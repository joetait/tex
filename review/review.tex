\documentclass[article,11pt]{memoir}

\begin{document}
Ian Stewart's \emph{``17 equations that changed to world"} takes an historical tour of the most fundamental, influential and important equations in mathematics. The book ranges from pure geometry in the form of Pythagoras' theorem, to physics via relativity and Schr\"odinger's equation, and even the world of finance, ending with the Black-Scholes equation.

Each chapter builds on what has gone before. The chapters on thermodynamics and gravity, for example, make clear that they fundamentally rely on the previously discussed concepts of Fourier transforms and calculus. Especially for the lay reader, this helps stress that even the most abstract ideas are used in day-to-day life (albeit often indirectly). Mentioning concepts more than once also adds a feeling of continuity --- rather than an endless string of facts, one thinks of a web of ideas. The further one progresses the more one can place these ideas in a larger context.

Stewart does not shy away from going in to details. Some of these are relegated to the appendices (I can appreciate that not every reader will want to see the formal definition of a limit), but it is nice to see them included. The majority of the details are in the main content however, and a very good job is done of mixing them with examples and analogies. This allows the insightful reader to infer more than is said, whilst still being understandable to everyone.

There is an ongoing social commentary, which particularly adds to the historical context, making the equations, and more importantly the mathematics, more memorable. A particular highlight of this is the final chapter on the Black-Scholes equation. Great care is taken to show the very real effects that one equation can have on our world (no matter how abstract it may be). Ian Stewart points out that the current global financial problems were predicted in mathematics community, and were at least contributed to by the misuse and misunderstanding of a simple mathematical equation. He takes the time to explain how, whilst one can blindly use an equation and get a meaningful result, a lack of understanding and context will almost invariably lead to the result being misused. 

In review, this book gives a well rounded history of mathematics in a novel and interesting manner. By giving himself the broad topic of equations, Stewart has managed to present a medley of different fields and areas, yet still link them to each other and to the real world. This also limits the detail with which any one topic can be consider, and thus the book is probably best suited to mathematical enthusiasts. Such a reader will have a chance to see the beauty in mathematics that can be discovered in a short time by considering the right ideas. However, even the seasoned academic should appreciate this reminder of the breadth of his or her field.
\end{document}
