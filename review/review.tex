\documentclass[article,11pt]{memoir}

\begin{document}
Ian Stewart's "17 equations that changed to world" takes an historical tour of the most fundamentl, influential and important equations in mathematics. He ranges from pure geometry in the form of Pythagoras' theorem, to physics, with relativity and Schrodinger's equation, and even the world of finance, ending with the Balck-Scholes equation.

Each chapter builds on what has gone before, reminding the reader that without calculus or fourier transfroms, our understanding of thermodynamics and *put somehting else in here* could not exist. Especially for the lay reader, this helps to remind *one* that even the most abstract ideas are used in day to day life, if indirectly. It also gives an incredibly nice feel continuity - rather than feeling like an endless string of facts, one thinks of a web of ideas, and the more one reads the more one can put these ideas in a larger context.

Ian Stewart does not shy away from going in to details either. Some of this is relegated to the appendices (I can appreciate that not every reader will want to see the formal definition for a limit), but it is still nice to see it included. Much of it is still in the main content of the book though. A very good job is done of mixing examples and analogies with the technial *parts*, so that an insightful *reader* can infer more than is said, whilst still being understandable to everyone.

There is an ongoing social commentary which *put something here*. A particular highlight is chapter on the Balck-Scholes equation. In this last chapter of the book great care is taken to show the very real effects that one equation can have on the world (no matter how abstract it may look). He points out how the now ubiquitous global financial problems were predicted in mathematics community well in advance, was at least contributed to by the misuse and misunderstanding of a simple mathematical equation. He takes the time to explain how, whilst one can blindly use an equation by putting in the numbers and get out a meaningful result, a lack of understanding and context will almost invariably lead to the result being misused. 

In sum, this book is an excellent, well rounded approach to giving highlights of the history of mathematics. By giving himself the broad scope of "equations", Ian Stewart has managed to give a medlee of different topics and areas, yet still link them both to each other and the real world. The book will give non-mathematicians a chance to see the depth and beauty that can be gained in a short time by looking at the right *things*, whilst still reminding the seasoned academic of the breadth of the field.
\end{document}
