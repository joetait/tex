\documentclass[article,11pt]{memoir}

\begin{document}
Ian Stewart's ``17 equations that changed to world" takes an historical tour of the most fundamental, influential and important equations in mathematics. The book ranges from pure geometry in the form of Pythagoras' theorem, to physics via relativity and Schr\"odinger's equation, and even the world of finance by ending with the Black-Scholes equation.

Each chapter builds on what has gone before. For example, the reader is reminded that without calculus or Fourier transforms, our understanding of gravity and thermodynamics could not exist. Especially for the lay reader, this helps in stressing that even the most abstract ideas are used in day to day life (albeit often indirectly). It also adds a feeling of continuity --- rather than being an endless string of facts, one thinks of a web of ideas, and the further one progresses the more one can put these ideas in a larger context.

Ian Stewart does not shy away from going in to details either. Some of this is relegated to the appendices (I can appreciate that not every reader will want to see the formal definition of a limit), but it is still nice to see it included. The majority is in the main content of the book however, and a very good job is done of mixing examples and analogies in these technical sections. This allows the insightful reader to infer more than is said, whilst still being understandable to everyone.

There is also an ongoing social commentary which adds context, making the equations and importantly the mathematics more memorable. A particular highlight of this is the final chapter on the Black-Scholes equation, where great care is taken to show the very real effects that one equation can have on the world (no matter how abstract it may look). Ian Stewart points out how the now ubiquitous global financial problems were predicted in mathematics community well in advance, was at least contributed to by the misuse and misunderstanding of a simple mathematical equation. He takes the time to explain how, whilst one can blindly use an equation by putting in the numbers and get out a meaningful result, a lack of understanding and context will almost invariably lead to the result being misused. 

In sum, this book gives a well rounded history of mathematics in a novel and interesting manner. By giving himself the broad scope of ``equations", Ian Stewart has managed to consider a medley of different topics and areas, yet still link them to each other and to the real world. The book is probably best suited to mathematical enthusiasts, giving a chance to see the depth and beauty that can be gained in a short time by considering the right ideas. However, even the seasoned academic will appreciate a reminder of the breadth of his or her field.
\end{document}
