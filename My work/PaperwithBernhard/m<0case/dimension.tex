% !TEX TS-program = pdflatex
% !TEX encoding = UTF-8 Unicode

% This is a simple template for a LaTeX document using the "article" class.
% See "book", "report", "letter" for other types of document.

\documentclass[11pt]{article} % use larger type; default would be 10pt

\usepackage[utf8]{inputenc} % set input encoding (not needed with XeLaTeX)

%%% Examples of Article customizations
% These packages are optional, depending whether you want the features they provide.
% See the LaTeX Companion or other references for full information.

%%% PAGE DIMENSIONS
\usepackage{geometry} % to change the page dimensions
\geometry{a4paper} % or letterpaper (US) or a5paper or....
% \geometry{margins=2in} % for example, change the margins to 2 inches all round
% \geometry{landscape} % set up the page for landscape
%   read geometry.pdf for detailed page layout information

\usepackage{graphicx} % support the \includegraphics command and options

\usepackage[parfill]{parskip} % Activate to begin paragraphs with an empty line rather than an indent

%%% PACKAGES
\usepackage{booktabs} % for much better looking tables
\usepackage{array} % for better arrays (eg matrices) in maths
\usepackage{paralist} % very flexible & customisable lists (eg. enumerate/itemize, etc.)
\usepackage{verbatim} % adds environment for commenting out blocks of text & for better verbatim
\usepackage{subfig} % make it possible to include more than one captioned figure/table in a single float
% These packages are all incorporated in the memoir class to one degree or another...

%%% HEADERS & FOOTERS
\usepackage{fancyhdr} % This should be set AFTER setting up the page geometry
\pagestyle{fancy} % options: empty , plain , fancy
\renewcommand{\headrulewidth}{0pt} % customise the layout...
\lhead{}\chead{}\rhead{}
\lfoot{}\cfoot{\thepage}\rfoot{}

%%% SECTION TITLE APPEARANCE
\usepackage{sectsty}
\allsectionsfont{\sffamily\mdseries\upshape} % (See the fntguide.pdf for font help)
\usepackage{amsmath}
\usepackage{amsthm}
\usepackage{amsfonts}
\usepackage{mathrsfs}
\usepackage{amsopn}
\usepackage{amssymb}
\usepackage{natbib}
% (This matches ConTeXt defaults)

%%% ToC (table of contents) APPEARANCE
\usepackage[nottoc,notlof,notlot]{tocbibind} % Put the bibliography in the ToC
\usepackage[titles,subfigure]{tocloft} % Alter the style of the Table of Contents
\renewcommand{\cftsecfont}{\rmfamily\mdseries\upshape}
\renewcommand{\cftsecpagefont}{\rmfamily\mdseries\upshape} % No bold!

%Theorems and stuff
\newtheorem{defn}{Definition}
\newtheorem{thm}{Theorem}
\newtheorem{cor}{Corollary}
\newtheorem{lem}{Lemma}
\newtheorem{prop}{Proposition}

\DeclareMathOperator{\ord}{ord}

%%% END Article customizations

%%% The "real" document content comes below...

\title{Template}
\author{J Tait}
%\date{} % Activate to display a given date or no date (if empty),
         % otherwise the current date is printed 

\begin{document}
\maketitle

\begin{lem}
  Suppose that $m<0$. Then
  \begin{equation*}
    \dim_k(H^0(X,\Omega_X^{\otimes m})^G)=
      \begin{cases}
	0 & \mbox{if }\deg\left\lfloor \frac{m\pi_*(R)}{n} \right\rfloor\leq 2m-2\\
	1-2m+\left\lfloor \frac{m\pi_*(R)}{n} \right\rfloor & \mbox{otherwise}
      \end{cases}
  \end{equation*}
\end{lem}
\begin{proof}
  Let $K_Y$ be a canonical divisor on $Y$. Then we define $K_X:=\pi^*(K_Y)+R$, and hence $mK_X=m\pi^*(K_Y)+mR$; note that $K_X$ is a canonical divisor by \citep[Chap. IV,\ prop 2.3]{hart} and is also $G$-invariant by definition. Moreover, by ? we have an equivariant isomorphism between th $G$-sheaves $\mathscr{O}_X(K_X))$ and $\Omega_X$. We have that
	\[ 
		\Big{\lfloor}{\frac{\pi_*(mK_X)}{n}} \Big{\rfloor}
		= \Big{\lfloor}{\frac{\pi_*(m\pi^*(K_Y))+\pi_*(mR)}{n}} \Big{\rfloor}
		= mK_Y+\Big{\lfloor} {\frac{m\pi_*(R)}{n}}\Big{\rfloor}.
	\]
	
Hence, by the lemma in \citep{faithfulaction}, we see that 
	\[
		\pi_*^G(\Omega_X^{\otimes m})\cong \pi_*^G(\mathscr{O}_X(mK_X))\cong \mathscr{O}_Y\Big(mK_Y 			+\Big{\lfloor}{\frac{m\pi_*(R)}{n}}\Big{\rfloor}\Big).
	\]

and so
	\begin{eqnarray*}
		\mbox{dim}_k(H^0(X,\Omega_{X}^{\otimes m})^G)&=&
		\mbox{dim}_k(H^0(Y,\pi_*^G(\Omega_{X}^{\otimes m})))\\
		&=&\mbox{dim}_k\Big(H^0\Big(Y,\mathscr{O}_Y\Big(mK_Y+\Big{\lfloor}{\frac{m\pi_*(R)}						{n}}\Big{\rfloor}\Big)\Big)\Big).
	\end{eqnarray*}
Since $g_X=0$ then clearly $g_Y=0$ too.
We know look at two cases.
Firstly, when $\deg(mK_Y+\left\lfloor \frac{m\pi_*(R)}{n} \right\rfloor)\leq -2$ then clearly the dimension of the fixed space is $0$, as the degree is negative.
Since $g_Y=0$ then $\deg(mK_Y)=-2m$, hence re-arranging the above formula gives us the first case.

Now if $\deg(mK_Y+\left\lfloor \frac{m\pi_*(R)}{n} \right\rfloor)>-2$ then the divisor is ample and hence the dimension is precisely
  \begin{equation}
    \deg(mK_Y+\left\lfloor \frac{m\pi_*(R)}{n} \right\rfloor)+1-g_Y=1-2m+\left\lfloor \frac{m\pi_*(R)}{n} \right\rfloor.
  \end{equation}
\end{proof}


\bibliography{/home/jtait/Documents/Maths/Bibliography/biblio.bib}
\bibliographystyle{plain}

\end{document}