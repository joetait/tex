% !TEX TS-program = pdflatex
% !TEX encoding = UTF-8 Unicode

% This is a simple template for a LaTeX document using the "article" class.
% See "book", "report", "letter" for other types of document.

\documentclass[11pt]{article} % use larger type; default would be 10pt

\usepackage[utf8]{inputenc} % set input encoding (not needed with XeLaTeX)

%%% Examples of Article customizations
% These packages are optional, depending whether you want the features they provide.
% See the LaTeX Companion or other references for full information.

%%% PAGE DIMENSIONS
\usepackage{geometry} % to change the page dimensions
\geometry{a4paper} % or letterpaper (US) or a5paper or....
% \geometry{margins=2in} % for example, change the margins to 2 inches all round
% \geometry{landscape} % set up the page for landscape
%   read geometry.pdf for detailed page layout information

\usepackage{graphicx} % support the \includegraphics command and options

\usepackage[parfill]{parskip} % Activate to begin paragraphs with an empty line rather than an indent

%%% PACKAGES
\usepackage{booktabs} % for much better looking tables
\usepackage{array} % for better arrays (eg matrices) in maths
\usepackage{paralist} % very flexible & customisable lists (eg. enumerate/itemize, etc.)
\usepackage{verbatim} % adds environment for commenting out blocks of text & for better verbatim
\usepackage{subfig} % make it possible to include more than one captioned figure/table in a single float
% These packages are all incorporated in the memoir class to one degree or another...

%%% HEADERS & FOOTERS
\usepackage{fancyhdr} % This should be set AFTER setting up the page geometry
\pagestyle{fancy} % options: empty , plain , fancy
\renewcommand{\headrulewidth}{0pt} % customise the layout...
\lhead{}\chead{}\rhead{}
\lfoot{}\cfoot{\thepage}\rfoot{}

%%% SECTION TITLE APPEARANCE
\usepackage{sectsty}
\allsectionsfont{\sffamily\mdseries\upshape} % (See the fntguide.pdf for font help)
\usepackage{amsmath}
\usepackage{amsthm}
\usepackage{amsfonts}
\usepackage{mathrsfs}
\usepackage{amsopn}
\usepackage{amssymb}
\usepackage{natbib}
% (This matches ConTeXt defaults)

%%% ToC (table of contents) APPEARANCE
\usepackage[nottoc,notlof,notlot]{tocbibind} % Put the bibliography in the ToC
\usepackage[titles,subfigure]{tocloft} % Alter the style of the Table of Contents
\renewcommand{\cftsecfont}{\rmfamily\mdseries\upshape}
\renewcommand{\cftsecpagefont}{\rmfamily\mdseries\upshape} % No bold!

%Theorems and stuff
\newtheorem{defn}{Definition}
\newtheorem{thm}{Theorem}
\newtheorem{cor}{Corollary}
\newtheorem{lem}{Lemma}
\newtheorem{prop}{Proposition}
\theoremstyle{remark}\newtheorem*{rem}{Remark}

\newcommand{\cO}{{\cal O}}
\newcommand{\ra}{\rightarrow}
\newcommand{\NN}{{\mathbb N}}
\newcommand{\PP}{{\mathbb P}}
\newcommand{\ZZ}{{\mathbb Z}}
\newcommand{\cL}{{\cal L}}

\DeclareMathOperator{\ord}{ord}
\DeclareMathOperator{\di}{div}
\DeclareMathOperator{\cha}{char}
\DeclareMathOperator{\gal}{Gal}


%%% END Article customizations

%%% The "real" document content comes below...

\title{Using ``Galois structure of Zariski cohomology''}
\author{J Tait}
%\date{} % Activate to display a given date or no date (if empty),
         % otherwise the current date is printed 

\begin{document}
\maketitle
Let $X$ be a smooth connected projective algebraic curve over an algebraically closed field $k$ of charactersitic $p>0$.
Suppose that $G$ is a cyclic group of prime order $l$ (not equal $p$) which acts faithfully on $X$.
We denote by $Y$ the quotient curve $X/G$, and by $\pi$ the canonical projection $X\rightarrow Y$.
Then let $K_X$ and $K_Y$ denote the canonical divisors of $X$ and $Y$ respectively, and similarly let $g_X$ and $g_Y$ denote the genera of the curves.
Moreover, we assume that $g_X\geq 2$ and that $g_Y = 0$.


We wish to use \citep[Thm 4.5]{galiosstruc} to find when $G$ acts trivially on $H^0(X,\Omega^{\otimes m})$.
In order to do this we will first need to introduce some notation.

We first denote by $\chi_P$ the representation of the decomposition group $G_P$ on the cotangent space $\frak m_P/\frak m_P^2$.
We also introduce the unique projective module $N_{G,X}$ satisfying
\begin{equation*}
 \bigoplus^l N_{G,X} = \bigoplus_{P\in X} \bigoplus_{d=1}^{e^t_p-1} \bigoplus^d {\rm Ind}_{G_P}^G(\chi_P^d),
\end{equation*}

where $e^t_p$ denotes that (tame) ramification index at a point $P\in X$, as in \citep[Thm 4.3]{galiosstruc}.

Fix an integer $m\geq 2$. Now if we write $mK_X = \sum_{P\in X} n_P[P]$ then we can define $r_P\in \{0,\ldots, e^t_p - 1\}$ and $s_P\in \ZZ$ to be the unique values such that
\[
 n_P = r_P + s_Pe_P^t.
\]

Finally, if for any $Q\in Y$ we let $\bar Q \in X$ be an element of the fibre with respect to $\pi$, we can state \citep[Thm 4.5]{galiosstruc}, which says that we have the following equality in $K_0(k[G])$:
\begin{equation}\label{eq}
 \chi(G,X,\Omega^{\otimes m}) = -[N_{G,X}] +\sum_{Q\in Y}\sum^{r_{\bar Q}}_{d=1} [{\rm Ind}_{G_{\bar Q}}^G(\chi_{\bar Q}^d)]  + \left( 1 + \sum_{Q\in Y} s_{\bar Q} \right) [k[G]],
\end{equation}
where for any $k[G]$ module $A$ then $[A]$ denotes that class of $A$ in $K_0[k[G]]$.

We wish to find when the action of $G$ on $H^0(X,\Omega^{\otimes m})$ is trival.
Note that $[H^0(X,\Omega^{\otimes m})]$ is in the same class as $\chi(G,X,\Omega^{\otimes m})$ in $K_0(k[G])$, since $mK_X$ is non-special under our assumptions.
Hence the action will be trivial precisely when we have equality between the dimension and the fixed dimension.
However, rather than compute this in its entirety we will look at the weaker question of when each component of the right hand side of \ref{eq} has dimension equal to its fixed dimension.
We will see that even this can only tell us that $l=2$ and that $m$ is either odd or 2.


Before we look at the first two parts, we recall from \citep[Rem. 4.30]{introtoreps} that for a representation $V$ of a group $G$ and a subgroup $H\subseteq G$ we have
\[
  \dim \rm{Ind}_H^G(V) = \dim(V)\cdot\frac{|G|}{|H|}.
\]

Since our group is of prime order, any subgroup is either trivial, or the whole group.
If $G_P = G$ then $G$ acts trivially on $P$, and $P$ is a ramification point, and has ramification index $l=|G|$.
In the other case, when $G_P$ is trivial, then $P$ is unramified, and has ramification index one.
Now in both $N_{G,X}$ and $\sum_{Q\in Y}\sum^{r_{\bar Q}}_{d=1} {\rm Ind}_{G_P}^G(\chi_P^d)$ we sum over all the points in $X$ and $Y$ respectively, and then we sum to one less than their ramification index.
So we have $\dim {\rm Ind}_{G_P}^G(\chi_P^d) = \dim \chi_P^d = 1$ for all contributing components.

Following this we can write 
\[
 l\cdot \dim N_{G,X} = \sum_{P\in X} \sum_{d=1}^{e^t_P -1} d\cdot \dim {\rm Ind}_{G_P}^G\chi_P^d = \sum_{P\ \rm{ ramified}} \sum_{d=1}^{l -1} d
\]
since the dimension of $\chi^d_P$ is $1$.
Now the fixed dimension of $\chi^d_P$ is precisely 0 for $d>0$, and hence we also have
\[
 l\cdot \dim N_{G,X}^G = \sum_{P\ \rm{ ramified}} \sum_{d=1}^{l-1} 0 = 0.
\]

We now compute the difference of these to see when they are zero:
\begin{eqnarray*}
 l\cdot \dim N_{G,X} - l\cdot \dim N_{G,X}^G & = & l\cdot \sum_{P\ \rm{ ramified}} \left( \frac{(l-1)l(2l-1)}{6} - \frac{(l-1)l}{2}\right) \\
  & = & l\cdot \sum_{P\ \rm{ ramified}} \frac{(2l-1)((l-1)l) - 3(l-1)l}{6}\\
  & = & l\cdot \sum_{P\ \rm{ ramified}} \frac{(2l-4)(l-1)l}{6}.
\end{eqnarray*}
The sum is clearly zero if and only if all it's terms are, which is true if and only if $l\in \{0,1,2\}$.
But $l=0$ is not possible, and $l=1$ would mean that we have the trivial group, so $l=2$ is the only interesting case.

We now move on to the second term in the sum, and compute its dimension which is
\[
 \dim \left(\sum_{Q\in Y} \sum_{d=1}^{r_{\bar Q}} \rm{Ind}_{G_{\bar Q}}^G (\chi_{\bar Q}^{-d})\right) = \sum_{Q\in Y} \sum_{d=1}^{r_{\bar Q}} d.
\]
On the other hand the fixed dimension is
\[
 \sum_{Q\in Y} \sum_{d=1}^{r_{\bar Q}} 1.
\]

So these dimensions will be equal if and only if $r_{\bar Q}$ is equal to 0 or 1 for all $Q \in Y$.
Since we know that $l=2$, we can be more explicit.
Recall that we can write $mK_X = mK_Y + mR$, where $R$ is the ramification divisor.
We can assume $K_Y = -2[P']$ for some unramified point $P'\in X$.
This gives us $-2m = r_{P'} + s_{P'}l$, and for any ramified point $P$ we also have $m(l-1) = r_P + s_Pl$ (since all the ramification is tame, all the non-zero coefficients of $R$ are $l-1)$).
Since $l=2$ the first of these identities tells us that $r_{P'}=0$ and $s_{P'}=m$.
From the second identity we see that if $P$ is a ramified point $r_P$ is $0$ if $m$ is even, and is $1$ otherwise.


Finally, we now compute when 
\[
 \dim \left( 1 + \sum_{Q\in Y} s_{\bar Q} \right) k[G] - \dim \left( 1 + \sum_{Q\in Y} s_{\bar Q} \right) k[G]^G = \left( 1 + \sum_{Q\in Y} s_{\bar Q} \right)(l-1) = 0.
\]
Since $l=2$ reduces to
\[
 \left( 1 + \sum_{Q\in Y} s_{\bar Q} \right)(l-1) = 1 + \sum_{Q\in Y} s_{\bar Q} = 0.
\]

We can rewrite this as
\begin{eqnarray*}
 -l &=& \sum_{Q\in Y} ls_{\bar Q}\\
 &= &\sum_{P\ \rm{ramified}}( m(l-1) -r_P )- (2m + r_{P'})\\
& =& \sum_{P\ \rm{ramified}}( m(l-1) -r_P )- 2m,
\end{eqnarray*}
since $r_{P'}=0$.
Now if $m$ is odd and there are two ramification points then this is satisfied; we just have $2(m-1) -2m = -2$, which is true.
If $m$ is even we can write $m = 2m'$, then for equality to hold we are asking that $-2=n2m'-4m'= 2m'(n-2)$. 
Since $m'(2-n) = 1$ if and only $m' = n = 1$, the only even value $m$ can take is two.

So we have deduced that $l=2$, and if even $m=2$.

\bibliography{/home/joe/files/Documents/Maths/Bibliography/biblio.bib}
\bibliographystyle{plain}

\end{document}