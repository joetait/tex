% !TEX TS-program = pdflatex
% !TEX encoding = UTF-8 Unicode

% This is a simple template for a LaTeX document using the "article" class.
% See "book", "report", "letter" for other types of document.

\documentclass[11pt]{article} % use larger type; default would be 10pt

\usepackage[utf8]{inputenc} % set input encoding (not needed with XeLaTeX)

%%% Examples of Article customizations
% These packages are optional, depending whether you want the features they provide.
% See the LaTeX Companion or other references for full information.

%%% PAGE DIMENSIONS
\usepackage{geometry} % to change the page dimensions
\geometry{a4paper} % or letterpaper (US) or a5paper or....
% \geometry{margins=2in} % for example, change the margins to 2 inches all round
% \geometry{landscape} % set up the page for landscape
%   read geometry.pdf for detailed page layout information

\usepackage{graphicx} % support the \includegraphics command and options

\usepackage[parfill]{parskip} % Activate to begin paragraphs with an empty line rather than an indent

%%% PACKAGES
\usepackage{booktabs} % for much better looking tables
\usepackage{array} % for better arrays (eg matrices) in maths
\usepackage{paralist} % very flexible & customisable lists (eg. enumerate/itemize, etc.)
\usepackage{verbatim} % adds environment for commenting out blocks of text & for better verbatim
\usepackage{subfig} % make it possible to include more than one captioned figure/table in a single float
% These packages are all incorporated in the memoir class to one degree or another...

%%% HEADERS & FOOTERS
\usepackage{fancyhdr} % This should be set AFTER setting up the page geometry
\pagestyle{fancy} % options: empty , plain , fancy
\renewcommand{\headrulewidth}{0pt} % customise the layout...
\lhead{}\chead{}\rhead{}
\lfoot{}\cfoot{\thepage}\rfoot{}

%%% SECTION TITLE APPEARANCE
\usepackage{sectsty}
\allsectionsfont{\sffamily\mdseries\upshape} % (See the fntguide.pdf for font help)
\usepackage{amsmath}
\usepackage{amsthm}
\usepackage{amsfonts}
\usepackage{mathrsfs}
\usepackage{amsopn}
\usepackage{amssymb}
\usepackage{natbib}
% (This matches ConTeXt defaults)

%%% ToC (table of contents) APPEARANCE
\usepackage[nottoc,notlof,notlot]{tocbibind} % Put the bibliography in the ToC
\usepackage[titles,subfigure]{tocloft} % Alter the style of the Table of Contents
\renewcommand{\cftsecfont}{\rmfamily\mdseries\upshape}
\renewcommand{\cftsecpagefont}{\rmfamily\mdseries\upshape} % No bold!

%Theorems and stuff
\newtheorem{defn}{Definition}
\newtheorem{thm}{Theorem}
\newtheorem{cor}{Corollary}
\newtheorem{lem}{Lemma}
\newtheorem{prop}{Proposition}
\theoremstyle{remark}\newtheorem*{rem}{Remark}

\newcommand{\cO}{{\cal O}}
\newcommand{\ra}{\rightarrow}
\newcommand{\NN}{{\mathbb N}}
\newcommand{\PP}{{\mathbb P}}
\newcommand{\ZZ}{{\mathbb Z}}
\newcommand{\cL}{{\cal L}}

\DeclareMathOperator{\ord}{ord}
\DeclareMathOperator{\di}{div}
\DeclareMathOperator{\cha}{char}
\DeclareMathOperator{\gal}{Gal}


%%% END Article customizations

%%% The "real" document content comes below...

\title{Artin-Schreier extensions}
\author{J Tait}
%\date{} % Activate to display a given date or no date (if empty),
         % otherwise the current date is printed 

\begin{document}
\maketitle
Let $k$ be an algebraically closed field of characteristic 2.
Consider the Artin-Schreier extension $k(x,y)$ of $k(x)$ by $y$, where $y$ satisfies
\begin{equation}\label{1}
  y^2 - y = f(x)
\end{equation}
for some polynomial $f(x)\in k[x]$.
Then let $C$ be the smooth projective curve associated to this extension.
Let $g$ be the genus of the curve, and let $\pi$ be the canonical projection $C\rightarrow \mathbb P_1^k$.
We denote the space of global holomorphic differentials of $C$ by $\Omega$.

We now prove the following theorem, determining a basis of the space of global holomorphic differentials as a vector space over $k$.\\




\begin{thm}
 The elements $dx, xdx, \ldots , x^{(g-1)}dx$ form a basis of  of $\Omega$ over $k$.
\end{thm}
\begin{proof}
First note that if $C$ is birationally equivalent to $\mathbb P_1^k$, then $g=0$, there are no basis elements, and the theorem holds.

We may now assume that $g\geq 1$.
Considering $\mathbb A_k^2$ as a subspace of $\mathbb P_k^2$, let $C'$ the part of the curve in $\mathbb A_k^2$.
Then $\pi$ restricts to a map $C'\rightarrow \mathbb A^1_k$, the projection on the $x$ co-ordinate.
Let $x'\in \mathbb A_k^1$ be a point in the image of $\pi$.
Then if $(y',x')$ is a point in $\pi^{-1}(x')$, so is the distinct point $(y'+1,x')$.
Since the extension is degree two, this shows that there are no ramified points in the affine part.

However, from the Hurwitz formula we see that
\[
 2g+2 = \deg (R),
\]
where $R$ is the ramification divisor of $\pi$ (recall that $\pi$ maps to $\mathbb P_k^1$, which has genus zero).
Hence the degree of the ramification divisor is positive, and there must exist a ramification point i.e. the point at infinity, $[0:1:0]\in \mathbb P_k^2$.


We have already shown that there are two points, $P_0$ and $P_0'$, in $\pi^{-1}(0)$, and one point in $\pi^{-1}(\infty)$, which we call $P$.
Then $\di(\pi) = [P_0] + [P_0'] - 2[P]$.

Now we look at the differential $d x$.
Let $Q$ be a point in $C$ other than $P$, and let $a= \pi(Q)$ be the image of $Q$ in $\mathbb A_k^1$.
Then $x-a$ is a local parameter at $a$, and since $a$ is unramified, it is also a local parameter at $Q$.
Hence $\ord_Q(dx)=\ord_Q(d(x-a)) = 0$, by  \citep[\S 8.5]{fulton}.

Since $0\neq dx \in \Omega$, then $\di (dx)$ is a canonical divisor; see, for example, \citep[\S 8.5]{fulton}.
Hence we conclude that $\di (dx) = (2g-2)[P]$.
Then for $Q\notin \{P,P_0,P_0'\}$ we see that $\ord_Q(x^idx)=0$ for all $i\geq 0$.
If $Q$ is either $P_0$ or $P_0'$, then $\ord_Q(x^idx)=i$ for all $i\geq 0$.
Finally, $\ord_P(x^idx)=-2i+2g-2$ for all $i\geq 0$.
Hence $\deg(\di (x^idx))>0$ for $0 \leq i \leq g-1$, and $x^idx$ is a holomorphic differential for $0 \leq i\leq g-1$.
There are $g$ of them, so if we show that they are linearly independent, we will be done, since $\dim(\Omega)=g$.



If they were not linearly independent there would exist $a_i\in k$ for $0\leq i \leq g-1$ such that $\sum_{i=0}^{g-1} a_ix^idx = 0$.
Since $\Omega$ is a one dimensional vector space over the field of rational functions of $X$, this would imply $\sum_{i=0}^{g-1} a_ix^i = 0$.
Hence $a_i = 0$ for all $i$, and the differentials are linearly independent. 
\end{proof}



\bibliography{/home/jtait/files/Documents/Maths/Bibliography/biblio.bib}
\bibliographystyle{plain}

\end{document}