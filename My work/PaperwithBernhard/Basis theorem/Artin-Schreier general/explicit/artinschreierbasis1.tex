% !TEX TS-program = pdflatex
% !TEX encoding = UTF-8 Unicode

% This is a simple template for a LaTeX document using the "article" class.
% See "book", "report", "letter" for other types of document.

\documentclass[11pt]{article} % use larger type; default would be 10pt

\usepackage[utf8]{inputenc} % set input encoding (not needed with XeLaTeX)

%%% Examples of Article customizations
% These packages are optional, depending whether you want the features they provide.
% See the LaTeX Companion or other references for full information.

%%% PAGE DIMENSIONS
\usepackage{geometry} % to change the page dimensions
\geometry{a4paper} % or letterpaper (US) or a5paper or....
% \geometry{margins=2in} % for example, change the margins to 2 inches all round
% \geometry{landscape} % set up the page for landscape
%   read geometry.pdf for detailed page layout information

\usepackage{graphicx} % support the \includegraphics command and options

\usepackage[parfill]{parskip} % Activate to begin paragraphs with an empty line rather than an indent

%%% PACKAGES
\usepackage{booktabs} % for much better looking tables
\usepackage{array} % for better arrays (eg matrices) in maths
\usepackage{paralist} % very flexible & customisable lists (eg. enumerate/itemize, etc.)
\usepackage{verbatim} % adds environment for commenting out blocks of text & for better verbatim
\usepackage{subfig} % make it possible to include more than one captioned figure/table in a single float
% These packages are all incorporated in the memoir class to one degree or another...

%%% HEADERS & FOOTERS
\usepackage{fancyhdr} % This should be set AFTER setting up the page geometry
\pagestyle{fancy} % options: empty , plain , fancy
\renewcommand{\headrulewidth}{0pt} % customise the layout...
\lhead{}\chead{}\rhead{}
\lfoot{}\cfoot{\thepage}\rfoot{}

%%% SECTION TITLE APPEARANCE
\usepackage{sectsty}
\allsectionsfont{\sffamily\mdseries\upshape} % (See the fntguide.pdf for font help)
\usepackage{amsmath}
\usepackage{amsthm}
\usepackage{amsfonts}
\usepackage{mathrsfs}
\usepackage{amsopn}
\usepackage{amssymb}
%\usepackage{natbib}
% (This matches ConTeXt defaults)

%%% ToC (table of contents) APPEARANCE
\usepackage[nottoc,notlof,notlot]{tocbibind} % Put the bibliography in the ToC
\usepackage[titles,subfigure]{tocloft} % Alter the style of the Table of Contents
\renewcommand{\cftsecfont}{\rmfamily\mdseries\upshape}
\renewcommand{\cftsecpagefont}{\rmfamily\mdseries\upshape} % No bold!

%Theorems and stuff
\newtheorem{defn}{Definition}
\newtheorem{thm}{Theorem}
\newtheorem{cor}{Corollary}
\newtheorem{lem}{Lemma}
\newtheorem{prop}{Proposition}
\theoremstyle{remark}\newtheorem*{rem}{Remark}

\newcommand{\cO}{{\cal O}}
\newcommand{\ra}{\rightarrow}
\newcommand{\NN}{{\mathbb N}}
\newcommand{\PP}{{\mathbb P}}
\newcommand{\ZZ}{{\mathbb Z}}
\newcommand{\cL}{{\cal L}}

\DeclareMathOperator{\ord}{ord}
\DeclareMathOperator{\di}{div}
\DeclareMathOperator{\cha}{char}
\DeclareMathOperator{\gal}{Gal}


%%% END Article customizations

%%% The "real" document content comes below...

\title{Artin-Schreier extensions}
\author{J Tait}
%\date{} % Activate to display a given date or no date (if empty),
         % otherwise the current date is printed 

\begin{document}
\maketitle
Let $k$ be an algebraically closed field of characteristic 2.
Let $k(x)$ be the rational function field in one variable $x$ and let $K|k(x)$ be a field extension of degree 2, i.e.\ an Artin-Schreier extension.
Let $\pi:C \rightarrow \mathbb P_k^1$ be the corresponding morphism of smooth projective curves over $k$ and let $g$ denote the genus of $C$.
By \cite[\S 7.4.3]{liu}, there exists a $y\in K$ such that $K=k(x,y)$ and $y$ satisfies the equation
\begin{equation}\label{1}
  y^2 - h(x)y = f(x)
\end{equation}
for some polynomials $h(x), f(x)\in k[x]$, with maximum degrees of $g+1$ and $2g+2$ respectively.
In order for the curve to be smooth, we require that $h(x)$ and $h'(x)^2f(x) + f'(x)^2$ are prime to each other.
This comes from the Jacobian criterion.

Let $d$ be the degree of $h(x)$.
We denote the space of global holomorphic differentials of $C$ by $\Omega$.

We first describe the ramified points of $\pi$, in order to compute the ramification divisor.
Since we required \ref{1} to satisfy the Jacobian criterion, we can consider the affine curve defined by this equation, which will be smooth.
We denote this curve by $C'$.
Then $\pi$ restricts to a map $C'\rightarrow \mathbb A^1_k$, the projection on the $x$ co-ordinate.
Let $a\in \mathbb A_k^1$.
Then if $(a,b)$ is a point in $\pi^{-1}(a)$, so is the point $(a,b+h(a))$, which is clearly distinct if and only if $h(a)\neq 0$.
Since the extension is degree two, this shows that the ramified points in the affine part correspond to the roots of $h(x)$.
If $h(x)$ is non-constant, we denote zeroes of $h(x)$ by $a_i$ for $1\leq i \leq k$ where $k \leq g+1$.
For each $a_i$ there is a corresponding $b_i$, which we define to be the root of $f(a_i)$.
We will also denote the corresponding ramification point by $P_i$.
If these are not all the ramification points, then the point at infity is also ramified.
In this case the point that maps to infinity will be written $P_{\infty}$.
If this is not the case we will denote the two points that map to infinity as $P_{\infty}'$ and $P_{\infty}''$.

We will now compute the ramification divisor to see exactly when $P_1,\ldots,P_k$ are all the ramified points.\\

\begin{lem}
 Let $n_i$ be the order of $h(x)$ at $a_i\in \mathbb A_k^1$.
Then the ramification divisor of $\pi$ is
 \[
  R = \left\{ \begin{array}{ll}
               \sum_{i=1}^k 2n_i[P_i] & \mbox{if } \sum_{i=1}^k 2n_i = 2g+2 \\
		\\
               \sum_{i=1}^k 2n_i[P_i] + (2g+2-2\sum_{i=1}^kn_i)[P_{\infty}] & \mbox{otherwise.}
             \end{array} \right. \]

\end{lem}
\begin{rem}
 We will show that the point at infity is ramified if and only if $\sum_{i=1}^k n_i = \deg(h(x)) < g+1$
 (or equivalently it is unramified if and only  $\sum_{i=1}^k n_i = g+1$).
 Hence the equality will always make sense in the second case.
\end{rem}
\begin{proof}
 First note that by the Hurwitz formula $\deg(R) = 2g+2$.
 Then if the ramification at each $P_i$ is $n_i$, and $\sum_{i=1}^k 2n_i < 2g+2$, there must exist another point, i.e. $P_{\infty}$, which is ramified.
 Then the ramification at this point is $2g+2-\sum _{i=1}^k2n_i$.
 So it will suffice to prove that coefficient of $[P_i]$ is $2n_i$ for $1\leq i \leq k$.
 
 We start by showing that $y-b_i$ is a uniformising parameter in $\mathscr O_{P_i}$.
 Now $x-a_i$ is a uniformising parameter at $\pi(P_i)$, so it will be sufficient to show that $x-a_i = u(y-b_i)^2$ for some unit $u\in \mathscr O_{P_i}$.
 We write $y^2-b_i^2 = yh(x)-(f(x)-b_i^2)$, and then write $h(x)=(x-a_i)w(x)$ and $f(x)-b_i^2=(x-a_i)v(x)$.
 So 
 \[
x-a_i = \frac{y^2-b_i^2}{yw(x)-v(x)} = \frac{(y-b_i)^2}{yw(x)-v(x)},
 \]
and it remains to show that the denominator, $r(x,y) = yw(x)-v(x)$, is a unit.

To show that $r(x,y)$ is indeed a unit we consider two cases, and show that in both the polynomial is not in the maximal ideal $\mathfrak M_{P_i}=(x-a_i,y-b_i)$.
Firstly, if $b_i=0$ then $y\in \mathfrak{M}_{P_i}$, and hence $r\in \mathfrak M_{P_i}$ if and only if $v(x)$ is.
But then we would have $(x-a_i)|v(x)$, and hence $(x-a_i)^2|f(x)$, and this would contradict the smoothness of the curve.
Alternatively, if $b_i\neq 0$, then $r(x,y)\in \mathfrak M_{P_i}$ if and only if $(x-a_i)|r(x,y)$.
But then $(x-a_i)^2|yh(x)-f(x) + b_i^2$, and so $(x-a_i)|yh'(x)-f'(x)$, which again would contradict smoothness (recall that $(x-a_i)|h(x)$ by virture of $P_i$ being a ramification point).
So we have shown that $r(x,y)\in \mathscr O_{P_i}^*$.

Now write $h(x)=(x-a_i)^{n_i}s(x)$ where $s(x)\in \mathscr O_p^*$.
Note that $\sigma (y)=y+h(x)$, by an argument similar to that used to show that the solutions to $h(x)$ give ramified points.
Odd $N_{P_i}\in \mathbb N$ defined by $N_{P_i}+1=\ord_P(\sigma(y - b_i) - (y-b_i))$ exist by {\bf reference}, which are the coefficients of $P_i$ in $R$.
We compute these $N_{P_i}+1$ explicitly,
\begin{eqnarray*}
 \ord_{P_i}(\sigma(y-b_i)-(y-b_i)) & = & \ord_{P_i}(y-b_i-h(x)-y+b_i)\\
 & = & \ord_{P_i}((x-a_i)^{n_i}s(x))\\
 & = & 2n_i,
\end{eqnarray*}
finishing the proof.
\end{proof}



We now prove the following theorem, determining a basis of the space of global holomorphic differentials as a vector space over $k$.\\


\begin{thm}
 The elements $\frac{dx}{h(x)}, \frac{xdx}{h(x)}, \ldots , \frac{x^{(g-1)}dx}{h(x)}$ form a basis of  of $\Omega$ over $k$.
\end{thm}
\begin{proof}

First note that if $C$ is birationally equivalent to $\mathbb P_1^k$, then $g=0$, there are no basis elements, and the theorem holds.

We may now assume that $g\geq 1$.
Without loss of generality we may assume the $0\in \mathbb A_k^1$ is unramified.
Hence there are two points, $P_0$ and $P_0'$, in $\pi^{-1}(0)$.


First we notice that $\ord_{P_0}(x^i)=\ord_{P_0'}(x^i)=i$.
According to whether infinity is ramified, then $\ord_{P_{\infty}}(x^i)=-2i$ or $\ord_{P_{\infty}'}(x^i)=\ord_{P_{\infty}''}(x^i)=-i$.
The order of $x^i$ is zero at any other point.


Now we compute the divisor of $dx$. 
If $a\in \mathbb A_k^1$ is unramified, then $x-a$ is a uniformising parameter at both points in the preimage of $\pi^{-1}(a)$.
Then if $P$ maps to $a$, then $\ord_P(dx)=\ord_P(d(x-a))=0$, by \cite[\S 8.5]{fulton}.
Now consider any ramified point $a_i\in \mathbb A_k^1$, and as in the proof of the above lemma, we can write $x-a_i$ as $\frac{(y-b_i)^2}{yw(x)-v(x)}$.
This gives us that
\begin{eqnarray*}
 dx & = & d(x-a_i) \\
 & = & d \left( \frac{(y-b_i)^2}{yw(x)-v(x)} \right) \\
 & = & \frac{2(y-b_i)}{yw-v} dy + (y-b_i)^2\left( d\frac{1}{yw-v}\right)\\
 & = & \left( \frac{y-b_i}{yw+v}\right)^2d(yw+v)\\
 & = & \left( \frac{y-b_i}{yw+v}\right)^2(d(yw)+dv)
 \end{eqnarray*}
 and 
 \[
\ord_{P_i}\left( \left( \frac{y-b_i}{yw+v}\right)^2(d(yw)+dv)\right) = \ord_{P_i}\left( \left( \frac{y-b_i}{yw+v}\right)^2d(yw)\right)
\]
since $v$ is a unit.
If $(x-a_i)\nmid w$ then $w$ is a unit and we see that the order of $dx$ at $P_i$ is 2.
Otherwise, we write $w(x)=(x-a_i)w(x)'$, and iterate the process.
This gives $\ord_{P_i}(dx)=2n_i$.
Note that the case where $w$ is a unit is precisely the case where $n_i=1$.
Since $dx\neq 0$ then $\di (dx)$ is a canonical divisor and has degree $2g-2$.
It follows that $\ord_{P_{\infty}}(dx)=2(g-1-k)$.

Finally we look at $\frac{1}{h(x)}$.
If infity is ramified then $\ord_{P_{\infty}}\left(\frac{1}{h(x)}\right) = -\ord_{P_{\infty}}(h(x)) = 2d$.
If it is not ramified, then $\ord_{P_{\infty}'}\left(\frac{1}{h(x)}\right) = \ord_{P_{\infty}''}\left(\frac{1}{h(x)}\right)=g+1$.
For the ramified points $P_i$, $1\leq i \leq k$, then $\ord_{P_i}\left(\frac{1}{h(x)}\right) = -\ord_{P_i}(h(x))= -2n_i$.
At any other point the order of $\frac{1}{h(x)}$ is clearly zero.


Combining these divisors we see that if infinity is ramified then
\begin{eqnarray*}
 \di \left( \frac{x^idx}{h(x)}\right) & = & i[P_0]+ i[P_0'] -2i[P_{\infty}] +\sum_{i=1}^{k}2n_i[P_i] + 2(g-1-d)[P_{\infty}] - \sum_{i=1}^{k}2n_i[P_i] +2d[P_{\infty}]\\
 & = & i[P_0]+ i[P_0'] +2(g-1 -i)[P_{\infty}].
 \end{eqnarray*}
 If infity is not ramified then 
 \begin{eqnarray*}
  \di \left( \frac{x^idx}{h(x)}\right) & =  i([P_0]+ [P_0'] -[P_{\infty}']-[P_{\infty}'']) +\sum_{i=1}^{k}2n_i[P_i] & + (g-1-d)([P_{\infty}']+[P_{\infty}''])\\
  & & - \sum_{i=1}^{k}2n_i[P_i] +d[P_{\infty}']+d[P_{\infty}'']\\
 & =  i[P_0]+ i[P_0'] +(g-1 -i)([P_{\infty}']+[P_{\infty}'']).&
 \end{eqnarray*}

Hence the divisor is positive for $0\leq i \leq g-1$.




It remains to show that the elements are linearly independent.
This follows from the fact that $dx$ is a basis of the space of meromorphic functions over $K$, and the fact that the functions $\frac{1}{h(x)},\ldots , \frac{x^{g-1}}{h(x)}$ are linearly independent over $k$.
\end{proof}



\bibliography{/home/jtait/files/Documents/Maths/Bibliography/biblio.bib}
\bibliographystyle{alphanum}

\end{document}