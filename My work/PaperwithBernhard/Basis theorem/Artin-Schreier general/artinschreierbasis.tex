% !TEX TS-program = pdflatex
% !TEX encoding = UTF-8 Unicode

% This is a simple template for a LaTeX document using the "article" class.
% See "book", "report", "letter" for other types of document.

\documentclass[11pt]{article} % use larger type; default would be 10pt

\usepackage[utf8]{inputenc} % set input encoding (not needed with XeLaTeX)

%%% Examples of Article customizations
% These packages are optional, depending whether you want the features they provide.
% See the LaTeX Companion or other references for full information.

%%% PAGE DIMENSIONS
\usepackage{geometry} % to change the page dimensions
\geometry{a4paper} % or letterpaper (US) or a5paper or....
% \geometry{margins=2in} % for example, change the margins to 2 inches all round
% \geometry{landscape} % set up the page for landscape
%   read geometry.pdf for detailed page layout information

\usepackage{graphicx} % support the \includegraphics command and options

\usepackage[parfill]{parskip} % Activate to begin paragraphs with an empty line rather than an indent

%%% PACKAGES
\usepackage{booktabs} % for much better looking tables
\usepackage{array} % for better arrays (eg matrices) in maths
\usepackage{paralist} % very flexible & customisable lists (eg. enumerate/itemize, etc.)
\usepackage{verbatim} % adds environment for commenting out blocks of text & for better verbatim
\usepackage{subfig} % make it possible to include more than one captioned figure/table in a single float
% These packages are all incorporated in the memoir class to one degree or another...

%%% HEADERS & FOOTERS
\usepackage{fancyhdr} % This should be set AFTER setting up the page geometry
\pagestyle{fancy} % options: empty , plain , fancy
\renewcommand{\headrulewidth}{0pt} % customise the layout...
\lhead{}\chead{}\rhead{}
\lfoot{}\cfoot{\thepage}\rfoot{}

%%% SECTION TITLE APPEARANCE
\usepackage{sectsty}
\allsectionsfont{\sffamily\mdseries\upshape} % (See the fntguide.pdf for font help)
\usepackage{amsmath}
\usepackage{amsthm}
\usepackage{amsfonts}
\usepackage{mathrsfs}
\usepackage{amsopn}
\usepackage{amssymb}
%\usepackage{natbib}
% (This matches ConTeXt defaults)

%%% ToC (table of contents) APPEARANCE
\usepackage[nottoc,notlof,notlot]{tocbibind} % Put the bibliography in the ToC
\usepackage[titles,subfigure]{tocloft} % Alter the style of the Table of Contents
\renewcommand{\cftsecfont}{\rmfamily\mdseries\upshape}
\renewcommand{\cftsecpagefont}{\rmfamily\mdseries\upshape} % No bold!

%Theorems and stuff
\newtheorem{defn}{Definition}
\newtheorem{thm}{Theorem}
\newtheorem{cor}{Corollary}
\newtheorem{lem}{Lemma}
\newtheorem{prop}{Proposition}
\theoremstyle{remark}\newtheorem*{rem}{Remark}

\newcommand{\cO}{{\cal O}}
\newcommand{\ra}{\rightarrow}
\newcommand{\NN}{{\mathbb N}}
\newcommand{\PP}{{\mathbb P}}
\newcommand{\ZZ}{{\mathbb Z}}
\newcommand{\cL}{{\cal L}}

\DeclareMathOperator{\ord}{ord}
\DeclareMathOperator{\di}{div}
\DeclareMathOperator{\cha}{char}
\DeclareMathOperator{\gal}{Gal}


%%% END Article customizations

%%% The "real" document content comes below...

\title{Artin-Schreier extensions}
\author{J Tait}
%\date{} % Activate to display a given date or no date (if empty),
         % otherwise the current date is printed 

\begin{document}
\maketitle
Let $k$ be an algebraically closed field of characteristic 2.
Let $k(x)$ be the rational function field in one variable $x$ and let $K|k(x)$ be a field extension of degree 2, i.e.\ an Artin-Schreier extension.
Let $\pi:C \rightarrow \mathbb P_k^1$ be the corresponding morphism of smooth projective curves over $k$ and let $g$ denote the genus of $C$.
By \cite[\S 7.4.3]{liu}, there exists a $y\in K$ such that $K=k(x,y)$ and $y$ satisfies the equation
\begin{equation}\label{ext}
  y^2 - h(x)y = f(x)
\end{equation}
for some polynomials $h(x), f(x)\in k[x]$, with maximum degrees of $g+1$ and $2g+2$ respectively.
In order for the curve to be smooth, we require that $h(x)$ and $h'(x)^2f(x) + f'(x)^2$ are prime to each other.
This is equivalent to the Jacobian criterion.

Let $d$ be the degree of $h(x)$.
We denote the space of global holomorphic differentials of $C$ by $\Omega$.

We first describe the ramified points of $\pi$, in order to compute the ramification divisor.
Since we required \eqref{ext} to satisfy the Jacobian criterion, we can consider the affine curve defined by this equation, which will be smooth.
We denote this curve by $C'$.
Then $\pi$ restricts to a map $C'\rightarrow \mathbb A^1_k$, the projection on the $x$ co-ordinate.
Let $a\in \mathbb A_k^1$.
Then if $(a,b)$ is a point in $\pi^{-1}(a)$, so is the point $(a,b+h(a))$, which is clearly distinct if and only if $h(a)\neq 0$.
Since the extension is degree two, this shows that the ramified points in the affine part correspond to the roots of $h(x)$.
If $h(x)$ is non-constant, we denote zeroes of $h(x)$ by $a_i$ for $1\leq i \leq k$ where $k \leq g+1$.
For each $a_i$ there is a corresponding $b_i$, which is the root of $f(a_i)$.
We will also denote the corresponding ramification point by $P_i$.
If these are not all the ramification points, then the point at infinity is also a branch point.
In this case the point that maps to infinity will be written $P_{\infty}$.
If this is not the case we will denote the two points that map to infinity as $P_{\infty}'$ and $P_{\infty}''$.

We will now compute the ramification divisor to see exactly when $P_1,\ldots,P_k$ are all the ramified points.\\

\begin{lem}
 Let $n_i$ be the order of $h(x)$ at $a_i\in \mathbb A_k^1$.
Then the coefficient $\delta_P$ of the ramification divisor $R$ at $P\in X$ is given by
\[
 \delta_P = \left\{
 \begin{array}{ll}
 2n_i & {\rm if }\ P=P_i\ {\rm for\ some }\ i \in \{1,\ldots ,k\}, \\
 2(g+1-d) & {\rm if }\ d<g+1\ {\rm and }\ P=P_\infty, \\
 0 & {\rm otherwise.} 
 \end{array}
\right.
\]
\end{lem}
\begin{proof}
 First note that by the Hurwitz formula $\deg(R) = 2g+2$.
 Then if the ramification index at each $P_i$ is $2n_i$, and $\sum_{i=1}^k 2n_i = 2d < 2g+2$, there must exist another point, i.e. $P_{\infty}$, which is ramified.
 Then the ramification at this point is $2g+2-2d$.
 So it will suffice to prove that coefficient of $[P_i]$ is $2n_i$ for $1\leq i \leq k$.
 
 In order to prove this we first need to show that $y-b_i$ is a local parameter at $P_i$.
 To see this, note that the maximal ideal $\mathfrak M_{P_i}$ of the local ring $\cO_{X,P_i}$ at $P_i$ is generated by $x-a_i$ and $y-b_i$.
 But $x-a_i\in \mathfrak M_{P_i}^2$ since $\pi$ is ramified at $P_i$ with ramification index 2 {\bf At least 2??}.
 By Nakayama's lemma, $y-a_i$ is therefore a local parameter at $P_i$.
 
 Now we can compute $\delta_P$ for $P\in \{P_1,\ldots , P_k\}$, using Hilbert's formula \cite[Prop. 4, \S 1, Ch. IV]{localfields},
 \begin{eqnarray*}
  \delta_P & = & \sum_{i\geq 0} \left(\ord(G_i(P))-1\right) \\
  & = & {\rm max\ }\left\{ i\in \NN | G_i(P)\neq \{1\}\right\} \\
  & = & \ord_{P_i}(\sigma(y-b_i) - (y-b_i)).
  \end{eqnarray*}
  
 By an argument similar to that used to show the correspondence between the solutions of $h(x)$ and the ramification points, it is clear $\sigma(y)=y+h(x)$.
 The following calculation then concludes the proof,
  \begin{eqnarray*}
  \delta_P & = & \ord_{P_i}(\sigma(y-b_i) - (y-b_i)) \\
  & = & \ord_{P_i}(y-b_i+h(x) - y + b_i) \\
  & = & 2\ord_{a_i}(h(x)) \\
  & = & 2n_i.
 \end{eqnarray*}
\end{proof}



We now prove the following theorem, determining a basis of the space of global holomorphic differentials as a vector space over $k$.\\


\begin{thm}
 The elements $\frac{dx}{h(x)}, \frac{xdx}{h(x)}, \ldots , \frac{x^{(g-1)}dx}{h(x)}$ form a basis of $\Omega$ over $k$.
\end{thm}
\begin{proof}

First note that if $C$ is birationally equivalent to $\mathbb P_1^k$, then $g=0$, there are no basis elements, and the theorem holds.

We may now assume that $g\geq 1$.
If $0\in \mathbb A_k^1$ is a branch point, then we will denote the point that maps to it by $P_0$.
If it is not ramified then we will denote the points above it by $P_0'$ and $P_0''$.

In the rest of the proof we will refer to the divisors $D_0$ and $D_\infty$, defined as
\[
 D_0=\left\{ \begin{array}{ll}
             2P_0 &\ {\rm If\ 0\ is\ a\ branch\ point,}\\
             P_0'+P_0'' &\ \rm{otherwise,}
            \end{array}
\right.
\]
and
\[
 D_\infty = \left\{ \begin{array}{ll}
                     2P_\infty &\ {\rm if\ \infty\ is\ a\ branch\ point,}\\
                     P_\infty' + P_\infty'' &\ {\rm otherwise.}
                    \end{array}\right.
\]
These will allow us to consider the multiple cases simultaneously.

Now we will compute the divisors of the separate components of the elements we claim form a basis.

It is clear that $\di (x^i) = i(D_0-D_\infty)$.


Now we compute the divisor of $dx$.
We will need to consider the divisor of $dx$ both as a differential on $C$ and and on $\mathbb P^1$. 
We will use the notation of $\di_C(dx)$ and $\di_{\mathbb P^1}(dx)$ to do do this.
Then, the Hurwitz formula {\bf reference} states that
\[
 \di_C( dx) = \pi^*\di_{\mathbb P^1}(dx) + R.
\]
Now we have already computed $R$ in the previous lemma, and $\pi^*\di_{\mathbb P^1}(dx) = -2D_\infty$, so we understand $\di_C(dx)$.


Finally we look at $\frac{1}{h(x)}$.
If infinity is ramified then $\ord_{P_{\infty}}\left(\frac{1}{h(x)}\right) = -\ord_{P_{\infty}}(h(x)) = 2d$.
If it is not ramified, then $\ord_{P_{\infty}'}\left(\frac{1}{h(x)}\right) = \ord_{P_{\infty}''}\left(\frac{1}{h(x)}\right)=g+1$.
For the ramified points $P_i$, $1\leq i \leq k$, then $\ord_{P_i}\left(\frac{1}{h(x)}\right) = -\ord_{P_i}(h(x))= -2n_i$.
At any other point the order of $\frac{1}{h(x)}$ is clearly zero.
So $\di\left(\frac{1}{h(x)}\right) = (g+1)D_\infty -R$.


Combining these divisors gives us
\begin{eqnarray*}
 \di \left(\frac{x^idx}{h(x)}\right) & = & i(D_0-D_\infty)-2D_\infty + R + (g+1)D_\infty -R \\
 & = & iD_0 + (g-i-1)D_\infty.
\end{eqnarray*}

Hence the divisor is positive for $0\leq i \leq g-1$.




It remains to show that the elements are linearly independent.
This follows from the fact that $dx$ is a basis of the space of meromorphic functions over $K$, and the fact that the functions $\frac{1}{h(x)},\ldots , \frac{x^{g-1}}{h(x)}$ are linearly independent over $k$.
\end{proof}



\bibliography{/home/jtait/files/Documents/Maths/Bibliography/biblio.bib}
\bibliographystyle{alpha}

\end{document}