% !TEX TS-program = pdflatex
% !TEX encoding = UTF-8 Unicode

% This is a simple template for a LaTeX document using the "article" class.
% See "book", "report", "letter" for other types of document.

\documentclass[11pt]{article} % use larger type; default would be 10pt

\usepackage[utf8]{inputenc} % set input encoding (not needed with XeLaTeX)

%%% Examples of Article customizations
% These packages are optional, depending whether you want the features they provide.
% See the LaTeX Companion or other references for full information.

%%% PAGE DIMENSIONS
\usepackage{geometry} % to change the page dimensions
\geometry{a4paper} % or letterpaper (US) or a5paper or....
% \geometry{margins=2in} % for example, change the margins to 2 inches all round
% \geometry{landscape} % set up the page for landscape
%   read geometry.pdf for detailed page layout information

\usepackage{graphicx} % support the \includegraphics command and options

\usepackage[parfill]{parskip} % Activate to begin paragraphs with an empty line rather than an indent

%%% PACKAGES
\usepackage{booktabs} % for much better looking tables
\usepackage{array} % for better arrays (eg matrices) in maths
\usepackage{paralist} % very flexible & customisable lists (eg. enumerate/itemize, etc.)
\usepackage{verbatim} % adds environment for commenting out blocks of text & for better verbatim
\usepackage{subfig} % make it possible to include more than one captioned figure/table in a single float
% These packages are all incorporated in the memoir class to one degree or another...

%%% HEADERS & FOOTERS
\usepackage{fancyhdr} % This should be set AFTER setting up the page geometry
\pagestyle{fancy} % options: empty , plain , fancy
\renewcommand{\headrulewidth}{0pt} % customise the layout...
\lhead{}\chead{}\rhead{}
\lfoot{}\cfoot{\thepage}\rfoot{}

%%% SECTION TITLE APPEARANCE
\usepackage{sectsty}
\allsectionsfont{\sffamily\mdseries\upshape} % (See the fntguide.pdf for font help)
\usepackage{amsmath}
\usepackage{amsthm}
\usepackage{amsfonts}
\usepackage{mathrsfs}
\usepackage{amsopn}
\usepackage{amssymb}
\usepackage{natbib}
% (This matches ConTeXt defaults)

%%% ToC (table of contents) APPEARANCE
\usepackage[nottoc,notlof,notlot]{tocbibind} % Put the bibliography in the ToC
\usepackage[titles,subfigure]{tocloft} % Alter the style of the Table of Contents
\renewcommand{\cftsecfont}{\rmfamily\mdseries\upshape}
\renewcommand{\cftsecpagefont}{\rmfamily\mdseries\upshape} % No bold!

%Theorems and stuff
\newtheorem{defn}{Definition}
\newtheorem{thm}{Theorem}
\newtheorem{cor}{Corollary}
\newtheorem{lem}{Lemma}
\newtheorem{prop}{Proposition}

\DeclareMathOperator{\ord}{ord}

%%% END Article customizations

%%% The "real" document content comes below...

\title{Extension of ``Faithful Action on the space of Global Differentials of an Algebraic curve"}
\author{J Tait}
%\date{} % Activate to display a given date or no date (if empty),
         % otherwise the current date is printed 

\begin{document}
\maketitle

Let $\pi :X \rightarrow Y$ be a finite Galois covering of connected, complete, non-singular curves over an algebraically closed field $k$ of characteristic $p$, with genus $g_X$ and $g_Y$ respectively. Let $G=\mbox{Gal}(X/Y)$ and assume that $G$ is cyclic of order $p$. Let $P_1,\ldots ,P_r\in X$ be the ramification points of $\pi$. For $i=1,\ldots ,r$, let $N_i$ be defined by $N_i+1=\ord_{P_i}(\sigma\cdot\pi_i-\pi_i)$, where $\sigma$ is a generator of $G$, and $\pi_i$ is a funtion such that $\ord_{P_i}(\pi_i)=1$. We know from lemma 1 on p. 87 of \citep{naka} that $p\nmid N_i$. Also, for a rational number $a$ we will denote the fractional part by $<a>$.

Also, if we let $V$ be the $k[G]$ module with $k$-basis $\{e_1,\ldots ,e_p\}$ and $G$-action defined by $\sigma\cdot e_l=e_l+e_{l-1}$, with $1\leq l\leq p$ and $e_0=0$, then the subspaces $V_j$ spanned by $\{e_1,\ldots ,e_j\}$ are all the indecomposable $k[G]$ modules up to isomorphism.
\\
\begin{prop}
If $D$ is a $G$-invariant divisor on $X$ such that $\deg(D)>2g_X-2$ then the action of $G$ on $H^0(X,\mathscr{O}_X(D))$ is trivial if and only if
\[ (p-1)\deg(D)=p\Big(g_X-g_Y-\sum_{i=1}^r\Big<\frac{n_i}{p}\Big>\Big),\]
where the $n_i$ are the coefficients of the ramification points in $D$. 
\end{prop}

\begin{proof}
First note that the action is trivial if and only if \[ \dim_kH^0(X,\mathscr{O}_X(D))=\dim_kH^0(X,\mathscr{O}_X(D))^G.\]

Now from Nakajimas paper \citep{naka}, we know that $H^0(X,\mathscr{O}_X(D))\cong \oplus _{j=1}^p m_j\cdot V_j$, with 
	\begin{equation*}
		m_j=
			\begin{cases}
				\sum_{i=1}^r\Big(\frac{N_i}{p}+\Big{<}\frac{n_i-jN_i}													{p}\Big{>}-\Big{<}\frac{n_i-(j-1)N_i}{p}\Big{>}\Big) & \mbox{if }1\leq j\leq p-1 \\
				\frac{1}{p}\deg(D)-g_Y+1-\sum_{i=1}^r\Big(\frac{(p-1)N_i}{p}+\Big<\frac{n_i-(p-1)N_i}					{p}\Big>\Big) & \mbox{if }j=p.
			\end{cases}
	\end{equation*}
It is also clear that the only $G$-invariant part of each submodule $V_j$ is $e_1$. Hence $\dim_kH^0((X,\mathscr{O}_X(D))^G) = \sum_{j=1}^p m_j$.

Note that this sum cancels in a very natural manner; we have that
	\begin{eqnarray*}
		\dim_k(H^0((X,\mathscr{O}_X(D))^G) & = & \sum_{j=1}^p m_j \\
		& = & \sum_{j=1}^{p-1}  \sum_{i=1}^r\Big(\frac{N_i}{p}+\Big<\frac{n_i-jN_i}							{p}\Big>-\Big<\frac{n_i-(j-1)N_i}{p}\Big>\Big) \\
		& + & \frac{1}{p}\deg(D)-g_Y+1+\sum_{i=1}^r\Big(\frac{N_i}{p}-N_i-\Big<\frac{n_i-(p-1)N_i}				{p}\Big>\Big) \\
		& = & \frac{1}{p}\deg(D)-g_Y+1-\sum_{i=1}^r \Big<\frac{n_i}{p}\Big>.
	\end{eqnarray*}

Now as $\deg(D)>2g_X-2$ then $\dim_kH^0((X,\mathscr{O}_X(D))) =\deg(D)+1-g_X$ by the Riemann Roch theorem. So the action of $G$ on $H^0(X,\mathscr{O}_X(D))$ is trivial if and only if
	\begin{equation*}
		\deg(D)+1-g_X  = \frac{1}{p}\deg(D)-g_Y+1-\sum_{i=1}^r\Big<\frac{n_i}{p}\Big>. \label{hi}
	\end{equation*}

This then clearly rearranges to $(p-1)\deg(D)=p\Big(g_X-g_Y-\sum_{i=1}^r\Big<\frac{n_i}{p}\Big>\Big)$, as we want.
\end{proof}

\begin{cor}
Suppose that $\deg(D)\geq 2g_X$ and $g_Y=0$. Then the action of $G$ on $H^0(X,\mathscr{O}_X(D))$ is trivial if and only if $p | n_i$ for all $i$, $\deg(D)=2g_X$ and, unless $g_X=0$, $p=2$.
\end{cor}
\begin{proof}
The following string of inequalities always hold:
	\[
		(p-1)\deg(D)\geq (p-1)2g_X \geq pg_X \geq pg_X-p\sum_{i=1}^r\Big<\frac{n_i}{p}\Big>.
	\]
Now the first inequality is an equality if and only if $\deg(D)=2g_X$. The second is an equality if and only if either $g_X=0$ or $p=2$. Lastly, the third inequality is an equality if and only if $\sum_{i=1}^r\Big<\frac{n_i}{p}\Big>=0$, which is the case if and only if each $n_i$ is divisible by $p$. Now proposition 1 implies corollary 1.
\end{proof}

\begin{cor}
Suppose that $\deg(D)= 2g_X-1$ and $g_Y=0$. Then the action of $G$ on $H^0(X,\mathscr{O}_X(D))$ is trivial if and only if
	\begin{itemize}
		\item
			$g_X=0$ or
		\item
			$p=2$ and $\sum_{i=1}^r\Big<\frac{n_i}{p}\Big>=\frac{1}{2}$ or
		\item
			$g_X=1$ and $\sum_{i=1}^r\Big<\frac{n_i}{p}\Big>=\frac{1}{p}$ or
		\item
			$g_X=2$, $p=3$ and $p\mid n_i$ for all $i$.
	\end{itemize}.
It should be noted that in the last two cases the Hurwitz formula implies that $r\leq 2$. If $r=1$ then the conditions $\sum_{i=1}^r\Big<\frac{n_i}{p}\Big>=\frac{1}{p}$ and $p\mid n_i$ for all $i$ are already implied by $\deg(D)=2g_X-1$.
\end{cor}
\begin{proof}
Firstly, if $g_X=0$ then $\deg(D)=-1<0$, so $\dim_kH^0(X,\mathscr{O}_X(D))=0$ and the action is trivial.

Now note that as $\deg(D)=2g_X-1$ we conclude from proposition 1 that the action is trivial if and only if 
	\begin{equation*}
		(p-1)(2g_X-1)=p\Big(g_X-\sum_{i=1}^r\Big<\frac{n_i}{p}\Big>\Big).
	\end{equation*}
If $p=2$ then this is equivalent to $2g_X-1=2g_X-2\sum_{i=1}^r\Big<\frac{n_i}{p}\Big>$, and hence to $\sum_{i=1}^r\Big<\frac{n_i}{p}\Big>=\frac{1}{2}$.

If $g_X=1$ then this is equivalent to $p-1=p-p\sum_{i=1}^r\Big<\frac{n_i}{p}\Big>$, and hence to $\sum_{i=1}^r\Big<\frac{n_i}{p}\Big>=\frac{1}{p}$.

Lastly, if $p\geq 3$ and $g_X\geq 2$ then we have that $g_X\geq \frac{p-1}{p-2}$ which is equivalent to the first inequality in the chain
\begin{equation*}
	(p-1)(2g_X-1)\geq pg_X\geq pg_X-p\sum_{i=1}^r\Big<\frac{n_i}{p}\Big>.
\end{equation*}
Hence the action is trivial if and onlly if both inequlaities are equalities, which is the case if and only if $p=3,\ g_X=2$ and $p\mid n_i$ for all $i$.
\end{proof}
We will now assume that $K_X$ denotes a $G$-invariant divisor on $X$ such that $\mathscr{O}_X(K_X)\cong \Omega_X$ as $G$-sheaves. Remembering that $\Omega_X^{\otimes m}\cong \mathscr{O}_X(mK_X)$, we will now use these results to prove the following statement about the action of $G$ on the group $H^0(X,\Omega_X^{\otimes m})$, where $m\geq 2$. \\

\begin{cor}
Let $m\geq 2$. We assume that $p\geq 2$, that $G$ is a cyclic group of prime order $p$ and also that $g_Y=0$. Then $G$ acts trivially on $H^0(X,\Omega_X^{\otimes m})$ if and only if
	\begin{itemize}
		\item
			$g_X=0$ or
		\item
			$g_X=1$ or
		\item
			$p=g_X=m=2$.
	\end{itemize}
\end{cor}
\begin{proof}
If $g_X=0$ then $\mbox{deg}(mK_X)=-2m<0$; hence $\mbox{dim}_k(H^0(X,\Omega_X^{\otimes m}))=0$ by \citep[prop. 3, {\S}8]{fulton} and the action must be trivial.

We now look at the case when $g_X=1$. By \citep[Chap. IV,\ Example 1.3.6]{hart} we know that $K_X$ is equivalent to the zero divisor. Hence $H^0(X,\Omega_X^{\otimes m})$ is isomorphic to $H^0(X,\mathscr{O}_X(mK_X))=H^0(X,\mathscr{O}_X(0))=k$, the space of constant functions, which is $G$-invariant. Hence the action of $G$ on $H^0(X,\Omega_X^{\otimes m})$ is trivial.

If $g_X\geq 2$ then we have that $\deg(mK_X)\geq 2g_X$. So by Corollary 1, we have a trivial action if and only if $p=2,\ \deg(D)=2g_X$ and $p|n_i$ for all $i$. Now $\deg(mK_X)=2g_X$ means that $m(2g_X-2)=2g_X$, so $m(g_X-1)=g_X$, and hence that $m=g_X=2$. This proves the only-if-direction of the proposition. To prove the if-direction it suffices to show that the co-efficents of the ramification points $K_X$ (and hence of $mK_X$) are always divisible by $p$ if $p=2$. To start with note that we can write $K_X$ as $\pi^*(K_Y)+R$, where $R=\sum_{i=1}^r(N_i+1)(p-1)[P]$ is the ramification divisor. First note that as $p\nmid N_i$ and $p=2$ then $p\mid N_i+1$. Also, $\pi^*$ multiplies the co-efficents of ramification points by $p$, hence the coefficents of all ramification points of $\pi^*(K_Y)$ are divisible by $p$. So in the sum $\pi^*(K_Y)+R$ all co-efficients of ramification points are divisible by $p$, and we are done.
\end{proof}


\bibliography{/home/jtait/Desktop/Work/Bibliography/biblio.bib}
\bibliographystyle{plain}

\end{document}