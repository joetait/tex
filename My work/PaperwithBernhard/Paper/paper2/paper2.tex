% !TEX TS-program = pdflatex
% !TEX encoding = UTF-8 Unicode

% This is a simple template for a LaTeX document using the "article" class.
% See "book", "report", "letter" for other types of document.

\documentclass[11pt]{article} % use larger type; default would be 10pt

\usepackage[utf8]{inputenc} % set input encoding (not needed with XeLaTeX)

%%% Examples of Article customizations
% These packages are optional, depending whether you want the features they provide.
% See the LaTeX Companion or other references for full information.

%%% PAGE DIMENSIONS
\usepackage{geometry} % to change the page dimensions
\geometry{a4paper} % or letterpaper (US) or a5paper or....
% \geometry{margins=2in} % for example, change the margins to 2 inches all round
% \geometry{landscape} % set up the page for landscape
%   read geometry.pdf for detailed page layout information

\usepackage{graphicx} % support the \includegraphics command and options

\usepackage[parfill]{parskip} % Activate to begin paragraphs with an empty line rather than an indent

%%% PACKAGES
\usepackage{booktabs} % for much better looking tables
\usepackage{array} % for better arrays (eg matrices) in maths
\usepackage{paralist} % very flexible & customisable lists (eg. enumerate/itemize, etc.)
\usepackage{verbatim} % adds environment for commenting out blocks of text & for better verbatim
\usepackage{subfig} % make it possible to include more than one captioned figure/table in a single float
% These packages are all incorporated in the memoir class to one degree or another...

%%% HEADERS & FOOTERS
\usepackage{fancyhdr} % This should be set AFTER setting up the page geometry
\pagestyle{fancy} % options: empty , plain , fancy
\renewcommand{\headrulewidth}{0pt} % customise the layout...
\lhead{}\chead{}\rhead{}
\lfoot{}\cfoot{\thepage}\rfoot{}

%%% SECTION TITLE APPEARANCE
\usepackage{sectsty}
\allsectionsfont{\sffamily\mdseries\upshape} % (See the fntguide.pdf for font help)
\usepackage{amsmath}
\usepackage{amsthm}
\usepackage{amsfonts}
\usepackage{mathrsfs}
\usepackage{amsopn}
\usepackage{amssymb}
% (This matches ConTeXt defaults)

%%% ToC (table of contents) APPEARANCE
\usepackage[nottoc,notlof,notlot]{tocbibind} % Put the bibliography in the ToC
\usepackage[titles,subfigure]{tocloft} % Alter the style of the Table of Contents
\renewcommand{\cftsecfont}{\rmfamily\mdseries\upshape}
\renewcommand{\cftsecpagefont}{\rmfamily\mdseries\upshape} % No bold!

%Theorems and stuff
\newtheorem{defn}{Definition}
\newtheorem{thm}{Theorem}
\newtheorem{cor}{Corollary}
\newtheorem{lem}{Lemma}
\newtheorem{prop}{Proposition}

\newcommand{\cO}{{\cal O}}
\newcommand{\ra}{\rightarrow}
\newcommand{\NN}{{\mathbb N}}
\newcommand{\PP}{{\mathbb P}}
\newcommand{\ZZ}{{\mathbb Z}}
\newcommand{\cL}{{\cal L}}

\DeclareMathOperator{\ord}{ord}
\DeclareMathOperator{\di}{div}
\DeclareMathOperator{\cha}{char}
\DeclareMathOperator{\gal}{Gal}

%%% END Article customizations

%%% The "real" document content comes below...

\title{Paper}
\author{J Tait}
%\date{} % Activate to display a given date or no date (if empty),
         % otherwise the current date is printed 

\begin{document}
\maketitle

\begin{abstract}
  Given a faithful action of afinite group $G$ on an algebraic curve of genus at least 2, we
prove that the induced action on the space of global
holomorphic differentials is faithful as well unless the
characteristic of the base field is  2 and $G$ contains a
hyperelliptic involution.
We show also that the induced action on the poly-differentials is faithful unless [insert term for when $m=2$], 
the genus is precisely 2 and the group contains a hyperelliptic involution.

\end{abstract}

Let $X$ be a connected smooth projective algebraic curve over and algebraically closed field $k$ equipped with a faithful action of a finite group $G$ of order $n$.
Then $G$ also acts on the vector space $H^0(X,\Omega_X^{\otimes m})$ for $m\geq 1$ of (poly)-differentials on $X$.
Let $\pi:X\rightarrow Y$ be the projection from $X$ to the quotient curve $X/G$.
We wish to know whether $G$ acts faithfully on the spaces $H^0(X,\Omega_X^{\otimes m})$ for $m\geq 1$.
We give the following answer.
Let $g_X$ and $g_Y$ be the genera of $X$ and $Y$ respectively, and let $p$ denote the characteristic of $k$.
We recall the a hyper-elliptic involution of $X$ is an automorphism $\sigma$ of $X$ of order 2 such that the quotient curve $X/\langle \sigma \rangle$ is isomorphic to $\mathbb{P}_k^1$.\\

  \begin{thm}
    Suppose that $g_X\geq 2$ and let $m\geq1$. 
    Then $G$ does not act faithfully on $X$ if and only if $G$ contains a hyperelliptic involution and one of the following conditions holds:
      \begin{itemize}
	\item $m=1$ and $\cha(k)=2$
	\item $m=2$ and $g_X=2$.
      \end{itemize}
  \end{thm}
The proof of this theorem is given after the proof of Proposition \ref{dim1} below.\\

  \begin{cor}
    Let $g_X \ge 2$. 
    If $G$ does not act faithfully on $H^0(X,\Omega_X)$ then $p=2$, $g_Y =0$ and the projection $\pi$ is not tamely ramified.
  \end{cor}
  \begin{proof}
    By the previous theorem we have $p=2$ and there exists a hyper-elliptic involution $\sigma \in G$. 
    Then the Hurwitz formula (see Corollary~2.4 on p.~301 in \cite{Ha}) applied to the projection $X\rightarrow X/\langle \sigma \rangle \cong \PP^1_k$ shows that $X\rightarrow X/\langle \sigma \rangle$ is not unramified and hence not tamely ramified; then $\pi$ is not tamely ramified either. 
    And the Hurwitz formula applied to the projection $\PP^1_k\cong X/\langle \sigma \rangle \ra  Y$ shows that the genus of $Y$ is $0$ as well.
  \end{proof}

The following example describes some (mostly trivial) cases when the action of $G$ on $H^0(X, \Omega_X^{\otimes m})$ (for some $m\geq 2$) is in fact trivial.

{\bf Example.}\\
  (a) If $g_X = 0$ then $G$ obviously acts trivially on $H^0(X,\Omega_X^{\otimes m}) = \{0\}$ for $m\geq 1$.\\

  (b) Suppose $g_X =1 $ (i.e. suppose $X$ is an elliptic curve) and that $G$ is a finite subgroup of $X(k)$ acting on $X$ by translations.
  Then $G$ leaves invariant any global non-vanishing holomorphic differential and hence $G$ acts trivially on $H^0(X,\Omega_X)$.\\

  (c) Let $p=2$. 
  If $n=2$ and $g_Y =0$, then $G \cong \ZZ/2\ZZ$ acts trivially on $H^0(X, \Omega_X)$ by Proposition~2 below.
  For instance, let $r$ be an odd natural number, let $k(x,y)$ be the extension of the rational function field $k(x)$ given by the Artin-Schreier equation $y^2-y = x^r$ and let $\pi: X \rightarrow \PP^1_k$ be the corresponding cover of non-singular curves over $k$; 
  then $G$ acts trivially on the vector space $H^0(X,\Omega_X)$ whose dimension is $\frac{r-1}{2}$ by Example~2.5 on p.~1095 in \cite{Ko}.

The next lemma is crucial for the proof of Proposition~1 which in turn is the main idea for the proof of our theorem. 
We begin by introducing some notations. 
For any $G$-invariant divisor $D$ on $X$ let $\cO_X(D)$ denote the corresponding equivariant invertible $\cO_X$-module, as usual. 
Furthermore let $\pi_*^G(\cO_X(D))$ denote the sub-sheaf of the direct image $\pi_*(\cO_X(D))$ fixed by the obvious action of $G$ on $\pi_*(\cO_X(D))$ and let $\left\lfloor \frac{\pi_*(D)}{n}
\right \rfloor$ denote the divisor on $Y$ obtained from the push-forward $\pi_*(D)$ by replacing the co-efficient $m_Q$ of $Q$ in $\pi_*(D)$ with the integral part $\left \lfloor \frac{m_Q}{n} \right \rfloor$ of $\frac{m_Q}{n}$ for every $Q \in Y$. 
The function fields of $X$ and $Y$ are denoted by $K(X)$ and $K(Y)$, respectively. 
Finally, for any $P \in X$, let $e_P$ denote the ramification index of $\pi$ at $P$ and let $\textrm{ord}_P$ and $\textrm{ord}_Q$ denote the respective valuations of $K(X)$ and $K(Y)$ at $P$ and $Q:=\pi(P)$. \\

  \begin{lem}
    Let $D$ be a $G$-invariant divisor on $X$.
    Then the sheaves $\pi_*^G(\cO_X(D))$ and $\cO_Y\left(\left\lfloor \frac{\pi_*(D)}{n}\right \rfloor\right)$ are equal as sub-sheaves of the constant sheaf $K(Y)$ on $Y$. 
    In particular the sheaf $\pi_*^G(\cO_X(D))$ is an invertible $\cO_Y$-module.
  \end{lem}
  \begin{proof}
    For every open subset $V$ of $Y$ we have 
      \[
	 \pi_*^G(\cO_X(D))(V) = \cO_X(D) (\pi^{-1}(V))^G \subseteq K(X)^G = K(Y).
      \]
    In particular both sheaves are sub-sheaves of the constant sheaf $K(Y)$ as stated. 
    It therefore suffices to check that their stalks are equal. 
    Let $Q \in Y$, let $P \in \pi^{-1}(Q)$ and let $n_P$ denote the coefficient of $D$ at $P$. 
    Then we have
      \begin{eqnarray*}
	 \lefteqn{\pi_*^G\left(\cO_X(D)\right)_Q = \cO_X(D)_P \cap K(Y)}\\
	  &=& \left\{f \in K(Y): \textrm{ord}_P(f) \ge -n_P\right\}\\
	  &=& \left\{f \in K(Y): \textrm{ord}_Q(f) \ge - \frac{n_P}{e_P}\right\}\\
	  &=& \left\{ f \in K(Y): \textrm{ord}_Q(f) \ge - \left\lfloor\frac{n_P}{e_P} \right\rfloor \right\}\\
	  &=& \cO_Y\left(\left\lfloor \frac{\pi_*(D)}{n} \right\rfloor\right)_Q,
      \end{eqnarray*}
    as desired.
  \end{proof}

Let $R := \sum_{P\in X} \textrm{dim}_k (\Omega_{X/Y}) [P]$ denote the ramification divisor of $\pi$. 
The following proposition computes the dimension of the subspace of $H^0(X,\Omega_X^{\otimes m})$ fixed by $G$.

  \begin{prop}\label{dim1}
    The variables $m,\ g_Y$ and $\deg \left\lfloor\frac{m\pi_*(R)}{n} \right\rfloor$ completely determine $\dim_k(H^0(X,\Omega_X^{\otimes m})^G)$.
    The dimension of $H^0(X,\Omega_X^{\otimes m})^G$ is:
      \begin{itemize}
	\item $g_Y$ if $m=1$ and $\deg \left\lfloor\frac{m\pi_*(R)}{n} \right\rfloor=0$;
	\item $1$ if $m\geq 1$, $g_Y=1$ and $\deg \left\lfloor\frac{m\pi_*(R)}{n} \right\rfloor=0$;
	\item $0$ if $m\geq 1$, $g_Y=0$ and $\deg \left\lfloor\frac{m\pi_*(R)}{n} \right\rfloor< 2m$;
	\item and $(2m-1)(g_Y-1)+\deg \left\lfloor\frac{m\pi_*(R)}{n} \right\rfloor$ otherwise.
      \end{itemize}
  \end{prop}
  \begin{proof}
    Let $K_X$ be a $G$-invariant canonical divisor on $X$, that is we have an equivariant isomorphism $\cO_X(K_X)\cong \Omega_X$. 
    Let the divisor $K_Y$ on $Y$ be defined by the equality $\pi^*(\Omega_Y) = \cO_X(\pi^*(K_Y))$ of sub-sheaves of $\cO_X(K_X)$. 
    Note that we consider $\pi^*(\Omega_Y)$ as a sub-sheaf of $\Omega_X \cong \cO_X(K_X)$ and that we have a short exact sequence
      \[0 \rightarrow \pi^* \Omega_Y \rightarrow \Omega_X \rightarrow \Omega_{X/Y} \rightarrow 0. \]
    In particular we have $K_X = \pi^* K_Y + R$ and hence
      \[\left\lfloor \frac{\pi_*(mK_X)}{n} \right \rfloor = 
      \left \lfloor \frac{\pi_*\pi^*(mK_Y) + \pi_*(mR)}{n} \right \rfloor =
      mK_Y + \left \lfloor \frac{m\pi_*(R)}{n} \right\rfloor.\]
    Using the previous lemma we conclude that $\pi_*^G(\Omega_X^{\otimes m}) \cong \cO_Y\left(mK_Y + \left \lfloor \frac{m\pi_*(R)}{n}\right\rfloor\right)$ and finally that
      \begin{eqnarray*}
	\lefteqn{\textrm{dim}_k \left(H^0(X,\Omega_X^{\otimes m})^G \right)}\\
	& = & \textrm{dim}_k \left(H^0\left(Y, \pi_*^G(\Omega_X^{\otimes m})\right)\right) \\
	 & = & \textrm{dim}_k \left(H^0\left(Y, \cO_Y\left(mK_Y+ \left\lfloor \frac{m\pi_*(R)}{n}\right\rfloor \right) \right) \right).
      \end{eqnarray*}
  In the first case in the proposition, if $m=1$ and $\deg \left\lfloor\frac{m\pi_*(R)}{n} \right\rfloor=0$ then $\left\lfloor\frac{m\pi_*(R)}{n} \right\rfloor$ is the zero divisor and we conclude that 
    \begin{equation*}
	\dim_k\left(H^0(X,\Omega_X)^G\right) = \dim_k\left(H^0(Y, \Omega_Y)\right) = g_Y
    \end{equation*}


  In the second case then $mK_Y+\left\lfloor \frac{m\pi_*(R)}{n} \right\rfloor$ is equivalent to the zero divisor; 
  $g_Y=1$ implies $K_Y$ is equivalent to $0$, and hence $mK_Y$ is too; 
  as $\left( \left\lfloor \frac{m\pi_*(R)}{n} \right\rfloor \right)$ is positive and of degree zero it is the the zero divisor. 
  Overall, this means
    \begin{eqnarray*}
      \dim_k\left( H^0\left( Y,\cO_Y\left( mK_Y+\left\lfloor \frac{m\pi_*(R)}{n} \right\rfloor \right) \right) \right) & = & \dim_k \left( H^0\left( Y,\cO_Y\left( 0 \right) \right) \right)\\
      & = & 1.
    \end{eqnarray*}


  In the third case it suffices to show that $\deg \left( mK_Y+\left\lfloor \frac{m\pi_*(R)}{n} \right\rfloor \right) < 0$.
  Now $g_Y=0$ then $\deg(K_Y)=-2$, so $\deg(mK_Y)=-2m$. 
  As $\deg \left\lfloor\frac{m\pi_*(R)}{n} \right\rfloor<2m$, then $\deg \left( mK_Y+\left\lfloor \frac{m\pi_*(R)}{n} \right\rfloor \right)$ is indeed negative.



  We now show that in all other cases the divisor is ample, and then the Riemann-Roch formula will give 
     \begin{eqnarray*}
	\dim_k\left(H^0\left(Y,\mathscr{O}_Y\left(mK_Y+\left\lfloor{\frac{m\pi_*(R)}{n}}\right\rfloor \right)\right)\right)= & 1-g_Y+\deg\left(mK_Y+\left\lfloor{\frac{m\pi_*(R)}{n}}\right\rfloor\right) \\
	= & (2m-1)(g_Y-1)+\deg\left(\left\lfloor{\frac{m\pi_*(R)}{n}}\right\rfloor\right),
      \end{eqnarray*}
  completing the proof.


  All that remains is to show that $\deg(D)>\deg(K_Y)$ in all other cases.
  Firstly, if $g_Y=0$ and $\deg \left\lfloor\frac{m\pi_*(R)}{n} \right\rfloor \geq 2m$ then as $mK_Y=-2m$ we have $\deg \left( mK_Y+\left\lfloor \frac{m\pi_*(R)}{n} \right\rfloor \right) \geq 0 > \deg(K_Y)$.
  Similarly, if $g_Y=1$ and $\deg \left\lfloor\frac{m\pi_*(R)}{n} \right\rfloor >0$ then as $\deg \left( mK_Y \right)=0$ we have $\deg \left( mK_Y+\left\lfloor \frac{m\pi_*(R)}{n} \right\rfloor \right) > 0 = \deg (K_Y)$.
  If $m=1$ and $\deg \left\lfloor\frac{m\pi_*(R)}{n} \right\rfloor >0$ then clearly $\deg \left( K_Y+ \left\lfloor \frac{\pi_*(R)}{n} \right\rfloor \right) > \deg (K_Y)$.
  Lastly, if $m\geq 2$ and $g_Y\geq 2$ then as $\deg (K_Y) > 0$ we have 
    \begin{equation*}
      \deg \left( mK_Y + \left\lfloor \frac{m\pi_*(R)}{n} \right\rfloor \right) \geq \deg\left( mK_Y \right) > \deg (K_Y).
    \end{equation*}
  So we have shown that in all cases the divisor is ample, and the proof is complete.
  \end{proof}



  \begin{thm}
    Suppose that $g_X\geq 2$ and let $m\geq1$. 
    Then $G$ does not act faithfully on $X$ if and only if $G$ contains a hyper-elliptic involution and one of the following conditions holds:
      \begin{itemize}
	\item $m=1$ and $\cha(k)=2$
	\item $m=2$ and $g_X=2$.
      \end{itemize}
  \end{thm}
  \begin{proof}
    We first show the if direction, supposing that $G$ does not act faithfully on $H^0(X,\Omega_X^{\otimes m})$. 
    By replacing $G$ with the (non-trivial) kernel $H$ if necessary, we may assume that $G$ is non-trivial and acts trivially on $H^0(X,\Omega_X^{\otimes m})$.


    We start the proof by showing $g_Y=0$, which is shown separately for the cases when $m=1$ and when $m\geq 2$.
    In the case when $m=1$ we start by showing $\deg \left( \left\lfloor \frac {\pi_*(R)}{n} \right\rfloor \right)>0$ by contradiction.
    Suppose that $\deg \left( \left\lfloor \frac{\pi_*(R)}{n} \right\rfloor \right) =0$.
    As $G$ acts trivially then by Proposition \ref{dim1} we have:
      \begin{equation*}
	g_X=\dim_k H^0(X,\Omega_X)=\dim_k H(Y,\Omega_Y)=g_Y.
      \end{equation*}
    Substituting this in to the Hurwitz formula 
      \begin{equation}\label{eq:hur}
	2(g_X-1)=2n(g_Y-1)+\deg(R)
      \end{equation}
    yields the desired contradiction as $g_X\geq 2, n\geq 2$ and $\deg(R)>0$.

    Now we show that $g_Y=0$ when $m=1$. 
    As $\deg\left( \left\lfloor \frac{\pi_*(R)}{n} \right\rfloor \right) >0$ then Proposition \ref{dim1} gives us that
      \begin{equation*}
	g_X=g_Y-1+\deg\left\lfloor \frac{\pi_*(R)}{n} \right\rfloor.
      \end{equation*}
    Substituting this in to the Hurwitz formula \eqref{eq:hur}, we see that
      \begin{equation*}
	2\left(g_Y - 1 + \textrm{deg}\left \lfloor \frac{\pi_*(R)}{n} \right \rfloor -1 \right) = 2n (g_Y -1) + \textrm{deg}(R).
      \end{equation*}
    For any $Q \in Y$ let $n_Q$ denote the coefficient of the ramification divisor $R$ at any $P \in \pi^{-1}(Q)$ and let $e_Q := e_P$ for any $P \in \pi^{-1}(Q)$. 
    Rewriting the previous equation yields
      \begin{eqnarray*}
	\lefteqn{(2n-2)g_Y = 2n-4 + 2 \,\textrm{deg}\left \lfloor \frac{\pi_*(R)}{n}\right \rfloor - \textrm{deg}(R)}\\
	&=& 2 \left(n-2 + \sum_{Q \in Y} \left(\left\lfloor \frac{n}{e_Q} \frac{n_Q}{n} \right\rfloor - \frac{n}{e_Q} \frac{n_Q}{2}\right) \right)\\
	&=& 2 \left(n-2 + \sum_{Q \in Y} \left( \left\lfloor \frac{n_Q}{e_Q} \right\rfloor - \frac{n_Q}{e_Q} \frac{n}{2} \right)\right)\\
	& \le & 2(n-2)
      \end{eqnarray*}
    because $\frac{n}{2} \ge 1$ and $\left\lfloor \frac{n_Q}{e_Q}\right\rfloor \le \frac{n_Q}{e_Q}$ for all $Q \in Y$. 
    Hence we obtain $g_Y \le \frac{n-2}{n-1} < 1$ and therefore $g_Y =0$, as desired.

    We now show that $g_Y=0$ when $m\geq 2$. 
    Since $g_X\geq 2$ then we have that $\deg(mK_X)=m(2g_X-2)>2g_X-2=\deg(K_X)$. 
    So by the Riemann-Roch theorem,
      \begin{equation}\label{eq:riem}
	\dim_k(H^0(X,\Omega_X^{\otimes m}))=\deg(mK_X)+1-g_X=2m(g_X-1)+1-g_X=(2m-1)(g_X-1).
      \end{equation}
    But, again as both $m$ and $g_X$ are at least 2, then $(2m-1)(g_X-1)>1$.
    There is only one case in Proposition \ref{dim1} such that $m\geq 2$ and $\dim_k H^0(X,\Omega_X^{\otimes m})>1$, so 
      \begin{equation*}
	(2m-1)(g_X-1)=(2m-1)(g_Y-1)+\deg\left(\left\lfloor \frac{n\pi_*(R)}{n} \right\rfloor \right).
      \end{equation*}
    Combining this with the Hurwitz formula we see that
      \begin{eqnarray*}
	2(2m-1)(g_Y-1)+2\deg\left(\left\lfloor\frac{m\pi_*(R)}{n}\right\rfloor\right) & = & 2(2m-1)(g_X-1)\\
	& = & 2n(2m-1)(g_Y-1)+(2m-1)\deg(R)
      \end{eqnarray*}
    which can be re-arranged as
      \begin{equation*}
	(2m-1)(2n-2)(g_Y-1)=2\deg\left(\left\lfloor\frac{m\pi_*(R)}{n}\right\rfloor\right)-(2m-1)\deg(R).
      \end{equation*}
    So if we can show that $2\deg\left(\left\lfloor\frac{m\pi_*(R)}{n}\right\rfloor\right)-(2m-1)\deg(R)<0$ then we will have $g_Y-1<0$ and hence $g_Y=0$.

    Using the same notation as before we show this here;
    
      \begin{align*}
	\lefteqn{2\deg\left(\left\lfloor\frac{m\pi_*(R)}{n}\right\rfloor\right)-(2m-1)\deg(R)}\\
	& = \sum_{Q \in Y} \left(2\left\lfloor m\cdot \frac{n}{e_Q}\frac{n_Q}{n}\right\rfloor -n(2m-1)\left(\frac{n_Q}{e_Q}\right)\right) \\
	&\leq   \sum_{Q\in Y}\left( m\cdot\frac{n_Q}{e_Q}-n(2m-1)\frac{n_Q}{e_Q}\right) \\
	& =  (2m-n(2m-1))\sum_{Q\in Y }\frac{n_Q}{e_Q}.
      \end{align*}

    Now as $n,m\geq 2$ then we have $2m-n(2m-1)\leq 2m-2(2m-1)=2(1-m)<0$ and we are done as $\sum_{Q\in Y}\frac{n_Q}{e_Q}$ is positive.

    So we have shown that in all cases, if $G$ does not act faithfully then $g_Y=0$. 
    In the case when $m\geq 2$ then the result will follow from Proposition \ref{triv}, which shows that a cyclic group acts trivially on $H^0(X,\Omega^{\otimes m})$ for $m\geq 2$ if and only if it is of order $2$, and $m=g_X=2$.
    
    The $m=1$ case will follow from the next proposition after we show that there is a cyclic subgroup of order $p$ that acts trivially. 
    First note that $\pi$ must be wildly ramified.
    Indeed, if it were not then $R=\sum_{P\in X} (e_P-1)[P]$, and $\deg\left\lfloor \frac{\pi_*(R)}{n} \right\rfloor=0$, which we have already shown cannot be the case.
    Hence the characteristic of $k$ must be positive, and there is a cyclic subgroup of order $p$ which acts trivially.
  \end{proof}

  \begin{prop}
     Let $p  > 0$ and let $G$ be cyclic of order $p$. 
    We furthermore assume that $g_Y = 0$. 
    Then $G$ acts trivially on $H^0(X,\Omega_X)$ if and only if one of the following three conditions holds:\\
      (i) $p=2$.\\
      (ii) $g_X =0$.\\
      (iii) $p=3$ and $g_X=1$.
  \end{prop}
  \begin{proof}
    Let $P_1, \ldots, P_r \in X$ be the ramified points of $\pi: X \ra Y$ and, for $i=1, \ldots, r$, define $N_i \in \NN$ by $\textrm{ord}_{P_i}(\sigma(\pi_i) - \pi_i) = N_i +1$ where $\pi_i$ is a local parameter at $P_i$ and $\sigma$ is a generator of $G$. 
    From Lemma~1 on p.~87 in \cite{Na} we know that $p$ does not divide $N_i$, a fact we will use several times below. 
    The ramification divisor $R$ of $\pi$ is equal to $\sum_{i=1}^r(N_i+1)(p-1)[P_i]$ by Hilbert's formula for the order of the different (see Prop.~4, \S 1, Ch.~IV on p.~72 in \cite{Se}). 
    Let $N:= \sum_{i=1}^r N_i$. 
    Using the Hurwitz formula we obtain
      \[
	2g_X - 2 = -2p + (N+r)(p-1)
      \] 
    and hence
      \[
	\textrm{dim}_k\left(H^0(X,\Omega_X)\right) = g_X =\frac{(N+r-2)(p-1)}{2}.
      \] 
    Since $g_X \ge 0$ we obtain $r \ge 1$; that is, $\pi$ is not unramified. 
    Therefore we have 
      \[
	\textrm{deg} \left\lfloor \frac{\pi_*(R)}{p} \right\rfloor =
	\sum_{i=1}^r \left\lfloor \frac{(N_i+1)(p-1)}{p}\right\rfloor 
	\ge \sum_{i=1}^r \left\lfloor \frac{2(p-1)}{p}\right\rfloor = r > 0.
      \] 
    From Proposition~1 we then conclude that
      \begin{eqnarray*}
	\lefteqn{\textrm{dim}_k \left(H^0(X,\Omega_X)^G\right) = g_Y -1 + \textrm{deg} \left\lfloor \frac{\pi_*(R)}{p} \right\rfloor}\\
	&=& -1 + \sum_{i=1}^r \left\lfloor \frac{(N_i+1)(p-1)}{p}\right\rfloor\\
	&=& -1 + N +r + \sum_{i=1}^r \left\lfloor -\frac{N_i+1}{p}\right\rfloor.
      \end{eqnarray*}
    If $p=2$ the dimension of both $H^0(X,\Omega_X)$ and $H^0(X,\Omega_X)^G$ is therefore equal to $\frac{N+r-2}{2}$. 
    If $g_X = 0$ both dimensions are obviously equal to~$0$.
    If $p=3$ and $g_X =1$ we obtain $N+r=3$ and hence $r=1$ and $N=2$; thus both dimensions are equal to $1$.
    Therefore in all three of these cases $G$ acts trivially on $H^0(X,\Omega_X)$.
    This finishes the proof of the if-direction in Proposition~2.\\
    To prove the other direction we now assume that $G$ acts trivially on $H^0(X, \Omega_X)$ and that $p \ge 3$ and prove that condition~(ii) or condition~(iii) holds. 
    For each $i=1, \ldots, r$, we write $N_i = s_i p +t_i$ with $s_i \in \NN$ and $t_i \in \{1, \ldots, p-1\}$. 
    We furthermore put $S:=\sum_{i=1}^r s_i$ and $T:= \sum_{i=1}^r t_i \ge r$. 
    Then we have
      \[ 
	 \frac{(N+r-2)(p-1)}{2} =\textrm{dim}_k(H^0(X,\Omega_X))  = \textrm{dim}_k\left(H^0(X,\Omega_X)^G\right) = N-S-1 .
      \]
    Rearranging this equation we obtain
      \[
	 (3-p)N - 2S = (r-2)(p-1) +2  
      \]
    and hence
      \[
	 (-p^2 + 3p -2)S = (r-2)(p-1) +2 - (3-p)T.
      \]
    Since $-p^2+3p-2 = - (p-1)(p-2)$ and $p \ge 3$ this equation implies that
      \[ 
	S = \frac{(r-2)(1-p)-2 + T (3-p)}{(p-1)(p-2)}. 
      \]
    Because $S \ge 0$ the numerator of this fraction is non-negative, that is
      \begin{eqnarray*}
	\lefteqn{0 \le (r-2)(1-p) - 2 + T (3-p)}\\
	&\le & (r-2)(1-p) - 2 + r (3-p)\\
	&=& 2 (r-1)(2-p).
      \end{eqnarray*}
    Hence we have $r=1$ and that numerator is $0$. 
    We conclude that $S=0$ and hence that $T=1$ or $p=3$. 
    If $T=1$ we also have $N=1$ and finally
      \[
	g_X = \frac{(N+r-2)(p-1)}{2} = 0,
      \]
    i.e.\ condition~(ii) holds. 
    If $T \not=1$ and $p=3$ we obtain $N=T=2$ and finally 
      \[
	g_X = \frac{(N+r-2)(p-1)}{2} =1,
      \] 
    i.e.\
    condition~(iii) holds.
  \end{proof}

Having completed the proof of the main theorem for the case when $m=1$ we now turn to the case where $m\geq 2$.
We use \cite{Na} to prove some results regarding when a group acts trivially on $H^0(X,\cO(D))$ for a $G$-invariant divisor $D$ of degree greater than $2g_X-2$.
This is both for the interest of the reader and to shorten the proof of Proposition \ref{triv} which completes the proof of Theorem 1.
We first remind the reader of the notation in \cite{Na}.
Let $G$ be a cyclic group of order $p=\cha (k)$ and let $\sigma$ be a generator of $G$.
Let $V$ be a $k[G]$ module with $k$-basis $e_1,\ldots ,e_p$ and $G$ action defined by $\sigma\cdot e_i=e_i+e_{i-1}$, $1\leq i \leq p,\ e_0=0$.
Then define $V_j$ to be the subspace of $V$ spanned $e_1,\ldots ,e_j$ over $k$ is also a $k[G]$ module.
In fact, the sub-modules $V_1,\ldots ,V_p$ form the set of all irreducible sub-modules of $V$ up to isomorphism.
Hence, for any $G$ invariant divisor $D$ then $H^0(X,\cL (D))\cong \bigoplus_{j=1}^p m_j\cdot V_j$ as $k[G]$ modules for some $m_j$. \\

  \begin{prop}\label{nakaj}
    If $D$ is a $G$-invariant divisor on $X$ such that $\deg(D)>2g_X-2$ then the action of $G$ on $H^0(X,\mathscr{O}_X(D))$ is trivial if and only if
      \[ 
	(p-1)\deg(D)=p\Big(g_X-g_Y-\sum_{i=1}^r\Big<\frac{n_i}{p}\Big>\Big).
      \]
    Here, the $n_i$ are the coefficients of the ramification points in $D$, and $\langle a \rangle$ denotes the fractional part of $a$, i.e. $a-\lfloor a \rfloor$.
  \end{prop}
  \begin{proof}
    First note that the action is trivial if and only if
      \[
	\dim_k H^0(X,\mathscr{O}_X(D))=\dim_k H^0(X,\mathscr{O}_X(D))^G.
      \]
    Now from Nakajima's paper \cite{Na}, we know that $H^0(X,\mathscr{O}_X(D))\cong \oplus _{j=1}^p m_j\cdot V_j$, with 

      \begin{equation*}
      m_j=
	\begin{cases}
	    \sum_{i=1}^r\left(\frac{N_i}{p}+\left\langle \frac{n_i-jN_i}{p}\right\rangle -\left\langle \frac{n_i-(j-1)N_i}{p}\right\rangle \right) & \mbox{if }1\leq j\leq p-1 \\
	    \frac{1}{p}\deg(D)-g_Y+1-\sum_{i=1}^r\left(\frac{(p-1)N_i}{p}+\left\langle \frac{n_i-(p-1)N_i}{p}\right\rangle \right) & \mbox{if }j=p.
	\end{cases}
      \end{equation*}

	
    It is also clear that the only $G$-invariant part of each sub-module $V_j$ is $e_1$. 
    Hence $\dim_kH^0((X,\mathscr{O}_X(D))^G) = \sum_{j=1}^p m_j$.

    Note that this sum cancels in a very natural manner; we have that
      \begin{eqnarray*}
	\lefteqn{\dim_k(H^0((X,\mathscr{O}_X(D))^G) = \sum_{j=1}^p m_j} \\
	& = & \sum_{j=1}^{p-1}  \sum_{i=1}^r\left(\frac{N_i}{p}+\left\langle\frac{n_i-jN_i}{p}\right\rangle-\left\langle\frac{n_i-(j-1)N_i}{p}\right\rangle\right) \\
	& + & \frac{1}{p}\deg(D)-g_Y+1+\sum_{i=1}^r\left(\frac{N_i}{p}-N_i-\left\langle\frac{n_i-(p-1)N_i}{p}\right\rangle\right) \\
	& = & \frac{1}{p}\deg(D)-g_Y+1-\sum_{i=1}^r \left\langle\frac{n_i}{p}\right\rangle.
      \end{eqnarray*}

    Now as $\deg(D)>2g_X-2$ then $\dim_kH^0(X,\mathscr{O}_X(D)) =\deg(D)+1-g_X$ by the Riemann Roch theorem. 
    So the action of $G$ on $H^0(X,\mathscr{O}_X(D))$ is trivial if and only if
      \begin{equation*}
	\deg(D)+1-g_X  = \frac{1}{p}\deg(D)-g_Y+1-\sum_{i=1}^r\left\langle\frac{n_i}{p}\right\rangle. \label{hi}
      \end{equation*}

    This then rearranges to $(p-1)\deg(D)=p\left(g_X-g_Y-\sum_{i=1}^r\left\langle\frac{n_i}{p}\right\rangle\right)$, as desired.
    \end{proof}

  \begin{cor}
    Suppose that $\deg(D)\geq 2g_X$ and $g_Y=0$. Then the action of $G$ on $H^0(X,\mathscr{O}_X(D))$ is trivial if and 
    only if $p | n_i$ for all $i$, $\deg(D)=2g_X$ and either $g_X=0$ or $p=2$.
  \end{cor}
  \begin{proof}
    The following inequalities always hold under the stated assumptions:
      \[
	(p-1)\deg(D)\geq (p-1)2g_X \geq pg_X \geq pg_X-p\sum_{i=1}^r\left\langle\frac{n_i}{p}\right\rangle.
      \]
    Now the first inequality is an equality if and only if $\deg(D)=2g_X$. 
    The second is an equality if and only if either $g_X=0$ or $p=2$. 
    Lastly, the third inequality is an equality if and only if $\sum_{i=1}^r\left\langle\frac{n_i}{p}\right\rangle=0$, which is the case if and only if each $n_i$ is divisible by $p$. 
    Given these observations, Proposition \ref{nakaj} implies this Corollary.
  \end{proof}

  \begin{cor}
    Suppose that $\deg(D)= 2g_X-1$ and $g_Y=0$. Then the action of $G$ on $H^0(X,\mathscr{O}_X(D))$ is trivial if and only if one of the following conditions hold:
      \begin{itemize}
	\item
	  $g_X=0$
	\item
	  $p=2$ and $\sum_{i=1}^r\left\langle\frac{n_i}{p}\right\rangle=\frac{1}{2}$
	\item
	  $g_X=1$ and $\sum_{i=1}^r\left\langle\frac{n_i}{p}\right\rangle=\frac{1}{p}$
	\item
	  $g_X=2$, $p=3$ and $p\mid n_i$ for all $i$.
      \end{itemize}
  \end{cor}
It should be noted that in the last two cases the Hurwitz formula implies that $r\leq 2$. 
If $r=1$ then the conditions ``$\sum_{i=1}^r\left\langle\frac{n_i}{p}\right\rangle=\frac{1}{p}$" and ``$p\mid n_i$ for all $i$" are already implied by ``$\deg(D)=2g_X-1$".
  \begin{proof}
    Firstly, if $g_X=0$ then $\deg(D)=-1<0$, so $\dim_kH^0(X,\mathscr{O}_X(D))=0$ and the action is trivial.

    Now note that as $\deg(D)=2g_X-1$ we conclude from Proposition \ref{nakaj} that the action is trivial if and only if 
      \begin{equation*}
	(p-1)(2g_X-1)=p\left(g_X-\sum_{i=1}^r\left\langle\frac{n_i}{p}\right\rangle\right).
      \end{equation*}
    If $p=2$ then this is equivalent to $2g_X-1=2g_X-2\sum_{i=1}^r\left\langle\frac{n_i}{p}\right\rangle$; i.e. $\sum_{i=1}^r\left\langle\frac{n_i}{p}\right\rangle=\frac{1}{2}$.

    If $g_X=1$ then this is equivalent to $p-1=p-p\sum_{i=1}^r\left\langle\frac{n_i}{p}\right\rangle$; i.e. $\sum_{i=1}^r\left\langle\frac{n_i}{p}\right\rangle=\frac{1}{p}$.

    Lastly, if $p\geq 3$ and $g_X\geq 2$ then we have that $g_X\geq \frac{p-1}{p-2}$ which is equivalent to the first inequality in the chain
      \begin{equation*}
	(p-1)(2g_X-1)\geq pg_X\geq pg_X-p\sum_{i=1}^r\left\langle\frac{n_i}{p}\right\rangle.
      \end{equation*}
    Hence the action is trivial if and only if both inequalities are equalities, which is the case if and only if $p=3,\ g_X=2$ and $p\mid n_i$ for all $i$.
  \end{proof}

The following proposition now completes the proof of Theorem 1 for the case when $m\geq 2$, thus finishing the entire proof.\\

  \begin{prop}\label{triv}
    Let $m \geq 2$. 
    Suppose $G$ is a cyclic group of prime order $l$ (which may or may not be equal to $\cha(k)=p$). 
    Then $G$ acts trivially on $H^0(X,\Omega_X^{\otimes m})$ if and only if one of the following conditions holds:
      \begin{itemize}
	\item $g_X=0$.
	\item $g_X=1$.
	\item $g_X=m=l=2$.
      \end{itemize}
  \end{prop}
  \begin{proof}
    If $g_X=0$ then $\deg(mK_X)=-2m<0$; hence $\dim_k(H^0(X,\Omega_X^{\otimes m}))=0$ by \cite[prop. 3, {\S}8]{fulton} and the action must be trivial.

    We now look at the case when $g_X=1$. By \cite[Chap. IV,\ Example 1.3.6]{Ha} we know that $K_X$ is equivalent to the zero divisor. 
    Hence $H^0(X,\Omega_X^{\otimes m})\cong H^0(X,\mathscr{O}_X(mK_X)) \cong H^0(X,\mathscr{O}_X(0))=k$, the space of constant functions, which is $G$ invariant. Hence the action of $G$ on $H^0(X,\Omega_X^{\otimes m})$ is trivial.



    First we suppose that $l=p$, and recall that $g_Y=0$.
    As $g_X\geq 2$ then we have that $\deg(mK_X)\geq 2g_X$. 
    So by Corollary 3, the action is trivial if and only if $p=2,\ \deg(mK_X)=2g_X$ and $p|n_i$ for all $i$. 
    Now $\deg(mK_X)=2g_X$ means that $m(2g_X-2)=2g_X$, so $m(g_X-1)=g_X$, and hence that $m=g_X=2$. 
    This proves the only-if-direction of the proposition. 
    To prove the if-direction it suffices to show that the co-efficients of the ramification points $K_X$ (and hence of $mK_X$) are always divisible by $p$ if $p=2$. 
    To start with note that we can write $K_X$ as $\pi^*(K_Y)+R$, where $R=\sum_{i=1}^r(N_i+1)(p-1)[P]$ is the ramification divisor. 
    First observe that as $p\nmid N_i$ and $p=2$ then $p\mid N_i+1$. 
    Also, $\pi^*$ multiplies the co-efficients of ramification points by $p$, hence the coefficients of all ramification points of $\pi^*(K_Y)$ are divisible by $p$. 
    So in the sum $\pi^*(K_Y)+R$ all co-efficients of ramification points are divisible by $p$, and we are done for the case when $l=p$.


    Now if $l\neq p$ then we know that the ramification coefficients $n_i=l-1$ for all $i$. 
    To show the if direction in this case, first note that from \eqref{eq:riem} we see that $\dim_k(H^0(X,\Omega_X^{\otimes m}))=3$. 
    On the other hand, the Hurwitz formula, $2g_X-2=-2l+\deg(R)=2g_X+r(l-1)$, implies that $r=6$. 
    Finally Proposition \ref{dim1} gives us
      \begin{eqnarray*}
	\dim_kH^0(X,\Omega_X^{\otimes m})^G & = & -(2m-1) + \sum_{i=1}^r \left\lfloor \frac{m\cdot n_i}{l} \right\rfloor\\
	& = & -3 +\sum_{i=1}^6 \left\lfloor \frac{m(l-1)}{l} \right\rfloor\\
	& = & 3.
      \end{eqnarray*}
    since $m=l=2$.
    As the dimensions are equal the action is trivial.


    Now, for the final section of the proof, suppose that $G$ acts trivially on $H^0(X,\Omega_X^{\otimes m})$. 
    We will show that, given that $g+X\geq 2$, the only possibility is that $g_X=2$, which occurs when $l=2$.
    Then 
      \begin{eqnarray*}
	\lefteqn{(2m-1)(g_X-1)=\dim_k(H^0(X,\Omega_X^{\otimes m}))} \\
	& = & \dim_k(H^0(X,\Omega_X^{\otimes m})^G)=-(2m-1)+\sum_{i=1}^r \left\lfloor \frac{m\cdot n_i}{l} \right\rfloor.
      \end{eqnarray*}
    This in turn gives us
      \begin{eqnarray*}
	(2m-1)g_X & = & \sum_{i=1}^r \left\lfloor \frac{m\cdot n_i}{l} \right\rfloor\\
	& = & \sum_{i=1}^r \left\lfloor \frac{m(l-1)}{l} \right\rfloor\\
	& = & r\left( m+\left\lfloor \frac{-m}{l} \right\rfloor \right).
      \end{eqnarray*}
    Note that we can rewrite this by choosing $s\in \{1,\ldots ,l\}$ and $q\in \mathbb{N}$ such that $m=ql+s$. 
    Then we have
      \begin{equation}\label{eq:mult}
	(2m-1)g_X=r(m-q-1).
      \end{equation}
    If we multiply \eqref{eq:mult} by $l-1$ and then substitute in for the $r(l-1)$ term in the Hurwitz formula we get
      \begin{equation*}
	(2m-1)(l-1)g_X=(2g_X+2(l-1))(m-q-1).
      \end{equation*}
    By rearranging we are able to compute $g_X$ in terms of $m,l$ and $q$;
      \begin{equation*}
	g_X=\frac{2(l-1)(m-q-1)}{(2m-1)(l-1)-2(m-q-1)}.
      \end{equation*}
    We then simplify this expression
      \begin{eqnarray}\label{eq:equ}
	g_X & = & \frac{2(l-1)(m-q-1)}{(2m-1)(l-1)-2(m-q-1)} \nonumber \\
	& = & 1 + \frac{2(m-q-1)-(2q+1)(l-1)}{(2m-1)(l-1)-2(m-q-1)} \nonumber \\
	& = & 1 + \frac{2s-1-l}{(2m-1)(l-1)-2(m-q-1)} \nonumber \\
	& = & 1 + \frac{2(s-1)+1-l}{(2m-1-2q)(l-1)-2(s-1)}. 
      \end{eqnarray}
    We now use \eqref{eq:equ} to complete the proof.
    First, we show that if $l\geq 3$ the equation cannot hold whilst $g_X\geq 2$.
    Observe that the denominator is strictly greater than $l-1$, remembering that $m=ql+s$;
      \begin{eqnarray*}
	(2m-1-2q)(l-1)-2(s-1) & = & ((2q(l-1)+2s-1)(l-1)-2(s-1) \\
	& \geq & (2s-1)(l-1)-2(s-1) \\
	& > & (2s-1)(l-1)-2(s-1)(l-1) \\
	& = & l-1.
      \end{eqnarray*}
    Now the numerator is at most $l-1$, occurring when $s=l$. 
    Hence if $l\geq 3$ the fraction in \eqref{eq:equ} will be less than one and $g_X < 2$, which we have already looked at.
    Now if $l=2$, then $s$ is either 1 or 2.
    If $s=1$ the fraction in \eqref{eq:equ} is negative, and $g_X<1$, which again we have already considered.
    Finally, if $s=2$ then $g_X\leq 2$, with equality if and only if $q=0$ i.e. if $m=2$.
    So if $g_X \geq 2$ then the action being trivial implies that $g_X=l=m=2$, and the proof is complete.    
  \end{proof}


\begin{thebibliography}
  \bibitem[K\"o]{Ko} \textsc{B.~K\"ock}, Galois
    structure  of
    Zariski cohomology for weakly ramified covers of curves,
    \textit{American Journal of Mathematics}~\textbf{126} (2004),
    1085-1107.
\bibitem[Na]{Na} \textsc{S.\ Nakajima}, Action of an
    automorphism of order $p$ on cohomology groups of an
    algebraic curve, \textit{J.\ Pure Appl.\ Algebra}~\textbf{42}
    (1986), 85-94.
\bibitem[Se]{Se} \textsc{J.-P.~Serre}, Corps locaux, \textit{
    Publications de l'Institut de Math\'ematique de l'Universit\'e de
  Nancago VIII}, Hermann, Paris 1962.
\bibitem[Fu]{fulton} \textsc{W.\ Fulton}, Algebraic Curves, 
    \textit{Advanced Book Classics. Addison-Wesley Publishing 
    Company Advanced Book Program}, Redwood CIty, CA, 1989.
\bibitem[Ha]{Ha} \textsc{R.~Hartshorne}, Algebraic
    Geometry, \textit{Grad.\ Texts in Math.}, vol.~52, Springer, New York 1977.
\end{thebibliography}

\end{document}