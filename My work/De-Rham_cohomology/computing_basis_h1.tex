% !TEX TS-program = pdflatex
% !TEX encoding = UTF-8 Unicode

% This is a simple template for a LaTeX document using the "article" class.
% See "book", "report", "letter" for other types of document.

\documentclass[draft, 11pt]{article} % use larger type; default would be 10pt

\usepackage[utf8]{inputenc} % set input encoding (not needed with XeLaTeX)

%%% Examples of Article customizations
% These packages are optional, depending whether you want the features they provide.
% See the LaTeX Companion or other references for full information.

%%% PAGE DIMENSIONS
\usepackage{geometry} % to change the page dimensions
\geometry{a4paper} % or letterpaper (US) or a5paper or....
% \geometry{margins=2in} % for example, change the margins to 2 inches all round
% \geometry{landscape} % set up the page for landscape
%   read geometry.pdf for detailed page layout information

\usepackage{graphicx} % support the \includegraphics command and options

\usepackage[parfill]{parskip} % Activate to begin paragraphs with an empty line rather than an indent

%%% PACKAGES
\usepackage{mathtools}
\usepackage{booktabs} % for much better looking tables
\usepackage{array} % for better arrays (eg matrices) in maths
\usepackage{paralist} % very flexible & customisable lists (eg. enumerate/itemize, etc.)
\usepackage{verbatim} % adds environment for commenting out blocks of text & for better verbatim
\usepackage{subfig} % make it possible to include more than one captioned figure/table in a single float
% These packages are all incorporated in the memoir class to one degree or another...

%%% HEADERS & FOOTERS
\usepackage{fancyhdr} % This should be set AFTER setting up the page geometry
\pagestyle{fancy} % options: empty , plain , fancy
\renewcommand{\headrulewidth}{0pt} % customise the layout...
\lhead{}\chead{}\rhead{}
\lfoot{}\cfoot{\thepage}\rfoot{}

%%% SECTION TITLE APPEARANCE
\usepackage{sectsty}
\allsectionsfont{\sffamily\mdseries\upshape} % (See the fntguide.pdf for font help)
\usepackage{amsmath}
\usepackage{amsthm}
\usepackage{amsfonts}
\usepackage{mathrsfs}
\usepackage{amsopn}
\usepackage{amssymb}
\usepackage{natbib}
% (This matches ConTeXt defaults)

%%% ToC (table of contents) APPEARANCE
\usepackage[nottoc,notlof,notlot]{tocbibind} % Put the bibliography in the ToC
\usepackage[titles,subfigure]{tocloft} % Alter the style of the Table of Contents
\renewcommand{\cftsecfont}{\rmfamily\mdseries\upshape}
\renewcommand{\cftsecpagefont}{\rmfamily\mdseries\upshape} % No bold!

%Theorems and stuff
\theoremstyle{plain}
\newtheorem{defn}{Definition}[section]
\newtheorem{thm}[defn]{Theorem}
\newtheorem{cor}[defn]{Corollary}
\newtheorem{lem}[defn]{Lemma}
\newtheorem{prop}[defn]{Proposition}
\newtheorem{ex}[defn]{Example}
\newtheorem*{unnumthm}{Theorem}
\newtheorem{defnlem}[defn]{Definition/Lemma}
\newtheorem{defnthm}[defn]{Theorem/Definition}
\theoremstyle{remark}
\newtheorem*{rem}{Remark}


\newcommand{\cO}{{\cal O}}
\newcommand{\ra}{\rightarrow}
\newcommand{\NN}{{\mathbb N}}
\newcommand{\PP}{{\mathbb P}}
\newcommand{\ZZ}{{\mathbb Z}}
\newcommand{\cL}{{\mathcal L}}
\newcommand{\cA}{{\mathcal A}}
\newcommand{\cD}{{\mathcal D}}


\DeclareMathOperator{\aut}{Aut}
\DeclareMathOperator{\res}{Res}
\DeclareMathOperator{\ord}{ord}
\DeclareMathOperator{\di}{div}
\DeclareMathOperator{\cha}{char}
\DeclareMathOperator{\gal}{Gal}
\DeclareMathOperator{\Tr}{Tr}

%%% END Article customizations

%%% The "real" document content comes below...

\title{}
\author{}
%\date{} % Activate to display a given date or no date (if empty),
         % otherwise the current date is printed 

\begin{document}
\maketitle


Let $X$ be a smooth, projective, connected hyperelliptic curve of genus $g$ over an algebraically closed field $k$ of characteristic unequal to 2.
Let $x:X\rightarrow \mathbb P_k^1$ be the corresponding holomorphic map of degree two.
There is then a corresponding field extension of $k(x)$ by $y$, where $y$ is defined by
\[
y^2 = f(x),
\]
for some $f(x)\in k[x]$.
By an automorphism of $X$ we may assume that $0$ and $\infty$ are branch points.
We let $P_0$ and $P_\infty$ be the unique points in $x^{-1}(0)$ and $x^{-1}(\infty)$ respectively.
We may also assume that $f(x) = a_{2g+1}x^{2g+1} + a_{2g}x^{2g} + \ldots + a_0$ for some $a_i \in k$.
If $a_{2g}=0$ we can use a third automorphism to adjust the roots, and hence we may assume that $a_{2g} \neq 0$.


In the paper of Hortsch, we take the pairing $\langle y^{-1}x^idx, yx^{-j} \rangle = \res(x^{i-j}dx)$.
We consider the same pairing to start with.
In order to compute the residue we require there to be a pole at $P_0$ and $P_\infty$. 
To demonstrate this we compute the divisor
\begin{eqnarray}
	\di(x^{i-j}dx) & = &  2(i-j)P_0 +  2(j-i)P_\infty + R - 2D_\infty \\
	& = & (2i-2j+1)P_0 +(R-P_0-P_\infty) +(2j-2i-3)P_\infty.
\end{eqnarray}

Recall that since both $P_\infty$ and $P_0$ are ramification points it follows that $R-P_0-P_\infty$ is positive.
Hence we have a pole at both $P_0$ and $P_\infty$ if and only if $i=j+1$.
So we will be considering the differential $\frac{1}{x}dx$.


Now we need to compute the residue of $x^{-1}dx$ at $P_\infty$.
To start with we note that $t:= \frac{y}{x^{g+1}}$ is a uniformising parameter at $P_\infty$.
We can compute the order of $t$ at $P_\infty$; the computation follows simply from the fact that $t^2 = \frac{f(x)}{x^{2g+2}}$, which gives
\[
ord_{P_\infty}(t) = \frac{1}{2}\ord_{P_\infty}(f(x)) - \frac{1}{2}d_{P_\infty}(x^{2g+2}) = -(2g+1) + (2g+2) = 1.
\]

We now write $\frac{1}{x}dx$ in terms of $dt$.
First, 
\[
dt^2 = f\cdot d\frac{1}{x^{2g+2}} + \frac{1}{x^{2g+2}} \cdot df.
\]
Now 
\[
f\cdot d\frac{1}{x^{2g+2}} = f\cdot \frac{2g+1}{x^{2g+3}} dx \ {\rm and}\ \frac{1}{x^{2g+2}} \cdot df = \frac{1}{x^{2g+2}}\cdot f' \cdot dx.
\]
So in total 
\[
dt^2 = \left (\frac{(2g+1)\cdot f}{x^{2g+3}} - \frac{f'}{x^{2g+2}} \right ) dx.
\]

Finally, we have $\frac{1}{x}dx = \frac {2\cdot x^{2g+1}}{\left ( (2g+1)\cdot \frac{f}{x} - f' \right )} \cdot t dt$.

Let $h(x) = (2g+1)\cdot \frac{f}{x} - f'$.
Note that $\deg(h(x)) = 2g-1$ since we assumed that $a_{2g}\neq 0$.
Hence $\ord_{P_\infty}\left ( \frac{2\cdot x^{2g+1}}{h(x)} \right ) = -2(2g+1) + 2(2g-1) = -4$.
We want to compute the coefficent of $t^{-2}$ in the laurent series of this.
This is because the residue of $\frac{1}{x}dx$ is the coefficient of $t^{-1}$ in $\frac{2x^{2g+1}}{h(x)}\cdot t$, which is the coefficient of $t^{-2}$ in $\frac{2x^{2g+1}}{h(x)}$.


Note that 
\begin{eqnarray}\label{h(x)}
h(x) & = & \left( \frac{a_{2g}(2g+1)x^{2g-1}}{x^{2g+1}} - \frac{2g\cdot a_{2g}\cdot x^{2g-1}}{x^{2g+1}} \right) \nonumber \\
& ~ & + \left(\frac{a_{2g-1}\cdot(2g+1)\cdot x^{2g-2}}{x^{2g+1}} - \frac{(2g-1)\cdot a_{2g-1} \cdot x^{2g-2}}{x^{2g+1}} \right) + \ldots \nonumber \\
& = & \frac{a_{2g}}{x^2} + \frac{2\cdot a_{2g-1}}{x^3} + \ldots
\end{eqnarray}

Also note that 
\begin{eqnarray}\label{t^4}
t^4 & = & \frac{f(x)^2}{x^{4g+4}} \nonumber \\
& = & \frac{a_{2g+1}^2 x^{4g+2}}{x^{4g+4}} + \frac{2\cdot a_{2g}\cdot a_{2g+1} \cdot x^{4g+1}}{x^{4g+4}} + \ldots \nonumber \\
& = & \frac{a_{2g+1}^2}{x^2} + \frac{2\cdot a_{2g} \cdot a_{2g+1}}{x^3} + \ldots
\end{eqnarray}

So if we write $h(x) = \sum_{i\geq 4} c_i t^i$ then we must have $c_4 = \frac{a_{2g}}{a_{2g+1}^2}$ in order that the coefficients of $\frac{1}{x^2}$ are equal.

We now compute the $c_6$ expansions of $h(x)$.

We know that the lead term of $t^6 = \frac{f(x)^3}{x^{6g+6}}$ is $\frac{2a_{2g}^2}{a_{2g+1}x^3}$.
So $c_6$ must satisfy
\[
\frac{2a_{2g-1}}{x^3} = \frac{c_4\cdot 2 \cdot a_{2g} \cdot a_{2g+1}}{x^3} + \frac{c_6 \cdot a_{2g+1}^3}{x^3},
\]
from \eqref{h(x)} and \eqref{t^4}.

Hence we can easily derive $c_6 = \frac{2\cdot a_{2g-1} \cdot a_{2g+1} - 2a_{2g}^2}{a_{2g+1}^4}$.


We then suppose that $\frac{1}{h}  = \sum_{i\geq -4} d_it^i$, and then we have $\sum c_it^i \cdot \sum d_it^i = 1$.
As described earlier we want to compute $d_{-2}$.
Expanding the product of the sums we see
\[
1 = c_4\cdot d_{-4} + (c_6\cdot d_{-4} + c_4\cdot d_{-2})t^2 + \ldots
\]
So $c_4 \cdot d_{-4} = 1$ and hence $d_{-4} = \frac{1}{c_4}$.



Then we also have $c_6d_{-4} + c_4d_{-2} = 0$ and hence $d_{-2} = -\frac{c_6d_{-4}}{c_4} = -\frac{c_6}{c_4}$.
So in total  
\[
d_{-2} = \frac{a_{2g+1}^4}{a_{2g}^2} \cdot \left( \frac{2a_{2g-1}a_{2g+1} - 2a_{2g}^2}{a_{2g+1}^4} \right) = \frac{2a_{2g-1}a_{2g+1} - 2a_{2g}^2}{a_{2g}^2}.
\]

Now we can assume that $a_{2g+1} = 1$, since this will not affect the curve.
This reduces the task of showing that the residue is non-zero to showing that $a_{2g-1} -a_{2g}^2 \neq 0$.
But if we write the roots of $f(x)$ as $\alpha_i \in k$ with $1\leq i \leq 2g+1$ then we can see that $a_{2g-1} -a_{2g}^2 = \sum_{i=1}^{2g+1}\alpha_i^2$.
 
{\bf If this sum is zero, then by an automorphism of the curve we can change one of the roots by a small value, and hence make it non-zero. 
Indeed, we can fix all but one root, say $\alpha_1$. Then there is only one value of $\alpha_1$ such that $a_{2g}=0$, and there are only two values of $\alpha_1$ such that $\sum_{i=1}^{2g+1}\alpha_i^2 = 0$. Since $k$ is infinite we can certainly ensure that our automorphism of $X$ sends $\alpha_1$ to an element of $k$ distinct from these three values.}


We now want to find, given $yx^{-i}$ for some $1\leq i \leq g$, two differentials $\omega_1$ and $\omega_2$, satisfying the following:
\begin{itemize}
	\item $df = \omega_1 - \omega_2$,
	\item $\omega_1$ doesn't have a pole at any point other than $P_0$,
	\item $\omega_2$ has a pole at no point other than $P_\infty$.
\end{itemize}

One step towards this is to compute the order of $df = dyx^{-i}$ at $P_0$ and $P_\infty$.
Note that $\ord_{P_0}(yx^{-i}) = \ord_{P_0}(y) + \ord_{P_0}(x^{-i}) = 1 -2i$.
Hence $yx^{-i} = ct^{1-2i} + {\rm \ higher \ order \ terms }$, for some uniformising parameter $t$ at $P_0$ and some constant $c$.
Hence $\ord_{P_0}(yx^{-i}) = -2i$.

We now compute the order of the differential at $P_\infty$.
First note that $\ord_{P_\infty}(yx^{-i})= \ord_{P_\infty}(y) + \ord_{P_\infty}(x^{-i}) = -2g-1+2i.$
Hence if we let $t$ be a uniformising parameter at $P_\infty$ then $yx^{-i} = ct^{-2g-1+2i} + {\rm \ higher \ order \ terms \ }$, and since $dt^{-2g-1+2i} = (2i-(2g+1)) t^{-2g-2+2i}dt$ it follows that $\ord_{P_\infty}(dyx^{-i}) = 2i-2g-2$.


Obviously $dyx^{-i} = x^{-i}dy + ydx^{-i}$.
Now $y$ is a uniformising parameter at $P_0$, so $\ord_{P_0}(x^{-i}dy) = -2i$.
Also, since $ydx^{-i} = \frac{y}{x^{i-1}}tdt$ for some uniformising parameter $t$ at $P_\infty$, we have $\ord_{P_\infty}(ydx^{-i}) = 2i-2g-2$.
So if we let $\omega_1 = x^{-i}dy$ and $\omega_2 = ydx^{-i}$ then we have two differentials with the desired orders at $P_0$ and $P_\infty$ respectively, and whose sum is equal to $f= yx^{-i}$.
However, they are not regular on the rest of the space.
The order at $P_\infty$ of $dy$ is $-2g$, so the order of $\omega_1$ is $2i-2g$ at $P_\infty$.
We can see that the order of $dy$ at $P_\infty$ is $-2g$, since
\[
2ydy = dy^2 = df = f'dx = f'x^2d\frac{1}{x} = 2f'x^2tdt
\]
for some uniformising parameter.
Then $\ord_{P_\infty}(dy) = \ord_{P_\infty}\left (\frac{2f'x^2t}{y}dt \right) = -4g - 2 +1 +2g+1 = -2g$.
So the order at $P_\infty$ of $x^{-i}dy$ is $2i-2g$.
The order at $P_0$ of $ydx^{-i}$ is $-2i$.


So we need a function $h$ such that either 
\begin{itemize}
 	\item $h$ has a zero of order $2g$ at $P_0$ and $h-1$ has a zero of order $2g$ at $P_\infty$ {\bf OR}
	\item $h$ has a zero of order $2g$ at $P_\infty$ and $h-1$ has a zero of order $2g$ at $P_0$.
\end{itemize}

There is also the issue of potential other poles and zeroes of each differential.
We first consider $x^{-i}dy$.
Now $ dy = \frac{f'}{y}dx$, so \begin{eqnarray*} \di (dy) & = & \di(f') - \di (y) + \di (dx) \\ & = & \di(f') - (R - (2g+2)[P_\infty]) + (R - 4[P_\infty])\\  & = & \di(f') +(2g-2)[P_\infty].\end{eqnarray*}
Note that we use the Hurwitz formula to compute the divisor of $dx$.
So \begin{eqnarray} \di(x^{-i}dy) & = &  \di(f') -2i[P_0] + (2i +2g-2)P_{\infty} \\ & = & \di_0(f') - 2i[P_0] + (2i-2)[P_\infty]. \end{eqnarray}

We now consider $ydx^{-i}$.
Firstly, \[ y dx^{-i} = y\cdot \frac{i}{x^{i-1}}d\frac{1}{x} = y \cdot \frac{i}{x^{i+1}} dx.\]
Hence \begin{eqnarray*} \di (ydx^{-i}) & = & (R - (2g+2)[P_\infty]) -(2(i+1)[P_0] - 2(i+1)[P_\infty]) + (R - 4[P_\infty]) \\ & = &  2R - 2(i+1)[P_0] + 2(i-g-2)[P_\infty] \\ & = & 2\left ( \sum_{\text{Ramified } P\neq P_0, P_\infty}[P]\right) - 2i[P_0] +2(i-g-1)P_\infty .\end{eqnarray*}


We also compute the divisor of $dyx^{-i}$.
To start with, note that 
\begin{eqnarray}
2yx^{-i} & = & d(yx^{-i})^2 \\
& = & dfx^{-2i} \\
& = & fdx^{-2i} + x{-2i}df \\
& = & -f\cdot\frac{2i}{x^{2i+1}}dx + \frac{f'}{x^{2i}}dx \\
& = & \frac{1}{x^{2i}}\left( f' - 2i\frac{f}{x}\right) dx 
\end{eqnarray}

Hence it follows that 
\begin{equation*}
dyx^{-i} = \frac{1}{2yx^{i}}\left( f' - \frac{2if}{x} \right) dx
\end{equation*}
and so
\[
\di (dxy^{-i}) = \di_0(h) -2i[P_0] +2(i-g-1)[P_\infty],
\]
where $h = \left ( f'-\frac{2if}{x} \right )$.
We use $h$ as we will not in general know what its zeroes are, though we do know that $\di_0(h)$ is a positive divisor of degree $2g$ (note that it cannot have a zero at a ramification point, since $f$ and $f'$ have distinct roots).



\bibliography{/home/jtait/files/Documents/Maths/Bibliography/biblio.bib}
%\bibliography{/home/joe/files/Documents/Maths/Bibliography/biblio.bib}
\bibliographystyle{plain}


\end{document}
