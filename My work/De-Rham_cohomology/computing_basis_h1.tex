% !TEX TS-program = pdflatex
% !TEX encoding = UTF-8 Unicode

% This is a simple template for a LaTeX document using the "article" class.
% See "book", "report", "letter" for other types of document.

\documentclass[draft, 11pt]{article} % use larger type; default would be 10pt

\usepackage[utf8]{inputenc} % set input encoding (not needed with XeLaTeX)

%%% Examples of Article customizations
% These packages are optional, depending whether you want the features they provide.
% See the LaTeX Companion or other references for full information.

%%% PAGE DIMENSIONS
\usepackage{geometry} % to change the page dimensions
\geometry{a4paper} % or letterpaper (US) or a5paper or....
% \geometry{margins=2in} % for example, change the margins to 2 inches all round
% \geometry{landscape} % set up the page for landscape
%   read geometry.pdf for detailed page layout information

\usepackage{graphicx} % support the \includegraphics command and options
\usepackage{todonotes}

\usepackage[parfill]{parskip} % Activate to begin paragraphs with an empty line rather than an indent

%%% PACKAGES
\usepackage{mathtools}
\usepackage{booktabs} % for much better looking tables
\usepackage{array} % for better arrays (eg matrices) in maths
\usepackage{paralist} % very flexible & customisable lists (eg. enumerate/itemize, etc.)
\usepackage{verbatim} % adds environment for commenting out blocks of text & for better verbatim
\usepackage{subfig} % make it possible to include more than one captioned figure/table in a single float
% These packages are all incorporated in the memoir class to one degree or another...

%%% HEADERS & FOOTERS
\usepackage{fancyhdr} % This should be set AFTER setting up the page geometry
\pagestyle{fancy} % options: empty , plain , fancy
\renewcommand{\headrulewidth}{0pt} % customise the layout...
\lhead{}\chead{}\rhead{}
\lfoot{}\cfoot{\thepage}\rfoot{}

%%% SECTION TITLE APPEARANCE
\usepackage{sectsty}
\allsectionsfont{\sffamily\mdseries\upshape} % (See the fntguide.pdf for font help)
\usepackage{amsmath}
\usepackage{amsthm}
\usepackage{amsfonts}
\usepackage{mathrsfs}
\usepackage{amsopn}
\usepackage{amssymb}
\usepackage{natbib}
% (This matches ConTeXt defaults)

%%% ToC (table of contents) APPEARANCE
\usepackage[nottoc,notlof,notlot]{tocbibind} % Put the bibliography in the ToC
\usepackage[titles,subfigure]{tocloft} % Alter the style of the Table of Contents
\renewcommand{\cftsecfont}{\rmfamily\mdseries\upshape}
\renewcommand{\cftsecpagefont}{\rmfamily\mdseries\upshape} % No bold!

%Theorems and stuff
\theoremstyle{plain}
\newtheorem{defn}{Definition}[section]
\newtheorem{thm}[defn]{Theorem}
\newtheorem{cor}[defn]{Corollary}
\newtheorem{lem}[defn]{Lemma}
\newtheorem{prop}[defn]{Proposition}
\newtheorem{ex}[defn]{Example}
\newtheorem*{unnumthm}{Theorem}
\newtheorem{defnlem}[defn]{Definition/Lemma}
\newtheorem{defnthm}[defn]{Theorem/Definition}
\theoremstyle{remark}
\newtheorem*{rem}{Remark}


\newcommand{\cO}{{\cal O}}
\newcommand{\ra}{\rightarrow}
\newcommand{\NN}{{\mathbb N}}
\newcommand{\PP}{{\mathbb P}}
\newcommand{\ZZ}{{\mathbb Z}}
\newcommand{\cL}{{\mathcal L}}
\newcommand{\cA}{{\mathcal A}}
\newcommand{\cD}{{\mathcal D}}


\DeclareMathOperator{\aut}{Aut}
\DeclareMathOperator{\res}{Res}
\DeclareMathOperator{\ord}{ord}
\DeclareMathOperator{\di}{div}
\DeclareMathOperator{\cha}{char}
\DeclareMathOperator{\gal}{Gal}
\DeclareMathOperator{\Tr}{Tr}

%%% END Article customizations

%%% The "real" document content comes below...

\title{}
\author{}
%\date{} % Activate to display a given date or no date (if empty),
         % otherwise the current date is printed 

\begin{document}
\maketitle


Let $X$ be a smooth, projective, connected hyperelliptic curve of genus $g$ over an algebraically closed field $k$ of characteristic unequal to 2.
Let $x:X\rightarrow \mathbb P_k^1$ be the corresponding holomorphic map of degree two.
There is then an associated field extension of $k(x)$ by $y$, where $y$ is defined by
\[
y^2 = f(x),
\]
for some $f(x)\in k[x]$.
By an automorphism of $\mathbb P_k^1$ we may assume that $0$ and $\infty$ are branch points.
We let $P_0$ and $P_\infty$ be the unique points in $x^{-1}(0)$ and $x^{-1}(\infty)$ respectively.
We may also assume that $f(x) = a_{2g+1}x^{2g+1} + a_{2g}x^{2g} + \ldots $ for some $a_i \in k$, $a_{2g+1} \neq 0$; note that the constant term is zero since $0\in \mathbb P_k^1$ is a branch point.
Since the constant term is zero, we can conclude that $a_1 \neq 0$, else $x^2$ would divide $f(x)$, but we do not allow repeated roots.

We wish to compute a basis for the de-Rham hypercohomology of this curve, which we denote by $H^1_{dR}(X/k)$. \todo{check notation - currently copied from Hortsch paper}
To do this we se the fact that we have the following short exact sequence:
\begin{equation}\label{ses}
0 \ra H^0(X,\Omega_X) \ra H^1_{dR}(X/k) \ra H^1(X,\cO_x) \ra 0.
\end{equation}

We use \v{C}ech cohomology for our computations.
We have an affine open cover by $U_1 = X\backslash P_0$ and $U_2 = X \backslash P_\infty$, which we denote by $U = \{ U_1, \ U_2\}$.
\todo{find citation}{Now by definition}  $\check{H}_{dR}^1(U)$ is the quotient of the $k$-vector space 
\[
\{(\omega_1, \omega_2, f_{12}) | \omega_i\in \Omega_{X/k}(U_i), f_{12}\in \cO_X(U_1 \cap U_2), df_{12} = \omega_1 - \omega_2\}
\]

by the subspace
\[
\{ (df_1, df_2, f_1-f_2)|f_i \in \cO_x(U_i)\}.
\]

Now given the short exact sequence \eqref{ses} we can compute when a group $G$ acting on $X$ acts faithfully on $H^1_{dR}(X/k)$ by computing a basis of the other terms in the sequence and seeing when $G$ acts faithfully on them.
We know that $\{x^iy^{-1}dx\}_{0\leq i \leq g-1}$ form a basis of $H^0(X,\Omega_X)$ from previous work, so that leaves us to compute a basis of $H^1(X,\cO_X)$.

We start doing this by using Serre duality.
We recall that Serre duality gives a map $\text{Res}: H^0(X,\Omega_X) \times H^1(X,\cO_X) \rightarrow k$.
This is defined by  $(\omega,f) \rightarrow \sum_{P\in X}\text{res}_P(f\omega)$ for some $(\omega, f) \in   H^0(X,\Omega_X) \times H^1(X,\cO_X)$.
We consider the pair $(y^{-1}x^idx,yx^{-j})$ 
% \in  H^0(X,\Omega_X) \times H^1(X,\cO_X) 
, which maps to $\sum_{P\in X} \text{res}_P (x^{i-j}dx)$. \todo{As we're using Cech cohomology, we compute each open set. The only poles are at P0 and Pinfinity, so it is regular on the insterstion}
This can be calculated by summing the residues over the points in $U_1$ or by the residue theorem we can equivalently can compute the negative of the residue at $P_0$.
So for $\text{res}(x^{i-j})dx$ to be non-zero there must be a pole at at least one point in $U_1$ and at $P_0$.
We can check when this occurs by computing the divisor of $x^{i-j}dx$, which we do presently:
\begin{eqnarray}
	\di(x^{i-j}dx) & = &  2(i-j)P_0 +  2(j-i)P_\infty + R - 2D_\infty \\
	& = & (2i-2j+1)P_0 +(R-P_0-P_\infty) +(2j-2i-3)P_\infty,
\end{eqnarray}
where $R$ is the ramification divisor of $x$, and $D_\infty = 2[P_\infty]$.

Recall that since both $P_\infty$ and $P_0$ are ramification points it follows that $R-P_0-P_\infty$ is positive.
Moreover, since $X$ is a hyperelliptic curve the coefficient of every point in the ramification divisor is 1, we have a pole at both $P_0$ and $P_\infty$ if and only if $j=i+1$.
So we will be considering the differential $\frac{1}{x}dx$, as this is the only differential of the form $x^{i-j}dx$ which can possibly have non-zero residue at the required points.


Now we need to compute the residue of $x^{-1}dx$ at $P_\infty$.
To start with we note that $t:= \frac{y}{x^{g+1}}$ is a uniformising parameter at $P_\infty$.
Indeed we can compute the order of $t$ at $P_\infty$; the computation follows simply from the fact that $t^2 = \frac{f(x)}{x^{2g+2}}$, which gives
\[
\ord_{P_\infty}(t) = \frac{1}{2}\ord_{P_\infty}(f(x)) - \frac{1}{2}\ord_{P_\infty}(x^{2g+2}) = -(2g+1) + (2g+2) = 1.
\]

We now write $\frac{1}{x}dx$ in terms of $dt$.
First, 
\[
dt^2 = f\cdot d\frac{1}{x^{2g+2}} + \frac{1}{x^{2g+2}} \cdot df.
\]
Now 
\[
f\cdot d\frac{1}{x^{2g+2}} = -f\cdot \frac{2g+2}{x^{2g+3}} dx \ {\rm and}\ \frac{1}{x^{2g+2}} \cdot df = \frac{1}{x^{2g+2}}\cdot f' \cdot dx.
\]
\begin{comment}
So in total 
\[
dt^2 = \left (\frac{(2g+1)\cdot f}{x^{2g+3}} - \frac{f'}{x^{2g+2}} \right ) dx.
\]

Finally, we have $\frac{1}{x}dx = \frac {2\cdot x^{2g+1}}{\left ( (2g+1)\cdot \frac{f}{x} - f' \right )} \cdot t dt$.

Let $h(x) = (2g+1)\cdot \frac{f}{x} - f'$.
Note that $\deg(h(x)) = 2g-1$ since we assumed that $a_{2g}\neq 0$.
Hence $\ord_{P_\infty}\left ( \frac{2\cdot x^{2g+1}}{h(x)} \right ) = -2(2g+1) + 2(2g-1) = -4$.
We want to compute the coefficent of $t^{-2}$ in the laurent series of this.
This is because the residue of $\frac{1}{x}dx$ is the coefficient of $t^{-1}$ in $\frac{2x^{2g+1}}{h(x)}\cdot t$, which is the coefficient of $t^{-2}$ in $\frac{2x^{2g+1}}{h(x)}$.


Note that 
\begin{eqnarray}\label{h(x)}
h(x) & = & \left( \frac{a_{2g}(2g+1)x^{2g-1}}{x^{2g+1}} - \frac{2g\cdot a_{2g}\cdot x^{2g-1}}{x^{2g+1}} \right) \nonumber \\
& ~ & + \left(\frac{a_{2g-1}\cdot(2g+1)\cdot x^{2g-2}}{x^{2g+1}} - \frac{(2g-1)\cdot a_{2g-1} \cdot x^{2g-2}}{x^{2g+1}} \right) + \ldots \nonumber \\
& = & \frac{a_{2g}}{x^2} + \frac{2\cdot a_{2g-1}}{x^3} + \ldots
\end{eqnarray}

Also note that 
\begin{eqnarray}\label{t^4}
t^4 & = & \frac{f(x)^2}{x^{4g+4}} \nonumber \\
& = & \frac{a_{2g+1}^2 x^{4g+2}}{x^{4g+4}} + \frac{2\cdot a_{2g}\cdot a_{2g+1} \cdot x^{4g+1}}{x^{4g+4}} + \ldots \nonumber \\
& = & \frac{a_{2g+1}^2}{x^2} + \frac{2\cdot a_{2g} \cdot a_{2g+1}}{x^3} + \ldots
\end{eqnarray}

So if we write $h(x) = \sum_{i\geq 4} c_i t^i$ then we must have $c_4 = \frac{a_{2g}}{a_{2g+1}^2}$ in order that the coefficients of $\frac{1}{x^2}$ are equal.

We now compute the $c_6$ expansions of $h(x)$.

We know that the lead term of $t^6 = \frac{f(x)^3}{x^{6g+6}}$ is $\frac{2a_{2g}^2}{a_{2g+1}x^3}$.
So $c_6$ must satisfy
\[
\frac{2a_{2g-1}}{x^3} = \frac{c_4\cdot 2 \cdot a_{2g} \cdot a_{2g+1}}{x^3} + \frac{c_6 \cdot a_{2g+1}^3}{x^3},
\]
from \eqref{h(x)} and \eqref{t^4}.

Hence we can easily derive $c_6 = \frac{2\cdot a_{2g-1} \cdot a_{2g+1} - 2a_{2g}^2}{a_{2g+1}^4}$.


We then suppose that $\frac{1}{h}  = \sum_{i\geq -4} d_it^i$, and then we have $\sum c_it^i \cdot \sum d_it^i = 1$.
As described earlier we want to compute $d_{-2}$.
Expanding the product of the sums we see
\[
1 = c_4\cdot d_{-4} + (c_6\cdot d_{-4} + c_4\cdot d_{-2})t^2 + \ldots
\]
So $c_4 \cdot d_{-4} = 1$ and hence $d_{-4} = \frac{1}{c_4}$.



Then we also have $c_6d_{-4} + c_4d_{-2} = 0$ and hence $d_{-2} = -\frac{c_6d_{-4}}{c_4} = -\frac{c_6}{c_4}$.
So in total  
\[
d_{-2} = \frac{a_{2g+1}^4}{a_{2g}^2} \cdot \left( \frac{2a_{2g-1}a_{2g+1} - 2a_{2g}^2}{a_{2g+1}^4} \right) = \frac{2a_{2g-1}a_{2g+1} - 2a_{2g}^2}{a_{2g}^2}.
\]

Now we can assume that $a_{2g+1} = 1$, since this will not affect the curve.
This reduces the task of showing that the residue is non-zero to showing that $a_{2g-1} -a_{2g}^2 \neq 0$.
But if we write the roots of $f(x)$ as $\alpha_i \in k$ with $1\leq i \leq 2g+1$ then we can see that $a_{2g-1} -a_{2g}^2 = \sum_{i=1}^{2g+1}\alpha_i^2$.
 
{\bf If this sum is zero, then by an automorphism of the curve we can change one of the roots by a small value, and hence make it non-zero. 
Indeed, we can fix all but one root, say $\alpha_1$. Then there is only one value of $\alpha_1$ such that $a_{2g}=0$, and there are only two values of $\alpha_1$ such that $\sum_{i=1}^{2g+1}\alpha_i^2 = 0$. Since $k$ is infinite we can certainly ensure that our automorphism of $X$ sends $\alpha_1$ to an element of $k$ distinct from these three values.}
\end{comment}

Hence
\[
dt^2 = \frac{1}{x^{2g+2}}\left(f' - \frac{(2g+2)f}{x}\right) dx
\]
and it then follows that
\[
\frac{1}{x}dx = \frac{-2tx^{2g+1}}{\left(\frac{(2g+2)f}{x} - f'\right)} dt.
\]
We now let $h = \left(\frac{(2g+2)f}{x} - f'\right)$, and noting that the coefficient of $x^{2g}$ in $h$ is $(2g+2)a_{2g+1} - (2g+1)a_{2g+1} = a_{2g+1}$, we see that $h$ is a degree $2g$ polynomial in $x$.
We wish to compute the coefficient of $t^{-1}$ in the expansion of $\frac{1}{x}dx$ at $P_\infty$ and computing the first coefficient of $\frac{h}{x^{2g+1}}$ is a key step.

Now
\[
\frac{h}{x^{2g+1}} = \frac{a_{2g+1}x^{2g}}{x^{2g+1}} + \ldots = \frac{a_{2g+1}}{x} + \ldots 
\]
Similarly 
\[
t^2 = \frac{f}{x^{2g+2} } = \frac{a_{2g+1}}{x} + \ldots
\]

Since $\ord_{P_\infty}\left(\frac{h}{x^{2g+1}}\right) = 2$ we have that $\frac{h}{x^{2g+1}} = \sum_{i\geq 2} c_it^i$ for $c_i\in k$, and from the above computations we can see that $c_2 = 1$.
We can also write $\frac{x^{2g+1}}{h} = \sum_{i\geq -2} d_it^i$, for some $d_i\in k$.
Since \[
\frac{h}{x^{2g+1}} \cdot \frac{x^{2g+1}}{h} = \left( \sum_{i\geq 2} c_it^i\right) \cdot \left( \sum_{i\geq -2} d_it^i\right) = 1
\]
 we conclude that $d_{-2} = 1$. 
Now 
\[
\frac{1}{x}dx = \left( -2t \cdot \sum_{i\geq -2} d_it^i\right) dt = \left( \sum_{i\geq -2}-2d_it^{i+1} \right) dt
\]
so we can see that the residue is $-2$.

To make sure that this is part of the basis for $H^1(\Omega_X,\cO_X)$ we need to find suitable differentials $\omega_1$ and $\omega_2$, regular on $U_1$ and $U_2$ respectively, such that $d(yx^{-i}) = \omega_1 - \omega_2$.
\begin{comment}
We now want to find, given $yx^{-i}$ for some $1\leq i \leq g$, two differentials $\omega_1$ and $\omega_2$, satisfying the following:
\begin{itemize}
	\item $df = \omega_1 - \omega_2$,
	\item $\omega_1$ doesn't have a pole at any point other than $P_0$,
	\item $\omega_2$ has a pole at no point other than $P_\infty$.
\end{itemize}

One step towards this is to compute the order of $df = dyx^{-i}$ at $P_0$ and $P_\infty$.
Note that $\ord_{P_0}(yx^{-i}) = \ord_{P_0}(y) + \ord_{P_0}(x^{-i}) = 1 -2i$.
Hence $yx^{-i} = ct^{1-2i} + {\rm \ higher \ order \ terms }$, for some uniformising parameter $t$ at $P_0$ and some constant $c$.
Hence $\ord_{P_0}(yx^{-i}) = -2i$.

We now compute the order of the differential at $P_\infty$.
First note that $\ord_{P_\infty}(yx^{-i})= \ord_{P_\infty}(y) + \ord_{P_\infty}(x^{-i}) = -2g-1+2i.$
Hence if we let $t$ be a uniformising parameter at $P_\infty$ then $yx^{-i} = ct^{-2g-1+2i} + {\rm \ higher \ order \ terms \ }$, and since $dt^{-2g-1+2i} = (2i-(2g+1)) t^{-2g-2+2i}dt$ it follows that $\ord_{P_\infty}(dyx^{-i}) = 2i-2g-2$.


Obviously $dyx^{-i} = x^{-i}dy + ydx^{-i}$.
Now $y$ is a uniformising parameter at $P_0$, so $\ord_{P_0}(x^{-i}dy) = -2i$.
Also, since $ydx^{-i} = \frac{y}{x^{i-1}}tdt$ for some uniformising parameter $t$ at $P_\infty$, we have $\ord_{P_\infty}(ydx^{-i}) = 2i-2g-2$.
So if we let $\omega_1 = x^{-i}dy$ and $\omega_2 = ydx^{-i}$ then we have two differentials with the desired orders at $P_0$ and $P_\infty$ respectively, and whose sum is equal to $f= yx^{-i}$.
However, they are not regular on the rest of the space.
The order at $P_\infty$ of $dy$ is $-2g$, so the order of $\omega_1$ is $2i-2g$ at $P_\infty$.
We can see that the order of $dy$ at $P_\infty$ is $-2g$, since
\[
2ydy = dy^2 = df = f'dx = f'x^2d\frac{1}{x} = 2f'x^2tdt
\]
for some uniformising parameter.
Then $\ord_{P_\infty}(dy) = \ord_{P_\infty}\left (\frac{2f'x^2t}{y}dt \right) = -4g - 2 +1 +2g+1 = -2g$.
So the order at $P_\infty$ of $x^{-i}dy$ is $2i-2g$.
The order at $P_0$ of $ydx^{-i}$ is $-2i$.


So we need a function $h$ such that either 
\begin{itemize}
 	\item $h$ has a zero of order $2g$ at $P_0$ and $h-1$ has a zero of order $2g$ at $P_\infty$ {\bf OR}
	\item $h$ has a zero of order $2g$ at $P_\infty$ and $h-1$ has a zero of order $2g$ at $P_0$.
\end{itemize}


We can, of course, rewrite $d(yx^{-i})$ as $ydx^{-i} + x^{-i}dy$.
We will start by studying these two divisors, and use them to find an appropriate $\omega_1$ and $\omega_2$.

We first consider $x^{-i}dy$.
Now $ dy = \frac{f'}{y}dx$, so \begin{eqnarray*} \di (dy) & = & \di(f') - \di (y) + \di (dx) \\ & = & \di(f') - (R - (2g+2)[P_\infty]) + (R - 4[P_\infty])\\  & = & \di(f') +(2g-2)[P_\infty].\end{eqnarray*}
Note that we use the Hurwitz formula to compute the divisor of $dx$.
So \begin{eqnarray} \di(x^{-i}dy) & = &  \di(f') -2i[P_0] + (2i +2g-2)P_{\infty} \\ & = & \di_0(f') - 2i[P_0] + (2i-2)[P_\infty]. \end{eqnarray}

We now consider $ydx^{-i}$.
Firstly, \[ y dx^{-i} = y\cdot \frac{i}{x^{i-1}}d\frac{1}{x} = y \cdot \frac{i}{x^{i+1}} dx.\]
Hence \begin{eqnarray*} \di (ydx^{-i}) & = & (R - (2g+2)[P_\infty]) -(2(i+1)[P_0] - 2(i+1)[P_\infty]) + (R - 4[P_\infty]) \\ & = &  2R - 2(i+1)[P_0] + 2(i-g-2)[P_\infty] \\ & = & 2\left ( \sum_{\text{Ramified } P\neq P_0, P_\infty}[P]\right) - 2i[P_0] +2(i-g-1)P_\infty .\end{eqnarray*}
\end{comment}

We begin by computing the divisor of $dyx^{-i}$.
To start with, note that 
\begin{eqnarray*}
2yx^{-i}d(yx^{-i}) & = & d(yx^{-i})^2 \\
& = & dfx^{-2i} \\
& = & fdx^{-2i} + x{-2i}df \\
& = & -f\cdot\frac{2i}{x^{2i+1}}dx + \frac{f'}{x^{2i}}dx \\
& = & \frac{1}{x^{2i}}\left( f' - 2i\frac{f}{x}\right) dx 
\end{eqnarray*}

Hence it follows that 
\begin{equation*}
dyx^{-i} = \frac{1}{2yx^{i}}\left( f' - \frac{2if}{x} \right) dx
\end{equation*}
and so
\[
\di (dxy^{-i}) = \di_0(s) -2i[P_0] +2(i-g-1)[P_\infty],
\]
where $s(x) = \left ( f'-\frac{2if}{x} \right )$.
We use $s$ as we will not in general know what its zeroes are, though we do know that $\di_0(s)$ is a \todo{actually, degree is 4g - check if this affects other things}{positive divisor of degree $4g$}.
This is true since $2i <2g+1$, thus the coefficient of $x^{2g}$ in $s$, which is $(2g+1)a_{2g+1}- 2ia_{2g+1}$, is non-zero. 
We will implicitly be using the fact that $s$ cannot have a zero at a ramification point, since $f$ and $f'$ have distinct roots, in our future computations.
\begin{comment}
We could try writing $dyx^{-i}$ as $\frac{1}{2yx^i}dx - (1-s)\frac{1}{2yx^i}dx$.
We then  have 
\[
\di \left( \frac{1}{2yx^i}dx\right) = -\di(y) - \di(x^i) + \di(dx) = (2g+2i-2)[P_\infty] - 2i[P_0]
\]
which has the correct pole at $P_0$ and is otherwise regular.

On the other hand we have
\[
\di\left((1-s)\frac{1}{2yx^i}\right ) dx = \di_0(1-s) -4g[P_\infty] +2(g+i-1)[P_\infty]-2i[P_0] = \di_0(1-s) +2(i-g-1)[P_\infty] - 2i[P_0].
\]

\end{comment}



We write $s(x) = b_{2g}x^{2g} + b_{2g-1}x^{2g-1} + \ldots + b_1x + b_0$.
We then let $\phi = b_{2g} x^{2g} b_{2g-1}x^{2g-1} + \ldots + b_{g+1}x^{g+1} $ and $\psi = s - \phi = b_gx^g + \ldots + b_0$.

It is then clear that $\frac{\phi }{2yx^i}dx - \frac{-\psi}{2yx^i}dx = \frac{\phi + \psi}{2yx^i}dx = \frac{s}{2yx^i}dx$.

We then compute the divisors of both of these differentials to check that they are regular on the appropriate sets.
Firstly
\begin{eqnarray*}
\di\left( \frac{\phi}{2yx^i}dx \right) & = & \di(\phi) - ( R - 2(g+1)[P_\infty]) - (2i[P_0] - 2i[P_\infty]) + (R - 4[P_\infty]) \\
& = & \di_0\left( \frac{\phi}{x^{g+1}}\right) + 2(g+1)[P_0] - 4g[P_\infty] - 2i[P_0] + 2(g-1)[P_\infty] \\
& = & \di_0\left( \frac{\phi}{x^{g+1}} \right) + 2(i-g-1)[P_\infty] + 2(g-i+1)[P_0].
\end{eqnarray*}

On the other hand
\begin{eqnarray*}
\di \left( \frac{-\psi}{2yx^i}dx\right) & = & \di (\psi ) -R + (2g+2)[P_\infty] - 2i[P_0] + 2i[P_\infty] + R -4[P_\infty] \\
& = & \di_0(\psi) - 2g[P_\infty] + 2(g+i-2)[P_\infty] -2i[P_0] \\
& = & \di_0(\psi) + 2(i-1)[P_\infty] - 2i[P_0].
\end{eqnarray*}

It should be noted that we required the coefficient $b_0$ in $s$ to be non-zero for this to be true, as then $\di_0(\psi)$ has a coefficient of zero for $[P_0]$. This is indeed the case, since $b_0 = a_1 - 2ia_1 = (1-2i)a_1 \neq 0$ since $a_1$ is non-zero. 


So if we now let $\omega_1 = \frac{-\psi}{2yx^i}dx$ and $\omega_2 = \frac{\phi}{2yx^i}dx$ then we have our two differentials.

\newpage

\section{characteristic 2}

We now consider the same question when $\cha (k) = 2$.
Our field extension corresponding to $X$ is now defined differently; it is $k(x,y)$, where
\[
y^2 - yh(x) = f(x)
\]
for some polynomials $f$ and $h$.
Since we are still assuming that $0$ and infinity are branch points, it follows that $\deg(f) = 2g+1$ and that $\deg(h)  = d = \leq g$. \todo{cite liu}
We suppose that $h(x) = \prod_i^m (x-a_i)^{n_i}$, and that $P_i\in X$ are the corresponding ramification points.
We then have that the ramification divisor has coefficients $\delta_P$ where
\[
\delta_P = \left \{ \begin{array}{ll}
2n_i & \text{if } P= P_i \text{ for some } i=1, \ldots , m \\
2(g+1-d) & \text{if } P=P_\infty \\
0 & \text{else.}
\end{array}
\right.
\]


We now recall the divisors of some other elements of the function field which we will be using.
\begin{eqnarray*}
\di (dx) & = & R - 4[P_\infty] \\
\di (h(x)) & = & R - 2(g+1)[P_\infty] \\
\di (x^i) & = & 2i[P_0] - 2i[P_\infty]\\
\di (y) & = & \di_0(y) - (2g+1)[P_\infty].
\end{eqnarray*}

Though we have not properly described the zero divisor of $y$, it should be recalled that since $P_0$ is a ramification, $\di_0(y)$ will have a coefficient of one for $[P_0]$.
Details of the above are in my 18 month report.

As before, we will consider a pairing in $H^0(X,\Omega_X) \times H^1(X,\cO_X)$, and start by showing that is has poles at both $P_0$ and $P_\infty$.
As shown in the 18 month report, a basis of $H^0(X,\Omega_X)$ is $\frac{x^idx}{h(x)}$ for $0 \leq i \leq g-1$.
We consider the pairing $\left(\frac{x^idx}{h(x)}, x^{-j}y\right) \mapsto \frac{x^{i-j}ydx}{h(x)}$.
We compute the divisor
\begin{eqnarray*}
\di\left(\frac{x^{i-j}ydx}{h(x)}\right)  & = & 2(i-j)[P_0] - 2(i-j)[P_\infty] + \di_0(y) - (2g+1)[P_\infty] +R -4[P_\infty] -R +2(g+2)[P_\infty] \\
& = &  \di_0(y) + 2(i-j)[P_0] +(2(j-i)-3)[P_\infty]
\end{eqnarray*}


\bibliography{/home/jtait/files/Documents/Maths/Bibliography/biblio.bib}
%\bibliography{/home/joe/files/Documents/Maths/Bibliography/biblio.bib}
\bibliographystyle{plain}


\end{document}
