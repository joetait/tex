% !TEX TS-program = pdflatex
% !TEX encoding = UTF-8 Unicode

% This is a simple template for a LaTeX document using the "article" class.
% See "book", "report", "letter" for other types of document.

\documentclass[draft, 11pt]{article} % use larger type; default would be 10pt

\usepackage[utf8]{inputenc} % set input encoding (not needed with XeLaTeX)

%%% Examples of Article customizations
% These packages are optional, depending whether you want the features they provide.
% See the LaTeX Companion or other references for full information.

%%% PAGE DIMENSIONS
\usepackage{geometry} % to change the page dimensions
\geometry{a4paper} % or letterpaper (US) or a5paper or....
% \geometry{margins=2in} % for example, change the margins to 2 inches all round
% \geometry{landscape} % set up the page for landscape
%   read geometry.pdf for detailed page layout information

\usepackage{graphicx} % support the \includegraphics command and options

\usepackage[parfill]{parskip} % Activate to begin paragraphs with an empty line rather than an indent

%%% PACKAGES
\usepackage{mathtools}
\usepackage{booktabs} % for much better looking tables
\usepackage{array} % for better arrays (eg matrices) in maths
\usepackage{paralist} % very flexible & customisable lists (eg. enumerate/itemize, etc.)
\usepackage{verbatim} % adds environment for commenting out blocks of text & for better verbatim
\usepackage{subfig} % make it possible to include more than one captioned figure/table in a single float
% These packages are all incorporated in the memoir class to one degree or another...

%%% HEADERS & FOOTERS
\usepackage{fancyhdr} % This should be set AFTER setting up the page geometry
\pagestyle{fancy} % options: empty , plain , fancy
\renewcommand{\headrulewidth}{0pt} % customise the layout...
\lhead{}\chead{}\rhead{}
\lfoot{}\cfoot{\thepage}\rfoot{}

%%% SECTION TITLE APPEARANCE
\usepackage{sectsty}
\allsectionsfont{\sffamily\mdseries\upshape} % (See the fntguide.pdf for font help)
\usepackage{amsmath}
\usepackage{amsthm}
\usepackage{amsfonts}
\usepackage{mathrsfs}
\usepackage{amsopn}
\usepackage{amssymb}
\usepackage{natbib}
% (This matches ConTeXt defaults)

%%% ToC (table of contents) APPEARANCE
\usepackage[nottoc,notlof,notlot]{tocbibind} % Put the bibliography in the ToC
\usepackage[titles,subfigure]{tocloft} % Alter the style of the Table of Contents
\renewcommand{\cftsecfont}{\rmfamily\mdseries\upshape}
\renewcommand{\cftsecpagefont}{\rmfamily\mdseries\upshape} % No bold!

%Theorems and stuff
\theoremstyle{plain}
\newtheorem{defn}{Definition}[section]
\newtheorem{thm}[defn]{Theorem}
\newtheorem{cor}[defn]{Corollary}
\newtheorem{lem}[defn]{Lemma}
\newtheorem{prop}[defn]{Proposition}
\newtheorem{ex}[defn]{Example}
\newtheorem*{unnumthm}{Theorem}
\newtheorem{defnlem}[defn]{Definition/Lemma}
\newtheorem{defnthm}[defn]{Theorem/Definition}
\theoremstyle{remark}
\newtheorem*{rem}{Remark}


\newcommand{\cO}{{\cal O}}
\newcommand{\ra}{\rightarrow}
\newcommand{\NN}{{\mathbb N}}
\newcommand{\PP}{{\mathbb P}}
\newcommand{\ZZ}{{\mathbb Z}}
\newcommand{\cL}{{\mathcal L}}
\newcommand{\cA}{{\mathcal A}}
\newcommand{\cD}{{\mathcal D}}


\DeclareMathOperator{\aut}{Aut}
\DeclareMathOperator{\res}{Res}
\DeclareMathOperator{\ord}{ord}
\DeclareMathOperator{\di}{div}
\DeclareMathOperator{\cha}{char}
\DeclareMathOperator{\gal}{Gal}
\DeclareMathOperator{\Tr}{Tr}

%%% END Article customizations

%%% The "real" document content comes below...

\title{}
\author{}
%\date{} % Activate to display a given date or no date (if empty),
         % otherwise the current date is printed 

\begin{document}
\maketitle


We assume the the characteristic of our field is $p\geq 3$.
We make the assumption that our hyperelliptic curve is ramified at both infinity and zero.
This can be done by automorphism without loss of generality, and makes the initial compuations simpler.
We can also assume that the equation $y^2 = f(x)$ defines the affine part of the curve with \[
f(x) = a_{2g+1}x^{2g+1} + a_{2g}x^{2g} +\ldots + a_0.
\]
By a third automorphism, if required, we may assume that $a_{2g} \neq =0$.


In the paper of Hortsch, we take the pairing $\langle y^{-1}x^idx, yx^{-j} \rangle = \res(x^{i-j}dx)$.
We consider the same pairing to start with.
A condition taken directly from the same paper, we require a pole at $P_0$ (the point above 0), and a pole at any other point as well.
We do indeed also have a pole at the point $P_\infty$ (the point above $\infty$).
To demonstrate this:
\begin{eqnarray}
	\di(x^{i-j}dx) & = &  2(i-j)P_0  2(j-i)P_\infty + R - 2D_\infty \\
	& = & (2i-2j+1)P_0 +(R-P_0-P_\infty) +(2j-2i-3)P_\infty.
\end{eqnarray}

Recall that since both $P_\infty$ and $P_0$ are ramification points then $R-P_0-P_\infty$ is positive.
So we have a pole at both $P_0$ and $P_\infty$ if and only if $i=j+1$.

Now we need to compute the residue of $x^{-1}dx$ at $P_\infty$.
To start with we note that $t:= \frac{y}{x^{g+1}}$.
We can compute the order of $t$ at $P_\infty$; the computation follows simply from the fact that $t^2 = \frac{f(x)}{x^{2g+2}}$.
Then $\ord_{P_\infty}(t) = \frac{1}{2}\ord_{P_\infty}(f(x)) - \frac{1}{2}d_{P_\infty}(x^{2g+2}) = 1$.

We now write $dx$ in terms of $dt$.
First, $dt^2 = f\cdot d\frac{1}{x^{2g+2}} + \frac{1}{x^{2g+2}} \cdot df$.
Now $f\cdot d\frac{1}{x^{2g+2}} = f\cdot 2g+1 \cdot \frac{1}{x^{2g+3}} dx$ and $\frac{1}{x^{2g+2}} \cdot df = \frac{1}{2g+2}\cdot f' \cdot dx$.
So in total $dt^2 = \left (2g+1 \cdot \frac{f}{x^{2g+3}} - \frac{f'}{x^{2g+2}} \right ) dx$.

Finally, we have $\frac{1}{x}dx = \frac {2\cdot x^{2g+1}}{\left ( (2g+1)\cdot \frac{f}{x} - f' \right )} \cdot t dt$.

Let $h(x) = (2g+1)\cdot \frac{f}{x} - f'$.
Note that $\deg(h(x)) \leq 2g-1$.
Hence $\ord_{P_\infty}\left ( \frac{2\cdot x^{2g+1}}{h(x)} \right ) \leq -2(2g+1) + 2(2g-1) = -4$.
We want to compute the coefficent of $t^{-2}$ in the laurent series of this.


Note that 
\begin{eqnarray}
h(x) & = & \left( \frac{a_{2g}(2g+1)x^{2g-1}}{x^{2g+1}} - \frac{2g\cdot a_{2g}\cdot x^{2g-1}}{x^{2g+1}} \right) + \left(\frac{a_{2g-1}\cdot(2g+1)\cdot x^{2g-2}}{x^{2g+1}} - \frac{(2g-1)\cdot a_{2g-1} \cdot x^{2g-2}}{x^{2g+1}} \right) + \ldots \\
& = & \frac{a_{2g}}{x^2} + \frac{2\cdot a_{2g-1}}{x^3} + \ldots
\end{eqnarray}

Also note that 
\begin{eqnarray}
t^4 & = & \frac{f(x)^2}{x^{4g+4}} \\
& = & \frac{a_{2g+1}^2 x^{4g+2}}{x^{4g+4}} + \frac{2\cdot a_{2g}\cdot a_{2g+1} \cdot x^{4g+1}}{x^{4g+4}} + \ldots \\
& = & \frac{a_{2g+1}^2}{x^2} + \frac{2\cdot a_{2g} \cdot a_{2g+1}}{x^3} + \ldots
\end{eqnarray}

So if we write $h(x) = \sum_{i\geq 4} c_i t^i$ then we must have $c_4 = \frac{a_{2g}}{a_{2g+1}^2}$ in order that the coefficients of $\frac{1}{x^2}$ are equal.

We now compute the $c_6$ coefficient in the sum.

We know that the lead term of $t^6 = \frac{f(x)^3}{x^{6g+6}}$ is $\frac{2a_{2g}^2}{a_{2g+1}x^3}$.
So $c_6$ must satisfy
\[
\frac{2a_{2g-1}}{x^3} = \frac{c_4\cdot 2 \cdot a_{2g} \cdot a_{2g+1}}{x^3} + \frac{c_6 \cdot a_{2g+1}^3}{x^3}.
\]

Hence we can easily derive $c_6 = \frac{2\cdot a_{2g-1} \cdot a_{2g+1} - 2a_{2g}^2}{a_{2g+1}^4}$.

Then, since $c_6d_{-4} + c_4d_{-2} = 0$ we have that $d_{-2} = -\frac{c_6d_{-4}}{c_4} = -\frac{c_6}{c_4}$.

So if we write $\frac{1}{h}  = \sum_{i\geq -4} d_it^i$, then we have $\sum c_it^i \cdot \sum d_it^i = 1$.
Since the residue is $2d_{-2}$ (recall that $\frac{1}{x}dx = \frac{2}{h}\cdot t dt$, thus increasing the index by 1) we only need to compute enough to discover that term.
So  we have 
\[
d_{-2} = \frac{a_{2g+1}^4}{a_{2g}^2} \cdot \left( \frac{2a_{2g-1}a_{2g+1} - 2a_{2g}^2}{a_{2g+1}^4} \right) = \frac{2a_{2g-1}a_{2g+1} - 2a_{2g}^2}{a_{2g}^2}.
\]

Now we can assume that $a_{2g+1} = 1$, since this will not affect the curve.
This reduces the task of showing that the residue is non-zero to showing that $a_{2g-1} -a_{2g}^2 \neq 0$.
But if we write the roots of $f(x)$ as $\alpha_i \in k$ with $1\leq i \leq 2g+1$ then we can see that $a_{2g-1} -a_{2g}^2 = \sum_{i=1}^{2g+1}\alpha_i^2$.
{\bf If this sum is zero, then by an automorphism of the curve we can change one of the roots by a small value, and hence make it non-zero.}


\[
1 = c_4\cdot d_{-4} + (c_6\cdot d_{-4} + c_4\cdot d_{-2})t^2 + \ldots
\]
So $c_4 \cdot d_{-4} = 1$ and hence $d_{-4} = \frac{1}{c_4}$.


Try using a different uniformising parameter.
Let $t=\frac{y}{x^{g+1}+x}$.
Then 
\begin{eqnarray}
dt^2 & = & d\frac{f(x)}{(x^{g+1}+x)^2} \\
& = & 	\frac{1}{(x^{g+1}+x)^2}df + fd\left ( \frac{1}{(x^{g+1}+x)^2} \right ) \\
& = & 	\frac{1}{(x^{g+1}+x)^2}\cdot f'dx - \frac{2f((g+1)x^g + 1)}{(x^{g+1}+x)^3}dx \\
& = &  \frac{1}{x(x^g1)^2} \left ( f' - \frac{2f((g+1)x^{g+1}+1)}{(x^{g+1}+x)} \right) \frac{1}{x}dx 
\end{eqnarray}

We would like to compute $x$ in terms of the local parameter $t$.
We can possibly do this by writing $\frac{1}{x}$ in terms of $t$, and then using the fact that $x\cdot \frac{1}{x} = 1$ to complete the exercise.

In particular 
\[
\frac{1}{x}dx  = 
\]
Note - if the lead terms of the demoninator (i.e. $h(x)$) do not cancel then the residue should be easy to compute.
\bibliography{/home/jtait/files/Documents/Maths/Bibliography/biblio.bib}
%\bibliography{/home/joe/files/Documents/Maths/Bibliography/biblio.bib}
\bibliographystyle{plain}


\end{document}
