% !TEX TS-program = pdflatex
% !TEX encoding = UTF-8 Unicode

% This is a simple template for a LaTeX document using the "article" class.
% See "book", "report", "letter" for other types of document.

\documentclass[draft, 11pt]{article} % use larger type; default would be 10pt

\usepackage[utf8]{inputenc} % set input encoding (not needed with XeLaTeX)

%%% Examples of Article customizations
% These packages are optional, depending whether you want the features they provide.
% See the LaTeX Companion or other references for full information.

%%% PAGE DIMENSIONS
\usepackage{geometry} % to change the page dimensions
\geometry{a4paper} % or letterpaper (US) or a5paper or....
% \geometry{margins=2in} % for example, change the margins to 2 inches all round
% \geometry{landscape} % set up the page for landscape
%   read geometry.pdf for detailed page layout information

\usepackage{graphicx} % support the \includegraphics command and options
\usepackage{todonotes}

\usepackage[parfill]{parskip} % Activate to begin paragraphs with an empty line rather than an indent

%%% PACKAGES
\usepackage{mathtools}
\usepackage{booktabs} % for much better looking tables
\usepackage{array} % for better arrays (eg matrices) in maths
\usepackage{paralist} % very flexible & customisable lists (eg. enumerate/itemize, etc.)
\usepackage{verbatim} % adds environment for commenting out blocks of text & for better verbatim
\usepackage{subfig} % make it possible to include more than one captioned figure/table in a single float
% These packages are all incorporated in the memoir class to one degree or another...

%%% HEADERS & FOOTERS
\usepackage{fancyhdr} % This should be set AFTER setting up the page geometry
\pagestyle{fancy} % options: empty , plain , fancy
\renewcommand{\headrulewidth}{0pt} % customise the layout...
\lhead{}\chead{}\rhead{}
\lfoot{}\cfoot{\thepage}\rfoot{}

%%% SECTION TITLE APPEARANCE
\usepackage{sectsty}
\allsectionsfont{\sffamily\mdseries\upshape} % (See the fntguide.pdf for font help)
\usepackage{amsmath}
\usepackage{amsthm}
\usepackage{amsfonts}
\usepackage{mathrsfs}
\usepackage{amsopn}
\usepackage{amssymb}
\usepackage{natbib}
% (This matches ConTeXt defaults)

%%% ToC (table of contents) APPEARANCE
\usepackage[nottoc,notlof,notlot]{tocbibind} % Put the bibliography in the ToC
\usepackage[titles,subfigure]{tocloft} % Alter the style of the Table of Contents
\renewcommand{\cftsecfont}{\rmfamily\mdseries\upshape}
\renewcommand{\cftsecpagefont}{\rmfamily\mdseries\upshape} % No bold!

%Theorems and stuff
\theoremstyle{plain}
\newtheorem{defn}{Definition}[section]
\newtheorem{thm}[defn]{Theorem}
\newtheorem{cor}[defn]{Corollary}
\newtheorem{lem}[defn]{Lemma}
\newtheorem{prop}[defn]{Proposition}
\newtheorem{ex}[defn]{Example}
\newtheorem*{unnumthm}{Theorem}
\newtheorem{defnlem}[defn]{Definition/Lemma}
\newtheorem{defnthm}[defn]{Theorem/Definition}
\theoremstyle{remark}
\newtheorem*{rem}{Remark}


\newcommand{\cO}{{\cal O}}
\newcommand{\ra}{\rightarrow}
\newcommand{\NN}{{\mathbb N}}
\newcommand{\PP}{{\mathbb P}}
\newcommand{\ZZ}{{\mathbb Z}}
\newcommand{\cL}{{\mathcal L}}
\newcommand{\cA}{{\mathcal A}}
\newcommand{\cD}{{\mathcal D}}
\newcommand{\cU}{{\mathcal U}}


\DeclareMathOperator{\aut}{Aut}
\DeclareMathOperator{\res}{Res}
\DeclareMathOperator{\ord}{ord}
\DeclareMathOperator{\di}{div}
\DeclareMathOperator{\cha}{char}
\DeclareMathOperator{\gal}{Gal}
\DeclareMathOperator{\Tr}{Tr}

%%% END Article customizations

%%% The "real" document content comes below...

\title{}
\author{}
%\date{} % Activate to display a given date or no date (if empty),
         % otherwise the current date is printed 

\begin{document}
\maketitle

\section{Characteristic not 2}


Let $X$ be a smooth, projective, connected hyperelliptic curve of genus $g \geq 2$ over an algebraically closed field $k$ of characteristic unequal to 2.
Let $x:X\rightarrow \mathbb P_k^1$ be the corresponding morphism of degree two such that the associated field extension of $k(x)$ is given by $y$, where $y$ is defined by
\[
y^2 = f(x),
\]
for some $f(x)\in k[x]$ with no repeated roots.
By an automorphism of $\mathbb P_k^1$ we may assume that $0$ and $\infty$ are branch points.
We let $P_0$ and $P_\infty$ be the unique points in $x^{-1}(0)$ and $x^{-1}(\infty)$ respectively.
We may also assume that $f(x) = a_{2g+1}x^{2g+1} + a_{2g}x^{2g} + \ldots + a_1x$ for some $a_i \in k$, and $a_{2g+1} \neq 0$; note that the constant term is zero since $0\in \mathbb P_k^1$ is a branch point.
As the constant term is zero, we can conclude that $a_1 \neq 0$, else $x^2$ would divide $f(x)$ and we do not allow repeated roots.

We wish to compute a basis for the first de-Rham hypercohomology of this curve, which we denote by \todo{define this}{$H^1_{dR}(X/k)$.} 
To do this we use the fact that we have the following short exact sequence:
\begin{equation}\label{ses}
0 \ra H^0(X,\Omega_X) \ra H^1_{dR}(X/k) \ra H^1(X,\cO_X) \ra 0.
\end{equation}

We use \v{C}ech cohomology for our computations.
We have an affine open cover by $U_1 = X\backslash P_0$ and $U_2 = X \backslash P_\infty$, which we denote by ${\cal U} = \{ U_1, \ U_2\}$.
\todo{find citation}{By definition}  $\check{H}_{dR}^1(\cU)$ is the quotient of the $k$-vector space 
\[
\{(\omega_1, \omega_2, f_{12}) | \omega_i\in \Omega_{X/k}(U_i), f_{12}\in \cO_X(U_1 \cap U_2), df_{12} = \omega_1 - \omega_2\}
\]

by the subspace
\[
\{ (df_1, df_2, f_1-f_2)|f_i \in \cO_X(U_i)\}.
\]

\todo[inline]{check that I have been consistent with whether it is $(U_1,U_2,X)$ or $(U_2, U_1,X)$}
\begin{prop}\label{basish1char0}

Let $1 \leq g$.
Furthermore, let $s(x) = \left( f'(x) - \frac{2if}{x}\right) = b_{2g}x^{2g} + \ldots + b_0$ and then let 
\[
\phi = b_{2g} x^{2g} + \ldots b_{g+1}x^{g+1}
\]
and
\[ 
\psi = b_gx^g + \ldots + b_0.
\]
Then a basis of $H^1(X, \cO_X)$ is given by
\[
\left( \left(\frac{-\psi}{2xy^i}\right) dx, \left( \frac{\phi}{2yx^i} \right) dx, yx^{-i} \right).
\]

\end{prop}

Given the short exact sequence \eqref{ses} we can \todo{confirm this for above theorem}{compute when a group $G$ acting on $X$ acts faithfully on $H^1_{dR}(X/k)$ by computing a basis of the other terms in the sequence and seeing when $G$ acts faithfully on them.}
We know that the elements $\frac{dx}{y}, \frac{xdx}{y}, \ldots , \frac{x^{g-1}dx}{y}$ form a basis of $H^0(X,\Omega_X)$ from \cite[\S 7, Prop. 4.26]{liu}, so we only need to compute a basis of $H^1(X,\cO_X)$.\todo{add Bernhard's note}

%We start doing this by using Serre duality.
\todo{find citation}{We recall} that Serre duality gives a map $\text{Res}: H^0(X,\Omega_X) \times H^1(X,\cO_X) \rightarrow k$.
\todo{just sum over $U_1$ and not the whole of $X$?}This is defined by  \todo{what is f}{$(\omega,f) \rightarrow \sum_{P\in X}\text{res}_P(f\omega)$} for some $(\omega, f) \in   H^0(X,\Omega_X) \times H^1(X,\cO_X)$.
\todo[inline]{do I want to quantify $i$ and $j$? Certainly not $j$, maybe $i$}
We consider the pair $(y^{-1}x^idx,yx^{-j})$% \in  H^0(X,\Omega_X) \times H^1(X,\cO_X) 
, which maps to $\sum_{P\in X} \text{res}_P (x^{i-j}dx)$. \todo{As we're using Cech cohomology, we compute each open set. The only poles are at P0 and Pinfinity, so it is regular on the insterstion}
This can be calculated by summing the residues over the points in $U_1$ or by the residue theorem we can equivalently can compute the negative of the residue at $P_0$.
So for $\text{Res}(x^{i-j}dx)$ to be non-zero there must be a pole at at least one point in $U_1$ and at $P_0$.
We can check when this occurs by computing the divisor of $x^{i-j}dx$, which we do presently:
\begin{eqnarray}
	\di(x^{i-j}dx) & = &  \di(x^{i-j}) + \di(dx) \\
	& = & 2(i-j)[P_0] +  2(j-i)[P_\infty] + R - 4[P_\infty] \\
	& = & (2i-2j+1)[P_0] +(R-[P_0]-[P_\infty]) +(2j-2i-3)[P_\infty],
\end{eqnarray}
where $R$ is the ramification divisor of $x$.

Recall that since both $P_\infty$ and $P_0$ are ramification points it follows that $R-[P_0]-[P_\infty]$ is positive.
Moreover, since $X$ is a hyperelliptic curve the coefficient of every point in the ramification divisor is 1, we have a pole at both $P_0$ and $P_\infty$ if and only if $j=i+1$.
It follows that the only divisor of the form $x^{i-j}dx$ that can possibly have non-zero residue at $P_0$ and $P_\infty$ is $\frac{1}{x}dx$.
We will now compute exactly what the residue of $\frac{1}{x}dx$ is at $P_\infty$ to check that it is non-zero.


We now turn to computing precisely the residue of $x^{-1}dx$ at $P_\infty$.
To start with we note that $t:= \frac{y}{x^{g+1}}$ is a uniformising parameter at $P_\infty$.
Indeed we can compute the order of $t$ at $P_\infty$ as follows 
\begin{eqnarray}
	\ord_{P_\infty}(t) & = & \frac{1}{2}\ord_{P_\infty}(t^2) \\
 	& = & \frac{1}{2}\ord_{P_\infty}\left( \frac{f}{x^{2g+2}} \right) \\
	& = & \frac{1}{2}\ord_{P_\infty}(f(x)) - \frac{1}{2}\ord_{P_\infty}(x^{2g+2})\\
	& = & -(2g+1) + (2g+2) \\
	& = & 1.
\end{eqnarray}

\todo[inline]{rewrite using the quotient rule}
We now write $\frac{1}{x}dx$ in terms of $dt$.
First, 
\[
dt^2 = f\cdot d\frac{1}{x^{2g+2}} + \frac{1}{x^{2g+2}} \cdot df.
\]
Now 
\[
f\cdot d\frac{1}{x^{2g+2}} = -f\cdot \frac{2g+2}{x^{2g+3}} dx \ {\rm and}\ \frac{1}{x^{2g+2}} \cdot df = \frac{1}{x^{2g+2}}\cdot f' \cdot dx.
\]
\todo[inline]{check if $P_0$ would be easier - I don't think so...}

Hence
\[
dt^2 = \frac{1}{x^{2g+2}}\left(f' - \frac{(2g+2)f}{x}\right) dx
\]
and it then follows that
\[
\frac{1}{x}dx = \frac{-2tx^{2g+1}}{\left(\frac{(2g+2)f}{x} - f'\right)} dt.
\]
We now let $h = \left(\frac{(2g+2)f}{x} - f'\right)$, and noting that the coefficient of $x^{2g}$ in $h$ is $(2g+2)a_{2g+1} - (2g+1)a_{2g+1} = a_{2g+1}$, we see that $h$ is a degree $2g$ polynomial in $x$.
We wish to compute the coefficient of $t^{-1}$ in the expansion of $\frac{1}{x}dx$ at $P_\infty$ and computing the first coefficient of $\frac{h}{x^{2g+1}}$ is a key step.

Now
\[
\frac{h}{x^{2g+1}} = \frac{a_{2g+1}x^{2g}}{x^{2g+1}} + \ldots = \frac{a_{2g+1}}{x} + \ldots 
\]
Similarly 
\[
t^2 = \frac{f}{x^{2g+2} } = \frac{a_{2g+1}}{x} + \ldots
\]

\todo[inline]{check all summation notation}
Since $\ord_{P_\infty}\left(\frac{h}{x^{2g+1}}\right) = 2$ we know that $\frac{h}{x^{2g+1}} = \sum_{i\geq 2} c_it^i$ for some $c_i\in k$, and from the above computations we can see that $c_2 = 1$.
We also know that $\frac{x^{2g+1}}{h} = \sum_{i\geq -2} d_it^i$, for some $d_i\in k$.
Since
\[
1 = \left( \sum_{i\geq 2} c_it^i\right) \cdot \left( \sum_{i\geq -2} d_it^i\right)
\]
 and $c_2 = 1$ we conclude that $d_{-2} = 1$. 
Now 
\[
\frac{1}{x}dx = \left( -2t \cdot \sum_{i\geq -2} d_it^i\right) dt = \left( \sum_{i\geq -2}-2d_it^{i+1} \right) dt
\]
so we see that the residue is $-2$.

To make sure that $\frac{1}{x}dx$ is part of the basis for $H^1(\Omega_X,\cO_X)$ we need to find suitable differentials $\omega_1$ and $\omega_2$, regular on $U_1$ and $U_2$ respectively, such that $d(yx^{-j}) = \omega_1 - \omega_2$.
\todo[inline]{change all i's to j's}


We begin by computing the divisor of $dyx^{-i}$.
To start with, note that 
\begin{eqnarray*}
2yx^{-i}d(yx^{-i}) & = & d(yx^{-i})^2 \\
& = & dfx^{-2i} \\
& = & fdx^{-2i} + x^{-2i}df \\
& = & -f\cdot\frac{2i}{x^{2i+1}}dx + \frac{f'}{x^{2i}}dx \\
& = & \frac{1}{x^{2i}}\left( f' - \frac{2if}{x}\right) dx 
\end{eqnarray*}

Hence it follows that 
\begin{equation*}
d \left( yx^{-i} \right) = \frac{1}{2yx^{i}}\left( f' - \frac{2if}{x} \right) dx
\end{equation*}
and so
\todo[inline]{check compuation below, possibly note on a seperate piece of paper}
\todo[inline]{change $s$'s to $s_i$}
\[
\di (dyx^{-i}) = \di_0(s) -2i[P_0] +(2i-2g-3)[P_\infty],
\]
where $s(x) = \left ( f'-\frac{2if}{x} \right )$ and $\di_0(s(x))$ is the divisor of zeroes of $s$.
In general $s$ will be a polynomial of degree $2g$ in $x$.
\todo{check about $p$ dividing coefficients}However, if $p$ divides $(2g+1)a_{2g+1}- 2ia_{2g+1}$, the coefficient of $x^{2g}$, then the degree will be less, and in this case we will let $d_s$ be the degree of $s$.
Note that we will not in general know what the zeroes of $s$ are, but when we use $\di_0(s)$ this will not affect the coefficients of $[P_0]$ and $[P_\infty]$ in the divisors we are computing, and these are the relevant parts of the divisor (recall that $f$ has no repeated roots and has a zero at $0$, so $f'(0) \neq 0$).


We write $s(x) = b_{2g}x^{2g} + b_{2g-1}x^{2g-1} + \ldots + b_1x + b_0$.
We then let $\phi = b_{2g} x^{2g} b_{2g-1}x^{2g-1} + \ldots + b_{g+1}x^{g+1} $ and $\psi = s - \phi = b_rx^g + \ldots + b_0$.
Here we are denoting by $b_r$ the non-zero coefficient of $s(x)$ with greatest index less than or equal to $g$.

It is then clear that $\frac{\phi }{2yx^i}dx - \frac{-\psi}{2yx^i}dx = \frac{\phi + \psi}{2yx^i}dx = \frac{s}{2yx^i}dx$.

We then compute the divisors of both of these differentials to check that they are regular on the appropriate sets.
Firstly, if $d_s \leq g$ then $\phi = 0$ and is regular everywhere.
So we assume that $d_s > g$, and then we have
\begin{eqnarray*}
\di\left( \frac{\phi}{2yx^i}dx \right) & = & \di(\phi) -\di(y) - \di(x^i) - \di (dx) \\
& = & \di(\phi) - ( R - 2(g+1)[P_\infty]) - (2i[P_0] - 2i[P_\infty]) + (R - 4[P_\infty]) \\
& = & \di_0\left( \frac{\phi}{x^{g+1}}\right) + 2(g+1)[P_0] - 2d_s[P_\infty] - 2i[P_0] + 2(g-1)[P_\infty] \\
& \geq & \di_0\left( \frac{\phi}{x^{g+1}}\right) + 2(g+1)[P_0] - 4g[P_\infty] - 2i[P_0] + 2(g-1)[P_\infty] \\
& = & \di_0\left( \frac{\phi}{x^{g+1}} \right) + 2(i-g-1)[P_\infty] + 2(g-i+1)[P_0].
\end{eqnarray*}

Hence the differential $\frac{\phi}{2yx^i}$ is regular on $U_2$.

On the other hand
\begin{eqnarray*}
\di \left( \frac{-\psi}{2yx^i}dx\right) & = & \di(\psi) - \di(y) - \di(x^i) + \di (dx) \\
& = & \di (\psi ) -R + (2g+2)[P_\infty] - 2i[P_0] + 2i[P_\infty] + R -4[P_\infty] \\
& =  & \di_0(\psi) - 2b_r[P_\infty] + 2(g+i-2)[P_\infty] -2i[P_0] \\
& \geq & \di_0(\psi) - 2g[P_\infty] + 2(g+i-2)[P_\infty] -2i[P_0] \\
& = & \di_0(\psi) + 2(i-1)[P_\infty] - 2i[P_0].
\end{eqnarray*}

Hence the divisor is regular on $U_2$. 
Of course, we are assuming that $b_r$ is non-zero, as otherwise $\psi$ is zero and hence the differential is zero and hence trivially regular on all of $X$.


So if we now let $\omega_1 = \frac{-\psi}{2yx^i}dx$ and $\omega_2 = \frac{\phi}{2yx^i}dx$ then we have our two differentials.

\newpage

\section{characteristic 2}

We now consider the same question when $\cha (k) = 2$.
Our field extension corresponding to $X$ is now defined differently; it is $k(x,y)$, where
\[
y^2 - yh(x) = f(x)
\]
for some polynomials $f$ and $h$.
Since we are still assuming that $0$ and infinity are branch points, it follows that $\deg(f) = 2g+1$ and that $\deg(h)  = d  \leq g$. \todo{cite liu}
We suppose that $h(x) = \prod_1^m (x-a_i)^{n_i} = \beta_d x^d + \ldots + \beta_{n_1} x^{n_1}$, and that $P_i\in X$ are the corresponding ramification points.
We will assume that $a_1 = 0$.
It then follows that the ramification divisor has coefficients $\delta_P$ where
\[
\delta_P = \left \{ \begin{array}{ll}
2n_i & \text{if } P= P_i \text{ for some } i=1, \ldots , m \\
2(g+1-d) & \text{if } P=P_\infty \\
0 & \text{otherwise.}
\end{array}
\right.
\]


We now recall the divisors of some other elements of the function field which we will be using.
\begin{eqnarray*}
\di (dx) & = & R - 4[P_\infty] \\
\di (h(x)) & = & R - 2(g+1)[P_\infty] \\
\di (x^i) & = & 2i[P_0] - 2i[P_\infty]\\
\di (y) & = & \di_0(y) - (2g+1)[P_\infty]
\end{eqnarray*}

Though we have not properly described the zero divisor of $y$, it should be recalled that since $P_0$ is a ramification, $\di_0(y)$ will have a coefficient of at most one for $[P_0]$.
Details of the above are in my 18 month report.

As before, we will consider a pairing in $H^0(X,\Omega_X) \times H^1(X,\cO_X)$, and start by showing that it has poles at both $P_0$ and $P_\infty$.
As shown in the 18 month report, a basis of $H^0(X,\Omega_X)$ is $\frac{x^idx}{h(x)}$ for $0 \leq i \leq g-1$.
We consider the pairing $\left(\frac{x^idx}{h(x)}, x^{-j}y\right) \mapsto \frac{x^{i-j}ydx}{h(x)}$.
We compute the divisor
\begin{eqnarray*}
\di\left(\frac{x^{i-j}ydx}{h(x)}\right)  & = & 2(i-j)[P_0] - 2(i-j)[P_\infty] + \di_0(y) - (2g+1)[P_\infty] +R -4[P_\infty] \\
& & -R +2(g+2)[P_\infty] \\
& = &  \di_0(y) + 2(i-j)[P_0] +(2(j-i)-3)[P_\infty].
\end{eqnarray*}

Clearly we have a pole at both $P_0$ and $P_\infty$ if and only if $j=i+1$, and in this case the order of the pole at both points is $-1$.

We now wish to compute the residue of $\frac{ydx}{xh(x)}$ at $P_\infty$.
We start by noting that $t = \frac{y}{x^{g+1}}$ is a uniformising parameter at $P_\infty$; we check this by computing the divisor:
\[
\di(t) = \di_0(y) - (2g+1)[P_\infty] -(2g+2)[P_0] + (2g+2)[P_\infty] = \di_0(y)-(2g+2)[P_0] + [P_\infty].
\]

So clearly $t$ is a uniformising parameter at $P_\infty$.

We now wish to write $\frac{y}{xh(x)}dx$ as $r(x,y)dt$ for some $r \in k(x,y)$.
We first write $dy$ in terms of $dx$.
Since
\begin{eqnarray*}
0 & = & dy^2 \\
& = & f'dx + hdy + yh'dx
\end{eqnarray*}
we conclude that 
\[
dy = \left( \frac{f'+yh'}{h} \right) dx.
\]
We now rewrite $dt$ as follows:
\begin{eqnarray*}
dt & = & yd\frac{1}{x^{g+1}} + \frac{1}{x^{g+1}}dy \\
& = & \frac{1}{x^{g+1}} \left( \frac{(g+1)y}{x} + \frac{f'+yh'}{h} \right) dx \\
& = & \frac{1}{x^{g+1}} \left( \frac{xf'}{y} + xh' + (g+1)h \right) \frac{y}{hx} dx.
\end{eqnarray*}

In total we then have
\[
\frac{y}{xh(x)}dx = \frac{x^{g+1}y}{s(x,y)}dt
\]
where $s(x,y) = xf' + yxh' + (g+1)yh$.

\todo{show that the order of this function is -1 at $P_\infty$}{We know that $\frac{x^{g+1}y}{s(x,y)} = \sum_{i\geq -1} c_i t^i$, with $c_i \in k$, and we wish to compute $c_{-1}$.}


We shall do this by computing the coefficient of $t$ in the expansion $\frac{s(x,y)}{x^{g+1}y} = \sum_{i\geq 1}d_it^i$.
We can split up $\frac{s(x,y)}{x^{g+1}y}$ in to three terms, namely $\frac{xf'}{x^{g+1}y}$, $\frac{yxh'}{x^{g+1}y}$ and $\frac{yh}{x^{g+1}y}$.
Each of these terms can of course be written as a power series in $t$, but only the first term mentioned has order 1 at $P_\infty$, and hence this term will uniquely determine $d_1$.
Suppose that $f = \alpha_{2g+1}x^{2g+1} + \alpha_{2g}x^{2g} + \ldots + \alpha_1x^1 + \alpha_0$.
Then $xf'= \alpha_{2g+1}x^{2g+1} + \alpha_{2g-1}x^{2g-1} + \ldots + \alpha_1x^1$; i.e. the terms with an even power are removed.

So the only term in $\frac{xf'}{x^{g+1}y}$ of order 1 at $P_\infty$ is $\frac{\alpha_{2g+1}x^{2g+1}}{x^{g+1}y} = \frac{\alpha_{2g+1}x^{g}}{y}$.
Since
\[
\frac{\alpha_{2g+1}x^g}{y} = \frac{\alpha_{2g+1}}{x}t^{-1}
\]
and $\frac{1}{x} = \sum_{i\geq 2}e_it^i$ for some $e_i \in k$, if we compute $e_2$ then we will have effectively computed $d_1$.
Now $t^2 = \frac{f }{x^{2g+2}}+ \frac{hy}{x^{2g+2}}$, and clearly $\frac{hy}{x^{2g+2}}$ has no terms of the form $\frac{c}{x}$ for some $c \in k$. 
On the other hand
\[
\frac{f}{x^{2g+2}} = \frac{\alpha_{2g+1}}{x} + \ldots
\]
Hence we conclude that $e_2 = \frac{1}{\alpha_{2g+1}}$.
It follows that $d_1 = \alpha_{2g+1} \cdot \frac{1}{\alpha_{2g+1}} = 1$.


We finally use this to compute $c_{-1}$.
Since $1 = \frac{s(x,y)}{x^{g+1}y}\cdot \frac{x^{g+1}y}{s(x,y)} = \left( \sum_{i\geq 1}d_it^i \right) \cdot \left( \sum_{i\geq -1}c_it^i\right)$, we conclude that $c_{-1} = \frac{1}{d_{1}} = 1$.


So now we need to find $\omega_1$ and $\omega_2$, regular on $U_1$ and $U_2$ respectively, such that $d(yx^{-i}) = \omega_1 - \omega_2$.
We start by rewriting $dyx^{-i}$ as a divisor of $dx$, as follows
\begin{eqnarray*}
dyx^{-i} & = & x^{-i}dy + ydx^{-i} \\
& = & \frac{1}{x^i} \left( \frac{f' + yh'}{h}\right) dx + \frac{iy}{x^{i+1}}dx \\
& = & \left( \frac{f'}{x^ih} + \frac{yh'}{x^ih} + \frac{iy}{x^{i+1}} \right) dx.
\end{eqnarray*}

We compute the divisors of each term to see when they are regular on $U_1$ or $U_2$.
To start with we have
\begin{eqnarray*}
\di \left( \frac{f'dx}{x^ih} \right) & = & \di (f') + \di ( dx ) - \di (x^i) - \di (h) \\
& = & \di_0 (f') - 4g[P_\infty] + R - 4[P_\infty] - 2i[P_0] + 2i[P_\infty] -R + (2g+2)[P_\infty] \\
& = & \di_0(f') - 2(g+)i[P_0] +2(i-1)[P_\infty].
\end{eqnarray*}


Before computing the next divisor we define $H' := \di_0(h') - 2(n_1 -1)[P_0]$, and hence we have $\di_0(h') = H' + 2(n_1-1)[P_0]$.
We can now compute the divisor of the second term as
\begin{eqnarray*}
\di\left( \frac{yh'dx}{x^ih} \right)  & = & \di_0(y) - (2g+1)[P_\infty] + \di_0(h') - 2(d-1)[P_\infty] + R - 4[P_\infty] \\
& & - 2i[P_0] + 2i[P_\infty] - R + (2g+2)[P_\infty] \\
& = & \di_0(y) - 2g[P_\infty] - [P_\infty] + \di_0(h') - 2d[P_\infty] + 2[P_\infty] - 4[P_\infty] -\\
& & 2i[P_0] + 2i[P_\infty]+ 2g[P_\infty] + 2[P_\infty]  \\
& = & \di_0(y) + (2i-2d-1)[P_\infty] + \di_0(h') - 2i[P_0] \\
& = & \di_0(y) + (2i -2d -1) [P_\infty] + H' + (2(n_1 - 1 - i)[P_0].
\end{eqnarray*}

Note that $H'$ is a positive  divisor with no $[P_0]$ or $[P_\infty]$ terms, so it will not affect whether the differential is regular on $U_1$ or $U_2$.


We now define $R' = R - 2(g+1-d)[P_\infty] - 2n_1[P_0]$, so that $R = R' + 2(g+1-d)[P_\infty] + 2n_1[P_0]$.
Then the third divisor is
\begin{eqnarray*}
\di \left( \frac{iydx}{x^{i+1}} \right) & = & \di_0(y) - (2g+1)[P_\infty] + R - 4[P_\infty] - 2(i+1)[P_0] + 2(i+1)[P_\infty] \\
& = & \di_0(y) - 2g[P_\infty] - [P_\infty] + R' + 2g[P_\infty] + 2[P_\infty] - 4[P_\infty] - 2d[P_\infty] +\\
& &  2n_1[P_0] - 2i[P_0] -2[P_0] + 2i[P_\infty] +2[P_\infty]  \\
& = & \di_0(y) + R' + (2n_1 - 2i -2)[P_0] + (2i -2d -1)[P_\infty]
\end{eqnarray*}
when $i$ is odd, and is zero if $i$ is even.

As with $H'$, $R'$ is a positive divisor with no $[P_0]$ or $[P_\infty]$ terms.



We start by writing $\frac{fdx}{x^ih}$ as the sum of two divisors, one regular on $U_1$ and the other regular on $U_2$.
We start by writing $f = \alpha_{2g+1} x^{2g+1} + \ldots + \alpha_1x$, and hence we have $f' = \alpha_{2g+1}x^{2g} + \ldots + \alpha_1$.
We let $\phi (x) = \alpha_{2g+1}x^{2g} + \ldots + \alpha_{i+1}x^i$ and $\psi (x) = \alpha_{i-1}x^{i-2} + \ldots + \alpha_1$.
Of course we do not know that any of the coefficients other $\alpha_{2g+1}$ are non-zero.
In particular, this means that $\psi(x)$ has a pole of order less than or equal to $2(i-1)$ at $P_\infty$, and that $\phi(x)$ has a zero of order at least $2i$ at $P_0$, facts we will use below.

Then we have
\begin{eqnarray*}
\di \left( \frac{\phi dx}{x^i h} \right) & = & \di(\phi) + R - 4[P_\infty] - 2i[P_0] + 2i[P_\infty] - R + (2g+2) [P_\infty] \\
& = & \di_0(\phi) - 4g[P_\infty] - 4[P_\infty] - 2i[P_0] + 2i[P_\infty] + 2g[P_\infty] + 2[P_\infty] \\
& = & \di_0(\phi) - 2i[P_0] + 2(i-g-1)[P_\infty] \\
& = & \di_0\left( \frac{\phi}{x^i} \right) + \di_0( x^i) - 2i[P_0] + 2(i-g-1)[P_\infty] \\
& = & \di_0 \left( \frac{\phi}{x^i} \right) + 2(i-g-1)[P_\infty].
\end{eqnarray*}
Notice that $\frac{\phi}{x}$ may still have a zero at $P_0$, but this will not affect the above differential being regular on $U_1$.




Similarly 
\begin{eqnarray*}
\di\left( \frac{\psi dx}{x^ih} \right) & = & \di(\psi) + R - 4[P_\infty] - 2i[P_0] + 2i[P_\infty] - R + 2(g+1)[P_\infty] \\
& \geq & \di_0(\psi ) - 2(i-1)[P_\infty] - 4[P_\infty] - 2i[P_0] + 2(g+i + 1)[P_\infty] \\
& = & \di_0(\psi) - 2i[P_0] + 2(g+1)[P_\infty].
\end{eqnarray*}
Note that we have equality on the second line if and only if $\alpha_{i-1}$ is non-zero.
Of course, even if we do not have equality, the differential is still regular on $U_2$.



So we have rewritten the $\frac{fdx}{x^ih}$ as the sum of two differentials (dependent on $i$), one of which is regular on $U_1$ and the other regular on $U_2$.



If we now suppose that $i$ is even, then the only term left to consider is
$\frac{yh'dx}{x^ih}$.
If $i$ is greater than $d$ or less than $n_1$ then we can see from the divisor of the differential that it will be regular on $U_2$ and $U_1$ respectively, so we suppose that $d \geq i \geq n_1$.
If $h'$ is zero then the differential is trivially regular on both sets, so we assume $h'$ is not zero.
Then we suppose that the highest term in $h'$ with non-zero coefficient has degree $d'$ and that the lowest term with non-zero coefficient has degree $e'$.
Then we can write $h' = \beta_{d'+1} x^{d'} + \ldots + \beta_{e'+1} x^{e'}$.
\todo{specify that this is when i is between d' and e'}{We then define two polynomials, $\theta = \beta_{d'+1} x^{d'} + \ldots + \beta_{i+1}x^i$ and $\psi = \beta_{i-1}x^{i-2} + \ldots + \beta_{e'+1}x^{e'}$, whose sum is $h'$.}
Again, either of these polynomials could be zero, and if so the differentials whose divisors we compute next will obviously be zero, and the differentials will trivially be regular on both $U_1$ and $U_2$.
Hence we will assume in the computations below that neither polynomial is the zero function.



It should also be noted that since we are assuming $i$ is even we know that $\psi$ does not have an $x^{i-1}$ term, since the differential of $x^i$ is zero.
Also, as in the previous case, we know only know that $\theta$ has a zero of order at least $2i$ at $P_0$, rather than precisely $2i$.
We now compute the divisors of $\frac{y\theta dx}{x^ih}$ and $\frac{y\psi dx}{x^ih}$. 
\begin{eqnarray*}
\di\left(\frac{y\theta dx}{x^ih}\right) & = & \di_0(y) - (2g+1)[P_\infty] + \di(\theta) + R - 4[P_\infty] -2i[P_0] + 2i[P_\infty] - R \\
& &  + (2g+2)[P_\infty] \\
& = & \di_0(y) + \di(\theta) -2i[P_0] + (2i -3)[P_\infty] \\
& = & \di_0(y) + \di_0(\theta) - 2d'[P_\infty] - 2i[P_0] + (2i-3)[P_\infty] \\
& = & \di_0(y) + \di_0\left(\frac{\theta}{x^i}\right) + 2i[P_0] -2i[P_0] + (2i-2d'-3)[P_\infty]\\
& = & \di_0(y) + \di_0\left(\frac{\theta}{x^i} \right) + (2i-2d'-3)[P_\infty]
\end{eqnarray*}

and

\begin{eqnarray*}
\di\left( \frac{y\psi dx}{x^ih} \right) & = & \di_0(y) - (2g+1)[P_\infty] + \di(\psi) + R - 4[P_\infty] - 2i[P_0] + 2i[P_\infty] \\
& & - R + (2g+2)[P_\infty] \\
& = & \di_0(y) + \di(\psi) -2i[P_0] + (2i-3)[P_\infty] \\
& \geq & \di_0(y) + \di_0(\psi) - 2(i-2)[P_\infty] - 2i[P_0] + (2i-3)[P_\infty] \\
& = & \di_0(y) + \di_0(\psi) -2i[P_0] + [P_\infty].
\end{eqnarray*}
Note that we have equality in the third line if and only if $\beta_{i-2}$ is non-zero.
So we have now written $\frac{yh'dx}{x^ih}$ as the sum of two divisors, one regular on $U_1$ and the other regular on $U_2$, and this completes our basis in the case that $i$ is even.




We now consider the case what happens when $i$ is odd, and in this case we will consider the last two terms simultaneously.
It is clear that
\[
\left( \frac{yh'}{x^ih} + \frac{iy}{x^{i+1}} \right) dx  =  y \left( \frac{xh' + h}{hx^{i+1}} \right) dx.
\]

Now $xh' + h$ is precisely the polynomial formed by removing all the even terms from $h$.
Of course, this could be zero, but in that case the differential above is trivially regular on both $U_1$ and $U_2$ and there is nothing to do.
Also, as in the previous case, we need only consider what happens when $d \geq i \geq n_1$.
We assume the polynomial is non-zero, and of degree $d''$, with lowest non-zero term of degree $e''$.
Hence we know that $xh' + h = \beta_{d''+1} x^{d''} + \ldots + \beta_{e''+1}x^{e''}$.
Of course, this could be zero, but in that case the differential above is trivially regular on both $U_1$ and $U_2$ and there is nothing to do.
So we assume the polynomial is non-zero, and of degree $d''$, with lowest non-zero term of degree $e''$.



The only case we need to consider is when $e'' \leq i \leq d''$, and in this case we define $\phi = \beta_{d''+1} x^{d''} + \ldots + \beta_{i+1}^{i+1}$ and $\psi = \beta_{i-1} + x^{i-1} + \ldots + \beta_{e''+1}x^{e''}$.
Then it is clear that $\frac{y\psi}{x^{i+1}}dx + \frac{y\phi}{x^{i+1}}dx = y\left( \frac{xh' + h}{x^{i+1}} \right) dx$.
We then compute the divisors of these differentials, as follows
\begin{eqnarray*}
 \di \left( \frac{y \phi}{hx^{i+1}} dx \right) & = & \di_0(y) - (2g+1)[P_\infty] + \di( \phi) + R - 4[P_\infty] - 2(i+1)[P_0] \\
& ~ & + 2(i+1)[P_\infty] -R + 2(g+1)[P_\infty] \\
& = & \di_0(y) + \di(\phi) -(2g+1)[P_\infty] + R' + 2(g+1-d)[P_\infty] + 2n_1[P_0] - 2(i+1)[P_0] \\ 
& ~ & + 2(i-1)[P_\infty] -R' - 2n_1[P_0]- 2(g+1-d)[P_\infty] +2(g+1)[P_\infty] \\
& = & \di_0(y) + \di(\phi) + R' - 2(i +1)[P_0] + (2i-2)[P_\infty] \\
& \geq & \di_0(y) + \di_0(\phi) + R' - 2(i + 1)[P_0] + (2i - 2d'' - 2)[P_\infty] \\
& = & \di_o(y) + \di_0\left( \frac{\phi}{x^{i+1}} \right) + R' + 2(i - d'' -1)[P_\infty]. 
\end{eqnarray*}

Similarly 
\begin{eqnarray*}
\di \left( \frac{\psi y}{hx^{i+1}} dx \right) & = & \di_0(y) -(2g+1)[P_\infty] + \di(\psi) +R -4[P_\infty] - 2(i+1)[P_0] \\
&  & + 2(i+1)[P_\infty] - R + 2(g+1)[P_\infty] \\
& = & \di_0(y) + \di(\psi) -2(i+1)[P_0] + (2i-1)[P_\infty] \\
& \geq & \di_0(y) + \di_0(\psi) - 2(i-1)[P_\infty] - 2(i+1)[P_0] + (2i+1)[P_\infty] \\
& = & \di_0(y) + \di_0(\psi) - 2(i+1)[P_0] + [P_\infty]
\end{eqnarray*}


So when $i$ is odd we can also write the last two terms as a sum of differentials, with one regular on $U_1$ and the other regular on $U_2$. 

In total we can always write the differentials as a sum of two others, regular on the correct sets, thus we have a basis of $H^1(X,\cO_x)$ when $\cha (k) =2$.


\bibliography{/home/jtait/files/Documents/Maths/Bibliography/biblio.bib}
%\bibliography{/home/joe/files/Documents/Maths/Bibliography/biblio.bib}
\bibliographystyle{plain}


\end{document}
