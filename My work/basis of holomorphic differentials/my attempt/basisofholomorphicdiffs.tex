% !TEX TS-program = pdflatex
% !TEX encoding = UTF-8 Unicode

% This is a simple template for a LaTeX document using the "article" class.
% See "book", "report", "letter" for other types of document.

\documentclass[11pt]{article} % use larger type; default would be 10pt

\usepackage[utf8]{inputenc} % set input encoding (not needed with XeLaTeX)

%%% Examples of Article customizations
% These packages are optional, depending whether you want the features they provide.
% See the LaTeX Companion or other references for full information.

%%% PAGE DIMENSIONS
\usepackage{geometry} % to change the page dimensions
\geometry{a4paper} % or letterpaper (US) or a5paper or....
% \geometry{margins=2in} % for example, change the margins to 2 inches all round
% \geometry{landscape} % set up the page for landscape
%   read geometry.pdf for detailed page layout information

 \usepackage{graphicx} % support the \includegraphics command and options

\parindent0ex % Activate to begin paragraphs with an empty line rather than an indent

%%% PACKAGES
\usepackage{mathtools}
\usepackage{booktabs} % for much better looking tables
\usepackage{array} % for better arrays (eg matrices) in maths
\usepackage{paralist} % very flexible & customisable lists (eg. enumerate/itemize, etc.)
\usepackage{verbatim} % adds environment for commenting out blocks of text & for better verbatim
\usepackage{subfig} % make it possible to include more than one captioned figure/table in a single float
% These packages are all incorporated in the memoir class to one degree or another...

%%% HEADERS & FOOTERS
\usepackage{fancyhdr} % This should be set AFTER setting up the page geometry
\pagestyle{fancy} % options: empty , plain , fancy
\renewcommand{\headrulewidth}{0pt} % customise the layout...
\lhead{}\chead{}\rhead{}
\lfoot{}\cfoot{\thepage}\rfoot{}

%%% SECTION TITLE APPEARANCE
\usepackage{sectsty}
\allsectionsfont{\sffamily\mdseries\upshape} % (See the fntguide.pdf for font help)
\usepackage{amsmath}
\usepackage{amsthm}
\usepackage{amsfonts}
\usepackage{mathrsfs}
\usepackage{amsopn}
\usepackage{amssymb}
\usepackage{natbib}
% (This matches ConTeXt defaults)

%%% ToC (table of contents) APPEARANCE
\usepackage[nottoc,notlof,notlot]{tocbibind} % Put the bibliography in the ToC
\usepackage[titles,subfigure]{tocloft} % Alter the style of the Table of Contents
\renewcommand{\cftsecfont}{\rmfamily\mdseries\upshape}
\renewcommand{\cftsecpagefont}{\rmfamily\mdseries\upshape} % No bold!

%Theorems and stuff
\theoremstyle{plain}
\newtheorem{defn}{Definition}[section]
\newtheorem{thm}[defn]{Theorem}
\newtheorem{cor}[defn]{Corollary}
\newtheorem{lem}[defn]{Lemma}
\newtheorem{prop}[defn]{Proposition}
\newtheorem{ex}[defn]{Example}
\newtheorem*{unnumthm}{Theorem}
\newtheorem{defnlem}[defn]{Definition/Lemma}
\newtheorem{defnthm}[defn]{Theorem/Definition}
\theoremstyle{remark}
\newtheorem*{rem}{Remark}


\newcommand{\cO}{{\cal O}}
\newcommand{\ra}{\rightarrow}
\newcommand{\NN}{{\mathbb N}}
\newcommand{\PP}{{\mathbb P}}
\newcommand{\ZZ}{{\mathbb Z}}
\newcommand{\cL}{{\mathcal L}}
\newcommand{\cA}{{\mathcal A}}
\newcommand{\cD}{{\mathcal D}}


\DeclareMathOperator{\aut}{Aut}
\DeclareMathOperator{\ord}{ord}
\DeclareMathOperator{\di}{div}
\DeclareMathOperator{\cha}{char}
\DeclareMathOperator{\gal}{Gal}
\DeclareMathOperator{\Tr}{Tr}

%%% END Article customizations

%%% The "real" document content comes below...

\title{More hyperelliptic curves}
\author{J Tait}
%\date{} % Activate to display a given date or no date (if empty),
         % otherwise the current date is printed 

\begin{document}
\maketitle

  \subsection{Hyperelliptic curves, characteristic unequal to 2}\label{charneq2}
  Let $X$ be a smooth, projective, connected hyperelliptic algebraic curve of genus $g$ over an algebraically closed field $k$, with $\cha (k) \neq 2$.
Since $X$ is hyperelliptic there is, by definition, a degree two holomorphic map $x:X\rightarrow \mathbb P_k^1$.
The map $x$ generates a function field $k(x)$, which is isomorphic to the function field of $\mathbb P_k^1$.
Then by \cite[\S 7.4.3]{liu} there exists some $f(x) \in k[x]$ with no repeated roots, such that if we define $y$ by
\begin{equation}\label{definingequation}
	y^2 = f(x)
\end{equation}
then $k(x,y)$, the degree two extension of $k(x)$ by $y$, is isomorphic to the function field of $X$.


We will now compute the ramification divisor of $x$; note that in the following lemma we will not distinguish between points in $\mathbb P_k^1\backslash \{\infty\}$ and their corresponding points in $\mathbb A_k^1$.

\begin{lem}
The ramification divisor $R$ of $x$ is
\begin{equation*}
R = [P_1] + \ldots + [P_{2g+2}]
\end{equation*}
for distinct points $P_1, \ldots , P_{2g+2} \in X$.
We let $a_i := x(P_i)$ for $1\leq i \leq 2g+2$, and then one of the following statements holds:
\begin{enumerate}[(i)]
	\item Infinity is a branch point, and if we let $a_1 = \infty$, then $a_2,\ldots ,a_{2g+2}$ are all the roots of $f(x)$ in $\mathbb A_k^1$.
	\item Infinity is not a branch point and $a_1, \ldots , a_{2g+2}$ are all the roots of $f(x)$ in $\mathbb A_k^1$.
\end{enumerate}
\end{lem}
\begin{proof}
It follows from the Riemann-hurwitz formula, Corollary (??), that $\deg(R) = 2g - 2 +2\cdot2 = 2g+2$.
Since the degree of $x$ is two and $x$ is only tamely ramified, the coefficents of the ramification divisor are all 1.
Hence $R= [P_1] + \ldots + [P_{2g+2}]$ for some distinct points $P_i \in X$.


To study these points we will need to consider $X'$, the affine curve defined by \eqref{definingequation}, and $x':X'\rightarrow \mathbb A_k^1$, the restriction of $x$ to $X'$.
Note that as $f(x)$ has no repeated roots then $X'$ is smooth.
Clearly for any $a\in \mathbb A_k^1$ which is not a solution of $f(x)$ there are two points in the pre-image of $a$ under $x'$, namely $(a, \pm \sqrt {f(a)})$, and hence $a$ is not a branch point.
On the other hand, if $a \in \mathbb A_k^1$ is a solution of $f(x)$ then the only point in the pre-image of $a$ under $x'$ is $(a,0)$, and hence $a$ is a branch point.
It follows that the only branch points of $x$ in $\mathbb P_k^1$ are those corresponding to solutions of $f(x)$ and possibly infinity.

Hence if infinity is not a branch point then $a_1, \ldots , a_{2g+2}$ are precisely all the solutions of $f(x)$. 
On the other hand, if infinity is a branch point and $a_1=x(P_1)=\infty$, it follows that $a_2,\ldots ,a_{2g+2}$ must be all the solutions to $f(x)$.
\end{proof}

We can easily see from this lemma that if infinity is a branch point then $\deg(f(x)) = 2g+1$, and otherwise $\deg(f(x)) = 2g+2$.


\begin{comment}
We can define $j\in \aut(X)$ to be the map defined $q_1 \mapsto q_2$ for $q_1, q_2 \in x^{-1}(a)$. 
Obviously this is the identity on the ramification points, and since $x$ is of degree $2$, $j^2$ is the identity map.
We will now use this to show the existence of one more function, $y$, which is necessary to define the basis of $H^0(X,\Omega_X^{\otimes m})$.

If we let $D$ be the divisor $(g+1)p + (g+1)q$ on $k$, then there exists a $y \in \cL(D)$ (unique up to a factor) such that $j^* (y) = -y$.
To start, we have by the Riemann-Roch theorem that $l(D) = \dim\cL(D) = (2g+2)-g+1 = g+3$.
So $\cL(D) \cong k^{g+3}$.
Since $x(p) = x(q) = \infty$ then $j^*$ defines a linear map $\cL(D) \rightarrow \cL(D)$.
Since $j^{*2} = 1$, the eigenvalues must be $\pm 1$, and as such we can decompose $\cL(D)$ in two subspaces, $\cL(D)^+$ and $\cL(D)^-$, corresponding to the eigenspaces of eigenvalues $1$ and $-1$ respectively.
Note that $\cL(D) = \cL(D)^+ \oplus \cL(D)^-$.

If $f\in \cL(D)^+$ then this means that $j^*(f(q))=f(j(q))=f(q)$ for all $q \in X$.
But $x$ maps $q$ and $j(q)$, and only these two points, to the same point in $\mathbb P_k^1$ for all $q\in X$.
So any meromorphic function in $\cL(D)^+$ can be written as a composition of $x$ and a meromorphic function on $\mathbb P_k^1$, such that the composition only has poles at $p$ and $q$ (i.e. the meromorphic function on $\mathbb{P}_k^1$ only has poles at infinity.)
The order of the pole at $p$ or $q$ cannot exceed $g+1$, hence $1,x,\ldots ,x^{g+1}$ forms a basis of $\cL(D)^+$.
As this implies that $\dim\cL(D)^+ = g+2$, and as $\dim\cL(D) = g+3$, we see that $\dim\cL(D)^- = 1$, and so there is a non-trivial meromorphic function $y \in \cL(D)$ such that $j^*(y) = -y$.
This is the $y$ we will use in the following proposition.\\
\end{comment}


Now we define precisely what the divisor of a poly-differential is.
If we consider an element of the tensor product $\omega \in \Omega_X^{\otimes m}$ then it can be written as $y dx_1\otimes \ldots \otimes dx_m$, where $y, x_j \in K(X)$ for $1 \leq j \leq m$.
Let $P$ be a point in~$X$.
Since each $dx_j$ can be written as $y_j dt$ for some $y_j\in K(X)$ and some uniformising parameter $t$ at $P$, we can rewrite $\omega$ as $y' dt \otimes \ldots \otimes dt$, where $y' = y \cdot y_1 \cdots y_m$.
We then define the order of $\omega$ at $P$ to be $\ord_P(\omega ) := \ord_P(y')$.
Now we restrict to the case where $\omega = ydx \otimes \ldots \otimes ydx = y^mdx^{\otimes m}$.
For any $P \in X$ then $dx = zdt$ for some $z,t \in K(X)$, where $t$ is a uniformising parameter at $P$.
Hence $\omega = y^m z^m dt^{\otimes m}$, and we have
\begin{equation*}
	\ord_P(\omega) = \ord_P(y^mz^m) = m\ord_P(yz) = m\ord_P(ydx),
\end{equation*}
from which it follows that $\di(\omega) = m\di(ydx)$.


We now prove the main proposition of this section.
\vskip1em



\begin{prop}\label{propneq2}
Let $m\geq 1$.
Let $X$, $x$ and $y$ be as above, and let $\omega := \frac{dx^{\otimes m}}{y^m}$. 
Then if $g\geq 2$, a basis of $H^0(X,\Omega_X^{\otimes m})$ is given by


$\begin{cases}
 \omega, x\omega, \ldots , x^{g-1}\omega &  \mbox{if}\ m=1 \\
 \omega, x\omega, x^2\omega & \mbox{if}\ m=g=2 \\
 \omega, x\omega, \ldots, x^{m(g-1)}\omega;\  y\omega, xy\omega, \ldots ,x^{(m-1)(g-1)-2}y\omega & \mbox{otherwise.}
 \end{cases}
$
\end{prop}

\vskip1em

\begin{rem}
 Note that the case where $m=1$ is treated in \cite[7.4.3]{liu} and \cite[$\S$4, Ch. IV]{griffiths}.
\end{rem}

\begin{proof}
We first check that the number of poly-differentials listed in the proposition is equal to the dimension of $H^0(X,\Omega_X^{\otimes m})$.
When $m=1$ we have $g$ differentials, and by Proposition (??) the dimension of $H^0(X,\Omega_X^{\otimes m})$ is $g$.
If $g,m\geq 2$ then $\dim_kH^0(X,\Omega_X^{\otimes m}) = (2m-1)(g-1)$.
When $m=g=2$ then $(2m-1)(g-1) = 3$, and this is how many poly-differentials are listed in this case.
Otherwise there are $m(g-1) + (m-1)(g-1)-2 = 2mg - 2m - g -1 = (2m-1)(g-1)$ poly-differentials listed, as desired.

We now show that the elements are linearly independent over $k$.
Since $\omega$ is fixed, it is equivalent to show that the coefficients of $\omega$ are linearly independent over $k$ - i.e. that $1,x,\ldots ,x^n, y, xy, \ldots ,x^ly$ are linearly independent over $k$ for any $n$ and $l$.
The only time this linear independence is not immediate is when we consider whether one term of the form $x^i$ is linearly independent of another of the form $x^jy$.
But if two such elements were linearly dependent then the extension $k(x,y)$ of $k(x)$ would be degree one, which is not the case.

Finally we need to show that the listed elements in the proposition are indeed holomorphic differentials, i.e. that their divisors are non-negative.
We first compute the divisors associated to $x$, $y$ and $dx^{\otimes m}$.
\begin{comment}
For any $f\in K(X)$ we will denote by $(f)_0$ and $(f)_\infty$ the divisor of zeroes and divisor of poles of $f$ respectively.
In particular, 
\[
          (f)_0 := \sum_{\{P\in X|\ord_P(f)>0\}}\ord_P(f)[P],      
 \]
 and 
 \[
 (f)_\infty = \sum_{\{P\in X |\ord_P(f)<0\}} -\ord_P(f)[P].
 \]
\end{comment}

Let $D_0 = [P']+[Q']$ with $P',Q' \in X$ (note that we could have $P' = Q'$) be the divisor of zeroes of $x$. 
Then $ i D_0$ is the divisor of zeroes of $x^i$. 
Similarly, let $D_\infty = [P''] + [Q'']$ with $P'', Q'' \in X$ (again, we could have $P''=Q''$) be the divisor of poles of $x$. 
Again, $iD_\infty$ is the divisor of poles of $x^i$, and hence $\div(x^i) = iD_0 - iD_\infty$.

To compute $\di (dx^{\otimes m})$, we first note that it suffices to compute the divisor of $dx$, since $\di (dx^{\otimes m}) =m\di (dx)$, as above.
Since $x$ can be viewed either as the projection of $X$ on to the projective line or as a function on the projective line, we use $\di_X (dx)$ and $\di_{\mathbb P^1} (dx)$ to differentiate these cases.
Recalling the Riemann-Hurwitz formula (Theorem \ref{detailedhurwitz}) we see that
\[
 \di_X (dx) = x^*( \di_{\mathbb P^1}(dx)) + R = R - 2D_{\infty},
\]
since $\di_{\mathbb P^1}(dx) = -D_\infty$ and hence $x^* (\di_{\mathbb P^1}(dx)) = -2D_\infty$.

Finally, we compute $\di (y)$.
Since $\di (y^2) = \di (f(x))$ and hence $\di(y) = \frac{1}{2}\di(f(x))$, we need only compute the divisor of $f(x)$.
As noted earlier, the solutions to $f$ correspond to the ramification points.
So for any $P\notin x^{-1}(\infty)$ then we have $\ord_P(y) =  \frac{1}{2}\ord_P(f(x)) = 1$ if $P$ is a ramification point, and $\ord_P(y) = \frac{1}{2}\ord_P(f(x)) = 0$ otherwise.
If infinity is not a branch point and $P\in x^{-1}(\infty)$ then $\ord_P(y)  = \frac{1}{2}\ord_P(f(x)) = - \frac{1}{2} (2g+2) = -(g+1)$.
So in this case we have
\[
 \di(y) = \sum_{i=1}^{2g+2} [P_i]- (g+1)D_\infty = R - (g+1)D_{\infty}.
 \]
On the other hand, if infinity is a branch point and $P\in x^{-1}(\infty)$, then $\ord_P(y)  = \frac{1}{2}\ord_P(f(x)) = - (2g+1)$.
Then in this case we have 
\[
 \di(y) = \sum_{i=2}^{2g+2} [P_i] - (2g+1)[P_1] = R - (2g+2)[P_1] = R-(g+1)D_\infty.
 \]
\begin{comment}
Firstly, since $y(p_i)= y(j(p_i)) = j^*(y)(p_i)= -y(p_i)$ for all $i$ we see that $y(p_i)=0$.
Therefore $\deg(y)_0 \geq \deg\left(\sum_{i = 1}^{2g +2} [p_i] \right) = 2g+2$.
But since $y\in \cL(D)$, we know that $\deg(y)_{\infty} \leq \deg((g+1)D) = 2g + 2$, and as $\deg(y) = \deg(y)_0 - \deg(y)_{\infty} = 0$ then $\deg(y)_0 = \deg(y)_{\infty} = 2g+2$.
So
\[
 (y)_0 = \sum_{i=1}^{2g+2} [p_i], \ (y)_{\infty} = (g+1)D_\infty.
\]
\end{comment}

We now show that the differentials listed in Proposition \ref{propneq2} are holomorphic.
We have that
\begin{eqnarray*}
 \di(x^i\omega) & = & \di \left( \frac{x^idx^{\otimes m}}{y^m} \right)\\ & = & i(D_0 -D_\infty) + m(R-2D_\infty) -m(R-(g+1)D_\infty) \\
 & = & iD_0 + (mg -m -i)D_\infty \\
 & = & iD_0 + (m(g-1) -i)D_\infty,
\end{eqnarray*}
which is positive for $0\leq i \leq m(g-1)$.
Hence all the differentials and poly-differentials in the first two cases, and also the first $m(g-1)+1$ differentials in the third case, are holomorphic.

\begin{comment}
If $m=g=2$ then we have that
\begin{eqnarray*}
 \di\left(\frac{x^idx^{\otimes 2}}{y^2}\right) & = & i(D_0-D_\infty) + 2(R-2D_\infty) - 2(R-(g+1)D_\infty) \\
 & = & iD_0+ ((2g-2)-i)D_\infty \\
 & = & iD_0 +(2-i)D_\infty,
\end{eqnarray*}
which is positive for $0\leq 2$.
By Lemma \ref{dim3} then $\dim_kH^0(X,\Omega_X^{\otimes 2})=3$, so again, we have three linearly independent elements, so they must form a basis.
\end{comment}


On the other hand we have
\begin{eqnarray*}
 \di (x^iy\omega) & = & \di(x^i\omega) + R -(g+1)D_\infty \\
 & = & iD_0 + R +((m-1)(g-1)-2-i)D_\infty,
\end{eqnarray*}
which is holomorphic only for $0\leq i \leq (m-1)(g-1)-2$.
Hence the last $(m-1)(g-1) -1$ poly-differentials in the third case are holomorphic.
This completes the proof.
\end{proof}

We denote by $\sigma$ the automorphism of $X$ of order 2, which maps each point to its corresponding point in the pre-image of $x$ (note that this map fixes the ramification points).
Since $\sigma(y) = -y$, and $\sigma$ acts trivially on $x$, we see from the previous proposition that when $m=g=2$ the action will be trivial, since the only $y$ term in the basis in this case is $y^2$ and of course $y^2 = (-y)^2$.
If either $g>2$ or $m>2$ then we have an odd power of $y$ in the basis, and hence the action of $\sigma$ is not trivial.
This is a new proof of Theorem (??) for hyperelliptic curves over a field with characteristic not equal to 2.




\subsection{Hyperelliptic curves, characteristic 2}
Let $X$ be a smooth, projective, connected hyperelliptic algebraic curve of genus $g$ over an algebraically closed field $k$ of characteristic two.
Since $X$ is hyperelliptic there is, by definition, a degree two holomorphic map $x:X\rightarrow \mathbb P_k^1$.
Let $k(x)$ be the function field of $\mathbb P_k^1$.
Then by \cite[\S 7.4.3]{liu} there are polynomials $h(x),f(x) \in k[x]$, with maximum degrees $g+1$ and $2g+2$ respectively, such that if we define $y$ by
\begin{equation}\label{ext}
	y^2 - yh(x) = f(x)
\end{equation}
then $k(x,y)$, the degree two extension of $k(x)$ by $y$, is isomorphic to the function field of $X$.
\vskip1em



\begin{lem}\label{smoothness}
The polynomials $h(x)$ and $h'(x)^2 f(x) + f'(x)$ in $k[x]$ have no common zeroes on $X$.
\end{lem}
\begin{proof}
 The Jacobian criterion (see, for example, \cite[Thm. 2.19]{liu}), states that the partial derivatives of \eqref{ext} with respect to $x$ and with respect to $y$ are non-zero at all points $P\in X$ if and only if the curve is smooth.
 Clearly \[
          \frac{\partial}{\partial y} (y^2 -h(x)y -f(x)) = h(x)
         \]
 since the characteristic of $k$ is 2.
 On the other hand
 \begin{equation*}\label{xderivative}
  \frac{\partial}{\partial x} (y^2 - h(x)y -f(x)) = h'(x)y - f'(x).
 \end{equation*}
Now it is obvious that $h'(x)y - f'(x)$ is zero at a point $P\in X$ if and only if
\[
 (h'(x)y-f'(x))^2 = h'(x)^2y^2 -f'(x)^2 = h'(x)^2h(x)y + h'(x)^2f(x) - f'(x)^2
 \]
is also zero at $P$.

 Then it follows from the Jacobian criterion and the smoothness of $X$ that $h(x)$ and $h'(x)^2f(x) + f'(x)$ can have no common zeroes on $X$. 
\end{proof}


We now introduce some notation.
Let $a\in \mathbb P_k^1$.
If $a$ is a branch point then we denote the point in the pre-image of $a$ under $x$ by $P_a$, otherwise we denote the two points in the pre-image by $P_a'$ and $P_a''$.
To simplify future calculations we also define $D_a := x^*([a])$.
We let $d$ be the degree of $h(x)$, and denote the number of distinct roots of $f(x)$ and $h(x)$ in $\mathbb A_k^1$ by $l$ and $k$ respectively.



We now describe the ramified points of $x$.
It follows from Lemma \ref{smoothness} that $X'$, the affine curve defined by \eqref{ext}, is smooth.
Then $x$ restricts to a map $X'\rightarrow \mathbb A^1_k$, the projection on to the $x$ co-ordinate.
For any $a\in \mathbb A_k^1$, if $(a,b)$ is in $x^{-1}(a)$ then so is $(a,b+h(a))$.
Since it is clear that $(a,b)$ and $(a,b+h(a))$ are distinct if and only if $h(a)\neq 0$, and
since $x$ is degree two, we see that the ramified points in $X'$ correspond to the roots of $h(x)$.
We denote the zeroes of $h(x)$ by $a_i$ for $1\leq i \leq k$.
For each $a_i$ there is a corresponding $\alpha_i$, which is the square root of $f(a_i)$.
We will also denote the corresponding ramification points by $P_i := P_{a_i}$ (note that if $P_i \neq \infty$ then $P_i = (a_i,\sqrt {f(a_i)})$ as an affine point) and the divisors by $D_i:=D_{a_i}$.


By \cite[\S 7.4.3]{liu}, the degree of $h(x)$ is less than $g+1$ if and only if $\infty \in \mathbb P_k^1$ is a branch point.
In the case when infinity is a branch point the degree of $f(x)$ must be precisely $2g+1$.
On the other hand, if infinity is not a branch point, the degree of $f(x)$ can be anything from 1 to $2g+2$.

Recall that as $X$ is a hyperelliptic curve there is an automorphism of $X$ of order two, which we will denote $\sigma$.
If $P\in X$ is a branch point then $\sigma$ will fix $P$.
If $P$ is not a branch point then $\sigma$ permutes $P$ with the unique distinct point in $X$ with the same image as $P$ under $x$.

We will now compute the ramification divisor of $x$, which we will denote $R$.
\vskip1em


\begin{lem}\label{char2ramification}
 Let $n_i$ be the order of $h(x)$ at $a_i\in \mathbb A_k^1$.
Then the coefficient $\delta_P$ of the ramification divisor $R$ at $P\in X$ is given by
\[
 \delta_P = \left\{
 \begin{array}{ll}
 2n_i & {\rm if }\ P=P_i\ {\rm for\ some }\ i \in \{1,\ldots ,k\}, \\
 2(g+1-d) & {\rm if }\ \infty \  {\rm is\ a\ branch\ point\ and\ }\  P=P_\infty, \\
 0 & {\rm otherwise.} 
 \end{array}
\right.
\]
\end{lem}
\begin{proof}

 Let $P=P_i$ for some $i\in \{1,\ldots , k\}$.
 Then $y-b_i$ is a local parameter at $P$.
 To see this, note that the maximal ideal $\mathfrak m_{P}$ of the local ring $\cO_{X,P}$ at $P$ is generated by $x-a_i$ and $y-b_i$.
 But $x-a_i\in \mathfrak m_{P}^2$ since $x$ is ramified at $P$ with ramification index 2.
 By Nakayama's lemma \cite[Prop. 2.6]{atiyahmacdonald}, $y-b_i$ is therefore a local parameter at $P$.
 
 Using Hilbert's formula (Theorem (??)) we obtain
 \begin{eqnarray*}
  \delta_P & = & \sum_{i\geq 0} \left(\ord(G_i(P))-1\right) \\
  & = & {\rm max\ }\left\{ i\in \NN | G_i(P)\neq \{1\}\right\} + 1 \\
  & = & \ord_{P}(\sigma(y-b_i) - (y-b_i)).
  \end{eqnarray*}
  
It is clear from \eqref{ext} that $\sigma(y)=y+h(x)$.
 It then follows that
  \begin{eqnarray*}
  \delta_P & = & \ord_{P}(\sigma(y-b_i) - (y-b_i)) \\
  & = & \ord_{P}(y-b_i+h(x) - y + b_i) \\
  & = & 2\ord_{a_i}(h(x)) \\
  & = & 2n_i.
 \end{eqnarray*}
 
 Obviously if $P \neq P_i$ for $1\leq i \leq k$, then $\delta_P \neq 0$ if and only if $P \in x^{-1}(\infty)$.
Recall that by the Riemann-Hurwitz formula (Corollary (??)) $\deg(R) = 2g+2$.
If infinity is not a branch point then $\deg\left( \sum_{i=1}^k 2n_i[P_i]\right) = 2g+2$, since $\deg(h(x)) = g+1$. 
 If infinity is a branch point then the coefficient at $P_\infty$ is $\deg(R) - \sum_{i=1}^k2n_i$, which is equal to $2g+2-2d = 2(g+1-d)$, as stated.
 
\end{proof}


We will now compute the divisors associated to $h(x)$, $x$ and $y$ in $K(X)$, and also to $dx$.
\vskip1em

\begin{lem}\label{xchar2}
 The divisor of $x\in K(X)$ is 
 \[
  \di (x)= D_0 - D_\infty.
\]
\end{lem}
\begin{proof}
 Given our notation above, this is clear.
\end{proof}


\begin{lem}\label{dxchar2}
 The divisor associated to the differential $dx$ is 
 \[
  \di (dx) =\sum_{i=1}^kn_iD_i + (g-1-d)D_\infty.
 \]
\end{lem}
\begin{proof}
We will need to consider the divisor of $dx$ both as a differential on $X$ and on $\mathbb P_k^1$; 
we will use the notation of $\di_X(dx)$ and $\di_{\mathbb P^1}(dx)$ to differentiate between the two cases.
Then the Riemann-Hurwitz formula (Theorem \ref{detailedhurwitz}) states that
\[
 \di_X( dx) = x^*\di_{\mathbb P^1}(dx) + R.
\]
Now $x^*\di_{\mathbb P^1}(dx) = -2D_\infty$, 
and hence we have
\[
 \di_X( dx) = \sum_{i=1}^k 2n_iP_i + (g+1-d)D_\infty - 2D_\infty = \sum_{i=1}^k n_iD_i + (g-1-d)D_\infty.
\]
 \end{proof}

 \vskip1em

 
 \begin{lem}\label{h(x)char2}
  The divisor associated to $\frac{1}{h(x)}$ is
  \[
  \di \left(\frac{1}{h(x)}\right) = dD_\infty - \sum_{i=1}^k n_iD_i.
 \]
 \end{lem}
\begin{proof}
If $x$ is ramified at infinity we have $\ord_{P_{\infty}}\left(\frac{1}{h(x)}\right) = -\ord_{P_{\infty}}(h(x)) = 2d$.
If $x$ is not ramified at infinity we have $\ord_{P_{\infty}'}\left(\frac{1}{h(x)}\right) = \ord_{P_{\infty}''}\left(\frac{1}{h(x)}\right)=d$.
For $1\leq i \leq k$ we have $\ord_{P_i}\left(\frac{1}{h(x)}\right) = -\ord_{P_i}(h(x))= -2n_i$.
At any other point of $X$ the order of $\frac{1}{h(x)}$ is clearly zero.
\end{proof}

We again consider the affine curve $X'$, and we let $b_j \in \mathbb A_k^1$, for $1\leq j\leq l \leq 2g+2$, be the zeroes of $f(x)$.
If $b_j$ corresponds to a branch point then we denote the unique point in the pre-image of $b_j$ uner $x$ by $Q_j$.
Otherwise there are two points in the pre-image, and we denote these \[ Q_j:=(b_j,0)\ {\rm and}\ Q_j':=(b_j,h(b_j)).\]


~


\begin{lem}\label{ychar2}
 Let $m_j$ be the order of $f(x)$ at $b_j\in \mathbb A_k^1$.
 Then if infinity is a branch point we have
 \[
  \di(y) = \sum_{j=1}^l m_j[Q_j] - (2g+1)[P_\infty].
 \]

 Otherwise we have either
 \begin{equation*}
  \di(y) = \sum_{j=1}^l m_j[Q_j] + (g + 1-\deg(f(x)))[P_\infty'] -(g+1)[P_\infty'']
 \end{equation*}
 or
  \begin{equation*}
  \di(y) = \sum_{j=1}^l m_j[Q_j] + (g + 1-\deg(f(x)))[P_\infty''] -(g+1)[P_\infty'].
 \end{equation*}
\end{lem}


\begin{proof}
 First note that if $P = (a,b) \in X'$ is such that $f(a) \neq 0$ (i.e. $P\neq Q_j$ for any $j\in \{ 1,\ldots, l\}$), then it is clear that $y(P) \neq 0$.
 Indeed, if $y$ were zero at $P$ then we would have $f(a) = y(P)^2 + h(a)y(P) = 0$, a contradiction.
 
 If $x$ is not ramified at $Q_j$ then $h(b_j) \neq 0$, and $y(Q_j) = 0$, hence $y+h(x)$ is a unit at $Q_j$.
 Since $y(y+h(x)) = f(x)$, it follows that
 \begin{equation}\label{zeroesofy}
  \ord_{Q_j}(y) = \ord_{Q_j}\left(\frac{f(x)}{y+h(x)}\right) = \ord_{Q_j}(f(x)) -\ord_{Q_j}(y+h(x)) = m_j.
 \end{equation}

 On the other hand, when $Q_j$ is a ramification point we must have $m_j=1$.
 Otherwise we would have $f'(b_j) = 0$.
Since we also have $h(b_j) = f(b_j) = 0$, this would contradict Lemma \ref{smoothness}.
 So if we let $\tilde f(x) := \frac{f(x)}{x-b_j}$ and $\tilde h(x) := \frac{h(x)}{x-b_j}$ then $\tilde f(x)$ is a unit at $Q_j$.
We have \[y^2 = f(x) - yh(x) = (x-b_j)(\tilde f(x) - \tilde h(x)y),\]
 and hence
 \begin{equation*}
   \ord_{Q_i}(y^2) = \ord_{Q_i}(x-b_j) + \ord_{Q_i}(\tilde f(x) - \tilde h(x)y).
  \end{equation*}
Note that the function $\tilde f(x) - \tilde h(x)y$ does not have an affine pole and so $y$ has a zero at $Q_j$. So $h(x)y$ has a zero at $Q_j$, and also $\tilde f(x)$ is a unit at $Q_j$. Combining this we see that $\ord_{Q_j}(\tilde f(x)-\tilde h(x)y) = 0$.
Hence $\ord_{Q_i}(y) = \frac{1}{2}\ord_{Q_i}(x-b_j) = 1$ and we have determined the zero divisor of $y$. 


If infinity is a branch point there is only one point of $X$ at which $y$ can have a pole, namely $P_\infty$.
Since the degree of $\di (y)$ is zero, then the pole divisor of $y$ must be
$
 (2g+1)[P_\infty],
$
and this completes the proof when infinity is a branch point.

We now suppose that infinity is not a branch point; recall that this implies that the degree of $h(x)$ is $g+1$.
We consider \eqref{zeroesofy} according to whether $\ord_{P_\infty'}(y)$ is greater than, less than or equal to $-(g+1)$.
First note that we cannot have $\ord_{P_\infty'}(y) < -(g+1)$, else $\ord_{P_\infty'}(y+h(x)) = \ord_{P_\infty'}(y)< -(g+1)$, and this would then imply \[ \ord_{P_\infty'}(y+h(x)) + \ord_{P_\infty'}(y) < -2(g+1) \leq \ord_{P_\infty'}(f(x)), \] contradicting \eqref{zeroesofy}.
On the other hand, if $\ord_{P_\infty'} (y) = -(g+1)$, then by \eqref{zeroesofy} we see that $\ord_{P_\infty'}(y+h(x)) = -\deg(f(x)) +g+1$.
Lastly, if $\ord_{P_\infty'}(y) > -(g+1)$, then $\ord_{P_\infty'}(y+h(x)) = -(g+1)$, and hence $\ord_{P_\infty'} (y) = -\deg(f(x)) +g+1$ by \eqref{zeroesofy}.

Note that in all cases we have $\ord_{P_\infty'}(y) + \ord_{P_\infty''}(y) = -\deg(f(x))$, and so the degree of $\di (y)$ is zero.
\end{proof}



Recalling that we have the involution $\sigma$ of $X$, we see from the proof of the above lemma that when infinity is not a branch point one of $\ord_{P_\infty '}(y)$ or $\ord_{P_\infty '}(\sigma(y))$ is $g + 1-\deg(f(x))$, and the other is $-(g+1)$.
 Since $\sigma(P_\infty') = P_\infty''$, we will assume for the rest of the report that $\ord_{P_\infty'}(y) = g+1-\deg(f(x))$ if infinity is not a branch (and hence $\ord_{P_\infty''}(y) = -(g+1)$).


We now prove the following proposition, determining a basis of the space of global holomorphic poly-differentials on $X$ of order $m$.
\vskip1em

\begin{prop}
Let $m\geq 1$.
Assume that $g\geq 2$ and let $\omega:= \frac{dx^{\otimes m}}{h(x)^m}$. 
Then a basis of $H^0(X,\Omega_X^{\otimes m})$ is given by

$\begin{cases}
 \omega, x\omega, \ldots , x^{g-1}\omega &  \mbox{if}\ m=1 \\
 \omega, x\omega, x^2\omega & \mbox{if}\ m=g=2 \\
 \omega, x\omega, \ldots, x^{m(g-1)}\omega;\  y\omega, xy\omega, \ldots ,x^{(m-1)(g-1)-2}y\omega & \mbox{otherwise.}
 \end{cases}
$
\end{prop}

\begin{rem}
 Note that the case where $m=1$ can be found in \cite[\S 7.4.3]{liu}.
\end{rem}

\begin{proof}
\begin{comment}
 We first assume that the above elements are holomorphic poly-differentials, and show that they then form a basis.
To show that the elements are linearly independent over $k$ we need only show that the coefficients are, since $\omega$ is fixed.
The only case where this is not clear is when the coefficients contain both $x$ and $y$ terms.
But since the $y$ terms are all linear, and the extension is of degree two, it must follow that coefficients are linearly independent.
 
 
 In the case that $m=1$ then we have that $\dim_k H^0(X,\Omega_X) =g$ by Lemma \ref{dim3}, and there are $g$ elements described in the statement of the proposition in this case, so they must form a basis.
 If $m \geq 2$ then $\dim_k H^0(X,\Omega_X^{\otimes m}) = (2m-1)(g-1)$.
 If $m=g=2$ then $(2m-1)(g-1) = 3$, and there are three elements listed in the proposition.
 On the other hand if $m\geq 2$ and $g > 2$ the proposition lists
 \[
  m(g-1)+1 + (g-1)(m-1)-2+1 = 2mg -2m -g +1 = (2m-1)(g-1)
 \]
 elements, and again they must form a basis.
 \end{comment}
 
 The proof of the fact that the number of elements listed is the same as the dimension of the $H^0(X,\Omega_X^{\otimes m})$ is the same as that in Proposition \ref{propneq2}, as is the proof of the linear independence of the listed elements.
 
 So we need to show that the listed poly-differentials are holomorphic, i.e. that their divisors are non-negative.
 Recall that in the last subsection we saw that $\di (dx^{\otimes m}) = m\di (dx)$.
 Then we have
 \begin{eqnarray*}\label{nonydifferentials}
  \di(x^i\omega) & = & \di \left( \frac{x^i dx^{\otimes m}}{h(x)^m} \right) \nonumber \\ \nonumber & = & i(D_0 - D_\infty) +2m\sum_{i=1}^k n_i[P_i] + m(g-1-d)D_\infty \nonumber \\
  & & -2m\sum_{i=1}^k n_i [P_i] + dmD_\infty \nonumber \\
  & = & iD_0 + (m(g-1) -i)D_\infty
 \end{eqnarray*}
  by Lemmas \ref{xchar2}, \ref{dxchar2} and \ref{h(x)char2}, and this is clearly non-negative if and only if $0\leq i \leq m(g-1)$.
  
  Similarly the divisor 
  \begin{eqnarray*}
   \di(x^iy\omega) & = & \di \left( \frac{x^i ydx^{\otimes m}}{h(x)^m} \right)\nonumber \\ & = & iD_0 + (m(g-1) -i)D_\infty + \sum_{i=1}^l m_i[Q_i] - (2g+1)[P_\infty] \nonumber \\
    & = & iD_0 +  \sum_{i=1}^l m_i[Q_i] + (2m(g-1) -(2g+1) -2i)[P_\infty] \nonumber  \\
   & = & iD_0 +  \sum_{i=1}^l m_i[Q_i] + (2((m-1)(g-1) -1 -i)-1)[P_\infty]
  \end{eqnarray*}
 is clearly non-negative if and only if $0 \leq i \leq (g-1)(m-1)-2$.

 
 Finally, we consider the case where infinity is not a branch point.
 Then 
 \begin{eqnarray}\label{unramifiedydifferential}
  \di(y\omega x^i) & = & iD_0 -iD_\infty + (m(g-1)-i)D_\infty \\ & & 
  +\sum_{i=1}^l m_i[Q_i] +(g+1-\deg(f(x)))[P_\infty'] -(g+1)[P_\infty''] \\
 & = & iD_0 + \sum_{i=1}^l m_i[Q_i] + ((m+1)(g-1) + 2 - \deg(f(x)) -i)[P_\infty']\\
 & & +((m-1)(g-1)-2-i)[P_\infty''].
 \end{eqnarray}
 
For this to be holomorphic we require three conditions to hold: that $i\geq 0$, $i\leq (m+1)(g-1)-2$ and that $i \leq (m-1)(g+1) - \deg(f(x)) + 2$.
Recalling that $1 \leq \deg(f(x)) \leq 2g+2$, we see that 
\begin{eqnarray*}
 (m+1)(g-1)+2 - \deg(f(x)) & \geq & (m+1)(g-1) + 2 - 2g-2 \\
 & = & mg + g - m - 1 - 2g \\
 & = & mg - g -m -1 \\
 &= & (m-1)(g-1) - 2.
\end{eqnarray*}
Hence the differentials in \eqref{unramifiedydifferential} are holomorphic if $0 \leq i \leq (m-1)(g-1)-2$.
This completes the proof.
\end{proof}

  Since $\sigma$ acts trivially on $x$, it is clear that in both the $m=1$ case and the case where $m=g=2$ that the group action is trivial.
  On other hand, since $\sigma(y) = y+h(x)$, we can see that in the other cases the action is not trivial, as clearly
  \[
    x^iy\omega \neq x^i(y+h(x))\omega = \sigma(x^iy)\omega.
  \]

  Hence this is an alternative proof of Theorem (??) in the case of hyperelliptic curves over a field of characteristic two.


\bibliography{/home/jtait/files/Documents/Maths/Bibliography/biblio.bib}
%\bibliography{/home/joe/files/Documents/Maths/Bibliography/biblio.bib}
\bibliographystyle{plain}


\end{document}
