% !TEX TS-program = pdflatex
% !TEX encoding = UTF-8 Unicode

% This is a simple template for a LaTeX document using the "article" class.
% See "book", "report", "letter" for other types of document.

\documentclass[draft, 11pt]{article} % use larger type; default would be 10pt

\usepackage[utf8]{inputenc} % set input encoding (not needed with XeLaTeX)

%%% Examples of Article customizations
% These packages are optional, depending whether you want the features they provide.
% See the LaTeX Companion or other references for full information.

%%% PAGE DIMENSIONS
\usepackage{geometry} % to change the page dimensions
\geometry{a4paper} % or letterpaper (US) or a5paper or....
% \geometry{margins=2in} % for example, change the margins to 2 inches all round
% \geometry{landscape} % set up the page for landscape
%   read geometry.pdf for detailed page layout information

\usepackage{graphicx} % support the \includegraphics command and options

\usepackage[parfill]{parskip} % Activate to begin paragraphs with an empty line rather than an indent

%%% PACKAGES
\usepackage{mathtools}
\usepackage{booktabs} % for much better looking tables
\usepackage{array} % for better arrays (eg matrices) in maths
\usepackage{paralist} % very flexible & customisable lists (eg. enumerate/itemize, etc.)
\usepackage{verbatim} % adds environment for commenting out blocks of text & for better verbatim
\usepackage{subfig} % make it possible to include more than one captioned figure/table in a single float
% These packages are all incorporated in the memoir class to one degree or another...

%%% HEADERS & FOOTERS
\usepackage{fancyhdr} % This should be set AFTER setting up the page geometry
\pagestyle{fancy} % options: empty , plain , fancy
\renewcommand{\headrulewidth}{0pt} % customise the layout...
\lhead{}\chead{}\rhead{}
\lfoot{}\cfoot{\thepage}\rfoot{}

%%% SECTION TITLE APPEARANCE
\usepackage{sectsty}
\allsectionsfont{\sffamily\mdseries\upshape} % (See the fntguide.pdf for font help)
\usepackage{amsmath}
\usepackage{amsthm}
\usepackage{amsfonts}
\usepackage{mathrsfs}
\usepackage{amsopn}
\usepackage{amssymb}
\usepackage{natbib}
% (This matches ConTeXt defaults)

%%% ToC (table of contents) APPEARANCE
\usepackage[nottoc,notlof,notlot]{tocbibind} % Put the bibliography in the ToC
\usepackage[titles,subfigure]{tocloft} % Alter the style of the Table of Contents
\renewcommand{\cftsecfont}{\rmfamily\mdseries\upshape}
\renewcommand{\cftsecpagefont}{\rmfamily\mdseries\upshape} % No bold!

%Theorems and stuff
\theoremstyle{plain}
\newtheorem{defn}{Definition}[section]
\newtheorem{thm}[defn]{Theorem}
\newtheorem{cor}[defn]{Corollary}
\newtheorem{lem}[defn]{Lemma}
\newtheorem{prop}[defn]{Proposition}
\newtheorem{ex}[defn]{Example}
\newtheorem*{unnumthm}{Theorem}
\newtheorem{defnlem}[defn]{Definition/Lemma}
\newtheorem{defnthm}[defn]{Theorem/Definition}
\theoremstyle{remark}
\newtheorem*{rem}{Remark}


\newcommand{\cO}{{\cal O}}
\newcommand{\ra}{\rightarrow}
\newcommand{\NN}{{\mathbb N}}
\newcommand{\PP}{{\mathbb P}}
\newcommand{\ZZ}{{\mathbb Z}}
\newcommand{\cL}{{\mathcal L}}
\newcommand{\cA}{{\mathcal A}}
\newcommand{\cD}{{\mathcal D}}


\DeclareMathOperator{\aut}{Aut}
\DeclareMathOperator{\ord}{ord}
\DeclareMathOperator{\di}{div}
\DeclareMathOperator{\cha}{char}
\DeclareMathOperator{\gal}{Gal}
\DeclareMathOperator{\Tr}{Tr}

%%% END Article customizations

%%% The "real" document content comes below...

\title{}
\author{Joe Tait}
%\date{} % Activate to display a given date or no date (if empty),
         % otherwise the current date is printed 

\begin{document}
\maketitle
\thispagestyle{empty}
\newpage
\subsection{Riemann Surfaces - basic definitions}

\begin{defn} A Riemann surface is a second countable, connected, Hausdorff topological space $X$ with an atlas of pairwise compatible charts, such that if $(\phi_!,U_1)$ and $(\phi_2,U_2)$ are charts then $\phi_1\circ \phi_2:\mathbb C \rightarrow \mathbb C$ is holomorphic.
\end{defn}

Note that every Riemann surface is a real surface (2 dimensional manifold), and hence we have a well defined genus for all compact Riemann surfaces.

Some initial examples of Riemann surfaces are:
\begin{list}
 \item The (Riemann) sphere. We have the natural projection from the south and north poles. If we identify the south pole with $0\in \mathbb C$ then we can consider the north pole to be infinity.

\item The projective line, $\mathbb{CP}^1$. We can cover this with 
\[
	U_0 = \{[1:w]|w\in \mathbb C\}
\]
and
\[
	U_1 = \{[z:1]|z\in \mathbb C\}.
\]

Here we map $[1:w]$ to $w$, and $[z:1]$ to $z$.

\item The torus can be viewed as the quotient of $\mathb C$ by a lattice $L=\mathbb Z + \lambda \mathbb Z$, where $\lambda$ is not real.
Then we have a natural projection $\pi : \mathbb C \rightarrow T = \mathbb C/L$, and using this we can obtain chart maps.
\end{list}

We now relate Riemann surfaces and complex curves; this relation allows us to use algebraic techniques essential to the proof of Riemann-Roch.

\begin{thm} 
Let $f(z,w)$ be a non-singular, irreducible polynomial .
Then it's locus of roots (i.e. the set of points in $\mathbb C^2$ for which $f$ is zero) is a Riemann surface.
\end{thm}

By non-singular we mean there is no point for which $f$ and both of its first derivatives are zero.

The crux of proving this comes from the implicit function theorem.

\begin{thm}[Implicit function theorem]
Let $f(z,w)$ be a non-singular, irreducible polynomiali.
Let $X$ be its zero set.
Suppose $p\in X$ is such that $\frac{\partial f}{\partial w}(p) \neq 0$.
Then there is some holomorphic $g(z)$, where $z$ is a local co-ordinate for $p$ around $z_0$, such that near $z_0$ we have $X=\{(z,w)\in \mathbb C^2|g(z)-w = 0\}.$
\end{thm}

We then get our chart maps on the curve by writing $X$ lcoally as the zero set of the curve, and projecting the co-ordinate which $g$ is a variable in. Obviously one has to check that at points where neither derivative is zero that the chart maps are compatible.
\bibliography{/home/jtait/files/Documents/Maths/Bibliography/biblio.bib}
%\bibliography{/home/joe/files/Documents/Maths/Bibliography/biblio.bib}
\bibliographystyle{plain}


\end{document}
