\documentclass[12pt,leqno]{article}

\usepackage{amsfonts}
\usepackage{amssymb}


\oddsidemargin=0.3cm \evensidemargin=0.3cm \topmargin=-1cm
\textwidth=15.3cm \textheight=23cm
\parskip=5pt
\parindent=0pt




\newcommand{\ra}{\rightarrow}
\newcommand{\minusquad}{\!\!\!\!\!\!}
\newcommand{\scrE}{\mathcal{E}}
\newcommand{\scrF}{\mathcal{F}}
\newcommand{\scrI}{\mathcal{I}}
\newcommand{\scrO}{\mathcal{O}}
\newcommand{\frcm}{\mathfrak{m}}
\newcommand{\frcn}{\mathfrak{n}}
\newcommand{\frca}{\mathfrak{a}}
\newcommand{\biject}{\stackrel{\sim}{\longrightarrow}}
\newcommand{\modd}{\!\!\!\mod}
\newcommand{\Fq}{\mathds{F}_{q}}


%\liff{ref}
\newcommand{\liff}[1]{\stackrel{#1}{\Longleftrightarrow}}

%\lformula {label}{mathcode}
\newcommand{\lformula}[2]{\begin{equation} \label{#1} #2
\end{equation}}

%\formula{mathcode}
\newcommand{\formula}[1]{\begin{equation} #1 \end{equation}}

%\intlabel{label}
%\newcommand{\intlabel}[1]{\label{#1} \textbf{[Label: #1 ]} \newline}
\newcommand{\intlabel}[1]{\label{#1}}


\newcommand{\cA}{{\cal A}}
\newcommand{\cC}{{\cal C}}
\newcommand{\cB}{{\cal B}}
\newcommand{\cD}{{\cal D}}
\newcommand{\cE}{{\cal E}}
\newcommand{\cF}{{\cal F}}
\newcommand{\cG}{{\cal G}}
\newcommand{\cH}{{\cal H}}
\newcommand{\cI}{{\cal I}}
\newcommand{\cJ}{{\cal J}}
\newcommand{\cK}{{\cal K}}
\newcommand{\cL}{{\cal L}}
\newcommand{\cM}{{\cal M}}
\newcommand{\cN}{{\cal N}}
\newcommand{\cO}{{\cal O}}
\newcommand{\cP}{{\cal P}}
\newcommand{\cQ}{{\cal Q}}
\newcommand{\cR}{{\cal R}}
\newcommand{\cV}{{\cal V}}


\newcommand{\ZZ}{{\mathbb Z}}
\newcommand{\CC}{{\mathbb C}}
\newcommand{\FF}{{\mathbb F}}
\newcommand{\RR}{{\mathbb R}}
\newcommand{\NN}{{\mathbb N}}
\renewcommand{\AA}{{\mathbb A}}
\newcommand{\PP}{{\mathbb P}}
\newcommand{\QQ}{{\mathbb Q}}
\newcommand{\HH}{{\mathbb H}}




\begin{document}
\begin{center} {\bf\Large Faithful Action on the Space of Global\\
\vspace*{0.2cm} Differentials of an Algebraic Curve} \end{center}
\begin{center}\sc Bernhard K\"ock \end{center}

\bigskip

\begin{quote}
\footnotesize {\bf Abstract.} Given a faithful action of a
finite group $G$ on an algebraic curve of genus at least 2, we
prove that the induced action on the space of global
holomorphic differentials is faithful as well unless the
characteristic of the base field is  2 and $G$ contains a
hyperelliptic involution.


{\bf Mathematics Subject Classification 2000.} 14H30; 14F10;
11R32.

\end{quote}

\bigskip

Let $X$ be a connected smooth projective algebraic curve over
an algebraically closed field $k$ equipped with a faithful
action of a finite group $G$ of order $n$. Then $G$ also acts
on the vector space $H^0(X, \Omega_X)$ of global holomorphic
differentials on $X$. A widely studied problem is to determine
the structure of $H^0(X,\Omega_X)$ as module over the group
ring $k[G]$. It goes back to the Chevalley-Weil when $k = \CC$,
see \cite{CW}. If the canonical projection $\pi: X \rightarrow
Y$ from $X$ to the quotient curve $Y = X/G$ is tamely ramified,
a fairly explicit answer to this problem has been given in 1986
by Kani in \cite{Ka}. For more recent answers to (related)
questions in more general situations the reader is referred to
the papers \cite{Bo} and \cite{FWK}. In the case of arbitrary
wild ramification the explicit calculation of the
$k[G]$-isomorphism class of $H^0(X,\Omega_X)$ is still an open
problem.

This note is concerned with the weaker question whether $G$
acts faithfully on the space~$H^0(X,\Omega_X)$. We give the
following answer to this question. Let $g_X$ and $g_Y$ denote
the genus of $X$ and $Y$, respectively, and let $p$ denote the
characteristic of $k$. We recall that a hyperelliptic
involution of $X$ is an automorphism $\sigma$ of $X$ of order
$2$ such that the quotient curve $X/\langle \sigma \rangle$ is
isomorphic to $\PP^1_k$.

{\bf Theorem.} {\em We assume that $g_X \ge 2$. Then $G$ acts
faithfully on $H^0(X,\Omega_X)$ if and only if $p \not=2$ or if
$G$ does not contain any hyperelliptic involution.}

The proof of this theorem will be given after the proof of
Proposition~1 below.


{\bf Corollary.}  {\em Let $g_X \ge 2$. If $G$ does not act
faithfully on $H^0(X,\Omega_X)$ then $p=2$, $g_Y =0$ and the
projection $\pi$ is not tamely ramified.}

{\em Proof.} By the previous theorem we have $p=2$ and there
exists a hyperelliptic involution $\sigma \in G$. Then the
Hurwitz formula (see Corollary~2.4 on p.~301 in \cite{Ha})
applied to the projection $X\rightarrow X/\langle \sigma
\rangle \cong \PP^1_k$ shows that $X\rightarrow X/\langle
\sigma \rangle$ is not unramified and hence not tamely
ramified; then $\pi$ is not tamely ramified either. And the
Hurwitz formula applied to the projection $\PP^1_k\cong
X/\langle \sigma \rangle \ra  Y$ shows that the genus of $Y$ is
$0$ as well.

The following example describes some (mostly trivial) cases
when the action of $G$ on $H^0(X, \Omega_X)$ is in fact
trivial.



{\bf Example.}\\
(a) If $g_X = 0$ then $G$
obviously acts trivially on $H^0(X,\Omega_X) = \{0\}$.\\
(b) If $g_X =1 $ (that is if $X$ is an elliptic curve) and if
$G$ is a finite subgroup of $X(k)$ acting on $X$ by
translations then $G$ leaves invariant any global non-vanishing
holomorphic differential and hence $G$ acts trivially on
$H^0(X,\Omega_X)$.\\
(c) Let $p=2$. If $n=2$ and $g_Y =0$, then $G \cong \ZZ/2\ZZ$
acts trivially on $H^0(X, \Omega_X)$ by Proposition~2 below.
For instance, let $r$ be an odd natural number, let $k(x,y)$ be
the extension of the rational function field $k(x)$ given by
the Artin-Schreier equation $y^2-y = x^r$ and let $\pi: X
\rightarrow \PP^1_k$ be the corresponding cover of nonsingular
curves over $k$; then $G$ acts trivially on the vector space
$H^0(X,\Omega_X)$ whose dimension is $\frac{r-1}{2}$ by
Example~2.5 on p.~1095 in \cite{Ko}.


The next lemma is crucial for the proof of Proposition~1 which
in turn is the main idea for the proof of our theorem. We begin
by introducing some notations. For any $G$-invariant divisor
$D$ on $X$ let $\cO_X(D)$ denote the corresponding equivariant
invertible $\cO_X$-module, as usual. Furthermore let
$\pi_*^G(\cO_X(D))$ denote the subsheaf of the direct image
$\pi_*(\cO_X(D))$ fixed by the obvious action of $G$ on
$\pi_*(\cO_X(D))$ and let $\left\lfloor \frac{\pi_*(D)}{n}
\right \rfloor$ denote the divisor on $Y$ obtained from the
push-forward $\pi_*(D)$ by replacing the coefficient $m_Q$ of
$Q$ in $\pi_*(D)$ with the integral part $\left \lfloor
\frac{m_Q}{n} \right \rfloor$ of $\frac{m_Q}{n}$ for every $Q
\in Y$. The function fields of $X$ and $Y$ are denoted by
$K(X)$ and $K(Y)$, respectively. Finally, for any $P \in X$,
let $e_P$ denote the ramification index of $\pi$ at $P$ and let
$\textrm{ord}_P$ and $\textrm{ord}_Q$ denote the respective
valuations of $K(X)$ and $K(Y)$ at $P$ and $Q:=\pi(P)$.

{\bf Lemma.} {\em Let $D$ be a $G$-invariant divisor on $X$.
Then the sheaves $\pi_*^G(\cO_X(D))$ and
$\cO_Y\left(\left\lfloor \frac{\pi_*(D)}{n}\right
\rfloor\right)$ are equal as subsheaves of the constant sheaf
$K(Y)$ on $Y$. In particular the sheaf $\pi_*^G(\cO_X(D))$ is
an invertible $\cO_Y$-module. }

For the reader's convenience we include a proof of this lemma
although it may already exist in the literature.

{\em Proof.} For every open subset $V$ of $Y$ we have
\[\pi_*^G(\cO_X(D))(V) = \cO_X(D) (\pi^{-1}(V))^G \subseteq K(X)^G = K(Y).\]
In particular both sheaves are subsheaves of the constant sheaf
$K(Y)$ as stated. It therefore suffices to check that their
stalks are equal. Let $Q \in Y$, let $P \in \pi^{-1}(Q)$ and
let $n_P$ denote the coefficient of $D$ at $P$. Then we have
\begin{eqnarray*}
\lefteqn{\pi_*^G\left(\cO_X(D)\right)_Q = \cO_X(D)_P \cap
K(Y)}\\
&=& \left\{f \in K(Y): \textrm{ord}_P(f) \ge -n_P\right\}\\
&=& \left\{f \in K(Y): \textrm{ord}_Q(f) \ge - \frac{n_P}{e_P}\right\}\\
&=& \left\{ f \in K(Y): \textrm{ord}_Q(f) \ge - \left\lfloor
\frac{n_P}{e_P} \right\rfloor \right\}\\
&=& \cO_Y\left(\left\lfloor \frac{\pi_*(D)}{n} \right\rfloor
\right)_Q,
\end{eqnarray*}
as desired. \hfill $\Box$


Let $R := \sum_{P\in X} \textrm{dim}_k (\Omega_{X/Y}) [P]$
denote the ramification divisor of $\pi$. The following
proposition computes the dimension of the subspace of
$H^0(X,\Omega_X)$ fixed by $G$.


{\bf Proposition~1.}
\[\dim_k \left(H^0(X,\Omega_X)^G\right) = \left\{
\begin{array}{ll}
g_Y & \textrm{if } \textrm{deg} \left\lfloor \frac{\pi_*(R)}{n} \right\rfloor = 0\\
\\
g_Y-1 + \textrm{deg}\left\lfloor \frac{\pi_*(R)}{n} \right\rfloor &
\textrm{if } \textrm{deg}\left\lfloor \frac{\pi_*(R)}{n} \right\rfloor > 0.
\end{array}\right.\]


{\em Proof.} Let $K_X$ be a $G$-invariant canonical divisor on
$X$, that is we have an equivariant isomorphism $\cO_X(K_X)
\cong \Omega_X$. Let the divisor $K_Y$ on $Y$ be defined by the
equality $\pi^*(\Omega_Y) = \cO_X(\pi^*(K_Y))$ of subsheaves of
$\cO_X(K_X)$. Note that we consider $\pi^*(\Omega_Y)$ as a
subsheaf of $\Omega_X \cong \cO_X(K_X)$ and that we have a
short exact sequence
\[0 \rightarrow \pi^* \Omega_Y \rightarrow \Omega_X \rightarrow
\Omega_{X/Y} \rightarrow 0. \]
 In particular we have $K_X = \pi^* K_Y + R$ and hence
\[\left\lfloor \frac{\pi_*(K_X)}{n} \right \rfloor = \left
\lfloor \frac{\pi_*\pi^*(K_Y) + \pi_*(R)}{n} \right \rfloor =
K_Y + \left \lfloor \frac{\pi_*(R)}{n} \right\rfloor.\]
 Using the previous lemma we conclude that $\pi_*^G(\Omega_X) \cong
\cO_Y\left(K_Y + \left \lfloor
\frac{\pi_*(R)}{n}\right\rfloor\right)$ and finally that
\begin{eqnarray*}
\lefteqn{\textrm{dim}_k \left(H^0(X,\Omega_X)^G \right)}\\
& = & \textrm{dim}_k \left(H^0\left(Y, \pi_*^G(\Omega_X)\right)\right) \\
& = & \textrm{dim}_k
\left(H^0\left(Y, \cO_Y\left(K_Y+ \left\lfloor \frac{\pi_*(R)}{n}\right\rfloor \right) \right) \right).
\end{eqnarray*}
If $\textrm{deg}\left\lfloor \frac{\pi_*(R)}{n} \right \rfloor
= 0$ then $\left \lfloor \frac{\pi_*(R)}{n} \right\rfloor$ is
the zero divisor and we conclude that
\[\textrm{dim}_k\left(H^0(X,\Omega_X)^G\right) =
\textrm{dim}_k\left(H^0(Y, \Omega_Y)\right) = g_Y,\] as desired.
If $\textrm{deg}\left\lfloor \frac{\pi_*(R)}{n} \right \rfloor
> 0$ the divisor $K_Y + \left \lfloor \frac{\pi_*(R)}{n} \right
\rfloor$ is non-special and using the Riemann-Roch theorem (see
Theorem~1.3 on p.~295 and Example~1.3.4 on p.~296 in \cite{Ha})
we obtain
\begin{eqnarray*}
\lefteqn{\dim_k\left(H^0(X, \Omega_X)^G \right)}\\
& = & \textrm{deg}\left(K_Y + \left \lfloor \frac{\pi_*(R)}{n} \right \rfloor \right) + 1 - g_Y \\
& = & g_Y - 1 + \textrm{deg} \left \lfloor \frac{\pi_*(R)}{n} \right\rfloor ,
\end{eqnarray*}
as stated. \hfill $\Box$

{\em Proof of Theorem.} To prove the if-direction we suppose
that $G$ does not act faithfully on $H^0(X, \Omega_X)$. By
replacing $G$ with the (non-trivial) kernel $H$ of the action
of $G$ on $H^0(X, \Omega_X)$ we may assume that $G$ is
non-trivial and that $G$ acts trivially on $H^0(X,\Omega_X)$.\\
We first prove that $\pi$ is not tamely ramified. Suppose that
$\pi$ is tamely ramified. Then we have $R= \sum_{P \in X}
(e_P-1)[P]$ by Proposition 2.2(c) on p.~300 in \cite{Ha}; hence
$\left \lfloor \frac{\pi_*(R)}{n} \right \rfloor$ is the zero
divisor. Therefore we obtain
\[g_X = \textrm{dim}_k\left(H^0(X, \Omega_X)\right) =
\textrm{dim}\left(H^0(X,\Omega_X)^G\right) = g_Y\] by
Proposition~1. Substituting this equality into the Hurwitz
formula
\[2(g_X -1) = 2n (g_Y-1) + \textrm{deg}(R)\]
yields the desired contradiction because $n \ge 2$, $g_X \ge 2$
and $\textrm{deg}(R) \ge 0$. \\
We next prove that $g_Y =0$. By the argument used in the
previous paragraph we know that $\left \lfloor
\frac{\pi_*(R)}{n} \right \rfloor$ is not the zero divisor.
Then Proposition~1 tells us that
\[g_X = g_Y-1 + \textrm{deg} \left \lfloor \frac{\pi_*(R)}{n}
\right \rfloor.\] Substituting this equality into the Hurwitz
formula we obtain
\[2\left(g_Y - 1 + \textrm{deg}\left \lfloor \frac{\pi_*(R)}{n}
\right \rfloor -1 \right) = 2n (g_Y -1) + \textrm{deg}(R).\]
For any $Q \in Y$ let $n_Q$ denote the coefficient of the
ramification divisor $R$ at any $P \in \pi^{-1}(Q)$ and let
$e_Q := e_P$ for any $P \in \pi^{-1}(Q)$. Rewriting the
previous equation yields
\begin{eqnarray*}
\lefteqn{(2n-2)g_Y = 2n-4 + 2 \,\textrm{deg}\left \lfloor
\frac{\pi_*(R)}{n}\right \rfloor - \textrm{deg}(R)}\\
&=& 2 \left(n-2 + \sum_{Q \in Y}
\left(\left\lfloor \frac{n}{e_Q} \frac{n_Q}{n} \right\rfloor - \frac{n}{e_Q} \frac{n_Q}{2}\right) \right)\\
&=& 2 \left(n-2 + \sum_{Q \in Y}
\left( \left\lfloor \frac{n_Q}{e_Q} \right\rfloor - \frac{n_Q}{e_Q} \frac{n}{2} \right)\right)\\
& \le & 2(n-2)
\end{eqnarray*}
because $\frac{n}{2} \ge 1$ and $\left\lfloor \frac{n_Q}{e_Q}
\right\rfloor \le \frac{n_Q}{e_Q}$ for all $Q \in Y$. Hence we
obtain $g_Y \le \frac{n-2}{n-1} < 1$ and therefore $g_Y =0$, as
desired. \\
As $\pi$ is not tamely ramified, the characteristic of $k$ is
positive and we may furthermore replace $G$ by a cyclic
subgroup of $G$ of order $p$ (that is still supposed to act
trivially on $H^0(X,\Omega_X)$). In the previous paragraph we
have shown that $g_Y=0$. To finish the proof of the
if-direction of our theorem it therefore suffices to show that
$p=2$.  This and the other direction follow from the following
proposition. \hfill $\Box$

{\bf Proposition~2.} {\em Let $p  > 0$ and let $G$ be cyclic of
order $p$. We furthermore assume that $g_Y = 0$. Then $G$ acts
trivially on $H^0(X,\Omega_X)$ if and only if one of the
following three conditions holds:\\
(i) $p=2$.\\
(ii) $g_X =0$.\\
(iii) $p=3$ and $g_X=1$. }

{\em Proof.} Let $P_1, \ldots, P_r \in X$ be the ramified
points of $\pi: X \ra Y$ and, for $i=1, \ldots, r$, define $N_i
\in \NN$ by $\textrm{ord}_{P_i}(\sigma(\pi_i) - \pi_i) = N_i
+1$ where $\pi_i$ is a local parameter at $P_i$ and $\sigma$ is
a generator of $G$. From Lemma~1 on p.~87 in \cite{Na} we know
that $p$ does not divide $N_i$, a fact we will use several
times below. The ramification divisor $R$ of $\pi$ is equal to
$\sum_{i=1}^r(N_i+1)(p-1)[P_i]$ by Hilbert's formula for the
order of the different (see Prop.~4, \S 1, Ch.~IV on p.~72 in
\cite{Se}). Let $N:= \sum_{i=1}^r N_i$. Using the Hurwitz
formula we obtain
\[2g_X - 2 = -2p + (N+r)(p-1)\] and hence
\[\textrm{dim}_k\left(H^0(X,\Omega_X)\right) = g_X =
\frac{(N+r-2)(p-1)}{2}.\] Since $g_X \ge 0$ we obtain $r \ge
1$; that is, $\pi$ is not unramified. Therefore we have
\[\textrm{deg} \left\lfloor \frac{\pi_*(R)}{p} \right\rfloor =
\sum_{i=1}^r \left\lfloor \frac{(N_i+1)(p-1)}{p}\right\rfloor
\ge \sum_{i=1}^r \left\lfloor \frac{2(p-1)}{p}\right\rfloor = r
> 0.\] From Proposition~1 we then conclude that
\begin{eqnarray*}
\lefteqn{\textrm{dim}_k
\left(H^0(X,\Omega_X)^G\right) = g_Y -1 + \textrm{deg}
\left\lfloor \frac{\pi_*(R)}{p} \right\rfloor}\\
&=& -1 + \sum_{i=1}^r \left\lfloor \frac{(N_i+1)(p-1)}{p}\right\rfloor\\
&=& -1 + N +r + \sum_{i=1}^r \left\lfloor -\frac{N_i+1}{p}\right\rfloor.
\end{eqnarray*}
If $p=2$ the dimension of both $H^0(X,\Omega_X)$ and
$H^0(X,\Omega_X)^G$ is therefore equal to $\frac{N+r-2}{2}$. If
$g_X = 0$ both dimensions are obviously equal to~$0$. If $p=3$
and $g_X =1$ we obtain $N+r=3$ and hence $r=1$ and $N=2$; thus
both dimensions are equal to $1$. Therefore in all three of
these cases $G$ acts trivially on $H^0(X,\Omega_X)$.
This finishes the proof of the if-direction in Proposition~2.\\
To prove the other direction we now assume that $G$ acts
trivially on $H^0(X, \Omega_X)$ and that $p \ge 3$ and prove
that condition~(ii) or condition~(iii) holds. For each $i=1,
\ldots, r$, we write $N_i = s_i p +t_i$ with $s_i \in \NN$ and
$t_i \in \{1, \ldots, p-1\}$. We furthermore put $S:=
\sum_{i=1}^r s_i$ and $T:= \sum_{i=1}^r t_i \ge r$. Then we
have
\[ \frac{(N+r-2)(p-1)}{2} =\textrm{dim}_k(H^0(X,\Omega_X))  =
\textrm{dim}_k\left(H^0(X,\Omega_X)^G\right) = N-S-1 .\]
Rearranging this equation we obtain
\[(3-p)N - 2S = (r-2)(p-1) +2  \]
and hence
\[(-p^2 + 3p -2)S = (r-2)(p-1) +2 - (3-p)T.\]
Since $-p^2+3p-2 = - (p-1)(p-2)$ and $p \ge 3$ this equation
implies that
\[ S = \frac{(r-2)(1-p)-2 + T (3-p)}{(p-1)(p-2)}. \]
Because $S \ge 0$ the numerator of this fraction is
non-negative, that is
\begin{eqnarray*}
\lefteqn{0 \le (r-2)(1-p) - 2 + T (3-p)}\\
&\le & (r-2)(1-p) - 2 + r (3-p)\\
&=& 2 (r-1)(2-p).
\end{eqnarray*}
Hence we have $r=1$ and that numerator is $0$. We conclude that
$S=0$ and hence that $T=1$ or $p=3$. If $T=1$ we also have
$N=1$ and finally
\[g_X = \frac{(N+r-2)(p-1)}{2} = 0,\]
i.e.\ condition~(ii) holds. If $T \not=1$ and $p=3$ we obtain
$N=T=2$ and finally \[g_X = \frac{(N+r-2)(p-1)}{2} =1,\] i.e.\
condition~(iii) holds. \hfill $\Box$

{\em Acknowledgments.} The question underlying this paper goes
back to Michel Matignon. I would like to thank Niels Borne for
communicating this question to me, for pointing me to the above
lemma and for sketching a proof of the fact that $G$ acts
faithfully on $H^0(X,\Omega_X)$ in the tamely ramified case.








\begin{thebibliography}{FWK}
\bibitem[Bo]{Bo} \textsc{N.~Borne}, Cohomology of
    $G$-sheaves in positive characteristic, \textit{Adv.\ Math.}~\textbf{201} (2006),
    454-515.
\bibitem[CW]{CW} \textsc{C.~Chevalley} and \textsc{A.~Weil},
    \"Uber
    das
    Verhalten der Integrale erster Gattung bei Automorphismen
    des Funktionenk\"orpers, \textit{Hamb.\ Abh.}~\textbf{10} (1934),
    358-361.
\bibitem[FWK]{FWK} \textsc{H.~Fischbacher-Weitz} and
    \textsc{B.~K\"ock}, Equivariant Riemann-Roch theorems for
    curves over perfect fields, \textit {Manuscripta
    Math.}~\textbf{128} (2009), 89-105.
\bibitem[Ha]{Ha} \textsc{R.~Hartshorne}, Algebraic
    Geometry, \textit{Grad.\ Texts in Math.}, vol.~52, Springer, New York 1977.
\bibitem[Ka]{Ka} \textsc{E.~Kani}, The Galois-module structure
    of the space of holomorphic differentials of a curve,
    \textit{J.\ Reine Angew. Math.}~\textbf{367} (1986),
    187-206.
\bibitem[K\"o]{Ko} \textsc{B.~K\"ock}, Galois
    structure  of
    Zariski cohomology for weakly ramified covers of curves,
    \textit{American Journal of Mathematics}~\textbf{126} (2004),
    1085-1107.
\bibitem[Na]{Na} \textsc{S.\ Nakajima}, Action of an
    automorphism of order $p$ on cohomology groups of an
    algebraic curve, \textit{J.\ Pure Appl.\ Algebra}~\textbf{42}
    (1986), 85-94.
\bibitem[Se]{Se} \textsc{J.-P.~Serre}, Corps locaux, \textit{
    Publications de l'Institut de Math\'ematique de l'Universit\'e de
  Nancago VIII}, Hermann, Paris 1962.
\end{thebibliography}

\bigskip

\bigskip


School of Mathematics, University of Southampton, Southampton
SO17 1BJ, UK. {\em E-mail:} B.Koeck@soton.ac.uk.


\end{document}
