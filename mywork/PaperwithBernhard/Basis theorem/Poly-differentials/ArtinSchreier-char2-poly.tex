% !TEX TS-program = pdflatex
% !TEX encoding = UTF-8 Unicode

% This is a simple template for a LaTeX document using the "article" class.
% See "book", "report", "letter" for other types of document.

\documentclass[11pt]{article} % use larger type; default would be 10pt

\usepackage[utf8]{inputenc} % set input encoding (not needed with XeLaTeX)

%%% Examples of Article customizations
% These packages are optional, depending whether you want the features they provide.
% See the LaTeX Companion or other references for full information.

%%% PAGE DIMENSIONS
\usepackage{geometry} % to change the page dimensions
\geometry{a4paper} % or letterpaper (US) or a5paper or....
% \geometry{margins=2in} % for example, change the margins to 2 inches all round
% \geometry{landscape} % set up the page for landscape
%   read geometry.pdf for detailed page layout information

\usepackage{graphicx} % support the \includegraphics command and options

\usepackage[parfill]{parskip} % Activate to begin paragraphs with an empty line rather than an indent

%%% PACKAGES
\usepackage{booktabs} % for much better looking tables
\usepackage{array} % for better arrays (eg matrices) in maths
\usepackage{paralist} % very flexible & customisable lists (eg. enumerate/itemize, etc.)
\usepackage{verbatim} % adds environment for commenting out blocks of text & for better verbatim
\usepackage{subfig} % make it possible to include more than one captioned figure/table in a single float
% These packages are all incorporated in the memoir class to one degree or another...

%%% HEADERS & FOOTERS
\usepackage{fancyhdr} % This should be set AFTER setting up the page geometry
\pagestyle{fancy} % options: empty , plain , fancy
\renewcommand{\headrulewidth}{0pt} % customise the layout...
\lhead{}\chead{}\rhead{}
\lfoot{}\cfoot{\thepage}\rfoot{}

%%% SECTION TITLE APPEARANCE
\usepackage{sectsty}
\allsectionsfont{\sffamily\mdseries\upshape} % (See the fntguide.pdf for font help)
\usepackage{amsmath}
\usepackage{amsthm}
\usepackage{amsfonts}
\usepackage{mathrsfs}
\usepackage{amsopn}
\usepackage{amssymb}
\usepackage{natbib}
% (This matches ConTeXt defaults)

%%% ToC (table of contents) APPEARANCE
\usepackage[nottoc,notlof,notlot]{tocbibind} % Put the bibliography in the ToC
\usepackage[titles,subfigure]{tocloft} % Alter the style of the Table of Contents
\renewcommand{\cftsecfont}{\rmfamily\mdseries\upshape}
\renewcommand{\cftsecpagefont}{\rmfamily\mdseries\upshape} % No bold!

%Theorems and stuff
\newtheorem{defn}{Definition}
\newtheorem{thm}{Theorem}
\newtheorem{cor}{Corollary}
\newtheorem{lem}{Lemma}
\newtheorem{prop}{Proposition}
\theoremstyle{remark}\newtheorem*{rem}{Remark}

\newcommand{\cO}{{\cal O}}
\newcommand{\ra}{\rightarrow}
\newcommand{\NN}{{\mathbb N}}
\newcommand{\PP}{{\mathbb P}}
\newcommand{\ZZ}{{\mathbb Z}}
\newcommand{\cL}{{\cal L}}

\DeclareMathOperator{\ord}{ord}
\DeclareMathOperator{\di}{div}
\DeclareMathOperator{\cha}{char}
\DeclareMathOperator{\gal}{Gal}


%%% END Article customizations

%%% The "real" document content comes below...

\title{Basis of holomorphic poly-differentials of a hyper-elliptic curve part 1: $\cha (k)=2$}
\author{J Tait}
%\date{} % Activate to display a given date or no date (if empty),
         % otherwise the current date is printed 
         
         
\begin{document}
\maketitle


Let $k$ be an algebraically closed field of characteristic 2.
Let $k(x)$ be the rational function field in one variable $x$ and let $K|k(x)$ be a field extension of degree 2, i.e. an Artin-Schreier extension.
Let $\pi:C \rightarrow \mathbb P_k^1$ be the corresponding morphism of smooth projective curves over $k$ and let $g$ denote the genus of $C$.
By \cite[\S 7.4.3]{liu}, there exists a $y\in K$ such that $K=k(x,y)$ and $y$ satisfies the equation
\begin{equation}\label{1}
  y^2 - h(x)y = f(x)
\end{equation}
for some polynomials $h(x), f(x)\in k[x]$, with maximum degrees of $g+1$ and $2g+2$ respectively.
Let $d$ be the degree of $h(x)$.
We denote the space of global holomorphic differentials of $C$ by $\Omega$.

We first describe the ramified points of $\pi$, in order to compute the ramification divisor.
Considering $\mathbb A_k^2$ as a subspace of $\mathbb P_k^2$, let $C'$ the part of the curve in $\mathbb A_k^2$.
Then $\pi$ restricts to a map $C'\rightarrow \mathbb A^1_k$, the projection on the $x$ co-ordinate.
Let $a\in \mathbb A_k^1$ be a point in the image of $\pi$.
Then if $(a,b)$ is a point in $\pi^{-1}(a)$, so is the point $(a,b+h(a))$, which is clearly distinct if and only if $h(a)\neq 0$.
Since the extension is degree two, this shows that the only ramified points in the affine part are the roots of $h(x)$.
If $h(x)$ is non-constant, we denote zeroes of $h(x)$ by $P_i=(a_i,b_i)$ for $1\leq i \leq k$ where $k \leq g+1$.
If these are not all the ramification points, then the point at infinity is also ramified.
In this case it will be written $P_{\infty}$.
If this is not the case we will denote the two points that map to infinity as $P_{\infty}'$ and $P_{\infty}''$.

We will now compute the ramification divisor, which will allow us to see exactly when $P_1,\ldots,P_k$ are all the ramified points.\\

\begin{lem}
 Let $n_i$ be the order of $h(x)$ at $a_i\in \mathbb A_k^1$.
Then we can write the ramification divisor of $\pi$ is precisely
 \[
  R = \left\{ \begin{array}{ll}
               \sum_{i=1}^k 2n_i[P_i] & \mbox{if } \sum_{i=1}^k 2n_i = 2g+2 \\
		\\
               \sum_{i=1}^k 2n_i[P_i] + (2g+2-2\sum_{i=1}^kn_i)[P_{\infty}] & \mbox{otherwise.}
             \end{array} \right. \]

\end{lem}
\begin{rem}
 We will show that the point at infinity is ramified if and only if $\sum_{i=1}^k n_i = \deg(h(x)) < g+1$
 (or equivalently it is unramified if and only  $\sum_{i=1}^k n_i = g+1$).
 Hence the equality will always make sense in the second case.
\end{rem}
\begin{proof}
 First note that by the Hurwitz formula $\deg(R) = 2g+2$.
 Then if $\sum_{i=1}^k n_i < 2g+2$ is all the ramification from the points $P_i$, there must exist another point, i.e. $P_{\infty}$, which has ramification $2g+2-\sum _{i=1}^k2n_i$.
 So it will suffice to prove that coefficient of $[P_i]$ is $2n_i$ for $1\leq i \leq k$.
 
 We start by showing that $y-b_i$ is a uniformising parameter in $\mathscr O_{P_i}$.
 Now $x-a_i$ is a uniformising parameter at $\pi(P_i)$, so it will be sufficient to show that $x-a_i = u(y-b_i)^2$ for some unit $u\in \mathscr O_{P_i}$.
 We write $y^2-b_i^2 = yh(x)-(f(x)-b_i^2)$, and then write $h(x)=(x-a_i)w(x)$ and $f(x)-b_i^2=(x-a_i)v(x)$.
 So 
 \[
x-a_i = \frac{y^2-b_i^2}{yw(x)-v(x)} = \frac{(y-b_i)^2}{yw(x)-v(x)},
 \]
and it remains to show that the denominator, $r(x,y) = yw(x)-v(x)$, is a unit.

To show that $r(x,y)$ is indeed a unit we consider two cases, and show that in both the polynomial is not in the maximal ideal $\mathfrak M_{P_i}=(x-a_i,y-b_i)$.
Firstly, if $b_i=0$ then $y\in \mathfrak{M}_{P_i}$, and hence $r\in \mathfrak M_{P_i}$ if and only if $v(x)$ is.
But then we would have $(x-a_i)|v(x)$, and hence $(x-a_i)^2|f(x)$, and this would contradict the smoothness of the curve.
Alternatively, if $b_i\neq 0$, then $r(x,y)\in \mathfrak M_{P_i}$ if and only if $(x-a_i)|r(x,y)$.
But then $(x-a_i)^2|yh(x)-f(x)$, and so $(x-a_i)|yh'(x)-f'(x)$, which again would contradict smoothness (recall that $(x-a_i)|h(x)$ by virture of $P_i$ being a ramification point).
So we have shown that $r(x,y)\in \mathscr O_{P_i}^*$.

Now write $h(x)=(x-a_i)^ns(x)$ where $s(x)\in \mathscr O_p^*$.
Note that $\sigma (y)=y+h(x)$, by an argument similar to that used to show that the solutions to $h(x)$ give ramified points.
So there exist odd $N_{P_i}\in \mathbb N$ defined by $N_{P_i}+1=\ord_P(\sigma(y - b_i) - (y-b_i))$, which are the coefficients of $P_i$ in $R$.
We compute these $N_{P_i}+1$ explicitly,
\begin{eqnarray*}
 \ord_{P_i}(\sigma(y-b_i)-(y-b_i)) & = & \ord_{P_i}(y-b_i-h(x)-y+b_i)\\
 & = & \ord_{P_i}((x-a_i)^ns(x))\\
 & = & \ord_{P_i}((y-b_i)^{2n}r(x,y)s(x))^{-1}\\
 & = & 2n_i,
\end{eqnarray*}
finishing the proof.
\end{proof}


We first compute the divisor associated to $x\in K$.

\begin{lem}
 The divisor of $x\in K(X)$ is 
 \[
  \di (x)= \left\{ 
  \begin{array}{ll}
    P_0+P_0' - 2P_{\infty} & \mbox{if } d<g+1\\
    P_0+P_0' - P_{\infty}' - P_{\infty}'' & \mbox{otherwise.}
  \end{array}\right.
\]
\end{lem}
\begin{proof}
 Given our assumptions above, this is clear.
\end{proof}

\begin{lem}
 The divisor associated to the differential $dx^{\otimes m}$ is 
 \[
  \di (dx^{\otimes m}) = \left\{ 
  \begin{array}{ll}
   2m\sum_{i=1}^k n_iP_i + 2m(g-1-d)P_{\infty} & \mbox{if } d<g+1\\
   2m\sum_{i=1}^k n_iP_i - 2m(P_{\infty}' + P_{\infty}'') & \mbox{otherwise.}
  \end{array}\right.
 \]
\end{lem}
\begin{proof}
 We first note that it suffices to compute $\di (dx)$, since $\di (dx^{\otimes m}) = m\di (dx)$.

 If $a\in \mathbb A_k^1$ is unramified, then $x-a$ is a uniformising parameter at both points in the preimage.
Then if $P$ maps to $a$, then $\ord_P(dx)=\ord_P(d(x-a))=0$, by \cite[\S 8.5]{fulton}.
Now consider any ramified point $a_i\in \mathbb A_k^1$, and as in the proof of the above lemma, we can write $x-a_i$ as $\frac{(y-b_i)^2}{yw(x)-v(x)}$.
This gives us that
\begin{eqnarray*}
 dx & = & d(x-a_i) \\
 & = & d \left( \frac{(y-b_i)^2}{yw(x)-v(x)} \right) \\
 & = & \frac{2y}{yw-v} dy + (y-b_i)^2\left( d\frac{1}{yw-v}\right)\\
 & = & \left( \frac{y-b_i}{yw+v}\right)^2d(yw+v)\\
 & = & \left( \frac{y-b_i}{yw+v}\right)^2(d(yw)+dv)\\
 & = & \left( \frac{y-b_i}{yw+v}\right)^2d(yw),
 \end{eqnarray*}
 and 
 \[
\ord_{P_i}\left( \left( \frac{y-b_i}{yw+v}\right)^2(d(yw)+dv)\right) = \ord_{P_i}\left( \left( \frac{y-b_i}{yw+v}\right)^2d(yw)\right)
\]
since $v$ is a unit.
If $(x-a_i)\nmid w$ then $w$ is a unit and we see that the order of $dx$ at $P_i$ is 2.
Otherwise, we write $w(x)=(x-a_i)w(x)'$, and iterate the process.
This gives $\ord_{P_i}(dx)=2n_i$.
Note that the case where $w$ is a unit is precisely the case where $n_i=1$.

Finally, since $dx\neq 0$ then $\di (dx)$ is a canonical divisor and has degree $2g-2$.
It follows that $\ord_{P_{\infty}}(dx)=2(g-1-d)$ if $d<g+1$, and $\ord_{P_{\infty}'}(dx) =\ord_{P_{\infty}''}(dx)=(g-1-d) = -2$ otherwise.
 \end{proof}

 \begin{lem}
  The divisor associated to $\frac{1}{h(x)}$ is
  \[
  \di \left(\frac{1}{h(x)}\right) = \left \{
  \begin{array}{ll}
   -2\sum_{i=1}^k n_i P_i + 2dP_{\infty} & \mbox{if } d<g+1\\
   -2\sum_{i=1}^k n_i P_i + d(P_{\infty}' + P_{\infty}'') & \mbox{otherwise.}
  \end{array}\right.
 \]
 \end{lem}
\begin{proof}
If infinity is ramified then $\ord_{P_{\infty}}\left(\frac{1}{h(x)}\right) = -\ord_{P_{\infty}}(h(x)) = 2d$.
If it is not ramified, then $\ord_{P_{\infty}'}\left(\frac{1}{h(x)}\right) = \ord_{P_{\infty}''}\left(\frac{1}{h(x)}\right)=d$.
For the ramified points $P_i$, $1\leq i \leq k$, then $\ord_{P_i}\left(\frac{1}{h(x)}\right) = -\ord_{P_i}(h(x))= -2n_i$.
At any other point the order of $\frac{1}{h(x)}$ is clearly zero.
\end{proof}

 
 \begin{lem}
  The divisor associated to $y$ is 
  \[
   \di (y) = \left\{ 
   \begin{array}{ll}
   \sum_{i=1}^l m_i Q_i - (2g+1)P_\infty & \mbox{if } d<g+1 \\
    \ldots -(g+1)(P_\infty' + P_\infty'') & \mbox{otherwise},
   \end{array}\right.
 \]
 where $Q_i$ are the solutions to $f(x)$, $m_i$ are the corresponding multiplicities. and $l$ is the number of distinct roots of $f(x)$. 
 \end{lem}
\begin{proof}
Clearly if $(a,0)$ is a point in the affine part of the curve, then $f(a) = 0$, and their is a corresponding point $(a,h(a))$. 
Also, if $f(a) = 0$ then one of the two points in $\pi^{-1}(a)$ is $(a,0)$.
So we have a bijection between the zeroes of $f(x)$ and $y$ {\bf (need to consider $h(a)=0$).}
This shows that the positive part of $\di (y)$ is $\sum_{i=1}^l m_i Q_i$.

Now if $d<g+1$ then $\deg(f(x)) = 2g+1$ and also $P_\infty$ is ramified.
It is clear that $\ord_{P_\infty}(f(x)) = -2(2g+1)$, and also that $\ord_{P_\infty}(h(x))\geq-2g>-2(2g+1)$, hence the order of $y$ at $P_\infty$ is also negative.
Then 
\begin{eqnarray*}
 -\ord_{P_{\infty}}(y^2-h(x)y) & = & {\rm max}\{ -2\ord_{P_{\infty}}(y),-\ord_{P_{\infty}}(h(x)y) \}\\
 & = & -ord_{P_\infty}(f(x)) \\
 & = & 2(2g+1).
\end{eqnarray*}

 Since $\deg(h(x))< g+1$ then we conclude that $2\ord_{Q_{\infty}}(y)$ must be the maximum.
 So $\ord_{\infty}(y) = -(2g+1)$, and 
 \[
  \di(y) =\sum_{i=1}^l Q_i - (2g+1)P_{\infty}.
  \]

 
If $d=g+1$, we know that 
\[
 \ord_{P_\infty'}(y^2)={\rm min} \{\ord_{P_\infty'}(f(x)),\ord_{P_\infty'}(h(x)y)\}.
\]
Now if $\ord_{P_\infty'}(f(x))$ is the minimum, then $\ord_{P_\infty'}(y)=\frac{1}{2}\ord_{P_\infty'}(f(x))$.
But then since $\ord_{P_\infty'}(h(x)) \leq \ord_{P_\infty'}(f(x))$ (recall that $\deg(f(x)) \leq 2g+2$, and that $P_\infty$ is a pole), then this would imply that $\ord_{P_\infty'}(h(x)y)$ is the minimum.
According to whether or not $\deg(f(x))=2g+2$, this implies that we can choose either polynomials, or that $\ord_{P_\infty'}(h(x)y)$ is the minimum.
So we have shown that
\[
 \ord_{P_\infty'}(y^2)=\ord_{P_\infty'}(h(x)y),
\]
whence it follows that $\ord_{P_\infty'}(y)=g+1$.
The same argument holds for $P_\infty''$.


\end{proof}


We now prove the following theorem, determining a basis of the space of global holomorphic poly-differentials as a vector space over $k$.\\

\begin{prop}
Let $C$, $x$ and $y$ be as above. 
Then
\[ \frac{dx^{\otimes m}}{h(x)^m}, \frac{xdx^{\otimes m}}{h(x)^m}, \ldots , \frac{x^{m(g-1)}dx^{\otimes m}}{h(x)^m}, 
\frac{ydx^{\otimes m}}{h(x)^m},\ldots , \frac{x^{(g-1)(m-1)-2}ydx^{\otimes m}}{h(x)^m}\]
form a basis of $\Omega^m$.
\end{prop}
\begin{proof}
 Using the above lemmas we compute the divisors of $\frac{x^idx^{\otimes m}}{h(x)^m}$ for $0\leq i \leq m(g-1)$ and $\frac{yx^jdx^{\otimes m}}{h(x)^m}$ for $0\leq j \leq (g-1)(m-1)-2$.
 
 Let $D_\infty=2 P_\infty$ if $d<g+1$, and let $D_\infty=P_\infty' + P_\infty''$ otherwise.
 Then in the first case, 
 \begin{eqnarray*}
  \di\left(\frac{x^idx^{\otimes m}}{h(x)^m}\right) & = &  iP_0+iP_0' - iD_{\infty} + m\sum_{i=1}^n n_iP_i + m(g-1-d)D_{\infty} \\
  & &  \quad \quad \quad\quad\quad\quad\quad\quad\quad\quad\quad\quad -m\sum_i^n n_i P_i +mdD_{\infty}\\
  & = & iP_0 + iP_0' + (m(g-1)-i)D_{\infty}.
 \end{eqnarray*}
 So clearly this is a global holomorphic differential for the stated values of $i$.
 
 In the second case we need to consider seperately whether $d=g+1$ or not.
 If $d<g+1$ then we have 
 \begin{eqnarray*}
  \di\left(\frac{yx^jdx^{\otimes m}}{h(x)^m}\right) & = &\sum_i^l m_iQ_i - (2g+1)P_\infty + jP_0+jP_0' - 2jP_\infty + \\
  & & m\sum_{i=1}^n n_i P_i + 2m(g-1-d)P_{\infty} -m\sum_i^n n_i P_i +2mdP_{\infty} \\
  & = & \sum_i^l m_iQ_i  + jP_0+jP_0' + (2m(g-1)-2j -(2g+1))P_{\infty} \\
  & = & \sum_i^l m_iQ_i  + jP_0+jP_0' + (2(g-1)(m-1)-3-2j)P_\infty,
  \end{eqnarray*}
  which finishes the proof in the case.
  
  Finally, we look at when $d=g+1$.
  Then
  \begin{eqnarray*}
     \di\left(\frac{yx^jdx^{\otimes m}}{h(x)^m}\right) & = & \ldots- (g+1)(P_\infty'+P_\infty'') + jP_0+jP_0' - j(P_\infty'+P_\infty'') + \\
  & & m\sum_{i=1}^n n_i P_i + m(g-1-d)(P_{\infty}+P_\infty) -m\sum_i^n n_i P_i + md(P_{\infty}'+P_\infty'') \\
  & = & \ldots + jP_0+jP_0' + (m(g-1)-j -(g+1))(P_{\infty}'+P_\infty) \\
  & = & \ldots  + jP_0+jP_0' + ((g-1)(m-1)-2-j)(P_\infty'+P_\infty''),
  \end{eqnarray*}
completing the proof.
 
  So all the elements described above are global holomorphic differentials.
\end{proof}


\bibliography{/home/jtait/files/Documents/Maths/Bibliography/biblio.bib}
\bibliographystyle{plain}

\end{document}
