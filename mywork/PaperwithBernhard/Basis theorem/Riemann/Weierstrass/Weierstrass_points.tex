% !TEX TS-program = pdflatex
% !TEX encoding = UTF-8 Unicode

% This is a simple template for a LaTeX document using the "article" class.
% See "book", "report", "letter" for other types of document.

\documentclass[11pt]{article} % use larger type; default would be 10pt

\usepackage[utf8]{inputenc} % set input encoding (not needed with XeLaTeX)

%%% Examples of Article customizations
% These packages are optional, depending whether you want the features they provide.
% See the LaTeX Companion or other references for full information.

%%% PAGE DIMENSIONS
\usepackage{geometry} % to change the page dimensions
\geometry{a4paper} % or letterpaper (US) or a5paper or....
% \geometry{margins=2in} % for example, change the margins to 2 inches all round
% \geometry{landscape} % set up the page for landscape
%   read geometry.pdf for detailed page layout information

\usepackage{graphicx} % support the \includegraphics command and options

\usepackage[parfill]{parskip} % Activate to begin paragraphs with an empty line rather than an indent

%%% PACKAGES
\usepackage{booktabs} % for much better looking tables
\usepackage{array} % for better arrays (eg matrices) in maths
\usepackage{paralist} % very flexible & customisable lists (eg. enumerate/itemize, etc.)
\usepackage{verbatim} % adds environment for commenting out blocks of text & for better verbatim
\usepackage{subfig} % make it possible to include more than one captioned figure/table in a single float
% These packages are all incorporated in the memoir class to one degree or another...

%%% HEADERS & FOOTERS
\usepackage{fancyhdr} % This should be set AFTER setting up the page geometry
\pagestyle{fancy} % options: empty , plain , fancy
\renewcommand{\headrulewidth}{0pt} % customise the layout...
\lhead{}\chead{}\rhead{}
\lfoot{}\cfoot{\thepage}\rfoot{}

%%% SECTION TITLE APPEARANCE
\usepackage{sectsty}
\allsectionsfont{\sffamily\mdseries\upshape} % (See the fntguide.pdf for font help)
\usepackage{amsmath}
\usepackage{amsthm}
\usepackage{amsfonts}
\usepackage{mathrsfs}
\usepackage{amsopn}
\usepackage{amssymb}
\usepackage{natbib}
% (This matches ConTeXt defaults)

%%% ToC (table of contents) APPEARANCE
\usepackage[nottoc,notlof,notlot]{tocbibind} % Put the bibliography in the ToC
\usepackage[titles,subfigure]{tocloft} % Alter the style of the Table of Contents
\renewcommand{\cftsecfont}{\rmfamily\mdseries\upshape}
\renewcommand{\cftsecpagefont}{\rmfamily\mdseries\upshape} % No bold!

%Theorems and stuff
\newtheorem{defn}{Definition}
\newtheorem{thm}{Theorem}
\newtheorem{cor}{Corollary}
\newtheorem{lem}{Lemma}
\newtheorem{prop}{Proposition}
\newtheorem{ex}{Example}

\DeclareMathOperator{\ord}{ord}
\DeclareMathOperator{\di}{div}
\DeclareMathOperator{\cha}{char}
\DeclareMathOperator{\gal}{Gal}
%%% END Article customizations

%%% The "real" document content comes below...

\title{Weierstrass points}
\author{J Tait}
%\date{} % Activate to display a given date or no date (if empty),
         % otherwise the current date is printed 

\begin{document}
\maketitle
Let $X$ be a Riemann surface.
Fix a point $P\in X$.\\

\begin{defn}
 We call $n \in \mathbb{N}$ a gap number for $|K|$ at $P$ if $L(K-nP) \neq L(K-(n-1)P)$.
\end{defn}

By Riemann-Roch we have
\[
 L(K-np) = 2g-2-n+1-g + (L(nP)),
\]
where $L$ denotes the dimension of the space associated to the divisor.

So 
\begin{eqnarray*}
 L(K-(n-1)P) - L(K-nP) & = & g-1-(n-1) + L((n-1)P) -(g-1-n + L(nP))\\
  & = & L((n-1)P) + 1 - L(nP).
\end{eqnarray*}
So $n$ is a gap number if and only if $L(nP)=L((n-1)P)$.

We can look at the nested sequence 
\[
 \{0\}\subset l(0) \subseteq l(P) \subseteq \ldots \subseteq l(nP) \subseteq \ldots
\]
and we have equality from $l((n-1)p)$ to $l(np)$ if and only if $n$ is a gap number.

At each step the dimension can increase by at most one. We know that once $n \geq 2g-1$ the dimension will increase by precisely one each time.
We also know that $L((2g - 2)P) = g$, whilst $L(0) = 1$. So we have $2g-1$ steps, each with a an increase of at most 1, to go from dimension 1 to dimension $g$.
Hence there must be $2g-1 - (g-1)=g$ integers where the dimension does not increase, and we call these the gap numbers, denoted $G_P(|K|)$.\\

\begin{defn}
 We call $P$ a Weierstrass point if $G_P(|K|) \neq \{1,2,\ldots, g\}$. 
 i.e. if $l(0)=l(P)=\ldots = l(gP) \subsetneq l((g+1)P) \subsetneq \ldots $.
\end{defn}

Note that this definition is equivalent to saying the $L(gP)\geq 2$.

\subsection{Questions:}
1) {\em Show that $2\notin G_P(|K|)$ if and only if $X$ is hyperelliptic and $P$ is a ramification map for the double covering $\pi: X\rightarrow \mathbb P^1$.
If so, show that $G_P(|k|) = \{1,3,5,\ldots ,2g-1\}$.}

Suppose $X$ is hyperelliptic and that $P$ is a ramification point.
Let $a = \pi (P)$. 
Then by identifying $\mathbb P^1$ with $k\cup \infty$ we can define a map $f:\mathbb P^1 \rightarrow k\cup \infty$ which has a zero of order 1 at $a$. 
i.e. $f= (x-a)$.
Then, since $X$ is hyperelliptic and $P$ is a ramification point, we ave $\ord_P(f\circ \pi) = 2$.
So there is a meromorphic map with a pole of exactly degree 2 at $P$ and no other poles, and hence $2\notin G_P(|K|)$.
In this case, since we have a function with order 2 at $P$, we can take powers to get a function with order $n$ for each even $n \in \mathbb N$.
Hence $n \notin G_P(|K|)$ for $n$ even, which means that $G_P(|K|) \supseteq \{1,3,\ldots 2g-1\}$. 
But since $G_P(|K|)$ contains $g$ elements, and so does this set, we have equality.

Now if $X$ is hyperelliptic but $P$ is not a branch point, then the preimage of $a$ contains two points, say $P'$ and $P$.
So if a meromorphic function has a pole at $P$ then it must also have a pole at $P'$.
Hence it cannot be in $l(nP)$.


If $X$ is not hyperelliptic, then first note that $P$ must a totally ramified, else there will be multiple points in the preimage, and we can run the same argument as above.
Since $X$ is hyperelliptic, any map $\pi:X \rightarrow \mathbb P^1$ must de degree at least 3.
So any map is determined by $f:\mathbb P^1 \rightarrow k\cup \infty$ that has a pole at $a = \pi (P)$, must have order equal to the degree.
Hence it's order is not precisely 2, and we are done.

2) {\em Show that if $X$ is hyperelliptic then there are exactly $2g+2$ Weierstrass points on $X$, each having the above set as the set of gap numbers.}

The only points with positive weights are the ramification points, or equivalently by the previous exercise, the points with $G_P(|K|) = \{1,3,\ldots ,2g-1\}$.
In this case the weight is the sum of the first $g-1$ even numbers, which is $g(g-1)$. 
We then halve this due to the degree of the map.
By Corollary 4.17 in \cite[Ch. VII]{mir} we then see that the number of Weierstrass points is $x$ where $x\cdot \frac{g(g-1)}{2} = g(g+1)(g-1)$.
Hence $x$ clearly equals $2g+2$.



\bibliography{/home/jtait/files/Desktop/Work/Bibliography/biblio.bib}
\bibliographystyle{plain}


\end{document}