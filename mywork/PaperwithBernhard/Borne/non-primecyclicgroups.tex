% !TEX TS-program = pdflatex
% !TEX encoding = UTF-8 Unicode

% This is a simple template for a LaTeX document using the "article" class.
% See "book", "report", "letter" for other types of document.

\documentclass[11pt]{article} % use larger type; default would be 10pt

\usepackage[utf8]{inputenc} % set input encoding (not needed with XeLaTeX)

%%% Examples of Article customizations
% These packages are optional, depending whether you want the features they provide.
% See the LaTeX Companion or other references for full information.

%%% PAGE DIMENSIONS
\usepackage{geometry} % to change the page dimensions
\geometry{a4paper} % or letterpaper (US) or a5paper or....
% \geometry{margins=2in} % for example, change the margins to 2 inches all round
% \geometry{landscape} % set up the page for landscape
%   read geometry.pdf for detailed page layout information

\usepackage{graphicx} % support the \includegraphics command and options

\usepackage[parfill]{parskip} % Activate to begin paragraphs with an empty line rather than an indent

%%% PACKAGES
\usepackage{booktabs} % for much better looking tables
\usepackage{array} % for better arrays (eg matrices) in maths
\usepackage{paralist} % very flexible & customisable lists (eg. enumerate/itemize, etc.)
\usepackage{verbatim} % adds environment for commenting out blocks of text & for better verbatim
\usepackage{subfig} % make it possible to include more than one captioned figure/table in a single float
% These packages are all incorporated in the memoir class to one degree or another...

%%% HEADERS & FOOTERS
\usepackage{fancyhdr} % This should be set AFTER setting up the page geometry
\pagestyle{fancy} % options: empty , plain , fancy
\renewcommand{\headrulewidth}{0pt} % customise the layout...
\lhead{}\chead{}\rhead{}
\lfoot{}\cfoot{\thepage}\rfoot{}

%%% SECTION TITLE APPEARANCE
\usepackage{sectsty}
\allsectionsfont{\sffamily\mdseries\upshape} % (See the fntguide.pdf for font help)
\usepackage{amsmath}
\usepackage{amsthm}
\usepackage{amsfonts}
\usepackage{mathrsfs}
\usepackage{amsopn}
\usepackage{amssymb}
\usepackage{natbib}
% (This matches ConTeXt defaults)

%%% ToC (table of contents) APPEARANCE
\usepackage[nottoc,notlof,notlot]{tocbibind} % Put the bibliography in the ToC
\usepackage[titles,subfigure]{tocloft} % Alter the style of the Table of Contents
\renewcommand{\cftsecfont}{\rmfamily\mdseries\upshape}
\renewcommand{\cftsecpagefont}{\rmfamily\mdseries\upshape} % No bold!

%Theorems and stuff
\newtheorem{defn}{Definition}
\newtheorem{thm}{Theorem}
\newtheorem{cor}{Corollary}
\newtheorem{lem}{Lemma}
\newtheorem{prop}{Proposition}
\newtheorem{ex}{Example}

\DeclareMathOperator{\ord}{ord}
\DeclareMathOperator{\di}{div}
\DeclareMathOperator{\cha}{char}
\DeclareMathOperator{\gal}{Gal}
%%% END Article customizations

%%% The "real" document content comes below...

\title{Cyclic groups of order $p^{\nu}$ acting on curves}
\author{J Tait}
%\date{} % Activate to display a given date or no date (if empty),
         % otherwise the current date is printed 

\begin{document}
\maketitle

Suppose that $G$ is a cyclic group of order $p^{\nu}$ (i.e. $G\cong \mathbb{Z}/p^{\nu}\mathbb{Z}$) acting fatihfully on a projective
curve $X$ over $k$, an algebraically closed field of positive characteristic $p$. Let $Y:=X/G$, the quotient curve of $X$, and let $g_Y$ and $g_X$ be the genus of $X$ and $Y$ respectively. We denote the sub-modules 
$k[G]/(\sigma-1)^j$ of $k[G]$. Now if $D$ is a $G$-invariant divisor on $X$ and $\deg(D)>2g_X-2$ then $H^0(X,\mathscr{L}_X(D))\cong \oplus_{j=1}^{p^{\nu}}
V_j^{\oplus m_j}$. Here the integers $m_j$ for $1\leq j \leq p^{\nu}$ are given by
   \begin{equation*}
      \begin{cases}
	m_1=2a_1-a_2 &\\
	m_j=-a_{j-1}+2a_j-a_{j+1} & \mbox{if } 2\leq j \leq p^{\nu}-1 \\
	m_{p^{\nu}}=a_{p^{\nu}}-a_{P^{\nu}-1} &
      \end{cases}
   \end{equation*}
where $a_j=j(1-g_Y)+\sum_{i=1}^j\deg(\pi_{\nu *}^{\alpha_0(i)}\ldots\pi_{1*}^{\alpha_{\nu-1}}(i)D)$ and $\alpha_0(l),\ldots ,\alpha_{\nu -1}(l)$ are the digits 
in the $p$-adic writing of $l-1$ defined by $l-1=\sum_{h=0}^{\nu -1}\alpha_h(l)p^h$ with $0 \leq \alpha_h(j)\leq p-1$ for $1\leq l \leq p^{\nu}$.

\begin{thm}
  The dimension of the $G$-invariant subspace of $H^0(X,\mathscr{L}_X(D))$ is $\dim_kH^0(X,\mathscr{L}_X(D))^G=\sum_{i=1}^{p^{\nu}}m_j$.
\end{thm}
\begin{proof}
  It suffices to show that the $G$ invarinat $\cong$

\end{proof}

\begin{thm}
  
\end{thm}


\bibliography{/home/jtait/Desktop/Work/Bibliography/biblio.bib}
\bibliographystyle{plain}


\end{document}