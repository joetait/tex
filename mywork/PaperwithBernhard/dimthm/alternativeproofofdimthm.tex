% !TEX TS-program = pdflatex
% !TEX encoding = UTF-8 Unicode

% This is a simple template for a LaTeX document using the "article" class.
% See "book", "report", "letter" for other types of document.

\documentclass[11pt]{article} % use larger type; default would be 10pt

\usepackage[utf8]{inputenc} % set input encoding (not needed with XeLaTeX)

%%% Examples of Article customizations
% These packages are optional, depending whether you want the features they provide.
% See the LaTeX Companion or other references for full information.

%%% PAGE DIMENSIONS
\usepackage{geometry} % to change the page dimensions
\geometry{a4paper} % or letterpaper (US) or a5paper or....
% \geometry{margins=2in} % for example, change the margins to 2 inches all round
% \geometry{landscape} % set up the page for landscape
%   read geometry.pdf for detailed page layout information

\usepackage{graphicx} % support the \includegraphics command and options

\usepackage[parfill]{parskip} % Activate to begin paragraphs with an empty line rather than an indent

%%% PACKAGES
\usepackage{booktabs} % for much better looking tables
\usepackage{array} % for better arrays (eg matrices) in maths
\usepackage{paralist} % very flexible & customisable lists (eg. enumerate/itemize, etc.)
\usepackage{verbatim} % adds environment for commenting out blocks of text & for better verbatim
\usepackage{subfig} % make it possible to include more than one captioned figure/table in a single float
% These packages are all incorporated in the memoir class to one degree or another...

%%% HEADERS & FOOTERS
\usepackage{fancyhdr} % This should be set AFTER setting up the page geometry
\pagestyle{fancy} % options: empty , plain , fancy
\renewcommand{\headrulewidth}{0pt} % customise the layout...
\lhead{}\chead{}\rhead{}
\lfoot{}\cfoot{\thepage}\rfoot{}

%%% SECTION TITLE APPEARANCE
\usepackage{sectsty}
\allsectionsfont{\sffamily\mdseries\upshape} % (See the fntguide.pdf for font help)
\usepackage{amsmath}
\usepackage{amsthm}
\usepackage{amsfonts}
\usepackage{mathrsfs}
\usepackage{amsopn}
\usepackage{amssymb}
\usepackage{natbib}
% (This matches ConTeXt defaults)

%%% ToC (table of contents) APPEARANCE
\usepackage[nottoc,notlof,notlot]{tocbibind} % Put the bibliography in the ToC
\usepackage[titles,subfigure]{tocloft} % Alter the style of the Table of Contents
\renewcommand{\cftsecfont}{\rmfamily\mdseries\upshape}
\renewcommand{\cftsecpagefont}{\rmfamily\mdseries\upshape} % No bold!

%Theorems and stuff
\newtheorem{defn}{Definition}
\newtheorem{thm}{Theorem}
\newtheorem{cor}{Corollary}
\newtheorem{lem}{Lemma}
\newtheorem{prop}{Proposition}
\theoremstyle{remark}\newtheorem*{rem}{Remark}

\newcommand{\cO}{{\cal O}}
\newcommand{\ra}{\rightarrow}
\newcommand{\NN}{{\mathbb N}}
\newcommand{\PP}{{\mathbb P}}
\newcommand{\ZZ}{{\mathbb Z}}
\newcommand{\cL}{{\cal L}}

\DeclareMathOperator{\ord}{ord}
\DeclareMathOperator{\di}{div}
\DeclareMathOperator{\cha}{char}
\DeclareMathOperator{\gal}{Gal}


%%% END Article customizations

%%% The "real" document content comes below...

\title{Template}
\author{J Tait}
%\date{} % Activate to display a given date or no date (if empty),
         % otherwise the current date is printed 

\begin{document}
\maketitle


 We use the paper {\bf reference} to give an alternative proof of {\bf ref} in the case that $G$ is a cyclic $p$-group of order $p^l$.

\begin{thm}
  If $G$ is a cyclic group of order $p^l$, then
\[\dim_k \left(H^0(X,\Omega_X)^G\right) = \left\{
\begin{array}{ll}
g_Y & \textrm{if } \textrm{deg} \left\lfloor \frac{\pi_*(R)}{n} \right\rfloor = 0\\
\\
g_Y-1 + \textrm{deg}\left\lfloor \frac{\pi_*(R)}{n} \right\rfloor &
\textrm{if } \textrm{deg}\left\lfloor \frac{\pi_*(R)}{n} \right\rfloor > 0.
\end{array}\right.\]
\end{thm}
\begin{proof}
Let $V:=H^0(X,\Omega_X)$. 
Then by {\bf ref} $V$ is a direct sum $V=\oplus_{i=1}^{p^l} V_i^{d_i}$, where $V_i$ are the irreducible submodules of $V$ with $k$ basis $v_1,\ldots ,v_i$.
If $\sigma$ is a generator of $G$ then we define the $G$ action on $V_m$ by $\sigma v_i=v_i + v_{i+1}$ if $1\leq i \leq m-1$ and by $\sigma v_m=v_m$ otherwise.
Clearly the part of each $V_i$ fixed by $G$ is one dimensional, so $\dim_k(V^G)=\sum_{i=1}^{p^l} d_i$.
Using the notation of {\bf ref} this is the same as $c(0,0)$, which is $g_Y-1+\Gamma_{0,0}+\Lambda_{0,0}$.
Here, $\Lambda_{0,0}$ is $1$ if $\Gamma_{0,0}=0$, and is zero otherwise.
Hence it suffices to show that $\Gamma_{0,0}=\deg \left\lfloor \frac{\pi_*(R)}{n} \right\rfloor$.

To this end, for each ramification point $P_i$ let $\delta_i$ be the exponent of the different, and let $e_i$ be the common ramification index above $P_i$.
Also note that, in the notation of {\bf ref}, $w_0$ is $1$, hence it's valuation is zero.
Now it is clear that in our case $\Gamma_{0,0}$ simplifies to $\sum_{i=1}^r\left\lfloor \frac{\delta_i}{e_i} \right\rfloor$, as desired.
$\Beta$
\end{proof}




\bibliography{/home/jtait/Documents/Maths/Bibliography/biblio.bib}
\bibliographystyle{plain}

\end{document}