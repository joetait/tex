% !TEX TS-program = pdflatex
% !TEX encoding = UTF-8 Unicode

% This is a simple template for a LaTeX document using the "article" class.
% See "book", "report", "letter" for other types of document.

\documentclass[11pt]{article} % use larger type; default would be 10pt

\usepackage[utf8]{inputenc} % set input encoding (not needed with XeLaTeX)

%%% Examples of Article customizations
% These packages are optional, depending whether you want the features they provide.
% See the LaTeX Companion or other references for full information.

%%% PAGE DIMENSIONS
\usepackage{geometry} % to change the page dimensions
\geometry{a4paper} % or letterpaper (US) or a5paper or....
% \geometry{margins=2in} % for example, change the margins to 2 inches all round
% \geometry{landscape} % set up the page for landscape
%   read geometry.pdf for detailed page layout information

\usepackage{graphicx} % support the \includegraphics command and options

\usepackage[parfill]{parskip} % Activate to begin paragraphs with an empty line rather than an indent

%%% PACKAGES
\usepackage{booktabs} % for much better looking tables
\usepackage{array} % for better arrays (eg matrices) in maths
\usepackage{paralist} % very flexible & customisable lists (eg. enumerate/itemize, etc.)
\usepackage{verbatim} % adds environment for commenting out blocks of text & for better verbatim
\usepackage{subfig} % make it possible to include more than one captioned figure/table in a single float
% These packages are all incorporated in the memoir class to one degree or another...

%%% HEADERS & FOOTERS
\usepackage{fancyhdr} % This should be set AFTER setting up the page geometry
\pagestyle{fancy} % options: empty , plain , fancy
\renewcommand{\headrulewidth}{0pt} % customise the layout...
\lhead{}\chead{}\rhead{}
\lfoot{}\cfoot{\thepage}\rfoot{}

%%% SECTION TITLE APPEARANCE
\usepackage{sectsty}
\allsectionsfont{\sffamily\mdseries\upshape} % (See the fntguide.pdf for font help)
\usepackage{amsmath}
\usepackage{amsthm}
\usepackage{amsfonts}
\usepackage{mathrsfs}
\usepackage{amsopn}
\usepackage{amssymb}
% (This matches ConTeXt defaults)

%%% ToC (table of contents) APPEARANCE
\usepackage[nottoc,notlof,notlot]{tocbibind} % Put the bibliography in the ToC
\usepackage[titles,subfigure]{tocloft} % Alter the style of the Table of Contents
\renewcommand{\cftsecfont}{\rmfamily\mdseries\upshape}
\renewcommand{\cftsecpagefont}{\rmfamily\mdseries\upshape} % No bold!

%Theorems and stuff
\newtheorem{defn}{Definition}
\newtheorem{thm}{Theorem}
\newtheorem{cor}{Corollary}
\newtheorem{lem}{Lemma}
\newtheorem{prop}{Proposition}
\newtheorem*{unnumthm}{Theorem}
\theoremstyle{remark}\newtheorem*{rem}{Remark}


\newcommand{\cO}{{\cal O}}
\newcommand{\ra}{\rightarrow}
\newcommand{\NN}{{\mathbb N}}
\newcommand{\PP}{{\mathbb P}}
\newcommand{\ZZ}{{\mathbb Z}}
\newcommand{\cL}{{\cal L}}

\DeclareMathOperator{\ord}{ord}
\DeclareMathOperator{\di}{div}
\DeclareMathOperator{\cha}{char}
\DeclareMathOperator{\gal}{Gal}

%%% END Article customizations

%%% The "real" document content comes below...

\title{Faithful Action on Spaces of Global Poly-Differentials on an Algebraic Curve}
\author{Bernhard K\"ock and Joseph Tait}
%\date{} % Activate to display a given date or no date (if empty),
         % otherwise the current date is printed 

\begin{document}
\maketitle

\begin{quote}
 

  {\bf Abstract} Given a faithful action of a finite group $G$ on an algebraic curve~$X$ of genus $g_X\geq 2$, 
we prove that the induced action on the space of global holomorphic poly-differentials of order $m$ is faithful as well 
unless $G$ contains a hyperelliptic involution and either both $m=1$ and the characteristic of the base field is $2$ or both $m=2$ and $g_X=2$.\\
\end{quote}

  \section{Introduction}

  Let $X$ be a connected smooth projective algebraic curve over
an algebraically closed field $k$ equipped with a faithful
action of a finite group $G$ of order $n$. Then $G$ also acts
on the vector space $H^0(X, \Omega_X)$ of global holomorphic
differentials on $X$. A widely studied problem is to determine
the structure of $H^0(X,\Omega_X)$ as module over the group
ring $k[G]$. It goes back to the Chevalley-Weil when $k = \mathbf{C}$,
see \cite{CW}. If the canonical projection $\pi: X \rightarrow
Y$ from $X$ to the quotient curve $Y = X/G$ is tamely ramified,
a fairly explicit answer to this problem has been given in 1986
by Kani in \cite{Ka}. For more recent answers to (related)
questions in more general situations the reader is referred to
the papers \cite{Bo} and \cite{FWK}. In the case of arbitrary
wild ramification the explicit calculation of the
$k[G]$-isomorphism class of $H^0(X,\Omega_X)$ is still an open
problem.

Here we look at the weaker question of whether $G$ acts faithfully on $H^0(X,\Omega_X)$ and also whether $G$ acts faithfully on the space $H^0(X,\Omega_X^{\otimes m})$ of global holomorphic poly-differentials of order $m$ (for $m\geq 2$).
Our answer to the question is the following theorem.


Let $p$ denote the characteristic of $k$ and let $g_X$ denote the genus of $X$.
We also recall that a hyperelliptic involution of $X$ is an automorphism $\sigma$ of $X$ of order $2$ such that the quotient curve $X/\langle \sigma \rangle$ is isomorphic to $\mathbb{P}_k^1$. \\
\newpage
  \begin{unnumthm}
    Suppose that $g_X\geq 2$ and let $m\geq1$. 
    Then $G$ does not act faithfully on $H^0(X,\Omega_X^{\otimes m})$ if and only if $G$ contains a hyper-elliptic involution and one of the following two sets of conditions holds:
      \begin{itemize}
	\item $m=1$ and $p=2$;
	\item $m=2$ and $g_X=2$.
      \end{itemize}
  \end{unnumthm}
To show this, we first compute the dimension of the fixed space of global holomorphic differentials and poly-differentials.
We then use this to find when a cyclic group of prime order acts trivially on the space.
Finally, in the proof of the main theorem we reduce the proof to the case where $G$ is cyclic of prime order, reducing it to an earlier problem.

We now give some further notation for the paper.
We write
  \[
      R=\sum_{P\in X}\delta_P[P]
  \]
for the ramification divisor of $\pi:X \rightarrow Y$.
Recall that
  \[
     \delta_P=\sum_{j=0}^{\infty}(\ord(G_j(P))-1),
  \]
where $G_j(P)$ is the $j^{th}$ ramification group at $P$ in lower notation, see \cite[Ch. IV, {\S}~1]{localfields}.
For any $P\in X$, let $e_P=\ord(G_0(P))$ denote the ramification index at $P$.
For any $Q\in Y$ we write $\delta_Q$ for $\delta_P$ and $e_Q$ for $e_P$ where $P\in \pi^{-1}(Q)$.
As usual, the sheaf of differentials on $X$ is denoted by $\Omega_X$ and it's $m^{th}$ tensor power by $\Omega_X^{\otimes m}$ for any $m\geq 1$.
Sections of $\Omega_X^{\otimes m}$ are called {\em poly-differentials of order $m$} and, if $m=2$, {\em quadratic differentials}.
We let $K_Y$ be a canonical divisor on $Y$. 
Then the divisor $K_X:=\pi^*(K_Y)+R$ is a $G$-invariant canonical divisor on $X$ by \cite[{\S} IV, Prop. 2.3]{hart} and $\mathscr{O}_X(mK_X)$ and $\Omega_X^{\otimes m}$ are isomorphic as $G$-sheaves.

  \section{Dimension Formulas}

In this section we compute the dimension of $H^0(X,\Omega_X^{\otimes m})$ and of $H^0(X,\Omega_X^{\otimes m})^G$, the subspace of $H^0(X,\Omega_X^{\otimes m})$ fixed by $G$.
In later sections, by comparing the two dimensions, we will ascertain when the action of $G$ on $H^0(X,\Omega_X^{\otimes m})$ is trivial.

By definition we have $\dim_kH^0(X,\Omega_X)=g_X$.
If $m\geq 2$, we have the following formula for $\dim_kH^0(X,\Omega_X^{\otimes m})$.\\


\begin{lem}\label{dim3}
  Let $m\geq 2$. Then
    \begin{equation}
      \dim_kH^0(X,\Omega_X^{\otimes m}) =
	\begin{cases}
	  0 & \mbox{if } g_X=0,\\
	  1 & \mbox{if } g_X=1,\\
	  (2m-1)(g_X-1) & \mbox{otherwise}.
	\end{cases}
    \end{equation}
\end{lem}
\begin{proof}
  If $g_X=0$ then $\deg(mK_X)=-2m<0$, and hence $\dim_kH^0(X,\Omega_X^{\otimes m})=0$.

  If $g_X=1$ then by \cite[Ch. IV, Example 1.3.6]{hart} we know that $K_X$ is equivalent to the zero divisor.
  Hence $mK_X$ is too and $\dim_kH^0(X,\Omega_X^{\otimes m})=1$.

  Finally, if $g_X\geq 2$ then $\deg(K_X)\geq1$, so $\deg(mK_X)>\deg(K_X)$, so by the Riemann-Roch theorem \cite[Ch. 8, {\S}6]{fulton} we see that $\dim_kH^0(X,\Omega_X^{\otimes m})=\deg(mK_X)+1-g_X=(2m-1)(g_X-1)$.
\end{proof}

We now introduce some notations. 
For any $G$-invariant divisor $D$ on $X$ let $\cO_X(D)$ denote the corresponding equivariant invertible $\cO_X$-module, as usual. 
Furthermore let $\pi_*^G(\cO_X(D))$ denote the sub-sheaf of the direct image $\pi_*(\cO_X(D))$ fixed by the obvious action of $G$ on $\pi_*(\cO_X(D))$ and let $\left\lfloor \frac{\pi_*(D)}{n}
\right \rfloor$ denote the divisor on $Y$ obtained from the push-forward $\pi_*(D)$ by replacing the coefficient $m_Q$ of $Q$ in $\pi_*(D)$ with the integral part $\left \lfloor \frac{m_Q}{n} \right \rfloor$ of $\frac{m_Q}{n}$ for every $Q \in Y$. 
The function fields of $X$ and $Y$ are denoted by $K(X)$ and $K(Y)$, respectively. 
Finally, for any $P \in X$ let $\textrm{ord}_P$ and $\textrm{ord}_Q$ denote the respective valuations of $K(X)$ and $K(Y)$ at $P$ and $Q:=\pi(P)$.



The next lemma is the main idea in the proof of our formula for $\dim_kH^0(X,\Omega_X^{\otimes m})^G$, see Proposition \ref{dim}. \\


  \begin{lem}
    Let $D$ be a $G$-invariant divisor on $X$.
    Then the sheaves $\pi_*^G(\cO_X(D))$ and $\cO_Y\left(\left\lfloor \frac{\pi_*(D)}{n}\right \rfloor\right)$ are equal as sub-sheaves of the constant sheaf $K(Y)$ on $Y$. 
    In particular the sheaf $\pi_*^G(\cO_X(D))$ is an invertible $\cO_Y$-module.
  \end{lem}
  \begin{proof}
    For every open subset $V$ of $Y$ we have 
      \[
	 \pi_*^G(\cO_X(D))(V) = \cO_X(D) (\pi^{-1}(V))^G \subseteq K(X)^G = K(Y).
      \]
    In particular both sheaves are sub-sheaves of the constant sheaf $K(Y)$ as stated. 
    It therefore suffices to check that their stalks are equal. 
    Let $Q \in Y$, let $P \in \pi^{-1}(Q)$ and let $n_P$ denote the coefficient of $D$ at $P$. 
    Then we have
      \begin{eqnarray*}
	 \lefteqn{\pi_*^G\left(\cO_X(D)\right)_Q = \cO_X(D)_P \cap K(Y)}\\
	  &=& \left\{f \in K(Y): \textrm{ord}_P(f) \ge -n_P\right\}\\
	  &=& \left\{f \in K(Y): \textrm{ord}_Q(f) \ge - \frac{n_P}{e_P}\right\}\\
	  &=& \left\{ f \in K(Y): \textrm{ord}_Q(f) \ge - \left\lfloor\frac{n_P}{e_P} \right\rfloor \right\}\\
	  &=& \cO_Y\left(\left\lfloor \frac{\pi_*(D)}{n} \right\rfloor\right)_Q,
      \end{eqnarray*}
    as desired.
  \end{proof}

The following proposition contains the aforementioned formula for the dimension of the subspace of $H^0(X,\Omega_X^{\otimes m})$ fixed by $G$.
In particular we see that the dimension is completely determined by $m$, $g_Y$ and $\deg \left\lfloor \frac{m\pi_*(R)}{n} \right\rfloor$.\\

  \begin{prop}\label{dim}
    Let $m\geq 1$. Then the dimension of $H^0(X,\Omega_X^{\otimes m})^G$ is equal to
	\begin{equation*}
	   \begin{cases}
	    g_Y & \mbox{if } m=1 \mbox{ and } \deg\left\lfloor\frac{m\pi_*(R)}{n}\right\rfloor = 0, \\
\\
	    1 & \mbox{if } g_Y=1 \mbox{ and } \deg\left\lfloor\frac{m\pi_*(R)}{n}\right\rfloor = 0, \\ 
\\
	    0 & \mbox{if } g_Y=0 \mbox{ and } \deg\left\lfloor\frac{m\pi_*(R)}{n}\right\rfloor < 2m, \\
\\
	    (2m-1)(g_Y-1) + \deg\left\lfloor\frac{m\pi_*(R)}{n} \right\rfloor & \mbox{otherwise}.
	  \end{cases}
      \end{equation*}
  \end{prop}
  \begin{proof}

    Let $E$ denote the divisor $\left\lfloor \frac{\pi_*(mK_X)}{n} \right\rfloor$ on $Y$. As $K_X=\pi^*(K_Y)+R$ we have
      \[ E = 
      \left \lfloor \frac{\pi_*\pi^*(mK_Y) + \pi_*(mR)}{n} \right \rfloor =
      mK_Y + \left \lfloor \frac{m\pi_*(R)}{n} \right \rfloor.\]
    Using the previous lemma we conclude that $\pi_*^G(\Omega_X^{\otimes m}) \cong \cO_Y (E)$ and finally that
      \begin{equation*}
	\textrm{dim}_k H^0(X,\Omega_X^{\otimes m})^G 
	 = \textrm{dim}_k H^0\left(Y, \pi_*^G(\Omega_X^{\otimes m})\right)
	  = \textrm{dim}_k H^0\left(Y, \cO_Y\left( E \right) \right).
      \end{equation*}


  In the first case of the proposition, i.e. if $m=1$ and $\deg \left\lfloor\frac{m\pi_*(R)}{n} \right\rfloor=0$, then $\left\lfloor\frac{m\pi_*(R)}{n} \right\rfloor$ is the zero divisor and we conclude that 
    \begin{equation*}
	\dim_kH^0(X,\Omega_X)^G = \dim_kH^0(Y, \Omega_Y) = g_Y
    \end{equation*}


  In the second case again $\left\lfloor \frac{m\pi_*(R)}{n} \right\rfloor$ is the zero divisor. 
  Furthermore, as $g_Y=1$, $K_Y$ is equivalent to the zero divisor, and hence $mK_Y$ is too. 
  This means that
    \begin{equation*}
      \dim_kH^0(X,\Omega_X^{\otimes m})^G = \dim_kH^0\left( Y,\cO_Y\left( E \right) \right) 
      = \dim_k  H^0\left( Y,\cO_Y\left( 0 \right) \right)
      = 1.
    \end{equation*}


  For the third case it suffices to show that $\deg \left( E \right) < 0$.
  As $g_Y=0$ we have $\deg(K_Y)=-2$, so $\deg(mK_Y)=-2m$, and $\deg \left( E \right)$ is indeed negative.



  We now show that in all other cases $\deg(E)\geq \deg(K_Y)$, and then the Riemann-Roch formula \cite[Ch. IV, {\S}1, Theorem 1.3]{hart} will give 
     \begin{eqnarray*}
	\lefteqn{\dim_kH^0(X,\Omega_X^{\otimes m})^G = \dim_kH^0\left(Y,\mathscr{O}_Y\left( E \right)\right)} \\
	& = & 1-g_Y+\deg\left(mK_Y+\left\lfloor{\frac{m\pi_*(R)}{n}}\right\rfloor\right) \\
	& = & (2m-1)(g_Y-1)+\deg\left\lfloor{\frac{m\pi_*(R)}{n}}\right\rfloor,
      \end{eqnarray*}
  completing the proof.


  All that remains is to show that $\deg(E)\geq \deg(K_Y)$ in all other cases.
  Firstly, if $g_Y=0$ and $\deg \left\lfloor\frac{m\pi_*(R)}{n} \right\rfloor \geq 2m$ then, as $\deg(mK_Y)=-2m$, we have $\deg \left( E \right) \geq 0 > \deg(K_Y)$.
  Similarly, if $g_Y=1$ and $\deg \left\lfloor\frac{m\pi_*(R)}{n} \right\rfloor >0$ then, as $\deg \left( mK_Y \right)=0$, we have $\deg \left( E \right) > 0 = \deg (K_Y)$.
  If $m=1$ and $\deg \left\lfloor\frac{m\pi_*(R)}{n} \right\rfloor >0$ then clearly $\deg \left( E \right) > \deg (K_Y)$.
  Lastly, if $m\geq 2$ and $g_Y\geq 2$ then $\deg (K_Y) > 0$ and we have 
    \begin{equation*}
      \deg \left( E \right) \geq \deg\left( mK_Y \right) > \deg (K_Y).
    \end{equation*}
  So in all other cases we have $\deg(E)\geq \deg(K_Y)$, and the proof is complete.
  \end{proof}


If $m=1$ we reformulate Proposition \ref{dim} in the following slightly more concrete way. 
Let $S$ denote the set of all points $Q\in Y$ such that $\pi$ is not tamely ramified at $Q$ and let $s$ denote the cardinality of $S$. 
Note that $s=0$ if $p$ does not divide $n$.\\

\begin{cor}\label{dim2}
  We have 
    \begin{eqnarray*}
      \dim_kH^0(X,\Omega_X)^G = 
	\begin{cases}
	  g_Y & \mbox{if } s=0, \\
	  g_Y-1+\sum_{Q\in S}\left\lfloor \frac{\delta_Q}{e_Q} \right\rfloor & \mbox{otherwise}.
	\end{cases}
    \end{eqnarray*}
\end{cor}
\begin{proof}
  We have
    \[
	\deg\left\lfloor\frac{\pi_*(R)}{n} \right\rfloor = \sum_{Q\in Y}\left\lfloor\sum_{P\mapsto Q} \frac{\delta_P}{n} \right\rfloor = \sum_{Q\in Y} \left\lfloor \frac{\delta_Q}{e_Q} \right\rfloor.
    \]
Furthermore we have $\left\lfloor \frac{\sigma_Q}{e_Q} \right\rfloor = 0$ if and only if $\sigma_Q<e_Q$, i.e. if and only $Q\in S$. 
Thus Corollary \ref{dim2} follows from Proposition \ref{dim}.
\end{proof}

\begin{rem}
  Note that if $p>0$ and $G$ is cyclic then Corollary \ref{dim2} can be derived from proposition 6 in the recent pre-print
 \cite{kako} by Karanikolopoulos and Kontogeorgis.
\end{rem}



  \section{Trivial action in the cyclic case}

   In this section, rather than looking at a general group $G$, we determine when specific classes of cyclic groups will act trivially on $\dim_kH^0(X,\Omega_X^{\otimes m})$.
   By limiting ourselves to these groups this question will be a lot easier, and then in the final section we will show how to reduce the general situation to these cases.\\

  In this section, $P_1,\ldots ,P_r \in X$ denote the ramification points of $\pi$ and we write $e_i$ and $\delta_i$ for $e_{P_i}$ and $\delta_P{_i}$.
  Also, for $i=1, \ldots, r$, we define $N_i \in \NN$ by $\textrm{ord}_{P_i}(\delta(\pi_i) - \pi_i) = N_i +1$ where $\pi_i$ is a local parameter at the ramification point $P_i$ and $\sigma$ is a generator of $G(P_i)$.\\ 


  \begin{prop}\label{m=1}
    Let $p  > 0$ and let $G$ be cyclic of order $p$.
    Furthermore, we assume that $g_Y=0$.
    Then $G$ acts trivially on $H^0(X,\Omega_X)$ if and only if one of the following three conditions holds:
	 \begin{itemize}
	  \item $g_X=0$;
	  \item $g_X=1$ and $p=3$;
	  \item $p=2$.
	 \end{itemize}
  \end{prop}
  \begin{proof}
    From Lemma~1 on p.~87 in \cite{Naka} we know that $p$ does not divide $N_i$, a fact we will use several times below. 
    The ramification divisor $R$ of $\pi$ is equal to $\sum_{i=1}^r(N_i+1)(p-1)[P_i]$ by Hilbert's formula for the order of the different (see Prop.~4, \S 1, Ch.~IV on p.~72 in \cite{localfields}). 
    Let $N:= \sum_{i=1}^r N_i$. 
    Recall that the Hurwitz formula states
      \begin{equation}\label{hur}
	2(g_X-1)=2n(g_Y-1) + \deg(R).
      \end{equation}
    Using this we obtain
      \begin{equation}\label{hur2}
	2g_X - 2 = -2p + (N+r)(p-1)
      \end{equation}
    and hence
      \[
	\textrm{dim}_kH^0(X,\Omega_X) = g_X =\frac{(N+r-2)(p-1)}{2}.
      \] 
    Since $g_X \ge 0$ we obtain $r \ge 1$; that is, $\pi$ is not unramified. 
    Therefore we have 
      \[
	\textrm{deg} \left\lfloor \frac{\pi_*(R)}{p} \right\rfloor =
	\sum_{i=1}^r \left\lfloor \frac{(N_i+1)(p-1)}{p}\right\rfloor 
	\ge \sum_{i=1}^r \left\lfloor \frac{2(p-1)}{p}\right\rfloor = r > 0.
      \] 
    From Proposition~1 we then conclude that
      \begin{eqnarray*}
	\lefteqn{\textrm{dim}_k H^0(X,\Omega_X)^G = g_Y -1 + \textrm{deg} \left\lfloor \frac{\pi_*(R)}{p} \right\rfloor}\\
	&=& -1 + \sum_{i=1}^r \left\lfloor \frac{(N_i+1)(p-1)}{p}\right\rfloor\\
	&=& -1 + N +r + \sum_{i=1}^r \left\lfloor -\frac{N_i+1}{p}\right\rfloor.
      \end{eqnarray*}
    If $p=2$ the dimension of both $H^0(X,\Omega_X)$ and $H^0(X,\Omega_X)^G$ is therefore equal to $\frac{N+r-2}{2}$. 
    If $g_X = 0$ both dimensions are obviously equal to~$0$.
    If $p=3$ and $g_X =1$ we obtain $N+r=3$ and hence $r=1$ and $N=2$; thus both dimensions are equal to $1$.
    Therefore in all three of these cases $G$ acts trivially on $H^0(X,\Omega_X)$.
    This finishes the proof of the if-direction in Proposition~2.\\
    To prove the other direction we now assume that $G$ acts trivially on $H^0(X, \Omega_X)$ and that $p \ge 3$ and prove that the first or second condition holds. 
    For each $i=1, \ldots, r$, we write $N_i = s_i p +t_i$ with $s_i \in \NN$ and $t_i \in \{1, \ldots, p-1\}$. 
    We furthermore put $S:=\sum_{i=1}^r s_i$ and $T:= \sum_{i=1}^r t_i \ge r$. 
    Then we have
      \[ 
	 \frac{(N+r-2)(p-1)}{2} =\textrm{dim}_k(H^0(X,\Omega_X))  = \textrm{dim}_k\left(H^0(X,\Omega_X)^G\right) = N-S-1 .
      \]
    Rearranging this equation we obtain
      \[
	 (3-p)N - 2S = (r-2)(p-1) +2  
      \]
    and hence
      \[
	 (-p^2 + 3p -2)S = (r-2)(p-1) +2 - (3-p)T.
      \]
    Since $-p^2+3p-2 = - (p-1)(p-2)$ and $p \ge 3$ this equation implies that
      \[ 
	S = \frac{(r-2)(1-p)-2 + T (3-p)}{(p-1)(p-2)}. 
      \]
    Because $S \ge 0$ the numerator of this fraction is non-negative, that is
      \begin{eqnarray*}
	\lefteqn{0 \le (r-2)(1-p) - 2 + T (3-p)}\\
	&\le & (r-2)(1-p) - 2 + r (3-p)\\
	&=& 2 (r-1)(2-p).
      \end{eqnarray*}
    Hence we have $r=1$ and that numerator is $0$. 
    We conclude that $S=0$ and hence that $T=1$ or $p=3$. 
    If $T=1$ we also have $N=1$ and finally
      \[
	g_X = \frac{(N+r-2)(p-1)}{2} = 0,
      \]
    i.e. the first condition holds. 
    If $T \not=1$ and $p=3$ we obtain $N=T=2$ and finally 
      \[
	g_X = \frac{(N+r-2)(p-1)}{2} =1,
      \] 
    i.e.\
    the second condition holds.
  \end{proof}

  \begin{prop}\label{triv}
    Let $m \geq 2$. 
    Suppose that $G$ is a cyclic group of prime order $l$ (which may or may not be equal to $\cha(k)=p$) and that $g_Y=0$. 
    Then $G$ acts trivially on $H^0(X,\Omega_X^{\otimes m})$ if and only if one of the following conditions holds:
      \begin{itemize}
	\item $g_X=0$;
	\item $g_X=1$;
	\item $g_X=m=l=2$.
      \end{itemize}
  \end{prop}
\begin{proof}
    If $g_X=0$ then $\deg(mK_X)=-2m<0$; hence $\dim_k(H^0(X,\Omega_X^{\otimes m})=0$ by \cite[{\S} 8, Prop. 3]{fulton} and the action must be trivial.

    We now look at the case when $g_X=1$. By \cite[Chap. IV,\ Example 1.3.6]{hart} we know that $K_X$ is equivalent to the zero divisor. 
    Hence 
      \[
	H^0(X,\Omega_X^{\otimes m})\cong H^0(X,\mathscr{O}_X(mK_X)) \cong H^0(X,\mathscr{O}_X(0))=k,
      \]
    the space of constant functions, which is $G$ invariant. 
    Hence the action of $G$ on the space $H^0(X,\Omega_X^{\otimes m})$ is trivial.

    From now on we assume that $g_X\geq 2$. 
    We have different proofs according to whether or not the order $l$ of the group is the same as the characteristic $p$ of the field.


	First we assume that $l=p$. 
	Due to (\ref{hur2}) we have $2g_X-2=-2p+k(p-1)$, and combining this with Lemma \ref{dim3} we can write
		\[
		\dim_kH^0(X,\Omega_X^{\otimes m})=(2m-1)(g_X-1)=(2m-1)\Big(\frac{k(p-1)-2p}{2}\Big).
		\]

	Now if $\deg\left \lfloor{\frac{m\pi_*(R)}{p}}\right\rfloor\geq 2m$ then by Proposition 2 we have
		\begin{eqnarray*}	
			\lefteqn{\dim_kH^0(X,\Omega_X^{\otimes m})^G = 1-2m+\deg\Big{\lfloor}{\frac{m\pi_*(R)}{p}}\Big{\rfloor}} \\
			&= & 1-2m+\sum_{i=1}^r \Big{\lfloor}{\frac{m(N_i+1)(p-1)}{p}}\Big{\rfloor} \\
			&= & 1-2m+mk+\sum_{i=1}^r\Big{\lfloor}{\frac{-m(N_i+1)}{p}}\Big{\rfloor}. 
		\end{eqnarray*}
	
	If we have $p=g_X=m=2$ then on the one hand we see that $\dim_kH^0(X,\Omega_X^{\otimes m})=3$. 
	On the other hand, we first note that the equation $2g_X-2=-2p+k(p-1)$ implies that $k=6$. 
	So $\deg\left\lfloor \frac{m\pi_*(R)}{p}\right\rfloor=mk+\sum_{i=1}^r\Big{\lfloor}{\frac{-m(N_i+1)}{p}}\Big{\rfloor}=2k-k>2m$. 	
	Then we can compute $\dim_kH^0(X,\Omega_X^{\otimes m})^G=9+\sum\lfloor-N_i-1\rfloor=3$. 
	So the dimensions are equal and the action of $G$ on $H^0(X,\Omega_X^{\otimes m})$ is trivial. 
	This completes the if direction of the proof.

	Now we assume that the action is trivial. This first implies that 
	$\deg \left\lfloor\frac{m\pi_*(R)}{n}\right\rfloor \geq 2m$ because otherwise we would 
	have $\dim_k(H^0(X,\Omega_X^{\otimes m})^G)=0$ by the previous proposition, but we know that 
	$\dim_k(H^0(X,\Omega_X^{\otimes m}))=(2m-1)(g_X-1)$ (6) is strictly positive. So we have:
		\begin{eqnarray}
			\lefteqn{(2m-1)\frac{k(p-1)-2p}{2} = \dim_k(H^0(X,\Omega_X^{\otimes m}))} \nonumber\\
			& = & \dim_k(H^0(X,\Omega_X^{\otimes m})^G) \nonumber\\
			& = & 1-2m+mk+\sum_{i=1}^r\Big{\lfloor}\frac{-m(N_i+1)}{p}\Big{\rfloor}\nonumber \\
			& \leq & 1-2m+mk+\sum_{i=1}^r\frac{-m(N_i+1)}{p}\nonumber \\
			& = & 1-2m+mk-\frac{mk}{p}.
		\end{eqnarray}

	If we then multiply by $2p$ we obtain
		\begin{equation}
			(2mk-k-4m+2)p^2+(-4mk+k-2+4m)p+2mk\leq 0.
		\end{equation}

	The left hand side of this then factorises as
		\begin{multline*}
			(k-2)(2m-1)p^2-((k-2)(2m-1)+2mk)p+2mk = \\
			(p-1)((k-2)(2m-1)p-2mk)
		\end{multline*}

	As (\ref{hur2}) tells us that $-2p+k(p-1)=2g_X-2 \geq 2$ then we see that 
		\begin{equation}
			k\geq \frac{2+2p}{p-1}=2+\frac{4}{p-1}>2.
		\end{equation}

	So from (7) and (8) we see that
		\begin{eqnarray}
			p & \leq & \frac{2mk}{(k-2)(2m-1)}\nonumber\\
			& = & \frac{k}{k-2}\cdot\frac{2m}{2m-1}\nonumber\\
			& = & \Big( 1+\frac{2}{k-2} \Big) \Big(1+\frac{1}{2m-1} \Big)\\
			& \leq & 4, \nonumber	
		\end{eqnarray}

	i.e. $p=2$ or $p=3$. 

	Suppose that $p=3$. Then from (8) we have $k\geq 4$. However, from (10) and (9) we also have that 
		\begin{eqnarray*}
			3 & \leq &\Big( 1+\frac{2}{k-2} \Big) \Big(1+\frac{1}{2m-1} \Big)\\
			& \leq & \Big( 1+\frac{2}{k-2} \Big) \frac{4}{3}\\
			& \leq & \frac{8}{3},
		\end{eqnarray*}

	a contradiction.

	Lastly, we come to the case when $p=2$. From (9) we see that $2\leq \Big(1+\frac{2}{k-2}\Big)\frac{4}{3}$ 
	and hence $k\leq 6$. However, from (8) we also see that $k\geq 6$, so $k=6$. Then from (6) $2m-1=1-2m+6m-3m$
	and $m=2$. Finally, (\ref{hur2}) gives us that $2g_X-2=-4+6=2$ and hence $g_X=2$. 
	This completes the only if direction of the proof when $l=p$.

    Now if $l\neq p$ then we know that all the coefficients of the ramification divisor are $l-1$. 
    To show the if direction in this case, first note that from Lemma \ref{dim3} we see that $\dim_kH^0(X,\Omega_X^{\otimes m})=3$. 
    On the other hand, the Hurwitz formula, $2g_X-2=-2l+\deg(R)=-2l+r(l-1)$, implies that $r=6$. 
    Finally Proposition \ref{dim} gives us
      \begin{equation*}
	\dim_kH^0(X,\Omega_X^{\otimes m})^G = -(2m-1) + \sum_{i=1}^r \left\lfloor \frac{m\cdot n_i}{l} \right\rfloor
	= -3 +\sum_{i=1}^6 \left\lfloor \frac{m(l-1)}{l} \right\rfloor
	= 3
      \end{equation*}
    since $m=l=2$.
    As the dimensions of $H^0(X,\Omega_X^{\otimes m})$ and $H^0(X,\Omega_X^{\otimes m})^G$ are equal the action is trivial.


    Now, for the final section of the proof, we still assume that $g_X\geq 2$, and furthermore suppose that $G$ acts trivially on $H^0(X,\Omega_X^{\otimes m})$. 
    We will show that, given that $g_X\geq 2$, the only possibility is that $g_X=2$, which occurs when $l=2$.
    Hence we will have proven that $g_X=l=m=2$, as desired.
    From Lemma \ref{dim3} and Proposition~\ref{dim} we obtain
      \begin{eqnarray*}
	\lefteqn{(2m-1)(g_X-1)=\dim_kH^0(X,\Omega_X^{\otimes m})} \\
	& = & \dim_kH^0(X,\Omega_X^{\otimes m})^G=-(2m-1)+\sum_{i=1}^r \left\lfloor \frac{m\cdot n_i}{l} \right\rfloor
      \end{eqnarray*}
    and hence
      \begin{equation*}
	(2m-1)g_X = \sum_{i=1}^r \left\lfloor \frac{m\cdot n_i}{l} \right\rfloor
	= \sum_{i=1}^r \left\lfloor \frac{m(l-1)}{l} \right\rfloor
	= r\left( m+\left\lfloor \frac{-m}{l} \right\rfloor \right).
      \end{equation*}
    By choosing $s\in \{1,\ldots ,l\}$ and $q\in \mathbb{N}$ such that $m=ql+s$ we can rewrite this as
      \begin{equation}\label{eq:mult}
	(2m-1)g_X=r(m-q-1).
      \end{equation}
    If we multiply (\ref{eq:mult}) by $l-1$ and then substitute in for the $r(l-1)$ term in the Hurwitz formula we get
      \begin{equation*}
	(2m-1)(l-1)g_X=(2g_X+2(l-1))(m-q-1).
      \end{equation*}
    By rearranging we are able to compute $g_X$ in terms of $m,l$ and $q$:
      \begin{eqnarray*}
	\lefteqn{g_X = \frac{2(l-1)(m-q-1)}{(2m-1)(l-1)-2(m-q-1)}} \\
	& = & 1 + \frac{2(m-q-1)-(2q+1)(l-1)}{(2m-1)(l-1)-2(m-q-1)} \\
	& = & 1 + \frac{2s-1-l}{(2m-1)(l-1)-2(m-q-1)}  \\
	& = & 1 + \frac{2(s-1)+1-l}{(2m-1-2q)(l-1)-2(s-1)}. 
      \end{eqnarray*}
    First, we show that if $l\geq 3$ the equation cannot hold whilst $g_X\geq 2$.
    Observe that the denominator is strictly greater than $l-1$, remembering that $m=ql+s$:
      \begin{eqnarray*}
	(2m-1-2q)(l-1)-2(s-1) & = & ((2q(l-1)+2s-1)(l-1)-2(s-1) \\
	& \geq & (2s-1)(l-1)-2(s-1) \\
	& \geq & (2s-1)(l-1)-2(s-1)(l-1) \\
	& = & l-1;
      \end{eqnarray*}
    here the two inequalities are equalities if and only if $q=0$ and $s=1$, respectively, and, as $m\geq 2$, not both equalities can be equalities.
    Now the numerator is at most $l-1$, occurring when $s=l$. 
    Hence if $l\geq 3$ the fraction will be less than one and $g_X < 2$, which we have already looked at.
    Now if $l=2$, then $s$ is either 1 or 2.
    If $s=1$ the fraction is negative, and $g_X<1$, which again we have already considered.
    Finally, if $s=2$ then $g_X\leq 2$, with equality if and only if $q=0$ i.e. if $m=2$.
    So if $g_X \geq 2$ then the action being trivial implies that $g_X=l=m=2$, and the proof is complete.    
  \end{proof}

For the rest of this section we assume that $p>0$ and that $G$ is a cyclic group of order $p^l$ for some $l$.
What we are now going to do will not be used in the final section, but is included because it generalises the previous results.
More precisely, we do not restrict ourselves to looking at $H^0(X,\Omega_X^{\otimes m})$, but using \cite{kako} we study $H^0(X,\mathscr{O}(D))$ for any $G$-invariant divisor $D$ such that $\deg(D)>2g_X-2$.


We first introduce some notation.
Let $D = \sum_{P\in X} n_P[P]$ be a $G$-invariant divisor on $X$.
Then let $\langle a \rangle$ denote the fractional part of any $a\in \mathbb{R}$, i.e. $\langle a \rangle = a - \lfloor a \rfloor$.
Also, for any $Q\in Y$ let $n_Q$ be equal to $n_P$ for any $P\in \pi^{-1}(Q)$.\\



  \begin{prop}\label{nakaj}
    Suppose $p>0$ and $G$ is a cyclic group of order $p^l$ for some $l$.
    Let $D$ be a $G$-invariant divisor on $X$ such that $\deg(D)>2g_X-2$.
    Then the action of $G$ on $H^0(X,\mathscr{O}_X(D))$ is trivial if and only if
      \[ 
	(p^l-1)\deg(D)=p^l\left(g_X-g_Y-\sum_{Q\in Y}\left\langle \frac{n_Q}{e_Q} \right\rangle\right).
      \]
  \end{prop}
  \begin{proof}
We first remind the reader of the notation in \cite{kako}.
Let $\sigma$ be a generator of $G$.
Let $V$ be the $k[G]$ module with $k$-basis $e_1,\ldots ,e_{p^l}$ and $G$-action defined by $\sigma\cdot e_i=e_i+e_{i-1}$, $1\leq i \leq p^l,\ e_0=0$.
Then $V_j$, defined to be the subspace of $V$ spanned by $e_1,\ldots ,e_j$ over $k$, is also a $k[G]$ module.
In fact, the modules $V_1,\ldots ,V_{p^l}$ form a complete set of representatives for the set of isomorphism classes of indecomposable $k[G]$-modules. For each $j=1,\ldots,p^l$ let $m_j$ denote the multiplicity of $V_j$ in the $k[G]$-module $H^0(X,\cO_x(D))$, i.e. we have $H^0(X,\cO_x(D))\cong \oplus_{j=1}^{p^l}m_jV_j$.



    First note that the action of $G$ on $H^0(X,\cO_X(D))$ is trivial if and only if
      \begin{equation}\label{triva}
	\dim_k H^0(X,\mathscr{O}_X(D))^G =\dim_k H^0(X,\mathscr{O}_X(D)).
      \end{equation}
	
    It is clear that the $G$-invariant part of each sub-module $V_j$ is spanned by $e_1$. 
    Hence $\dim_kH^0(X,\mathscr{O}_X(D))^G = \sum_{j=1}^{p^l} m_j$.
    By \cite[Theorem 2.1]{quaddiffequi}, which relies on \cite{Bo}, we have
      \begin{eqnarray*}
	\sum_{j=1}^{p^l} m_j & = & 1- g_Y +\sum_{Q\in Y} \left\lfloor \frac{n_Q}{e_Q}\right\rfloor\\
	& = & 1- g_Y + \sum_{Q\in Y} \left( \frac{n_Q}{e_Q} - \left\langle \frac{n_Q}{e_Q}\right\rangle \right) \\
  	& = & 1 - g_Y + \frac{1}{p^l}\deg(D) - \sum_{Q\in Y} \left\langle \frac{n_Q}{e_Q} \right\rangle.
      \end{eqnarray*}

    Now as $\deg(D)>2g_X-2$ we have $\dim_kH^0(X,\mathscr{O}_X(D)) =\deg(D)+1-g_X$ by the Riemann-Roch theorem. 
    So the action of $G$ on $H^0(X,\mathscr{O}_X(D))$ is trivial if and only if
      \begin{equation*}
	\deg(D)+1-g_X  = 1 - g_Y + \frac{1}{p^l}\deg(D) - \sum_{Q\in Y}\left\langle \frac{n_Q}{e_Q} \right\rangle. \label{hi}
      \end{equation*}

    This then rearranges to $(p^l-1)\deg(D)=p^l\left(g_X-g_Y-\sum_{Q\in Y}\left\langle \frac{n_Q}{e_Q} \right\rangle\right)$, as desired.
    \end{proof}

  \begin{cor}\label{this}
    Suppose that $\deg(D)\geq 2g_X$. Then the action of $G$ on $H^0(X,\mathscr{O}_X(D))$ is trivial if and 
    only if $e_Q | n_Q$ for all $Q\in Y$, $\deg(D)=2g_X$ and either $g_X=0$ or $p^l=2$.
  \end{cor}
  \begin{proof}
    The following inequalities always hold under the stated assumptions:
      \[
	(p^l-1)\deg(D)\geq (p^l-1)2g_X \geq p^lg_X \geq p^lg_X-p^l\sum_{Q\in Y}\left\langle\frac{n_Q}{e_Q}\right\rangle.
      \]
    Now the first inequality is an equality if and only if $\deg(D)=2g_X$. 
    The second is an equality if and only if either $g_X=0$ or $p^l=2$. 
    Lastly, the third inequality is an equality if and only if $\sum_{Q\in Y}\left\langle\frac{n_Q}{e_Q}\right\rangle=0$, which is the case if and only if each $n_Q$ is divisible by~$e_Q$. 
    Given these observations, Proposition \ref{nakaj} implies Corollary~\ref{this}.
  \end{proof}
    
The following Corollary is stated for completeness sake.\\

  \begin{cor}
    Suppose that $\deg(D)= 2g_X-1$. Then the action of $G$ on $H^0(X,\mathscr{O}_X(D))$ is trivial if and only if one of the following conditions hold:
      \begin{itemize}
	\item $g_X=0$;
	\item  $p^l=2$ and $\sum_{Q\in Y}\left\langle\frac{n_Q}{e_Q}\right\rangle=\frac{1}{2}$;
	\item  $g_X=1$ and $\sum_{Q\in Y}\left\langle\frac{n_Q}{e_Q}\right\rangle=\frac{1}{p^l}$;
	\item  $g_X=2$, $p^l=3$ and $e_Q\mid n_Q$ for all $Q\in Y$.
      \end{itemize}
  \end{cor}~


  \begin{rem}
    It can easily be shown that in the last case the Hurwitz formula implies that $r\leq 4$. 
    Furthermore, if $r=1$ then the conditions ``$\sum_{Q\in Y}\left\langle\frac{n_Q}{e_Q}\right\rangle=\frac{1}{p^l}$" and ``$e_Q\mid n_Q$ for all $Q\in Y$" are already implied by ``$\deg(D)=2g_X-1$".
  \end{rem}

  \begin{proof}
    Firstly, if $g_X=0$ then $\deg(D)=-1<0$, so $\dim_kH^0(X,\mathscr{O}_X(D))=0$ and the action is trivial.

    Now note that, as $\deg(D)=2g_X-1$, we conclude from Proposition \ref{nakaj} that the action is trivial if and only if 
      \begin{equation*}
	(p^l-1)(2g_X-1)=p^l\left(g_X-\sum_{Q\in Y}\left\langle\frac{n_Q}{e_Q}\right\rangle\right).
      \end{equation*}
    If $p^l=2$ then this is equivalent to $2g_X-1=2g_X-2\sum_{Q\in Y}\left\langle\frac{n_Q}{e_Q}\right\rangle$ and hence to $\sum_{Q\in Y}\left\langle\frac{n_Q}{e_Q}\right\rangle=\frac{1}{2}$.

    If $g_X=1$ then this is equivalent to $p^l-1=p^l-p^l\sum_{Q\in Y}\left\langle\frac{n_Q}{e_Q}\right\rangle$ and hence also to $\sum_{Q\in Y}\left\langle\frac{n_Q}{e_Q}\right\rangle=\frac{1}{p^l}$.

    Lastly, if $p^l\geq 3$ and $g_X\geq 2$ then we have that $g_X\geq \frac{p^l-1}{p^l-2}$ which is equivalent to the first inequality in the chain
      \begin{equation*}
	(p^l-1)(2g_X-1)\geq p^lg_X\geq p^lg_X-p\sum_{Q\in Y}\left\langle\frac{n_Q}{e_Q}\right\rangle.
      \end{equation*}
    Hence the action is trivial if and only if both inequalities are equalities, which is the case if and only if $p^l=3,\ g_X=2$ and $e_Q\mid n_Q$ for all $Q\in Y$.
  \end{proof}

The following Corollary generalises the only-if direction of the $l=p$ part of the proof of of Proposition $3$.\\

  \begin{cor}
    Let $m\geq 2$. 
    We assume that $g_X\geq 2$ and that $G$ is a cyclic group of order~$p^l$. 
    If $G$ acts trivially on $H^0(X,\Omega_X^{\otimes m})$, then $p^l=g_X=m=2$   .
  \end{cor}
  \begin{proof}
    As $g_X\geq 2$ and $m\geq 2$ we have that $\deg(mK_X)\geq 2g_X$. 
    So by Corollary \ref{this}, the action is trivial if and only if $p^l=2,\ \deg(mK_X)=2g_X$ and $e_Q | n_Q$ for all $Q\in Y$. 
    Now $\deg(mK_X)=2g_X$ means that $m(2g_X-2)=2g_X$, so $m(g_X-1)=g_X$, and hence that $m=g_X=2$. 
  \end{proof}

  \section{The Main Theorem}
  In this section we prove the main theorem of the paper, describing exactly when $G$ will act faithfully on $H^0(X,\Omega_X^{\otimes m})$.


  We begin with some examples when the action of $G$ on $H^0(X, \Omega_X^{\otimes m})$ is in fact trivial.

{\bf Example.}\\
  (a) If $g_X = 0$ then $\deg(K_X)=-2$ and so $\deg(mK_X)<0$ for $m\geq1$. 
  Hence  $H^0(X,\Omega_X^{\otimes m}) = \{0\}$ and $G$ acts trivially for all $m\geq 1$.

  
  (b) Suppose $g_X =1 $ (i.e. suppose $X$ is an elliptic curve) and that $G$ is a finite subgroup of $X(k)$ acting on $X$ by translations.
  Then $G$ leaves invariant any global non-vanishing holomorphic differential and hence $G$ acts trivially on $H^0(X,\Omega_X)$.
  Since $\Omega_X\cong \Omega_X^{\otimes m}$ for $m\geq1$ when $g_X=1$, this means that $G$ also acts trivially on  $H^0(X,\Omega_X^{\otimes m})$.


  (c) Let $p=2$. 
  If $n=2$ and $g_Y =0$, then $G \cong \ZZ/2\ZZ$ acts trivially on $H^0(X, \Omega_X)$ by Proposition~\ref{m=1}.
  For instance, let $r$ be an odd natural number, let $k(x,y)$ be the extension of the rational function field $k(x)$ given by the Artin-Schreier equation $y^2-y = x^r$ and let $\pi: X \rightarrow \PP^1_k$ be the corresponding cover of non-singular curves over $k$; 
  then $G$ acts trivially on the vector space $H^0(X,\Omega_X)$ whose dimension is $\frac{r-1}{2}$ by Example~2.5 on p.~1095 in \cite{galoisstruc}.

  (d) Let $k(x,y)$ be the extension of the rational function field $k(x)$ given by $y^2=(x-x_1)\ldots (x-x_6)$, where $x_1, \ldots , x_6$ are pairwise distinct.
      Then the corresponding natural projection $\pi:X\rightarrow \mathbb{P}_k^1$ is of degree $2$ and ramified exactly over $x_1,\ldots , x_6\in \mathbb{P}_k^1$.
      Now by Lemma \ref{dim3} and Proposition \ref{dim} respectively $\dim_kH^0(X,\Omega_X^{\otimes 2})$ and $\dim_kH^0(X,\Omega_X^{\otimes 2})^G$ are both $3$, so the action is trivial.\\

  \begin{unnumthm}
    Suppose that $g_X\geq 2$ and let $m\geq1$. 
    Then $G$ does not act faithfully on $H^0(X,\Omega_X^{\otimes m})$ if and only if $G$ contains a hyper-elliptic involution and one of the following conditions holds:
      \begin{itemize}
	\item $m=1$ and $p=2$;
	\item $m=2$ and $g_X=2$.
      \end{itemize}
  \end{unnumthm}
  \begin{proof}
    We first show the if direction. 
    In the case when $m=1$, the hyper-elliptic involution contained in $G$ generates a subgroup of order $2$.
    Since $p=2$, this acts trivially by Proposition \ref{m=1}, and hence $G$ does not act faithfully.
    In the case when $m=2$, then again looking at the subgroup generated by the hyper-elliptic involution, we have a group of order $2$ acting on $H^0(X,\Omega_X^{\otimes m})$.
    So, by Proposition \ref{triv} and since $g_X=m=2$, the action of this subgroup is trivial, and again, this means that $G$ does not act faithfully.


    We now start the proof of the only if direction, supposing that $G$ does not act faithfully on $H^0(X,\Omega_X^{\otimes m})$. 
    By replacing $G$ with the (non-trivial) kernel $H$ if necessary, we may assume that $G$ is non-trivial and acts trivially on $H^0(X,\Omega_X^{\otimes m})$.


    We start the proof by showing that $g_Y=0$, which is shown separately for the cases when $m=1$ and when $m\geq 2$.
    In the case when $m=1$ we start by showing that $\deg  \left\lfloor \frac {\pi_*(R)}{n} \right\rfloor >0$ by contradiction.
    Suppose that $\deg\left\lfloor \frac{\pi_*(R)}{n} \right\rfloor =0$.
    As $G$ acts trivially then by Proposition~\ref{dim} we have:
      \begin{equation*}
	g_X=\dim_k H^0(X,\Omega_X)=\dim_k H^0(X,\Omega_X)^G=g_Y.
      \end{equation*}
    Substituting this into the Hurwitz formula (\ref{hur}) yields the desired contradiction because $g_X\geq 2, n\geq 2$ and $\deg(R)\geq 0$.

    Thus $\deg\left( \left\lfloor \frac{\pi_*(R)}{n} \right\rfloor \right) >0$. 
    Now Proposition~\ref{dim} gives us that
      \begin{equation*}
	g_X=\dim_k H^0(X,\Omega_X)=\dim_k H^0(X,\Omega_X)^G= g_Y-1+\deg\left\lfloor \frac{\pi_*(R)}{n} \right\rfloor.
      \end{equation*}
    Substituting this in to the Hurwitz formula (\ref{hur}), we see that
      \begin{equation*}
	2\left(g_Y - 1 + \textrm{deg}\left \lfloor \frac{\pi_*(R)}{n} \right \rfloor -1 \right) = 2n (g_Y -1) + \textrm{deg}(R).
      \end{equation*}
    For any $Q \in Y$ let $n_Q$ denote the coefficient of the ramification divisor $R$ at any $P \in \pi^{-1}(Q)$ and let $e_Q := e_P$ for any $P \in \pi^{-1}(Q)$. 
    Rewriting the previous equation yields
      \begin{eqnarray*}
	\lefteqn{(2n-2)g_Y = 2n-4 + 2 \,\textrm{deg}\left \lfloor \frac{\pi_*(R)}{n}\right \rfloor - \textrm{deg}(R)}\\
	&=& 2 \left(n-2 + \sum_{Q \in Y} \left(\left\lfloor \frac{n}{e_Q} \frac{n_Q}{n} \right\rfloor - \frac{n}{e_Q} \frac{n_Q}{2}\right) \right)\\
	&=& 2 \left(n-2 + \sum_{Q \in Y} \left( \left\lfloor \frac{n_Q}{e_Q} \right\rfloor - \frac{n_Q}{e_Q} \frac{n}{2} \right)\right)\\
	& \le & 2(n-2)
      \end{eqnarray*}
    because $\frac{n}{2} \ge 1$ and $\left\lfloor \frac{n_Q}{e_Q}\right\rfloor \le \frac{n_Q}{e_Q}$ for all $Q \in Y$. 
    Hence we obtain $g_Y \le \frac{n-2}{n-1} < 1$ and therefore $g_Y =0$, as desired.

    We now show that $g_Y=0$ when $m\geq 2$. 
    Since $g_X\geq 2$ we have that $\deg(mK_X)=m(2g_X-2)>2g_X-2=\deg(K_X)$.
    By Lemma \ref{dim3}, and as both $m$ and $g_X$ are at least 2, then $\dim_kH^0(X,\Omega_X^{\otimes m})^G=\dim_kH^0(X,\Omega_X^{\otimes m})=(2m-1)(g_X-1)>1$.
    There is only one case in Proposition \ref{dim} such that $m\geq 2$ and $\dim_k H^0(X,\Omega_X^{\otimes m})^G>1$, so 
      \begin{equation*}
	(2m-1)(g_X-1)=(2m-1)(g_Y-1)+\deg\left(\left\lfloor \frac{m\pi_*(R)}{n} \right\rfloor \right).
      \end{equation*}
    Combining this with the Hurwitz formula (\ref{hur}) we see that
      \begin{eqnarray*}
	2(2m-1)(g_Y-1)+2\deg\left(\left\lfloor\frac{m\pi_*(R)}{n}\right\rfloor\right) & = & 2(2m-1)(g_X-1)\\
	& = & 2n(2m-1)(g_Y-1)+(2m-1)\deg(R)
      \end{eqnarray*}
    which can be re-arranged as
      \begin{equation*}
	(2m-1)(2n-2)(g_Y-1)=2\deg\left(\left\lfloor\frac{m\pi_*(R)}{n}\right\rfloor\right)-(2m-1)\deg(R).
      \end{equation*}
    So if we can show that the right hand side of this equation is negative then we will have $g_Y-1<0$ and hence $g_Y=0$, as desired.

    Using the same notation as in the case when $m=1$, we calculate:
      \begin{align*}
	\lefteqn{2\deg\left(\left\lfloor\frac{m\pi_*(R)}{n}\right\rfloor\right)-(2m-1)\deg(R)}\\
	& = \sum_{Q \in Y} \left(2\left\lfloor m\cdot \frac{n}{e_Q}\frac{n_Q}{n}\right\rfloor -n(2m-1)\frac{n_Q}{e_Q}\right) \\
	&\leq   \sum_{Q\in Y}\left( 2m\cdot\frac{n_Q}{e_Q}-n(2m-1)\frac{n_Q}{e_Q}\right) \\
	& =  (2m-n(2m-1))\sum_{Q\in Y }\frac{n_Q}{e_Q}.
      \end{align*}

    Now as $n,m\geq 2$ then we have $2m-n(2m-1)\leq 2m-2(2m-1)=2(1-m)<0$ and we are done as $\sum_{Q\in Y}\frac{n_Q}{e_Q}$ is positive.

    So we have shown that in all cases, if $G$ does not act faithfully then $g_Y=0$.
    Now if $m\geq 2$ then firstly note that $G$ must contain a cyclic subgroup of prime order, and this will act trivially on $H^0(X,\Omega_X^{\otimes m})$ as $G$ does.
    Now Proposition \ref{triv} tells us that $m=g_X=2$, and that the order of this subgroup must also be 2.
    Hence, if we denote this subgroup by $K$, then $X/K\cong \mathbb{P}_k^1$, thus completing the only if direction for $m\geq 2$.
    
    Similarly, the $m=1$ case of the only-if direction will follow from Proposition \ref{m=1} after we show that $p>0$ and there is a cyclic subgroup of order $p$ that acts trivially. 
    This is true as $\pi$ cannot be tamely ramified.
    Indeed, if it were then $R=\sum_{P\in X} (e_P-1)[P]$ \cite[{\S} IV, Cor. 2.4]{hart}, and $\deg\left\lfloor \frac{\pi_*(R)}{n} \right\rfloor=0$, which we have already shown cannot be the case.
    Hence $p$ must be positive, and there is a cyclic subgroup of order $p$ which acts trivially.
  \end{proof}

\begin{rem}
  Note that the existence of a hyper-elliptic involution $\sigma$ in $G$ means not only that the genus of $X/\langle \sigma \rangle$ but also the genus of $Y=X/G$ is $0$ (by the Hurwitz formula).
  If moreover $p=2$, then the canonical projection $X\rightarrow X/\langle \sigma \rangle$ is not unramified (again by the Hurwitz formula) and hence not tamely ramified; then $\pi$ cannot be tamely ramified either.
\end{rem}

{\em Acknowledgements.} A prototype of the question underlying this paper goes
back to Michel Matignon. The first-named author would like to thank Niels Borne for
communicating this question to him and for sketching a proof of the fact that $G$ acts
faithfully on $H^0(X,\Omega_X)$ in the tamely ramified case.

\bibliography{/home/jtait/files/Documents/Maths/Bibliography/biblio.bib}
\bibliographystyle{plain}

\end{document}