% !TEX TS-program = pdflatex
% !TEX encoding = UTF-8 Unicode

% This is a simple template for a LaTeX document using the "article" class.
% See "book", "report", "letter" for other types of document.

\documentclass[11pt]{article} % use larger type; default would be 10pt

\usepackage[utf8]{inputenc} % set input encoding (not needed with XeLaTeX)

%%% Examples of Article customizations
% These packages are optional, depending whether you want the features they provide.
% See the LaTeX Companion or other references for full information.

%%% PAGE DIMENSIONS
\usepackage{geometry} % to change the page dimensions
\geometry{a4paper} % or letterpaper (US) or a5paper or....
% \geometry{margins=2in} % for example, change the margins to 2 inches all round
% \geometry{landscape} % set up the page for landscape
%   read geometry.pdf for detailed page layout information

\usepackage{graphicx} % support the \includegraphics command and options

\usepackage[parfill]{parskip} % Activate to begin paragraphs with an empty line rather than an indent

%%% PACKAGES
\usepackage{booktabs} % for much better looking tables
\usepackage{array} % for better arrays (eg matrices) in maths
\usepackage{paralist} % very flexible & customisable lists (eg. enumerate/itemize, etc.)
\usepackage{verbatim} % adds environment for commenting out blocks of text & for better verbatim
\usepackage{subfig} % make it possible to include more than one captioned figure/table in a single float
% These packages are all incorporated in the memoir class to one degree or another...

%%% HEADERS & FOOTERS
\usepackage{fancyhdr} % This should be set AFTER setting up the page geometry
\pagestyle{fancy} % options: empty , plain , fancy
\renewcommand{\headrulewidth}{0pt} % customise the layout...
\lhead{}\chead{}\rhead{}
\lfoot{}\cfoot{\thepage}\rfoot{}

%%% SECTION TITLE APPEARANCE
\usepackage{sectsty}
\allsectionsfont{\sffamily\mdseries\upshape} % (See the fntguide.pdf for font help)
\usepackage{amsmath}
\usepackage{amsthm}
\usepackage{amsfonts}
\usepackage{mathrsfs}
\usepackage{amsopn}
\usepackage{amssymb}
\usepackage{natbib}
% (This matches ConTeXt defaults)

%%% ToC (table of contents) APPEARANCE
\usepackage[nottoc,notlof,notlot]{tocbibind} % Put the bibliography in the ToC
\usepackage[titles,subfigure]{tocloft} % Alter the style of the Table of Contents
\renewcommand{\cftsecfont}{\rmfamily\mdseries\upshape}
\renewcommand{\cftsecpagefont}{\rmfamily\mdseries\upshape} % No bold!

%Theorems and stuff
\newtheorem{defn}{Definition}
\newtheorem{thm}{Theorem}
\newtheorem{cor}{Corollary}
\newtheorem{lem}{Lemma}
\newtheorem{prop}{Proposition}
\newtheorem{ex}{Example}
\theoremstyle{remark}\newtheorem*{rem}{Remark}

\DeclareMathOperator{\ord}{ord}
\DeclareMathOperator{\di}{div}
\DeclareMathOperator{\cha}{char}
\DeclareMathOperator{\gal}{Gal}
%%% END Article customizations

%%% The "real" document content comes below...

\title{Serre's local fields notes}
\author{J Tait}
%\date{} % Activate to display a given date or no date (if empty),
         % otherwise the current date is printed 

\begin{document}
\maketitle
\section{DVRs and Dedekind Domains}

\begin{defn}
 A ring $A$ is a discrete valuation ring (DVI) if it is a local PID.
\end{defn}

\begin{defn}
 We call $A/m(A)$ the residue field of $A$.
\end{defn}

The maximal ideal of a DVI is of the form $\pi A$ for some irreducible $\pi \in A$, called a uniformiser.
The other ideals are all of the form $\pi^nA$ for some $n\in \mathbb{N}$.

\begin{defn}
 We define a valuation $v:\rightarrow \mathbb N$ by writing $x=\pi^n u$ for a unit $u$, and then defining $v(x)=n$.
If $x\in K$, the field of fractions of $A$, then $n$ is now in $\mathbb Z$.
So we can extend this to a map $K\rightarrow \mathbb Z$ by defining $v(0)=\infty$ and otherwise $v(x):=n$.
\end{defn}

\begin{rem}
 The map $v:K^*\rightarrow \mathbb Z$ is surjective. Also, $v(x+y) \geq inf(v(x),v(y))$.
\end{rem}

We can now give an alternative definiton for a DVI, in terms of valuations.

\begin{defn}
 If $v$ is a valuation on a field $K$, then $A:=\{x\in K| v(x)\geq 0\}$ is a discrete valuation ring.
\end{defn}

\section{Couple of equivalent properties and a useful lemma}

\begin{prop}
 Let $A$ be a commutative ring.
Then $A$ is a DVR if and only if $A$ is a Noetherian local ring, whose maximal ideal is generated by a non-nilpotent element.
\end{prop}
\begin{rem}
 The ony if part is trivial.
 The other direction is a long argument, but is a lot simpler if $A$ is integral.
\end{rem}

\begin{prop}
 Let $A$ be Noetherian integral.
Then $A$ is a DVR if and only if it is integrally closed and has a unique non-zero prime ideal.
\end{prop}

\begin{lem}
 Let $A$ be a DVR, and let $x_i$ be in the field of fractions of $A$, such that $v(x_i)>v(x_1)$ for $i\neq 1$.
Then
\[
 x_1+x_2+\ldots +x_n \neq 0.
\]
\end{lem}
\begin{proof}
 If we assume that $x_1=1$ (which we can, by dividing by $x_1$), then we have $v(x_i)>0$ for $i\neq 1$.
So $x_1\notin m(A)$, but for all other $x_i$ this is not true.
Hence $x_1+\ldots+x_n \in m(A)$, and is not zero.
\end{proof}

\section{Dedekind Domains}
For this section, let $A$ be an integral domain and let $K$ be it's field of fractions.
Also, let $S$ be a multiplicatively closed subset not conatining zero. 
We can write $S^{-1}A$ for the localisation of $A$ at $S$.

\begin{prop}\label{remark}
 If $A$ is a noetherian integral domain then the following are equivalent:
 \begin{list}{}{}
  \item For all prime $p\neq 0$ of $A$ then $A_p$ is a DVR.\\
  \item $A$ is integrally closed with dimension at most 1.
 \end{list}
\end{prop}
\begin{rem}
 The dimension of a ring is the maximal height of any prime ideal.
\end{rem}

\begin{proof}
 1 implies 2: If $p\subseteq p'$ then $pA_{p'}\subseteq p'A_{p'}$ and is a prime ideal. 
 It then follows from the fact that $A_{p'}$ is a DVR that $p=0$ or $p=p'$, and $A$ is dimension at most 1.
 
Now if $a$ is integral over $A$ then it is integral over $A_p$, and hence by a previous proposition $a\in A_p$.
So we can write $a = \frac{b}{c}\in A_p$, where $b\in A$ and $c\notin p$ for any prime ideal $p$.
Hence $c$ is not in any maximal ideal of $A$, and in particular is invertible.
Hence $a \in A$.

To show the other direction, we notice that by a previous proposition it suffices to $A_p$ is integrally closed, since we already know that it has a maximal ideal.
Suppose $x$ is integral over $A_p$., so 
\[
 x^n + a_{n-1}x^{n-1} + \ldots + a_0 = 0
\]
for some $a_i\in A_p$. 
We can multiply through by $s^n$ for some $s\in A\\p$ to get
\[
 s^nx^n + sa_{n-1}s^{n-1}x^{n-1} + \ldots + a_0s^n = 0.
\]
Hence $sx\in A$, and thus $x\in A_p$.
 \end{proof}
\begin{rem}
 If $A\subseteq K$ and $S$ is as before, then $x\in K$ is integral over $S^{-1}A$ if and only if $a=\frac{a'}{s}$. with $a'$ integral over $A$ and $s\in S$.
\end{rem}

\begin{defn}
 We call $A$ a Dedekind domain if it has either of the properties in the above proposition.
\end{defn}

\begin{prop}
 In a Dedekind domain all non-zero fractional ideals are invertible.
\end{prop}
\begin{proof}
 In a DVR all the fractional ideals are of the form $\pi^nA$ for some $n\in \mathbb Z$, and hence they are all invertible.
 Since all fractional ideals are sums of products of prime ideals, whose localisations are DVR's by the above proposition, the following relations conclude the proof:
 \begin{eqnarray*}
  (I\cdot J)_p & = & I_p\cdot J_p\\
  (I+J)_p & = & I_p + J_p\\
  (I:J)_p & = & (I_p:J_p).
 \end{eqnarray*}
\end{proof}

 \begin{defn}
  The ideal group of a Dedekind domain is the group of fractional ideals, in which $A$ is the identity.
 \end{defn}
 
 \begin{prop}
  If $x\in A$, $x\neq 0$, then only finitely many primes contain $x$
 \end{prop}

 This gives us the following corollary.
 \begin{cor}
  If we let $v_p$ denote the on $K$ defined by $A_p$, then for all $x\in K$, $v_p(x)$ is almost always zero.
 \end{cor}
 
We now give a number of equivalent definitions of $v_p(I)$ for some fractional ideal $I$.
First note that at most a finite number of primes can contain $I$ for an arbitrary fractional ideal.
Now we can write the image $I_p$ of $I$ in $A_p$ as $I_p = (pA_p)^{v_p(I)}$, with almost all $v_p(I)=0$.
Note that by $I_p$ we mean $I\cdot A_p$, so we may add inverses for elements not in $p$.
If we let $I_1=\prod_p p^{v_p(I)}$ and let $I_2 = \{x\in K | v_p(x)\geq v_p(I)\}$, then $I=I_1=I_2$.

\begin{prop}
 Every fractional ideal of $A$ can be written uniquely in the form $I=\prod_p p^{v_p(I)}$, where almost all the $v_p(I)$ are zero.
\end{prop}

\begin{rem}
 We have the following equalities:
 \begin{eqnarray*}
  v_p(a\cdot b) & = & v_p(a) + v_p(b)\\
  v_p((b:a)) &= &v_p(ba^{-1}) = v_p(b) - v_p(a)\\
  v_p(a+b) & = & inf\{v_p(a),v_p(b)\}\\
  v_p(x) & = & v_p(xA)
 \end{eqnarray*}
\end{rem}

\begin{lem}[Approximation lemma]
 Let $k\in \mathbb N$ and let $p_1,\ldots , p_k$ be distinct prime ideals.
 Also, let $x_i\in K$ and $n_i\in \mathbb Z$ for $1\leq i \leq k$.
 Then there exists $x\in K$ such that $v_{p_i}(x-x_i)\geq n_i$ for all $i$, and $v_q \geq 0$ for all $q\neq p_i$ for some $i$.
\end{lem}
\begin{proof}
 Suppose that $x_i \in A$.
 We can assume by linearity that $x_2  = \ldots = x_k = $ (not sure why?).
 Let $I=P_1^{n_1} + P_2^{n_2}\ldots P_k^{n_k}$.
 Then $v_p(I)=0$ for all $p$, and hence $I=A$.
 Let $x_1 = x+ y$ for some $y\in p_1^{n_1}$ and $x\in p_2^{n_2}\ldots p_k^{n_k}$.
 Then $x$ will satisfy the conditions required.
 
 In general, we can write $x_i=\frac{a_i}{s}$ for some $a_i\in A$ and $s\in A\\{0}$, and we can write $x=\frac{a}{s}$.
 Then $a$ must satisfy 
 \[
  v_q(a) \geq v_q(s)
 \]
for $q\notin \{p_1,\ldots ,p_k\}$ and 
\[
 v_q(a-a_i) \geq n_i + v_{p_i}(s).
\]
But we have shown above that $a_i$ can satisfy this. We may have to add $q$ such that $v_q(s)>0$ to the above.
\end{proof}

\begin{cor}
 A Dedekind domain with finitely many prime ideals is principal.
\end{cor}

\section{Extensions}
For this section, let $K$ be a field, and $L$ a finite extension of $K$, of degree $n$.
Let $A$ be a Noetherian integrally closed domain, wiht field of fractions $K$.
Let $B$ be the integral closure of $A$ in $L$.
Using the same technique as that at the end of the proof of Proposition \ref{remark} we see that $K\cdot B = L$.

We now make the hypothesis that $B$ is a finitely generated $A$-module, and call this hypothesis (F).

\begin{prop}
 If $L/K$ is a seperable extension then (F) is true.
\end{prop}
\begin{proof}
 Proof ommitted.
\end{proof}

\begin{prop}
 If $A$ is a Dedekind domain then so is $B$.
\end{prop}
\begin{proof}
 By our assumptions, $B$ is Noetherian and by (F) it is integrally closed.
 So by a previous proposition it remains to show that the dimension of $B$ is at most 1.
 We suppose that ${\cal B}_0\subsetneq {\cal B}_1 \subsetneq {\cal B}_2$ are prime ideals of $B$.
 Assume without loss of generality that $A\cap {\cal B}_0 = A\cap {\cal B}_1$.
 We will show that this implies that ${\cal B}_0 = {\cal B}_1$ for a contradiction.
 Quotienting by ${\cal B}_0$ if necessary, we may assume that ${\cal B}_0= 0$.
 So if ${\cal B}_1\neq {\cal B}_0$ then there exists some non-zero $x \in {\cal B}_1$.
 Let \[
      x^n + a_{n-1}x^{n-1} + \ldots + a_0 = 0\rm{ a_i\in A}
     \]
be a minimal equation over $A$.
This exists because $B$ is the integral closure of $A$ in $L$, hence all elements are integral over $A$.
Now $a_0\neq 0$ since the equation is minimal, and so $a_0\in (x)$.
Hence $a_0\in {\cal B}_1\cap A$, and ${\cal B}_0\cap A \neq {\cal B}_1 \cap A$, as desired.
This completes the proof.
\end{proof}

We now explain ramification in local fields.
Let ${\cal B} \neq 0$ be a prime ideal of $B$ and ${\cal P} = {\cal B} \cap A$, we say that $\cal B$ divides $\cal P$ (denoted $\cal B | \cal P$).
Equivalently, $\cal B$ divides $\cal P$ if ${\cal B} \supseteq {\cal P} B$. 
We let $e_{\cal B}$ denote the exponent of $\cal B$ in the factorisation of ${\cal P} B$ in to primes i.e. $e_{\cal B} = v_{\cal B}({\cal PB})$.
 This is called the ramification index of $\cal B$ in $L/K$.
 
 Now if $\cal B|P$ then $B/\cal B$ is an extension of $A/\cal P$.
 This is clear since ${\cal B} \supseteq {\cal P}B$ and so $B/{\cal B} \subseteq B/\cal P$.
 Recalling that $B$ is finitely generated over $A$, it is clear that $B/{\cal P}$ is finitely generated $A/\cal P$.
 The degree of this extension is called the residue degree, denoted
 \[
  f_{\cal B} =\left[ B/{\cal B} : A/{\cal P}\right].
 \]
If there is a unique prime $\cal B$ above $\cal P$ and $f_{\cal B} = 1$ then we say $L/K$ is totally ramified at $\cal P$.
If $e_{\cal B} = 1$ and the extension $\left[ B/{\cal B}:A/{\cal P}\right]$ is seperable then we say that $L/K$ is unramified at $\cal B$.
If $L/K$ is unramified for all primes dividing ${\cal P} \subseteq A$ then we say $L/K$ is unramified above $\cal P$.

\begin{prop}
 Let $cal P$ be a non-zero prime ideal of $A$. 
 Then the ring $B/{\cal P}B$ is an $A/\cal P$ algebra of degree $n=[L:K]$, isomorphic to $\prod _{\cal B} B/{\cal B}^{e_{\cal B}}$.
 We have the following equality :
 \[
  n = \sum_{\cal B|P} e_{\cal B}\cdot f_{\cal B}.
 \]
\end{prop}
\begin{proof}
 We start by letting $S=A\ \cal P$, $A' = A_{\cal P} = S^{-1}A$ and finally that $B' = S^{-1}B$.
 Since $A$ is a Dedekind domain then $A'$ is a DVR, and $B'$ is it's integral closure in $L$ (see the end of the proof of prop. \ref{remark}).
 It is clear that $A'/{\cal P}A' = A/\cal P$ and $B'/{\cal P}B' = B/{\cal P}B$.
 As $A'$ is principal then $B'$ is a free module of rank $n$.
 This follows since the maximal ideal is generated by one element, hence we must add $n$ elements to $B'$ for $B\cdot K = L$ to be an $n$-dimensional vector space over $K$.
 Hence $B'/{\cal P} B'$ is free of rank $n$ over $A'/{\cal P}A'$.
 
 
 Notice that ${\cal P} B = \cap {\cal B}^{e_{\cal P}}$, and hence 
 \[
  B/{\cal P}B \rightarrow \prod_{\cal B|P} B/{\cal B}^{e_{\cal B}}
 \]
is injective i.e. an element maps to zero on the RHS if it is in ${\cal B}^e_{\cal B}$ for each $\cal B$ above $\cal P$, which means it is ${\cal P} B$.
The approximation lemma implies the surjectivity of the map - if we represent each element in the product by $x_i$ then we can choose an $x$ as in the lemma, which will map on the them in the quotients.

Finally, comparing the degrees, one sees that $n$ is the sum of the degrees
\[
 n_{\cal B} = \left[B/{\cal B}^{e_{\cal B}} : A/{\cal P}\right].
\]

Then one has 
\begin{eqnarray*}
 n_{\cal B} & = & \sum_{i=0}^{i = e_{\cal B} - 1} \left[ {\cal B}^i/{\cal B}^{i + 1}: A/{\cal P}\right]\\
 & = & e_{\cal B} \cdot \left[ B/{\cal B}:A/{\cal P}\right] \\
 & = & e_{\cal B}\cdot f_{\cal B},
\end{eqnarray*}
as desired.
\end{proof}

\begin{cor}
 Given a prime ideal ${|cal P}\subset A$ there is at least one prime ${\cal B} \subset B$ above $\cal P$, and at  most $n$.
\end{cor}
\begin{rem}
 The above corollary shows that if $A$ has finitely many prime ideals so does $B$.
\end{rem}

\begin{defn}
 Let ${\cal B}$ be a prime ideal of $B$, and let ${\cal P} = {\cal B} \cap A$.
 Then for all $x\in K$ it is true that $v_{\cal B} (x) = e_{\cal B} v_{\cal P} (x)$, and we say that $v_{\cal B}$ prolongs or extends $v_{\cal P}$ with index $e_{\cal B}$.
\end{defn}

\begin{prop}
 Let $w$ be a valuation of $L$, prolonging $v_{\cal P}$ with index $e$.
 Then there exists a prime divisor ${\cal B}$ of $\cal P$ with $w = v_{\cal B}$ and $e= e_{\cal B}$.
 i.e. valuations are equivalent to prime divisors.
\end{prop}
\begin{proof}
 Let $W$ be the ring defined by $w$, and let $\cal Q$ be it's maximal ideal.
 Clearly $A$ is contained in $W$, and it is integrally closed with field of fractions $L$ (this follows from the fact that it is a DVR).
 Therefore $B\subseteq W$.
 Let ${\cal B} = {\cal Q} \cap B$.
 Obviously ${\cal B} \cap A = {\cal P}$, so $\cal B | P$.
 Now $W\supseteq B_{\cal B}$, since all of $W\\ \cal Q$ is invertible.
 but DVRs are maximal subrings, hence $W= B_{\cal B}$, and we are done.
\end{proof}

\section{Norm and Inclusion}
We keep the conditions from the previous section.
Let $I_A$ and $I_B$ be the ideal groups of $A$ and $B$.
We will define homomorphisms $i:I_A\rightarrow I_B$ and $N:I_B \rightarrow I_A$.
Since the ideal groups are free abelian groups generated by prime ideals, it suffices to define the functions on the prime ideals.
With this in mind, let
\[
 i({\cal P}) = {\cal P}B = \prod_{\cal B|P} {\cal B}^{e_{\cal B}}
\]
and 
\[
 N({\cal P}) = {\cal P}^{f_{\cal B}}
\]
if $\cal B|P$.
By a previous proposition for any ideal ${\cal A}\subset A$ then $N(i({\cal A})) = {\cal A}^n$.


We can alternatively define these functions using Grothendieck groups.
Let ${\cal C}_A$ be the categroy of finite length $A$-modules. 
If $M\in {\cal C}_A$ is length $m$ then it has a composition series
\[
 0 = M_0 \subset M_1 \subset \ldots \subset M_m = M
\]
where each quotient $M_i/M_{i+1}$ is simple, i.e. it is isomorphic to $A/{\cal P}_i$ for some prime ${\cal P}_i$ since we are in a Dedekind domain.
By the Jordan-H{\" o}lder theorem these ${\cal P}_i$ are unique up to order.
Hence we can define
\[
 \chi_A(M) = \prod {\cal P}_i.
\]

\begin{rem}
 If $M=\cal A / A$ is a quotient of fractional ideals, ${\cal A} \subset \cal B$, then $\chi_A(M) = {\cal B}^{-1}\cdot \cal A$.
 This is clear since we can write $M = A/{\cal B}^{-1}\cal A$.
 In particular, if $M= A/\cal A$ then $\chi_A(M) = \cal A$.
\end{rem}

Now $\chi_A :{\cal C}_A \rightarrow  I_A$ is multiplicative in the sense that if
\[
 0\rightarrow M' \rightarrow M \rightarrow M'' \rightarrow 0
\]
is a short exact sequence in ${\cal C}_A$ then $\chi_A(M) = \chi_A(M')\chi_A(M'')$.
Also, the map is universal in the sense that every homomorphism $f:{\cal C}_A \rightarrow G$ to an abelian group $G$ factor through $\chi_A$.
Hence we can identify ${\cal C}_A$ and $I_A$.

We have a similar map $\chi_B:{\cal B}\rightarrow I_B$, which also associates the respective category and group.
Since every finitely generated $B$-module is also a finitely generated $A$-module there is an exact functor ${\cal C}_B\rightarrow {\cal C}_A$.
This functor induces a map $I_B\rightarrow I_A$ which is the norm map.
\begin{prop}
 If $M$ is a finitely generated $B$-module then $\chi_A(M) = N(\chi_B(M))$.
\end{prop}
\begin{proof}
 We only need to check $M=B/\cal B$, by linearity.
 Indeed, if $B/\cal B$ is a length $m$ module over $B$ then viewing it as an $A$ module will increase the length by the degree of $[B/{\cal B} : A/{\cal P}]$, since ${\cal B} \supseteq {|cal P}B$.
 i.e. we will get ${\cal B}^{e_{\cal B}}$. 
 \end{proof}
 
 Every $A$-module $M$ of finite length when tensored with $B$ gives a $B$-module of finite length , say $M_B$.
 This functor ${\cal C}_A\rightarrow {\cal C}_B$ is exact and gives the inclusion homomorphism.
 
 \begin{prop}
  If $M$ is a finite length $A$-module then $\chi_B(M) = i(\chi_A(M))$.
 \end{prop}
\begin{proof}
 We just need to consider $M=A/\cal P$, so $M_B = B/{\cal P} B$.
 Then this is clear, since we have the chain $0\subset B/{\cal B} \subset \ldots \subset B/{\cal B}^{e_{\cal B}} = B/{\cal P}B$.
\end{proof}

\begin{prop}
 If $x\in L$ then $N(xB) = N_{L/K}(x)A$.
\end{prop}
\begin{proof}
 We can assume that $x$ is integral over $A$, and that $A$ is principal (localising if necessary).
 Then $B$ is a free $A$-module of rank $n$.
 Let $u_x:B\rightarrow B$ be multiplication by $x$ in $B$.
 Then $N_{L/K} (x) = \det (u_x)$ and $N(xB) = \chi_A(B/xB) = \chi_A ({\rm coker }\ u_x)$.
 The next lemma completes the proof.
\end{proof}

\begin{lem}
 Let $A$ be a PID and let $u:A\rightarrow A$ be a linear map such that $\det (u) \neq 0$.
 Then $\det (u) A = \chi_A ({\rm coker }\ u)$.
\end{lem}
\begin{proof}
 First note that $\det (u) A$ does not change if we multiply $u$ by an invertible linear map, so we can reduce $u$ to be diagonal.
 We then proceed by induction on $n$.
 If $n=1$ then this is just $\chi_A (A/{\cal A}) = \cal A$ as has already been noted.
\end{proof}

\section{An example}
Let $A$ be a local ring, and let $k$ be the residue field of $A$.
Let $n$ be a positive integer and let $f\in A[X]$ be a monic polynomial of degree $n$.
Let $B_f$ be the quotient ring $A[X]/(f)$.
It is a free $A$-algebra of finite type, with basis
\[
\{1,X,\ldots , X^{n-1}\}.
\]

We first want to find the maximal ideals of $B_f$.
Let $m\subset A$ be the maxmimal ideal of $A$, and let $\bar B_f := B_f/mB_f = A[X]/(m,f)$.
Let $\bar f$ be the image of $f$ in $A[X]/m = k[X]$.
Then $\bar B_f = k[X]/(\bar f)$.
We write the indecomposable decomposition of $\bar f$ as $\prod _{i\in I} \phi_i^{e_i}$.
For each $i \in I$ let $g)i \in A[X]$ be a polynomial such that $\bar g_i = \phi_i$.

\begin{lem}
Let $m_i = (m,g_i)$ be the ideal generated by $m$ and the image of $g_i$ in $B_f$.
Then $m_i$ are maximal and distinct, and make up all of the maximal ideals of $B_f$.
We also have the following isomorphism:
\[
B_f \cong k_i := k[X]/(\phi_i).
\]
\end{lem}

\begin{proof}
Let $\bar m_i$ be the ideal in $\bar B_f$ generated by $\phi_i$.
This is clearly maximal, since $\phi_i$ is irreducible.
It is also clear that $m_i$ is the inverse of $\bar m_i$.
Since $\bar B_f /(\phi_i) = k_i$, clearly $m_i$ is maximal and $b_f/m_i = k_i$.
It suffices to show that any maximal ideal $n$ contains $m$, since then it must be the inverse of $\phi_i$ in $\bar B_f$, which are the only maximal ideals in $\bar B_f$.
Now if $n \supsetneq m$, then $n + mB-f = B_f$.
As $B_f$ is a finitely generated $A$-module then Nakayama's lemma would imply that $n = B_f$, a contradiction.
\end{proof}

For the rest of this section we assume that $A$ is a DVR. We look at two cases, assuming that $B_f$ is also a DVR.
In the first case we will look is the unramified case.

\begin{prop}
If $A$ is a DVR and $\bar f$ is irreducible, then $B_f$ is a DVR with maximal ideal $mB_f$ and residue field $k[X]/(\bar f)$.
\end{prop}
\begin{proof}
By the lemma, $B_f$ is local and has maximal ideal $mB_f$ and residue field $k[X]/(\bar f)$.
If $\pi$ generates $m$, then the image of $\pi$ in $B_f$ generates $mB_f$ and is not nilpotent.
Hence $B_f$ is a DVR.
\end{proof}

\begin{cor}
If $K$ is the field of fractions of $A$, then $f$ is irreducible in $K[X]$.
If $L=K[X]/(f)$, then $B_f$ is the integral closure of $A$ in $L$.
\end{cor}
\begin{proof}
It is clear that $K[X]/(f) = B_f\otimes_A K$.
Now $B_f$ is an integral domain, hence $B_f\otimes_A K$ is also.
This means that $K[X]/(f)$ is a an integral domain, and hence a field, since prime ideals are maximal in polynomial rings in one variable.
Hence $f$ is irreducible.
Now $B_f$ is integrally closed since it is a DVR (see earlier proposition).
It is clear that $L$ is it's field of fractions and that $A\subseteq B_f$, so $A$ is the integral closure of $A$ in $L$.
\end{proof}

\begin{cor}
If $\bar f$ is a seperable then the extension $L/K$ is unramified.
\end{cor}
\begin{proof}
In some field $L'$ we can write $f$ as a combination of linear factors.
Each of these makes a prime ideal $\cal B$ and $e_{\cal B} = 1$ for each $\cal B$ since the factors are linear.
\end{proof}

\begin{prop}
Let $A$ be a DVR wtih field of fractions $K$, and let $L$ be an extension of degree $n$.
Let $B$ be the integral closure of $A$ in $L$.
Suppose that $B$ is a DVR also, and that the residue field $\bar L$ of $B$ is a simple extension of degree $n$ of $k = \bar K$, the residue field of $A$.
Let $x$ be any element of $B$ such that $\bar x \in \bar L$ is a generator of $\bar L$ over $k$.
We denote by $f$ the characteristic polynomial of $x$ over $k$.
Then the homorphism $A[X]\rightarrow B$, defined by $X \mapsto x$ defines, by passing to the quotient, an isomorphism $B_f\rightarrow B$.
\end{prop}
\begin{proof}
First note that the coefficients of $f$ are integral over $A$.
They are also in $K$, by definition, and it then follows from the fact that $A$ is integrally closed that they are in $A$.
The fact that $f(x) = 0$ shows that $A[X]\rightarrow B$ factors in to $A[X]\rightarrow B_f \rightarrow B$.
Since $\bar f (\bar x) = $ and $\bar x$ is degree $n$ over $k$, $\bar f$ is the minimal polynomial of $\bar x$ over $k$, and hence is irreducible.
The result now follows from the above corollary.
\end{proof}

We now look at the other case, which is the totally ramified case.
\begin{prop}
Suppose that $A$ is a DVR, and that
\[
f = X^n + a_1X^{n-1} + \ldots + a_n
\]
where $a_i\in m$ and $a_n \notin m^2$.
Then $B_f$ is a DVR and has maximal ideal generated by the image $x$ of $X$ and has residue field $k$.
\begin{proof}
By definition, $\bar f = x^n$.
The lemma then shows that $B_f$ is local with maximal ideal generated by $(m,x)$.
Moreover, there is a uniformiser $\pi$ of $A$ such that $-\pi = x^n +a_1x^{n-1} + \ldots + a_{n-1}x$, and we see that $\pi \in (x)$.
So $(m,x) = (x)$.
Since $\pi$ is not nilpotent, $x$ is not either, and $B_f$ is a DVR.
\end{proof}

\begin{cor}
Similarly to before, $f$ is irreducible in $K[X]$ and if $L=K[X]/(f)$, then $B_f$ is the integral closure of $A$ in $L$.
\end{cor}
\begin{proof}
Proof ommitted - similar to corresponding corollary above.
\end{proof}

\begin{prop}
Suppose that $A$ is a DVR, $K$ is it's field of fractions and $L$ is a degree $n$ extension of $K$.
Let $B$ be the integral closure of $A$ in $L$.
Then suppose that $B$ is a DVR and it's valuation extends that of $A$ with index $n$.
Let $x$ be a uniformiser of $B$, with characteristic polynomial $f$ in $K[X]$.
Then $f$ is an Eisenstein polynomial (as described above), and the homomorphism of $A[X]$ in to $B$ that maps $X\mapsto x$ defines, by passage to the quotient, an isomorphism of $B_f$ on to $B$.
\end{prop}
\begin{proof}
Let $w$ be the valuation of $B$.
Then $w(x) = 1$ and $w(a) \cong 0 \ {\rm mod } \ n$ for $a \in A$.
Let $r = {\rm inf}(w(a_ix^{n-i}))$ with $0\leq i \leq n$.
By an earlier lemma, there exists unequal $i$ and $j$ such that 
\[
r = w(a_ix^{n-i}) = w(a_jx^{n-j}).
\]
Hence $i-j = w(a_i/a_j) \cong 0 \ {\rm mod } \ n$, which is only possible if $i =0$ and $j= n$.
So $r = n$ and $w(a_n) = n$.
Also $w(a_i)\geq n-i$ for all $i$.
It follows that $f$ is Eisenstein, and the follows from the corollary.
\begin{proof}

An exercise asks to show that in Lemma 4, if $e_i=1$ then $(B_f)_{m_i}$ is a DVR.
Well, this follows since $e_i= 1$ if and only if $\phi_i$ is degree one, hence the result is clear.




\bibliography{/home/jtait/files/Desktop/Work/Bibliography/biblio.bib}
\bibliographystyle{plain}


\end{document}
