% !TEX TS-program = pdflatex
% !TEX encoding = UTF-8 Unicode

% This is a simple template for a LaTeX document using the "article" class.
% See "book", "report", "letter" for other types of document.

\documentclass[11pt]{article} % use larger type; default would be 10pt

\usepackage[utf8]{inputenc} % set input encoding (not needed with XeLaTeX)

%%% Examples of Article customizations
% These packages are optional, depending whether you want the features they provide.
% See the LaTeX Companion or other references for full information.

%%% PAGE DIMENSIONS
\usepackage{geometry} % to change the page dimensions
\geometry{a4paper} % or letterpaper (US) or a5paper or....
% \geometry{landscape} % set up the page for landscape
% read geometry.pdf for detailed page layout information

\usepackage{graphicx} % support the \includegraphics command and options
\usepackage[obeyDraft]{todonotes}

%\usepackage[parfill]{parskip} % Activate to begin paragraphs with an empty line rather than an indent

%%% PACKAGES
\usepackage{mathtools}
\usepackage{booktabs} % for much better looking tables
\usepackage{array} % for better arrays (eg matrices) in maths
\usepackage{paralist} % very flexible & customisable lists (eg. enumerate/itemize, etc.)
\usepackage{verbatim} % adds environment for commenting out blocks of text & for better verbatim
\usepackage{subfig} % make it possible to include more than one captioned figure/table in a single float
% These packages are all incorporated in the memoir class to one degree or another...

%\usepackage[activate={true,nocompatibility},final,tracking=true,kerning=true,spacing=true,factor=1100,stretch=10,shrink=10]{microtype}
%\microtypecontext{spacing=nonfrench}
% activate={true,nocompatibility} - activate protrusion and expansion
% final - enable microtype; use "draft" to disable
% tracking=true, kerning=true, spacing=true - activate these techniques
% factor=1100 - add 10% to the protrusion amount (default is 1000)
% stretch=10, shrink=10 - reduce stretchability/shrinkability (default is 20/20)

%%% HEADERS & FOOTERS
\usepackage{fancyhdr} % This should be set AFTER setting up the page geometry
\pagestyle{fancy} % options: empty , plain , fancy
\renewcommand{\headrulewidth}{0pt} % customise the layout...
\lhead{}\chead{}\rhead{}
\lfoot{}\cfoot{\thepage}\rfoot{}

%%% SECTION TITLE APPEARANCE
\usepackage{sectsty}
\allsectionsfont{\sffamily\mdseries\upshape} % (See the fntguide.pdf for font help)
\usepackage{amsmath}
\usepackage{amsthm}
\usepackage{amsfonts}
\usepackage{mathrsfs}
\usepackage{amsopn}
\usepackage{amssymb}
%\usepackage{natbib}
% (This matches ConTeXt defaults)

%%% ToC (table of contents) APPEARANCE
\usepackage[nottoc,notlof,notlot]{tocbibind} % Put the bibliography in the ToC
\usepackage[titles,subfigure]{tocloft} % Alter the style of the Table of Contents
\renewcommand{\cftsecfont}{\rmfamily\mdseries\upshape}
\renewcommand{\cftsecpagefont}{\rmfamily\mdseries\upshape} % No bold!
%\renewcommand{\familydefault}{\sfdefault}
%\usepackage{cabin}
%\usepackage{libertine}
%\usepackage[T1]{fontenc}

%Theorems and stuff
\theoremstyle{plain}
\newtheorem{defn}{Definition}[section]
\newtheorem{thm}[defn]{Theorem}
\newtheorem{cor}[defn]{Corollary}
\newtheorem{lem}[defn]{Lemma}
\newtheorem{prop}[defn]{Proposition}
\newtheorem{ex}[defn]{Example}
\newtheorem*{unnumthm}{Theorem}
\newtheorem{defnlem}[defn]{Definition/Lemma}
\newtheorem{defnthm}[defn]{Theorem/Definition}
\theoremstyle{remark}
\newtheorem*{rem}{Remark}
\newtheorem*{note}{Note}


\newcommand{\cO}{{\cal O}}
\newcommand{\ra}{\rightarrow}
\newcommand{\NN}{{\mathbb N}}
\newcommand{\PP}{{\mathbb P}}
\newcommand{\ZZ}{{\mathbb Z}}
\newcommand{\cL}{{\mathcal L}}
\newcommand{\cA}{{\mathcal A}}
\newcommand{\cD}{{\mathcal D}}
\newcommand{\cU}{{\mathcal U}}
\newcommand{\cech}{\v{C}ech }
\newcommand{\hzero}{{H^0(X,\Omega_X)}}
\newcommand{\hone}{H^1(X,\mathcal{O}_X)}
\newcommand{\cechhone}{\check{H}^1(\mathcal U,\mathcal O_X)}
\newcommand{\derhamhone}{H_{\text {dR}}^1(X/k)}


\DeclareMathOperator{\aut}{Aut}
\DeclareMathOperator{\res}{Res}
\DeclareMathOperator{\ord}{ord}
\DeclareMathOperator{\di}{div}
\DeclareMathOperator{\cha}{char}
\DeclareMathOperator{\gal}{Gal}
\DeclareMathOperator{\Tr}{Tr}
\DeclareMathOperator{\Ima}{Im}

%%% END Article customizations

%%% The "real" document content comes below...

\title{Actions on spaces of differentials on algebraic curves}
\author{Joseph Tait}
%\date{} % Activate to display a given date or no date (if empty),
         % otherwise the current date is printed

\begin{document}
\maketitle
\newpage
%\listoftodos

\tableofcontents

\chapter{Introduction} \label{Chapter:introduction}


Geometry and topology provide perhaps the greatest source of intuition and vision in mathematics, whilst algebra balances the scales, being the exemplar of precision and abstraction. 
Perhaps the best example of the interplay between these two areas is the triple equivalence of Riemann surfaces, complex function fields and complex curves.
On the one hand, compact Riemann surfaces constitute all spaces that occur in the topological classification of connected, compact, orientable surfaces.
On the other hand, complex function fields lie strongly in the algebraic end of the spectrum, with strong relations to number theory and Galois theory.
Finally, it is algebraic curves that most clearly unites algebra and geometry.

The genus is arguably the most important invariant of topological surfaces.
It is possible to use it to define the Euler characteristic, and it also benefits from being very easy to describe --- the genus of a connected, compact, orientable surface is just the number of ``holes" or ``handles" it has.
Given this, any theory that claims to be equivalent to the study of Riemann surfaces would do well to explain how it gives rise to the concept of genus.

In the case of algebraic curves, it is Riemann-Roch theory that allows us to extend the definition of genus.
Originally only for Riemann surfaces, the theory focusses on meromorphic functions and differentials.
It is this focus which allows the definition to be generalised, first just to complex algebraic curves, then to curves over any algebraically closed field $k$.
The genus appears as a constant in the Riemann-Roch theory of Riemann surfaces, most notably as the dimension of the vector space of holomorphic differentials and in the Riemann-Roch theorem itself.
Since this constant remains after generalising to algebraic curves, it is natural to continue calling this constant the genus.
The fact that the genus can be defined in terms of differentials demonstrates why differentials, and in particular holomorphic differentials, play such an important role in the theory of algebraic curves.

%change new page so that it fills pads out the other paragraphs
\vbox{On the other hand, we recall the famous quote
\begin{quotation}
``Whenever you have to deal with a structure endowed entity, try to determine its automorphism group" --- Hermann Weyl
\end{quotation}}
With this in mind, determining the automorphism group of collections of differentials on a curve $X$, such as $\hzero$, and more generally how group actions on such objects behave, is a worthwhile pursuit in its own right.
Moreover, the automorphism groups of algebraic curves, and in particular Riemann surfaces, have given rise to many interesting theories.
For example, it is known that every finite group is the full automorphism group of some Riemann surface \cite[Thm.\ 6']{greenberg}.

The main focus of this thesis will lie in considering the $k[G]$-module structure of various spaces of differentials on a curve $X$, and related spaces, for a subgroup $G$ of the automorphism group $\aut(X)$.
Moreover, we will focus on what happens in positive characteristic.
Of course, if the characteristic divides the order of $G$ the theory is often a lot more complex --- for example, we no longer have Maschke's theorem, a fundamental result in representation theory.

The thesis is broken in to four main chapters (excluding this one).
The first gives background and fixes notation.
We proceed to describe and motivate the other three chapters below.

\section{Bases of spaces of (poly)differentials on hyperelliptic curves}

Hyperelliptic curves are a classically studied class of algebraic curves, characterised by being double covers of the projective line.
They can be viewed as a natural extension of elliptic curves to higher genera, sharing a similar defining equation of $y^2 = f(x)$ (if $\cha(k) \neq 2$).
It is this concrete and relatively simple defining equation that allows explicit calculations to be made for them.
Added to this, there exist hyperelliptic curves with every possible genus (except one and zero), so in this sense they are not a very restrictive class to consider.
Moreover, hyperelliptic curves also have a number of nice geometric properties --- for example, they can be characterised entirely in terms of Weierstrass points, and also every genus 2 curve is hyperelliptic.

It is for these reasons that we study hyperelliptic curves throughout this thesis.
However, despite being commonplace in algebraic geometry, it is not always easy to find precise statements in the literature.
This is especially true when working over a field of characteristic two, where hyperelliptic curves behave very differently.
Because of this we split Chapter $3$ in to two sections, according to the characteristic of $k$, and start each section by collecting results that will be needed, either later in the chapter or the rest of thesis.

The highlights of the chapter are Proposition \ref{propbasishyperellipticp=2} and Proposition \ref{prophyperellipticbasispnot2}, which give bases of the space of holomorphic differentials and polydifferentials of a hyperelliptic curve $X$ when the characteristic of $k$ is two and is not two, respectively.
We first state the basis when the characteristic of $k$ is not 2, recalling that in this case the function field $K(X)$ is equal to $k(x,y)$, where $y$ satisfies $y^2 = f(x)$ for some polynomial $f(x) \in k[x]$.
    \begin{unnumprop}
    Let $m\geq 1$ and let $\omega := \frac{dx^{\otimes m}}{y^m}$. 
    Then if $g\geq 2$, a basis of $H^0(X,\Omega_X^{\otimes m})$ is given by:
        {\centering 
        \begin{tabular}{c c}
        $\omega, x\omega, \ldots , x^{g-1}\omega$ &  if $m=1$, \\
        $\omega, x\omega, x^2\omega$  &  if $m=g=2$, \\
        $\omega, x\omega, \ldots, x^{m(g-1)}\omega;\  y\omega, xy\omega, \ldots, x^{(m-1)(g-1)-2}y\omega$ &  otherwise.
        \end{tabular}\par
        }
    \end{unnumprop}
    
Note that the case where $m=1$ is already in the literature, see \cite[Prop. 7.4.26]{liu} or \cite[Ch. IV, \S 4, Prop. 4.3]{griffiths}.
 
On the other hand, if $\cha(k) = 2$ then $K(X)$ is still equal to an extension of $k(x)$ of the form $k(x,y)$, but this time we require $y$ to satisfy $y^2 + H(x)y = F(x)$, where $F(x)$ and $H(x)$ are polynomials in $k[x]$, whose degrees will determine the genus.
We now give another basis for this case.
    \begin{unnumprop}
    Let $m\geq 1$ and let $\omega:= \frac{dx^{\otimes m}}{H(x)^m}$. 
    Then if $g\geq 2$, a basis of $H^0(X,\Omega_X^{\otimes m})$ is given by:\\
        {\centering
        \begin{tabular}{c c}
        $\omega, x\omega, \ldots , x^{g-1}\omega$ &  if $m=1$, \\
        $\omega, x\omega, x^2\omega$ & if $m=g=2$, \\
        $\omega, x\omega, \ldots, x^{m(g-1)}\omega;\  y\omega, xy\omega, \ldots, x^{(m-1)(g-1)-2}y\omega$ & otherwise.
        \end{tabular}\par
        }
    \end{unnumprop}
    
Note that the case where $m=1$ can again be found in \cite[Prop. 7.4.26]{liu}.

Equipped with the knowledge of these explicit bases we can examine group actions on $H^0(X,\Omega_X^{\otimes m})$ much more readily.
For example, in Chapter 5 we compute the action of the hyperelliptic involution $\sigma$ on the above basis.
Using this we can see when the group generated by $\sigma$ acts faithfully on $H^0(X,\Omega_X^{\otimes m})$, explicating the main theorem of the chapter in this case.

\section{Group actions on algebraic de Rham cohomology}

In the study of smooth manifolds de Rham cohomology is a well-established tool, which determines to what extent closed differential forms on a smooth manifold $M$ fail to be exact.
Moreover, in 1931 Georges de Rham \cite{derhamstheorem} proved de Rham's theorem.
    \begin{unnumthm}[de Rham's theorem]
    Let $k$ be either $\RR$ or $\CC$.
    For any smooth manifold $M$ over $k$ we have an isomorphism
        \[
        H^n_{\text{dR}}(M) \cong H^n(M;k),
        \]
    between the de Rham cohomology groups and the singular cohomology groups of $M$.
    \end{unnumthm}

Of course, if de Rham cohomlogy can be defined on complex manifolds, one can compute the de Rham cohomology of Riemann surfaces.
This leads to the obvious question as to whether one can define analog of de Rham cohomology for algebraic curves.
Grothendieck answered this in a letter to Atiyah \cite{grothendiecklettertoatiyah}, where he in fact defined the algebraic de Rham cohomology of a scheme.
The Hodge-de Rham spectral sequence arose from this definition, and has been much studied.
In particular, Deligne and Illusie proved that if, for example $X$ is a complex, projective variety then
    \begin{equation*}
    H^n_{\text{dR}}(X) \cong \bigoplus_{i=0}^n H^i(X,\Omega_X^{n-i}),
    \end{equation*}
see \cite{deligneillusie}.
Of course, when $X$ is a curve this is equivalent to saying that we have a canonical short exact sequence
    \begin{equation}\label{equationafterhodgetheorydecomposition}
    0 \ra \hzero \ra \derhamhone \ra \hone \ra 0.
    \end{equation}
Moreover, in general (for example, when $\cha(k) = 0$), this sequence splits not only as $k$ vector spaces, but also as $k[G]$-modules, where $G$ is a subgroup of $\aut(X)$.
However, this is not always the case --- in particular, if $\cha(k) = p >0$ divides the order $G$, the sequence may not split.
In \cite{canonicalrepresentation} Hortsch demonstrated that if $X$ is a hyperelliptic curve over $k$, an algebraically closed field of characteristic $p$, and has $y^2 = x^p-x$ as a defining equation, then \eqref{equationafterhodgetheorydecomposition} does not split.

In Chapter 4 we generalise this result to the following theorem.
    \begin{unnumthm}
    Let $X$ be a hyperelliptic curve over an algebraically closed field $k$ of characteristic $p \geq 3$.
    Suppose there exists $\tau \in \aut(X)$ such that the induced automorphism $\bar \tau \colon \mathbb P_k^1 \ra \mathbb P_k^1$ is given by $x \mapsto x+a$ for some $0 \neq a \in k$.
    We let $G = \langle \tau \rangle$ be the subgroup of $\aut(X)$ generated by $\tau$, and further suppose that $X$ is not ramified above $\infty \in \PP_k^1$.
    Then the sequence \eqref{equationafterhodgetheorydecomposition} does not split as a sequence of $k[G]$-modules.
    \end{unnumthm}
Such curves exist in every genus and every characteristic greater than 2, and we give examples of such curves.%, and also give an example from \cite{automorphismshyperellipticmodular} of a curve that is as described in theorem, except that it is not ramified above $\infty \in \PP_k^1$, and show that for this curve the short exact sequence \eqref{equationafterhodgetheorydecomposition} does split.

We prove the above theorem by first computing explicit bases of each of the spaces in \eqref{equationafterhodgetheorydecomposition}.
If $\pi \colon X \ra \PP_k^1$ is a degree two projection, then by \cech cohomology we have
        \[
        \hone \cong \frac{\cO_X\left(U_0 \cap U_\infty\right)}{\{ f_0 - f_\infty | f_i \in \cO_X(U_i)\}},
        \]
where $U_0 = X \backslash \pi^{-1}(0)$ and $U_\infty = X \backslash \pi^{-1}(\infty)$.
As mentioned, in the preceding chapter we already computed a basis of $\hzero$, and we use this along with Serre duality and the above identity to compute a basis of $\hone$.
    \begin{unnumthm}
    Then the elements $\frac{y}{x}, \ldots, \frac{y}{x^g} \in K(X)$ are regular on $U_0 \cap U_\infty$, and their residue classes $\left [ \frac{y}{x} \right ],  \ldots, \left [ \frac{y}{x^g} \right]$ form a basis of $\hone$.
    \end{unnumthm}
 It should be noted that this basis is the same regardless of characteristic --- since this is not the case for the dual space $\hzero$, this may be surprising.
 We then go on to compute the basis of $\derhamhone$ as well, given in Theorem \ref{theorembasisofderham}, but this is slightly more involved, so we do not state it here.


 \section{Faithful actions on Riemann-Roch spaces}

A significant open problem is to, given a smooth, projective curve $X$ over an algebraically closed field $k$, completely determine the $k[G]$-module structure of $\hzero$, for any subgroup $G$ of $\aut(X)$.
This was done for the case $k = \CC$ by Chevalley and Weil in 1934, see \cite{chev}.
The result was later broadened to a curve over any algebraically closed field of characteristic zero by Lewittes \cite{lewittes}, and Broughton's paper \cite{broughton} gives another method of generalising to this case.
The question has also been answered by Kani \cite{Kani} and Nakajima \cite{naka2}, if the projection $\pi \colon X \ra Y:= X/G$ is tamely ramified.
Valentini and Madan \cite{valmadan} determined the structure when $\pi$ may be wildly ramified, but they assume that $G$ is a cyclic group of order $p^n$, and this was recently generalised by Karanikolopoulos and Kontogeorgis to any cyclic group \cite{kako}.
%Other papers look at this - read intros to and then write a sentence about other papers already referenced in our paper.

A weaker, but naturally related, question is: "When does a group $G$ act faithfully on $H^0(X,\Omega_X)$."
We answer this in full generality, and also extend the result to look at the space of holomorphic polydifferentials denoted $H^0(X,\Omega_X^{\otimes m})$.
In particular we obtain the following theorem.
    \begin{unnumthm}
    Suppose that $g_X\geq 2$ and let $m\geq1$. 
    Then $G$ does not act faithfully on $H^0(X,\Omega_X^{\otimes m})$ if and only if $G$ contains a hyperelliptic involution and one of the following two sets of conditions holds:
        \begin{itemize}
        \item $m=1$ and $p=2$;
        \item $m=2$ and $g_X=2$.
        \end{itemize}
    \end{unnumthm}
Our main method of attack in proving this is comparing the dimension of $H^0(X,\Omega_X^{\otimes m})$ to its fixed space, $H^0(X,\Omega_X^{\otimes m})^G$, for a cyclic group $G$ of prime order.
We then aim to reduce the general case down to this somewhat more manageable example.

We use similar techniques to compute when $G$ acts trivially on more general Riemann-Roch spaces, and in particular we prove the following proposition.
    \begin{unnumprop}\label{nakaj}
    Suppose $p>0$ and $G$ is a cyclic group of order $p^l$ for some $l\geq 1$.
    Let $D$ be a $G$-invariant divisor on $X$ such that $\deg(D)>2g_X-2$.
    Then the action of~$G$ on $H^0(X,\cO_X(D))$ is trivial if and only if
        \[ 
        (p^l-1)\deg(D)=p^l\left(g_X-g_Y-\sum_{Q\in Y}\left\langle \frac{n_Q}{e_Q} \right\rangle\right).
        \]
    \end{unnumprop}




\begin{comment}
\section{Introduction}

In topology, the number of holes in a compact, orientable surface is an important invariant, called the genus, and classifies compact, orientable surfaces up to homeomorphism.
In particular the genus is an important topological invariant of compact Riemann surfaces.
It is well known that for any compact Riemann surface the genus is also equal to the dimension of the space of global holomorphic differentials.
Furthermore there is a correspondence between compact Riemann surfaces and smooth projective algebraic curves over $\mathbb C$, and the notion of holomorphic differentials can be extended to such curves.
In fact we can extend this even further, by defining the genus of any curve over an algebraically closed field to the dimension of the space of global holomorphic differentials.
The above alone makes it obvious that the space of global holomorphic differentials is a fundamental object in the theory of algebraic curves.
The general motivation underlying this report is to study this space as a representation of a subgroup of the automorphism group of the given curve.

Let $X$ be a smooth connected projective curve over an algebraically closed field $k$.
Given a subgroup $G$ of the automorphism group of $X$ then a classic problem pertaining to $H^0(X,\Omega_X)$, the space of holomorphic differentials (see section \ref{chapterbackground}), is determining its $k[G]$-module structure.
This originally dates back to 1934, and a paper of Chevalley and Weil \cite{chev}.
They only considered the case when $k= \mathbb C$, but the complete structure has since been discovered in the case where the projection from $X$ to the quotient curve is tamely ramified.
This was done by Kani in 1986 \cite{Kani}.
Progress has also been made recently in the case where the projection is wildly ramified; in particular Karanikolopoulos and A. Kontogeorgis \cite{kako} have computed the $k[G]$-module structure for any cyclic group $G$.
Also, in 1986 Broughton \cite{Broughton} computed the $k[G]$-module structure of the space of global holomorphic poly-differentials, $H^0(X,\Omega_X^{\otimes m})$ (see Section \ref{charneq2}), in the case where $\cha(k) = 0$.

In this report we will not look directly at the $k[G]$-module structure, but rather at the related question of determining when the action of $G$ on $H^0(X,\Omega_X)$, and also on $H^0(X,\Omega_X^{\otimes m})$, is faithful.
The following is our main result:

    \begin{unnumthm}
    Suppose that $g_X\geq 2$ and let $m\geq1$. 
    Then $G$ does not act faithfully on $H^0(X,\Omega_X^{\otimes m})$ if and only if $G$ contains a hyperelliptic involution and one of the following two sets of conditions holds:
        \begin{itemize}
        \item $m=1$ and $p=2$;
        \item $m=2$ and $g_X=2$.
        \end{itemize}
    \end{unnumthm}

The format of the report is now briefly outlined.

In the first section we prove the strong form of the Riemann-Hurwitz formula (Theorem \ref{theoremdetailedhurwitz}).
The Riemann-Hurwitz formula relates the genus of two curves when there is a surjective map from one to the other, via the degree of the map and the degree of the ramification divisor.
However, this can obscure the fact that the canonical divisors (see Section \ref{chapterbackground}) themselves are related.
The strong form of the theorem states that given two curves and a surjective map $\pi:X\rightarrow Y$ of degree $n$ between the curves, with ramification divisor $R$ (see Section \ref{chapterbackground}), we have
    \[
    \di (\pi^* (\omega)) = \pi^*(\di (\omega)) + R,
    \]
where $\omega$ is a non-zero differential on $Y$, and $\pi^*$ is the pull-back induced by $\pi$.
This section closely follows Stichtenoth's book, see \cite{stichtenoth}.

The second section looks at computing the dimension of various spaces, but the most significant is the dimension of the subspace of $H^0(X,\Omega_X^{\otimes m})$ fixed by $G$, where $m\geq 1$.
This result, along with two other results in section three, forms the heart of the proof of the main theorem.
The dimension itself is dependent, essentially, on the genus of the quotient curve, $Y=X/G$, the degree of the projection map $\pi:X\rightarrow Y$, the ramification divisor $R$ of $\pi$ and $m$.
By using the Riemann-Roch theorem we can easily compute the dimension of $H^0(X,\Omega_X^{\otimes m})$, and the comparison of these two dimensions is what will be used in the third section.


In the third section we consider when a group of prime power order acts trivially on $H^0(X,\Omega_X^{\otimes m})$.
By only considering cyclic groups of prime power order our computations are made considerably easier.
Initially we only consider groups of prime order.
In this case, if the characteristic of $k$ is different to $p$ then the projection map is tamely ramified, and hence we know that the coefficients of the ramification divisor are $p-1$.
This makes it considerably easier to compute the dimension of the fixed space, as of all the parameters it depends on, the ramification divisor is the most difficult to deal with.
When the characteristic of the field and the order of the group are the same (\ie when wild ramification could occur), then the computations are longer, but still made considerably easier by our assumptions.
At the end of the third section we use results of \cite{kako} to make the same computations for groups whose orders are powers of $p$.
The results we use are somewhat more technical, but they do extend the original results.
We also make some computations for a general divisor $D$, which in general should have degree greater than $2g-2$, and give criteria for when the action is trivial on the associated Riemann-Roch space $H^0(X,\cO_X(D))$.

In the fourth section we prove the main result.
This builds on sections two and three, by reducing from a group which does not act faithfully on $H^0(X,\Omega_X^{\otimes m})$, to a subgroup that acts trivially.
We consider the cases where $m=1$ and $m\geq 2$ separately; despite similar methods being employed, there are technical details that need to be changed according to which case is being considered.
These technical details show clearly in the statement: if $m=1$ then the characteristic of $k$ must be 2 for the action to not be faithful, but the genus is not relevant at all.
In contrast, if we consider $m\geq 2$, the characteristic now does not matter, but the genus must be 2 (as does $m$).

In the final section we consider examples to illuminate what has been done in the previous sections.
We start by considering the rather trivial cases where $g_X=0$ and $g_X= 1$.
We give explicit proofs of when the action is faithful, as these cases where not included in the proof of the main theorem.
We then go on to construct a basis for the space of holomorphic poly-differentials for any given hyperelliptic curve.
This serves to explicitly prove the main theorem for this class of curves, and helps to enlighten the reader by way of a concrete example.
In particular, it helps to show why we have a different type of result according to whether $m=1$ or $m\geq 2$.
\end{comment}

\chapter{The Riemann-Hurwitz formula} \label{Chapter:hurwitzformula}

\todo[inline]{Use "T" as a variable for polynomials throughout chapter}

Let $X$ and $Y$ be two projective, non-singular curves over an algebraically closed field $k$, with a map $\pi \colon X \rightarrow Y$ of degree $n$.
We let $g_X$ and $g_Y$ be the genera of $X$ and $Y$ respectively.
Let $K(X)$ and $K(Y)$ be the function fields of $X$ and $Y$ respectively.
Note that $K(X)$ is a degree $n$ extension of $K(Y)$, and recall that $\pi$ induces a map $\pi^*\colon K(Y) \rightarrow K(X)$.
As usual, for any $P \in X$ and $Q \in Y$ we denote the local ring formed by functions which are regular at $P$ and $Q$ by $\cO_{X,P}$ and $\cO_{X,Q}$ respectively.

\section{Galois Theory}\label{sectiongaloistheory}
    
We recall some facts from Galois theory for a field extension $L/K$ of degree $n$.
Any element $\alpha \in L$ defines a $K$-linear map $\mu_{\alpha} \colon  L \rightarrow L$, given by multiplication by $\alpha$.
We let $\tr_{L/K}(\alpha)$ be the trace of the matrix in $\operatorname{Mat}_{n \times n}(K)$ corresponding to $\mu_\alpha$.
Hence we form the {\em trace map} $\tr_{L/K}\colon L \ra K$.
When considering the trace map between the function fields $K(X)$ and $K(Y)$ we let $\tr_{X/Y} := \tr_{K(X)/K(Y)}$.
Note that $\tr_{L/K}$ is an additive map, and that for $\alpha \in K$ we have $\tr_{L/K}(\alpha) = n \alpha$.
    \begin{prop}\label{proptracelemmasurjective}
    The trace map $\tr_{X/Y} \colon K(X) \ra K(Y)$ is surjective.
    \end{prop}
    \begin{proof}
    We first show the equality 
        \[
        \tr_{\overline{K(Y)}\otimes_{K(Y)}K(X)/\overline{K(Y)}} = \id_{\overline{K(Y)}}\otimes _{K(Y)} \tr_{X/Y} \colon \overline{K(Y)}\otimes_{K(Y)}K(X) \ra \overline{K(Y)},
        \]
    with the latter map given by $f\otimes h \mapsto f\tr_{X/Y}(h)$.
    It then follows that the surjectivity of $\id_{\overline{K(Y)}}\otimes _{K(Y)} \tr_{X/Y}$ and $\tr_{\overline{K(Y)}\otimes_{K(Y)}K(X)/\overline{K(Y)}}$ are equivalent.
    We let $h_1, \ldots, h_n$ be a $K(Y)$ basis of $K(X)$, and write $hh_j = \sum_{i=1}^n c_{ij}h_i$ for $c_{ij} \in K(Y)$.
    Then a $\overline{K(Y)}$-basis of $\overline{K(Y)}\otimes_{K(Y)}K(X)$ is given by $1 \otimes h_1, \ldots, 1 \otimes h_n$, and 
        \[
        (f \otimes h)(1\otimes h_j) = f \otimes hh_j = \sum_{i=1}^n c_{ij}(f \otimes h_i) = \sum_{i=1}^n c_{ij}f(1 \otimes h_i).
        \]  
    Hence the matrix corresponding to $\mu_{f \otimes h}$ is $c_{ij}f$, and so
        \[
        \tr_{\overline{K(Y)}\otimes_{K(Y)}K(X)/\overline{K(Y)}}(f \otimes h) = \sum_{i=1}^n c_{ii}f = f \tr_{X/Y}(h).
        \]  

    We now show that $\tr_{\overline{K(Y)}\otimes_{K(Y)}K(X)/\overline{K(Y)}}$ is surjective.
    Since $K(X)$ is a Galois extension of $K(Y)$, and hence also separable, we can find $\alpha$ separable over $K(Y)$ such that $K(X) = K(Y)(\alpha)$.
    We suppose that $\alpha$ has minimal (separable) polynomial $f(z) = (z - \beta_1)\cdots (z - \beta_n)$.
    Since
        \[
        K(X) \cong K(Y)(\alpha) \cong K(Y)(z)/(f(z)) \cong \bigoplus_{i=1}^n K(Y)(z)/((z-\beta_i)) \cong \bigoplus_{i=1}^n K(Y)
        \]  
    then $\overline{K(Y)}\otimes_{K(Y)} K(X) \cong \bigoplus_{i=1}^n \overline{K(Y)}$.
    Now $\tr_{\overline{K(Y)}\otimes_{K(Y)}K(X)/\overline{K(Y)}}$ is the sum of the trace map on each component.
    But the trace map $\tr_{\overline{K}/\overline{K}}$ is the identity.
    Hence $\tr_{\overline{K(Y)}\otimes_{K(Y)}K(X)/\overline{K(Y)}}$ is surjective, and so $\tr_{X/Y}$ is too.
    \end{proof}

    \begin{lem} 
    Suppose that $\alpha \in K(X)$ has minimum polynomial 
        \[
        f(x) = x^r + a_{r-1}x^{r-1} + \ldots +a_0 \in K(Y)[x]
        \]
    and $s= [K(X):K(Y)(\alpha)]$, then $\tr_{X/Y}(\alpha) = -sa_{r-1}$.
    \end{lem}
    \begin{proof}
    We first note that since $s = [K(X):K(Y)(\alpha)]$ it suffices to show the equality $\tr_{K(Y)(\alpha)/K(Y)}(\alpha) = -a_{r-1}$.
    The elements $1, \alpha, \ldots, \alpha^{r-1}$ form a $K(Y)$-basis of $K(Y)(\alpha)$.
    If $i \leq r-2$ then $\mu_{\alpha}(\alpha^i) = \alpha^{i+1}$, and hence the corresponding matrix has 1's below the diagonal and zeroes elsewhere on all columns except the final column.
    Since $\mu_{\alpha}(\alpha^{r-1}) = \alpha^r = -a_0 - a_1\alpha - \ldots - a_{r-1}\alpha^{r-1}$ then the final consists of $-a_0, \ldots, a_{r-1}$.
    Hence the only non-zero diagonal entry is $a_{r-1}$.
    \end{proof}

            
Given a basis $\{z_1,\ldots,z_n\}$ of $K(X)$ over $K(Y)$, we now introduce its dual basis with respect to the trace map.
We denote the {\em dual space} of $K(X)$ over $K(Y)$ by 
    \[
    K(X)^*:=\{\lambda \colon K(X) \rightarrow K(Y)| \lambda\ \text{is}\ K(Y)\text{-linear}\}.
    \]
We make $K(X)^*$ in to a one-dimensional $K(X)$-vector space by defining $z \lambda(w):=\lambda(z w)$.
It follows from Proposition \ref{proptracelemmasurjective} that $\tr_{X/Y}$ is non-zero, and hence there exists a unique $z\in K(X)$ for every $\lambda \in K(X)^*$ such that $\lambda = z\cdot \tr_{X/Y}$.
In particular, if we choose $\lambda_j\in K(X)^*$ such that $\lambda_j(z_i) = \delta_{ij}$ (the Kronecker symbol), then there exist $z_j^*$ such that $\lambda_j = z_j^*\cdot \tr_{X/Y}$.
Hence
    \[
    \tr_{X/Y}(z_iz_j^*) = (z_j^*\cdot \tr_{X/Y})(z_i) = \lambda_j(z_i) = \delta_{ij}.
    \]
The elements $z_1^*, \ldots , z_n^*$ form the {\em dual basis} of $z_1, \ldots , z_n$.


\section{Differentials and divisors}

We first introduce differentials in terms of function fields.
An {\em \adele on $X$} is a map $\alpha\colon  X \rightarrow K(X)$ such $\alpha(P) \in \cO_{X,P}$ for nearly all $P \in X$ (\ie all but a finite number of points).
We can represent $\alpha$ by some $(\alpha_P)_{P \in X} \in \prod_{P \in X} K(X)$, where $\alpha_P := \alpha(P)$.
We let the {\em \adele space} of $X$, denoted by $\cA_X$, be the space of all \adeles on $X$.
There is a canonical injection $K(X) \hookrightarrow  \cA_X$, defined by sending $x\in K(X)$ to the \adele $\alpha$ for which $\alpha_P =  x$ at each point $P\in X$.
The elements of $K(X)$ have natural valuations: for each $P\in X$ we choose a uniformising parameter $t\in \cO_{X,P}$, and then there is a unique $n \in \ZZ$ such that $x=ut^n$ for some unit $u\in \cO_{X,P}$.
We then define the valuation of $x$ at $P$ to be $v(x)_P := n$.
We can we can then extend this valuation to elements of $\cA_X$ by defining $v_P(\alpha) := v_P(\alpha_P)$.

Recall that a divisor on $X$ is a finitely supported formal sum over $X$, with coefficients in $\mathbb Z$, which we write as 
    \[
    D = \sum_{P\in X} n_P [P].
    \]
The degree of such a divisor $D$ is 
    \[
    \deg(D) := \sum_{P\in X} n_P.
    \]
We then define the order of $D$ at $P$ to be $v_P(D) := n_P$.

We let $\cD_X$ be the space of divisors of $X$.
Note that for any $x\in K(X)$ we have a naturally associated divisor, which is
    \[
    \di(x) := \sum_{P\in X} v_P(x) [P].
    \]

For any $D\in \cD_X$ we define the {\em \adele space associated to $D$} to be
    \[
    \cA_X(D):=\{\alpha \in \cA_X | v_P(\alpha) \geq -v_P(D)\ \text{for all} \ P\in X\}.
    \]
Then we define a {\em differential} on $X$ to be a $k$-linear map $\omega\colon  \cA_X \rightarrow k$ such that $\omega$ is zero on $\cA_X(D) +K(X) \subset \cA_X$ for some divisor $D\in \cD_X$.
We will denote the {\em space of differentials on $X$} by $\Omega_{K(X)}$.
    
    \begin{prop}
    The space of differentials, $\Omega_{K(X)}$, is a one-dimensional vector space over $K(X)$.
    \end{prop}
    \begin{proof}
    See \cite[Prop. 1.5.9]{stichtenoth}.
    \end{proof}

    \begin{rem}
    The definition of differential given here is particular to functions fields, and does not obviously correspond to the standard definition in algebraic geometry.
    At the end of the section we collect some results showing the correspondence of these definitions with the more classical definitions.
    \end{rem}

For any non-zero $\omega \in \Omega_{K(X)}$ we define
    \[
    M(\omega) := \{ D\in \cD_X | \omega \ \text{is zero on }\ \cA_X(D) + K(X)\}.
    \]

    \begin{lem}\label{lemmamaximaldivisorassociatedtoomega}
    For any non-zero differential $\omega \in \Omega_{K(X)}$ on $X$ there is a divisor $W$ in $M(\omega)$ such that for any $D \in M( \omega)$ then $D \leq W$.
    \end{lem}
    \begin{proof}
    See \cite[Lem. 1.5.10]{stichtenoth}.
    \end{proof}

For any non-zero $\omega \in \Omega_X$ we call the corresponding divisor from Lemma \ref{lemmamaximaldivisorassociatedtoomega} {\em the canonical divisor associated to $\omega$}, and we associate the zero divisor to $0 \in \Omega_X$.
We denote the canonical divisor associated to $\omega$ by $\di(\omega)$.
We call a divisor a {\em canonical divisor} if it is the canonical divisor associated to some $\omega \in \Omega_{K(X)}$.
    
    \begin{lem}
    Given any two non-zero canonical divisors, $W$ and $W'$, there exists an $x\in K(X)$ such that $W = W' + \di(x)$.
    Conversely, if $W$ is a canonical divisor, and $x\in K(X)$, then $W + \di(x)$ is a canonical divisor.
    \end{lem}
    \begin{proof}
    See \cite[Prop. 1.5.13]{stichtenoth}.
    \end{proof}

\todo[inline]{Introduce the idea of "the" canonical divisor $K_X$ here - maybe later? Maybe not at all in this section.}
We define a valuation $v_P$ of $\omega \in \Omega_{K(X)}$ at $P \in X$ via the associated canonical divisor $W$, by letting $v_P(\omega) := v_P(W)$.
We say that $\omega$ has a pole at $P \in X$ if $v_P(\omega) < 0$ and that $\omega$ has a zero at $P \in X$ if $v_P(\omega) >0$.
Then we call a differential $\omega \in \Omega_{K(X)}$ a {\em holomorphic differential} if it is has no poles (\ie if $v_P(\omega) \geq 0 $ for every $P \in X$).
We denote {\em the space of holomorphic differentials} by $\hzero$.
Also, for any divisor $D$ on $X$ we define 
    \[
    H^0(X,\cO(D)) := \{x\in K(X) | v_P(x) \geq -v_P(D)\}\cup \{0\}.
    \]\todo{change this notation to Fulton's or Stichtenoth's}
This is sheaf theoretic notation, which would not normally be used in function field theory, but we use it for consistency with the rest of the report.\todo{change this sentence after changing notation}
We can now state the celebrated Riemann-Roch theorem.

    \begin{thm}[Riemann-Roch Theorem]\label{theoremriemannroch}
    Let $W$ be a canonical divisor on $X$.
    Then for any divisor $D$ on $X$ 
        \[
        \dim H^0(X,\cO(D)) = \deg(D) + 1 - g_X + \dim H^0(X,\cO(W-D)).
        \]
    \end{thm}
    \begin{proof}
    See, for example, \cite[8.6]{fulton} or \cite[Thm. 1.5.15]{stichtenoth}.
    \end{proof}


    \begin{cor}\label{dim=gc}
    For any canonical divisor $W$ on $X$, we have 
        \[
        \deg(W) = 2g_X-2
        \]
    and 
        \[
        \dim H^0(X,\cO(W)) = g_X.
        \]
    \end{cor}
    \begin{proof}
    Since $\dim H^0(X,\cO(0)) = 1$ (recall that the only functions with no poles are the constant functions), we have by the Riemann-Roch Theorem, 
        \[
        1= \dim H^0(X,\cO(0)) = 0 + 1 -g_X + \dim H^0(X,\cO(W)).
        \]
    Rearranging this gives the second statement.
    The first statement then follows by rearranging
        \[
        g_X = \dim H^0(X,\cO(W)) = \deg(W) + 1 -g_X +  \dim H^0(X,\cO(W-W))= \deg(W) + 1 -g_X + 1.
        \]
    \end{proof}


Given a point $P\in X$ we define $\iota_P\colon K(X) \rightarrow \cA_X$ by
    \begin{equation}
    (\iota_P(x))_Q:= 
        \begin{cases}
        x & \text{if }\ P=Q\\
        0 & \text{otherwise},
        \end{cases}
    \end{equation}
for each $Q\in X$.
We then define $\omega_P\colon K(X) \rightarrow k$ to be the map $\omega_P(x) := \omega(\iota_P(x))$, for $\omega \in \Omega_{K(X)}$.
This is called the {\em local component} of $\omega$; we will use these definitions to prove the following proposition.\todo{change this sentence}

    \begin{prop}\label{propertyofomega}
    Let $\omega$ be a non-zero differential on $X$ and let $P\in X$. Then
        \[
        v_P(\omega) = \max \{r\in \mathbb{Z}|\omega_P(x) = 0\ \text{for all} \ x\in K(X) \ \text{with}\ v_P(x) \geq -r\}.
        \]
    In particular $\omega_P \neq 0$.\todo{make clear that the maximum exists and hence that this is well defined}
    \end{prop}
    \begin{proof}
    Let $W$ be the divisor associated to $\omega$.
    Let $s:=v_P(\omega)$ be the order of $\omega$ at $P$.
    If $x\in K(X)$ and $v_P(x)\geq -s$, then $\iota_P(x) \in \cA_X(W)$, by definition (recall that $0 \in \cA(D)$ for all divisors $D \in \cD_X$).
    Hence $\omega_P(x) = 0$.
    On the other hand, suppose $\omega_P(x) = 0$ for any $x\in K(X)$ satisfying that $v_P(x) \geq -s-1$.
    Let $\alpha = (\alpha_P)_{P \in X} \in \cA_X(W+[P])$.
    Then we have
        \[
        \alpha = (\alpha-\iota_P(\alpha_P)) + \iota_P(\alpha_P).
        \]
    Note that $\alpha - \iota_P(\alpha_P)\in \cA_X(W)$ and $v_P(\alpha_P) \geq -s-1$.
    Hence
        \[
        \omega(\alpha) = \omega(\alpha-\iota_P(\alpha_P))  + \omega_P(\alpha_P) = 0,
        \]
    and so $\omega$ is zero on $\cA_X(W+P)$.
    But this contradicts the maximality of $W$ in its definition.
    \end{proof}

\section{The complementary module}


We now prove some results giving algebraic relations between $K(X)$ and $K(Y)$.
Recall that if $S$ is a subring of $R$, then an element $x\in R$ is {\em integral over $S$} if there is a monic polynomial with coefficients in $S$ for which $x$ is a solution.
Then an {\em integral basis of $R$ over $S$} is a basis of $R$ over $S$, for which each basis elements is integral.
Furthermore, we say that $S$ is integrally closed in $R$ if any element of $R$ which is integral over $S$ lies in $S$.
Recall that $\cO_{X,P}$ is an integrally closed subring of $K(X)$ for any $P\in X$, and that $K(X)$ is the field of fractions of $\cO_{X,P}$.


    \begin{prop}
    For $z\in K(X)$ we let $\phi(T)\in K(Y)[T]$ be its minimal monic polynomial over $K(Y)$.
    Then $z$ is integral over $\cO_{Y,Q}$ if and only $\phi (T)\in \cO_{Y,Q}[T]$.
    \end{prop}
    \begin{proof}
    By definition $\phi(T)$ is the monic irreducible polynomial in $K(Y)[T]$ such that $\phi(z) = 0$. 
    Hence if $\phi (T)$ has coefficients in $\cO_{Y,Q}$ it follows that $z$ is integral over $\cO_{Y,Q}$ by definition.
    
    We now suppose that $z\in K(X)$ is integral over $\cO_{Y,Q}$.
    Then we can choose some monic polynomial $f(T)\in \cO_{Y,Q}[T] \subset K(Y)[T]$ such that $f(z) = 0$.
    Since $\phi(T)$ is minimal over $K(Y)$, then there exists some $\psi(T)\in K(Y)[T]$ such that $ f(T) = \phi(T)\cdot \psi(T)$.\todo{Could add proof of this - consider factorisation in the $K(X)$ adjoin the roots}
    We suppose that $\deg(\phi(T)) = m$ and fix $\alpha_i \in \overline{K(X)}$ such that $\phi(T)$ factorises as 
        \[  
        \phi(T) = (T-\alpha_1)\cdots (T-\alpha_m).
        \]
    We then let $F := K(X)(\alpha_1, \ldots, \alpha_m)$ be a finite extension of $K(X)$, and let $R$ be the integral closure of $\cO_{X,Q}$ in $F$.
    Now the roots of $f(T)$ are in $R$ by definition; in particular, we have $\alpha_1, \ldots, \alpha_m \in R$.
    The coefficients of $\phi(T)$ can be written as polynomials of $\alpha_1, \ldots, \alpha_m$, and hence $\phi(T) \in R[T]$.
    Since $R$ integrally closed $K(Y)\cap R = \cO_{Y,Q}$, and hence $\phi(T)\in \cO_{Y,Q}[T]$.
    \end{proof}

    \begin{cor}\label{traceinclosure}
    Let $Q$ be a point in $Y$ and let $x\in K(X)$ be integral over $\cO_{X,Q}$.
    Then $\tr_{X/Y}(x)\in \cO_{Y,Q}$.
    \end{cor}
    \begin{proof}
    As noted earlier, if $\phi(T)=T^r+a_{r-1}T^{r-1} + \ldots + a_0\in K(Y)[T]$ is the minimal polynomial of $x$ over $K(Y)$, then $\tr_{X/Y}(x)=-n_xa_{r-1}$, where $n_x : = [K(X):K(Y)(x)]$.
    Hence the corollary follows from the previous proposition.
    \end{proof}


We wish to define the ramification divisor, as this is essential for the statement of the Riemann-Hurwitz formula.
In order to do this we first define the complementary module and the ramification index.


    \begin{defn}
    Consider a point $P\in X$ in the pre-image of $Q\in Y$ under $\pi$.
    By \cite[Prop. 3.1.4]{stichtenoth} there is an integer $e_P$ such that for any $x\in K(X)$ the equality $v_P(x) = e_P\cdot v_Q(x)$ holds.
    This value is called the ramification index of $P$.
    If $e_P>1$ then we say that $\pi$ is ramified at $P$.
    \end{defn}

Given this we can associate to each $Q\in Y$ the divisor
    \[
    \pi^*([Q]) := \sum_{P\mapsto Q} e_P [P].
    \]
This can then be extended from a single point to a divisor on $Y$, in which case for a divisor $D = \sum_{Q\in Y}n_Q [Q]$ we have
    \[
    \pi^*(D) := \sum_{Q\in Y}n_Q \pi^*([Q]).
    \]

    \begin{rem}
    It should be noted that in the literature regarding function field theory, what we have denoted by $\pi^*(D)$ is normally called the conorm of $D$ and is denoted $\operatorname{Con}_{X/Y}(D)$. 
    We used the notation above, from algebraic geometry, to be consistent with the rest of the thesis.
    \end{rem}

    For any $Q\in Y$ we let $\cO_{Y,Q}'$ be the integral closure of $\cO_{Y,Q}$ in $K(X)$.

    \begin{defn}
    For any $Q in Y$ we define the complementary module over $\cO_{Y,Q}$ to be
        \[
        C_Q :=\{z\in K(X) | \tr_{X/Y}(z\cdot \cO_{Y,Q}') \subseteq \cO_{Y,Q}\}.
        \]\todo{is it okay to say $\tr_{X/Y}(z\cdot \cO_{Y,Q}')$?}
    \end{defn}


The following proposition lists a number of important properties of $C_Q$.

\begin{prop}\label{propfactsaboutc'}
    Fix $Q\in Y$. 
        \begin{enumerate}
        \item $C_Q$ is an $\cO_{Y,Q}'$-module, and $\cO_{Y,Q}' \subseteq C_Q$.
        \item If $z_1,\ldots ,z_n$ is a (necessarily integral) basis of $\cO_{Y,Q}'$ over $\cO_{Y,Q}$, then 
        \[
        C_Q = \sum_{i=1}^n \cO_{Y,Q}\cdot z_i^*,
        \]
        where $z_1^*, \ldots, z_n^*$ is the dual basis of $z_1, \ldots, z_n$.
        \item There is a $t\in K(X)$ such that $C_Q = t\cdot \cO_{Y,Q}'$ (note that $t$ depends on the choice of $Q$).
        Moreover, $v_P(t) \leq 0$ for all $P\in \pi^{-1}(Q)$, and if $t'\in K(X)$ then $C_Q=t'\cdot \cO_{Y,Q}'$ if and only if $v_P(t) = v_P(t')$ for all $P\in \pi^{-1}(Q)$.
        \end{enumerate}
    \end{prop}

However, before we can give a proof of this proposition, we require two lemmata.

    \begin{lem}[Approximation lemma]\label{lemmaapproximationlemma}
    Let $m$ be a positive integer. 
    For each $i\in \{1,\ldots, m\}$ let $P_i\in X$, let $\mathcal{P}_i$ be the corresponding maximal ideal of $\cO_{P_i}$, let $x_i$ be an element of $K(X)$ and let $n_i$ be an integer.\todo{fix line break}
    Then there is an $x\in K(X)$ such that $v_{P_i}(x-x_i) \geq n_i$ for all $i$.
    \end{lem}
    
    \begin{rem}
    This result can be strengthened to also say that $v_P(x) \geq 0$ for any $P\notin \{P_1,\ldots ,P_m\}$.\todo{fix line break}
    We will not prove this here, for the sake of brevity, but the proof can be found in \cite[Chap. 1, \S 3, pg. 12]{localfields}.
    \end{rem}

    \begin{proof}
    We let $R:= \cO_{P_i} \cap \ldots \cap \cO_{P_m}$, and we let $\mathcal{P}_i' := \mathcal{P}_i \cap R$.
    We first prove the lemma assuming that $x_i \in R$.
    We may increase the $n_i$ such that $n_i\geq 0$ for all $i$.
    By linearity we may assume that $x_2 = \ldots = x_m = 0$, since if we find an element for $x_1$ in this instance, we can similarly find an element for each $i$ and add them.
    Let $I = {\mathcal{P}_1'}^{n_1} + {\mathcal{P}_2'}^{n_2}\cdots {\mathcal{P}_n'}^{n_m}$.
    This is an ideal of $R$, and since it has elements whose valuation at any $P_i$ is zero, it is in fact equal to $R$.
    Hence we can write $x_1 = x + y$, where $y \in {\mathcal{P}_1'}^{n_1}$ and $x\in {\mathcal{P}_2'}^{n_2}\cdots {\mathcal{P}_n'}^{n_m}$.
    Since ${\mathcal{P}_1'}^{n_1} \subseteq {\mathcal{P}_1}^{n_1}$ and ${\mathcal{P}_2'}^{n_2}\cdots {\mathcal{P}_n'}^{n_m} \subseteq \mathcal{P}_2^{n_2}\cdots \mathcal{P}_n^{n_m}$, the $x$ above is as described in the lemma, and this finishes the proof in the case $x_i \in R$.
    
    In general one can write $x_i = \frac{a_i}{b}$ for $a_i\in R$ and $b\in R\backslash \{0\}$, and $x$ can be represented as $\frac{a}{s}$.
    Then we require that $v_{P_1}(a-a_i) \geq n_i + v_{P_i}(s)$ for all $i$ and that $v_P(a)_ \geq v_P(s)$ for all $P\notin \{P_1,\ldots ,P_m\}$.
    But after adding the points $P$ for which $v_P(s)$ is negative, this is precisely what we described in the statement of the lemma.
    \end{proof}

    \begin{lem}\label{lemmapidlemma}
    Let $Q\in Y$, and let $\cO_{X,Q}'$ be the integral closure of $\cO_{X,Q}$ in $K(X)$.
    Then $\cO_{X,Q}'$ is a principal ideal domain.
    \end{lem}
    \begin{proof}
    By \cite[Cor. 3.3.5]{stichtenoth}\todo{check citation}, we have $\cO_{X,Q}' = \{x\in K(X)|v_P(x) \geq 0 \ \text{for all} \ P\in \pi^{-1}(Q)\}$.
    Let $I$ be an ideal of $\cO_{X,Q}'$.
    If we let $\{P_1,\ldots, P_l\} = \pi^{-1}(Q)$ then we can choose $x_i$ for $1\leq i \leq l$ such that $v_{P_i}(x_i) \leq v_{P_i}(y)$ for all $y\in I$.
    By the Approximation Lemma there exist $z_i$ such that $v_P (z_i) = 0$ if $P=P_i$ and $v_{P_j}(z_i) > v_{P_j}(z_j)$ for $j\neq i$.
    Now let $x = \sum_{i=1}^l x_iz_i \in I$.
    Clearly $v_{P_i}(x) = v_{P_i}(x_i)$ for all $1\leq i\leq l$.
    
    Now we show that $I \subseteq x\cO_{X,Q}'$.
    If $y\in I$ then we let $z = x^{-1}y$.
    Then $v_{P_i}(z) = v_{P_i}(y) - v_{P_i}(x_i) \geq 0$ for all $1\leq i\leq l$.
    Hence $z\in \cO_{X,Q}'$ and so $y = xz \in x\cO_{X,Q}'$, completing the proof.
    \end{proof}

We now present a proof of Proposition \ref{propfactsaboutc}.

    \begin{proof}
        \begin{enumerate}
        \item It is clear that $C_Q$ is an $\cO_{X,Q}'$ module.
        Indeed, if $y\in \cO_{X,Q}'$, then $y\cdot \cO_{X,Q}' \subseteq \cO_{X,Q}'$ and hence $\tr_{X/Y}(z y\cdot \cO_{X,Q}')\subseteq \tr_{X/Y}(z\cdot \cO_{X,Q}') \subseteq \cO_{X,Q}$ for any $z\in C_Q$.
        The fact that $\cO_{X,Q}'$ is contained in $C_Q$ follows from Corollary \ref{traceinclosure}.
        \item We first show that $C_Q \subseteq \sum_{i=1}^n \cO_{X,Q}\cdot z_i^*$.
        Suppose $z\in C_Q$.
        Now $\{z_1^*, \ldots ,z_n^*\}$ is a basis of $K(X)$ over $K(Y)$, so there exist $x_1,\ldots , x_n\in K(Y)$ such that $z=\sum_{i=1}^n x_iz_i^*$.
        As $z\in C_Q$ and $z_1,\ldots ,z_n\in \cO_{X,Q}'$, it follows by definition of $C_Q$ that $\tr_{X/Y}(zz_j)\in \cO_{X,Q}$ for $j\in \{1,\ldots ,n\}$.
        We know that 
            \[
            \tr_{X/Y}(zz_j) = \tr_{X/Y}\left(\sum_{i=1}^nx_iz_iz_j^*\right) = \sum_{i=1}^nx_i \cdot \tr_{X/Y}(z_iz_j^*) = x_j,
            \]
        since $z_j^*$ is dual to $z_j$.
        Hence each $x_j$ is in $\cO_{X,Q}$, and $z\in \sum_{i=1}^n\cO_{X,Q}\cdot z_i^*$.
        
        Now suppose that $z\in \sum_{i=1}^n\cO_{X,Q}\cdot z_i^*$ and $u\in \cO_{X,Q}'$.
        Then we need to show that $\tr_{X/Y}(z u)\in \cO_{X,Q}$.
        We can find $x_i, y_j\in \cO_{X,Q}$ such that $z=\sum_{i=1}^n x_iz_i^*$ and $u=\sum_{i=1}^ny_jz_j$.
        Then
            \[
            \tr_{X/Y}(zu) = \tr_{X/Y}\left(\sum_{i,j=1}^n \left(x_iy_jz_i^*z_j\right)\right) = \sum_{i,j=1}^n x_iy_j\cdot \tr_{X/Y}(z_i^*z_j) = \sum_{i=1}^n x_iy_i.
            \]
        Since $\sum_{i=1}^n x_iy_i\in \cO_{X,Q}$, it follows that $z\in C_Q$.
        \item By the previous part, we can find $u_i\in K(X)$ such that $C_Q = \sum_{i=1}^n \cO_{X,Q} \cdot u_i$.
        Choose some $x\in K(Y)$ such that $v_Q(x)\geq 0$ and also $v_Q(x)\geq -v_P(u_i)$ for all $P\in \pi^{-1}(Q)$ and $i\in \{1,\ldots ,n\}$.
        By definition of $e_P$ it follows that
            \[ 
            v_P(xu_i) = e_Pv_Q(x) + v_P(u_i) \geq 0
            \]
        for all $i\in \{1,\ldots, n\}$ and all $P\in \pi^{-1}(Q)$.
        Since $\cO'_Q = \cap_{P\mapsto Q} \cO_{X,P}$ (see, for example, \cite[Cor. 3.3.5]{stichtenoth}), then $x\cdot C_Q \subseteq \cO_{X,Q}'$.
        Since $\cO_{X,Q}'$ is a principal ideal domain by Lemma \ref{lemmapidlemma}, it follows that $x\cdot C_Q = y\cdot \cO_{X,Q}'$ for some $y\in \cO_{X,Q}'$.
        If we let $t=x^{-1}y$ then $C_Q =t\cdot \cO_{X,Q}'$, proving the first part of the statement.
        
        
        
        Since $\cO_{X,Q}'\subseteq C_Q$ then $v_P(t)\leq 0$ for all $P\in \pi^{-1}(Q)$.
        Now $t\cdot \cO_{X,Q}' = t'\cdot \cO_{X,Q}'$ if and only if both $tt'^{-1}$ and $t^{-1}t'$ are in $\cO_{X,Q}'$.
        But this is the case if and only $v_P(tt'^{-1}) \geq 0$ and $v_P(t^{-1}t')\geq 0$ for all $P\in \pi^{-1}(Q)$, which is equivalent to $v_P(t)=v_P(t')$ for all such $P$.
        \end{enumerate}
    \end{proof}

    \begin{prop}\label{almostallqiny}
    For almost all $Q\in Y$ we have $C_Q= \cO_{X,Q}'$.
    \end{prop}
    \begin{proof}
    We first show that, given a basis of $K(X)$ over $K(Y)$, it is almost always integral over $\cO_{X,Q}$ given $Q\in Y$.
    Recall that $\cO_{X,Q}$ is integrally closed in $K(Y)$, and its quotient field is $K(Y)$.
    As before, we denote by $\cO_{X,Q}'$ the integral closure of $\cO_{X,Q}$ in $K(X)$, and we consider a basis $\{z_1,\ldots ,z_n\}$ of $K(X)$ over $K(Y)$.
    Let $\{z_1^*,\ldots ,z_n^*\}$ be the dual basis.
    Now the minimal polynomials of $z_1,\ldots, z_n,z_1^*,\ldots z_n^*$ have finitely many coefficients in $K(Y)$.
    Hence if $S\subseteq Y$ is the set of poles of these coefficients, then $S$ is finite and for $Q\notin S$ then we have
        \[
        z_1,\ldots,z_n,z_1^*,\ldots, z_n^*\in \cO_{X,Q}'.
        \]
    
    Now we assume that $\{z_1,\ldots ,z_n,z_1^*,\ldots z_n^*\}\subseteq \cO_{X,Q}'$ and then we show that 
        \[
        \cO_{X,Q}' \subseteq \sum_{i=1}^n \cO_{X,Q}\cdot z_i^*.
        \]
    
    If $z\in K(X)$ then there are $e_1,\ldots, e_n\in K(Y)$ such that $z=e_1z_1^*+\ldots +e_nz_n^*$.
    If $z\in \cO_{X,Q}'$ then $zz_j\in \cO_{X,Q}'$ for $1\leq j\leq n$, and hence $\tr_{X/Y}(zz_j)\in \cO_{X,Q}$, by Corollary \ref{traceinclosure}.
    As
        \[
        \tr_{X/Y}(zz_j) = \tr_{X/Y}\left(\sum_{i=1}^n e_iz_jz_i^*\right) = \sum_{i=1}^ne_i\cdot \tr_{X/Y}(z_jz_i^*) = e_j.
        \]
    Hence $e_j\in \cO_{X,Q}$, and $\cO_{X,Q}'\subseteq \sum_{i=1}^n\cO_{X,Q}\cdot z_i^*$.
    Since $\{z_1,\ldots, z_n\}$ also forms a basis we can run the same argument again to show that $\cO_{X,Q}'\subseteq \sum_{i=1}^n\cO_{X,Q}\cdot z_i$.
    
    We then have the following set of inequalities:
        \[
        \sum_{i=1}^n \cO_{X,Q}\cdot z_i \subseteq \cO_{X,Q}' \subseteq \sum_{i=1}^n\cO_{X,Q}\cdot z_i^* \subseteq \cO_{X,Q}'\subseteq \sum_{i=1}^n \cO_{X,Q}\cdot z_i.
        \]
    Since $C_Q = \sum_{i=1}^n \cO_{X,Q}\cdot z_i^*$ by part 2 of Proposition \ref{propfactsaboutc'}, the result follows.
    \end{proof}

We can now define the ramification divisor.


    \begin{defn}
    For a point $Q\in X$ choose $t\in K(X)$ such that $C_Q = t\cdot \cO_{X,Q}'$, as in part 3 of Proposition \ref{propfactsaboutc'}.
    Then for any $P\in \pi^{-1}(Q)$ we define the different exponent to be $\delta_P := -v_P(t)$.
    By Proposition \ref{propfactsaboutc'} this is well defined, and by Proposition \ref{almostallqiny} it is almost always zero.
    Hence we can define the ramification divisor to be
        \[
        R := \sum_{P\in X} \delta_P [P].
        \]
    \end{defn}

    \begin{rem}
    It should be noted that when considering the theory of function fields the ramification divisor is called the different, and denoted $\text{ Diff}_{X/Y}$.
    \end{rem}

We now define the \adele space of $X$ over $Y$.

    \begin{defn}
    We define $\mathcal {A}_{X/Y}$ as
        \[
        \mathcal{A}_{X/Y} := \{\alpha \in \mathcal{A}_X | \alpha_P =\alpha_{P'}\ \text{if}\ \cO_{X,P}\cap K(Y) = \cO_{P'}\cap K(Y)\}.
        \]
    We extend the trace function $\tr_{X/Y}\colon K(X)\rightarrow K(Y)$ to a map $\tr_{X/Y}\colon \mathcal{A}_{X/Y} \rightarrow \mathcal{A}_Y$,  by letting
        \[
        (\tr_{X/Y}(\alpha))_Q := \tr_{X/Y}(\alpha_P)
        \]
    for any $\alpha \in \mathcal{A}_{X/Y}$, $Q\in Y$ and $P\in \pi^{-1}(Q)$.
    
    For any divisor $D$ over $X$ we define
        \[
        \mathcal{A}_{X/Y}(D) := \mathcal{A}_X(D) \cap \mathcal{A}_{X/Y}.
        \]
    
    \end{defn}


    \begin{defnthm}\label{detailedhurwitz}
    For every differential $\omega\in H^0(Y,\Omega_Y)$ there is a unique $\omega'\in H^0(X,\Omega_X)$ such that
        \[
        \omega'(\alpha) = \omega\left(\tr_{X/Y}(\alpha)\right)
        \]
    for all $\alpha \in {\mathcal A}_{X/Y}$.
    
    This differential is called the \text{ pullback} of $\omega$, and is denoted $\pi^*(\omega)$. 
    If $\omega\neq 0$ and $(\omega)$ is the associated divisor, then 
        \[
        \di ( \pi^*(\omega)) = \pi^*(\di(\omega)) + R.
        \]
    \end{defnthm}
We first note that this immediately implies the Riemann-Hurwitz formula.




    \begin{cor}\label{hur}[Riemann-Hurwitz Formula]
    Given two non-singular projective curves $X$ and $Y$ of genera $g_X$ and $g_Y$ respectively, with a degree $n$ map $f\colon X \rightarrow Y$, then
        \[
        2g_X - 2 = n(2g_Y -2) + \deg(R),
        \]
    where $R$ is the ramification divisor of $f$.
    \end{cor}
    \begin{proof}
    This follows from Theorem \ref{detailedhurwitz} and Corollary \ref{dim=gc}, after taking degrees.
    \end{proof}


We first give a lemma that is necessary for the proof of the theorem.


    \begin{lem}\label{adelespacelemma}
    For any divisor $D$ over $X$ we have $\mathcal {A}_X = \mathcal{A}_{X/Y} + \mathcal{A}_X(D)$.
    \end{lem}
    \begin{proof}
    Let $\alpha \in \mathcal{A}_X$. Then by the approximation lemma there is an element $x_Q\in K(X)$ for each $Q\in Y$ such that 
        \[
        v_P(\alpha_P - x_Q) \geq -v_P(D)
        \]
    for each $P\in \pi^{-1}(Q)$. 
    We then define the \adele $\beta$ such that $\beta_P := x_Q$ for every $P\in \pi^{-1}(Q)$.
    Then $\beta \in \mathcal{A}_{X/Y}$ and the difference $\alpha - \beta$ is in $\mathcal{A}_X(D)$ by definition of $\beta$.
    Hence $\alpha = \beta + (\alpha - \beta) \in \mathcal{A}_{X/Y} + \mathcal{A}_X(D)$.
    \end{proof}
We now prove Theorem \ref{detailedhurwitz}.

    \begin{proof}
    We first construct a differential $\omega'\in H^0(X,\Omega_X)$ for every $\omega\in H^0(Y,\Omega_Y)$ such that $\omega'(\alpha) = \omega(\tr_{X/Y}(\alpha))$ for every $\alpha \in \cA_{X/Y}$.
    If $\omega = 0$ then we can clearly let $\omega' = 0$, so we assume that $\omega \neq 0$.
    We will use the following divisor, $W' := \pi^*(\di (\omega)) + R$, throughout the proof.
    
    We first prove two assertions about $\omega_1:= \omega\circ \tr_{X/Y} \colon \mathcal{A}_{X/Y} \rightarrow k$.
    Namely, we show that
        \begin{enumerate}[(i)]
        \item For any $\alpha \in \mathcal{A}_{X/Y}(W') + K(X)$ then $\omega_1(\alpha) = 0$;
        \item If $B'$ is a divisor on $X$ with $B' \nleq W'$ then there is a $\beta \in \mathcal{A}_{X/Y}(B')$ such that $\omega_1(\beta) \neq 0$.
        \end{enumerate}
    To show (i) we start by noting that since $\tr_{X/Y}$ and $\omega$ are $k$-linear, $\omega_1$ is too.
    Also, note that since $\omega$ is zero on $K(Y)$ then $\omega_1$ is zero on $K(X)$.
    To show that $\omega_1(\alpha) = 0$ for any $\alpha \in \mathcal{A}_{X/Y}(W')$ then it is sufficient to show that for any $Q\in Y$ and $P\in \pi^{-1}(Q)$ that
        \[
        v_Q(\tr_{X/Y}(\alpha_P)) \geq -v_Q(\omega).
        \]
    
    We choose $x\in K(Y)$ such that $v_Q(x) = v_Q(\omega)$.
    Then
        \begin{align}\label{remark}
        & v_P(x \alpha_P) = v_P(x) + v_P(\alpha_P) \geq e_P v_P(\omega) - v_P(W') \nonumber \\
        & = v_P(\pi^*((\omega)) - W') = -v_P(R) = -\delta_P. 
        \end{align}
    By the definition of $C_Q$ and the ramification divisor, it is clear that $z\in C_Q$ if and only if $v_P(z)\geq -\delta_P$ for all $P\in \pi^{-1}(Q)$.
    It then follows from \eqref{remark} that $x \alpha_P$ is in $C_Q$ and hence that $v_Q(\tr_{X/Y}(x \alpha_P)) \geq 0$, by definition of $C_Q$.
    Since $\tr_{X/Y}(x \alpha_P) = x\cdot \tr_{X/Y}(\alpha_P)$ and $v_Q(x) = v_Q(\omega)$, we have shown the first claim.
    
    To show (ii) we let $Q_0\in Y$ be a point such that there is a $P^*\in \pi^{-1}(Q_0)$ with $v_{P^*}(\pi^*((\omega)) - B') < -\delta_P$.
    We know that such a $Q_0$ exists since $B' \nleq W'$.
    As before we will denote by $\cO_{Q_0}'$ the integral closure of $\cO_{Q_0}$ in $K(X)$.
    We let 
        \[
        J := \{ z\in K(X) | v_{P^*}(z) \geq v_{P^*}(\pi^*(\di (\omega)) - B')\ \text{for all}\ P^*\in \pi^{-1}(Q_0)\}.
        \]
    By the approximation lemma, there exists a $u\in J$ such that 
        \[
        v_{P^*}(u) = v_{P^*}(\pi^*(\di (\omega))-B')
        \]
    for all $P^*\in \pi^{-1}(Q_0)$, and hence $J\nsubseteq C_{Q_0}$.
    (As noted earlier, $z\in \cO_{X,Q}$ if and only if $v_P(z) \geq -\delta_P$).
    It is clear that $J\cdot \cO_{Q_0} \subseteq J$, and hence that $\tr_{X/Y}(J) \nsubseteq \cO_{X,Q}$.
    We let $t$ be an element of $K(Y)$ such that $v_Q(t) = 1$.
    Then there is an $r\in \mathbb N$ such that $t^r\cdot J \subseteq \cO_{X,Q}$, and then 
        \[ 
        t^r\cdot \tr_{X/Y}(J) = \tr_{X/Y}(t^r\cdot J) \subseteq \cO_{X,Q}.
        \]
    It is clear that $t^r\cdot \tr_{X/Y} (J)$ is an ideal of $\cO_{X,Q}$, and so $\tr_{X/Y}(J) = t^s\cdot \cO_{X,Q}$ for some negative integer $s$.
    Hence 
        \begin{equation}\label{traceinring}
        t^{-1}\cdot \cO_{X,Q} \subseteq \tr_{X/Y}(J).
        \end{equation}
    By Proposition \ref{propertyofomega} we can find an $x\in K(Y)$ such that $v_Q(x) = -v_Q(\omega) - 1$ and $\omega_Q(x) \neq 0$.
    If we choose $y\in K(Y)$ such that $v_Q(y) = v_Q(\omega)$, then $xy \in t^{-1}\cO_{X,Q}$.
    Hence by \eqref{traceinring} there is a $z\in J$ such that $\tr_{X/Y} (z) = xy$.
    Let $\beta \in \mathcal{A}_{X/Y}$ be chosen  such that 
        \begin{equation*}
        \beta_P =
            \begin{cases}
            y^{-1}z & \text{if}\ P\in \pi^{-1}(Q) \\
            0 & \text{otherwise}.
            \end{cases}
        \end{equation*}
    Then for any $P\in \pi^{-1}(Q)$ we have
        \begin{align*}
        v_P(\beta) & =  -v_P(y) + v_P(z) \\
        & \geq  -v_P(\pi^*(\di (\omega))) + v_P(\pi^*(\di (\omega)) - \beta) \\
        & =  -v_P(B'),
        \end{align*}
    with the inequality following from the definition of $y$ and $J$.
    Hence $\beta \in \mathcal{A}_{X/Y}(B')$.
    Now $\omega_1(\beta) = \omega(\tr_{X/Y}(\beta)) = \omega_Q(x) \neq 0$.
    This shows (ii).
    
    By Lemma \ref{adelespacelemma} for each $\alpha \in \mathcal{A}_X$ there exists some $\beta \in \mathcal{A}_{X/Y}$ and $\gamma \in \mathcal{A}_X(W')$ such that $\alpha = \beta + \gamma$.
    We now define a new differential $\omega_2 \colon  \mathcal{A}_X \rightarrow k$ by letting $\omega_2(\alpha) := \omega_1(\beta)$.
    Suppose we have two representations of $\alpha$, say $\beta+ \gamma$ and $\beta' + \gamma'$, where $\beta, \beta' \in \mathcal{A}_{X/Y}$ and $\gamma, \gamma' \in \mathcal{A}_{X}(W')$.
    Then 
        \[
        \beta - \beta' = \gamma' - \gamma \in \mathcal{A}_{X/Y} \cap \mathcal{A}_X(W') = \mathcal{A}_{X/Y}(W').
        \]
    It then follows from (i) that 
        \[
        \omega_1(\beta) - \omega_1(\beta') = \omega_1(\beta - \beta') = 0.
        \]
    and hence $\omega_2$ is well defined.
    It is also clear that $\omega_2$ is $k$-linear.
    Also, by the first two points we proved, (i) and (ii), we have:
        \begin{enumerate}[(i$'$)]
        \item $\omega_2(\alpha) = 0$ for all $\alpha \in \mathcal{A}_X(W') + K(X)$.
        \item If $B'$ is a divisor on $X$ such that $B'\nleq W'$ then there is a $\beta \in \mathcal{A}_X(B')$ with $\omega_2(\beta) \neq 0$.
        \end{enumerate}
    
    
    Now it follows that for $\alpha \in \mathcal{A}_{X/Y}$ we have $\omega_2(\alpha) = \omega_1(\alpha) = \omega(\tr_{X/Y}(\alpha))$.
    This means that we have found the $\omega'$ in the statement of the theorem; namely $\omega' = \omega_2$.
    
    It is clear from (i$'$) and (ii$'$) that $\di(\omega') = W' = \pi^*(\di(\omega)) + R$.
    
    We finally prove the uniqueness of $\omega'$.
    Suppose that $\omega''$ also satisfies 
        \[
        \omega''(\alpha) = \omega(\tr_{X/Y}(\alpha))
        \]
    for all $\alpha\in \cA_{X/Y}$.
    So if we let $\theta = \omega'' - \omega'$, then $\theta(\alpha) = 0$ for all $\alpha \in \cA_{X/Y}$.
    But then if we choose a large enough divisor $D$ in Lemma \ref{adelespacelemma}, this implies that $\theta = 0$ and $\cA_X$, and hence $\omega' = \omega''$.
    \end{proof}





Let $C$ be a smooth, projective, connected algebraic curve over $k$.
We now show that our two different definitions of $\Omega_{K(C)}$ give rise to isomorphic differentials.
Recall that $K(x)$ is the field of rational functions of the projective line.
By \cite[Prop. 1.7.4]{stichtenoth} there exists a unique differential $\omega \in \Omega_{K(x)}$ such that $\di (\omega) = -2[P_{\infty}]$ and $\omega(\iota_\infty(x^{-1})) = -1$, where $P_\infty \in \mathbb P_k^1$ is the point at infinity on the projective line.


Now for any $z\in K(C) \backslash k$ we have that $k(z)$ is isomorphic to $K(x)$.
Hence if we let $G$ be the Galois group of the extension $[K(C):k(z)]$ then the quotient curve $Y:=X/G$ is isomorphic to the projective line.
We let $f_z\colon C \rightarrow \mathbb P_k^1$ be the corresponding surjective map, and we denote the unique differential described above by $\omega_z$.
We then define $\delta\colon K(C) \rightarrow \Omega_{K(C)}$ to be the map such that if $z\in K(C)\backslash k$ then $\delta (z) := f_z^*(\omega_z)$ and if $y\in k$ then $\delta(y):=0$.
This then induces a map $\mu\colon \Omega_{K(C)} \rightarrow \Omega_{K(C)}$, defined by $z\cdot dx \mapsto z\cdot \delta(x)$·

    \begin{thm}
    The map $\mu\colon \Omega_{K(C)} \rightarrow \Omega_{K(C)}$ is an isomorphism.
    \end{thm}
    \begin{proof}
    See \cite[Thm. 4.3.2]{stichtenoth}.
    \end{proof}

We now define the order of a poly-differential at a point.
If we consider an element of the tensor product $\omega \in \Omega_X^{\otimes m}$ then it can be locally written as $y dx_1\otimes \ldots \otimes dx_m$, where $x_i \in K(X)$ for all $1 \leq i \leq m$.
Let $P$ be a point in $X$.
Since each $dx_i$ can be written as $y_i dt$ for some $y_i\in K(X)$ and some uniformising parameter $t$ at $P$, we can rewrite $\omega$ as $y' dt \otimes \ldots \otimes dt$, where $y' = y \cdot y_1 \cdots y_m$.
We then define the order of $\omega$ at $P$ to be $\ord_P(\omega ) := \ord_P(y')$.
In the particular case where $\omega = fdx \otimes \ldots \otimes fdx = f^m dx^{\otimes m}$, then we have $y_1 = \ldots = y_m = z$ for some $z$ when we change $x$ to a uniformising parameter.
Hence in this instance 
    \[ 
    \ord_P(\omega) = \ord_P(z^m) = m\ord_P(z) = m\ord_P(dx).
    \]





It should be noted that in algebraic geometry differentials are normally defined in a different, but equivalent, manner.
Given a ring $K(X)$ and an $K(X)$-module $M$, we call any $K(X)$-linear map $D\colon K(X)\rightarrow M$ satisfying 
    \[
    D(ab) = aD(b) + D(a)b
    \]
for all $a,b\in K(X)$ a {\em derivation}.
There is a unique module, denoted $\Omega_{K(X)}$, with a map $d\colon K(X) \rightarrow \Omega_{K(X)}$, which every derivation must factor through; \ie if $D\colon K(X)\rightarrow M$ is a derivation then there is a unique map $f\colon \Omega_{K(X)}\rightarrow M$ such that $D = f\circ d$.\todo{change $f$}
Given any differential $\omega$, we can write it as $fdx$ for some $x$ and some $f$ in $K(X)$.
This is how we will consider differentials in the other sections of this report.
At the end of this section we will describe an isomorphism between $\Omega_X$ and $\Omega_{K(X)}$.



\chapter{Faithful actions on Riemann-Roch spaces} \label{Chapter:Faithfulactions}

In this chapter our main aim is to compute when a subgroup of the automorphism group of an algebraic curve acts faithfully on the space of global holomorphic differentials and polydifferentials.
Our approach uses the fact that if any finite group $G$ does not act faithfully on $H^0(X,\Omega_X^{\otimes m})$ then there exists a subgroup of $G$ which fixes at least one element of this $k$ vector space.

Given this, it would be useful to determine whether the fixed space is non-zero, and for this reason we start by computing the dimension of the fixed space $H^0(X,\Omega_X^{\otimes m})^G$.
We discover (Proposition \ref{dim}) that the dimension relies primarily on the genus of the quotient curve $Y:=X/G$, $m$ and the ramification divisor of $\pi \colon X \ra Y$.

Then we use this formula, along with results from the second chapter, to compute exactly when a cyclic group of prime order will act trivially on $H^0(X,\Omega_X^{\otimes m})$.
When we are considering holomorphic differentials (\ie when $m=1$), this depends solely on the characteristic of $k$, whilst for polydifferentials (\ie when $m \geq 2$) this is actually independent of $\cha (k)$, and is determined by the genus of $X$, $m$ and the order of the group.
In the same section we also extend these results to more general Riemann-Roch spaces.

We then move on to the main theorem (Theorem \ref{theoremfaithfulaction}), which answers the question of when $G$ acts faithfully on $H^0(X,\Omega_X^{\otimes m})$.
After proving this theorem we give examples which illustrate both when we do and do not have faithful actions.
In particular, we use results of Chapter 3 to explicitly show the result holds for hyperelliptic curves.

We close the chapter with an alternative proof of when a cyclic group of prime order acts faithfully on $\hzero$, by studying the $k[G]$-module structure of $\hzero$, which was determined in \cite{valmadan}.

\section{Dimension formulae}\label{dimsection}

Throughout this chapter, unless otherwise stated, we assume that $X$ is a connected, smooth, projective algebraic curve over an algebraically closed field $k$ of characteristic $p \geq 0$.
We furthermore assume that $G$ is a finite group of order $n$ that acts faithfully on $X$.
Note that $G$ also induces an action on the vector space $H^0(X,\Omega_X^{\otimes m})$ of global holomorphic poly-differentials of order $m$.
We let $Y$ denote the quotient curve $X/G$, and we let $\pi\colon X\rightarrow Y$ be the canonical projection.
Finally, we denote by $g_X$ and $g_Y$ the genus of $X$ and $Y$ respectively, and we let $K_X$ and $K_Y$ be canonical divisors on $X$ and $Y$.\todo{put comment in previous section regarding divisors}


In this section we compute the dimension of $H^0(X,\Omega_X^{\otimes m})$ and of $H^0(X,\Omega_X^{\otimes m})^G$, the subspace of $H^0(X,\Omega_X^{\otimes m})$ fixed by $G$.
We first recall that by Definition \ref{definitiongenus} we have $\dim_kH^0(X,\Omega_X)=g_X$.
We also computed the dimension of $H^0(X,\Omega_X^{\otimes m})$ when $g_X,m \geq 2$ in Lemma \ref{dim3}.
Finally, as we will see in examples (a) and (b) in Section \ref{sectionexamples}, if $g_X$ is zero or one then $\dim_k(H^0(X,\Omega_X^{\otimes m})= g_X$, for all $m \in \ZZ\geq 1$.


We now introduce some notations. 
Let $D=\sum_{P\in X}n_P[P]$ be a $G$-invariant divisor on $X$ (\ie $n_{\sigma(P)} = n_P$ for all $\sigma \in G$ and $P\in X$) and let $\cO_X(D)$ denote the corresponding equivariant invertible $\cO_X$-module. 
Furthermore, let $\pi_*^G(\cO_X(D))$ denote the sub-sheaf of the direct image $\pi_*(\cO_X(D))$ fixed by the obvious action of $G$ on $\pi_*(\cO_X(D))$.
We also let $\left\lfloor \frac{\pi_*(D)}{n} \right \rfloor$ denote the divisor on $Y$ obtained from the push-forward $\pi_*(D)$ by replacing the coefficient $m_Q$ of $Q$ in $\pi_*(D)$ with the integral part $\left \lfloor \frac{m_Q}{n} \right \rfloor$ of $\frac{m_Q}{n}$ for each $Q \in Y$. 
The function fields of $X$ and~$Y$ are denoted by $K(X)$ and $K(Y)$ respectively. 
Finally, for any $P \in X$ let $\ord_P$ and $\ord_Q$ denote the respective valuations of $K(X)$ and $K(Y)$ at $P$ and $Q:=\pi(P)$.

The next lemma is the main idea in the proof of our formula for $\dim_kH^0(X,\Omega_X^{\otimes m})^G$, see Proposition \ref{dim}. 

    \begin{lem}
    Let $D=\sum_{P\in X}n_P[P]$ be a $G$-invariant divisor on $X$.
    Then the sheaves $\pi_*^G(\cO_X(D))$ and $\cO_Y\left(\left\lfloor \frac{\pi_*(D)}{n}\right \rfloor\right)$ are equal as subsheaves of the constant sheaf $K(Y)$ on $Y$. 
    In particular, the sheaf $\pi_*^G(\cO_X(D))$ is an invertible $\cO_Y$-module.
    \end{lem}
    \begin{proof}
    For every open subset $V$ of $Y$ we have 
        \[
        \pi_*^G(\cO_X(D))(V) = \cO_X(D) (\pi^{-1}(V))^G \subseteq K(X)^G = K(Y).
        \]
    In particular, both sheaves are subsheaves of the constant sheaf $K(Y)$ as stated. 
    It therefore suffices to check that their stalks are equal. 
    For any $Q \in Y$ and $P \in \pi^{-1}(Q)$.
    We have
        \begin{align*}
        \pi_*^G\left(\cO_X(D)\right)_Q & = \cO_X(D)_P \cap K(Y) \\
        &= \left\{f \in K(Y): \ord_P(f) \ge -n_P\right\}\\
        &= \left\{f \in K(Y): \ord_Q(f) \ge - \frac{n_P}{e_P}\right\}\\
        &= \left\{ f \in K(Y): \ord_Q(f) \ge - \left\lfloor\frac{n_P}{e_P} \right\rfloor \right\}\\
        &= \cO_Y\left(\left\lfloor \frac{\pi_*(D)}{n} \right\rfloor\right)_Q,
        \end{align*}
    as desired.
    \end{proof}

The following proposition contains the aforementioned formula for the dimension of the subspace of $H^0(X,\Omega_X^{\otimes m})$ fixed by $G$.
In particular we see that this dimension is completely determined by $m$, $g_Y$ and $\deg \left\lfloor \frac{m\pi_*(R)}{n} \right\rfloor$.

    \begin{prop}\label{dim}
    Let $m\geq 1$. Then the dimension of $H^0(X,\Omega_X^{\otimes m})^G$ is equal to
        \[
        \dim_k \left( H^0(X,\Omega_X^{\otimes m})^G \right) = (2m-1)(g_Y-1) + \deg\left\lfloor\frac{m\pi_*(R)}{n} \right\rfloor,
        \]  
    unless 
        \begin{itemize}
        \item $m=1 \mbox{ and } \deg\left\lfloor\frac{m\pi_*(R)}{n}\right\rfloor = 0$ or
        \item $g_Y=1 \mbox{ and } \deg\left\lfloor\frac{m\pi_*(R)}{n}\right\rfloor = 0$ or
        \item  $g_Y=0 \mbox{ and } \deg\left\lfloor\frac{m\pi_*(R)}{n}\right\rfloor < 2m-1$,
        \end{itemize}
    in which case 
        \[
        \dim_k \left( H^0(X,\Omega_X^{\otimes m})^G \right) = g_Y.
        \]      
    \end{prop}
    \begin{proof}
    Let $E$ denote the divisor $\left\lfloor \frac{\pi_*(mK_X)}{n} \right\rfloor$ on $Y$. As $K_X=\pi^*(K_Y)+R$ we have
        \[ 
        E = \left \lfloor \frac{\pi_*\pi^*(mK_Y) + \pi_*(mR)}{n} \right \rfloor = mK_Y + \left \lfloor \frac{m\pi_*(R)}{n} \right \rfloor.
        \]
    Using the previous lemma we conclude that $\pi_*^G(\Omega_X^{\otimes m}) \cong \cO_Y (E)$ and finally that
        \begin{equation*}
        \dim_k H^0(X,\Omega_X^{\otimes m})^G = \dim_k H^0\left(Y, \pi_*^G(\Omega_X^{\otimes m})\right) = \dim_k H^0\left(Y, \cO_Y\left( E \right) \right).
        \end{equation*}
    
    
    In the first case of the proposition, \ie if $m=1$ and $\deg \left\lfloor\frac{m\pi_*(R)}{n} \right\rfloor=0$, then $\left\lfloor\frac{m\pi_*(R)}{n} \right\rfloor$ is the zero divisor and we conclude that 
        \begin{equation*}
        \dim_kH^0(X,\Omega_X)^G = \dim_kH^0(Y, \Omega_Y) = g_Y.
        \end{equation*}
    
    
    In the second case $\left\lfloor \frac{m\pi_*(R)}{n} \right\rfloor$ is again the zero divisor. 
    Furthermore, as $g_Y=1$, the divisor $K_Y$ is equivalent to the zero divisor, and hence $mK_Y$ is too. 
    This means that
        \begin{equation*}
        \dim_kH^0(X,\Omega_X^{\otimes m})^G = \dim_kH^0\left( Y,\cO_Y\left( E \right) \right) = \dim_k  H^0\left( Y,\cO_Y\left( 0 \right) \right) = 1.
        \end{equation*}
    
    
    For the third case, by \cite[Chap. IV, ex. 1.3.4]{hart} it suffices to show that $\deg \left( E \right) < 0$.
    As $g_Y=0$ we have $\deg(K_Y)=-2$, so $\deg(mK_Y)=-2m$, and $\deg \left( E \right)$ is indeed negative.
    
    
    
    We will show below that in all other cases $\deg(E) > \deg(K_Y)$, and then the Riemann-Roch formula (Theorem \ref{theoremriemannroch}) will give 
        \begin{align*}
        \dim_kH^0(X,\Omega_X^{\otimes m})^G  & = \dim_kH^0\left(Y,\cO_Y\left( E \right)\right) \\
        & =  1-g_Y+\deg\left(mK_Y+\left\lfloor{\frac{m\pi_*(R)}{n}}\right\rfloor\right) \\
        & =  (2m-1)(g_Y-1)+\deg\left\lfloor{\frac{m\pi_*(R)}{n}}\right\rfloor,
        \end{align*}
    completing the proof for this case.
    
    
    All that remains is to show that $\deg(E)>\deg(K_Y)$ in all other cases.
    Firstly, if $g_Y=0$ and $\deg \left\lfloor\frac{m\pi_*(R)}{n} \right\rfloor \geq 2m-1$ then, since $\deg(mK_Y)=-2m$, we have 
        \[
        \deg \left( E \right) \geq -1 >-2 = \deg(K_Y).
        \]
    Similarly, if $g_Y=1$ and $\deg \left\lfloor\frac{m\pi_*(R)}{n} \right\rfloor >0$ then, as $\deg \left( mK_Y \right)=0$, we have $\deg \left( E \right) > 0 = \deg (K_Y)$.
    If $m=1$ and $\deg \left\lfloor\frac{m\pi_*(R)}{n} \right\rfloor >0$ then clearly $\deg \left( E \right) > \deg (K_Y)$.
    Lastly, if $m\geq 2$ and $g_Y\geq 2$ then $\deg (K_Y) > 0$ and we have 
        \begin{equation*}
        \deg \left( E \right) \geq \deg\left( mK_Y \right) > \deg (K_Y).
        \end{equation*}
    So in all other cases $\deg(E)>\deg(K_Y)$, and the proof is complete.
    \end{proof}


If $m=1$ we reformulate Proposition \ref{dim} in the following slightly more concrete way. 
Let $S$ denote the set of all points $Q\in Y$ such that $\pi$ is not tamely ramified at $Q$ and let $s$ denote the cardinality of $S$. 
Note that $s=0$ if $p$ does not divide $n$.


For the next corollary we recall the notations $e_Q$ and $\delta_Q$ for any $Q\in Y$ defined before Theorem \ref{hilbertsformula}.


    \begin{cor}\label{dim2}
    We have 
        \begin{equation*}
        \dim_kH^0(X,\Omega_X)^G = 
            \begin{cases}
            g_Y & \mbox{if } s=0, \\
            g_Y-1+\sum_{Q\in S}\left\lfloor \frac{\delta_Q}{e_Q} \right\rfloor & \mbox{otherwise}.
            \end{cases}
        \end{equation*}
    \end{cor}
    \begin{proof}
    We have
        \[
        \deg\left\lfloor\frac{\pi_*(R)}{n} \right\rfloor = \sum_{Q\in Y}\left\lfloor\sum_{P\mapsto Q} \frac{\delta_P}{n} \right\rfloor = \sum_{Q\in Y} \left\lfloor \frac{\delta_Q}{e_Q} \right\rfloor.
        \]
    Furthermore we have $\left\lfloor \frac{\delta_Q}{e_Q} \right\rfloor = 0$ if and only if $\delta_Q<e_Q$, \ie if and only if $Q\notin S$. 
    Thus Corollary \ref{dim2} follows from Proposition \ref{dim}.
    \end{proof}

    \begin{rem}
    Note that if $p>0$ and $G$ is cyclic then Corollary \ref{dim2} can be derived from Proposition $6$ in the recent pre-print \cite{kako} by Karanikolopoulos and Kontogeorgis.\todo{check to see if preprint is now published}
    \end{rem}



\section{Trivial action in the cyclic case}

In this section we will look at the case where $G$ is a cyclic group of prime order, or a power of a prime, and determine when $G$ acts trivially on $H^0(X,\Omega_X^{\otimes m})$.
Compared to arbitrary groups, it is considerably easier to compute when these groups act trivially, and we will later see that we can reduce to this case, regardless of what the structure of $G$ is.


Throughout this section, $P_1,\ldots ,P_r \in X$ denote the ramification points of $\pi$ and we write $e_i$ and $\delta_i$ for $e_{P_i}$ and $\delta_P{_i}$.
Also, for $i=1, \ldots, r$, we define $N_i \in \NN$ by $\ord_{P_i}(\sigma(\pi_i) - \pi_i) = N_i +1$, where $\pi_i$ is a local parameter at the ramification point $P_i$ and $\sigma$ is a generator of $G(P_i)$. 
We also assume that $g_X \geq 2$.


    \begin{prop}\label{m=1}
    Let $p  > 0$ and let $G$ be cyclic of order $p$.
    Furthermore, we assume that $g_Y=0$.
    Then $G$ acts trivially on $H^0(X,\Omega_X)$ if and only if $p=2$. 
    \end{prop}
    \begin{proof}
    From \cite[Lem. 1]{Naka} we know that $p$ does not divide $N_i$ for $i=1,\ldots ,r$, a fact we will use several times below. \todo{more specific citation}
    Let $N:= \sum_{i=1}^r N_i$. 
    Using the Riemann-Hurwitz formula, Corollary \ref{corhurwitzformula}, we obtain
        \begin{equation}\label{hur2}
        2g_X - 2 = -2p + (N+r)(p-1)
        \end{equation}
    and hence
        \[
        \dim_kH^0(X,\Omega_X) = g_X =\frac{(N+r-2)(p-1)}{2}.
        \] 
    Since $g_X \ge 0$ we obtain $r \ge 1$; that is, $\pi$ is not unramified. 
    As $\cha(k) = p = \ord(G)$, the morphism $\pi$ is not tamely ramified, and the cardinality $s$ defined before Corollary \ref{dim2} is not zero.
    Therefore we have 
        \[
        \deg \left\lfloor \frac{\pi_*(R)}{p} \right\rfloor =
        \sum_{i=1}^r \left\lfloor \frac{(N_i+1)(p-1)}{p}\right\rfloor 
        \ge \sum_{i=1}^r \left\lfloor \frac{2(p-1)}{p}\right\rfloor = r > 0.
        \] 
    From Corollary \ref{dim2} we then conclude that 
        \begin{align*}
        \dim_kH^0\left(X,\Omega_X\right)^G & =  g_Y - 1 + \sum_{i=1}^r\left\lfloor \frac{\delta_i}{e_i}\right\rfloor \\
        & =  -1 + N + r \sum_{i=1}^r\left\lfloor -\frac{N_i+1}{p}\right\rfloor.
        \end{align*}
    
    If $p=2$, the dimension of both $H^0(X,\Omega_X)$ and $H^0(X,\Omega_X)^G$ is therefore equal to $\frac{N+r-2}{2}$. 
    This shows the if-direction in Proposition \ref{m=1}.
    
    
    
    To prove the other direction we now assume that $G$ acts trivially on $H^0(X, \Omega_X)$.
    For each $i=1, \ldots, r$, we write $N_i = s_i p +t_i$ with $s_i \in \NN$ and $t_i \in \{1, \ldots, p-1\}$. 
    We furthermore put $S:=\sum_{i=1}^r s_i$ and $T:= \sum_{i=1}^r t_i \ge r$. 
    Then we have
        \[ 
        \frac{(N+r-2)(p-1)}{2} =\dim_kH^0(X,\Omega_X)  = \dim_k H^0(X,\Omega_X)^G = N-S-1 .
        \]
    Rearranging this equation we obtain
        \[
        (3-p)N - 2S = (r-2)(p-1) +2  
        \]
    and hence
        \[
        (-p^2 + 3p -2)S = (r-2)(p-1) +2 - (3-p)T.
        \]
    Assuming that $p \ge 3$ this equation implies that
        \[ 
        S = \frac{(r-2)(1-p)-2 + T (3-p)}{(p-1)(p-2)}. 
        \]
    since $-p^2+3p-2 = - (p-1)(p-2)$. 
    
    Because $S \geq 0$, the numerator of this fraction is non-negative, that is
        \begin{align*}
        0 & \le (r-2)(1-p) - 2 + T (3-p)\\ 
        & \le  (r-2)(1-p) - 2 + r (3-p)\\
        &= 2 (r-1)(2-p).
        \end{align*}
    Hence we have that $r=1$ and that the numerator is $0$. 
    We conclude that $S=0$ and hence that $T=1$ or $p=3$. 
    If $T=1$ we also have $N=1$ and finally
        \[
        g_X = \frac{(N+r-2)(p-1)}{2} = 0,
        \]
    a contradiction.
    If $T \not=1$ and $p=3$ we obtain $N=T=2$ and finally 
        \[
        g_X = \frac{(N+r-2)(p-1)}{2} =1,
        \] 
    again a contradiction.
    \end{proof}

    \begin{prop}\label{triv}
    Let $m \geq 2$. 
    Suppose that $G$ is a cyclic group of prime order $l$ (which may or may not be equal to $p$) and that $g_Y=0$. 
    Then $G$ acts trivially on $H^0(X,\Omega_X^{\otimes m})$ if and only if $g_X=m=l=2$.
    \end{prop}
    \begin{proof}
    We have different proofs according to whether or not the order $l$ of the group is the same as the characteristic $p$ of the field.
    
    
    First we assume that $l=p$. 
    As in the proof of Proposition \ref{m=1}, we let $N=\sum_{i=1}^r N_i$, and we let $M=N+r$.
    Then due to (\ref{hur2}) we have
        \begin{equation}\label{simplehur}
        2g_X-2=-2p+M(p-1),
        \end{equation}
    and combining this with Lemma \ref{dim3} we can write
        \begin{equation}\label{altdim2}
        \dim_kH^0(X,\Omega_X^{\otimes m})=(2m-1)(g_X-1)=(2m-1)\left(\frac{M(p-1)-2p}{2}\right).
        \end{equation}
    
    Furthermore, we have
        \begin{equation}\label{altdim}
        \deg\left\lfloor \frac{m\pi_*(R)}{p} \right\rfloor = \sum_{i=1}^r\left\lfloor \frac{m(N_i+1)(p-1)}{p} \right\rfloor  = mM + \sum_{i=1}^r\left\lfloor \frac{-m(N_i+1)}{p} \right\rfloor.
        \end{equation}
    If we have $p=g_X=m=2$, then on the one hand we see that $\dim_kH^0(X,\Omega_X^{\otimes m}) =3$. 
    On the other hand, we first note that \eqref{simplehur} implies $M=6$.
    So 
        \begin{equation*}
        \deg\left\lfloor \frac{m\pi_*(R)}{p}\right\rfloor = 2M -M =6 > 3 = 2m-1.
        \end{equation*}  
    Then, by Proposition \ref{dim}, we obtain 
        \begin{equation*}
        \dim_kH^0(X,\Omega_X^{\otimes m})^G = (2m-1)(g_Y-1)+\deg\left\lfloor \frac{m\pi_*(R)}{p} \right\rfloor = -3 + 6 = 3.
        \end{equation*}
    So the two dimensions are equal and the action of $G$ on $H^0(X,\Omega_X^{\otimes m})$ is trivial. 
    This completes the if direction of the proof.
    
    Now we assume that the action is trivial. This first implies that 
    $\deg \left\lfloor\frac{m\pi_*(R)}{p}\right\rfloor \geq 2m-1$ because otherwise we would 
    have $\dim_kH^0(X,\Omega_X^{\otimes m})^G=0$ by Proposition \ref{dim}, but we know that 
    $\dim_kH^0(X,\Omega_X^{\otimes m})=(2m-1)(g_X-1)$ is strictly positive.\todo{rewrite sentence}
    So, using \eqref{altdim}, \eqref{altdim2} and Proposition \ref{dim} we see that
        \begin{align}\label{bound}
        (2m-1)\frac{M(p-1)-2p}{2} & = \dim_kH^0(X,\Omega_X^{\otimes m}) \nonumber\\
        & =  \dim_kH^0(X,\Omega_X^{\otimes m})^G \nonumber\\
        & =  1-2m+mM+\sum_{i=1}^r\left\lfloor\frac{-m(N_i+1)}{p}\right\rfloor\nonumber \\
        & \leq  1-2m+mM+\sum_{i=1}^r\frac{-m(N_i+1)}{p}\nonumber \\
        & =  1-2m+mM-\frac{mM}{p}.
        \end{align}
    
    After multiplying by $2p$ and rearranging we obtain
        \begin{align}\label{times2p}
        0 & \geq  (2mM-M-4m+2)p^2+(-4mM+M-2+4m)p+2mM \nonumber \\
            & =  (M-2)(2m-1)p^2-((M-2)(2m-1)+2mM)p+2mM \nonumber \\
        & =  (p-1)((M-2)(2m-1)p-2mM).
        \end{align}
    
    Furthermore from \eqref{hur2} we obtain that $-2p+M(p-1)=2g_X-2 \geq 2$ and hence that 
        \begin{equation}\label{greater2}
        M\geq \frac{2+2p}{p-1}=2+\frac{4}{p-1}>2.
        \end{equation}
    
    So from \eqref{times2p} and \eqref{greater2} we see that
        \begin{align}\label{plessthan4}
        p & \leq  \frac{2mM}{(M-2)(2m-1)}\nonumber\\
        & =  \frac{M}{M-2}\cdot\frac{2m}{2m-1}\nonumber\\
        & =  \left( 1+\frac{2}{M-2} \right) \left(1+\frac{1}{2m-1} \right)\\
        & \leq  4, \nonumber	
        \end{align}
    \ie $p=2$ or $p=3$. 
    
    Suppose that $p=3$. Then from \eqref{greater2} we have $M\geq 4$. However, from  \eqref{plessthan4} we also have that 
        \begin{align*}
        3 & \leq \left( 1+\frac{2}{M-2} \right) \left(1+\frac{1}{2m-1} \right)\\
        & \leq  \left( 1+\frac{2}{M-2} \right) \frac{4}{3}\\
        & \leq  \frac{8}{3},
        \end{align*}
    a contradiction.
    
    Lastly, we come to the case when $p=2$. From \eqref{plessthan4} we see that $2\leq \left(1+\frac{2}{M-2}\right)\frac{4}{3}$ 
    and hence $M\leq 6$. However, from \eqref{greater2} we know that $M\geq 6$, so $M=6$. Then from \eqref{bound}  we obtain that $2m-1=1-2m+6m-3m$
    and hence that $m=2$. Finally, (\ref{hur2}) gives us that $2g_X-2=-4+6=2$ and hence $g_X=2$. 
    This completes the only if direction of the proof when $l=p$.
    
    Now if $l\neq p$ then we know that all the coefficients $\delta_i$ of the ramification divisor are equal to $l-1$. 
    To show the if direction in this case, first note that $\dim_kH^0(X,\Omega_X^{\otimes m})=3$ by Lemma~\ref{dim3}. 
    On the other hand, the Riemann-Hurwitz formula (Corollary \ref{corhurwitzformula}) implies that $2 = 2g_X-2=-2l+\deg(R)=-2l+r(l-1)$, and hence that $r=6$. 
    Finally Proposition \ref{dim} gives us
        \begin{equation*}
        \dim_kH^0(X,\Omega_X^{\otimes m})^G = -(2m-1) + \sum_{i=1}^r \left\lfloor \frac{m\cdot \delta_i}{l} \right\rfloor
        = -3 +\sum_{i=1}^6 \left\lfloor \frac{m(l-1)}{l} \right\rfloor
        = 3,
        \end{equation*}
    since $m=l=2$.
    As the dimensions of $H^0(X,\Omega_X^{\otimes m})$ and $H^0(X,\Omega_X^{\otimes m})^G$ are equal, the action is trivial.
    
    
    Now, for the final section of the proof we suppose that $G$ acts trivially on the space $H^0(X,\Omega_X^{\otimes m})$.
    We then show that this implies that $g_X=l=m=2$.
    
    
    From Lemma \ref{dim3} and Proposition~\ref{dim} we obtain
        \begin{align*}
        (2m-1)(g_X-1)& =\dim_kH^0(X,\Omega_X^{\otimes m}) \\
        & =  \dim_kH^0(X,\Omega_X^{\otimes m})^G\\
        & =-(2m-1)+\sum_{i=1}^r \left\lfloor \frac{m\cdot \delta_i}{l} \right\rfloor
        \end{align*}
    and hence
        \begin{equation*}
        (2m-1)g_X = \sum_{i=1}^r \left\lfloor \frac{m\cdot \delta_i}{l} \right\rfloor
        = \sum_{i=1}^r \left\lfloor \frac{m(l-1)}{l} \right\rfloor
        = r\left( m+\left\lfloor \frac{-m}{l} \right\rfloor \right).
        \end{equation*}
    By choosing $s\in \{1,\ldots ,l\}$ and $q\in \mathbb{N}$ such that $m=ql+s$ we can rewrite this as
        \begin{equation}\label{eq:mult}
        (2m-1)g_X=r(m-q-1).
        \end{equation}
    If we multiply (\ref{eq:mult}) by $l-1$ and then substitute in for the $r(l-1)$ term in the Riemann-Hurwitz formula (Corollary \ref{corhurwitzformula}) we get
        \begin{equation*}
        (2m-1)(l-1)g_X=(2g_X+2(l-1))(m-q-1).
        \end{equation*}
    By rearranging we are able to compute $g_X$ in terms of $m,l$ and $q$:
        \begin{align}\label{equationgxintermsofmandlandq}
        g_X & = \frac{2(l-1)(m-q-1)}{(2m-1)(l-1)-2(m-q-1)} \nonumber \\
        & =  1 + \frac{2(m-q-1)-(2q+1)(l-1)}{(2m-1)(l-1)-2(m-q-1)}  \nonumber\\
        & =  1 + \frac{2s-1-l}{(2m-1)(l-1)-2(m-q-1)} \nonumber  \\
        & =  1 + \frac{2(s-1)+1-l}{(2m-1-2q)(l-1)-2(s-1)}. 
        \end{align}
    First, we show that if $l\geq 3$ the equation cannot hold whilst $g_X\geq 2$.
    Observe that the denominator is strictly greater than $l-1$, remembering that $m=ql+s$:
        \begin{align*}
        (2m-1-2q)(l-1)-2(s-1) & =  ((2q(l-1)+2s-1)(l-1)-2(s-1) \\
        & \geq  (2s-1)(l-1)-2(s-1) \\
        & \geq  (2s-1)(l-1)-2(s-1)(l-1) \\
        & =  l-1;
        \end{align*}
    here the two inequalities are equalities if and only if $q=0$ and $s=1$, respectively, and, as $m\geq 2$, not both inequalities can be equalities.
    Now the numerator is at most $l-1$, occurring when $s=l$. 
    Hence if $l\geq 3$ the fraction in \eqref{equationgxintermsofmandlandq} will be less than one and $g_X < 2$, contradicting our assumption.
    If $l=2$, then $s$ is either 1 or 2.
    If $s=1$ the fraction is negative, and $g_X<1$, which again contradicts our assumption.
    Finally, if $s=2$ then $g_X\leq 2$, with equality if and only if $q=0$, \ie~if and only if $m=2$.
    So if $g_X \geq 2$ then the action being trivial implies that $g_X=l=m=2$, and the proof is complete.    
    \end{proof}

For the rest of this section we assume that $p>0$ and that $G$ is a cyclic group of order $p^l$ for some $l \in \NN$.
What we are now going to do will not be used in the proof of the main theorem, but is included because it generalises the previous results.
More precisely, we do not restrict ourselves to looking at $H^0(X,\Omega_X^{\otimes m})$, but using a comparatively deep result from \cite{kako} we study $H^0(X,\cO(D))$ for any $G$-invariant divisor $D$ such that $\deg(D)>2g_X-2$.


We first introduce some notation.
Let $D = \sum_{P\in X} n_P[P]$ be a $G$-invariant divisor on $X$.
Then let $\langle a \rangle$ denote the fractional part of any $a\in \mathbb{R}$, \ie $\langle a \rangle = a - \lfloor a \rfloor$.
Also, for any $Q\in Y$ let $n_Q$ be equal to $n_P$ for any $P\in \pi^{-1}(Q)$.




    \begin{prop}\label{nakaj}
    Suppose $p>0$ and $G$ is a cyclic group of order $p^l$ for some $l\geq 1$.
    Let $D$ be a $G$-invariant divisor on $X$ such that $\deg(D)>2g_X-2$.
    Then the action of~$G$ on $H^0(X,\cO_X(D))$ is trivial if and only if
        \[ 
        (p^l-1)\deg(D)=p^l\left(g_X-g_Y-\sum_{Q\in Y}\left\langle \frac{n_Q}{e_Q} \right\rangle\right).
        \]
    \end{prop}
    \begin{proof}
    We first remind the reader of the notation in \cite{kako}.
    Let $\sigma$ be a generator of $G$.
    Let $V$ be the $k[G]$ module with $k$-basis $e_1,\ldots ,e_{p^l}$ and $G$-action defined by $\sigma( e_i)=e_i+e_{i-1}$, $1\leq i \leq p^l,\ e_0=0$.\todo{changed from $\sigma \cdot e_i$ to how is now. Check rest of work for issues}
    Then $V_j$, defined to be the subspace of $V$ spanned by $e_1,\ldots ,e_j$ over $k$, is also a $k[G]$ module.
    In fact, the modules $V_1,\ldots ,V_{p^l}$ form a complete set of representatives for the set of isomorphism classes of indecomposable $k[G]$-modules. For each $j=1,\ldots,p^l$ let $m_j$ denote the multiplicity of $V_j$ in the $k[G]$-module $H^0(X,\cO_x(D))$, \ie we have $H^0(X,\cO_x(D))\cong \oplus_{j=1}^{p^l}m_jV_j$.
    
    
    
    First note that the action of $G$ on $H^0(X,\cO_X(D))$ is trivial if and only if
        \begin{equation}\label{triva}
        \dim_k H^0(X,\cO_X(D))^G =\dim_k H^0(X,\cO_X(D)).
        \end{equation}
    
    It is clear that the $G$-invariant part of each submodule $V_j$ is spanned by $e_1$. 
    Hence $\dim_kH^0(X,\cO_X(D))^G = \sum_{j=1}^{p^l} m_j$.
    By \cite[Thm. 2.1]{quaddiffequi}, which relies on \cite{cohogsheaves}, we have
        \begin{align*}
        \sum_{j=1}^{p^l} m_j & =  1- g_Y +\sum_{Q\in Y} \left\lfloor \frac{n_Q}{e_Q}\right\rfloor\\
        & =  1- g_Y + \sum_{Q\in Y} \left( \frac{n_Q}{e_Q} - \left\langle \frac{n_Q}{e_Q}\right\rangle \right) \\
        & =  1 - g_Y + \frac{1}{p^l}\deg(D) - \sum_{Q\in Y} \left\langle \frac{n_Q}{e_Q} \right\rangle.
        \end{align*}
    
    Now as $\deg(D)>2g_X-2$ we have $\dim_kH^0(X,\cO_X(D)) =\deg(D)+1-g_X$ by the Riemann-Roch theorem. 
    So the action of $G$ on $H^0(X,\cO_X(D))$ is trivial if and only if
        \begin{equation*}
        \deg(D)+1-g_X  = 1 - g_Y + \frac{1}{p^l}\deg(D) - \sum_{Q\in Y}\left\langle \frac{n_Q}{e_Q} \right\rangle. \label{hi}
        \end{equation*}
    
    This then rearranges to $(p^l-1)\deg(D)=p^l\left(g_X-g_Y-\sum_{Q\in Y}\left\langle \frac{n_Q}{e_Q} \right\rangle\right)$, as desired.
    \end{proof}

    \begin{cor}\label{this}
    Suppose that $\deg(D)\geq 2g_X$. Then the action of $G$ on $H^0(X,\cO_X(D))$ is trivial if and 
    only if $g_Y = 0$, $e_Q | n_Q$ for all $Q\in Y$, $\deg(D)=2g_X$ and either $g_X=0$ or $p^l=2$.
    \end{cor}
    \begin{proof}
    The following inequalities always hold under the stated assumptions:
        \begin{multline}
        (p^l-1)\deg(D)\geq (p^l-1)2g_X \geq p^lg_X \geq p^lg_X-p^l\sum_{Q\in Y}\left\langle\frac{n_Q}{e_Q}\right\rangle \\ \geq p^l\left( g_X - g_Y -\sum_{Q\in Y}\left\langle \frac{n_Q}{e_Q} \right\rangle \right).
        \end{multline}
    Now the first inequality is an equality if and only if $\deg(D)=2g_X$. 
    The second is an equality if and only if either $g_X=0$ or $p^l=2$. 
    The third inequality is an equality if and only if $\sum_{Q\in Y}\left\langle\frac{n_Q}{e_Q}\right\rangle=0$, which is the case if and only if each $n_Q$ is divisible by~$e_Q$. 
    Lastly, the fourth inequality is an equality if and only if $g_Y = 0$.
    Given these observations, Proposition \ref{nakaj} implies Corollary~\ref{this}.
    \end{proof}

The following Corollary slightly strengthens the only if direction of the $l=p$ part of Proposition \ref{triv}
(from $\ord(G) = p$ to $\ord(G) = p^l$) and also provides a different proof for it;
note that this new proof relies on the comparatively deep result result in section 7 of \cite{cohogsheaves}.


    \begin{cor}
    Let $m \geq 2$ and let $G$ be a cyclic group of order $p^l$ for some $l$. 
    If $G$ acts trivially on $H^0(X,\Omega_X^{\otimes m})$, then $g_Y = 0$ and $p^l = g_X = m = 2$.
    \end{cor}
    \begin{proof}
    As $g_X \geq 2$ and $m\geq 2$ we have $\deg(mK_X) \geq 2g_X$. 
    So, if the action of $G$ on $H^0(X,\Omega_X^{\otimes m})$ is trivial, we obtain from Corollary \ref{this} that $\deg(mK_X) = 2g_X$, $p^l = 2$ and $g_y = 0$.
    Now $\deg (mK_X) = 2g_X$ implies that $m(2g_X -2 ) = 2g_X$, so $m(g_X -1) = g_X$ and hence $m=g_X=2$.
    \end{proof}

Similarly to the case $\deg(D)\geq 2g_X$ in Corollary \ref{this}, the following corollary derives necessary and sufficient conditions for trivial action from Proposition \ref{nakaj} in the case $\deg(D) =2g_X-1$.



    \begin{cor}
    Suppose that $\deg(D)= 2g_X-1$ and that $g_Y=0$. Then the action of $G$ on $H^0(X,\cO_X(D))$ is trivial if and only if one of the following conditions hold:
        \begin{itemize}
        \item  $p^l=2$ and $\sum_{Q\in Y}\left\langle\frac{n_Q}{e_Q}\right\rangle=\frac{1}{2}$;
        \item  $g_X=2$, $p^l=3$ and $e_Q\mid n_Q$ for all $Q\in Y$.
        \end{itemize}
    \end{cor}


    \begin{rem}
    It can easily be shown that in the last case the Riemann-Hurwitz formula implies that $r\leq 4$. 
    Furthermore, if $r=1$ then the conditions ``$\sum_{Q\in Y}\left\langle\frac{n_Q}{e_Q}\right\rangle=\frac{1}{p^l}$" and ``$e_Q\mid n_Q$ for all $Q\in Y$" are already implied by ``$\deg(D)=2g_X-1$".
    \end{rem}

    \begin{proof}
    Firstly, if $g_X=0$ then $\deg(D)=-1<0$, so $\dim_kH^0(X,\cO_X(D))=0$ and the action is trivial.
    
    Now note that, as $\deg(D)=2g_X-1$, we conclude from Proposition \ref{nakaj} that the action is trivial if and only if 
        \begin{equation*}
        (p^l-1)(2g_X-1)=p^l\left(g_X-\sum_{Q\in Y}\left\langle\frac{n_Q}{e_Q}\right\rangle\right).
        \end{equation*}
    If $p^l=2$ then this is equivalent to $2g_X-1=2g_X-2\sum_{Q\in Y}\left\langle\frac{n_Q}{e_Q}\right\rangle$ and hence to $\sum_{Q\in Y}\left\langle\frac{n_Q}{e_Q}\right\rangle=\frac{1}{2}$.
    
    If $g_X=1$ then this is equivalent to $p^l-1=p^l-p^l\sum_{Q\in Y}\left\langle\frac{n_Q}{e_Q}\right\rangle$ and hence is also equivalent to $\sum_{Q\in Y}\left\langle\frac{n_Q}{e_Q}\right\rangle=\frac{1}{p^l}$.
    
    Lastly, if $p^l\geq 3$ and $g_X\geq 2$ then we have that $g_X\geq \frac{p^l-1}{p^l-2}$ which is equivalent to the first inequality in the chain
        \begin{equation*}
        (p^l-1)(2g_X-1)\geq p^lg_X\geq p^lg_X-p^l\sum_{Q\in Y}\left\langle\frac{n_Q}{e_Q}\right\rangle \geq p^l\left( g_X - g_Y -\sum_{Q\in Y} \left\langle \frac{n_Q}{e_Q} \right\rangle \right).
        \end{equation*}
    Hence the action is trivial if and only if both inequalities are equalities, which is the case if and only if $p^l=3,\ g_X=2$, $e_Q\mid n_Q$ for all $Q\in Y$ and $g_Y = 0$.
    \end{proof}


\section{The main theorem}\label{maintheoremsection}
In this section we prove the main theorem of this chapter, describing exactly when $G$ will act faithfully on $H^0(X,\Omega_X^{\otimes m})$.


    \begin{thm}\label{theoremfaithfulaction}
    Suppose that $g_X\geq 2$ and let $m\geq1$. 
    Then $G$ does not act faithfully on $H^0(X,\Omega_X^{\otimes m})$ if and only if $G$ contains a hyperelliptic involution and one of the following two sets of conditions holds:
        \begin{itemize}
        \item $m=1$ and $p=2$;
        \item $m=2$ and $g_X=2$.
        \end{itemize}
    \end{thm}
    \begin{proof}
    We first show the if direction. 
    In the case when $m=1$, the hyperelliptic involution contained in $G$ generates a subgroup of order $2$.
    Since $p=2$, this acts trivially by Proposition \ref{m=1}, and hence $G$ does not act faithfully.
    In the case when $m=2$, then again looking at the subgroup generated by the hyperelliptic involution, we have a group of order $2$ acting on $H^0(X,\Omega_X^{\otimes m})$.
    So, by Proposition \ref{triv} and since $g_X=m=2$, the action of this subgroup is trivial, and again, this means that $G$ does not act faithfully.
    
    
    We now start the proof of the only if direction, supposing that $G$ does not act faithfully on $H^0(X,\Omega_X^{\otimes m})$. 
    By replacing $G$ with the (non-trivial) kernel $H$ if necessary, we may assume that $G$ is non-trivial and acts trivially on $H^0(X,\Omega_X^{\otimes m})$.
    
    
    We start the proof by showing that $g_Y=0$, which is shown separately for the cases when $m=1$ and when $m\geq 2$.
    In the case when $m=1$ we start by showing that $\deg  \left\lfloor \frac {\pi_*(R)}{n} \right\rfloor >0$ by contradiction.
    Suppose that $\deg\left\lfloor \frac{\pi_*(R)}{n} \right\rfloor =0$.
    As $G$ acts trivially it follows from Proposition~\ref{dim} that:
        \begin{equation*}
        g_X=\dim_k H^0(X,\Omega_X)=\dim_k H^0(X,\Omega_X)^G=g_Y.
        \end{equation*}
    Substituting this into the Riemann-Hurwitz formula (Corollary \ref{corhurwitzformula}) yields the desired contradiction because $g_X\geq 2, n\geq 2$ and $\deg(R)\geq 0$.
    
    Thus $\deg\left( \left\lfloor \frac{\pi_*(R)}{n} \right\rfloor \right) >0$. 
    Now Proposition~\ref{dim} gives us that
        \begin{equation*}
        g_X=\dim_k H^0(X,\Omega_X)=\dim_k H^0(X,\Omega_X)^G= g_Y-1+\deg\left\lfloor \frac{\pi_*(R)}{n} \right\rfloor.
        \end{equation*}
    Substituting this in to the Riemann-Hurwitz formula we see that
        \begin{equation*}
        2\left(g_Y - 1 + \deg\left \lfloor \frac{\pi_*(R)}{n} \right \rfloor -1 \right) = 2n (g_Y -1) + \deg(R).
        \end{equation*}
    For any $Q \in Y$ we let $\delta_Q$ denote the coefficient of the ramification divisor $R$ at any $P \in \pi^{-1}(Q)$ and let $e_Q := e_P$ for any $P \in \pi^{-1}(Q)$. 
    Rewriting the previous equation then yields
        \begin{align*}
        (2n-2)g_Y & = 2n-4 + 2 \,\deg\left \lfloor \frac{\pi_*(R)}{n}\right \rfloor - \deg(R)\\
        &= 2 \left(n-2 + \sum_{Q \in Y} \left(\left\lfloor \frac{n}{e_Q} \frac{\delta_Q}{n} \right\rfloor - \frac{n}{e_Q} \frac{\delta_Q}{2}\right) \right)\\
        &= 2 \left(n-2 + \sum_{Q \in Y} \left( \left\lfloor \frac{\delta_Q}{e_Q} \right\rfloor - \frac{\delta_Q}{e_Q} \frac{n}{2} \right)\right)\\
        & \le  2(n-2),
        \end{align*}
    because $\frac{n}{2} \ge 1$ and $\left\lfloor \frac{\delta_Q}{e_Q}\right\rfloor \le \frac{\delta_Q}{e_Q}$ for all $Q \in Y$. 
    Hence we obtain $g_Y \le \frac{n-2}{n-1} < 1$ and therefore $g_Y =0$, as desired.
    
    We now show that $g_Y=0$ when $m\geq 2$. 
    Since $g_X\geq 2$ we have that $\deg(mK_X)=m(2g_X-2)>2g_X-2=\deg(K_X)$.
    By Lemma \ref{dim3}, and as both $m$ and $g_X$ are at least 2, then $\dim_kH^0(X,\Omega_X^{\otimes m})^G=\dim_kH^0(X,\Omega_X^{\otimes m})=(2m-1)(g_X-1)>1$.
    There is only one case in Proposition \ref{dim} such that $m\geq 2$ and $\dim_k H^0(X,\Omega_X^{\otimes m})^G>1$, which yields 
        \begin{equation*}
        (2m-1)(g_X-1)=(2m-1)(g_Y-1)+\deg\left(\left\lfloor \frac{m\pi_*(R)}{n} \right\rfloor \right).
        \end{equation*}
    Combining this with the Riemann-Hurwitz formula, Corollary \ref{corhurwitzformula}, we see that
        \begin{align*}
        2(2m-1)(g_Y-1)+2\deg\left(\left\lfloor\frac{m\pi_*(R)}{n}\right\rfloor\right) & =  2(2m-1)(g_X-1)\\
        & =  2n(2m-1)(g_Y-1)+(2m-1)\deg(R),
        \end{align*}
    which can be re-arranged as
        \begin{equation*}
        (2m-1)(2n-2)(g_Y-1)=2\deg\left(\left\lfloor\frac{m\pi_*(R)}{n}\right\rfloor\right)-(2m-1)\deg(R).
        \end{equation*}
    So if we can show that the right hand side of this equation is negative then we will have $g_Y-1<0$ and hence $g_Y=0$, as desired.
    
    Using the same notation as in the case when $m=1$, we calculate:
        \begin{align*}
        2\deg\left(\left\lfloor\frac{m\pi_*(R)}{n}\right\rfloor\right)-(2m-1)\deg(R) & = \sum_{Q \in Y} \left(2\left\lfloor m\cdot \frac{n}{e_Q}\frac{\delta_Q}{n}\right\rfloor -n(2m-1)\frac{\delta_Q}{e_Q}\right) \\
        & \leq   \sum_{Q\in Y}\left( 2m\cdot\frac{\delta_Q}{e_Q}-n(2m-1)\frac{\delta_Q}{e_Q}\right) \\
        & =  (2m-n(2m-1))\sum_{Q\in Y }\frac{\delta_Q}{e_Q}.
        \end{align*}
    
    Now as $n,m\geq 2$ then we have $2m-n(2m-1)\leq 2m-2(2m-1)=2(1-m)<0$ and we are done as $\sum_{Q\in Y}\frac{\delta_Q}{e_Q}$ is positive.
    
    So we have shown for all $m\geq 1$, if the group $G$ acts trivially  on $H^0(X,\Omega_X^{\otimes m})$ then $g_Y=0$.
    Now if $m\geq 2$ then first note that $G$ must contain a cyclic subgroup of prime order, say $H$, such that $H$ acts trivially on $H^0(X,\Omega_X^{\otimes m})$.
    Now Proposition \ref{triv} tells us that $m=g_X=2$, and that the order of $H$ must also be 2.
    Hence $X/H\cong \mathbb{P}_k^1$, and this completes the only if direction for $m\geq 2$.
    
    Similarly, the $m=1$ case of the only if direction will follow from Proposition \ref{m=1} after we show that $p>0$ and there is a cyclic subgroup of $G$ of order $p$. 
    This is true since $\pi$ cannot be tamely ramified.
    Indeed, if it were then $R=\sum_{P\in X} (e_P-1)[P]$ \cite[Chap. IV, Cor. 2.4]{hart}, and $\deg\left\lfloor \frac{\pi_*(R)}{n} \right\rfloor=0$, which we have already shown cannot be the case.
    Hence $p$ must be positive, and there is a cyclic subgroup of order $p$ which acts trivially.
    \end{proof}

    \begin{rem}
    Note that the existence of a hyperelliptic involution $\sigma$ in $G$ means not only that the genus of $X/\langle \sigma \rangle$, but also the genus of $Y=X/G$, is $0$ (by the Riemann-Hurwitz formula).
    If, moreover $p=2$, then the canonical projection $X\rightarrow X/\langle \sigma \rangle$ is not unramified (again by the Riemann-Hurwitz formula) and hence not tamely ramified; then $\pi$ cannot be tamely ramified either.
    \end{rem}


\section{Examples}\label{sectionexamples}
We will now give some examples of a finite group acting on a curve, and the consequent action on the holomorphic poly-differentials. 
We start with some examples in which $G$ acts trivially on $H^0(X,\Omega_X^{\otimes m})$.
We then follow this with the example of hyperelliptic curves, for which we compute an explicit basis of $H^0(X,\Omega_X^{\otimes m})$, allowing us to see when the action is trivial.


\subsection{Trivial Examples}\label{examplessection}


(a) Let $g_X = 0$, \ie $X\cong \mathbb P_k^1$.
Then $\deg(K_X) = -2$ and so $\deg(mK_X) < 0$ for $m~\geq~1$.
Hence $H^0(X,\Omega_X^{\otimes m}) =\{0\}$ by \cite[Lem. 2, pg. 295]{hart}\todo{check citation} and $G$ acts trivially on $H^0(X,\Omega_X^{\otimes m})$ for all $m\geq 1$.

(b) Let $g_X = 1$, \ie $X$ is an elliptic curve.
If $G$ is a finite subgroup of $X(k)$ acting on $X$ by translations, then $G$ leaves invariant any global non-vanishing holomorphic differential $\omega$ and hence $G$ acts trivially on $H^0(X,\Omega_X)$;
since $\omega^{\otimes m}$ is a basis of $H^0(X,\Omega_X^{\otimes m})$ this means that $G$ acts trivially on $H^0(X,\Omega_X^{\otimes m})$ for all $m\geq 1$.

If $p>0$ and $G$ is a $p$-group, then the multiplicative character $G\rightarrow k^*$ afforded by the one-dimensional representation $H^0(X,\Omega_X^{\otimes m})$ of $G$ has to be trivial because $k$ doesn't contain any $p^{\mbox{th}}$ roots of unity;
in particular the action of $G$ on $H^0(X,\Omega_X^{\otimes m})$ is trivial as well.
On the other hand, if $p\neq 2$ and $X$ is given by the Weierstrass equation of the form $y^2 = f(x)$, then the involution $\sigma \colon  (x,y) \rightarrow (x,-y)$ maps the invariant differential $\omega = \frac{dx}{y}$ to $-\omega$.



Since $\sigma$ acts trivially on $x$, it is clear that in both the $m=1$ case and the case where $m=g=2$ that the group action is trivial.
On other hand, since $\sigma(y) = y+h(x)$, we can see that in the other cases the action is not trivial, as there are $y$ coefficients of the basis elements.
Hence this proves the main theorem in the case of hyperelliptic curves in characteristic two.


\section{$K[G]$-module structure of $\hzero$ when $|G|=p$}
Let $X$ be a smooth connected projective algebraic curve over an algebraically closed field $k$ of characteristic $p>0$, on which the cyclic group $G$ of prime order $p$ acts faithfully.
This induces an action of $G$ on $H^0(X,\Omega_X)$, and we will compute when this action is trivial.
This can be done with the paper of Valentini and Madan \cite{valmadan} when the genus of $X$ is at least $2$.

We start by phrasing the question in the language of the paper; rather than the curve $X$ we will refer to the corresponding function field $F$, and we will write $\Omega_F$ instead of $H^0(X,\Omega_X)$.
The subspace of $F$ fixed by $G$ (equivalently, the quotient of $X$ by the action of $G$) will be denoted $E$, and it's space of holomorphic differentials $\Omega_E$.
Finally, we will denote the genus of $E$ and $F$ by $g_E$ and $g_F$ respectively, and assume that $g_E=0$.

Note that the paper assumes $|G|=p^n$ for some $n \geq 1$, so where appropriate the value of $n$ will be substituted with $1$, after specifying that the substitution has been made.
 
Let $\sigma$ be a generator of $G$. 
There are $p$ unique indecomposable representations of $G$, which can be written $K[G]/(\sigma - 1)^k$, for $k\in \{1,\ldots, p\}$.
The trivial action corresponds to $k=1$ and the regular representation corresponds to $k=p$.
For a decomposition of the $G$-representation of $\Omega_F$ in to indecomposable representations, we will denote by $d_k$ the number of times the representation of degree $k$ occurs.
So if $\Omega_F = \oplus_{m=1}^t \Omega_m$ for some $t\leq g_F$ is a decomposition of $\Omega_F$ in to a direct sum of indecomposable submodules $\Omega_m$, then $d_k$ is the number of components of the sum isomorphic to $K[G]/(\sigma -1)^k.$
Note that if the action is trivial then the trivial representation will be the only indecomposable submodule. 
As such, showing that $G$ acts trivially is equivalent to showing that $d_1 = g_F$, and $d_k= 0 $ for $k \geq 1$.
We start by defining the following subspaces of $\Omega_F$,
\begin{equation*}
	\Omega_F^i = \{\omega\in \Omega_F | (\sigma - 1)^i\omega=0\}\ \mbox{ for } 0\leq i \leq p.
\end{equation*}
This is an increasing sequence of subspaces, from $0$ to $\Omega_F$, and we can see that $\dim_k\Omega_F^i = \sum_{m=1}^t\dim(\Omega_F^i \cap \Omega_m)$, where $\Omega_m$ are the same indecomposable submodules as earlier.
Now note that for the indecomposable module $k[G]/(\sigma - 1)^k$ it is true that 
\[
 \dim_k\{v \in k[G]/(\sigma - 1)^k | (\sigma - 1)^iv = 0\} = \left\{ \begin{array}{ll}
                                                                      i & i\leq k \\
\\
								      k & i > k
                                                                     \end{array} \right. \mbox{\textbf{prove this!}}
\]

Therefore $\dim_k\Omega_F^i = \sum _{k=1}^{i-1} kd_k + \sum_{k=i}^p id_k$, and it follows that 
\[
dim_k(\Omega_F^{i+1}/\Omega_F^i) = \sum_{k=i+1}^p d_k.
\]

As we wish to find $d_k$, we rewrite this as 
\begin{eqnarray*}
d_p & = & \dim_k(\Omega_F^p/\Omega_F^{p-1}) \\
d_k & = & \dim_k(\Omega_F^k/\Omega_F^{k-1}) - \dim_k(\Omega_F^{k-1}/\Omega_F^{k-1}).
\end{eqnarray*}
Now to compute $\dim_k(\Omega_F^i/\Omega_F^{i-1})$ we look at the extension $F/E$.


The extension $F/E$ is a degree $p$ Artin extension, so it has generation of the form 
\[
F=E(y), \mbox{	} y^p-y = b, \mbox{	} b\in E.
\]

Note that in the paper, there are $n$ intermediary fields, labelled $E_j$, each of which is an Artin-Schreier extension with equivalently labelled $b_j$ and $y_j$.
The $b$ and $y$ we are using correspond to $b_1$ and $y_1$ respectively, and $F$ and $E$ correspond to $E_1$ and $E_0$ respectively.


We can choose $b$ and $y$ to satisfy the conditions of Lemma 5.1 of appendix 5 of \cite{quaddiffequi}, which states:


\begin{lem}\label{koeck}
Let $L/K$ be a totally ramified Galois extension of degree $p$.
Then there exists an element $y \in L$ whose valuation is coprime to $p$ and
negative, say $-m$, such that $y^p - y \in K$ and $L = K(y)$. The greatest
integer $M$ such that the higher ramification group $G_M$ of $L/K$ does not vanish
is then equal to $m$.
\end{lem}


At this point in the paper, for $1\leq k \leq p^n-1$, $a_i^k$ is defined to be the co-efficient of $p^i$ in the p-adic expansion of $k$. 
Since we have $n=1$, we only have $a_0^k = k$ in this expansion, a fact we will later use.

Let $\bar{P_1},\dots , \bar{P_s}$ be the primes of $E$ that ramify, and let $P(i,1,1)$ denote the prime ideal above $\bar{P_i}$, in $F$.
(Note that we are using this notation to be consistent with the paper. 
There, the authors used $P(i,j,m)$, where $i$ denoted the prime ideal in $E$ it was above, $j$ denoted which extension the ideal was in, and $m$ was used to differentiate between these.
We set $j=1$ as $n=1$, as already commented.
Also, since the order of $G$ is prime, we may only have one prime above each $\bar{P_i}$, so $m=1$.


Now for each $\bar{P_i}$ the normalised valuation determined by $\bar{P_i}$ applied to $z\in E$ is denoted $v(i,1,1,z)$.
(Again, note that in the paper the authors use $v(i,j,m,z)$ to denote the valuation of a $z\in E_j$ determined by $P(i,j,m).$
Here however we will only require the valuation in $E$.)
We let $e_i$ be the ramification index of $\bar{P_i}$ and let $r=n-\max(e_i)$.
Since $n=1$, it is clear that $e_i=1$ for all $i$, unless the extension is unramified, and hence $r=0$.
Again, this is not necessarily the case if $n\geq 2$, and but we will use the notation of the paper.

From the proof of Lemma 2 in Valentini and Madan, if we set $\Phi(i,j) = \Phi(i,1) = -v(i,1,1,b)$, then we have the following formula to determine the exponent of the different at $P(i,1,m)$:

\begin{eqnarray}\label{eq}
\delta_i & = & (p-1)\sum_{j=n-e_i+1}^n (\Phi(i,j) + 1)p^{n-j} \\
	    & = & (p-1)(\Phi(i,1) + 1).
\end{eqnarray}

To determine $d_k$ in terms of this, then we set 
\[
v_{ik} = \left\lfloor \frac{\delta_i - k\Phi(i,1)}{p} \right\rfloor,
\]
for $0\leq k \leq p-1$, where $\lfloor c \rfloor$ denotes the largest integer not exceeding $c$ for any $c\in \mathbb{R}$.

For each $k$ we then denote the sum of these values of all the ramification points by $ \Gamma_k = \sum_{i=1}^s v_{ik}.$

Now we can apply Theorem $1$ in Valentini and Madan, which states:
\begin{thm}
Let $G$ be a cyclic group of automorphisms of $F$ of order $p^n$. 
Let $E$ be the fixed field of $G$ with $g_E$ its genus.
The regular representation of $G$ occurs $g_E-1+\alpha$ times in the representation of $G$ on $\Omega_F$, with $\alpha = 1$ if $r=0$ and $\alpha = 0$ otherwise.
For $k=1,\ldots p^n-1$, the indecomposable representation of degree $k$ occurs $\Gamma_{k-1}-\Gamma_k + \alpha_k$ times, with $\alpha_k = 1$ if $k= p^n-p^r +1$, $\alpha_k = -1$ if $k=p^n-p^r$ and $\alpha_k = 0$ otherwise.
\end{thm}

Now note that $g_E=0$ by assumption, and $r=0$ as commented earlier, so the degree of the regular representation is zero.

Now suppose that $p>2$ and that $g_X\geq 2$.
Suppose also that the action is trivial.


We first observe that $\Gamma_{p-1} = 0$. Indeed
\[
 v_{i(p-1)} = \delta_i - (p-1)\Phi(i,1) = (p-1)(\Phi(i,1) + 1) - (p-1)\Phi(i,1) = p-1,
\]
and $\left \lfloor \frac{p-1}{p} \right\rfloor = 0$. This gives $\Gamma_{p-1} = \sum_{i=1}^s 0 = 0$, as desired.

Now if we were to assume that the action is trivial, then for all $k\neq 1$ the representation of degree $k$ should not occur and it should be true that $d_k = 0$.
So if $k= p-1$ we have $\Gamma_{p-2} - \Gamma_{p-1} - 1 = \Gamma_{p-2} -1 = 0$, and hence $\Gamma_{p-2} = 1$.
Inductively for $2\leq k \leq p-2$, by the relation $\Gamma_{k-1} - \Gamma_k =0$, then $\Gamma_k = 1$.

Finally, we show a contradiction occurs when $k = 1$.
Since $p| \delta_i$, we can write
\begin{equation}\label{round}
 \Gamma_0 - \Gamma_1 = \left\lfloor \frac{\delta_i}{p} \right\rfloor -\left\lfloor \frac{\delta_i - \Phi(i,1)}{p} \right\rfloor = -\left\lfloor \frac{-\Phi(i,1)}{p} \right\rfloor.
\end{equation}

As 
\[
 1 = \Gamma_{p-2} = \sum_{i=1}^s\left\lfloor \frac{\Phi(i,1) + p -1}{p} \right\rfloor,
\]
 and as the sum is made of non-negative terms (by lemma \ref{koeck}, $-\Phi(i,1) \geq 0$) , we see that $p \leq \Phi(i,1) \leq p-1$ for one $i$, and $\Phi(i,1) = 0$ otherwise.
Without loss of generality we can assume the $\Phi(1,i)$ is the only non-zero term.
Combining this with \ref{round} and the action being trivial implies that $1 = -\left\lfloor \frac{-\Phi(1,1)}{p} \right\rfloor = \Gamma_0 - \Gamma_1 = g_F.$
Since $g_F \geq 2$ this is a contradiction and we are done.


\newpage


Putting $k=1$, we get 
\[
\Gamma_0 - \Gamma_1 = \sum_{i=1}^s\left(\left\lfloor \frac{\delta_i}{p} \right\rfloor \left\lfloor \frac{\delta_i - \Phi(i,1)}{p} \right\rfloor \right) = \sum_{i=1}^s - \left\lfloor \frac{\phi(i)}{p} \right\rfloor = g_F.
\]

Now since $\Gamma_{k-1}-\Gamma_k+\alpha_k = \Gamma_{k-1}-\Gamma_k = 0$ for $k=1,\ldots ,p-2$, it follows that $\Gamma_k=g_F$ for all $k\neq p-1$.

We now show that $\Gamma_{p-1} = 0$, which will contradict
\[
 \Gamma_{p-2} - \Gamma_{p-1} - 1 = g_F - \Gamma_{p-1} - 1 = 0.
\]


Indeed, for $k = p-1$, we have by \ref{eq}
\begin{eqnarray*}
v_{ik} & = & \left \lfloor \frac{\delta_i - k\Phi(i,1)}{p} \right\rfloor \\
	& = & \left\lfloor \frac{(p-1)(\Phi(i,1) + 1) - (p-1)\Phi(i,1)}{p} \right\rfloor\\
	& = & 0.
\end{eqnarray*}




% !TEX TS-program = pdflatex
% !TEX encoding = UTF-8 Unicode

% This is a simple template for a LaTeX document using the "article" class.
% See "book", "report", "letter" for other types of document.

\documentclass[draft, 11pt]{article} % use larger type; default would be 10pt

\usepackage[utf8]{inputenc} % set input encoding (not needed with XeLaTeX)

%%% Examples of Article customizations
% These packages are optional, depending whether you want the features they provide.
% See the LaTeX Companion or other references for full information.

%%% PAGE DIMENSIONS
\usepackage{geometry} % to change the page dimensions
\geometry{a4paper} % or letterpaper (US) or a5paper or....
% \geometry{landscape} % set up the page for landscape
% read geometry.pdf for detailed page layout information

\usepackage{graphicx} % support the \includegraphics command and options
\usepackage[obeyDraft]{todonotes}

%\usepackage[parfill]{parskip} % Activate to begin paragraphs with an empty line rather than an indent

%%% PACKAGES
\usepackage[all]{xy}
\usepackage{mathtools}
\usepackage{booktabs} % for much better looking tables
\usepackage{array} % for better arrays (eg matrices) in maths
\usepackage{paralist} % very flexible & customisable lists (eg. enumerate/itemize, etc.)
\usepackage{verbatim} % adds environment for commenting out blocks of text & for better verbatim
\usepackage{subfig} % make it possible to include more than one captioned figure/table in a single float
%\usepackage{hyperref}
% These packages are all incorporated in the memoir class to one degree or another...

%\usepackage[activate={true,nocompatibility},final,tracking=true,kerning=true,spacing=true,factor=1100,stretch=10,shrink=10]{microtype}
%\microtypecontext{spacing=nonfrench}
% activate={true,nocompatibility} - activate protrusion and expansion
% final - enable microtype; use "draft" to disable
% tracking=true, kerning=true, spacing=true - activate these techniques
% factor=1100 - add 10% to the protrusion amount (default is 1000)
% stretch=10, shrink=10 - reduce stretchability/shrinkability (default is 20/20)

%%% HEADERS & FOOTERS
\usepackage{fancyhdr} % This should be set AFTER setting up the page geometry
\pagestyle{fancy} % options: empty , plain , fancy
\renewcommand{\headrulewidth}{0pt} % customise the layout...
\lhead{}\chead{}\rhead{}
\lfoot{}\cfoot{\thepage}\rfoot{}

%%% SECTION TITLE APPEARANCE
\usepackage{sectsty}
\allsectionsfont{\sffamily\mdseries\upshape} % (See the fntguide.pdf for font help)
\usepackage{amsmath}
\usepackage{amsthm}
\usepackage{amsfonts}
\usepackage{mathrsfs}
\usepackage{amsopn}
\usepackage{amssymb}
\usepackage{etex}
%\usepackage{natbib}
% (This matches ConTeXt defaults)

%%% ToC (table of contents) APPEARANCE
\usepackage[nottoc,notlof,notlot]{tocbibind} % Put the bibliography in the ToC
\usepackage[titles,subfigure]{tocloft} % Alter the style of the Table of Contents
%\renewcommand{\cftsecfont}{\rmfamily\mdseries\upshape}
%\renewcommand{\cftsecpagefont}{\rmfamily\mdseries\upshape} % No bold!
%\renewcommand{\familydefault}{\sfdefault}
%\usepackage{cabin}
%\usepackage{libertine}
%\usepackage[T1]{fontenc}

%Theorems and stuff
\theoremstyle{plain}
\newtheorem{defn}{Definition}[section]
\newtheorem{thm}[defn]{Theorem}
\newtheorem{cor}[defn]{Corollary}
\newtheorem{lem}[defn]{Lemma}
\newtheorem{prop}[defn]{Proposition}
\newtheorem{ex}[defn]{Example}
\newtheorem*{unnumthm}{Theorem}
\newtheorem{defnlem}[defn]{Definition/Lemma}
\newtheorem{defnthm}[defn]{Theorem/Definition}
\theoremstyle{remark}
\newtheorem*{rem}{Remark}
\newtheorem*{note}{Note}


\newcommand{\cO}{{\cal O}}
\newcommand{\ra}{\rightarrow}
\newcommand{\NN}{{\mathbb N}}
\newcommand{\PP}{{\mathbb P}}
\newcommand{\ZZ}{{\mathbb Z}}
\newcommand{\cL}{{\mathcal L}}
\newcommand{\cA}{{\mathcal A}}
\newcommand{\cD}{{\mathcal D}}
\newcommand{\cU}{{\mathcal U}}
\newcommand{\cech}{\v{C}ech }
\newcommand{\hzero}{{H^0(X,\Omega_X)}}
\newcommand{\hone}{H^1(X,\mathcal{O}_X)}
\newcommand{\cechhone}{\check{H}^1(\mathcal U,\mathcal O_X)}
\newcommand{\derhamhone}{H_{\text {dR}}^1(X/k)}
\newcommand{\cechhzero}{{\check{H}^0(X,\Omega_X)}}
\newcommand{\ubar}{\underset{\bar{}}}


\DeclareMathOperator{\aut}{Aut}
\DeclareMathOperator{\res}{Res}
\DeclareMathOperator{\ord}{ord}
\DeclareMathOperator{\di}{div}
\DeclareMathOperator{\cha}{char}
\DeclareMathOperator{\gal}{Gal}
\DeclareMathOperator{\Tr}{Tr}
\DeclareMathOperator{\Ima}{Im}

%%% END Article customizations

%%% The "real" document content comes below...

\title{Group actions on de Rham cohomology of hyperelliptic curves}
\author{}
%\date{} % Activate to display a given date or no date (if empty),
         % otherwise the current date is printed

\begin{document}
\maketitle

\listoftodos

\section{Background}

\todo[inline]{change references to preprint when in thesis}
Let $X$ be a smooth, projective, connected hyperelliptic curve of genus $g \geq 2$ over an algebraically closed field $k$ of characteristic $p \geq 0$.
We  fix a map $\pi \colon X \rightarrow \mathbb P_k^1$ of degree two, which is unique up to an automorphism of $\mathbb P_k^1$.
We let $G$ denote a finite group acting faithfully on $X$.

In a similar vein to \cite{canonicalrepresentation}, we will be using \cech cohomology to compute both $\hone$ and the de Rham hypercohomology, denoted $\derhamhone$.
We therefore recall the \cech cohomology relevant to both of these, starting with $\hone$.

By Leray's theorem \cite[Thm 5.2.12]{liu} and Serre's affineness criterion \cite[Thm 5.2.23]{liu} we know that the first \cech cohomology group and $\hone$ will be isomorphic if the cover we use to compute the \cech cohomology is affine.
We let $U_a = X \backslash \pi^{-1}(a)$ for any $a \in \mathbb P_k^1$ and we let ${\cal U}$ be the affine cover of $X$ formed by $U_0$ and $U_\infty$.
Given any sheaf $\cal F$ on $X$ we have the \cech differential $\check{d}\colon {\cal F}(U_0) \times {\cal F} (U_\infty) \rightarrow {\cal F}(U_0 \cap U_\infty)$, defined by $(f_0,f_\infty) \mapsto f_0|_{U_0 \cap U_\infty} - f_\infty|_{U_0 \cap U_\infty}$.
In general we will suppress the notation denoting the restriction map.
Via this differential we have the following cochain complex
\begin{equation*}
0 \rightarrow \cO_X(U_0)\times \cO_X(U_\infty) \xrightarrow{\check{d}} \cO_X(U_0 \cap U_\infty) \rightarrow 0.
\end{equation*}
The first cohomology group of this chain is $\cechhone = \frac{\cO_X(U_0 \cap U_\infty)}{\Ima(\check{d})}$ and hence
\begin{equation}\label{cechhone}
\hone \cong \frac{\cO_X(U_0 \cap U_\infty)}{\Ima(\check{d})}  
 = \frac{\cO_X(U_0 \cap U_\infty)}{\{f_0 - f_\infty | f_i \in \cO_X(U_i) \}}.
\end{equation}

We now recall how to compute the algebraic de Rham cohomology of $X$ via \cech cohomology.
Since $X$ is a curve any differentials of degree greater than one on $X$ are zero.
Hence the de Rham complex $X$ is the hypercohomology of the complex
\begin{equation}\label{res}
0 \rightarrow \cO_X \xrightarrow{d} \Omega_X \rightarrow 0.
\end{equation}
Here $d$ denotes the differential map $f \mapsto df$, as defined in \cite[Chap II, \S 8, Pg. 172]{hart}.

We use the cover $\cal U$ and the \cech differentials as defined earlier to give us the \cech bicomplex of \eqref{res}, which is
\begin{equation}\label{bicomplex}
\xymatrix{ & 0 \ar[d] & 0 \ar[d] & \\
0 \ar[r] & \cO_X(U_0) \times \cO_X(U_\infty) \ar[d] \ar[r] & \Omega_X(U_0) \times \Omega_X(U_\infty) \ar[d] \ar[r] & 0 \\
0 \ar[r] & \cO_X(U_0\cap U_\infty) \ar[d] \ar[r] & \Omega_X(U_0 \cap U_\infty) \ar[r] \ar[d] & 0 \\
& 0 & 0 &}
\end{equation}
By a generalisation of Leray's theorem \cite[Cor 12.4.7]{EGA0III} we know that the $\derhamhone$ is isomorphic to the first cohomology of the total complex of \eqref{bicomplex}.
Note that this requires ${\check H}^p(U_\sigma, \cO_X)$ and ${\check H}^p(U_\sigma, \Omega_X)$ to be zero for any $\sigma$ in the nerve of $\cU$ and any $p \geq 1$ ---
since $U_0$ and $U_\infty$ are affine, this again follows from Serre's affineness criterion \cite[Thm 5.2.23]{liu}.



After computing the first cohomology group of the total complex of \eqref{bicomplex} we see that $\derhamhone$ is isomorphic to the space
\begin{equation}\label{derhamconditions}
\left\{(\omega_0, \omega_\infty, f_{0,\infty}) | \omega_i\in \Omega_{X/k}(U_i), f_{0,\infty}\in \cO_X(U_0 \cap U_\infty), df_{0,\infty} = \omega_0|_{U_0\cap U_\infty} - \omega_\infty|_{U_0\cap U_\infty} \right\}
\end{equation}
quotiented by the subspace
\begin{equation}\label{quotient}
\left\{  (df_0, df_\infty, f_0|_{U_0\cap U_\infty} -f_\infty|_{U_0\cap U_\infty} )|f_i \in \cO_X(U_i)\right\}.
\end{equation}

We wish to compute a $k$-basis of $\derhamhone$ in order that we can see how $G$ acts on $\derhamhone$ and to study the $k[G]$-module structure.
The following lemma shows that $\derhamhone$ fits in to a short exact sequence which we will use to compute the $k$-basis.
\begin{prop}\label{ses}
The following is a short exact sequence
\begin{equation}\label{equationses}
0 \ra H^0(X,\Omega_X) \xrightarrow{i} \derhamhone \xrightarrow{p} H^1(X,\cO_X) \ra 0, 
\end{equation}
where $i \colon \omega \mapsto (\omega, \omega)$ and $p \colon (\omega_0, \omega_\infty, f_{0 \infty}) \to f_{0 \infty}$.
\end{prop}
\begin{proof}
Let $T$ be the total complex of \eqref{bicomplex}.
Moreover we let $\cO$ and $\Omega$ be the complexes formed from the first and second (non-trivial) columns of \eqref{bicomplex} respectively.
Then let $\Omega[1]$ denote the complex obtained from shifting $\Omega$ by one, i.e. $\Omega[1]^{n+1} = \Omega^n$.
Then we obtain the following short exact sequence of complexes 
\[
\Omega[1] \hookrightarrow T \twoheadrightarrow \cO,
\]
giving rise to the following long exact sequence
\begin{equation*}
0 \ra H^0_{\text {dR}}(X/k) \ra H^0(X,\cO_X) \ra \\
\end{equation*}
\begin{equation}\label{longexactsequence}
 H^0(X,\Omega_X) \ra \derhamhone \ra \hone \ra \\
\end{equation}
\begin{equation*}
 H^1(X,\Omega_X) \ra H^2_{\text {dR}}(X/k) \ra 0.
\end{equation*}
The map $H^0(X,\cO_X) \ra \hzero$ is the map $f \mapsto df$.
Since the only globally holomorphic functions on $X$ are constant functions, it follows that this is the zero map, and hence $\hzero \ra \derhamhone$ is injective.

Since \eqref{longexactsequence} is exact, $p$ is surjective if and only if $\alpha \colon H^1(X,\Omega_X) \ra H^2_{\text {dR}}(X/k)$ is injective.
Now $H^1(X,\Omega_X)$ is isomorphic to $k$ via the residue map \cite[Chap. III, Thm. 7.14.1]{hart}, and if we can show that this isomorphism factors through $\alpha$ it will follow that $\alpha$ is injective.
Considering the \cech cohomology constructions of $H^1(X,\Omega_X)$ and $H^2_{\text {dR}}(X/k)$, it suffices to show that the residue map is zero on $\Ima \left( d \colon \cO_X(U_0 \cap U_\infty) \ra \Omega_X(U_0 \cap U_\infty) \right)$.
This follows from \cite[Chap. III, Thm. 7.14.1 (b)]{hart}, which says that $\res_P(df)=0$ for any $f \in \cO_X(U_0 \cap U_\infty)$.
Hence the residue isomorphism factors through $\alpha$, and hence $p$ is surjective.
\end{proof}


A basis of $H^0(X,\Omega_X)$ is already available in the literature (see, for example, \cite[Prop. 7.4.26]{liu}).
In the next section we compute a basis of $\hone$, in order that we can understand how $G$ acts on both ends of the short exact sequence described in Lemma \ref{ses}.

\section{Basis of $\derhamhone$}

The bases of the $k$-vector spaces in \eqref{equationses} will be given in terms of defining equations of the hyperelliptic curve $X$, so we describe these now.
Throughout we let $P_a$ and $P_a'$ denote the unique elements of $\pi^{-1}(a)$ for any point $a \in \mathbb P_k^1$ that is not a branch point.
If $a \in \mathbb P_k^1$ is a branch point we denote the unique point in $\pi^{-1}(a)$ by $P_a$.
We also define $D_a$ to be the divisor $\pi^*\left([a]\right)$ for any $a \in \mathbb P_k^1$, and hence
\begin{equation*}
D_a= 
\begin{cases}
 2[P_a] & \text{if $a$ is a branch point}, \\
 [P_a] + [P_a'] & \text{otherwise.}
\end{cases}
\end{equation*}
With this notation we also have
\begin{equation}\label{divxp=2}
\di (x)  = D_0 - D_\infty
\end{equation}
regardless of the characteristic of $k$.
Also, as in \cite[\S 6]{faithfulaction}, we have
\begin{equation}\label{differentialdivisor}
\di(dx) = R - 2D_\infty,
\end{equation}
where $R$ denotes the ramification divisor of $\pi$.
This is again regardless of characteristic.

If $p \neq 2$ then the extension $K(X)$ of $K(\mathbb P_k^1) = k(x)$ will be $k(x,y)$ where $y$ satisfies
\begin{equation}\label{definingequationpnot2}
y^2 = f(x)
\end{equation}
for some polynomial $f(x) \in k[x]$ which has no repeated roots and is of degree $2g+1$ or $2g+2$ \cite[Prop 7.4.24]{liu}.

Recall from \cite[\S 6]{faithfulaction} that
\begin{equation}\label{pnot2divisors}
\di(y)  = R - (g+1)D_\infty.
\end{equation}


If $p=2$, then the extension $K(X)$ of $k(x)$ will be $k(x,y)$, this time with $y$ satisfying the equation
\begin{equation}\label{definep=2}
y^2 - H(x)y = F(x)
\end{equation}
for some $H(x),F(x) \in k[x]$, such that $H(x)$ and $H'(x)^2F(x) + F'(x)^2$ share no roots.
We require that $\deg(H(x)) \leq g+1$, with equality if and only $\infty$ is not a branch point, and that $\deg(F(x)) \leq 2g+2$ with $\deg(F(x)) = 2g+1$ if $\infty$ is a branch point  \cite[Prop 7.4.24]{liu}.


We now recall from \cite[\S 6]{faithfulaction} that the divisor of $H(x)$ is
\begin{equation}\label{divisorofH}
\di (H(x))  = R - (g+1)D_\infty. 
\end{equation}
Finally, we describe the divisor of $y$ when $p=2$.
In order to do this we need to distinguish the zeroes of $F(x)$.
Suppose that $F(x)$ has $l \leq \deg(F)$ distinct zeroes, and let $\gamma_1, \ldots, \gamma_l \in k \subseteq \mathbb P_k^1$ be these zeroes.
Then if $\gamma_i$ is a branch point let $Q_i = (\gamma_i, 0)$ be the unique point in the pre-image $\pi^{-1}(\gamma_i)$.
If $\gamma_i$ is not a branch point then let $Q_i = (\gamma_i, 0)$ and $Q_i' = (\gamma_i, H(\gamma_i))$ be the unique points that form the pre-image $\pi^{-1}(\gamma_i)$.
Also, we denote the order of the zero of $F(x)$ at $\gamma_i \in k$ by $m_i$.


\begin{prop}\label{divyp=2}
Suppose that $p=2$.
Then, if $\infty$ is a branch point, the divisor of $y$ is
\begin{equation*}
\di(y) = 
 {\displaystyle \sum_{i=1}^l} m_i[Q_i] -(2g+1)[P_\infty].
\end{equation*}
On the other hand, if $\infty$ is not a branch point, then, after possibly swapping the notations for $P_\infty$ and $P_\infty'$ for the two points in $\pi^{-1}(\infty)$, we have
\begin{equation*}
 \di(y) = {\displaystyle \sum_{i=1}^l} m_i[Q_i] +(g+1-\deg(F(x)))[P_\infty] - (g+1)[P_\infty'].
\end{equation*}
\end{prop}
\begin{proof}
We first show that $\di_0(y)$, the divisor of the zeroes of $y$, is $\sum_{i=1}^l m_i [Q_i]$.

It is clear that the zeroes of $y$ can only occur in the affine part of the curve $X$ defined by \eqref{definep=2} i.e. in $U_\infty$.
Suppose $P\in U_\infty$.
If $\left. F \right|_P \neq 0$ then it follows that $y|_P \neq 0$, since $F(x) = y (y + H(x))$.
Hence $\di_0(y)$ has zero coefficients for any point in $U_\infty$ other than the $Q_i$.

Suppose that $P= Q_i = (\gamma_i, 0)$ is an unramified point in $U_\infty$.
Then $H(\gamma_i) \neq 0$ and $\left. y \right|_P = 0$, so $y + H(x)$ is a unit at $P$.
Since $y(y+H(x)) = F(x)$ we find that
\begin{equation*}
\ord_P(y) = \ord_P\left( \frac{F(x)}{y + H(x)} \right) = \ord_P(F(x)) = m_i.
\end{equation*}

We now look at when $P = Q_i = (\gamma_i, 0)$ is a ramification point.
Since $H(x)$ and $H'(x)^2F(x) + F'(x)^2$ cannot share roots it follows that $m_i = 1$.
Hence the function $\tilde F(x) := (x- \gamma_i)^{-1}F(x)$ is a unit at $P$.
We let $\tilde H(x) = (x- \gamma_i)^{-1}H(x)$.
Now $y^2 = F(x) - y H(x) = (x- \gamma_i) (\tilde F(x) - y \tilde H(x))$, and hence
\[
\ord_P(y^2 ) = \ord_P(x-\gamma_i) + \ord_P(\tilde F(x) - y \tilde H(x)).
\]
Since $\ord_P(x-\gamma_i) = 2$ and $\ord_P(\tilde F(x) - y \tilde H(x)) \geq 0$ we know that $\ord_P(y) \geq 1$.
Hence $\left. (y \tilde H(x)) \right|_P = 0$, and since $\tilde F(x)$ is a unit at $P$, we conclude that $\tilde F(x) - y \tilde H(x)$ is a unit at $P$.
Hence $\ord_P(y^2) = 2$, and so $\ord_P(y) = 1 = m_i$.
It follows that $\di_0(y) =  \sum_{i=1}^l m_i [Q_i]$.

We now consider the poles of $y$.
If $\infty$ is a branch point we know that $\deg(F(x)) = 2g+1$ and hence $\sum_{i=1}^l m_i = 2g+1$.
Since $y$ can only have a pole at $P_\infty$, we conclude that the degree of this pole is $2g+1$, and hence
\[
\di(y) = \sum_{i=1}^l m_i [Q_i] - (2g+1)[P_\infty]
\]
if $\infty$ is a branch point.

If $\infty$ is not a branch point then there are two points at which $y$ may have a pole, namely $P_\infty$ and $P_\infty'$.
We consider three cases, and recalling that $\ord_{P_\infty}(y + H(x)) = \ord_{P_\infty'}(y)$, using the automorphism given by $y \mapsto y+H(x)$.


Firstly, we suppose that $\ord_{P_\infty}(y) < -(g+1)$.
Then $\ord_{P_\infty}(y) < \ord_{P_\infty}(H(x))$ and hence $ \ord_{P_\infty}(y) = \ord_{P_\infty}(y+H(x))$.
But this contradicts $\ord_{P_\infty}(y) + \ord_{P_\infty}(y+H(x)) = \ord_{P_\infty}(F(x))$, since the left hand side is less than $-2(g+1)$, which is the minimum value of the right hand side.

We now suppose that $\ord_{P_\infty} (y) = -(g+1)$. Since $y(y+H(x)) = F(x)$ it follows that $-(g+1) + \ord_{P_\infty}(y+H(x)) = \ord_{P_\infty}(F(x))$, and hence $\ord_{P_\infty'}(y) = \ord_{P_\infty}(y+H(x)) = -\deg(F(x)) + g + 1$.

Finally, if $\ord_{P_\infty}(y) > -(g+1)$, then since $\ord_{P_\infty}(H(x)) = -(g+1)$ it follows that $\ord_{P_\infty'}(y) = \ord_{P_\infty} (y+H(x)) = -(g+1)$.
Then, from a computation similar to that in the previous paragraph we see that $\ord_{P_\infty}(y) = -\deg(F(x)) + g +1$, completing the proof.
\end{proof}



Finally, we will compute the divisor of $dy$, reminding the reader that we are considering the characteristic two case.
We start by noting that when we take the differential of \eqref{definep=2} we obtain
\[
dF = d\left(y^2 + yH \right) = d(yH) = Hdy + ydH
\]
and from this it follows that
\begin{equation}\label{divdyp=2}
dy = \frac{F'-yH'}{H}dx.
\end{equation}


To state the ramification points, and for later use, we suppose that 
\begin{equation*}
H(x) = \prod_{i=1}^d (x-a_i)^{n_i} = x^d + b_{d-1}x^{d-1} + \ldots + b_1x + b_0
\end{equation*}
for some $a_i, b_i \in  k$, $d \leq g+1$ and $n_i \in \mathbb N$.
Then the $a_i \in \mathbb A_k^1 \subset \mathbb P_k^1$ are the branch points of $\pi$ and we let $P_i \in X$ be the corresponding ramification points above $a_i$.
Given this, we can write the ramification divisor as
\[
R = \sum_{i=1}^d 2n_i[P_i] + (g+1-d)D_\infty.
\]
Details can be found in \cite[\S 6]{faithfulaction}.

We now take a brief detour to recall some of the details of Serre duality.
We let $\Omega_{K(X)}$ be the module of differentials of $K(X)$ over $k$.
\begin{lem}
The following sequence is exact
\begin{equation}\label{dualityses}
0 \rightarrow \hzero \ra \Omega_{K(X)} \ra \bigoplus_{P \in X}\Omega_{K(X)}/\Omega_{X,P} \xrightarrow{\delta} H^1(X,\Omega_X) \ra 0.
\end{equation}
\end{lem}
\begin{rem}
Note that the sequence formed by the last three terms can be found in \cite[Pg. 248]{hart}.
\end{rem}
\begin{proof}
We let $\underline{\Omega}_{K(X)}$ be the constant sheaf of $\Omega_{K(X)}$.
Then the short exact sequence
\begin{equation}\label{serredualityses}
0 \rightarrow \Omega_X \rightarrow \underline{\Omega}_{K(X)} \rightarrow \underline{\Omega}_{K(X)}/\Omega_X \rightarrow 0
\end{equation}
is a flasque resolution of $\Omega_X$ by \cite[Chap II, ex. 1.16]{hart}.

We view the module $\Omega_{K(X)}/\Omega_{X,P}$ as a sheaf on the singleton $\{P\}$, which has a natural embedding $i\colon \{P\} \rightarrow X$.
Hence for each $P\in X$ we have the induced sheaf $i_*\left(\Omega_{K(X)}/\Omega_{X,P}\right)$ on $X$.
If we consider the direct sum of these induced sheaves over all points $P\in X$ we have the following isomorphism
\begin{equation}\label{sheafisomorphism}
\underline{\Omega}_{K(X)}/\Omega_X\cong \bigoplus_{P\in X} i_*\left(\Omega_{K(X)}/\Omega_{X,P}\right).
\end{equation}
To explain this isomorphism we first construct a map from $\underline{\Omega}_{K(X)}/\Omega_{X,P}$ in to the product $\prod_{P \in X} i_*\left(\Omega_{K(X)}/\Omega_{X,P}\right)$, and then showing that this map has finite support.

Given $i\colon \{P\} \hookrightarrow X$ we have the following equalities
\begin{align*}
i^{-1}\left(\underline{\Omega}_{K(X)}/\Omega_X\right) & = \left(\underline{\Omega}_{K(X)}/\Omega_X\right)_P \\
& = \underline{\Omega}_{K(X),P}/\Omega_{X,P} \\
& = \Omega_{K(X)}/\Omega_{X,P}.
\end{align*}\todo{blame it on the adjoint}
It follows that for each $P\in X$ we can map $f \in \underline{\Omega}_{K(X)}/\Omega_X$ to $i_*i^*(f) \in i_*\left( \Omega_{K(X)}/\Omega_{X,P}\right)$, and that this map will be an isomorphism.
We recall that for any $f \in \Omega_{K(X)}$ we have $f \in \Omega_{X,P}$  for all but a finite number of points $P \in X$, and hence the image of any such $f$ in $\prod_{P\in X} i_*\left( \Omega_{K(X)}/\Omega_{X,P}\right)$ is zero in almost all factors.
In particular, the image lies in $\bigoplus_{P \in X} i_*(\Omega_{K(X)}/\Omega_{X,P} ) \subset \prod_{P\in X} i_*\left( \Omega_{K(X)}/\Omega_{X,P}\right)$.
Since the map $\underline{\Omega}_{K(X)}/\Omega_X \cong \bigoplus_{P \in X}i_*(\Omega_{K(X)}/\Omega_{X,P}$ given by $f \mapsto (i_*i^*(f))_{P \in X}$ is an isomorphism of stalks it follows that it is also an isomorphism at the level of sheaves, as claimed in \eqref{sheafisomorphism}.


Now we can take the \cech complex of \eqref{serredualityses} over the cover $\cU$, yielding
\begin{equation}\label{dualitydiagram2}
\xymatrix{\Omega_X(U_0)\times\Omega_X(U_\infty) \ar@{^{(}->}[r] \ar[d]^{d_1} & \underline{\Omega}_{K(X)} \times \underline{\Omega}_{K(X)} \ar[d]^{d_2} \ar@{->>}[r] & \bigoplus \limits_{P \in U_0} \Omega_{K(X)}/\Omega_{X,P} \times \bigoplus \limits_{P \in U_\infty} \Omega_{K(X)}/\Omega_{X,P} \ar[d]^{d_3} \\
\Omega_X(U_0 \cap U_\infty) \ar@{^{(}->}[r]  & \underline{\Omega}_{K(X)} \ar@{->>}[r] & \bigoplus \limits_{P\in U_0 \cap U_\infty} \Omega_{K(X)}/\Omega_{X,P} }
\end{equation}
We can now apply the snake lemma to this commutative diagram.
In the next paragraph we conclude the proof by showing that the exact sequence that arises from the applying snake lemma is precisely the sequence \eqref{dualityses} in the statement of the lemma.\todo{check underlining}

The fact that $H^0(X,\Omega_X) \cong \ker(d_1)$ and $H^1(X,\Omega_X) \cong {\rm coker}(d_1)$ follows from the above discussion of \cech cohomology.
The map $d_2\colon \Omega_{K(X)} \times \Omega_{K(X)} \ra \Omega_{K(X)}$ is the \cech differential given by $(\omega_1,\omega_2) \mapsto \omega_1 - \omega_2$.
Hence the map $\omega \mapsto (\omega, \omega)$ gives an isomorphism from $\Omega_{K(X)}$ to $\ker(d_2)$.
Finally, $d_3$ is defined by $(\omega_0, \omega_\infty) \mapsto \omega_0|_{U_0 \cap U_\infty} - \omega_\infty|_{U_0 \cap U_\infty}$.
Hence the kernel of $d_3$ is formed of pairs $(\omega_0, \omega_\infty) \in \bigoplus_{P \in U_0} \Omega_{K(X)}/\Omega_{X,P} \times \bigoplus_{P \in  U_\infty} \Omega_{K(X)}/\Omega_{X,P}$ such that $\omega_0$ and $\omega_\infty$ agree on $U_0 \cap U_\infty$.
It follows that the map $\bigoplus_{P \in X} \Omega_{K(X)}/\Omega_{X,P}\ra\ker(d_3)  $ given by 
\begin{equation*}
(\omega_P)_{P \in X} \to \left( (\omega_P)_{ P \in U_0}, (\omega_P)_{P \in U_\infty}) \right)
\end{equation*}
is an isomorphism.\todo{This is needed for the proof - it shows it is a short exact sequence}
\end{proof}


We now recall the definition of the trace map $t\colon H^1(X,\Omega_X) \ra k$.
Given a point $P\in X$ we have the residue map $\res_P \colon \Omega_{K(X)} \ra k$, defined by 
\begin{enumerate}
\item $\res_P(\omega) = 0$ for all $\omega \in \Omega_P$;
\item $\res_P(f^ndf)=0$ for all $f \in K(X)^*$, all $n \neq 1$;
\item $\res_P(f^{-1}df) = \ord_P(f)$, where $\ord_P(f)$ is the order of $f$ at $P$;
\end{enumerate}
as in \cite[Chap III, Thm. 7.14.1]{hart}.
This also gives a well defined map on the quotient $\Omega_{K(X)}/\Omega_P$, since elements of $\Omega_P$ have zero residue at $P$, and so $\Omega_P \subseteq \ker \left(\res_P\right)$.
Since we know that $\sum_{P\in X}\res_P(\omega) = 0$ for any $\omega \in \Omega_K(X)$ by the residue theorem \cite[Chap. III, Thm. 7.14.2]{hart},  it follows that the map $t \colon \bigoplus_{P \in X} \Omega_{K(X)}/\Omega_P \ra k$ given by $(\omega_P)_{P \in X} \mapsto \sum_{P\in X} \res_P(\omega_P)$ vanishes on the image of $\Omega_{K(X)}$.
Hence this map is well defined on the quotient, which is $H^1(X,\Omega_X)$, by Lemma \ref{dualityses}.
Given some $\bar \omega$ in the quotient $H^1(X,\Omega_X)$, which can be lifted to $\omega \in \bigoplus_{P \in X} \Omega_K(X)/\Omega_P$, we can define the trace map $t$ by
\[
t \colon H^1\left(X, \Omega_X\right) \ra k,\ \bar \omega \mapsto \sum_{P \in X} \res_P(\omega).
\]\todo{check this}

The following lemma will make it easier to compute $t(\bar \omega)$ for some $\bar \omega \in H^1(X,\Omega_X)$.
\begin{lem}\label{tracemaplemma}
For any $\bar \omega \in H^1(X,\Omega_X)$, with representative $\omega \in \bigoplus_{P \in X} \Omega_{K(X)}/\Omega_P$, we have the following equality:
\[
t(\bar \omega) = \sum_{P \in \pi^{-1}(\infty)}\res_P(\omega).
\]
\end{lem}
\begin{proof}
The proof follows from a diagram chase on \eqref{dualitydiagram2}.
Given a cocycle in $\bar \omega \in \hone$ we take a representative $\omega \in \Omega_X(U_0 \cap U_\infty)$.
This then injects in to $\underline{\Omega}_{K(X)}$, and since $d_2$ is surjective we can choose an element of $\underline{\Omega}_{K(X)} \times \underline{\Omega}_{K(X)}$ mapping to $\omega$.
In particular, we could choose $(\omega,0)$.
This then maps to 
\[
\psi = ((\bar{\omega}|_P)_{P\in U_0}, 0) \in \left( \bigoplus_{P \in U_0} \Omega_{K(X)}/\Omega_{X,P}\right) \times \left( \bigoplus_{P \in U_\infty} \Omega_{K(X)}/\Omega_{X,P} \right).
\]
Moreover, by commutativity of the diagram $\psi \in \ker(d_3)$.
Since $\omega$ is regular on $U_0 \cap U_\infty$, and $0$ is regular on $U_\infty$, then $\res_P(\psi) = 0$ for all $P \in U_\infty$.
Hence $t(\bar \omega) = \sum_{P \in X}\res_P(\bar \omega) = \sum_{P \in \pi^{-1}(\infty)} \res_P(\bar \omega) = \sum_{P \in \pi^{-1}(\infty)} \res_P(\omega)$.
\end{proof}

We now use the trace map to define a pairing between the $k$-vector spaces $\hone$ and $\hzero$.
There exists a canonical map 
\begin{equation}\label{productmap}
\hzero \times \hone \ra H^1\left(X, \Omega_X\right), \ (\omega, f) \mapsto f  \omega|_{U_0 \cap U_\infty},
\end{equation}
where the product $f \omega|_{U_0\cap U_\infty}$ is just the usual product of a function and a differential, but restricted to $U_0 \cap U_\infty$, and we consider $H^1(X,\Omega_X)$ and $\hone$ as $\check{H}^1(X,\Omega_X)$ and $\cechhone$ respectively.
We check that this map is well defined.
All elements of $\hzero$ and $\hone$ are regular on $U_0 \cap U_\infty$.
Hence the product $f \omega|_{U_0 \cap U_\infty}$ in \eqref{productmap} is regular on $U_0 \cap U_\infty$.
Moreover, if $f \in \cO_X(U_0 \cap U_\infty)$ is a representative of zero in $\hone$, i.e.~$f = f_0 - f_\infty$ for $f_i \in \cO_X(U_i)$, and $\omega  \in \hzero$ then $f \omega|_{U_0 \cap U_\infty} = f_0\omega|_{U_0 \cap U_\infty} - f_\infty \omega|_{U_0 \cap U_\infty}$, and $f_i \omega|_{U_0\cap U_\infty} \in \Omega_X(U_i)$ for $i\in \{0, \infty\}$.
Hence $f\omega$ is zero in $H^1(X,\Omega_X)$ and the map is well defined.

\begin{thm}\label{serredualitytheorem}
Via the pairing $\langle , \rangle$, the $k$-vector spaces $\hone$ and $\hzero$ are dual to each other.
\end{thm}
\begin{proof}
This is in fact a specialisation of \cite[Thm. 2, Chap. II]{algebraicgroupsandclassfields}.
\end{proof}

We now combine the product map in \eqref{productmap} with the trace map $t$ to get a map 
\[
(\omega, f) \to \langle \omega, f \rangle := t \left( f \omega |_{U_0 \cap U_\infty}\right) , \hzero \times \hone \ra k. 
\]
Thus if we fix any $\omega \in \hzero$ we produce a map $\theta(\omega)\colon \hone \ra k$, given by $\theta(\omega)(f) = \langle \omega , f\rangle$.
Similarly, if we fix any $f \in \hone$ then we get a map $\psi(f) \colon \hzero \ra k$.
Now $\psi$ and $\theta$ are isomorphisms and are dual to each other: in particular, this means that given a basis $e_1, \ldots, e_n$ of $\hzero$, we can find a basis $f_1, \ldots , f_n$ of $\hone$ such that $\theta(e_i)(f_i) = 1$ for all $1 \leq i \leq n$ and $\theta(e_i)(f_j) = 0$ if $i \neq j$ (and similarly for $\psi$).
Hence we have the following lemma.


We now use this fact to compute a basis of $H^1(X,\cO_X)$, and then give a basis of $\derhamhone$ in the following theorem.

\begin{prop}\label{basish1}
 Via the isomorphism \eqref{cechhone} the residue classes of $\frac{y}{x}, \ldots , \frac{y}{x^g} \in K(X)$, restricted to $U_0 \cap U_\infty$, form a basis of $H^1(X,\cO_X)$.
\end{prop}
\begin{proof}
We start by considering the case $p \neq 2$ and first check that the functions $\frac{y}{x}, \ldots, \frac{y}{x^g}$ are indeed regular on $U_0 \cap U_\infty$ (as required by \eqref{cechhone}) by computing their divisors.
From \eqref{divxp=2} and \eqref{pnot2divisors} we see that
\begin{align}\label{divisorofyoverx}
\di \left( \frac{y}{x^i} \right) & = \di (y) - \di ( x^i) \nonumber \\
& = R - (g+1)D_\infty - iD_0 + iD_\infty \nonumber \\
& = R - iD_0 - (g+1 - i)D_\infty.
\end{align}
Since $R$ is a positive divisor this is non-negative on $U_0 \cap U_\infty$ if $i\in \{0, \ldots, g-1\}$.


Recall that the differentials $\omega_0 = y^{-1}dx, \ldots, \omega_{g-1} = x^{g-1}y^{-1}dx$ form a basis of $\hzero$ (see, for example, \cite[Chap 7, Prop. 4.26]{liu}).
By Lemma \ref{tracelemma} we know that $\langle x^iy^{-1}dx, yx^{-j} \rangle = \sum_{P \in \pi^{-1}(\infty)}\res_P(x^{i-j}dx)$.
It follows immediately from \cite[Chap. III, Thm. 7.14.1(b)]{hart} that $\sum_{P \in \pi^{-1}(\infty)}\res_P(x^{i-j}dx) = -2$ if and is zero otherwise.
It then follows from Theorem \ref{serredualitytheorem} that the collection of elements $\{ yx^{-j}|_{U_0\cap U_\infty}\}_{ 1 \leq j \leq g}$ form a basis of $\hone$.



We now suppose that $p=2$, and again start by checking that for $i \in \{1, \ldots , g\}$ the function $yx^{-i}$ is regular on $U_0 \cap U_\infty$.
This follows once we compute the divisor of $yx^{-i}$, which is
\begin{equation*}
\di \left( \frac{y}{x^i} \right)  =  
{\displaystyle \sum_{i=1}^l} m_i[Q_i] -iD_0 -(2g+1 - 2i)[P_\infty]
\end{equation*}
if $\infty$ is a branch point and
\begin{equation*}
\di \left( \frac{y}{x^i} \right)  =  
{\displaystyle \sum_{i=1}^l} m_i[Q_i] - iD_0 +(g+1-\deg(F(x)) + i)[P_\infty] - (g+1-i)[P_\infty']
\end{equation*}
otherwise.
These equalities follow from Proposition \ref{divyp=2} and \eqref{divxp=2}.
The divisors are clearly positive on $U_0 \cap U_\infty$.

Next we recall from \cite[Chap 7, Prop. 4.26]{liu} that if $p=2$ a basis of $\hzero$ is given by $\frac{1}{H(x)}dx, \ldots, \frac{x^{g-1}}{H(x)}dx$.
We then deduce from Lemma \ref{tracemaplemma} that
\[
\left \langle \frac{x^i}{H(x)}dx, \frac{y}{x^j} \right \rangle = \res_{P_\infty} \left( \frac{yx^{i-j}}{H(x)}dx \right) + \res_{P_\infty'}\left( \frac{yx^{i-j}}{H(x)} dx \right).
\]
Then recall that in characteristic two we have an involution $\sigma \colon X \ra X$ given by $(x,y) \mapsto (x, y + H(x))$, and that $\res_P(\sigma^*(\omega)) = \res_{\sigma(P)}(\omega)$ for any $P \in X$ and $\omega\in \hzero$.
Then it follows that
\begin{align*}
\sum_{P \in \pi^{-1}(\infty)} \res_P \left( \frac{yx^{i-j}}{H(x)}dx \right) & = \res_{P_\infty} \left( \frac{yx^{i-j}}{H(x)} dx \right) + \res_{P_\infty'}\left( \frac{yx^{i-j}}{H(x)} dx\right) \\
& = \res_{P_ \infty} \left( \frac{yx^{i-j}}{H(x)}dx \right) + \res_{P_ \infty} \left( \frac{(y+H(x))x^{i-j}}{H(x)}dx \right) \\
& = \res_{P_\infty}(x^{i-j}dx),
\end{align*}
since we are assuming that $\cha(k) = 2$.
As in the previous case, it then follows from the definition of $\res_P$ that $\res_{P_\infty}(x^{i-j}dx) = -1$ if $i-j = -1$ and is zero otherwise.



If $P_\infty$ is a branch point then we start by computing the divisor of $ \frac{y}{x^j} \cdot \frac{x^i}{H(x)}dx$, using \eqref{divxp=2}, \eqref{differentialdivisor}, \eqref{divisorofH} and Proposition \ref{divyp=2}:
\begin{align*}
\di\left( \frac{yx^{i-j}}{H(x)}dx \right) & = \di(y) + \di(x^{i-j}) + \di( dx) - \di(H(x)) \\
& = \sum_{i=1}^l m_i[Q_i] - (2g+ 1 )[P_\infty] + (i-j)D_0 - (i-j)D_\infty + R - 2D_\infty \\
& \qquad - R + (g+1)D_\infty\\
& = \sum_{i=1}^l m_i[Q_i] + (2j-3-2i)[P_\infty] + (i-j)D_0
\end{align*}
We see that there is a pole of order one at $P_\infty$ if $2j - 3 - 2i = -1$, or equivalently if $j = i+1$.
Hence $\langle \frac{x^i}{H(x)}dx, \frac{y}{x^j} \rangle \neq 0$ in this case.\todo{what is the residue?}

We also need to check that if $j \neq i+1$ then $\left \langle \frac{x^i}{H(x)}dx, \frac{y}{x^j} \right \rangle = 0$.
Indeed, if $j-i \geq 2$ then clearly $\frac{yx^{i-j}}{H(x)}dx$ does not have a pole at $P_\infty$.
On the other hand, if $j-i \leq 0$ then the differential $\frac{yx^{i-j}}{H(x)}dx$ is regular on $U_\infty$, and hence the residue on this set is zero.
Since $X \backslash U_\infty = \{P_\infty\}$ it follows from the residue theorem the residue of $\frac{yx^{i-j}}{H(x)}dx$ at $P_\infty$ is also zero.
\end{proof}

%Mittag-Leffler style corollary
\begin{cor}
For each $P \in X$ fix some $f_P \in K(X)$, such that $f_P \in \cO_{X,p}$ for almost all $P$.
Then there exist unique $\alpha_1, \ldots , \alpha_g \in k$ such that, after $f_P$ is replaced by $f_P - \left( \alpha_1 \frac{y}{x} + \ldots + \alpha_g \frac{y}{x^g} \right)$ for all $P \in \pi^{-1}(\infty)$, there exists an $f \in K(X)$ such the laurent tail of $f$ at each $P$ in $X$ will be $f_P$.
\end{cor}
\todo[inline]{Check statement. Can't be all $P \in X$; maybe discrete subset? But then this is finite and the statement is trivial...}



In order to state a basis of $\derhamhone$, as well as to shorten the proof of the following theorem, we define the following polynomials. 
We suppose that $1 \leq i \leq g$.
Then when $p\neq 2$ we define
\[
s_i(x) := xf'(x) - 2if(x) \in k[x]
\]
and when $p = 2$ we define
\begin{equation}\label{capitals}
S_i(x,y) := xF'(x) + y(xH'(x) + iH(x))\in k[x]\oplus yk[x] \subseteq k(x,y).
\end{equation}

We now decompose these polynomials into two parts, which will be used in the sequel.
Firstly, we write $s_i(x)$ as $s_i(x) = \phi_i(x) + \psi_i(x)$, where $\psi_i(x)\in k(x)$ and $\phi_i(x) \in k[x]$ are the unique polynomials such that the degree of $\psi_i (x)$ is at most $g+1$ and $x^{g+2}$ divides $\phi_i(x)$.

We define $A_{j,i} \in k$ for $1 \leq j \leq 2g+2$, and $B_{k,i} \in k$ for $0\leq k \leq g+1$ by the equation
\[
S_i(x,y) = A_{2g+2,i}x^{2g+2} + \ldots + A_{1,i} x + y(B_{g+1,i} x^{g+1} + \ldots + B_{1,i} x + B_{0,i}).
\]
Note that many of these coefficients may be zero.
In particular we remark that the $x^i$ term of $xH'(x) + iH(x)$, and hence $B_{i,i}$, is always zero, since it is precisely $x \cdot b_ix^{i-1} + b_i x^i = 2B_{i,i}x^i = 0$.\todo{move back to where it previously was}


We now define the following polynomials:
\begin{equation}\label{Split}
\begin{split}
\Phi_i^x(x) & =  {}_iA_{2g+2}x^{2g+2} + \ldots + {}_iA_{i+1}x^{i+1} \\
\Psi_i^x(x) & =  {}_iA_ix^i + \ldots + {}_iA_1x \\
\Phi_i^y(x) & =  {}_iB_gx^g + \ldots {}_iB_{i+1}x^{i+1} \\
\Psi_i^y(x) & =  {}_iB_{i-1}x^{i-1} + \ldots + {}_iB_1x + {}_iB_0.
\end{split}
\end{equation}
Finally, we define $\Phi_i(x,y) = \Phi_i^x(x) + y \Phi^y_i(x)$ and $\Psi_i(x,y) = \Psi_i^x(x) + y \Psi_i^y(x)$, so that $S_i(x,y) = \phi_i(x,y) + \psi_i(x,y)$.

Viewing $\derhamhone$ as a quotient of \eqref{derhamconditions}, we now give a $k$-vector space basis of $\derhamhone$.
\begin{thm}\label{basis}

If $p \neq 2$ then the residue classes of 
\begin{equation}\label{one}
 \left( \left( \frac{\psi_i(x)}{2yx^{i+1}}\right) dx, \left(\frac{-\phi_i(x)}{2yx^{i+1}}\right) dx, x^{-i}y\right), i=1, \ldots ,g,
\end{equation}
along with the residue classes of 
\begin{equation}\label{two}
 \left( \frac{x^{i}}{y} dx , \frac{x^{i}}{y} dx, 0 \right), i = 0,\ldots ,g-1,
\end{equation}
form a basis of $\derhamhone$.

On the other hand, if $p=2$ then the residue classes of the elements 
\begin{equation}\label{three}
\left( \left(\frac{\Psi_i(x,y)}{x^{i+1}H(x)}\right) dx, \left( \frac{\Phi_i(x,y)}{x^{i+1}H(x)} \right) dx, x^{-i}y \right), i =1, \ldots , g,
\end{equation}
together with the residue classes of 
\begin{equation}\label{four}
\left( \frac{x^{i}}{H(x)} dx, \frac{x^{i}}{H(x)} dx, 0 \right), i=0, \ldots, g-1,
\end{equation}
form a basis of $\derhamhone$.
\end{thm}

Before proving this theorem we use it to prove the following corollary.

\begin{cor}
The action of $G$ on $\derhamhone$ is faithful unless $G$ contains a hyperelliptic involution and $p=2$, in which case the action of the hyperelliptic involution is trivial.
\end{cor}

\begin{proof}
Recall from Proposition \ref{ses} that $H^0(X,\Omega_X)$ injects into $\derhamhone$.
Then if $p \neq 2$ or $G$ does not contain a hyperelliptic involution it follows from \cite[Thm. 4.2]{faithfulaction} that $G$ acts faithfully on $H^0(X,\Omega_X)$, and hence $G$ acts faithfully on $\derhamhone$.

We now suppose that $p=2$ and that $G$ contains a hyperelliptic involution, which we denote by $\sigma$.
By the same theorem from \cite{faithfulaction} as used in the last paragraph, we know that $\sigma$ acts trivially on $\hzero$.

Since $\hzero$ is dual to $\hone$ we know that $\sigma$ acts trivially on $\hone$.
We can study exactly why this is from the view of \cech cohomology, and this will also help to determine the action of $\sigma$ on $\derhamhone$.
If we fix a natural number $i\in \{1, \ldots ,g\}$ then $\sigma$ maps $\frac{y}{x^i}$ to $\frac{y}{x^i} + \frac{H(x)}{x^i}$. 
Now we can split $\frac{H(x)}{x^i}$ as follows, 
\begin{equation*}
\frac{H(x)}{x^i} =  \frac{b_{i-1}x^{i-1} + \ldots + b_1x + b_0}{x^i} - \left( - \frac{x^d + b_{d-1}x^{d-1} + \ldots + b_ix^i}{x^i} \right),
\end{equation*}
and since this is clearly the difference of an element of $\cO_X(U_0)$ and an element of $\cO_X(U_\infty)$ we see that $\frac{H(x)}{x^i}$ is zero in $\hone$.
We let 
\[
H_{1,i}(x) = b_{i-1}x^{i-1} + \ldots + b_1x + b_0
\]
and 
\[
H_{2,i}(x) = -( x^d + b_{d-1}x^{d-1} + \ldots + b_ix^i).
\]

We now consider the action of $\sigma$ on the entries in \eqref{three}.
Firstly we see that
\begin{align*}
\sigma \left( \frac{-\Psi_i(x,y)}{x^{i+1}H(x)} dx\right) & = \frac{-\sigma(\Psi_i(x,y))}{x^{i+1} H(x)} dx \\
& = \frac{-\Psi_i(x,y)}{x^{i+1}H(x)}dx + \frac{H(x)(xH_{1,i}'(x) + iH_{1,i}(x))}{x^{i+1}H(x)}dx\\
& = \frac{-\Psi_i(x,y)}{x^{i+1}H(x)}dx + \frac{xH_{1,i}'(x) + iH_{1,i}(x)}{x^{i+1}}dx \\
& = \frac{-\Psi_i(x,y)}{x^{i+1}H(x)}dx +  \frac{H_{1,i}'(x)}{x^i}dx + \frac{iH_{1,i}(x)}{x^{i+1}}dx \\
& = \frac{-\Psi_i(x,y)}{x^{i+1}H(x)}dx +  \frac{1}{x^i}d\left( H_{1,i}(x) \right) + H_{1,i}(x) d \left( \frac{1}{x^i} \right) \\
& = \frac{-\Psi_i(x,y)}{x^{i+1}H(x)}dx + d\left( \frac{H_{1,i}(x)}{x^i} \right),
\end{align*}
where the second equality follows from \eqref{capitals} and the fact that $\sigma(y) = y + H(x)$.

Similarly we can derive
\begin{equation*}
\sigma \left( \frac{\Phi_i(x,y)}{x^{i+1}H(x)} dx \right)  = \frac{\Phi_i(x,y)}{x^{i+1}H(x)} dx + d \left( \frac{H_{2,i}(x)}{x^i} \right).
\end{equation*}
Lastly, it is clear that $\sigma(x^{-i}y) = x^{-i}(y+H(x))$.


We can now describe exactly how sigma acts on the elements of \eqref{three} using $H_{1,i}(x)$ and $H_{2,i}(x)$:
\begin{multline*}
\sigma \left( \left( \left(\frac{-\Psi_i(x,y)}{x^{i+1}H(x)}\right) dx, \left( \frac{\Phi_i(x,y)}{x^{i+1}H(x)} \right) dx, x^{-i}y \right)\right) = \\
 \left( \left(\frac{-\Psi_i(x,y)}{x^{i+1}H(x)} \right) dx + d\left(\frac{H_{1,i}(x)}{x^i}\right),  \left( \frac{\Phi_i(x,y)}{x^{i+1}H(x)} \right) dx+ d\left(\frac{H_{2,i}(x)}{x^i} \right), \frac{y+H(x)}{x^i} \right).
\end{multline*}
So the action of $\sigma$ on the basis elements in \eqref{three} amounts to adding 
\[
\left( d\left(\frac{H_{1,i}}{x^i}\right), d\left(\frac{H_{2,i}}{x^i}\right), \frac{H(x)}{x^i} \right),
\]
which clearly satisfies the conditions of \eqref{quotient} and hence is zero.
So the action of the involution $\sigma$ on $\derhamhone$ is trivial and hence the action of the group $G$ is not faithful.
\end{proof}

\begin{rem}
We briefly study the action of $\sigma$ on the elements \eqref{one} (when $p\neq 2$).
When $p \neq 2$ then $\sigma$ acts by $(x,y) \mapsto (x,-y)$.
If we let
\[
\nu_i = \left( \left( \frac{\psi_i(x)}{2yx^{i+1}}\right) dx, \left(\frac{-\phi_i(x)}{2yx^{i+1}}\right) dx, x^{-i}y\right)
\]
then 
\begin{equation*}
\sigma(\nu_i) = -\nu_i.
\end{equation*}
Similarly, if 
\[
\eta_i = \left( \frac{x^i}{y}dx, \frac{x^i}{y}dx, 0 \right)
\]
then 
\[
\sigma(\eta_i) = - \eta_i.
\]
Hence $\sigma$ acts by multiplication with $-1$ on $\derhamhone$.
\end{rem}


We now prove Theorem \ref{basis}.

\begin{proof}
We make use of the fact that the short exact sequence in Lemma \ref{ses} splits as a sequence of vector spaces over $k$, and that we know bases of the outer two terms.

It is clear that the elements in \eqref{two} and \eqref{four} are elements of the space \eqref{derhamconditions}. 
In fact, it follows from \cite[Thm 6.1]{faithfulaction} that they are the image of a basis of $H^0(X,\Omega_X)$ in $\derhamhone$.

Moreover, it is obvious that if the elements in \eqref{one} and \eqref{three} are well defined elements of the space \eqref{derhamconditions} then they will map to the basis of $\hone$ given in Lemma \ref{basish1}.
So we need only show that the terms in \eqref{one} and \eqref{three} satisfy the conditions stated in \eqref{derhamconditions}.
For the rest of the proof we fix $i \in \{1, \ldots ,g\}$.


We start with the case $p\neq 2$, and observe that
\begin{align*}
\left(  \frac{\psi_i(x)}{2yx^{i+1}}  - \frac{-\phi_i(x)}{2yx^{i+1}} \right) dx & =  \frac{s_i(x)}{2yx^{i+1}} dx \\
& =  \frac{1}{2yx^i} \left( f' - \frac{2if}{x} \right) dx \\
& =  \frac{x^i}{2y} \left( \frac{f'}{x^{2i}}dx -\frac{2if}{x^{2i+1}} dx \right) \\
& =  \frac{x^i}{2y} \left( fd\left(\frac{1}{x^{2i}}\right) + \frac{1}{x^{2i}}df \right) \\
& =  \frac{x^i}{2y}d(fx^{-2i}) \\
& =  \frac{x^i}{2y} d\left(\left(yx^{-i}\right)^2\right) \\
& =  d(yx^{-i}),
\end{align*}
with the penultimate line following from the defining equation \eqref{definingequationpnot2}.
This shows that the elements in \eqref{one} satisfy $df_{0, \infty} = \omega_0 - \omega_\infty$, one of the conditions of \eqref{derhamconditions}.
Since we saw in the proof of Lemma \ref{basish1} that $\frac{y}{x^i}$ is regular on $U_0\cap U_\infty$ it only remains to show that $\frac{\phi_i}{2yx^{i+1}}dx$ and $\frac{-\psi_i}{2yx^{i+1}}dx$ are regular on $U_\infty$ and $U_0$ respectively.


In order to do this we define $\alpha_{j,i} \in k$ for $0\leq j \leq 2g+2$ to satisfy the equation
\[
s_i(x) = \alpha_{2g+2,i}x^{2g+2} + \ldots + \alpha_{0,i},
\]
so that
\[
\phi_i(x) = \alpha_{2g+2,i}x^{2g+2} + \ldots + \alpha_{g+2,i}x^{g+2} \ {\rm and } \ \psi_i(x) = \alpha_{g+1,i}x^{g+1} + \ldots + \alpha_{0,i}.
\]
Note that it is possible for any of $\alpha_{j,i}$ to be zero. In fact, it is possible for either $\phi_i(x)$ or $\psi_i(x)$ to be zero.
Whenever they are non-zero we denote their degrees as polynomials in $x$ by $d_\phi$ and $d_\psi$ respectively. From the definition of $\phi_i(x)$ and $\psi_i(x)$ we know that $0 \leq d_\psi \leq g+1$ and $g+1 < d_\phi \leq 2g+2$.
\todo{remove d psi? - will keep for now in case useful later}


We now show that $\frac{-\phi_i}{2yx^{i+1}}dx$ and $\frac{\psi_i}{2yx^{i+1}}dx$ are regular on $U_\infty$ and $U_0$ respectively.
We may assume that $\phi_i$ and $\psi_i$ are non-zero, since the zero function is regular everywhere.
The divisor of $\frac{-\phi_i}{2yx^{i+1}}dx$ is

\begin{align*}
\di\left( \frac{-\phi_i}{2yx^{i+1}}dx \right) & =  \di(\phi_i) -\di(y) - \di(x^{i+1}) + \di (dx) \\
& =  \di(\phi_i) - ( R - (g+1)D_\infty) - ((i+1)D_0 - (i+1)D_\infty) \\
& \qquad + (R - 2D_\infty) \\
& =  \left( \di_0\left( \frac{\phi_i}{x^{g+2}}\right) + (g+2)D_0 - d_\phi D_\infty\right) - (i+1)D_0 + (g+i)D_\infty \\
& \geq  \di_0\left( \frac{\phi_i}{x^{g+2}}\right) + (g+2)D_0 - (2g+2)D_\infty - (i+1)D_0 + (g+i)D_\infty \\
& =  \di_0\left( \frac{\phi_i}{x^{g+2}} \right) + (i-g-2)D_\infty + (g-i+1)D_0,
\end{align*}
where the second equality makes use of \eqref{divxp=2} and \eqref{pnot2divisors}.
Since $i \leq g$ the differential $\frac{-\phi_i}{2yx^{i+1}}dx$ is regular on $U_\infty = X\backslash \pi^{-1}(\infty)$.

Similarly the divisor of $\frac{\psi_i}{2yx^{i+1}}dx$ is

\begin{align*}
\di \left( \frac{\psi_i}{2yx^{i+1}}dx\right) & =  \di(\psi_i) - \di(y) - \di(x^{i+1}) + \di (dx) \\
& =  \di (\psi_i ) -(R - (g+1)D_\infty) - ((i+1)D_0 - (i+1)D_\infty) \\ 
& \qquad + (R -2D_\infty) \\
& =  \di(\psi_i) + (g+i)D_\infty -(i+1)D_0 \\
& =  (\di_0(\psi_i) -d_\psi D_\infty) + (g+i)D_\infty -(i+1)D_0 \\
& \geq \left( \di_0(\psi_i) - (g+1)D_\infty \right) + (g+i)D_\infty -(i+1)D_0 \\
& =  \di_0(\psi_i) + (i-1)D_\infty - (i+1)D_0.
\end{align*}
Again, the second equality uses \eqref{divxp=2} and \eqref{pnot2divisors}, and since $i\geq 1$ we conclude that $\frac{\psi_i(x)}{2yx^{i+1}}dx$ is regular on $U_0 = X \backslash \pi^{-1}(0)$, completing the $p\neq 2$ case.


We now suppose that $p=2$.
We remind the reader that this allows us to change signs between positive and negative as we wish.
We see that
\begin{align*}
\left( \left( \frac{ \Psi_i}{x^{i+1}H} \right) + \left( \frac{\Phi_i}{x^{i+1}H} \right) \right) dx & =  \frac{S_i}{x^{i+1}H}dx \\
& =  \left( \frac{F'}{x^iH} + \frac{yH'}{x^iH} + \frac{iy}{x^{i+1}} \right) dx \\
& =  \frac{1}{x^i}\left( \frac{F' + yH'}{H} \right) dx + \frac{iy}{x^{i+1}}dx \\
& =  x^{-i}dy + yd \left( x^{-i}\right) \\
& =  d\left( yx^{-i}\right),
\end{align*}
with the fourth equality following from \eqref{divdyp=2}.
We have also already seen in the proof of Lemma \ref{basish1} that $\frac{y}{x^i}$ is regular on $U_0 \cap U_\infty$.
So in order to prove that for $i\in \{1, \ldots, g\}$ the elements of \eqref{three} are satisfy the conditions of \eqref{derhamconditions} it only remains to show that the differentials $\frac{\Phi_i}{x^{i+1}H(x)}dx$ and $\frac{\Psi_i}{x^{i+1}H(x)}dx$ are regular on $U_\infty$ and $U_0$ respectively.
We denote the degrees of the polynomials defined in \eqref{Split} by $d_{\Phi}^x, d_{\Psi}^x, d_{\Phi}^y$ and $d_{\Psi}^y$.


By \eqref{Split} $\Phi_i(x,y) = \Phi_i^x(x) + y\Phi_i^y(x)$ and $\Psi_i (x,y)= \Psi_i^x(x) + y\Psi_i^y(x)$, and we will use these splittings to show that $\frac{ \Phi_i(x) }{x^{i+1}H(x)}dx$ and $\frac{\Psi_i(x) }{x^{i+1}H(x)}dx$ are regular on $U_\infty$ and $U_0$ respectively.\todo{note that some stuff was removed before this para}

We start by computing the divisor of $\frac{1}{x^{i+1}H(x)}dx$, since it is a common component to all the differentials we need to look at.
This yields
\begin{align*}
\di \left( \frac{1}{x^{i+1}H(x)}dx \right) & = \di(dx) - \di (x^{i+1}) - \di (H(x)) \nonumber \\
& = (R-2D_\infty) - ((i+1)D_0 - (i+1)D_\infty) - (R - (g+1)D_\infty) \nonumber \\
& = (g+i)D_\infty - (i+1)D_0,
\end{align*}
using \eqref{differentialdivisor}, \eqref{divisorofH} and \eqref{divxp=2}.
We now use this along with Proposition \ref{divyp=2} and the polynomials \eqref{Split} to complete the proof.

We begin by computing the divisors associated to $\Phi_i(x,y)$.
Firstly,
\begin{align*}
\di \left( \frac{\Phi_i^x(x) }{x^{i+1} H(x)}dx \right)  = &  \di(\Phi_i^x(x)) -(i+1)D_0 + (g+i)D_\infty\\
 = & \left( \di_0(\Phi_i^x(x)) -d_\Phi^xD_\infty\right) -(i+1)D_0 + (g+i)D_\infty\\
 \geq & \di_0(\Phi_i^x(x)) - (2g+2)D_\infty - (i+1)D_0 + (g+i)D_\infty \\
 = &  \di_0(\Phi_i^x(x)) - (i+1)D_0 + (i-2-g)D_\infty \\
 =  & \di_0 \left( \frac{\Phi_i^x(x)}{x^{i+1}} \right) + (i-g-2)D_\infty.
\end{align*}
From this we see that the differential $\frac{\Phi_i^x(x)}{x^{i+1}H(x)}dx$ is clearly regular on $U_\infty = X \backslash \pi^{-1}(\infty)$.

We now compute the divisor of the other half of $\frac{\Phi_i(x,y)}{x^{i+1}H(x)}dx$, namely
\begin{align*}
\di\left(\frac{y\Phi_i^y(x) dx}{x^{i+1}H(x)} \right)  = & \di(y) + \di(\Phi_i^y(x)) -(i+1)D_0 + (g+i)D_\infty\\
 = & \di(y) + \di_0(\Phi_i^y(x)) - d_\Phi^yD_\infty -(i+1)D_0 + (g+i)D_\infty \\
 \geq & \di(y) + \di_0(\Phi_i^y(x)) - (g+1)D_\infty - (i+1)D_0 + (g+i)D_\infty \\
 = & \di(y) + \di_0\left(\frac{\Phi_i^y(x)}{x^{i+1}} \right) + (i-1)D_\infty.
\end{align*}
From Proposition \ref{divyp=2} we see that $y$ only has poles at points in $\pi^{-1}(\infty)$, and hence this completes the proof that $\frac{\Phi_i(x,y) }{x^{i+1}H(x)}dx$ is regular on $U_\infty = X \backslash \pi^{-1}(\infty)$.

Now we complete the same computations on $\Psi_i(x,y)$, starting with $\Psi_i^x(x)$:
\begin{align*}
\di\left( \frac{\Psi_i^x(x) }{x^{i+1}H(x)}dx \right)  & =   \di(\Psi_i^x(x))  - (i+1)D_0 + (g+i)D_\infty \\
& = (\di_0(\Psi_i^x(x)) -d_\Psi^xD_\infty) - (i+1)D_0 + (g+i)D_\infty \\
 & \geq   \di_0(\Psi_i^x(x) ) - iD_\infty - (i+1)D_0 + (g+i)D_\infty \\
 & =   \di_0(\Psi_i^x(x)) - (i+1)D_0 + gD_\infty,
\end{align*}
and it is clear that the divisor is positive on $U_0 = X \backslash \pi^{-1}(0)$.

For the other half of the differential we need to consider separate cases.
If we assume that $\infty$ is branch point then  using Proposition \ref{divyp=2} we see that
\begin{align*}
\di\left(\frac{y\Psi_i^y(x) }{x^{i+1}H(x)}dx \right)  =  & \di_0(y) - (2g+1)[P_\infty] + \di(\Psi_i^y(x)) - (i+1)D_0 + (g+i)[P_\infty] \\
 =  & \di_0(y) + \di(\Psi_i^y(x)) -(i+1)D_0 + (2i -1)[P_\infty] \\
 = &  \di_0(y) + \di_0(\Psi_i^y(x)) - d_\Psi^y[P_\infty] - (i+1)D_0 + (2i-1)[P_\infty] \\
 \geq &  \di_0(y) + \di_0(\Psi_i^y(x)) -(i-1)[P_\infty] -(i+1)D_0 + (2i-1)[P_\infty] \\
 =   &\di_0(y) + \di_0(\Psi_i^y(x)) -(i+1)D_0 + [P_\infty],
\end{align*}
which is clearly positive on $U_0$.
On the other hand, if $\infty$ is not a branch point then we have
\begin{align*}
\di\left(\frac{y\Psi_i^y(x) }{x^{i+1}H(x)}dx \right)  =  & \di(y) + \di(\Psi_i^y(x)) - (i+1)D_0 + (g+i)D_\infty \\
= & \di(y) + \di_0(\Psi_i^y(x)) - (i+1)D_0 + (g+i - d_\Psi^y)D_\infty \\
\geq & \di(y) + \di_0(\Psi_i^y(x)) - (i+1)D_0 + (g+1)D_\infty. \\
\end{align*}
Since we know from Proposition \ref{divyp=2} that $y$ cannot have a pole of order greater $g+1$ at $P_\infty$ or $P_\infty'$, and only has poles at these points, it follows that the differential $\frac{y\Psi_i^y(x) }{x^{i+1}H(x)}dx$ is regular on $U_0 = X \backslash \pi^{-1}(0)$.
Thus we have completed the proof.


\end{proof}

\section{Computing the residue}
\todo[inline]{check that don't need to change any basis elements after multiplying $s_i$ by $x$}

Introduction needs to be finalised once the proof in the previous section is completed.
In particular, if the exact residue can be computed then we may as well do so.
Currently we are assuming in this section that $\infty$ and 0 are a branch point.

\begin{lem}
Suppose that $\infty\in \mathbb P_k^1$ is a branch point of $\pi$.
If $p \neq 2$ then ${\rm res}_{P_\infty}(\frac{1}{x}dx) = -2$.
If $p=2$ then ${\rm res}_{P_\infty}\left(\frac{y}{xH(x)}dx\right) = 1$.
In particular, this shows that the basis of $\hone$ in Lemma \ref{basish1} is dual to the basis given for $\hzero$ in \cite{faithfulaction}.
\end{lem}

\begin{proof}

We first consider the case $p\neq 2$.
We note that $t:= \frac{y}{x^{g+1}}$ is a uniformising parameter at $P_\infty$ as can be seen by computing the order of $t$ at $P_\infty$ as follows
\begin{align*}
\ord_{P_\infty}(t) & =  \frac{1}{2}\ord_{P_\infty}(t^2) \\
  & =  \frac{1}{2}\ord_{P_\infty}\left( \frac{f}{x^{2g+2}} \right) \\
& =  \frac{1}{2}\ord_{P_\infty}(f(x)) - \frac{1}{2}\ord_{P_\infty}(x^{2g+2})\\
& =  -(2g+1) + (2g+2) \\
& =  1.
\end{align*}

We now write $\frac{1}{x}dx$ in terms of $dt$.
By the quotient rule we know that
\begin{align*}
dt^2 & =  d \left( \frac{f}{x^{2g+2}} \right) \\
& =  \frac{x^{2g+2}f' - (2g+2)x^{2g+1}f}{x^{4g+4}} dx \\
& =  \frac{1}{x^{2g+2}} \left( f' - \frac{(2g+2)f}{x} \right) dx
\end{align*}
from which we conclude that
\[
\frac{1}{x}dx = \frac{2tx^{2g+1}}{\left(\frac{f' - (2g+2)f}{x} \right)} dt.
\]


We now let $p(x) = \left(\frac{(2g+2)f(x)}{x} - f'(x)\right)$, and noting that the coefficient of $x^{2g}$ in $p(x)$ is $(2g+2)a_{2g+1} - (2g+1)a_{2g+1} = a_{2g+1}$, we see that $h(x)$ is a degree $2g$ polynomial in $x$.
We wish to compute the coefficient of $t^{-1}$ in the expansion of $\frac{1}{x}dx$ at $P_\infty$ and for this we require the following expansions
\begin{align*}
\frac{p(x)}{x^{2g+1}} = \frac{a_{2g+1}x^{2g}}{x^{2g+1}} + \ldots = \frac{a_{2g+1}}{x} + \ldots \qquad \text{and} \qquad t^2 = \frac{f}{x^{2g+2} } = \frac{a_{2g+1}}{x} + \ldots
\end{align*}

Since $\ord_{P_\infty}\left(\frac{p(x)}{x^{2g+1}}\right) = 2$ we know that $\frac{p(x)}{x^{2g+1}} = \sum_{j\geq 2} c_j t^j$ for some $c_j\in k$, and from the above computations we can see that $c_2 = 1$.
We also know that $\frac{x^{2g+1}}{p(x)} = \sum_{k\geq -2} d_kt^k$, for some $d_k\in k$, and clearly $d_{-2} = 1$.
Now
\[
\frac{1}{x}dx = \left( -2t \cdot \sum_{i\geq -2} d_it^i\right) dt 
\]
so we see that $\res_{P_\infty}\left( \frac{1}{x} dx\right) = 2$.
This completes the proof of the lemma when $p\neq 2$.

We now turn to the case when $p=2$.
We now wish to compute the residue of $\frac{y}{xH(x)}dx$ at $P_\infty$.
We start by noting that $t = \frac{y}{x^{g+1}}$ is a uniformising parameter at $P_\infty$ which we check by computing the divisor:
\begin{align*}
\di(t) & = \di_0(y) - (2g+1)[P_\infty] -(g+1)D_0 + (g+1)D_\infty \\
& = \di_0(y)-(g+1)D_0 + [P_\infty],
\end{align*}
using \eqref{divxp=2} and Proposition \ref{divyp=2}.
So clearly $t$ is a uniformising parameter at $P_\infty$.

We now wish to write $\frac{y}{xH(x)}dx$ as $r(x,y)dt$ for some $r \in k(x,y)$.
We first write $dy$ in terms of $dx$.
Since
\begin{align*}
0 & =  dy^2 \\
& =  d(F+yH) \\
& =  F'dx + Hdy + yH'dx
\end{align*}
we conclude that
\[
dy = \left( \frac{F'+yH'}{H} \right) dx.
\]

We also rewrite $dt$ as follows:
\begin{align*}
dt & =  d\left( \frac{y}{x^{g+1}} \right) \\
& =  yd\frac{1}{x^{g+1}} + \frac{1}{x^{g+1}}dy \\
& =  \frac{1}{x^{g+1}} \left( \frac{(g+1)y}{x} + \frac{F'+yH'}{H} \right) dx \\
& =  \frac{1}{x^{g+1}} \left( \frac{xF'}{y} + xH' + (g+1)H \right) \frac{y}{Hx} dx.
\end{align*}

In total we then have
\[
\frac{y}{xH(x)}dx = \frac{x^{g+1}y}{S_{g+1}(x,y)}dt
\]
where $S_{g+1}(x,y)$ is as defined in \eqref{capitals}.

Since we saw in the proof of Lemma \ref{basish1} that $\frac{y}{xH(x)}dx$ has a pole of order at $P_\infty$, we have the Taylor expansion $\frac{x^{g+1}y}{S_{g+1}(x,y)} = \sum_{i\geq -1} c_i t^i$, with $c_i \in k$, and we wish to compute $c_{-1}$.
We shall do this by computing the coefficient $d_1$ of $t$ in the expansion $\frac{S_{g+1}(x,y)}{x^{g+1}y} = \sum_{i\geq 1}d_it^i$.
We can split up $\frac{S_{g+1}(x,y)}{x^{g+1}y}$ into three terms, namely $\frac{xF'}{x^{g+1}y}$, $\frac{yxH'}{x^{g+1}y}$ and $\frac{(g+1)yH}{x^{g+1}y}$.
Each of these terms can of course be written as a power series in $t$.
However, to compute $d_1$ we need only compute the coefficient of $t$ in these power series.
Since $\infty$ is a branch point then $H$ has degree at most $g$ (and hence $xH'$ also has degree at most $g$).
It then follows that the order of both $\frac{xH'}{x^{g+1}}$ and $\frac{(g+1)H}{x^{g+1}}$ at $P_\infty$ is at least 2, and hence the coefficient of $t$ in their power series expansion is zero.
The order of $\frac{xF'}{x^{g+1}y}$ at $P_\infty$ on the other hand is precisely 1, and hence the coefficient of $t$ in the power series expansion of $\frac{xF'}{x^{g+1}y}$ will determine $d_1$.


Suppose that $F = \alpha_{2g+1}x^{2g+1} + \alpha_{2g}x^{2g} + \ldots + \alpha_1x^1 + \alpha_0$.
Then $xF'= \alpha_{2g+1}x^{2g+1} + \alpha_{2g-1}x^{2g-1} + \ldots + \alpha_1x^1$; i.e. the terms with an even power are removed.

So the only term in $\frac{xF'}{x^{g+1}y}$ of order 1 at $P_\infty$ is $\frac{\alpha_{2g+1}x^{2g+1}}{x^{g+1}y} = \frac{\alpha_{2g+1}x^{g}}{y}$.
Since
\[
\frac{\alpha_{2g+1}x^g}{y} = \frac{\alpha_{2g+1}}{x}t^{-1}
\]
and $\frac{1}{x} = \sum_{i\geq 2}e_it^i$ for some $e_i \in k$, if we compute $e_2$ then we will have effectively computed $d_1$.
Now $t^2 = \frac{F }{x^{2g+2}}+ \frac{Hy}{x^{2g+2}}$, and clearly $\frac{Hy}{x^{2g+2}}$ has no monomial term of the form $\frac{c}{x}$ with $c \in k$.
On the other hand
\[
\frac{F}{x^{2g+2}} = \frac{\alpha_{2g+1}}{x} + \ldots
\]
Hence we conclude that $e_2 = \frac{1}{\alpha_{2g+1}}$.
It follows that $d_1 = \alpha_{2g+1} \cdot \frac{1}{\alpha_{2g+1}} = 1$.
We finally use this to conclude that $c_{-1} = \frac{1}{d_{1}} = 1$.




\end{proof}
\begin{comment}

\section{Stability of modular curves in characteristic greater than 2}
In this section we consider some modular hyperelliptic curves and how their automorphism groups at on the first de Rham cohomology group of the curves.
The automorphism groups of these curves can be found in \cite{automorphismshyperellipticmodular}.
We start by considering the curve $X_0(22)$, which can defined by the equation
\begin{equation*}
y^2 = (x^3+ 4x^2 + 8x + 4)(x^3 + 8x^2 + 16x + 16).
\end{equation*}
The automorphism group of this curve is, in general, $\mathbb Z_2 \times \mathbb Z_2$, with the action given by 
\begin{align}
\omega_2\colon (x,y) & \mapsto \left( \frac{4}{x} , \frac{8y}{x^3} \right), \\
\omega_{11}\colon (x,y) & \mapsto (x, -y ).
\end{align}
Note that we are using the notation of \cite{automorphismshyperellipticmodular} for consistency.
We now study how this group action affects the basis of $\derhamhone$.
In particular, we show that the basis associated to $\hone$ in $\derhamhone$ is not stable under the action of the automorphism group.
For each $i \in \{ 1, \ldots , g\}$ we let 
\begin{equation*}
\nu_i = \left( \left( \frac{-\psi_i(x)}{2yx^i} \right) dx , \left( \frac{\phi_i(x)}{2yx^i} \right) dx, x^{-i}y \right).
\end{equation*}
We then note that 
\[
\omega_2(dx) = d( \omega_2(x)) = d\left( \frac{4}{x} \right) = \frac{-4}{x^2}dx.
\]
If we recall that 
\[
s_i(x) = \alpha_{2g+1}^ix^{2g+1} + \ldots + \alpha_0^i
\]
then we can see that the $\omega_2$ acts as follows:
\begin{equation}\label{omega11action}
\begin{split}
\omega_2 (\nu_i) & = \left( -\sum_{j = g+1}^{2g+1} \left( \frac{\alpha_j^i 4^{j-i}x^3}{2x^{j-i}y} \right) \frac{-4}{x^2} dx, \sum_{j = 0}^{g} \left(\frac{\alpha_j^i 4^{j-i} x^3}{2x^{j-i}y} \right) \frac{-4}{x^2}dx , \frac{8yx^i}{4^i x^3} \right) \\
& = \left( \sum_{j=g+1}^{2g+1} \left( \frac{\alpha_j4^{i-j+1}}{2yx^{j-i-1}} \right)dx, \sum_{j = 0}^{g} \left( \frac{-\alpha_j^i 4^{j-i+1}}{2yx^{j-i-1}} \right) dx, \frac{2yx^{i-3}}{4^{i-1}} \right).
\end{split}
\end{equation}

Now it appears that these terms are similar to those of $s_i \left( \frac{4}{x} \right)$, so it would be good to see if we can somehow write $y^2$ in terms of  some function of $\omega_2(x) = \frac{4}{x}$.
We can do this, since of course $\omega_2(y)^2 = \omega_2(y^2) = \omega_2(f(x)) = \omega_2\left( f \left(\frac{4}{x} \right) \right)$.
From this we obtain the identity
\begin{equation*}
y^2 = \frac{x^6 f \left(\frac{4}{x} \right)}{8^2}.
\end{equation*}
From this we obtain
\begin{align*}
dy & = \frac{1}{2y \cdot 8^2} \left( f'\left(\frac{4}{x} \right) x^6 d\left(  \frac{4}{x} \right) + 6 f\left( \frac{4}{x} \right) x^5 dx \right) \\
& = \frac{1}{2y \cdot 8^2} \left( f'\left(\frac{4}{x} \right) \cdot (-4 x^4) + 6 f\left( \frac{4}{x} \right) x^5  \right) dx \\
\end{align*}

We can then use this to compute $d (\omega_2(x^{-i}y)) = d \left( \frac{8yx^{i-3}}{4^i} \right)$, which we now do:
\begin{align*}
d\left( \frac{8yx^{i-3}}{4^i} \right) & = \frac{8}{4^i} \left( x^{i-3}dy + ydx^{i-3} \right) \\
& = \frac{8}{4^i} \left( \frac{x^{i-3}}{2y8^2} \left( \frac{-4}{x^2} \cdot f'\left( \frac{4}{x}  \right) x^6 + 6x^5 f\left( \frac{4}{x} \right) \right) + (i-3)x^{i-4}y \right) dx \\
& = \frac{8}{4^i} \left( \frac{1}{2y8^2} \left(\frac{-4}{x^2} \cdot  f'\left( \frac{4}{x} \right) x^{i+3} + 6x^{i+2} f\left( \frac{4}{x} \right) \right) + \frac{(i-3)x^{i-4}x^6f \left( \frac{4}{x} \right)  }{y8^2} \right) dx \\
& = \frac{8}{4^i} \cdot \frac{x^{i+3}}{2y8^2} \left(\frac{-4}{x^2} \cdot  f' \left( \frac{4}{x} \right)  + \frac{6}{x} f \left( \frac{4}{x} \right)  + \frac{2(i-3)f \left( \frac{4}{x} \right)  }{x} \right) dx \\
& = \frac{x^{i+3}}{2 \cdot 4^i \cdot 8 \cdot y} \left(\frac{-4}{x^2} \cdot  f'\left( \frac{4}{x} \right)  + \frac{ 2i f \left( \frac{4}{x} \right) }{x} \right) dx \\
& = \frac{x^{i+1}}{4^{i+1}y} \left( -f' \left(\frac{4}{x} \right) + \frac{2if \left( \frac{4}{x} \right) x}{4} \right) dx.
\end{align*}
\todo[noline]{Apart from the minus sign in front of $f' \left( \frac{4}{x} \right)$ and a factor of two almost matches up - check computations}Note that the function before $dx$ here is precisely $\omega_2\left(\frac{s_i(x)}{2yx^i} \right)$.

This (almost) shows that
\[
\left( \omega_2 \left( \left( \frac{s_i(x)}{2yx^i} \right) dx \right) , 0, \omega_2( x^{-i}y)\right)
\]
is the zero element in $\derhamhone$.

We will now finally show that the basis of $\hone$ in $\derhamhone$ is not stable under the group action --- in particular, it is not stable under the action of $\omega_2$.
Now we fix $i = g-1$.
We can then clearly see from \eqref{omega11action} that if $\alpha_{g}^i$ is zero all the $x$ terms in the second co-ordinate will have degree at most $g-1$.
\begin{note}
By the action of $\omega_2$ we have $\frac{8^2y^2}{x^6} = f\left(\frac{4}{x} \right)$.
\end{note}

\begin{note}
For $\infty$ is not a branch point the same basis should work. 
Consider $y^2 = f(x)$ where the degree of $f(x)$ is $2g+2$ (i.e. $\infty$ is not ramified), with projection map $\pi : X \rightarrow \mathbb P_k^1$.
Then we can take the cover to be $U_0 := X \backslash \{\pi^{-1}(0)\}$ and $U_\infty := X \backslash \{\pi^{-1}(\infty)\}$.
Then we have $\frac{x^idx}{y} \cdot \frac{y}{x^{j}} = x^{i-j}dx$.
The divisor of this is then $(i-j)D_0 -(i-j)D_\infty + R - 2D_\infty$.
Then if $i-j = -1$ this has poles at at all points supported at by $D_0$ and $D_\infty$.


If $p=2$ and $\infty$ is not a branch point then we need to consider the defining equation $y^2 - h(x)y = f(x)$ where $\deg(h)=g+1$, and $\deg(f)$ can be anything from 0 to $2g+2$.
From work (which I think was not in 18 month report), if we suppose that $P_\infty$ and $P_\infty'$ are the two points in the pre-image of $\infty \in \mathbb P_k^1$, then the divisor of $y$ is
\[
\di(y) = \di_0(y) +(g+1 - \deg(f))[P_\infty] - (g+1)[P_\infty'],
\]
up to parity of $P_\infty$ and $P_\infty'$.
So then, after some computation, we get
\begin{multline*}
\di\left(\frac{yx^{i-j}}{H(x)}dx \right) = \di_0(y) + (i-j)D_0 + (g-1-(i-j))D_\infty \\ + (g+1-\deg(f))D_\infty -(g+1)[P_\infty'].
\end{multline*}
So if $i-j = -1$ we clearly have a pole of order 1 at $P_\infty'$.
\end{note}

Now a short note about whether the main theorem still holds if we do not specify the ramification points.
\begin{note}
Clearly when $p\neq 2$ the same basis holds. In the proof we use the fact that $\phi_i$ has a factor of at least $x^{g+1}$, giving a sufficient zero at $P_0$.
This factor is still in the polynomial if we increase its potential degree.
Of course, we may have to replace $P_0$ with $D_0$, but the essential point still holds.
Similarly, after possibly replacing $P_\infty$ by $D_\infty$ we have almost the same argument for $\psi_i$.
So the basis is the same when $p\neq2$.

When $p=2$ the basis is also the same.
We first note that whilst the degree of the highest and lowest ordered terms may change in \eqref{Split}, it is still the case that the coefficient $B_i^i =0$, using exactly the same arguments.
This allows us to use essentially the same arguments, again with the exception of changing $P_\infty$ and $P_0$ to $D_\infty$ and $D_0$.
\end{note}
\end{comment}


\bibliography{biblio}
\bibliographystyle{amsalpha}


\end{document}


\bibliographystyle{amsalpha}
\bibliography{biblio}


\end{document}
