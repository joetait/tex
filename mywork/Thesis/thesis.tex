% !TEX TS-program = pdflatex
% !TEX encoding = UTF-8 Unicode

% This is a simple template for a LaTeX document using the "article" class.
% See "book", "report", "letter" for other types of document.

\documentclass[draft, 11pt]{article} % use larger type; default would be 10pt

\usepackage[utf8]{inputenc} % set input encoding (not needed with XeLaTeX)

%%% Examples of Article customizations
% These packages are optional, depending whether you want the features they provide.
% See the LaTeX Companion or other references for full information.

%%% PAGE DIMENSIONS
\usepackage{geometry} % to change the page dimensions
\geometry{a4paper} % or letterpaper (US) or a5paper or....
% \geometry{landscape} % set up the page for landscape
% read geometry.pdf for detailed page layout information

\usepackage{graphicx} % support the \includegraphics command and options
\usepackage[obeyDraft]{todonotes}

%\usepackage[parfill]{parskip} % Activate to begin paragraphs with an empty line rather than an indent

%%% PACKAGES
\usepackage{mathtools}
\usepackage{booktabs} % for much better looking tables
\usepackage{array} % for better arrays (eg matrices) in maths
\usepackage{paralist} % very flexible & customisable lists (eg. enumerate/itemize, etc.)
\usepackage{verbatim} % adds environment for commenting out blocks of text & for better verbatim
\usepackage{subfig} % make it possible to include more than one captioned figure/table in a single float
% These packages are all incorporated in the memoir class to one degree or another...

\usepackage[activate={true,nocompatibility},final,tracking=true,kerning=true,spacing=true,factor=1100,stretch=10,shrink=10]{microtype}
\microtypecontext{spacing=nonfrench}
% activate={true,nocompatibility} - activate protrusion and expansion
% final - enable microtype; use "draft" to disable
% tracking=true, kerning=true, spacing=true - activate these techniques
% factor=1100 - add 10% to the protrusion amount (default is 1000)
% stretch=10, shrink=10 - reduce stretchability/shrinkability (default is 20/20)

%%% HEADERS & FOOTERS
\usepackage{fancyhdr} % This should be set AFTER setting up the page geometry
\pagestyle{fancy} % options: empty , plain , fancy
\renewcommand{\headrulewidth}{0pt} % customise the layout...
\lhead{}\chead{}\rhead{}
\lfoot{}\cfoot{\thepage}\rfoot{}

%%% SECTION TITLE APPEARANCE
\usepackage{sectsty}
\allsectionsfont{\sffamily\mdseries\upshape} % (See the fntguide.pdf for font help)
\usepackage{amsmath}
\usepackage{amsthm}
\usepackage{amsfonts}
\usepackage{mathrsfs}
\usepackage{amsopn}
\usepackage{amssymb}
\usepackage{natbib}
% (This matches ConTeXt defaults)

%%% ToC (table of contents) APPEARANCE
\usepackage[nottoc,notlof,notlot]{tocbibind} % Put the bibliography in the ToC
\usepackage[titles,subfigure]{tocloft} % Alter the style of the Table of Contents
\renewcommand{\cftsecfont}{\rmfamily\mdseries\upshape}
\renewcommand{\cftsecpagefont}{\rmfamily\mdseries\upshape} % No bold!
%\renewcommand{\familydefault}{\sfdefault}
%\usepackage{cabin}
%\usepackage{libertine}
%\usepackage[T1]{fontenc}

%Theorems and stuff
\theoremstyle{plain}
\newtheorem{defn}{Definition}[section]
\newtheorem{thm}[defn]{Theorem}
\newtheorem{cor}[defn]{Corollary}
\newtheorem{lem}[defn]{Lemma}
\newtheorem{prop}[defn]{Proposition}
\newtheorem{ex}[defn]{Example}
\newtheorem*{unnumthm}{Theorem}
\newtheorem{defnlem}[defn]{Definition/Lemma}
\newtheorem{defnthm}[defn]{Theorem/Definition}
\theoremstyle{remark}
\newtheorem*{rem}{Remark}
\newtheorem*{note}{Note}


\newcommand{\cO}{{\cal O}}
\newcommand{\ra}{\rightarrow}
\newcommand{\NN}{{\mathbb N}}
\newcommand{\PP}{{\mathbb P}}
\newcommand{\ZZ}{{\mathbb Z}}
\newcommand{\cL}{{\mathcal L}}
\newcommand{\cA}{{\mathcal A}}
\newcommand{\cD}{{\mathcal D}}
\newcommand{\cU}{{\mathcal U}}
\newcommand{\cech}{\v{C}ech }
\newcommand{\hzero}{{H^0(X,\Omega_X)}}
\newcommand{\hone}{H^1(X,\mathcal{O}_X)}
\newcommand{\cechhone}{\check{H}^1(\mathcal U,\mathcal O_X)}
\newcommand{\derhamhone}{H_{\text {dR}}^1(X/k)}


\DeclareMathOperator{\aut}{Aut}
\DeclareMathOperator{\res}{Res}
\DeclareMathOperator{\ord}{ord}
\DeclareMathOperator{\di}{div}
\DeclareMathOperator{\cha}{char}
\DeclareMathOperator{\gal}{Gal}
\DeclareMathOperator{\Tr}{Tr}
\DeclareMathOperator{\Ima}{Im}

%%% END Article customizations

%%% The "real" document content comes below...

\title{Group actions on de-Rham cohomology of hyperelliptic curves}
\author{}
%\date{} % Activate to display a given date or no date (if empty),
         % otherwise the current date is printed

\begin{document}
\maketitle

%\listoftodos

\chapter{Introduction} \label{Chapter:introduction}

\section{Introduction}

In topology, the number of holes in a compact, orientable surface is an important invariant, called the genus, and classifies compact, orientable surfaces up to homeomorphism.
In particular the genus is an important topological invariant of compact Riemann surfaces.
It is well known that for any compact Riemann surface the genus is also equal to the dimension of the space of global holomorphic differentials.
Furthermore there is a correspondence between compact Riemann surfaces and smooth projective algebraic curves over $\mathbb C$, and the notion of holomorphic differentials can be extended to such curves.
In fact we can extend this even further, by defining the genus of any curve over an algebraically closed field to the dimension of the space of global holomorphic differentials.
The above alone makes it obvious that the space of global holomorphic differentials is a fundamental object in the theory of algebraic curves.
The general motivation underlying this report is to study this space as a representation of a subgroup of the automorphism group of the given curve.

\begin{comment} It is a well known and fundamental fact of the theory of Riemann surfaces that Riemann surfaces can be classified as $n$-tori (including the sphere as the zero torus).
This means that the topological invariant of genus is an invariant in this area of study too, where it is called the analytic genus.
Equally as important is the correspondence between compact Riemann surfaces and connected, projective, smooth algebraic curves over the complex numbers.
So it is natural to ask how the genus arises in the case of these curves.
As a corollary to the Riemann-Roch theorem it can be seen that the dimension of the space of {\em holomorphic differentials} (see Section \ref{Hurwitzsection}) over $\mathbb C$, which we denote by $H^0(X,\Omega_X)$, is what corresponds to the analytic genus.
This invariant is called the arithmetic genus, and can be extended to any curve over an algebraically closed field.
For this reason, among others, the space of holomorphic differentials is a fundamental object in algebraic geometry, and widely studied.
\end{comment}
Let $X$ be a smooth connected projective curve over an algebraically closed field $k$.
Given a subgroup $G$ of the automorphism group of $X$ then a classic problem pertaining to $H^0(X,\Omega_X)$, the space of holomorphic differentials (see section \ref{Hurwitzsection}), is determining its $k[G]$-module structure.
This originally dates back to 1934, and a paper of Chevalley and Weil \cite{chev}.
They only considered the case when $k= \mathbb C$, but the complete structure has since been discovered in the case where the projection from $X$ to the quotient curve is tamely ramified.
This was done by Kani in 1986 \cite{Kani}.
Progress has also been made recently in the case where the projection is wildly ramified; in particular Karanikolopoulos and A. Kontogeorgis \cite{kako} have computed the $k[G]$-module structure for any cyclic group $G$.
Also, in 1986 Broughton \cite{Broughton} computed the $k[G]$-module structure of the space of global holomorphic poly-differentials, $H^0(X,\Omega_X^{\otimes m})$ (see Section \ref{charneq2}), in the case where $\cha(k) = 0$.

In this report we will not look directly at the $k[G]$-module structure, but rather at the related question of determining when the action of $G$ on $H^0(X,\Omega_X)$, and also on $H^0(X,\Omega_X^{\otimes m})$, is faithful.
The following is our main result:

  \begin{unnumthm}{\bf 1}\label{maintheorem}
    Suppose that $g_X\geq 2$ and let $m\geq1$. 
    Then $G$ does not act faithfully on $H^0(X,\Omega_X^{\otimes m})$ if and only if $G$ contains a hyperelliptic involution and one of the following two sets of conditions holds:
      \begin{itemize}
	\item $m=1$ and $p=2$;
	\item $m=2$ and $g_X=2$.
      \end{itemize}
  \end{unnumthm}

  The format of the report is now briefly outlined.
  
  In the first section we prove the strong form of the Riemann-Hurwitz formula (Theorem \ref{hur}).
  The Riemann-Hurwitz formula relates the genus of two curves when there is a surjective map from one to the other, via the degree of the map and the degree of the ramification divisor.
  However, this can obscure the fact that the canonical divisors (see Section \ref{Hurwitzsection}) themselves are related.
  The strong form of the theorem states that given two curves and a surjective map $\pi:X\rightarrow Y$ of degree $n$ between the curves, with ramification divisor $R$ (see Section \ref{Hurwitzsection}), we have
  \[
 \di (\pi^* (\omega)) = \pi^*(\di (\omega)) + R,
  \]
where $\omega$ is a non-zero differential on $Y$, and $\pi^*$ is the pull-back induced by $\pi$.
This section closely follows Stichtenoth's book, see \cite{stichtenoth}.

The second section looks at computing the dimension of various spaces, but the most significant is the dimension of the subspace of $H^0(X,\Omega_X^{\otimes m})$ fixed by $G$, where $m\geq 1$.
This result, along with two other results in section three, forms the heart of the proof of the main theorem.
The dimension itself is dependent, essentially, on the genus of the quotient curve, $Y=X/G$, the degree of the projection map $\pi:X\rightarrow Y$, the ramification divisor $R$ of $\pi$ and $m$.
By using the Riemann-Roch theorem we can easily compute the dimension of $H^0(X,\Omega_X^{\otimes m})$, and the comparison of these two dimensions is what will be used in the third section.


In the third section we consider when a group of prime power order acts trivially on $H^0(X,\Omega_X^{\otimes m})$.
By only considering cyclic groups of prime power order our computations are made considerably easier.
Initially we only consider groups of prime order.
In this case, if the characteristic of $k$ is different to $p$ then the projection map is tamely ramified, and hence we know that the coefficients of the ramification divisor are $p-1$.
This makes it considerably easier to compute the dimension of the fixed space, as of all the parameters it depends on, the ramification divisor is the most difficult to deal with.
When the characteristic of the field and the order of the group are the same (\ie when wild ramification could occur), then the computations are longer, but still made considerably easier by our assumptions.
At the end of the third section we use results of \cite{kako} to make the same computations for groups whose orders are powers of $p$.
The results we use are somewhat more technical, but they do extend the original results.
We also make some computations for a general divisor $D$, which in general should have degree greater than $2g-2$, and give criteria for when the action is trivial on the associated Riemann-Roch space $H^0(X,\cO_X(D))$.

In the fourth section we prove the main result.
This builds on sections two and three, by reducing from a group which does not act faithfully on $H^0(X,\Omega_X^{\otimes m})$, to a subgroup that acts trivially.
We consider the cases where $m=1$ and $m\geq 2$ separately; despite similar methods being employed, there are technical details that need to be changed according to which case is being considered.
These technical details show clearly in the statement: if $m=1$ then the characteristic of $k$ must be 2 for the action to not be faithful, but the genus is not relevant at all.
In contrast, if we consider $m\geq 2$, the characteristic now does not matter, but the genus must be 2 (as does $m$).

In the final section we consider examples to illuminate what has been done in the previous sections.
We start by considering the rather trivial cases where $g_X=0$ and $g_X= 1$.
We give explicit proofs of when the action is faithful, as these cases where not included in the proof of the main theorem.
We then go on to construct a basis for the space of holomorphic poly-differentials for any given hyperelliptic curve.
This serves to explicitly prove the main theorem for this class of curves, and helps to enlighten the reader by way of a concrete example.
In particular, it helps to show why we have a different type of result according to whether $m=1$ or $m\geq 2$.

alignchapter{The Hurwitz formula} \label{Chapter:hurwitzformula}
\section{The Riemann-Hurwitz formula}\label{Hurwitzsection}

Let $X$ and $Y$ be two projective non-singular curves over an algebraically closed field $k$, with a map $\pi \colon X \rightarrow Y$ of degree $n$.
We let $g_X$ and $g_Y$ be the genera of $X$ and $Y$ respectively.
Let $K(X)$ and $K(Y)$ be the function fields of $X$ and $Y$ respectively.
Note that $K(X)$ is a degree $n$ extension of $K(Y)$, and recall that $\pi$ induces a map $\pi^*\colon K(Y) \rightarrow K(X)$.
As usual, for any $P \in X$ and $Q \in Y$ we denote the local ring formed by functions which are regular at $P$ and $Q$ by $\cO_{X,P}$ and $\cO_{X,Q}$ respectively.

We recall some facts from Galois theory.
Any element $\alpha \in K(X)$ defines a $K(Y)$~-~linear map $\mu_{\alpha} \colon  K(X) \rightarrow K(X)$, given by multiplication by $\alpha$.
We let $\Tr_{X/Y}(\alpha)$ be the trace of the matrix corresponding to $\mu_\alpha$.
We call $\Tr_{X/Y}(\alpha)$ the trace of $\alpha$, and $\Tr_{X/Y}$ is a map $K(X) \ra K(Y)$.
Note that $\Tr_{X/Y}$ is an additive map, and that for $\alpha \in K(Y)$ we have $\Tr_{X/Y}(\alpha) = n \alpha$.
Also, since our extension is Galois (and hence, in particular, separable) then the trace map is non-zero (see, for example, \cite[Appendix A]{stichtenoth}).\todo{more specific citation}
Finally, if $\alpha \in K(X)$ has minimum polynomial 
\[
 f(X) = X^r + a_{r-1}X^{r-1} + \ldots +a_0 \in K(Y)[X]
\]
 and $s= [K(X):K(Y)(\alpha)]$, then $\Tr_{X/Y}(\alpha) = -sa_{r-1}$ as in \cite[Appendix A]{stichtenoth}.\todo{more specific citation}

Given a basis $\{z_1,\ldots,z_n\}$ of $K(X)$ over $K(Y)$, we now introduce its dual basis with respect to the trace.
We denote the dual space of $K(X)$ over $K(Y)$ by \[K(X)^*:=\{\lambda \colon K(X) \rightarrow K(Y)| \lambda\ \text{is}\ K(Y)\text{-linear}\}.\]
We make $K(X)^*$ in to a one-dimensional $K(X)$-vector space by defining $z \lambda(w):=\lambda(z w)$.
As $\Tr_{X/Y}$ is non-zero, then there exists a unique $z\in K(X)$ for every $\lambda \in K(X)^*$ such that $\lambda = z\cdot \Tr_{X/Y}$.
In particular, if we choose $\lambda_j\in K(X)^*$ such that $\lambda_j(z_i) = \delta_{ij}$ (the Kronecker symbol), then there exist $z_j^*$ such that $\lambda_j = z_j^*\cdot \Tr_{X/Y}$.
Hence
\[
 \Tr_{X/Y}(z_iz_j^*) = (z_j^*\cdot \Tr_{X/Y})(z_i) = \lambda_j(z_i) = \delta_{ij}.
\]
The elements $z_1^*, \ldots , z_n^*$ form the {\em dual basis} of $z_1, \ldots , z_n$.

We now briefly introduce differentials in terms of function fields.
Let $C$ be any projective, non-singular curve over $k$, with function field $K(C)$, and with genus $g_C$.
Then an {\em \adele on $C$} is a map $\alpha\colon  C \rightarrow K(C)$ such $\alpha(P) \in \cO_{X,P}$ for nearly all $P \in C$ (\ie all but a finite number of points).
We can view $\alpha$ as an element of $\prod_{P\in C}K(C)$, and we write $\alpha_P := \alpha(P)$.
We let the {\em \adele space} of $C$, denoted by $\cA_C$, be the space of all \adeles on $C$.
There is a canonical injection $K(C) \hookrightarrow  \cA_C$, defined by sending $x\in K(C)$ to the \adele $\alpha$ for which $\alpha_P =  x$ at each point $P\in C$.
The elements of $K(C)$ have natural valuations; for each $P\in C$ we choose a uniformising parameter $t\in \cO_{X,P}$, and we then define $v_P(x):=n$, where $n$ is the unique integer such that $x=ut^n$ for some unit $u\in \cO_{X,P}$.
We can we can then extend this valuation to elements of $\cA_C$ by defining $v_P(\alpha) := v_P(\alpha_P)$.

Recall that a divisor on $C$ is a finitely supported formal sum over $C$, with coefficients in $\mathbb Z$, which we write as 
\[
 D = \sum_{P\in C} n_P [P],
\]
and that the degree of such a divisor $D$ is 
\[
 \deg(D) := \sum_{P\in X} n_P.
\]
We then define the order of $D$ at $P$ to be $v_P(D) := n_P$.

We let $\cD_C$ be the space of divisors of $C$.
Note that for any $x\in K(C)$ we have a naturally associated divisor, which is
\[
 \di(x) := \sum_{P\in C} v_P(x) [P].
\]

For any $D\in \cD_C$ we define $v_P(D)$ to be the coefficient of $[P]$ in $D$, and we let
\[
 \cA_C(D):=\{\alpha \in \cA_C | v_P(\alpha) \geq -v_P(D)\ \text{for all} \ P\in C\}
\]
be the {\em \adele space associated to $D$}.
Then we define a {\em differential} on $C$ to be a $k$-linear map $\omega\colon  \cA_C \rightarrow k$ such that $\omega$ is zero on $\cA_C(D) +K(C)$ for some divisor $D\in \cD_C$.
Note that via the canonical embedding of $K(C)$ into $\cA_C$, we can view $\cA_C(D) +K(C)$ as a subspace of $\cA_C$.
We will denote the space of differentials on $C$ by $\Omega_{K(C)}$.

Now we define 
\[
 M(\omega) := \{ D\in \cD_C | \omega \ \text{is zero on }\ \cA_C(D) + K(C)\}
\]
for any non-zero $\omega \in \Omega_{K(C)}$, and state the following lemma.

\begin{defnlem}
 For any non-zero differential $\omega$ on $C$ there is a unique maximal divisor $W$ in $M(\omega)$, which we call the {\rm canonical divisor associated to $\omega$} (we associate the zero divisor to $0\in \Omega_{K(C)}$).
 We denote the divisor associated to $\omega$ by $\di (\omega)$.
 We call a divisor a {\rm canonical divisor} if it is the canonical divisor associated to some $\omega \in \Omega_{K(C)}$.



 Given any two non-zero canonical divisors, $W$ and $W'$, there exists an $x\in K(C)$ such that $W = W' + \di(x)$.
 Conversely, if $W$ is a canonical divisor, and $x\in K(C)$, then $W + \di(x)$ is a canonical divisor.
\end{defnlem}
\begin{proof}
 See \cite[Lem. 1.5.10 and Prop. 1.5.13]{stichtenoth}.\todo{check citations}
\end{proof}

\todo[inline]{Introduce the idea of "the" canonical divisor $K_X$ here}
We can now define valuations on the differentials; namely, given some differential $\omega $ on $X$, and its associated divisor $W$, we set $v_P(\omega):=v_P(W)$ for $P\in X$.




\begin{prop}
 The space of differentials, $\Omega_{K(C)}$, is a one-dimensional vector space over $K(C)$.
\end{prop}
\begin{proof}
 See \cite[Prop. 1.5.9]{stichtenoth}.\todo{check citation}
\end{proof}



It should be noted that in algebraic geometry differentials are normally defined in a different, but equivalent, manner.
Given a ring $K(X)$ and an $K(X)$-module $M$, we call any $K(X)$-linear map $D\colon K(X)\rightarrow M$ satisfying \[D(ab) = aD(b) + D(a)b\] for all $a,b\in K(X)$ a {\em derivation}.
There is a unique module, denoted $\Omega_{K(X)}$, with a map $d\colon K(X) \rightarrow \Omega_{K(X)}$, which every derivation must factor through; \ie if $D\colon K(X)\rightarrow M$ is a derivation then there is a unique map $f\colon \Omega_{K(X)}\rightarrow M$ such that $D = f\circ d$.\todo{change $f$}
Given any differential $\omega$, we can write it as $fdx$ for some $x$ and some $f$ in $K(X)$.
This is how we will consider differentials in the other sections of this report.
At the end of this section we will describe an isomorphism between $\Omega_C$ and $\Omega_{K(C)}$.

Finally, we define a particular class of differentials that will play a crucial role in what follows.
A differential $\omega \in \Omega_{K(X)}$ is called {\em holomorphic} if its associated divisor, $W$, is non-negative (\ie if $v_P(W) \geq 0$ for all $P\in X$).
We denote {\em the space of holomorphic differentials} by $H^0(X,\Omega_{K(X)})$.\todo{Change notation - $\Omega_{K(X)}$ is not a sheaf}
Also, for any divisor $D$ on $X$ we define 
\[
H^0(X,\cO(D)) := \{x\in K(X) | v_P(x) \geq -v_P(D)\}\cup \{0\}.
\]\todo{change this notation to Fulton's or Stichtenoth's}
This is sheaf theoretic notation, which would not normally be used in function field theory, but we use it for consistency with the rest of the report.\todo{change this sentence after changing notation}
We can now state the celebrated Riemann-Roch theorem.

\begin{thm}[Riemann-Roch]\label{riemannroch}
Let $W$ be a canonical divisor on $C$.
Then for any divisor $D$ on $C$ 
\[
 \dim H^0(C,\cO(D)) = \deg(D) + 1 - g_C + \dim H^0(C,\cO(W-D)).
\]
\end{thm}
\begin{proof}
 See, for example, \cite[8.6]{fulton} or \cite[Thm. 1.5.15]{stichtenoth}.\todo{check citations}
\end{proof}


\begin{cor}\label{dim=gc}
 For any canonical divisor $W$ on $C$, we have 
 \[
  \deg(W) = 2g_C-2
 \]
and \[
     \dim H^0(C,\cO(W)) = g_C.
    \]
\end{cor}
\begin{proof}
Since $\dim H^0(C,\cO(0)) = 1$ (recall that the only functions with no poles are the constant functions), we have by Riemann-Roch, $ 1= \dim H^0(C,\cO(0)) = 0 + 1 -g_C + \dim H^0(C,\cO(W))$.
 Rearranging this gives the second statement.
 The first statement then follows by rearranging
\[
 g_C = \dim H^0(C,\cO(W)) = \deg(W) + 1 -g_C +  \dim H^0(C,\cO(W-W))= \deg(W) + 1 -g_C + 1.
\]
\end{proof}


There is an alternative way to map $K(C)$ in to $\cA_C$, which we will make use of shortly.
Given a point $P\in C$ we define $\iota_P\colon K(C) \rightarrow \cA_C$ by
\begin{equation}
 (\iota_P(x))_Q:= \begin{cases}
           x & \text{if }\ P=Q\\
           0 & \text{otherwise}.
           \end{cases}
\end{equation}
for $Q\in X$.
For $\omega \in \Omega_{K(X)}$ we then define $\omega_P\colon K(X) \rightarrow k$ to be the map $\omega_P(x) := \omega(\iota_P(x))$.
This is called the {\em local component} of $\omega$; we will use these definitions to prove the following proposition.\todo{change this sentence}

\begin{prop}\label{propertyofomega}
 Let $\omega \neq 0$ be a differential on $X$ and let $P\in X$. Then
 \[
  v_P(\omega) = \max \{r\in \mathbb{Z}|\omega_P(x) = 0\ \text{for all} \ x\in K(X) \ \text{with}\ v_P(x) \geq -r\}.
 \]
In particular $\omega_P \neq 0$.\todo{make clear that the maximum exists and hence that this is well defined}
\end{prop}
\begin{proof}
 Let $W$ be the divisor associated to $\omega$.
 Let $s:=v_P(\omega)$ be the order of $\omega$ at $P$.
 If $x\in K(X)$ and $v_P(x)\geq -s$, then $\iota_P(x) \in \cA_X(W)$, by definition.
 Hence $\omega_P(x) = 0$.
 On the other hand, suppose $\omega_P(x) = 0$ for any $x\in K(X)$ satisfying that $v_P(x) \geq -s-1$.
 Let $\alpha\in \cA_X(W+[P])$.
 Then we have
 \[
  \alpha = (\alpha-\iota_P(\alpha_P)) + \iota_P(\alpha_P).
 \]
Note that $\alpha - \iota_P(\alpha_P)\in \cA_X(W)$ and $v_P(\alpha_P) \geq -s-1$.
Hence
\[
 \omega(\alpha) = \omega(\alpha-\iota_P(\alpha_P))  + \omega_P(\alpha_P) = 0,
\]
and so $\omega$ is zero on $\cA_X(W+P)$.
But this contradicts the maximality of $W$ in its definition.
 
\end{proof}



We now give a small result regarding field extensions, after recalling some terminology.
Recall that if $S$ is a subring of $R$, then an element $x\in R$ is integral over $S$ if there is a monic polynomial with coefficients in $S$ for which $x$ is a solution.
Then an integral basis of $R$ over $S$ is a basis of $R$ over $S$, for which each basis elements is integral.
 Recall that $\cO_{X,Q}$ is an integrally closed subring of $K(Y)$ for any $Q\in Y$, and that $K(Y)$ is the field of fractions of $\cO_{X,Q}$.


\begin{prop}
 For $z\in K(X)$ we let $\phi(T)\in K(Y)[T]$ be its minimal monic polynomial over $K(Y)$.
 Then $z$ is integral over $\cO_{X,Q}$ if and only $\phi (T)\in \cO_{X,Q}[T]$.
\end{prop}
\begin{proof}
 By definition $\phi(T)$ is the monic irreducible polynomial in $K(Y)[T]$ such that $\phi(z) = 0$. 
 Hence if $\phi (T)$ has coefficients in $\cO_{X,Q}$, then $z$ is integral over $\cO_{X,Q}$ by definition.
 
 We now suppose that $z\in K(X)$ is integral over $\cO_{X,Q}$.
 Then we can choose some monic polynomial $f(T)\in \cO_{X,Q}[T]$ such that $f(z) = 0$.
 Since $\phi(T)$ is minimal over $K(Y)$, then there exists some $\psi(T)\in K(Y)[T]$ such that $f(T) = \phi(T)\cdot \psi(T)$.
 Let $F \supseteq K(X)$ be a finite extension of $K(X)$ containing all the roots of $\phi$, and let $R$ be the integral closure of $\cO_{X,Q}$ in $F$.
 Now the roots of $\phi(T)$ are also roots of $f(T)$, and hence are in $R$.
 Then the coefficients of $\phi(T)$ can be written as polynomials of the roots of $\phi(T)$, and as we just noted that these roots are in $R$ then $\phi(T) \in R[T]$.
 But then $\phi(T)\in K(Y)[T]$, and since $R$ is integrally closed it follows that $K(Y)\cap R = \cO_{X,Q}$, hence $\phi(T)\in \cO_{X,Q}[T]$.
\end{proof}

\begin{cor}\label{traceinclosure}
Let $Q$ be a point in $Y$ and let $x\in K(X)$ be integral over $\cO_{X,Q}$ under $\pi$.
 Then $\Tr_{X/Y}(x)\in \cO_{X,Q}$.
\end{cor}
\begin{proof}
 As noted earlier, if $\phi(T)=T^r+a_{r-1}T^{r-1} + \ldots + a_0\in K(Y)[T]$ is the minimal polynomial of $x$ over $K(Y)$, then $\Tr_{X/Y}(x)=-n_xa_{r-1}$, where $n_x : = [K(X):K(Y)(x)]$.
 Hence the corollary follows from the previous proposition.
\end{proof}


We wish to define the ramification divisor, as this is essential for the statement of the Riemann-Hurwitz formula.
In order to do this we first define the complementary module and the ramification index.


\begin{defn}
 Consider a point $P\in X$ in the pre-image of $Q\in Y$ under $\pi$.
 By \cite[Prop. 3.1.4]{stichtenoth}\todo{check citations} there is an integer $e_P$ such that for any $x\in K(X)$ the equality $v_P(x) = e_P\cdot v_Q(x)$ holds.
 This value is called the ramification index of $P$.
 If $e_P>1$ then we say that $\pi$ is ramified at $P$.
\end{defn}

Given this we can associate to each $Q\in Y$ the divisor
\[
 \pi^*([Q]) := \sum_{P\mapsto Q} e_P [P].
\]
This can then be extended from a single point to a divisor on $Y$, in which case for a divisor $D = \sum_{Q\in Y}n_Q [Q]$ we have
\[
 \pi^*(D) := \sum_{Q\in Y}n_Q \pi^*([Q]).
\]

\begin{rem}
It should be noted that in the literature regarding function field theory, what we have denoted by $\pi^*(D)$ is normally called the conorm of $D$ and is denoted ${\rm Con}_{X/Y}(D)$. 
We used the notation above, from algebraic geometry, to be consistent with the rest of this report.
\end{rem}

\begin{defn}
 For any $Q\in Y$, let $\cO_{X,Q}'$ be the integral closure of $\cO_{X,Q}$ in $K(X)$.\todo{define this in text, remove instances in this and next lemma}
 We then define the complementary module over $\cO_{X,Q}$ to be
 \[
  C_Q :=\{z\in K(X) | \Tr_{X/Y}(z\cdot \cO_{X,Q}') \subseteq \cO_{X,Q}\}.
 \]\todo{is it okay to say $\Tr_{X/Y}(z\cdot \cO_{X,Q}')$?}
\end{defn}


We will list some properties of the complementary module in a proposition, but first we require the following lemmas.

\begin{lem}[Approximation lemma]\label{approximationlemma}
Let $m$ be a positive integer. 
For each $i\in \{1,\ldots, m\}$ let $P_i\in X$, let $\mathcal{P}_i$ be the corresponding maximal ideal of $\cO_{P_i}$, let $x_i$ be an element of $K(X)$ and let $n_i$ be an integer.\todo{fix line break}
Then there is an $x\in K(X)$ such that $v_{P_i}(x-x_i) \geq n_i$ for all $i$.
\end{lem}

\begin{rem}
This result can be strengthened to also say that $v_P(x) \geq 0$ for any $P\notin \{P_1,\ldots ,P_m\}$.\todo{fix line break}
We will not prove this here, for the sake of brevity, but the proof can be found in \cite[Chap. 1, \S 3, pg. 12]{localfields}.
\end{rem}
\begin{proof}
We let $R:= \cO_{P_i} \cap \ldots \cap \cO_{P_m}$, and we let $\mathcal{P}_i' := \mathcal{P}_i \cap R$.
We first prove the lemma assuming that $x_i \in R$.
 We may increase the $n_i$ such that $n_i\geq 0$ for all $i$.
 By linearity we may assume that $x_2 = \ldots = x_m = 0$, since if we find an element for $x_1$ in this instance, we can similarly find an element for each $i$ and add them.
Let $I = {\mathcal{P}_1'}^{n_1} + {\mathcal{P}_2'}^{n_2}\cdots {\mathcal{P}_n'}^{n_m}$.
This is an ideal of $R$, and since it has elements whose valuation at any $P_i$ is zero, it is in fact equal to $R$.
Hence we can write $x_1 = x + y$, where $y \in {\mathcal{P}_1'}^{n_1}$ and $x\in {\mathcal{P}_2'}^{n_2}\cdots {\mathcal{P}_n'}^{n_m}$.
Since ${\mathcal{P}_1'}^{n_1} \subseteq {\mathcal{P}_1}^{n_1}$ and ${\mathcal{P}_2'}^{n_2}\cdots {\mathcal{P}_n'}^{n_m} \subseteq \mathcal{P}_2^{n_2}\cdots \mathcal{P}_n^{n_m}$, the $x$ above is as described in the lemma, and this finishes the proof in the case $x_i \in R$.

 
 In general one can write $x_i = \frac{a_i}{b}$ for $a_i\in R$ and $b\in R\backslash \{0\}$, and $x$ can be represented as $\frac{a}{s}$.
 Then we require that $v_{P_1}(a-a_i) \geq n_i + v_{P_i}(s)$ for all $i$ and that $v_P(a)_ \geq v_P(s)$ for all $P\notin \{P_1,\ldots ,P_m\}$.
 But after adding the points $P$ for which $v_P(s)$ is negative, this is precisely what we described in the statement of the lemma.
 \end{proof}

\begin{lem}\label{pidlemma}
 Let $Q\in Y$, and let $\cO_{X,Q}'$ be the integral closure of $\cO_{X,Q}$ in $K(X)$.
 Then $\cO_{X,Q}'$ is a principal ideal domain.
\end{lem}
\begin{proof}
By \cite[Cor. 3.3.5]{stichtenoth}\todo{check citation}, we have $\cO_{X,Q}' = \{x\in K(X)|v_P(x) \geq 0 \ \text{for all} \ P\in \pi^{-1}(Q)\}$.
 Let $I$ be an ideal of $\cO_{X,Q}'$.
 If we let $\{P_1,\ldots, P_l\} = \pi^{-1}(Q)$ then we can choose $x_i$ for $1\leq i \leq l$ such that $v_{P_i}(x_i) \leq v_{P_i}(y)$ for all $y\in I$.
 By the Approximation Lemma there exist $z_i$ such that $v_P (z_i) = 0$ if $P=P_i$ and $v_{P_j}(z_i) > v_{P_j}(z_j)$ for $j\neq i$.
 Now let $x = \sum_{i=1}^l x_iz_i \in I$.
 Clearly $v_{P_i}(x) = v_{P_i}(x_i)$ for all $1\leq i\leq l$.



 Now we show that $I \subseteq x\cO_{X,Q}'$.
 If $y\in I$ then we let $z = x^{-1}y$.
 Then $v_{P_i}(z) = v_{P_i}(y) - v_{P_i}(x_i) \geq 0$ for all $1\leq i\leq l$.
Hence $z\in \cO_{X,Q}'$ and so $y = xz \in x\cO_{X,Q}'$, completing the proof.
\end{proof}


\begin{prop}\label{factsaboutc'}
Fix $Q\in Y$. 
 \begin{enumerate}
  \item $C_Q$ is an $\cO_{X,Q}'$-module, and $\cO_{X,Q}' \subseteq C_Q$.
  \item If $\{z_1,\ldots ,z_n\}$ is a (necessarily integral) basis of $\cO_{X,Q}'$ over $\cO_{X,Q}$, then 
  \[
   C_Q = \sum_{i=1}^n \cO_{X,Q}\cdot z_i^*,
  \]
where $z_i^*$ is the dual basis of $z_i$ for $1\leq i \leq n$.\todo{reference dual basis defined earlier}
\item There is a $t\in K(X)$ such that $C_Q = t\cdot \cO_{X,Q}'$ (note that $t$ depends on the choice of $Q$).
  Moreover, $v_P(t) \leq 0$ for all $P\in \pi^{-1}(Q)$, and if $t'\in K(X)$ then $C_Q=t'\cdot \cO_{X,Q}'$ if and only if $v_P(t) = v_P(t')$ for all $P\in \pi^{-1}(Q)$.

 \end{enumerate}
\end{prop}
\begin{proof}
 \begin{enumerate}
  \item It is clear that $C_Q$ is an $\cO_{X,Q}'$ module.
	Indeed, if $y\in \cO_{X,Q}'$, then $y\cdot \cO_{X,Q}' \subseteq \cO_{X,Q}'$ and hence $\Tr_{X/Y}(z y\cdot \cO_{X,Q}')\subseteq \Tr_{X/Y}(z\cdot \cO_{X,Q}') \subseteq \cO_{X,Q}$ for any $z\in C_Q$.
	The fact that $\cO_{X,Q}'$ is contained in $C_Q$ follows from Corollary \ref{traceinclosure}.
  \item We first show that $C_Q \subseteq \sum_{i=1}^n \cO_{X,Q}\cdot z_i^*$.
	Suppose $z\in C_Q$.
	Now $\{z_1^*, \ldots ,z_n^*\}$ is a basis of $K(X)$ over $K(Y)$, so there exist $x_1,\ldots , x_n\in K(Y)$ such that $z=\sum_{i=1}^n x_iz_i^*$.
	As $z\in C_Q$ and $z_1,\ldots ,z_n\in \cO_{X,Q}'$, it follows by definition of $C_Q$ that $\Tr_{X/Y}(zz_j)\in \cO_{X,Q}$ for $j\in \{1,\ldots ,n\}$.
	We know that 
	\[
	 \Tr_{X/Y}(zz_j) = \Tr_{X/Y}\left(\sum_{i=1}^nx_iz_iz_j^*\right) = \sum_{i=1}^nx_i \cdot \Tr_{X/Y}(z_iz_j^*) = x_j,
	\]
	since $z_j^*$ is dual to $z_j$.
	Hence each $x_j$ is in $\cO_{X,Q}$, and $z\in \sum_{i=1}^n\cO_{X,Q}\cdot z_i^*$.
	
	Now suppose that $z\in \sum_{i=1}^n\cO_{X,Q}\cdot z_i^*$ and $u\in \cO_{X,Q}'$.
	Then we need to show that $\Tr_{X/Y}(z u)\in \cO_{X,Q}$.
	We can find $x_i, y_j\in \cO_{X,Q}$ such that $z=\sum_{i=1}^n x_iz_i^*$ and $u=\sum_{i=1}^ny_jz_j$.
	Then
	\[
	 \Tr_{X/Y}(zu) = \Tr_{X/Y}\left(\sum_{i,j=1}^n \left(x_iy_jz_i^*z_j\right)\right) = \sum_{i,j=1}^n x_iy_j\cdot \Tr_{X/Y}(z_i^*z_j) = \sum_{i=1}^n x_iy_i.
	\]
	Since $\sum_{i=1}^n x_iy_i\in \cO_{X,Q}$, it follows that $z\in C_Q$.
  \item By the previous part, we can find $u_i\in K(X)$ such that $C_Q = \sum_{i=1}^n \cO_{X,Q} \cdot u_i$.
	Choose some $x\in K(Y)$ such that $v_Q(x)\geq 0$ and also $v_Q(x)\geq -v_P(u_i)$ for all $P\in \pi^{-1}(Q)$ and $i\in \{1,\ldots ,n\}$.
	By definition of $e_P$ it follows that
	\[ v_P(xu_i) = e_Pv_Q(x) + v_P(u_i) \geq 0\]
	for all $i\in \{1,\ldots, n\}$ and all $P\in \pi^{-1}(Q)$.
	Since $\cO'_Q = \cap_{P\mapsto Q} \cO_{X,P}$ (see, for example, \cite[Cor. 3.3.5]{stichtenoth}), then $x\cdot C_Q \subseteq \cO_{X,Q}'$.
	Since $\cO_{X,Q}'$ is a principal ideal domain by Lemma \ref{pidlemma}, it follows that $x\cdot C_Q = y\cdot \cO_{X,Q}'$ for some $y\in \cO_{X,Q}'$.
	If we let $t=x^{-1}y$ then $C_Q =t\cdot \cO_{X,Q}'$, proving the first part of the statement.



	Since $\cO_{X,Q}'\subseteq C_Q$ then $v_P(t)\leq 0$ for all $P\in \pi^{-1}(Q)$.
	Now $t\cdot \cO_{X,Q}' = t'\cdot \cO_{X,Q}'$ if and only if both $tt'^{-1}$ and $t^{-1}t'$ are in $\cO_{X,Q}'$.
	But this is the case if and only $v_P(tt'^{-1}) \geq 0$ and $v_P(t^{-1}t')\geq 0$ for all $P\in \pi^{-1}(Q)$, which is equivalent to $v_P(t)=v_P(t')$ for all such $P$.
 \end{enumerate}
\end{proof}

\begin{prop}\label{almostallqiny}
 For almost all $Q\in Y$ we have $C_Q= \cO_{X,Q}'$.
\end{prop}
\begin{proof}
 We first show that, given a basis of $K(X)$ over $K(Y)$, it is almost always integral over $\cO_{X,Q}$ given $Q\in Y$.
 Recall that $\cO_{X,Q}$ is integrally closed in $K(Y)$, and its quotient field is $K(Y)$.
 As before, we denote by $\cO_{X,Q}'$ the integral closure of $\cO_{X,Q}$ in $K(X)$, and we consider a basis $\{z_1,\ldots ,z_n\}$ of $K(X)$ over $K(Y)$.
 Let $\{z_1^*,\ldots ,z_n^*\}$ be the dual basis.
 Now the minimal polynomials of $z_1,\ldots, z_n,z_1^*,\ldots z_n^*$ have finitely many coefficients in $K(Y)$.
 Hence if $S\subseteq Y$ is the set of poles of these coefficients, then $S$ is finite and for $Q\notin S$ then we have
 \[
  z_1,\ldots,z_n,z_1^*,\ldots, z_n^*\in \cO_{X,Q}'.
 \]

 
 
Now we assume that $\{z_1,\ldots ,z_n,z_1^*,\ldots z_n^*\}\subseteq \cO_{X,Q}'$ and then we show that 
\[
  \cO_{X,Q}' \subseteq \sum_{i=1}^n \cO_{X,Q}\cdot z_i^*.
\]

 If $z\in K(X)$ then there are $e_1,\ldots, e_n\in K(Y)$ such that $z=e_1z_1^*+\ldots +e_nz_n^*$.
 If $z\in \cO_{X,Q}'$ then $zz_j\in \cO_{X,Q}'$ for $1\leq j\leq n$, and hence $\Tr_{X/Y}(zz_j)\in \cO_{X,Q}$, by Corollary \ref{traceinclosure}.
 As
 \[
  \Tr_{X/Y}(zz_j) = \Tr_{X/Y}\left(\sum_{i=1}^n e_iz_jz_i^*\right) = \sum_{i=1}^ne_i\cdot \Tr_{X/Y}(z_jz_i^*) = e_j.
 \]
  Hence $e_j\in \cO_{X,Q}$, and $\cO_{X,Q}'\subseteq \sum_{i=1}^n\cO_{X,Q}\cdot z_i^*$.
  Since $\{z_1,\ldots, z_n\}$ also forms a basis we can run the same argument again to show that $\cO_{X,Q}'\subseteq \sum_{i=1}^n\cO_{X,Q}\cdot z_i$.
  
  We then have the following set of inequalities:
  \[
   \sum_{i=1}^n \cO_{X,Q}\cdot z_i \subseteq \cO_{X,Q}' \subseteq \sum_{i=1}^n\cO_{X,Q}\cdot z_i^* \subseteq \cO_{X,Q}'\subseteq \sum_{i=1}^n \cO_{X,Q}\cdot z_i.
  \]
Since $C_Q = \sum_{i=1}^n \cO_{X,Q}\cdot z_i^*$ by part 2 of Proposition \ref{factsaboutc'}, the result follows.
\end{proof}

We can now define the ramification divisor.


\begin{defn}
 For a point $Q\in X$ choose $t\in K(X)$ such that $C_Q = t\cdot \cO_{X,Q}'$, as in part 3 of Proposition \ref{factsaboutc'}.
 Then for any $P\in \pi^{-1}(Q)$ we define the different exponent to be $\delta_P := -v_P(t)$.
 By Proposition \ref{factsaboutc'} this is well defined, and by Proposition \ref{almostallqiny} it is almost always zero.
 Hence we can define the ramification divisor to be
 \[
  R := \sum_{P\in X} \delta_P [P].
 \]
\end{defn}

\begin{rem}
It should be noted that when considering the theory of function fields the ramification divisor is called the different, and denoted ${\rm Diff}_{X/Y}$.
\end{rem}

We now define the \adele space of $X$ over $Y$.

\begin{defn}
 We define $\mathcal {A}_{X/Y}$ as
 \[
  \mathcal{A}_{X/Y} := \{\alpha \in \mathcal{A}_X | \alpha_P =\alpha_{P'}\ \text{if}\ \cO_{X,P}\cap K(Y) = \cO_{P'}\cap K(Y)\}.
 \]
We extend the trace function $\Tr_{X/Y}\colon K(X)\rightarrow K(Y)$ to a map $\Tr_{X/Y}\colon \mathcal{A}_{X/Y} \rightarrow \mathcal{A}_Y$,  by letting
\[
 (\Tr_{X/Y}(\alpha))_Q := \Tr_{X/Y}(\alpha_P)
\]
for any $\alpha \in \mathcal{A}_{X/Y}$, $Q\in Y$ and $P\in \pi^{-1}(Q)$.

For any divisor $D$ over $X$ we define
\[
 \mathcal{A}_{X/Y}(D) := \mathcal{A}_X(D) \cap \mathcal{A}_{X/Y}.
\]

\end{defn}


\begin{defnthm}\label{detailedhurwitz}
 For every differential $\omega\in H^0(Y,\Omega_Y)$ there is a unique $\omega'\in H^0(X,\Omega_X)$ such that
 \[
  \omega'(\alpha) = \omega\left(\Tr_{X/Y}(\alpha)\right)
 \]
for all $\alpha \in {\mathcal A}_{X/Y}$.

This differential is called the {\rm pullback} of $\omega$, and is denoted $\pi^*(\omega)$. 
If $\omega\neq 0$ and $(\omega)$ is the associated divisor, then 
\[
 \di ( \pi^*(\omega)) = \pi^*(\di(\omega)) + R.
\]
\end{defnthm}
\begin{comment}
\begin{rem}
 In the language of algebraic geometry, where we would write differentials as $dx$ for some $x\in K(X)$, then this would be phrased differently.
 When we use the result in later sections, it will be written as
 \[
  \di (\pi^* (dx)) = \pi^*(\di (dx)) + R
 \]
for some holomorphic differential $dx$.

Also, it should be noted that in theory of function fields, what we have denoted by $\pi^*(\omega)$ is called the cotrace of $\omega$, and denoted ${\rm Cotr}_{X/Y}(\omega)$.
\end{rem}
\end{comment}
We first note that this immediately implies the Riemann-Hurwitz formula.




\begin{cor}\label{hur}[Riemann-Hurwitz Formula]
 Given two non-singular projective curves $X$ and $Y$ of genera $g_X$ and $g_Y$ respectively, with a degree $n$ map $f\colon X \rightarrow Y$, then
 \[
  2g_X - 2 = n(2g_Y -2) + \deg(R),
 \]
where $R$ is the ramification divisor of $f$.
\end{cor}
\begin{proof}
 This follows from Theorem \ref{detailedhurwitz} and Corollary \ref{dim=gc}, after taking degrees.
\end{proof}


We first give a lemma that is necessary for the proof of the theorem.


\begin{lem}\label{adelespacelemma}
 For any divisor $D$ over $X$ we have $\mathcal {A}_X = \mathcal{A}_{X/Y} + \mathcal{A}_X(D)$.
\end{lem}
\begin{proof}
 Let $\alpha \in \mathcal{A}_X$. Then by the approximation lemma there is an element $x_Q\in K(X)$ for each $Q\in Y$ such that 
 \[
  v_P(\alpha_P - x_Q) \geq -v_P(D)
 \]
for each $P\in \pi^{-1}(Q)$. 
We then define the \adele $\beta$ such that $\beta_P := x_Q$ for every $P\in \pi^{-1}(Q)$.
Then $\beta \in \mathcal{A}_{X/Y}$ and the difference $\alpha - \beta$ is in $\mathcal{A}_X(D)$ by definition of $\beta$.
Hence $\alpha = \beta + (\alpha - \beta) \in \mathcal{A}_{X/Y} + \mathcal{A}_X(D)$.
\end{proof}
\begin{comment}
\begin{lem}
 Let $V$ be a vector space over $K(X)$ and let $\mu\colon V\rightarrow K(Y)$ be a $K(Y)$ linear map.
 Then thee is a unique $K(X)$ linear map $\mu'\colon V\rightarrow K(X)$ such that $\Tr_{X/Y}\circ \mu' = \mu$.
\end{lem}
\begin{proof}
 {\bf is prop III.3.3 needed?}
 Let $K(X)^:=\{ \lambda\colon K(X) \rightarrow K(Y)| \lambda L-\text{linear}\}$ be the space of linear forms, which forms a vector space over $K(X)$.
 It is one dimensional over $K(X)$ {\bf (check why)}, and so any $\lambda \in K(X)^*$ can be written as $z\cdot \Tr_{X/Y}$ for some $z\in K(X)$.
 
 If we fix a $v\in V$ then we can define an $L$-linear map $\lambda \colon K(X) \rightarrow K(Y)$ as $\lambda_v(a)\mapsto \mu(av)$.
 So there exists a unique $z_v\in K(X)$ such that $\lambda_v = z_v\cdot \mu$, and we define $\mu'(v) := z_v$.
 Hence
 \[
  \mu(av) = (\mu'(v)\cdot \Tr_{X/Y})(a) = \Tr_{X/Y}(a\cdot \mu'(diffv))
 \]
for every $a\in K(X)$ and $v\in V$.
Since the trace function and $\mu$ are both linear, it follows that $\mu'$ is also.
If we let $a=1$ then the equality $\mu = \Tr_{X/Y} \circ \mu'$, which proves existence.
If there were another such map, say $\mu^*$, then the difference $\mu'-\mu^*\colon V\rightarrow K(X)$ would be surjective, but $\Tr_{X/Y}\circ (\mu'-\mu^*) = 0$.
This implies that $\Tr_{X/Y}=0$, a contradiction, hence $\mu'$is unique.
\end{proof}
\end{comment}
We now prove Theorem \ref{detailedhurwitz}.

\begin{proof}
 We first construct a differential $\omega'\in H^0(X,\Omega_X)$ for every $\omega\in H^0(Y,\Omega_Y)$ such that $\omega'(\alpha) = \omega(\Tr_{X/Y}(\alpha))$ for every $\alpha \in \cA_{X/Y}$.
 If $\omega = 0$ then we can clearly let $\omega' = 0$, so we assume that $\omega \neq 0$.
 We will use the following divisor, $W' := \pi^*(\di (\omega)) + R$, throughout the proof.
 
 We first prove two assertions about $\omega_1:= \omega\circ \Tr_{X/Y} \colon \mathcal{A}_{X/Y} \rightarrow k$.
 Namely, we show that
 \begin{enumerate}[(i)]
  \item For any $\alpha \in \mathcal{A}_{X/Y}(W') + K(X)$ then $\omega_1(\alpha) = 0$;
  \item If $B'$ is a divisor on $X$ with $B' \nleq W'$ then there is a $\beta \in \mathcal{A}_{X/Y}(B')$ such that $\omega_1(\beta) \neq 0$.
 \end{enumerate}
To show (i) we start by noting that since $\Tr_{X/Y}$ and $\omega$ are $k$-linear, $\omega_1$ is too.
Also, note that since $\omega$ is zero on $K(Y)$ then $\omega_1$ is zero on $K(X)$.
To show that $\omega_1(\alpha) = 0$ for any $\alpha \in \mathcal{A}_{X/Y}(W')$ then it is sufficient to show that for any $Q\in Y$ and $P\in \pi^{-1}(Q)$ that
\[
 v_Q(\Tr_{X/Y}(\alpha_P)) \geq -v_Q(\omega).
\]

We choose $x\in K(Y)$ such that $v_Q(x) = v_Q(\omega)$.
Then
\begin{align}\label{remark}
 & v_P(x \alpha_P) = v_P(x) + v_P(\alpha_P) \geq e_P v_P(\omega) - v_P(W') \nonumber \\
 & = v_P(\pi^*((\omega)) - W') = -v_P(R) = -\delta_P. 
\end{align}
By the definition of $C_Q$ and the ramification divisor, it is clear that $z\in C_Q$ if and only if $v_P(z)\geq -\delta_P$ for all $P\in \pi^{-1}(Q)$.
It then follows from \eqref{remark} that $x \alpha_P$ is in $C_Q$ and hence that $v_Q(\Tr_{X/Y}(x \alpha_P)) \geq 0$, by definition of $C_Q$.
Since $\Tr_{X/Y}(x \alpha_P) = x\cdot \Tr_{X/Y}(\alpha_P)$ and $v_Q(x) = v_Q(\omega)$, we have shown the first claim.

To show (ii) we let $Q_0\in Y$ be a point such that there is a $P^*\in \pi^{-1}(Q_0)$ with $v_{P^*}(\pi^*((\omega)) - B') < -\delta_P$.
We know that such a $Q_0$ exists since $B' \nleq W'$.
As before we will denote by $\cO_{Q_0}'$ the integral closure of $\cO_{Q_0}$ in $K(X)$.
We let 
\[
 J := \{ z\in K(X) | v_{P^*}(z) \geq v_{P^*}(\pi^*(\di (\omega)) - B')\ \text{for all}\ P^*\in \pi^{-1}(Q_0)\}.
\]
By the approximation lemma, there exists a $u\in J$ such that 
\[
 v_{P^*}(u) = v_{P^*}(\pi^*(\di (\omega))-B')
 \]
 for all $P^*\in \pi^{-1}(Q_0)$, and hence $J\nsubseteq C_{Q_0}$.
(As noted earlier, $z\in \cO_{X,Q}$ if and only if $v_P(z) \geq -\delta_P$).
It is clear that $J\cdot \cO_{Q_0} \subseteq J$, and hence that $\Tr_{X/Y}(J) \nsubseteq \cO_{X,Q}$.
We let $t$ be an element of $K(Y)$ such that $v_Q(t) = 1$.
Then there is an $r\in \mathbb N$ such that 
$t^r\cdot J \subseteq \cO_{X,Q},
 $
 and then \[ t^r\cdot \Tr_{X/Y}(J) = \Tr_{X/Y}(t^r\cdot J) \subseteq \cO_{X,Q}.\]
It is clear that $t^r\cdot \Tr_{X/Y} (J)$ is an ideal of $\cO_{X,Q}$, and so $\Tr_{X/Y}(J) = t^s\cdot \cO_{X,Q}$ for some negative integer $s$.
Hence 
\begin{equation}\label{traceinring}
 t^{-1}\cdot \cO_{X,Q} \subseteq \Tr_{X/Y}(J).
\end{equation}
By Proposition \ref{propertyofomega} we can find an $x\in K(Y)$ such that $v_Q(x) = -v_Q(\omega) - 1$ and $\omega_Q(x) \neq 0$.
If we choose $y\in K(Y)$ such that $v_Q(y) = v_Q(\omega)$, then $xy \in t^{-1}\cO_{X,Q}$.
Hence by \eqref{traceinring} there is a $z\in J$ such that $\Tr_{X/Y} (z) = xy$.
Let $\beta \in \mathcal{A}_{X/Y}$ be chosen  such that 
\begin{equation*}
 \beta_P = \begin{cases}
            y^{-1}z & \text{if}\ P\in \pi^{-1}(Q) \\
            0 & \text{otherwise}.
           \end{cases}
\end{equation*}
Then for any $P\in \pi^{-1}(Q)$ we have
\begin{align*}
 v_P(\beta) & =  -v_P(y) + v_P(z) \\
 & \geq  -v_P(\pi^*(\di (\omega))) + v_P(\pi^*(\di (\omega)) - \beta) \\
 & =  -v_P(B'),
\end{align*}
with the inequality following from the definition of $y$ and $J$.
Hence $\beta \in \mathcal{A}_{X/Y}(B')$.
Now $\omega_1(\beta) = \omega(\Tr_{X/Y}(\beta)) = \omega_Q(x) \neq 0$.
This shows (ii).

By Lemma \ref{adelespacelemma} for each $\alpha \in \mathcal{A}_X$ there exists some $\beta \in \mathcal{A}_{X/Y}$ and $\gamma \in \mathcal{A}_X(W')$ such that $\alpha = \beta + \gamma$.
We now define a new differential $\omega_2 \colon  \mathcal{A}_X \rightarrow k$ by letting $\omega_2(\alpha) := \omega_1(\beta)$.
Suppose we have two representations of $\alpha$, say $\beta+ \gamma$ and $\beta' + \gamma'$, where $\beta, \beta' \in \mathcal{A}_{X/Y}$ and $\gamma, \gamma' \in \mathcal{A}_{X}(W')$.
Then 
\[
  \beta - \beta' = \gamma' - \gamma \in \mathcal{A}_{X/Y} \cap \mathcal{A}_X(W') = \mathcal{A}_{X/Y}(W').
\]
It then follows from (i) that 
\[
 \omega_1(\beta) - \omega_1(\beta') = \omega_1(\beta - \beta') = 0.
 \]
and hence $\omega_2$ is well defined.
It is also clear that $\omega_2$ is $k$-linear.
Also, by the first two points we proved, (i) and (ii), we have:
\begin{enumerate}[(i$'$)]
 \item $\omega_2(\alpha) = 0$ for all $\alpha \in \mathcal{A}_X(W') + K(X)$.
 \item If $B'$ is a divisor on $X$ such that $B'\nleq W'$ then there is a $\beta \in \mathcal{A}_X(B')$ with $\omega_2(\beta) \neq 0$.
\end{enumerate}


Now it follows that for $\alpha \in \mathcal{A}_{X/Y}$ we have $\omega_2(\alpha) = \omega_1(\alpha) = \omega(\Tr_{X/Y}(\alpha))$.
This means that we have found the $\omega'$ in the statement of the theorem; namely $\omega' = \omega_2$.

It is clear from (i$'$) and (ii$'$) that $\di(\omega') = W' = \pi^*(\di(\omega)) + R$.

We finally prove the uniqueness of $\omega'$.
Suppose that $\omega''$ also satisfies 
\[
 \omega''(\alpha) = \omega(\Tr_{X/Y}(\alpha))
\]
for all $\alpha\in \cA_{X/Y}$.
So if we let $\theta = \omega'' - \omega'$, then $\theta(\alpha) = 0$ for all $\alpha \in \cA_{X/Y}$.
But then if we choose a large enough divisor $D$ in Lemma \ref{adelespacelemma}, this implies that $\theta = 0$ and $\cA_X$, and hence $\omega' = \omega''$.
\begin{comment}
If $\omega = 0$ this is clear.
The order of the differential at any point can be determined by the equality of divisors,
\[
(\pi^*(\omega) ) = \pi^*((\omega)) + R.
\]
If differential is a unit at a point, it's precise value can be determined by the equality
\[
\omega'(\alpha) = \Tr_{X/Y}(\omega(\alpha)).
\]
\todo{check this last part}
\end{comment}
\end{proof}


\begin{comment}
We now show the equivalence of our two definitions of differential.

Recall that for a curve $C$ a derivation of $K(C)$ is a $k$-linear map $d\colon K(C) \rightarrow M$ for some $K(C)$-module $M$ such that $d(ab) = ad(b) + d(a)b$ for all $a,b\in K(C)$.
Note that of course, as $K(C)$ is a field, $M$ is a vector space, but this does hold more generally for rings and modules.

By \cite[Prop. 4.1.4]{stichtenoth}, for each $x\in K(C)\backslash k$ there exists a unique derivation $d_x\colon K(C) \rightarrow K(C)$ such that $d_x(x) = 1$, which we call the {\em derivation with respect to $x$}.

If we let $Z:= \{(u,x)\in K(C)\times K(C) |x\notin k\}$, then we can define a relation on the elements of $Z$ by letting $(u,x) \sim (v,y)$ if $v = u\cdot d_y(x)$.
Then we denote the equivalence class of $(1,x)\in Z$ by $dx$.
It can then be shown that $Z$, when quotiented by the above relation, has the universal property of $\Omega_C$, and hence is isomorphic to $\Omega_C$ as a $K(C)$-module (see \cite[Prop. 4.1.8]{stichtenoth}).
\end{comment}

Let $C$ be a smooth, projective, connected algebraic curve over $k$.
We now show that our two different definitions of $\Omega_{K(C)}$ give rise to isomorphic differentials.
Recall that $K(x)$ is the field of rational functions of the projective line.
By \cite[Prop. 1.7.4]{stichtenoth} there exists a unique differential $\omega \in \Omega_{K(x)}$ such that $\di (\omega) = -2[P_{\infty}]$ and $\omega(\iota_\infty(x^{-1})) = -1$, where $P_\infty \in \mathbb P_k^1$ is the point at infinity on the projective line.


Now for any $z\in K(C) \backslash k$ we have that $k(z)$ is isomorphic to $K(x)$.
Hence if we let $G$ be the Galois group of the extension $[K(C):k(z)]$ then the quotient curve $Y:=X/G$ is isomorphic to the projective line.
We let $f_z\colon C \rightarrow \mathbb P_k^1$ be the corresponding surjective map, and we denote the unique differential described above by $\omega_z$.
We then define $\delta\colon K(C) \rightarrow \Omega_{K(C)}$ to be the map such that if $z\in K(C)\backslash k$ then $\delta (z) := f_z^*(\omega_z)$ and if $y\in k$ then $\delta(y):=0$.
This then induces a map $\mu\colon \Omega_{K(C)} \rightarrow \Omega_{K(C)}$, defined by $z\cdot dx \mapsto z\cdot \delta(x)$·

\begin{thm}
The map $\mu\colon \Omega_{K(C)} \rightarrow \Omega_{K(C)}$ is an isomorphism.
\end{thm}
\begin{proof}
See \cite[Thm. 4.3.2]{stichtenoth}.
\end{proof}

We now define the order of a poly-differential at a point.
If we consider an element of the tensor product $\omega \in \Omega_X^{\otimes m}$ then it can be locally written as $y dx_1\otimes \ldots \otimes dx_m$, where $x_i \in K(X)$ for all $1 \leq i \leq m$.
Let $P$ be a point in $X$.
Since each $dx_i$ can be written as $y_i dt$ for some $y_i\in K(X)$ and some uniformising parameter $t$ at $P$, we can rewrite $\omega$ as $y' dt \otimes \ldots \otimes dt$, where $y' = y \cdot y_1 \cdots y_m$.
We then define the order of $\omega$ at $P$ to be $\ord_P(\omega ) := \ord_P(y')$.
In the particular case where $\omega = fdx \otimes \ldots \otimes fdx = f^m dx^{\otimes m}$, then we have $y_1 = \ldots = y_m = z$ for some $z$ when we change $x$ to a uniformising parameter.
Hence in this instance \[ \ord_P(\omega) = \ord_P(z^m) = m\ord_P(z) = m\ord_P(dx).\]

\chapter{Faithful actions on Riemann-Roch spaces} \label{Chapter:Faithfulactions}

In this section our main aim is to compute when a subgroup of the automorphism group of an algebraic curve acts faithfully on the space of holomorphic differentials and polydifferentials.
Our approach uses the fact that if any finite group $G$ does not act faithfully on $H^0(X,\Omega_X^{\otimes m})$ then there exists a subgroup of $G$ which fixes at least one element of this $k$ vector space, and the dimension of the space fixed by this subgroup will be positive.

To this end, we start by computing the dimension of $H^0(X,\Omega_X^{\otimes m})$, and the dimension of the fixed space $H^0(X,\Omega_X^{\otimes m})^G$.
These dimensions rely primarily on the genus of the quotient curve $Y:=X/G$, $m$ and the ramification divisor of $\pi \colon X \ra Y$.

Then we use these formulae to compute exactly when a cyclic group of prime order will act trivially on $H^0(X,\Omega_X^{\otimes m})$.
In this case of holomorphic differentials (when $m=1$), this depends solely on the characteristic of $k$, whilst for polydifferentials (\ie when $m \geq 2$) this is independent of $\cha (k)$.
In the same section we also give similar results for more general Riemann-Roch spaces.

We then move on to the main theorem, which answers the question of when $G$ acts faithfully on $H^0(X,\Omega_X^{\otimes m})$.
After giving the main theorem we give examples which illustrate both when we do and do not have faithful actions.
In particular, we use results of Chapter 3 to explicitly show the result holds for hyperelliptic curves.

We close the chapter with an alternative proof of when a cyclic group of prime order acts faithfully on $\hzero$, by studying the $k[G]$-module structure, which was determined in \cite{valmadan}.

\section{Dimension formulae}\label{dimsection}

Throughout this chapter, unless otherwise stated, we assume that $X$ is a connected, smooth, projective algebraic curve over an algebraically closed field $k$ of characteristic $p \geq 0$.
We furthermore assume that $G$ is a finite group of order $n$ that acts faithfully on $X$.
Note that $G$ also induces an action on the vector space $H^0(X,\Omega_X^{\otimes m})$ of global holomorphic poly-differentials of order $m$.
We let $Y$ denote the quotient curve $X/G$, and we let $\pi:X\rightarrow Y$ be the canonical projection.
Finally, we denote by $g_X$ and $g_Y$ the genus of $X$ and $Y$ respectively, and we let $K_X$ and $K_Y$ be canonical divisors on $X$ and $Y$.\todo{put comment in previous section regarding divisors}


In this section we compute the dimension of $H^0(X,\Omega_X^{\otimes m})$ and of $H^0(X,\Omega_X^{\otimes m})^G$, the subspace of $H^0(X,\Omega_X^{\otimes m})$ fixed by $G$.


By Corollary \ref{dim=gc} we have $\dim_kH^0(X,\Omega_X)=g_X$.
If $m\geq 2$, we have the following formula for $\dim_kH^0(X,\Omega_X^{\otimes m})$.

    \begin{lem}\label{dim3}
    Let $m\geq 2$. Then
        \begin{equation}
        \dim_kH^0(X,\Omega_X^{\otimes m}) =
            \begin{cases}
            0 & \mbox{if } g_X=0,\\
            1 & \mbox{if } g_X=1,\\
            (2m-1)(g_X-1) & \mbox{otherwise}.
            \end{cases}
        \end{equation}
    \end{lem}
    \begin{proof}
    The trivial cases of $g_X =0$ and $g_X=1$ are explicitly explained in examples (a) and (b) in section \ref{examplessection}.
    
    If $g_X\geq 2$ then $\deg(K_X)\geq1$, so $\deg(mK_X)>\deg(K_X)$.
    Since $H^0(X,\Omega_X^{\otimes m}) \cong H^0(X,\Omega_X(mK_X))$ it then follows from the Riemann-Roch theorem (Theorem \ref{theoremriemannroch}) that
        \[
        \dim_kH^0(X,\Omega_X^{\otimes m})=\deg(mK_X)+1-g_X=(2m-1)(g_X-1).
        \]
    \end{proof}

We now introduce some notations. 
Let $D=\sum_{P\in X}n_P[P]$ be a $G$-invariant divisor on $X$ (\ie $n_{\sigma(P)} = n_P$ for all $\sigma \in G$ and $P\in X$) and let $\cO_X(D)$ denote the corresponding equivariant invertible $\cO_X$-module. 
Furthermore, let $\pi_*^G(\cO_X(D))$ denote the sub-sheaf of the direct image $\pi_*(\cO_X(D))$ fixed by the obvious action of $G$ on $\pi_*(\cO_X(D))$.
We also let $\left\lfloor \frac{\pi_*(D)}{n} \right \rfloor$ denote the divisor on $Y$ obtained from the push-forward $\pi_*(D)$ by replacing the coefficient $m_Q$ of $Q$ in $\pi_*(D)$ with the integral part $\left \lfloor \frac{m_Q}{n} \right \rfloor$ of $\frac{m_Q}{n}$ for each $Q \in Y$. 
The function fields of $X$ and~$Y$ are denoted by $K(X)$ and $K(Y)$ respectively. 
Finally, for any $P \in X$ let $\ord_P$ and $\ord_Q$ denote the respective valuations of $K(X)$ and $K(Y)$ at $P$ and $Q:=\pi(P)$.



The next lemma is the main idea in the proof of our formula for $\dim_kH^0(X,\Omega_X^{\otimes m})^G$, see Proposition \ref{dim}. 



    \begin{lem}
    Let $D=\sum_{P\in X}n_P[P]$ be a $G$-invariant divisor on $X$.
    Then the sheaves $\pi_*^G(\cO_X(D))$ and $\cO_Y\left(\left\lfloor \frac{\pi_*(D)}{n}\right \rfloor\right)$ are equal as subsheaves of the constant sheaf $K(Y)$ on $Y$. 
    In particular, the sheaf $\pi_*^G(\cO_X(D))$ is an invertible $\cO_Y$-module.
    \end{lem}
    \begin{proof}
    For every open subset $V$ of $Y$ we have 
        \[
        \pi_*^G(\cO_X(D))(V) = \cO_X(D) (\pi^{-1}(V))^G \subseteq K(X)^G = K(Y).
        \]
    In particular, both sheaves are subsheaves of the constant sheaf $K(Y)$ as stated. 
    It therefore suffices to check that their stalks are equal. 
    For any $Q \in Y$ and $P \in \pi^{-1}(Q)$.
    We have
        \begin{align*}
        \lefteqn{\pi_*^G\left(\cO_X(D)\right)_Q = \cO_X(D)_P \cap K(Y)}\\
        &= \left\{f \in K(Y): \ord_P(f) \ge -n_P\right\}\\
        &= \left\{f \in K(Y): \ord_Q(f) \ge - \frac{n_P}{e_P}\right\}\\
        &= \left\{ f \in K(Y): \ord_Q(f) \ge - \left\lfloor\frac{n_P}{e_P} \right\rfloor \right\}\\
        &= \cO_Y\left(\left\lfloor \frac{\pi_*(D)}{n} \right\rfloor\right)_Q,
        \end{align*}
    as desired.
    \end{proof}

The following proposition contains the aforementioned formula for the dimension of the subspace of $H^0(X,\Omega_X^{\otimes m})$ fixed by $G$.
In particular we see that this dimension is completely determined by $m$, $g_Y$ and $\deg \left\lfloor \frac{m\pi_*(R)}{n} \right\rfloor$.


    \begin{prop}\label{dim}
    Let $m\geq 1$. Then the dimension of $H^0(X,\Omega_X^{\otimes m})^G$ is equal to
        \[
        \dim_k \left( H^0(X,\Omega_X^{\otimes m})^G \right) = (2m-1)(g_Y-1) + \deg\left\lfloor\frac{m\pi_*(R)}{n} \right\rfloor,
        \]  
    unless one of the following conditions holds:
        \begin{itemize}
        \item $m=1 \mbox{ and } \deg\left\lfloor\frac{m\pi_*(R)}{n}\right\rfloor = 0$ or
        \item $g_Y=1 \mbox{ and } \deg\left\lfloor\frac{m\pi_*(R)}{n}\right\rfloor = 0$ or
        \item  $g_Y=0 \mbox{ and } \deg\left\lfloor\frac{m\pi_*(R)}{n}\right\rfloor < 2m-1$,
        \end{itemize}
    in which case 
        \[
        \dim_k \left( H^0(X,\Omega_X^{\otimes m})^G \right) = g_Y.
        \]      
    \end{prop}
    \begin{proof}
    Let $E$ denote the divisor $\left\lfloor \frac{\pi_*(mK_X)}{n} \right\rfloor$ on $Y$. As $K_X=\pi^*(K_Y)+R$ we have
        \[ 
        E = 
        \left \lfloor \frac{\pi_*\pi^*(mK_Y) + \pi_*(mR)}{n} \right \rfloor =
        mK_Y + \left \lfloor \frac{m\pi_*(R)}{n} \right \rfloor.
        \]
    Using the previous lemma we conclude that $\pi_*^G(\Omega_X^{\otimes m}) \cong \cO_Y (E)$ and finally that
        \begin{equation*}
        \dim_k H^0(X,\Omega_X^{\otimes m})^G 
        = \dim_k H^0\left(Y, \pi_*^G(\Omega_X^{\otimes m})\right)
        = \dim_k H^0\left(Y, \cO_Y\left( E \right) \right).
        \end{equation*}
    
    
    In the first case of the proposition, \ie if $m=1$ and $\deg \left\lfloor\frac{m\pi_*(R)}{n} \right\rfloor=0$, then $\left\lfloor\frac{m\pi_*(R)}{n} \right\rfloor$ is the zero divisor and we conclude that 
        \begin{equation*}
        \dim_kH^0(X,\Omega_X)^G = \dim_kH^0(Y, \Omega_Y) = g_Y.
        \end{equation*}
    
    
    In the second case $\left\lfloor \frac{m\pi_*(R)}{n} \right\rfloor$ is again the zero divisor. 
    Furthermore, as $g_Y=1$, the divisor $K_Y$ is equivalent to the zero divisor, and hence $mK_Y$ is too. 
    This means that
        \begin{equation*}
        \dim_kH^0(X,\Omega_X^{\otimes m})^G = \dim_kH^0\left( Y,\cO_Y\left( E \right) \right) 
        = \dim_k  H^0\left( Y,\cO_Y\left( 0 \right) \right)
        = 1.
        \end{equation*}
    
    
    For the third case, by \cite[Chap. IV, ex. 1.3.4]{hart} it suffices to show that $\deg \left( E \right) < 0$.
    As $g_Y=0$ we have $\deg(K_Y)=-2$, so $\deg(mK_Y)=-2m$, and $\deg \left( E \right)$ is indeed negative.
    
    
    
    We will show below that in all other cases $\deg(E) > \deg(K_Y)$, and then the Riemann-Roch formula (Theorem \ref{theoremriemannroch}) will give 
        \begin{align*}
        \lefteqn{\dim_kH^0(X,\Omega_X^{\otimes m})^G = \dim_kH^0\left(Y,\cO_Y\left( E \right)\right)} \\
        & =  1-g_Y+\deg\left(mK_Y+\left\lfloor{\frac{m\pi_*(R)}{n}}\right\rfloor\right) \\
        & =  (2m-1)(g_Y-1)+\deg\left\lfloor{\frac{m\pi_*(R)}{n}}\right\rfloor,
        \end{align*}
    completing the proof for this case.
    
    
    All that remains is to show that $\deg(E)>\deg(K_Y)$ in all other cases.
    Firstly, if $g_Y=0$ and $\deg \left\lfloor\frac{m\pi_*(R)}{n} \right\rfloor \geq 2m-1$ then, since $\deg(mK_Y)=-2m$, we have 
        \[
        \deg \left( E \right) \geq -1 >-2 = \deg(K_Y).
        \]
    Similarly, if $g_Y=1$ and $\deg \left\lfloor\frac{m\pi_*(R)}{n} \right\rfloor >0$ then, as $\deg \left( mK_Y \right)=0$, we have $\deg \left( E \right) > 0 = \deg (K_Y)$.
    If $m=1$ and $\deg \left\lfloor\frac{m\pi_*(R)}{n} \right\rfloor >0$ then clearly $\deg \left( E \right) > \deg (K_Y)$.
    Lastly, if $m\geq 2$ and $g_Y\geq 2$ then $\deg (K_Y) > 0$ and we have 
        \begin{equation*}
        \deg \left( E \right) \geq \deg\left( mK_Y \right) > \deg (K_Y).
        \end{equation*}
    So in all other cases $\deg(E)>\deg(K_Y)$, and the proof is complete.
    \end{proof}


If $m=1$ we reformulate Proposition \ref{dim} in the following slightly more concrete way. 
Let $S$ denote the set of all points $Q\in Y$ such that $\pi$ is not tamely ramified at $Q$ and let $s$ denote the cardinality of $S$. 
Note that $s=0$ if $p$ does not divide $n$.


For the next corollary we recall the notations $e_Q$ and $\delta_Q$ for any $Q\in Y$ defined before Theorem \ref{hilbertsformula}.


    \begin{cor}\label{dim2}
    We have 
        \begin{equation*}
        \dim_kH^0(X,\Omega_X)^G = 
            \begin{cases}
            g_Y & \mbox{if } s=0, \\
            g_Y-1+\sum_{Q\in S}\left\lfloor \frac{\delta_Q}{e_Q} \right\rfloor & \mbox{otherwise}.
            \end{cases}
        \end{equation*}
    \end{cor}
    \begin{proof}
    We have
        \[
        \deg\left\lfloor\frac{\pi_*(R)}{n} \right\rfloor = \sum_{Q\in Y}\left\lfloor\sum_{P\mapsto Q} \frac{\delta_P}{n} \right\rfloor = \sum_{Q\in Y} \left\lfloor \frac{\delta_Q}{e_Q} \right\rfloor.
        \]
    Furthermore we have $\left\lfloor \frac{\delta_Q}{e_Q} \right\rfloor = 0$ if and only if $\delta_Q<e_Q$, \ie if and only if $Q\notin S$. 
    Thus Corollary \ref{dim2} follows from Proposition \ref{dim}.
    \end{proof}

    \begin{rem}
    Note that if $p>0$ and $G$ is cyclic then Corollary \ref{dim2} can be derived from Proposition $6$ in the recent pre-print \cite{kako} by Karanikolopoulos and Kontogeorgis.\todo{check to see if preprint is now published}
    \end{rem}



\section{Trivial action in the cyclic case}

In this section we will look at the case where $G$ is a cyclic group of prime order, or a power of a prime, and determine when $G$ acts trivially on $H^0(X,\Omega_X^{\otimes m})$.
Compared to arbitrary groups, it is considerably easier to compute when these groups act trivially, and we will later see that we can reduce to this case, regardless of what the structure of $G$ is.


Throughout this section, $P_1,\ldots ,P_r \in X$ denote the ramification points of $\pi$ and we write $e_i$ and $\delta_i$ for $e_{P_i}$ and $\delta_P{_i}$.
Also, for $i=1, \ldots, r$, we define $N_i \in \NN$ by $\ord_{P_i}(\sigma(\pi_i) - \pi_i) = N_i +1$, where $\pi_i$ is a local parameter at the ramification point $P_i$ and $\sigma$ is a generator of $G(P_i)$. 
We also assume that $g_X \geq 2$.


    \begin{prop}\label{m=1}
    Let $p  > 0$ and let $G$ be cyclic of order $p$.
    Furthermore, we assume that $g_Y=0$.
    Then $G$ acts trivially on $H^0(X,\Omega_X)$ if and only if $p=2$. 
    \end{prop}
    \begin{proof}
    From \cite[Lem. 1]{Naka} we know that $p$ does not divide $N_i$ for $i=1,\ldots ,r$, a fact we will use several times below. \todo{more specific citation}
    Let $N:= \sum_{i=1}^r N_i$. 
    Using the Riemann-Hurwitz formula, Corollary \ref{corhurwitzformula}, we obtain
        \begin{equation}\label{hur2}
        2g_X - 2 = -2p + (N+r)(p-1)
        \end{equation}
    and hence
        \[
        \dim_kH^0(X,\Omega_X) = g_X =\frac{(N+r-2)(p-1)}{2}.
        \] 
    Since $g_X \ge 0$ we obtain $r \ge 1$; that is, $\pi$ is not unramified. 
    As $\cha(k) = p = \ord(G)$, the morphism $\pi$ is not tamely ramified, and the cardinality $s$ defined before Corollary \ref{dim2} is not zero.
    Therefore we have 
        \[
        \deg \left\lfloor \frac{\pi_*(R)}{p} \right\rfloor =
        \sum_{i=1}^r \left\lfloor \frac{(N_i+1)(p-1)}{p}\right\rfloor 
        \ge \sum_{i=1}^r \left\lfloor \frac{2(p-1)}{p}\right\rfloor = r > 0.
        \] 
    From Corollary \ref{dim2} we then conclude that 
        \begin{align*}
        \dim_kH^0\left(X,\Omega_X\right)^G & =  g_Y - 1 + \sum_{i=1}^r\left\lfloor \frac{\delta_i}{e_i}\right\rfloor \\
        & =  -1 + N + r \sum_{i=1}^r\left\lfloor -\frac{N_i+1}{p}\right\rfloor.
        \end{align*}
    
    If $p=2$, the dimension of both $H^0(X,\Omega_X)$ and $H^0(X,\Omega_X)^G$ is therefore equal to $\frac{N+r-2}{2}$. 
    This shows the if-direction in Proposition \ref{m=1}.
    
    
    
    To prove the other direction we now assume that $G$ acts trivially on $H^0(X, \Omega_X)$.
    For each $i=1, \ldots, r$, we write $N_i = s_i p +t_i$ with $s_i \in \NN$ and $t_i \in \{1, \ldots, p-1\}$. 
    We furthermore put $S:=\sum_{i=1}^r s_i$ and $T:= \sum_{i=1}^r t_i \ge r$. 
    Then we have
        \[ 
        \frac{(N+r-2)(p-1)}{2} =\dim_kH^0(X,\Omega_X)  = \dim_k H^0(X,\Omega_X)^G = N-S-1 .
        \]
    Rearranging this equation we obtain
        \[
        (3-p)N - 2S = (r-2)(p-1) +2  
        \]
    and hence
        \[
        (-p^2 + 3p -2)S = (r-2)(p-1) +2 - (3-p)T.
        \]
    Assuming that $p \ge 3$ this equation implies that
        \[ 
        S = \frac{(r-2)(1-p)-2 + T (3-p)}{(p-1)(p-2)}. 
        \]
    since $-p^2+3p-2 = - (p-1)(p-2)$. 
    
    Because $S \geq 0$, the numerator of this fraction is non-negative, that is
        \begin{align*}
        \lefteqn{0 \le (r-2)(1-p) - 2 + T (3-p)}\\
        &\le  (r-2)(1-p) - 2 + r (3-p)\\
        &= 2 (r-1)(2-p).
        \end{align*}
    Hence we have that $r=1$ and that the numerator is $0$. 
    We conclude that $S=0$ and hence that $T=1$ or $p=3$. 
    If $T=1$ we also have $N=1$ and finally
        \[
        g_X = \frac{(N+r-2)(p-1)}{2} = 0,
        \]
    a contradiction.
    If $T \not=1$ and $p=3$ we obtain $N=T=2$ and finally 
        \[
        g_X = \frac{(N+r-2)(p-1)}{2} =1,
        \] 
    again a contradiction.
    \end{proof}

    \begin{prop}\label{triv}
    Let $m \geq 2$. 
    Suppose that $G$ is a cyclic group of prime order $l$ (which may or may not be equal to $p$) and that $g_Y=0$. 
    Then $G$ acts trivially on $H^0(X,\Omega_X^{\otimes m})$ if and only if $g_X=m=l=2$.
    \end{prop}
    \begin{proof}
    We have different proofs according to whether or not the order $l$ of the group is the same as the characteristic $p$ of the field.
    
    
    First we assume that $l=p$. 
    As in the proof of Proposition \ref{m=1}, we let $N=\sum_{i=1}^r N_i$, and we let $M=N+r$.
    Then due to (\ref{hur2}) we have
        \begin{equation}\label{simplehur}
        2g_X-2=-2p+M(p-1),
        \end{equation}
    and combining this with Lemma \ref{dim3} we can write
        \begin{equation}\label{altdim2}
        \dim_kH^0(X,\Omega_X^{\otimes m})=(2m-1)(g_X-1)=(2m-1)\left(\frac{M(p-1)-2p}{2}\right).
        \end{equation}
    
    Furthermore, we have
        \begin{equation}\label{altdim}
        \deg\left\lfloor \frac{m\pi_*(R)}{p} \right\rfloor = \sum_{i=1}^r\left\lfloor \frac{m(N_i+1)(p-1)}{p} \right\rfloor  = mM + \sum_{i=1}^r\left\lfloor \frac{-m(N_i+1)}{p} \right\rfloor.
        \end{equation}
    If we have $p=g_X=m=2$, then on the one hand we see that $\dim_kH^0(X,\Omega_X^{\otimes m}) =3$. 
    On the other hand, we first note that \eqref{simplehur} implies $M=6$.
    So 
        \begin{equation*}
        \deg\left\lfloor \frac{m\pi_*(R)}{p}\right\rfloor = 2M -M =6 > 3 = 2m-1.
        \end{equation*}  
    Then, by Proposition \ref{dim}, we obtain 
        \begin{equation*}
        \dim_kH^0(X,\Omega_X^{\otimes m})^G = (2m-1)(g_Y-1)+\deg\left\lfloor \frac{m\pi_*(R)}{p} \right\rfloor = -3 + 6 = 3.
        \end{equation*}
    So the two dimensions are equal and the action of $G$ on $H^0(X,\Omega_X^{\otimes m})$ is trivial. 
    This completes the if direction of the proof.
    
    Now we assume that the action is trivial. This first implies that 
    $\deg \left\lfloor\frac{m\pi_*(R)}{p}\right\rfloor \geq 2m-1$ because otherwise we would 
    have $\dim_kH^0(X,\Omega_X^{\otimes m})^G=0$ by Proposition \ref{dim}, but we know that 
    $\dim_kH^0(X,\Omega_X^{\otimes m})=(2m-1)(g_X-1)$ is strictly positive.\todo{rewrite sentence}
    So, using \eqref{altdim}, \eqref{altdim2} and Proposition \ref{dim} we see that
        \begin{align}\label{bound}
        \lefteqn{(2m-1)\frac{M(p-1)-2p}{2} = \dim_kH^0(X,\Omega_X^{\otimes m})} \nonumber\\
        & =  \dim_kH^0(X,\Omega_X^{\otimes m})^G \nonumber\\
        & =  1-2m+mM+\sum_{i=1}^r\left\lfloor\frac{-m(N_i+1)}{p}\right\rfloor\nonumber \\
        & \leq  1-2m+mM+\sum_{i=1}^r\frac{-m(N_i+1)}{p}\nonumber \\
        & =  1-2m+mM-\frac{mM}{p}.
        \end{align}
    
    After multiplying by $2p$ and rearranging we obtain
        \begin{align}\label{times2p}
        0 & \geq  (2mM-M-4m+2)p^2+(-4mM+M-2+4m)p+2mM \nonumber \\
            & =  (M-2)(2m-1)p^2-((M-2)(2m-1)+2mM)p+2mM \nonumber \\
        & =  (p-1)((M-2)(2m-1)p-2mM).
        \end{align}
    
    Furthermore from \eqref{hur2} we obtain that $-2p+M(p-1)=2g_X-2 \geq 2$ and hence that 
        \begin{equation}\label{greater2}
        M\geq \frac{2+2p}{p-1}=2+\frac{4}{p-1}>2.
        \end{equation}
    
    So from \eqref{times2p} and \eqref{greater2} we see that
        \begin{align}\label{plessthan4}
        p & \leq  \frac{2mM}{(M-2)(2m-1)}\nonumber\\
        & =  \frac{M}{M-2}\cdot\frac{2m}{2m-1}\nonumber\\
        & =  \left( 1+\frac{2}{M-2} \right) \left(1+\frac{1}{2m-1} \right)\\
        & \leq  4, \nonumber	
        \end{align}
    \ie $p=2$ or $p=3$. 
    
    Suppose that $p=3$. Then from \eqref{greater2} we have $M\geq 4$. However, from  \eqref{plessthan4} we also have that 
        \begin{align*}
        3 & \leq \left( 1+\frac{2}{M-2} \right) \left(1+\frac{1}{2m-1} \right)\\
        & \leq  \left( 1+\frac{2}{M-2} \right) \frac{4}{3}\\
        & \leq  \frac{8}{3},
        \end{align*}
    a contradiction.
    
    Lastly, we come to the case when $p=2$. From \eqref{plessthan4} we see that $2\leq \left(1+\frac{2}{M-2}\right)\frac{4}{3}$ 
    and hence $M\leq 6$. However, from \eqref{greater2} we know that $M\geq 6$, so $M=6$. Then from \eqref{bound}  we obtain that $2m-1=1-2m+6m-3m$
    and hence that $m=2$. Finally, (\ref{hur2}) gives us that $2g_X-2=-4+6=2$ and hence $g_X=2$. 
    This completes the only if direction of the proof when $l=p$.
    
    Now if $l\neq p$ then we know that all the coefficients $\delta_i$ of the ramification divisor are equal to $l-1$. 
    To show the if direction in this case, first note that $\dim_kH^0(X,\Omega_X^{\otimes m})=3$ by Lemma~\ref{dim3}. 
    On the other hand, the Riemann-Hurwitz formula (Corollary \ref{corhurwitzformula}) implies that $2 = 2g_X-2=-2l+\deg(R)=-2l+r(l-1)$, and hence that $r=6$. 
    Finally Proposition \ref{dim} gives us
        \begin{equation*}
        \dim_kH^0(X,\Omega_X^{\otimes m})^G = -(2m-1) + \sum_{i=1}^r \left\lfloor \frac{m\cdot \delta_i}{l} \right\rfloor
        = -3 +\sum_{i=1}^6 \left\lfloor \frac{m(l-1)}{l} \right\rfloor
        = 3,
        \end{equation*}
    since $m=l=2$.
    As the dimensions of $H^0(X,\Omega_X^{\otimes m})$ and $H^0(X,\Omega_X^{\otimes m})^G$ are equal, the action is trivial.
    
    
    Now, for the final section of the proof we suppose that $G$ acts trivially on the space $H^0(X,\Omega_X^{\otimes m})$.
    We then show that this implies that $g_X=l=m=2$.
    
    
    From Lemma \ref{dim3} and Proposition~\ref{dim} we obtain
        \begin{align*}
        \lefteqn{(2m-1)(g_X-1)=\dim_kH^0(X,\Omega_X^{\otimes m})} \\
        & =  \dim_kH^0(X,\Omega_X^{\otimes m})^G=-(2m-1)+\sum_{i=1}^r \left\lfloor \frac{m\cdot \delta_i}{l} \right\rfloor
        \end{align*}
    and hence
        \begin{equation*}
        (2m-1)g_X = \sum_{i=1}^r \left\lfloor \frac{m\cdot \delta_i}{l} \right\rfloor
        = \sum_{i=1}^r \left\lfloor \frac{m(l-1)}{l} \right\rfloor
        = r\left( m+\left\lfloor \frac{-m}{l} \right\rfloor \right).
        \end{equation*}
    By choosing $s\in \{1,\ldots ,l\}$ and $q\in \mathbb{N}$ such that $m=ql+s$ we can rewrite this as
        \begin{equation}\label{eq:mult}
        (2m-1)g_X=r(m-q-1).
        \end{equation}
    If we multiply (\ref{eq:mult}) by $l-1$ and then substitute in for the $r(l-1)$ term in the Riemann-Hurwitz formula (Corollary \ref{corhurwitzformula}) we get
        \begin{equation*}
        (2m-1)(l-1)g_X=(2g_X+2(l-1))(m-q-1).
        \end{equation*}
    By rearranging we are able to compute $g_X$ in terms of $m,l$ and $q$:
        \begin{align}\label{equationgxintermsofmandlandq}
        \lefteqn{g_X = \frac{2(l-1)(m-q-1)}{(2m-1)(l-1)-2(m-q-1)}} \nonumber \\
        & =  1 + \frac{2(m-q-1)-(2q+1)(l-1)}{(2m-1)(l-1)-2(m-q-1)}  \nonumber\\
        & =  1 + \frac{2s-1-l}{(2m-1)(l-1)-2(m-q-1)} \nonumber  \\
        & =  1 + \frac{2(s-1)+1-l}{(2m-1-2q)(l-1)-2(s-1)}. 
        \end{align}
    First, we show that if $l\geq 3$ the equation cannot hold whilst $g_X\geq 2$.
    Observe that the denominator is strictly greater than $l-1$, remembering that $m=ql+s$:
        \begin{align*}
        (2m-1-2q)(l-1)-2(s-1) & =  ((2q(l-1)+2s-1)(l-1)-2(s-1) \\
        & \geq  (2s-1)(l-1)-2(s-1) \\
        & \geq  (2s-1)(l-1)-2(s-1)(l-1) \\
        & =  l-1;
        \end{align*}
    here the two inequalities are equalities if and only if $q=0$ and $s=1$, respectively, and, as $m\geq 2$, not both inequalities can be equalities.
    Now the numerator is at most $l-1$, occurring when $s=l$. 
    Hence if $l\geq 3$ the fraction in \eqref{equationgxintermsofmandlandq} will be less than one and $g_X < 2$, contradicting our assumption.
    If $l=2$, then $s$ is either 1 or 2.
    If $s=1$ the fraction is negative, and $g_X<1$, which again contradicts our assumption.
    Finally, if $s=2$ then $g_X\leq 2$, with equality if and only if $q=0$, \ie~if and only if $m=2$.
    So if $g_X \geq 2$ then the action being trivial implies that $g_X=l=m=2$, and the proof is complete.    
    \end{proof}

For the rest of this section we assume that $p>0$ and that $G$ is a cyclic group of order $p^l$ for some $l \in \NN$.
What we are now going to do will not be used in the proof of the main theorem, but is included because it generalises the previous results.
More precisely, we do not restrict ourselves to looking at $H^0(X,\Omega_X^{\otimes m})$, but using a comparatively deep result from \cite{kako} we study $H^0(X,\cO(D))$ for any $G$-invariant divisor $D$ such that $\deg(D)>2g_X-2$.


We first introduce some notation.
Let $D = \sum_{P\in X} n_P[P]$ be a $G$-invariant divisor on $X$.
Then let $\langle a \rangle$ denote the fractional part of any $a\in \mathbb{R}$, \ie $\langle a \rangle = a - \lfloor a \rfloor$.
Also, for any $Q\in Y$ let $n_Q$ be equal to $n_P$ for any $P\in \pi^{-1}(Q)$.




    \begin{prop}\label{nakaj}
    Suppose $p>0$ and $G$ is a cyclic group of order $p^l$ for some $l\geq 1$.
    Let $D$ be a $G$-invariant divisor on $X$ such that $\deg(D)>2g_X-2$.
    Then the action of~$G$ on $H^0(X,\cO_X(D))$ is trivial if and only if
        \[ 
        (p^l-1)\deg(D)=p^l\left(g_X-g_Y-\sum_{Q\in Y}\left\langle \frac{n_Q}{e_Q} \right\rangle\right).
        \]
    \end{prop}
    \begin{proof}
    We first remind the reader of the notation in \cite{kako}.
    Let $\sigma$ be a generator of $G$.
    Let $V$ be the $k[G]$ module with $k$-basis $e_1,\ldots ,e_{p^l}$ and $G$-action defined by $\sigma( e_i)=e_i+e_{i-1}$, $1\leq i \leq p^l,\ e_0=0$.\todo{changed from $\sigma \cdot e_i$ to how is now. Check rest of work for issues}
    Then $V_j$, defined to be the subspace of $V$ spanned by $e_1,\ldots ,e_j$ over $k$, is also a $k[G]$ module.
    In fact, the modules $V_1,\ldots ,V_{p^l}$ form a complete set of representatives for the set of isomorphism classes of indecomposable $k[G]$-modules. For each $j=1,\ldots,p^l$ let $m_j$ denote the multiplicity of $V_j$ in the $k[G]$-module $H^0(X,\cO_x(D))$, \ie we have $H^0(X,\cO_x(D))\cong \oplus_{j=1}^{p^l}m_jV_j$.
    
    
    
    First note that the action of $G$ on $H^0(X,\cO_X(D))$ is trivial if and only if
        \begin{equation}\label{triva}
        \dim_k H^0(X,\cO_X(D))^G =\dim_k H^0(X,\cO_X(D)).
        \end{equation}
    
    It is clear that the $G$-invariant part of each submodule $V_j$ is spanned by $e_1$. 
    Hence $\dim_kH^0(X,\cO_X(D))^G = \sum_{j=1}^{p^l} m_j$.
    By \cite[Thm. 2.1]{quaddiffequi}, which relies on \cite{cohogsheaves}, we have
        \begin{align*}
        \sum_{j=1}^{p^l} m_j & =  1- g_Y +\sum_{Q\in Y} \left\lfloor \frac{n_Q}{e_Q}\right\rfloor\\
        & =  1- g_Y + \sum_{Q\in Y} \left( \frac{n_Q}{e_Q} - \left\langle \frac{n_Q}{e_Q}\right\rangle \right) \\
        & =  1 - g_Y + \frac{1}{p^l}\deg(D) - \sum_{Q\in Y} \left\langle \frac{n_Q}{e_Q} \right\rangle.
        \end{align*}
    
    Now as $\deg(D)>2g_X-2$ we have $\dim_kH^0(X,\cO_X(D)) =\deg(D)+1-g_X$ by the Riemann-Roch theorem. 
    So the action of $G$ on $H^0(X,\cO_X(D))$ is trivial if and only if
        \begin{equation*}
        \deg(D)+1-g_X  = 1 - g_Y + \frac{1}{p^l}\deg(D) - \sum_{Q\in Y}\left\langle \frac{n_Q}{e_Q} \right\rangle. \label{hi}
        \end{equation*}
    
    This then rearranges to $(p^l-1)\deg(D)=p^l\left(g_X-g_Y-\sum_{Q\in Y}\left\langle \frac{n_Q}{e_Q} \right\rangle\right)$, as desired.
    \end{proof}

    \begin{cor}\label{this}
    Suppose that $\deg(D)\geq 2g_X$. Then the action of $G$ on $H^0(X,\cO_X(D))$ is trivial if and 
    only if $g_Y = 0$, $e_Q | n_Q$ for all $Q\in Y$, $\deg(D)=2g_X$ and either $g_X=0$ or $p^l=2$.
    \end{cor}
    \begin{proof}
    The following inequalities always hold under the stated assumptions:
        \begin{multline}
        (p^l-1)\deg(D)\geq (p^l-1)2g_X \geq p^lg_X \geq p^lg_X-p^l\sum_{Q\in Y}\left\langle\frac{n_Q}{e_Q}\right\rangle \\ \geq p^l\left( g_X - g_Y -\sum_{Q\in Y}\left\langle \frac{n_Q}{e_Q} \right\rangle \right).
        \end{multline}
    Now the first inequality is an equality if and only if $\deg(D)=2g_X$. 
    The second is an equality if and only if either $g_X=0$ or $p^l=2$. 
    The third inequality is an equality if and only if $\sum_{Q\in Y}\left\langle\frac{n_Q}{e_Q}\right\rangle=0$, which is the case if and only if each $n_Q$ is divisible by~$e_Q$. 
    Lastly, the fourth inequality is an equality if and only if $g_Y = 0$.
    Given these observations, Proposition \ref{nakaj} implies Corollary~\ref{this}.
    \end{proof}

The following Corollary slightly strengthens the only if direction of the $l=p$ part of Proposition \ref{triv}
(from $\ord(G) = p$ to $\ord(G) = p^l$) and also provides a different proof for it;
note that this new proof relies on the comparatively deep result result in section 7 of \cite{cohogsheaves}.


    \begin{cor}
    Let $m \geq 2$ and let $G$ be a cyclic group of order $p^l$ for some $l$. 
    If $G$ acts trivially on $H^0(X,\Omega_X^{\otimes m})$, then $g_Y = 0$ and $p^l = g_X = m = 2$.
    \end{cor}
    \begin{proof}
    As $g_X \geq 2$ and $m\geq 2$ we have $\deg(mK_X) \geq 2g_X$. 
    So, if the action of $G$ on $H^0(X,\Omega_X^{\otimes m})$ is trivial, we obtain from Corollary \ref{this} that $\deg(mK_X) = 2g_X$, $p^l = 2$ and $g_y = 0$.
    Now $\deg (mK_X) = 2g_X$ implies that $m(2g_X -2 ) = 2g_X$, so $m(g_X -1) = g_X$ and hence $m=g_X=2$.
    \end{proof}

Similarly to the case $\deg(D)\geq 2g_X$ in Corollary \ref{this}, the following corollary derives necessary and sufficient conditions for trivial action from Proposition \ref{nakaj} in the case $\deg(D) =2g_X-1$.



    \begin{cor}
    Suppose that $\deg(D)= 2g_X-1$ and that $g_Y=0$. Then the action of $G$ on $H^0(X,\cO_X(D))$ is trivial if and only if one of the following conditions hold:
        \begin{itemize}
        \item  $p^l=2$ and $\sum_{Q\in Y}\left\langle\frac{n_Q}{e_Q}\right\rangle=\frac{1}{2}$;
        \item  $g_X=2$, $p^l=3$ and $e_Q\mid n_Q$ for all $Q\in Y$.
        \end{itemize}
    \end{cor}


    \begin{rem}
    It can easily be shown that in the last case the Riemann-Hurwitz formula implies that $r\leq 4$. 
    Furthermore, if $r=1$ then the conditions ``$\sum_{Q\in Y}\left\langle\frac{n_Q}{e_Q}\right\rangle=\frac{1}{p^l}$" and ``$e_Q\mid n_Q$ for all $Q\in Y$" are already implied by ``$\deg(D)=2g_X-1$".
    \end{rem}

    \begin{proof}
    Firstly, if $g_X=0$ then $\deg(D)=-1<0$, so $\dim_kH^0(X,\cO_X(D))=0$ and the action is trivial.
    
    Now note that, as $\deg(D)=2g_X-1$, we conclude from Proposition \ref{nakaj} that the action is trivial if and only if 
        \begin{equation*}
        (p^l-1)(2g_X-1)=p^l\left(g_X-\sum_{Q\in Y}\left\langle\frac{n_Q}{e_Q}\right\rangle\right).
        \end{equation*}
    If $p^l=2$ then this is equivalent to $2g_X-1=2g_X-2\sum_{Q\in Y}\left\langle\frac{n_Q}{e_Q}\right\rangle$ and hence to $\sum_{Q\in Y}\left\langle\frac{n_Q}{e_Q}\right\rangle=\frac{1}{2}$.
    
    If $g_X=1$ then this is equivalent to $p^l-1=p^l-p^l\sum_{Q\in Y}\left\langle\frac{n_Q}{e_Q}\right\rangle$ and hence is also equivalent to $\sum_{Q\in Y}\left\langle\frac{n_Q}{e_Q}\right\rangle=\frac{1}{p^l}$.
    
    Lastly, if $p^l\geq 3$ and $g_X\geq 2$ then we have that $g_X\geq \frac{p^l-1}{p^l-2}$ which is equivalent to the first inequality in the chain
        \begin{equation*}
        (p^l-1)(2g_X-1)\geq p^lg_X\geq p^lg_X-p^l\sum_{Q\in Y}\left\langle\frac{n_Q}{e_Q}\right\rangle \geq p^l\left( g_X - g_Y -\sum_{Q\in Y} \left\langle \frac{n_Q}{e_Q} \right\rangle \right).
        \end{equation*}
    Hence the action is trivial if and only if both inequalities are equalities, which is the case if and only if $p^l=3,\ g_X=2$, $e_Q\mid n_Q$ for all $Q\in Y$ and $g_Y = 0$.
    \end{proof}


\section{The main theorem}\label{maintheoremsection}
In this section we prove the main theorem of this chapter, describing exactly when $G$ will act faithfully on $H^0(X,\Omega_X^{\otimes m})$.


    \begin{thm}\label{theoremfaithfulaction}
    Suppose that $g_X\geq 2$ and let $m\geq1$. 
    Then $G$ does not act faithfully on $H^0(X,\Omega_X^{\otimes m})$ if and only if $G$ contains a hyperelliptic involution and one of the following two sets of conditions holds:
        \begin{itemize}
        \item $m=1$ and $p=2$;
        \item $m=2$ and $g_X=2$.
        \end{itemize}
    \end{thm}
    \begin{proof}
    We first show the if direction. 
    In the case when $m=1$, the hyperelliptic involution contained in $G$ generates a subgroup of order $2$.
    Since $p=2$, this acts trivially by Proposition \ref{m=1}, and hence $G$ does not act faithfully.
    In the case when $m=2$, then again looking at the subgroup generated by the hyperelliptic involution, we have a group of order $2$ acting on $H^0(X,\Omega_X^{\otimes m})$.
    So, by Proposition \ref{triv} and since $g_X=m=2$, the action of this subgroup is trivial, and again, this means that $G$ does not act faithfully.
    
    
    We now start the proof of the only if direction, supposing that $G$ does not act faithfully on $H^0(X,\Omega_X^{\otimes m})$. 
    By replacing $G$ with the (non-trivial) kernel $H$ if necessary, we may assume that $G$ is non-trivial and acts trivially on $H^0(X,\Omega_X^{\otimes m})$.
    
    
    We start the proof by showing that $g_Y=0$, which is shown separately for the cases when $m=1$ and when $m\geq 2$.
    In the case when $m=1$ we start by showing that $\deg  \left\lfloor \frac {\pi_*(R)}{n} \right\rfloor >0$ by contradiction.
    Suppose that $\deg\left\lfloor \frac{\pi_*(R)}{n} \right\rfloor =0$.
    As $G$ acts trivially it follows from Proposition~\ref{dim} that:
        \begin{equation*}
        g_X=\dim_k H^0(X,\Omega_X)=\dim_k H^0(X,\Omega_X)^G=g_Y.
        \end{equation*}
    Substituting this into the Riemann-Hurwitz formula (Corollary \ref{corhurwitzformula}) yields the desired contradiction because $g_X\geq 2, n\geq 2$ and $\deg(R)\geq 0$.
    
    Thus $\deg\left( \left\lfloor \frac{\pi_*(R)}{n} \right\rfloor \right) >0$. 
    Now Proposition~\ref{dim} gives us that
        \begin{equation*}
        g_X=\dim_k H^0(X,\Omega_X)=\dim_k H^0(X,\Omega_X)^G= g_Y-1+\deg\left\lfloor \frac{\pi_*(R)}{n} \right\rfloor.
        \end{equation*}
    Substituting this in to the Riemann-Hurwitz formula we see that
        \begin{equation*}
        2\left(g_Y - 1 + \deg\left \lfloor \frac{\pi_*(R)}{n} \right \rfloor -1 \right) = 2n (g_Y -1) + \deg(R).
        \end{equation*}
    For any $Q \in Y$ we let $\delta_Q$ denote the coefficient of the ramification divisor $R$ at any $P \in \pi^{-1}(Q)$ and let $e_Q := e_P$ for any $P \in \pi^{-1}(Q)$. 
    Rewriting the previous equation then yields
        \begin{align*}
        \lefteqn{(2n-2)g_Y = 2n-4 + 2 \,\deg\left \lfloor \frac{\pi_*(R)}{n}\right \rfloor - \deg(R)}\\
        &= 2 \left(n-2 + \sum_{Q \in Y} \left(\left\lfloor \frac{n}{e_Q} \frac{\delta_Q}{n} \right\rfloor - \frac{n}{e_Q} \frac{\delta_Q}{2}\right) \right)\\
        &= 2 \left(n-2 + \sum_{Q \in Y} \left( \left\lfloor \frac{\delta_Q}{e_Q} \right\rfloor - \frac{\delta_Q}{e_Q} \frac{n}{2} \right)\right)\\
        & \le  2(n-2),
        \end{align*}
    because $\frac{n}{2} \ge 1$ and $\left\lfloor \frac{\delta_Q}{e_Q}\right\rfloor \le \frac{\delta_Q}{e_Q}$ for all $Q \in Y$. 
    Hence we obtain $g_Y \le \frac{n-2}{n-1} < 1$ and therefore $g_Y =0$, as desired.
    
    We now show that $g_Y=0$ when $m\geq 2$. 
    Since $g_X\geq 2$ we have that $\deg(mK_X)=m(2g_X-2)>2g_X-2=\deg(K_X)$.
    By Lemma \ref{dim3}, and as both $m$ and $g_X$ are at least 2, then $\dim_kH^0(X,\Omega_X^{\otimes m})^G=\dim_kH^0(X,\Omega_X^{\otimes m})=(2m-1)(g_X-1)>1$.
    There is only one case in Proposition \ref{dim} such that $m\geq 2$ and $\dim_k H^0(X,\Omega_X^{\otimes m})^G>1$, which yields 
        \begin{equation*}
        (2m-1)(g_X-1)=(2m-1)(g_Y-1)+\deg\left(\left\lfloor \frac{m\pi_*(R)}{n} \right\rfloor \right).
        \end{equation*}
    Combining this with the Riemann-Hurwitz formula, Corollary \ref{corhurwitzformula}, we see that
        \begin{align*}
        2(2m-1)(g_Y-1)+2\deg\left(\left\lfloor\frac{m\pi_*(R)}{n}\right\rfloor\right) & =  2(2m-1)(g_X-1)\\
        & =  2n(2m-1)(g_Y-1)+(2m-1)\deg(R),
        \end{align*}
    which can be re-arranged as
        \begin{equation*}
        (2m-1)(2n-2)(g_Y-1)=2\deg\left(\left\lfloor\frac{m\pi_*(R)}{n}\right\rfloor\right)-(2m-1)\deg(R).
        \end{equation*}
    So if we can show that the right hand side of this equation is negative then we will have $g_Y-1<0$ and hence $g_Y=0$, as desired.
    
    Using the same notation as in the case when $m=1$, we calculate:
        \begin{align*}
        2\deg\left(\left\lfloor\frac{m\pi_*(R)}{n}\right\rfloor\right)-(2m-1)\deg(R) & = \sum_{Q \in Y} \left(2\left\lfloor m\cdot \frac{n}{e_Q}\frac{\delta_Q}{n}\right\rfloor -n(2m-1)\frac{\delta_Q}{e_Q}\right) \\
        & \leq   \sum_{Q\in Y}\left( 2m\cdot\frac{\delta_Q}{e_Q}-n(2m-1)\frac{\delta_Q}{e_Q}\right) \\
        & =  (2m-n(2m-1))\sum_{Q\in Y }\frac{\delta_Q}{e_Q}.
        \end{align*}
    
    Now as $n,m\geq 2$ then we have $2m-n(2m-1)\leq 2m-2(2m-1)=2(1-m)<0$ and we are done as $\sum_{Q\in Y}\frac{\delta_Q}{e_Q}$ is positive.
    
    So we have shown for all $m\geq 1$, if the group $G$ acts trivially  on $H^0(X,\Omega_X^{\otimes m})$ then $g_Y=0$.
    Now if $m\geq 2$ then first note that $G$ must contain a cyclic subgroup of prime order, say $H$, such that $H$ acts trivially on $H^0(X,\Omega_X^{\otimes m})$.
    Now Proposition \ref{triv} tells us that $m=g_X=2$, and that the order of $H$ must also be 2.
    Hence $X/H\cong \mathbb{P}_k^1$, and this completes the only if direction for $m\geq 2$.
    
    Similarly, the $m=1$ case of the only if direction will follow from Proposition \ref{m=1} after we show that $p>0$ and there is a cyclic subgroup of $G$ of order $p$. 
    This is true since $\pi$ cannot be tamely ramified.
    Indeed, if it were then $R=\sum_{P\in X} (e_P-1)[P]$ \cite[Chap. IV, Cor. 2.4]{hart}, and $\deg\left\lfloor \frac{\pi_*(R)}{n} \right\rfloor=0$, which we have already shown cannot be the case.
    Hence $p$ must be positive, and there is a cyclic subgroup of order $p$ which acts trivially.
    \end{proof}

    \begin{rem}
    Note that the existence of a hyperelliptic involution $\sigma$ in $G$ means not only that the genus of $X/\langle \sigma \rangle$, but also the genus of $Y=X/G$, is $0$ (by the Riemann-Hurwitz formula).
    If, moreover $p=2$, then the canonical projection $X\rightarrow X/\langle \sigma \rangle$ is not unramified (again by the Riemann-Hurwitz formula) and hence not tamely ramified; then $\pi$ cannot be tamely ramified either.
    \end{rem}


\section{Examples}
We will now give some examples of a finite group acting on a curve, and the consequent action on the holomorphic poly-differentials. 
We start with some examples in which $G$ acts trivially on $H^0(X,\Omega_X^{\otimes m})$.
We then follow this with the example of hyperelliptic curves, for which we compute an explicit basis of $H^0(X,\Omega_X^{\otimes m})$, allowing us to see when the action is trivial.


\subsection{Trivial Examples}\label{examplessection}


(a) Let $g_X = 0$, \ie $X\cong \mathbb P_k^1$.
Then $\deg(K_X) = -2$ and so $\deg(mK_X) < 0$ for $m~\geq~1$.
Hence $H^0(X,\Omega_X^{\otimes m}) =\{0\}$ by \cite[Lem. 2, pg. 295]{hart}\todo{check citation} and $G$ acts trivially on $H^0(X,\Omega_X^{\otimes m})$ for all $m\geq 1$.

(b) Let $g_X = 1$, \ie $X$ is an elliptic curve.
If $G$ is a finite subgroup of $X(k)$ acting on $X$ by translations, then $G$ leaves invariant any global non-vanishing holomorphic differential $\omega$ and hence $G$ acts trivially on $H^0(X,\Omega_X)$;
since $\omega^{\otimes m}$ is a basis of $H^0(X,\Omega_X^{\otimes m})$ this means that $G$ acts trivially on $H^0(X,\Omega_X^{\otimes m})$ for all $m\geq 1$.

If $p>0$ and $G$ is a $p$-group, then the multiplicative character $G\rightarrow k^*$ afforded by the one-dimensional representation $H^0(X,\Omega_X^{\otimes m})$ of $G$ has to be trivial because $k$ doesn't contain any $p^{\mbox{th}}$ roots of unity;
in particular the action of $G$ on $H^0(X,\Omega_X^{\otimes m})$ is trivial as well.
On the other hand, if $p\neq 2$ and $X$ is given by the Weierstrass equation of the form $y^2 = f(x)$, then the involution $\sigma : (x,y) \rightarrow (x,-y)$ maps the invariant differential $\omega = \frac{dx}{y}$ to $-\omega$.





% !TEX TS-program = pdflatex
% !TEX encoding = UTF-8 Unicode

% This is a simple template for a LaTeX document using the "article" class.
% See "book", "report", "letter" for other types of document.

\documentclass[draft, 11pt]{article} % use larger type; default would be 10pt

\usepackage[utf8]{inputenc} % set input encoding (not needed with XeLaTeX)

%%% Examples of Article customizations
% These packages are optional, depending whether you want the features they provide.
% See the LaTeX Companion or other references for full information.

%%% PAGE DIMENSIONS
\usepackage{geometry} % to change the page dimensions
\geometry{a4paper} % or letterpaper (US) or a5paper or....
% \geometry{margins=2in} % for example, change the margins to 2 inches all round
% \geometry{landscape} % set up the page for landscape
%   read geometry.pdf for detailed page layout information

\usepackage{graphicx} % support the \includegraphics command and options
\usepackage{todonotes}

\usepackage[parfill]{parskip} % Activate to begin paragraphs with an empty line rather than an indent

%%% PACKAGES
\usepackage{mathtools}
\usepackage{booktabs} % for much better looking tables
\usepackage{array} % for better arrays (eg matrices) in maths
\usepackage{paralist} % very flexible & customisable lists (eg. enumerate/itemize, etc.)
\usepackage{verbatim} % adds environment for commenting out blocks of text & for better verbatim
\usepackage{subfig} % make it possible to include more than one captioned figure/table in a single float
% These packages are all incorporated in the memoir class to one degree or another...

%%% HEADERS & FOOTERS
\usepackage{fancyhdr} % This should be set AFTER setting up the page geometry
\pagestyle{fancy} % options: empty , plain , fancy
\renewcommand{\headrulewidth}{0pt} % customise the layout...
\lhead{}\chead{}\rhead{}
\lfoot{}\cfoot{\thepage}\rfoot{}

%%% SECTION TITLE APPEARANCE
\usepackage{sectsty}
\allsectionsfont{\sffamily\mdseries\upshape} % (See the fntguide.pdf for font help)
\usepackage{amsmath}
\usepackage{amsthm}
\usepackage{amsfonts}
\usepackage{mathrsfs}
\usepackage{amsopn}
\usepackage{amssymb}
\usepackage{natbib}
% (This matches ConTeXt defaults)

%%% ToC (table of contents) APPEARANCE
\usepackage[nottoc,notlof,notlot]{tocbibind} % Put the bibliography in the ToC
\usepackage[titles,subfigure]{tocloft} % Alter the style of the Table of Contents
\renewcommand{\cftsecfont}{\rmfamily\mdseries\upshape}
\renewcommand{\cftsecpagefont}{\rmfamily\mdseries\upshape} % No bold!

%Theorems and stuff
\theoremstyle{plain}
\newtheorem{defn}{Definition}[section]
\newtheorem{thm}[defn]{Theorem}
\newtheorem{cor}[defn]{Corollary}
\newtheorem{lem}[defn]{Lemma}
\newtheorem{prop}[defn]{Proposition}
\newtheorem{ex}[defn]{Example}
\newtheorem*{unnumthm}{Theorem}
\newtheorem{defnlem}[defn]{Definition/Lemma}
\newtheorem{defnthm}[defn]{Theorem/Definition}
\theoremstyle{remark}
\newtheorem*{rem}{Remark}


\newcommand{\cO}{{\cal O}}
\newcommand{\ra}{\rightarrow}
\newcommand{\NN}{{\mathbb N}}
\newcommand{\PP}{{\mathbb P}}
\newcommand{\ZZ}{{\mathbb Z}}
\newcommand{\cL}{{\mathcal L}}
\newcommand{\cA}{{\mathcal A}}
\newcommand{\cD}{{\mathcal D}}
\newcommand{\cU}{{\mathcal U}}


\DeclareMathOperator{\aut}{Aut}
\DeclareMathOperator{\res}{Res}
\DeclareMathOperator{\ord}{ord}
\DeclareMathOperator{\di}{div}
\DeclareMathOperator{\cha}{char}
\DeclareMathOperator{\gal}{Gal}
\DeclareMathOperator{\Tr}{Tr}

%%% END Article customizations

%%% The "real" document content comes below...

\title{}
\author{}
%\date{} % Activate to display a given date or no date (if empty),
         % otherwise the current date is printed 

\begin{document}
\maketitle

\section{Characteristic not 2}


Let $X$ be a smooth, projective, connected hyperelliptic curve of genus $g \geq 2$ over an algebraically closed field $k$ of characteristic $p$.

We wish to compute a basis for the first de-Rham hypercohomology of this curve, which we denote by \todo{define this}{$H^1_{dR}(X/k)$.} 
To do this we use the fact that we have the following short exact sequence:
\begin{equation}\label{ses}
0 \ra H^0(X,\Omega_X) \ra H^1_{dR}(X/k) \ra H^1(X,\cO_X) \ra 0.
\end{equation}

We use \v{C}ech cohomology for our computations.
We have an affine open cover by $U_1 = X\backslash P_0$ and $U_2 = X \backslash P_\infty$, which we denote by ${\cal U} = \{ U_1, \ U_2\}$.
\todo{find citation}{By definition}  $\check{H}_{dR}^1(\cU)$ is the quotient of the $k$-vector space 
\[
\{(\omega_1, \omega_2, f_{12}) | \omega_i\in \Omega_{X/k}(U_i), f_{12}\in \cO_X(U_1 \cap U_2), df_{12} = \omega_1 - \omega_2\}
\]

by the subspace
\[
\{ (df_1, df_2, f_1-f_2)|f_i \in \cO_X(U_i)\}.
\]

In order to compute this basis we will first give a basis of $H^1(X,\cO_X)$(a basis of $H^0(X,\Omega_X)$ is already available in the literature, for example \todo{insert reference}). This will depend upon the defining equations for the associated field extension of $K(x)$ that we choose, and this in turn will depend on whether $p=2$ or not.

If $p \neq 2$ then our extension of $k(x)$ will be $k(x,y)$ where $y$ satisfies 
\[
	y^2 = f(x)
\]
for some polynomial $f \in k[x]$ of degree $2g+1$ or $2g+2$ and with no repeated roots.
By an autopmorphism of $\mathbb P_k^1$ we may assume that $\deg(f) = 2g+1$ and moreover that $0$ and $\infty$ are branch points.
We will denote the pre-images of $0$ and $\infty$ under $x$ by $P_0$ and $P_\infty$ respectively.

We recall that $\di (R) = \di(y) + (g+1)[P_\infty]$ and that $\di( dx) = R - 4[P_\infty]$.

On the other hand, if $p=2$, then our extension will be $k(x,y)$, this time with $y$ defined by
\[
	y^2 - H(x)y = F(x),
\]
where $H(x)$ and $H'(x)^2F(x) + F'(x)^2$ share no roots.

\todo[inline]{note if I need to add other relations here}
We recall the following divisors, which we will make use of throughout this article:
\begin{eqnarray*}
\di (dx) & = & R - 4[P_\infty] \\
\di (h(x)) & = & R - 2(g+1)[P_\infty] \\
\di (x^i) & = & 2i[P_0] - 2i[P_\infty]\\
\di (y) & = & \di_0(y) - (2g+1)[P_\infty].
\end{eqnarray*}
Finally, to define the ramification divisor we suppose that \todo{check this notation doesn't clash with anything else}{$H(x) = \prod_{i=1}^d (x-a_i)^{n_i}$} for some $d \leq g$.
Then $a_i \in \mathbb P_k^1$ are the branch points of $x$ and we let $P_i \in X$ be the correspding ramification points of $x$.
Then 
\[
R = \sum_{i=1}^d 2n_i[P_i] - (g+1-d)[P_\infty].
\]


In order to state the basis of $H^1(X,\cO_X)$, as well as to shorten the proof of the following proposition, we define the following polynomial and differentials.
Suppose that $1 \leq i \leq g$
Firstly, when $p\neq 2$ we define 
\[
	s_i(x) := f'(x) - \frac{2if}{x}
\]
and is $p = 2$ we define
\[
	S_i(x) := xF'(x) + xyH'(x) + iyH(x)
\].

We then let $\phi_i$ and $\psi_i$ be all the terms of $s_i(x)$ with degrees greater than $g$ and less than or equal to $g$ respectively.
Similarly, we let $\Phi_i$ and $\Psi_i$ be the monomials of $S_i(x)$ where the degree of the $x$ component is greater than or equal to $i$ and less than $i$ respectively.

We can now give a basis for $H^1(X,\cO_X)$.

\begin{prop}\label{basis}

Let $i=1,\ldots g$. If $p\neq 2$ then a basis of $H^1(X,\cO_X)$ is given by 
\[
	\left( \left( \frac{-\psi}{2yx^i}\right) dx, \left(\frac{\phi}{2yx^i}\right) dx, x^{-i}y\right)
\]

and if $p=2$ then a basis is given by
\[
	\left( \left(\frac{-\Psi}{x^{i+1}h}\right) dx, \left( \frac{\Phi}{x^{i+1}h} \right) dx, x^{-i}y \right).
\]
\end{prop}


\begin{proof}

We start by showing that $x^{-i}y$ is a viable choice for the third component of both sets of bases.

From the earlier description of Serre duality, it is clear that if $\omega_i$ is a basis of $H^0(X, \Omega_X)$ then we require that the image of the residue map, ${\rm Res}(\omega_i, x^{-i}y)$ is non-zero.

If $p \neq 2$ then $\omega_i = \frac{x^{i-1}}{y}dx$ is a basis of $H^0(X, \Omega_X)$. 
Then ${\rm Res}(\omega_i, x^{-i}y) = res(x^{-1}dx)$ should be non-zero on $U_1$ and $U_2$, or equivalently $x^{-1}dx$ should have a pole of order one at both $P_0$ and $P_\infty$ (recall that the differential needs to be regular on $U_1 \cap U_2$).
The divisor of $x^{-1}dx$ is
\begin{eqnarray}
\di (x^{-1}dx) & = & \di(dx) - \di (x) \\
& = & R - 4[P_\infty] - 2[P_0] + 2[P_\infty] \\
& = & (R-[P_0]-[P_\infty]) - [P_0] - [P_\infty].
\end{eqnarray}
Since $R -[P_0]-[P_\infty]$ is a positive divisor with no $[P_0]$ or $[P_\infty]$ terms we see that $x^{-1}dx$ does indeed have a pole of order one at both $P_0$ and $P_\infty$.

If $p=2$ then a basis of $H^0(X,\Omega_X)$ is given by $\omega_i = \frac{x^{i-1}}{h(x)}dx$, and again we compute the divisor of $x^{-i}y\cdot \omega_i = \frac{y}{xh(x)}dx$.
\begin{eqnarray}
\di\left( \frac{y}{xh(x)}dx \right) & = & \di(y) + \di(dx) - \di(x) - \di(h(x)) \\
& = & \di_0(y) -2[P_0] -[P_\infty],
\end{eqnarray}
where $\di_0$ denotes the divisor of zeroes of a function, or the positive part of the divisor.
Now we know that $y$ has a zero of order at most 1 at $P_0$. \todo{justify order at most 1 statement}

Now that we know $xy^{-i}$ is the third part of the triple making the basis, we need to find two differentials, one regular on $U_1$ and the other regular on $U_2$, such that their difference is $d(x^{-i}y)$.

We again treat the cases $p=2$ and $p=\neq 2$ seperately.
We start by rewriting $d(x^{-i}y)$ in terms of $dx$.
We note that
\begin{eqnarray*}
2yx^{-i}d(yx^{-i}) & = & d(yx^{-i})^2 \\
& = & dfx^{-2i} \\
& = & fdx^{-2i} + x^{-2i}df \\
& = & -f\cdot\frac{2i}{x^{2i+1}}dx + \frac{f'}{x^{2i}}dx \\
& = & \frac{1}{x^{2i}}\left( f' - \frac{2if}{x}\right) dx 
\end{eqnarray*}

and hence it follows that 
\begin{equation*}
d \left( yx^{-i} \right) = \frac{1}{2yx^{i}}\left( f' - \frac{2if}{x} \right) dx = \frac{s_i(x)}{2yx^i}dx.
\end{equation*}

We define $\alpha^i_j$ for $1\leq j \leq 2g$, such that
\[
	$s_i(x) = \alpha^i_{2g}x^{2g} + \ldots + \alpha^i_0, 
\]
and then 

\end{proof}


\begin{comment}
We define $\alpha^i_j$ and $\Alpha^i_{j+1}$ for $0 \leq j \leq 2g$, and $\Beta_k^i$ for $1\leq k \leq g$, such that
\[
	$s_i(x) = \alpha^i_{2g}x^{2g} + \ldots + \alpha^i_0 \ {\rm and } \ S_i(x) = \alpha_{2g+1}^ix^{2g+1} + \ldots + \alpha^i_1 x + y(\Beta_g^i x^i + \ldots + \Beta_1^i x).
\] 
\end{comment}


	




\end{document}


\bibliography{biblio}
\bibliographystyle{plain}


\end{document}
