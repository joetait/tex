\section{$K[G]$-module structure of $\hzero$ when $|G|=p$}

In this subsection we give an alternative proof of Lemma \todo{reference}, using a result of Valentini and Madan \cite{valmadan}.
We let $X$ be a smooth, projective, connected algebraic curve curve of genus $g_X$ over an algebraically closed field $k$ of characteristic $p >0$.
We suppose that $G$ is a subgroup of $\aut(X)$ of order $p$.
We then define $Y:= X/G$ to be the quotient of $X$ by the action of $G$.
We let $\pi \colon X \ra Y$ be the quotient map and we let $g_Y$ be the genus of $Y$.
We will use \cite[Thm.\ 1]{valmadan} to compute the $k[G]$-module structure of $\hzero$, and from this we will show when the action of $G$ on $\hzero$ is trivial.

We remark that in \cite{valmadan} it is assumed that $|G|=p^n$ for some $n \in \NN$.
We have simplified this to the case $n=1$, since this will greatly simplify our computations, and since we do not require the general case.

Since $G$ is of prime order it is a cyclic group, and we assume that $\sigma$ is a generator.
There are $p$ unique indecomposable representations of $G$, which are 
    \[
    M_k := k[G]/((\sigma - 1)^k), \qquad k = 1, \ldots, p.
    \]
Note that the elements $e:= \sigma^0, \sigma, \ldots, \sigma^k$ form a $k$-vector space basis of $M_k$.

We let $d_k$ denote the number of times that $M_k$ occurs in the decomposition of the $k[G]$-module $\hzero$ in to irreducible $k[G]$-modules, so that
    \begin{equation}\label{equationdecompositionofhzero}
    \hzero \cong \bigoplus_{k=1}^p \bigoplus_{i=1}^{d_k} M_k.
    \end{equation}

Now if the action of $G$ on $\hzero$ is trivial then the only irreducible submodule of $\hzero$ will be the trivial module $M_1$.
Hence the action of $G$ is trivial if and only if $d_1 = g_X$ and $d_k = 0$ for $k \in \{ 2, \ldots, p \}$.

To show that the latter condition holds we start by defining
    \[
    \Omega_X^i := \{ \omega \in \hzero | (\sigma - 1)^i\omega = 0 \}
    \]
for $0 \leq i \leq p$.
Clearly $\Omega_X^{i+1} \supseteq \Omega_X^i$, and also $\Omega_X^0 = \{0\}$ and $\Omega_X^p = \hzero$.

By the decomposition \eqref{equationdecompositionofhzero} of $\hzero$ we know that
    \[
    \dim_k \left( \Omega_X^i\right) = \sum_{k=1}^p \sum_{i=1}^{d_k} \dim_k \left( \Omega_X^i \cap M_k \right).
    \]
Furthermore, since $e, \sigma, \ldots, \sigma^{k-1}$ forms a $k$-basis of $M_k$ it follows that 
    \[
    \dim_k\{ m \in M_k |(\sigma - 1)^iv = 0 \} = 
        \begin{cases}
        i & \text{if } i < k \\
        k & \text{otherwise.}
        \end{cases}
     \]   
Combining the above we conclude that 
    \[
    \dim_k \Omega_X^i = \sum_{k=1}^{i-1} kd_k + \sum_{k=i}^p id_k
    \]
from which it follows that
    \[
    \dim_k \left( \Omega_X^{i+1}/\Omega_X^i \right) = \sum_{k=i+1}^p d_k.
    \]


We then use this to make $d_k$ the subject of the equation, giving
    \[
    d_k = 
        \begin{cases}
        \dim_k \left( \Omega_X^p/\Omega_X^{p-1} \right) & \text{if } k = p \\
        \dim_k \left( \Omega_X^k/\Omega_X^{k-1} \right) - \dim_k \left( \Omega_X^{k+1}/\Omega_X^k \right) & \text{otherwise.}
        \end{cases}
    \]
We now determine $\dim_k\left( \Omega_X^i/\Omega_X^{i-1} \right)$ using the extension $K(X)/K(Y)$.

Since $G$ is cyclic of order $p$, $K(X)/K(Y)$ is a degree $p$ Artin extension, and hence $K(X) = K(Y)(y)$, where $y$ satisfies
    \[
    y^p - y = a
    \]
for some $a \in K(X)$.
We can and do choose $y$ and $b$ to satisfy the conditions of \cite[App.\ 5, Lem.\ 5.1]{quaddiffequi}, which we state below.

    \begin{lem}
    Let $L/K$ be a totally ramified Galois extension of degree $p$.
    Then there exists an element $y \in L$ with valuation $-m$, where $m$ is coprime to $p$ and positive, such that $y^p - y \in K$ and $L = K(y)$.
    Then if $M$ is the largest integer such that the higher ramification group $G_M$ doesn't vanish, we have $M = m$.
    \end{lem}
\todo[inline]{check this statement}

We let $Q_1, \ldots, Q_s \in Y$ be the branch points of $\pi$, and we let $P_1, \ldots, P_S$ be the corresponding ramifiction points (note that since $|G|$ is prime it follows that there is only one point in $\pi^{-1}(Q_i)$ for $1 \leq i \leq s$.
We let $v_i = v_{P_i}$ denote the valuation at each $Q_i \in Y$.
Now if we let $\alpha_i = -v_i(a) = m$\todo{why are they all equal to $m$?} then we can compute the exponent of the different \todo{check terminology} at $P_i$, which is
    \[
    \delta_i = (p-1)(\alpha_i + 1) = (p-1)(m+1).
    \]\todo{check that this is clear}
Now we set
    \[
    \gamma_{i,k} = \left\lfloor \frac{\delta_i - k \alpha_i}{p} \right\rfloor, \qquad k = 0, \ldots, p-1,
    \]\todo{make clear that and why this is the coefficient of the ramification divisor}
where $\lfloor c \rfloor$ denotes the largest integer less than $c$, for any $c \in \mathbb R$.
Then $\gamma_{i,k}$ is the coefficient of $P_i$ in $R$ the ramification divisor. \todo{give references for this}
We let $\Gamma_k = \sum_{i=1}^s \gamma_{i,k}$.

We now state the main theorem of \cite{valmadan}, making the appropriate substitutions given our assumptions.

    \begin{thm}
    Let $G$ be a cyclic group of automorphisms of $X$ of order $p$.
    Let $Y : = X/G$ be the quotient of $X$ by the action of $G$, with genus $g_Y$.
    The regular representation of $G$ occurs $g_Y$ times in the representation of $G$ on $\hzero$.
    For $k=1, \ldots, p-1$, the indecomposable representation of degree $k$ occurs $\Gamma_{k-1} - \Gamma_k + \alpha_k$ times, with $\alpha_k = -1$ if $k = p-1$, and zero otherwise.
    \end{thm}
    \begin{proof}
    See \cite[Thm.\ 1]{valmadan}.
    \end{proof}

Using the above theorem we now give an alternative proof of Lemma \todo{citation, here and below}.

    \begin{proof}[Proof of Lemma]
    Note that $g_Y= 0$ by assumption.\todo{check this}
    Now suppose that $p >2$ and that $g_X  \geq 2$, and that the action of $G$ on $\hzero$ is trivial.
    
    We first observe that for any $i \in \{1, \ldots, s \}$ we have
        \[
        \gamma_{i, (p-1)} = \left \lfloor \frac{\delta_i - (p-1)\alpha_i}{p} \right \rfloor  = \left \lfloor \frac{(p-1)(\alpha_i + 1) - (p-1)\alpha_i}{p} \right \rfloor  = \left \lfloor \frac{p-1}{p} \right \rfloor = 0.
        \]  
    and hence $\Gamma_{p-1} = \sum_{ i=1}^s \gamma_{i, (p-1)} = 0$.

    Now since we are assuming that the action of $G$ is trivial, it must follow that $d_k = 0$ for all $k \neq 1$.\todo{check this is already mentioned}
    By Theorem \todo{citation} it follows that $\Gamma_{p-2} - \Gamma_{p-1} -1 = 0$, and hence $\Gamma_{p-2} = 1$.
    We can then show inductively that for $2 \leq k \leq p-2$ then $\Gamma_k = 1$, using the relation $\Gamma_{k-1} - \Gamma_k = 0$ from theorem \todo{citation}.

    Finally, we derive a contradicton when $k =1$.
    Since $p | \delta_i$, we can write 
        \begin{equation}\label{equationdifferenceofgamma0andgamma1}
        \Gamma_0  - \Gamma_1 = \sum_{i=1}^s \left( \left \lfloor \frac{\delta_i}{p} \right \rfloor - \left \lfloor \frac{\delta_i - \alpha_i}{p} \right \rfloor \right) = \sum_{i=1}^s -\left\lfloor \frac{-\alpha_i}{p} \right \rfloor.
        \end{equation}
    Since
        \[
        1 = \Gamma_{p-2} = \sum_{i=1}^s \left\lfloor \frac{\alpha_i + p -1}{p} \right \rfloor,
        \]
    and also since the terms of this sum are all non-negative, we see that for one value of $i$ we have $1 \leq \alpha_i=0 \leq p-1$, and for all other $i$ we have $\alpha_i = 0$.
    Without loss of generality we may assume that $ 1 \leq \alpha_1 \leq p-1$.
    Combining this with \eqref{equationdifferenceofgamma0andgamma1} and the action being trivial \todo{is this second part actually needed?} we conclude that
        \[
        1 = - \left \lfloor \frac{\alpha_i}{p} \right \rfloor = \Gamma_0 - \Gamma_1 = g_X.
        \]  \todo{why is this equal to $g_X$?}
    Since we assumed that $g_X \geq 2$ we have a contradiction.
    \end{proof}



































