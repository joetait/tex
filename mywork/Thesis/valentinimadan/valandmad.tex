\section{$K[G]$-module structure of $\hzero$ when $|G|=p$}
Let $X$ be a smooth connected projective algebraic curve over an algebraically closed field $k$ of characteristic $p>0$, on which the cyclic group $G$ of prime order $p$ acts faithfully.
This induces an action of $G$ on $H^0(X,\Omega_X)$, and we will compute when this action is trivial.
This can be done with the paper of Valentini and Madan \cite{valmadan} when the genus of $X$ is at least $2$.

We start by phrasing the question in the language of the paper; rather than the curve $X$ we will refer to the corresponding function field $F$, and we will write $\Omega_F$ instead of $H^0(X,\Omega_X)$.
The subspace of $F$ fixed by $G$ (equivalently, the quotient of $X$ by the action of $G$) will be denoted $E$, and it's space of holomorphic differentials $\Omega_E$.
Finally, we will denote the genus of $E$ and $F$ by $g_E$ and $g_F$ respectively, and assume that $g_E=0$.

Note that the paper assumes $|G|=p^n$ for some $n \geq 1$, so where appropriate the value of $n$ will be substituted with $1$, after specifying that the substitution has been made.
 
Let $\sigma$ be a generator of $G$. 
There are $p$ unique indecomposable representations of $G$, which can be written $K[G]/(\sigma - 1)^k$, for $k\in \{1,\ldots, p\}$.
The trivial action corresponds to $k=1$ and the regular representation corresponds to $k=p$.
For a decomposition of the $G$-representation of $\Omega_F$ in to indecomposable representations, we will denote by $d_k$ the number of times the representation of degree $k$ occurs.
So if $\Omega_F = \oplus_{m=1}^t \Omega_m$ for some $t\leq g_F$ is a decomposition of $\Omega_F$ in to a direct sum of indecomposable submodules $\Omega_m$, then $d_k$ is the number of components of the sum isomorphic to $K[G]/(\sigma -1)^k.$
Note that if the action is trivial then the trivial representation will be the only indecomposable submodule. 
As such, showing that $G$ acts trivially is equivalent to showing that $d_1 = g_F$, and $d_k= 0 $ for $k \geq 1$.
We start by defining the following subspaces of $\Omega_F$,
\begin{equation*}
	\Omega_F^i = \{\omega\in \Omega_F | (\sigma - 1)^i\omega=0\}\ \mbox{ for } 0\leq i \leq p.
\end{equation*}
This is an increasing sequence of subspaces, from $0$ to $\Omega_F$, and we can see that $\dim_k\Omega_F^i = \sum_{m=1}^t\dim(\Omega_F^i \cap \Omega_m)$, where $\Omega_m$ are the same indecomposable submodules as earlier.
Now note that for the indecomposable module $k[G]/(\sigma - 1)^k$ it is true that 
\[
 \dim_k\{v \in k[G]/(\sigma - 1)^k | (\sigma - 1)^iv = 0\} = \left\{ \begin{array}{ll}
                                                                      i & i\leq k \\
\\
								      k & i > k
                                                                     \end{array} \right. \mbox{\textbf{prove this!}}
\]

Therefore $\dim_k\Omega_F^i = \sum _{k=1}^{i-1} kd_k + \sum_{k=i}^p id_k$, and it follows that 
\[
dim_k(\Omega_F^{i+1}/\Omega_F^i) = \sum_{k=i+1}^p d_k.
\]

As we wish to find $d_k$, we rewrite this as 
\begin{eqnarray*}
d_p & = & \dim_k(\Omega_F^p/\Omega_F^{p-1}) \\
d_k & = & \dim_k(\Omega_F^k/\Omega_F^{k-1}) - \dim_k(\Omega_F^{k-1}/\Omega_F^{k-1}).
\end{eqnarray*}
Now to compute $\dim_k(\Omega_F^i/\Omega_F^{i-1})$ we look at the extension $F/E$.


The extension $F/E$ is a degree $p$ Artin extension, so it has generation of the form 
\[
F=E(y), \mbox{	} y^p-y = b, \mbox{	} b\in E.
\]

Note that in the paper, there are $n$ intermediary fields, labelled $E_j$, each of which is an Artin-Schreier extension with equivalently labelled $b_j$ and $y_j$.
The $b$ and $y$ we are using correspond to $b_1$ and $y_1$ respectively, and $F$ and $E$ correspond to $E_1$ and $E_0$ respectively.


We can choose $b$ and $y$ to satisfy the conditions of Lemma 5.1 of appendix 5 of \cite{quaddiffequi}, which states:


\begin{lem}\label{koeck}
Let $L/K$ be a totally ramified Galois extension of degree $p$.
Then there exists an element $y \in L$ whose valuation is coprime to $p$ and
negative, say $-m$, such that $y^p - y \in K$ and $L = K(y)$. The greatest
integer $M$ such that the higher ramification group $G_M$ of $L/K$ does not vanish
is then equal to $m$.
\end{lem}


At this point in the paper, for $1\leq k \leq p^n-1$, $a_i^k$ is defined to be the co-efficient of $p^i$ in the p-adic expansion of $k$. 
Since we have $n=1$, we only have $a_0^k = k$ in this expansion, a fact we will later use.

Let $\bar{P_1},\dots , \bar{P_s}$ be the primes of $E$ that ramify, and let $P(i,1,1)$ denote the prime ideal above $\bar{P_i}$, in $F$.
(Note that we are using this notation to be consistent with the paper. 
There, the authors used $P(i,j,m)$, where $i$ denoted the prime ideal in $E$ it was above, $j$ denoted which extension the ideal was in, and $m$ was used to differentiate between these.
We set $j=1$ as $n=1$, as already commented.
Also, since the order of $G$ is prime, we may only have one prime above each $\bar{P_i}$, so $m=1$.


Now for each $\bar{P_i}$ the normalised valuation determined by $\bar{P_i}$ applied to $z\in E$ is denoted $v(i,1,1,z)$.
(Again, note that in the paper the authors use $v(i,j,m,z)$ to denote the valuation of a $z\in E_j$ determined by $P(i,j,m).$
Here however we will only require the valuation in $E$.)
We let $e_i$ be the ramification index of $\bar{P_i}$ and let $r=n-\max(e_i)$.
Since $n=1$, it is clear that $e_i=1$ for all $i$, unless the extension is unramified, and hence $r=0$.
Again, this is not necessarily the case if $n\geq 2$, and but we will use the notation of the paper.

From the proof of Lemma 2 in Valentini and Madan, if we set $\Phi(i,j) = \Phi(i,1) = -v(i,1,1,b)$, then we have the following formula to determine the exponent of the different at $P(i,1,m)$:

\begin{eqnarray}\label{eq}
\delta_i & = & (p-1)\sum_{j=n-e_i+1}^n (\Phi(i,j) + 1)p^{n-j} \\
	    & = & (p-1)(\Phi(i,1) + 1).
\end{eqnarray}

To determine $d_k$ in terms of this, then we set 
\[
v_{ik} = \left\lfloor \frac{\delta_i - k\Phi(i,1)}{p} \right\rfloor,
\]
for $0\leq k \leq p-1$, where $\lfloor c \rfloor$ denotes the largest integer not exceeding $c$ for any $c\in \mathbb{R}$.

For each $k$ we then denote the sum of these values of all the ramification points by $ \Gamma_k = \sum_{i=1}^s v_{ik}.$

Now we can apply Theorem $1$ in Valentini and Madan, which states:
\begin{thm}
Let $G$ be a cyclic group of automorphisms of $F$ of order $p^n$. 
Let $E$ be the fixed field of $G$ with $g_E$ its genus.
The regular representation of $G$ occurs $g_E-1+\alpha$ times in the representation of $G$ on $\Omega_F$, with $\alpha = 1$ if $r=0$ and $\alpha = 0$ otherwise.
For $k=1,\ldots p^n-1$, the indecomposable representation of degree $k$ occurs $\Gamma_{k-1}-\Gamma_k + \alpha_k$ times, with $\alpha_k = 1$ if $k= p^n-p^r +1$, $\alpha_k = -1$ if $k=p^n-p^r$ and $\alpha_k = 0$ otherwise.
\end{thm}

Now note that $g_E=0$ by assumption, and $r=0$ as commented earlier, so the degree of the regular representation is zero.

Now suppose that $p>2$ and that $g_X\geq 2$.
Suppose also that the action is trivial.


We first observe that $\Gamma_{p-1} = 0$. Indeed
\[
 v_{i(p-1)} = \delta_i - (p-1)\Phi(i,1) = (p-1)(\Phi(i,1) + 1) - (p-1)\Phi(i,1) = p-1,
\]
and $\left \lfloor \frac{p-1}{p} \right\rfloor = 0$. This gives $\Gamma_{p-1} = \sum_{i=1}^s 0 = 0$, as desired.

Now if we were to assume that the action is trivial, then for all $k\neq 1$ the representation of degree $k$ should not occur and it should be true that $d_k = 0$.
So if $k= p-1$ we have $\Gamma_{p-2} - \Gamma_{p-1} - 1 = \Gamma_{p-2} -1 = 0$, and hence $\Gamma_{p-2} = 1$.
Inductively for $2\leq k \leq p-2$, by the relation $\Gamma_{k-1} - \Gamma_k =0$, then $\Gamma_k = 1$.

Finally, we show a contradiction occurs when $k = 1$.
Since $p| \delta_i$, we can write
\begin{equation}\label{round}
 \Gamma_0 - \Gamma_1 = \left\lfloor \frac{\delta_i}{p} \right\rfloor -\left\lfloor \frac{\delta_i - \Phi(i,1)}{p} \right\rfloor = -\left\lfloor \frac{-\Phi(i,1)}{p} \right\rfloor.
\end{equation}

As 
\[
 1 = \Gamma_{p-2} = \sum_{i=1}^s\left\lfloor \frac{\Phi(i,1) + p -1}{p} \right\rfloor,
\]
 and as the sum is made of non-negative terms (by lemma \ref{koeck}, $-\Phi(i,1) \geq 0$) , we see that $p \leq \Phi(i,1) \leq p-1$ for one $i$, and $\Phi(i,1) = 0$ otherwise.
Without loss of generality we can assume the $\Phi(1,i)$ is the only non-zero term.
Combining this with \ref{round} and the action being trivial implies that $1 = -\left\lfloor \frac{-\Phi(1,1)}{p} \right\rfloor = \Gamma_0 - \Gamma_1 = g_F.$
Since $g_F \geq 2$ this is a contradiction and we are done.


\newpage


Putting $k=1$, we get 
\[
\Gamma_0 - \Gamma_1 = \sum_{i=1}^s\left(\left\lfloor \frac{\delta_i}{p} \right\rfloor \left\lfloor \frac{\delta_i - \Phi(i,1)}{p} \right\rfloor \right) = \sum_{i=1}^s - \left\lfloor \frac{\phi(i)}{p} \right\rfloor = g_F.
\]

Now since $\Gamma_{k-1}-\Gamma_k+\alpha_k = \Gamma_{k-1}-\Gamma_k = 0$ for $k=1,\ldots ,p-2$, it follows that $\Gamma_k=g_F$ for all $k\neq p-1$.

We now show that $\Gamma_{p-1} = 0$, which will contradict
\[
 \Gamma_{p-2} - \Gamma_{p-1} - 1 = g_F - \Gamma_{p-1} - 1 = 0.
\]


Indeed, for $k = p-1$, we have by \ref{eq}
\begin{eqnarray*}
v_{ik} & = & \left \lfloor \frac{\delta_i - k\Phi(i,1)}{p} \right\rfloor \\
	& = & \left\lfloor \frac{(p-1)(\Phi(i,1) + 1) - (p-1)\Phi(i,1)}{p} \right\rfloor\\
	& = & 0.
\end{eqnarray*}



