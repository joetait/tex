\chapter{Group actions on algebraic de-Rham cohomology} \label{Chapter:De-Rham cohomology}

Our aim in this chapter is to study the de Rham cohomology $\derhamhone$ of a hyperelliptic curve $X$ as a module over $k[G]$, where $G$ is a subgroup of $\aut(X)$.
In the first section we describe the ordinary cohomology groups $\hone$ and $\derhamhone$ via \cech cohomology.
We can do this particularly elegantly in the case of a hyperelliptic curve $X$, since we can choose a very simple affine cover, via the natural projection any hyperelliptic curve has on to the projective line.
We then use this to prove that the sequence of $k[G]$-modules
    \begin{equation}\label{equationintroductionshortexactsequence}
    0 \ra \hzero \ra \derhamhone \ra \hone \ra 0
    \end{equation}
is exact.
The rest of the chapter will then build towards showing that a particular class of hyperelliptic curves this sequence does not split.

Building on the \cech cohomology computations of the previous section, we then use Serre duality and the fact that we have already computed a $k$ vector space basis of $\hzero$ to compute a basis of $\hone$ (Theorem \ref{theorembasisofhone}), which surprisingly is the same whether $\cha(k) = 2$ or not.
As an application of this we then give a Mittag-Leffler style theorem for hyperelliptic curves.

In the next section we compute a $k$ vector space basis of $\derhamhone$, which features the bases of $\hzero$ and $\hone$ already mentioned, as well as other components.
Unlike the basis of $\hone$, this basis does depend on whether $\cha(k) = 2$ or not.

Using this basis we are able, after some computations, to determine precisely how certain automorphisms act on the de Rham cohomology of $X$.
In particular, we look at automorphisms on $X$ of the form $(x,y) \mapsto (x+a, y)$, for some non-zero $a \in k$.
Then we prove (see Theorem \ref{theoremsplittingtheorem}) that if $G$ contains such an automorphism, and $X$ ramifies above $\infty \in \PP_k^1$, then the short exact sequence \eqref{equationintroductionshortexactsequence} does not split as a sequence of $k[G]$-modules.
It should be noted that such hyperelliptic curves can occur in any genus greater than 1.

After this we give a number of examples to illustrate the details of what happens when the above suppositions are satisfied.
Moreover, we also give an example in which \eqref{equationintroductionshortexactsequence} does split when $\infty \in \PP_k^1$ is not a branch point.


\section{\cech cohomology and de Rham cohomology for {hyperelliptic}~curves}

In addition to the assumptions in Section 1, we now assume that $X$ is hyperelliptic of genus $g \geq 2$.
We recall that a curve is hyperelliptic if there exists a finite, separable morphism of degree two from the curve to $\mathbb P_k^1$.\index{Hyperelliptic curve}
We fix such a map $\pi \colon X \rightarrow \mathbb P_k^1$ of degree two, which is unique up to an automorphism of $\mathbb P_k^1$ (see \cite[Rem.\ 7.4.30]{liu}).
In this section we describe $\hone$ and $H^1(X,\Omega_X)$ concretely for such an $X$, using \cech cohomology.

By Leray's theorem \cite[Thm.\ 5.2.12]{liu} and Serre's affineness criterion \cite[Thm.\ 5.2.23]{liu} we know that, if we use an affine cover, the first \cech cohomology group of $\cO_X$ will be isomorphic to $\hone$.
We define $U_a = X \backslash \pi^{-1}(a)$ for any $a \in \mathbb P_k^1$ and we let ${\cal U}$ be the affine cover of $X$ formed by $U_0$ and $U_\infty$.
Given any sheaf $\cal F$ on $X$ we have the \cech differential $\check{d}\colon {\cal F}(U_0) \times {\cal F} (U_\infty) \rightarrow {\cal F}(U_0 \cap U_\infty)$, defined by $(f_0,f_\infty) \mapsto f_0|_{U_0 \cap U_\infty} - f_\infty|_{U_0 \cap U_\infty}$.
In general we will suppress the notation denoting the restriction map.
Via this differential we have the following cochain complex
    \begin{equation*}
    0 \rightarrow \cO_X(U_0)\times \cO_X(U_\infty) \xrightarrow{\check{d}} \cO_X(U_0 \cap U_\infty) \rightarrow 0.
    \end{equation*}
The first cohomology group of this chain complex is $\cechhone{\cal U} = \frac{\cO_X(U_0 \cap U_\infty)}{\Ima(\check{d})}$ and hence
    \begin{equation}\label{equationcechhoneisomorphismfunctions}
    \hone \cong \frac{\cO_X(U_0 \cap U_\infty)}{\Ima(\check{d})}  
     = \frac{\cO_X(U_0 \cap U_\infty)}{\{f_0 - f_\infty | f_i \in \cO_X(U_i) \}}.
    \end{equation}
When describing elements of $\hone$ using the isomorphism we will denote the residue class of $f \in \cO_X(U_0 \cap U_\infty)$ in the quotient by $[f]$.

If we replace $\cO_X$ by $\Omega_X$ in the previous paragraph then everything still holds, and we conclude that
    \begin{equation}\label{equationcechhoneisomorphismdifferentials}
    H^1(X,\Omega_X) \cong \frac{\Omega_X(U_0 \cap U_\infty)}{\Ima(\check{d})} = \frac{\Omega_X(U_0 \cap U_\infty)}{\{\omega_0 - \omega_ \infty | \omega_i \in \Omega_X(U_i)\}}.
    \end{equation}
Again, we denote the residue class of $\omega \in \Omega_X(U_0 \cap U_\infty)$ in $H^1(X,\Omega_X)$ by $[\omega]$.

We now describe how the trace map acts on $H^1(X,\Omega_X)$ via the presentation \eqref{equationcechhoneisomorphismdifferentials}.
    \begin{lem}\label{lemmatracemaplemma}
    Let $ \omega \in \Omega_X(U_0 \cap U_\infty)$ with residue class $[\omega]$ in $H^1(X,\Omega_X)$.
    Then we have
        \[
        t\left([ \omega ]\right) = \sum_{P \in \pi^{-1}(\infty)}\res_P(\omega).
        \]
    On the right hand side we consider $\omega$ as an element of $\Omega_{K(X)}$ via the canonical injection $\Omega_X(U_0 \cap U_\infty) \hookrightarrow \Omega_{K(X)}$.
    \end{lem}
    \begin{proof}
    We take the \cech complex of \eqref{equationshortexactsequencedifferentialsdirectsum} over the cover $\cU$, yielding the following bicomplex
        \begin{equation}\label{equationdualitycommutativediagramdifferentials}
        \xymatrix{\Omega_X(U_0)\times\Omega_X(U_\infty) \ar@{^{(}->}[r] \ar[d]^{d_1} & \Omega_{K(X)} \times \Omega_{K(X)} \ar[d]^{d_2} \ar@{->>}[r] & \bigoplus \limits_{P \in U_0} \Omega_{K(X)}/\Omega_{X,P} \times \bigoplus \limits_{P \in U_\infty} \Omega_{K(X)}/\Omega_{X,P} \ar[d]^{d_3} \\
        \Omega_X(U_0 \cap U_\infty) \ar@{^{(}->}[r]  & \Omega_{K(X)} \ar@{->>}[r] & \bigoplus \limits_{P\in U_0 \cap U_\infty} \Omega_{K(X)}/\Omega_{X,P} }
        \end{equation}\todo{recall reason that right hand horizontal maps are surjective}
    We can now apply the snake lemma to this diagram, giving a long exact sequence.
    We first note that $d_2$ is clearly surjective --- any $\omega \in \Omega_{K(X)}$ is mapped to by $(\omega, 0) \in \Omega_{K(X)} \times \Omega_{K(X)}$.
    Now recall that $d_3$ is defined by $((\omega_P)_{P \in U_0}, (\omega_P')_{P \in U_\infty}) \mapsto (\omega_P - \omega_P')_{P_ \in U_0 \cap U_\infty}$.
    Then given any element $(\omega_P)_{U_0 \cap U_\infty} \in \bigoplus \Omega_{K(X)}/\Omega_{X,P}$ we can define
        \[
        (\omega_P') := 
            \begin{cases}
            \omega_P & \text{if}\ P \in U_0 \cap U_\infty \\
            0 & \text{if} \ P = \infty.
            \end{cases}
        \]  
    Clearly $d_3((\omega_P')_{P \in U_0}, 0) = (\omega_P)_{P \in U_0 \cap U_\infty}$, and hence $d_3$ is also surjective.
    In particular, the fifth and sixth terms of the long exact sequence are zero.
    We now exhibit isomorphisms between $\ker(d_3)$ and $\operatorname{coker}(d_1)$ and, respectively, the third and fourth terms of \eqref{equationinitialdualitysesdifferentials}.
    The fact that $H^1(X,\Omega_X) \cong {\rm coker}(d_1)$ follows from the above discussion of \cech cohomology.
    To show the isomorphism $\ker(d_3) \cong \bigoplus_{P \in X} \Omega_{K(X)}/\Omega_{X,P}$ we first observe that the kernel of $d_3$ is formed of pairs $ ((\omega_P)_{P \in U_0}, (\omega_P')_{P \in U_\infty})\in \left( \bigoplus_{P \in U_0} \Omega_{K(X)}/\Omega_{X,P} \right) \times \left( \bigoplus_{P \in  U_\infty} \Omega_{K(X)}/\Omega_{X,P} \right)$ such that $\omega_P = \omega_P'$ for every $ P \in U_0 \cap U_\infty$.
    From this it follows that the map 
        \[
        \bigoplus_{P \in X} \Omega_{K(X)}/\Omega_{X,P}\ra\ker(d_3), \quad  (\omega_P)_{P \in X} \to \left( (\omega_P)_{ P \in U_0}, (\omega_P)_{P \in U_\infty} \right)
        \]
    is an isomorphism.
    
    The proof now follows from a diagram chase on \eqref{equationdualitycommutativediagramdifferentials}.
    We start with the residue class $[ \omega ] \in H^0(X,\Omega_X)$ of $\omega \in \Omega_X(U_0 \cap U_\infty)$.
    Then $\omega$ injects in to $\Omega_{K(X)}$, and since $d_2$ is surjective we can choose an element of $\Omega_{K(X)} \times \Omega_{K(X)}$ mapping to $\omega$.
    In particular, we choose $(\omega,0)$.
    This then maps to 
        \[
        \psi = (({\omega}_P)_{P\in U_0}, 0) \in \left( \bigoplus_{P \in U_0} \Omega_{K(X)}/\Omega_{X,P}\right) \times \left( \bigoplus_{P \in U_\infty} \Omega_{K(X)}/\Omega_{X,P} \right).
        \]
    By commutativity of the diagram $\psi \in \ker(d_3) \cong \sum_{P \in X}\Omega_{K(X)}/\Omega_{X,P}$.
    This means that $\omega_P$, and hence $\psi$, is zero for any $P \in U_0 \cap U_\infty$.
    Since $\psi$ is also zero for $P \in \pi^{-1}(0)$ it follows that 
        \[
        t\left([ \omega ] \right) = \sum_{P \in X}\res_P(\psi) = \sum_{P \in \pi^{-1}(\infty)} \res_P(\omega).
        \]
    \end{proof}

We now recall how to compute the algebraic de Rham cohomology of $X$ via \cech cohomology.
Since $X$ is a curve any differentials of degree greater than one on $X$ are zero.
Hence the de Rham complex of $X$ is \index{de Rham complex}
    \begin{equation}\label{equationderhamcomplex}
    0 \rightarrow \cO_X \xrightarrow{d} \Omega_X \rightarrow 0.
    \end{equation}
Here $d$ denotes the differential map $f \mapsto df$, as defined in \cite[Chap.\ II, Pg.\ 172]{hart}.
We then recall from \cite[Pg.\ 351]{grothendiecklettertoatiyah} that the algebraic de Rham cohomology of $X$ is defined to be the hypercohomology of \eqref{equationderhamcomplex}.\index{de Rham cohomology}\index{Algebraic de Rham cohomology|see{de Rham cohomology}}

We use the cover $\cal U$ and the \cech differentials defined earlier to give us the \cech bicomplex of \eqref{equationderhamcomplex}, which is
    \begin{equation}\label{equationcechbicomplex}
    \xymatrix{ & 0 \ar[d] & 0 \ar[d] & \\
    0 \ar[r] & \cO_X(U_0) \times \cO_X(U_\infty) \ar[d] \ar[r] & \Omega_X(U_0) \times \Omega_X(U_\infty) \ar[d] \ar[r] & 0 \\
    0 \ar[r] & \cO_X(U_0\cap U_\infty) \ar[d] \ar[r] & \Omega_X(U_0 \cap U_\infty) \ar[r] \ar[d] & 0 \\
    & 0 & 0 &}
    \end{equation}
By a generalisation of Leray's theorem \cite[Cor.\ 12.4.7]{EGA0III} we know that the $\derhamhone$ is isomorphic to the first cohomology of the total complex of \eqref{equationcechbicomplex}.
Note that this requires ${\check H}^p(U_\sigma, \cO_X)$ and ${\check H}^p(U_\sigma, \Omega_X)$ to be zero for any $\sigma$ in the nerve of $\cU$ and any $p \geq 1$ --- since $U_0$ and $U_\infty$ are affine, this follows from Serre's affineness criterion \cite[Thm.\ 5.2.23]{liu}.

Therefore $\derhamhone$ is isomorphic to the space
    \begin{equation}\label{equationderhamspace}
    \left\{(\omega_0, \omega_\infty, f_{0,\infty}) | \omega_i\in \Omega_X(U_i), f_{0,\infty}\in \cO_X(U_0 \cap U_\infty), df_{0,\infty} = \omega_0|_{U_0\cap U_\infty} - \omega_\infty|_{U_0\cap U_\infty} \right\}
    \end{equation}
quotiented by the subspace
    \begin{equation}\label{equationderhamquotient}
    \left\{  (df_0, df_\infty, f_0|_{U_0\cap U_\infty} -f_\infty|_{U_0\cap U_\infty} )|f_i \in \cO_X(U_i)\right\}.
    \end{equation}

Via the isomorphism \eqref{equationcechhoneisomorphismfunctions} and the description of $\derhamhone$ above, we can define the maps
    \begin{equation}\label{equationinjectionofhzerointoderham} 
    i \colon \hzero \ra \derhamhone, \qquad [\omega] \mapsto [(\omega, \omega, 0)]
    \end{equation}
and 
    \begin{equation}\label{equationsurjectionofderhamontohone}
    p \colon \derhamhone \ra \hone, \qquad [(\omega_0, \omega_\infty, f_{0 \infty})] \mapsto [f_{0 \infty}].
    \end{equation}
The following lemma shows that $\derhamhone$ fits in to a short exact sequence with $\hzero$ and $\hone$.
    
    \begin{prop}\label{propshortexactsequence}
    The following sequence is exact:
        \begin{equation*}
        0 \ra H^0(X,\Omega_X) \xrightarrow{i} \derhamhone \xrightarrow{p} H^1(X,\cO_X) \ra 0.
        \end{equation*}
    \end{prop}
    \begin{proof}
    Let $T$ be the total complex of \eqref{equationcechbicomplex}.
    Moreover, we let $\cO$ and $\Omega$ be the complexes formed from the first and second (non-trivial) columns of \eqref{equationcechbicomplex} respectively.
    Then let $\Omega[1]$ denote the complex obtained from shifting $\Omega$ by one, \ie~$\Omega[1]^{n+1} = \Omega^n$.
    From this we obtain the following short exact sequence of complexes 
        \[
        \Omega[1] \hookrightarrow T \twoheadrightarrow \cO,
        \]
    giving rise to the following long exact sequence
        \begin{align} \label{equationlongexactsequence}
        \begin{split}
        0 \ra & H^0_{\text {dR}}(X/k) \ra   H^0(X,\cO_X) \ra \\ 
        H^0(X,\Omega_X) \ra & \derhamhone \ra   \hone \ra  \\
        H^1(X,\Omega_X) \ra & H^2_{\text {dR}}(X/k) \ra   0, 
        \end{split}
        \end{align} 
    where the maps in the middle line are the maps $i$ \eqref{equationinjectionofhzerointoderham} and $p$ \eqref{equationsurjectionofderhamontohone}.


    The map $H^0(X,\cO_X) \ra \hzero$ is the map $f \mapsto df$.
    Since the only globally holomorphic functions on $X$ are constant functions, it follows that this is the zero map, and hence $\hzero \ra \derhamhone$ is injective.
    
    Since \eqref{equationlongexactsequence} is exact, $p$ is surjective if and only if $\alpha \colon H^1(X,\Omega_X) \ra H^2_{\text {dR}}(X/k)$ is injective.
    Now $H^1(X,\Omega_X)$ is isomorphic to $k$ via the trace map, and if we can show that this isomorphism factors through $\alpha$ it will follow that $\alpha$ is injective.
    Considering the \cech cohomology constructions of $H^1(X,\Omega_X)$ and $H^2_{\text {dR}}(X/k)$, it suffices to show that the trace map is zero on $\Ima \left( d \colon \cO_X(U_0 \cap U_\infty) \ra \Omega_X(U_0 \cap U_\infty) \right)$.
    This follows from Theorem \ref{theoremresiduemap}, which says that given any $f \in K(X)$ then $\res_P(df)=0$ for any $P \in X$, and in particular for any $P \in \pi^{-1}(\infty)$.
    Hence $t\left([df]\right) = 0$ by Lemma \ref{lemmatracemaplemma}.
    So the residue isomorphism factors through $\alpha$, and $p$ is surjective.
    \end{proof}

\section{Basis of $\hone$}

We now give concrete elements in $\cO_X(U_0 \cap U_\infty)$ whose classes in $\hone$, via the isomorphism \eqref{equationcechhoneisomorphismfunctions}, form a basis of $\hone$.
Note in particular that the basis is the same regardless of whether $p=2$ or $p\neq 2$.\todo{not strictly true, since $y$ is defined differently}
We then give a corollary which is of the same style as the Mittag-Leffler theorem \cite[Chap.\ 5, \S 2, Thm.\ 4]{ahlfors}.

    \begin{thm}\label{theorembasisofhone}
    The elements $\frac{y}{x}, \ldots, \frac{y}{x^g} \in K(X)$ are regular on $U_0 \cap U_\infty$, and their residue classes $\left [ \frac{y}{x} \right ],  \ldots, \left [ \frac{y}{x^g} \right]$ form a basis of $\hone$.
    \end{thm}
    \begin{proof}
    We start by considering the case $p \neq 2$ and first check that the functions $\frac{y}{x}, \ldots, \frac{y}{x^g}$ are indeed regular on $U_0 \cap U_\infty$ (as required by \eqref{equationcechhoneisomorphismfunctions}) by computing their divisors.
    From \eqref{equationdivxpis2} and \eqref{equationdivisorofypnot2} we see that
        \begin{equation*}\label{equationdivisorofyoverx}
        \begin{split}
        \di \left( \frac{y}{x^i} \right) & = \di (y) - \di ( x^i) \\
        & = R - (g+1)D_\infty - iD_0 + iD_\infty \\
        & = R - iD_0 - (g+1 - i)D_\infty.
        \end{split}
        \end{equation*}
    Since $R$ is a positive divisor this is non-negative on $U_0 \cap U_\infty$ for all $i \in \ZZ$, and hence in particular for $i\in \{0, \ldots, g-1\}$.
    
    
    Recall that the differentials $ y^{-1}dx, \ldots,  x^{g-1}y^{-1}dx$ form a basis of $\hzero$ (see Proposition \ref{prophyperellipticbasispnot2}).
    By Lemma \ref{lemmatracemaplemma} we know that $\langle x^iy^{-1}dx, yx^{-j} \rangle = \sum_{P \in \pi^{-1}(\infty)}\res_P(x^{i-j}dx)$.
    It follows immediately from Theorem \ref{theoremresiduemap} that $\sum_{P \in \pi^{-1}(\infty)}\res_P(x^{i-j}dx) = -2$ if $i-j=-1$ and is zero otherwise (regardless of whether $\infty$ is a branch point).
    It then follows from Theorem \ref{theoremserreduality} that the residue classes $\left[yx^{-1}\right],\ldots,\left[yx^{-g}\right]$ form a basis of $\hone$.
    
    We now suppose that $p=2$, and again start by checking that for $i \in \{1, \ldots , g\}$ the function $yx^{-i}$ is regular on $U_0 \cap U_\infty$.
    This follows once we compute the divisor of $yx^{-i}$, which is
        \begin{align*}
        \di \left( \frac{y}{x^i} \right)  & =  \di(y) - i\di(x) \\
        & = {\displaystyle \sum_{i=1}^l} m_i[Q_i] -iD_0 -(2g+1 - 2i)[P_\infty]
        \end{align*}
    if $\infty$ is a branch point and
        \begin{align*}
        \di \left( \frac{y}{x^i} \right)  & =  \di(y) - i\di(x) \\  
        & = {\displaystyle \sum_{i=1}^l} m_i[Q_i] - iD_0 +(g+1-\deg(F(x)) + i)[P_\infty] - (g+1-i)[P_\infty']
        \end{align*}
    otherwise.
    These equalities follow from Proposition \ref{propdivisorofypis2} and \eqref{equationdivxpis2}.
    The divisors are clearly positive on $U_0 \cap U_\infty$ for all $i \in \ZZ$, and hence for $ i \in \{1, \ldots, g \}$.
    
    Next we recall from Proposition \ref{propbasishyperellipticp=2}  that if $p=2$ a basis of $\hzero$ is given by $\frac{1}{H(x)}dx, \ldots, \frac{x^{g-1}}{H(x)}dx$.
    We then deduce from Lemma \ref{lemmatracemaplemma} that when $\infty$ is not a branch point
        \[
        \left \langle \frac{x^i}{H(x)}dx, \frac{y}{x^j} \right \rangle = \res_{P_\infty} \left( \frac{yx^{i-j}}{H(x)}dx \right) + \res_{P_\infty'}\left( \frac{yx^{i-j}}{H(x)} dx \right).
        \]
    Then recall that in characteristic two we have an involution $\sigma \colon X \ra X$ given by $(x,y) \mapsto (x, y + H(x))$, and that $\res_P(\sigma^*(\omega)) = \res_{\sigma(P)}(\omega)$ for any $P \in X$ and $\omega\in \hzero$.
    Then it follows that
        \begin{align*}
        \left \langle \frac{x^i}{H(x)}dx, \frac{y}{x^j} \right \rangle & = \res_{P_ \infty} \left( \frac{yx^{i-j}}{H(x)}dx \right) + \res_{P_ \infty} \left( \frac{(y+H(x))x^{i-j}}{H(x)}dx \right) \\
        & = 2\res_{P_\infty}\left( \frac{yx^{i-j}}{H(x)}dx \right) + \res_{P \infty}(x^{i-j}dx) \\
        & = \res_{P_\infty}(x^{i-j}dx),
        \end{align*}
    since we are assuming that $\cha(k) = 2$.
    As in the previous case, it follows from the definition of $\res_P$ that $\res_{P_\infty}(x^{i-j}dx) = -1$ if $i-j = -1$ and is zero otherwise.
    Hence , by Theorem \ref{theoremserreduality}, the residue classes of $\frac{y}{x}|_{U_0 \cap U_\infty}, \ldots, \frac{y}{x^g}|_{U_0 \cap U_\infty}$ form a basis of $\hone$ when $p = 2$ and $\infty$ is not ramified.
    
    
    
    If $P_\infty$ is a branch point then we compute the divisor of $ \frac{y}{x^j} \cdot \frac{x^i}{H(x)}dx$, using \eqref{equationdivxpis2}, \eqref{equationdifferentialdivisor}, \eqref{equationdivisorofH} and Proposition \ref{propdivisorofypis2}:
        \begin{align*}
        &\di\left( \frac{yx^{i-j}}{H(x)}dx \right)  = \di(y) + \di(x^{i-j}) + \di( dx) - \di(H(x)) \\
        & = \sum_{i=1}^l m_i[Q_i] - (2g+ 1 )[P_\infty] + (i-j)D_0 - (i-j)D_\infty + R - 2D_\infty  - R + (g+1)D_\infty\\
        & = \sum_{i=1}^l m_i[Q_i] + (2j-3-2i)[P_\infty] + (i-j)D_0.
        \end{align*}
    We see that there is a pole of order one at $P_\infty$ precisely if $2j - 3 - 2i = -1$, or equivalently if $j = i+1$.
    Hence $\left\langle \frac{x^i}{H(x)}dx, \frac{y}{x^j} \right\rangle = \res_{P_ \infty}\left( \frac{yx^{i-j}}{H(x)}dx\right)  \neq 0$ in this case.\todo[inline]{what is the residue?}
    
    We also check that if $j \neq i+1$ then $\left \langle \frac{x^i}{H(x)}dx, \frac{y}{x^j} \right \rangle = 0$.
    Indeed, if $j-i \geq 2$ then clearly $\frac{yx^{i-j}}{H(x)}dx$ does not have a pole at $P_\infty$.
    On the other hand, if $j-i \leq 0$ then the differential $\frac{yx^{i-j}}{H(x)}dx$ is regular on $U_\infty$, and hence the residue on this set is zero.
    Since $X \backslash U_\infty = \{P_\infty\}$ it follows from the residue theorem (Theorem \ref{theoremresiduetheorem}) that the residue of $\frac{yx^{i-j}}{H(x)}dx$ at $P_\infty$ is also zero, and hence the residue classes of the elements $\left[\frac{y}{x}\right], \ldots, \left[\frac{y}{x^g}\right]$ form a basis of $\hone$, in all cases.
    \end{proof}

%Mittag-Leffler style corollary

We now give a corollary to Theorem \ref{theorembasisofhone}, which is of the same style as the Mittag-Leffler theorem. For a description of the classical Mittag-Leffler problem see \cite[Pgs.\ 180-181]{miranda}.\index{Mittag-Leffler problem}

    \begin{cor}
    For each $P \in X$ we fix $f_P \in K(X)/\cO_{X,P}$, such that $f_P = 0$ for almost all $P \in X$.
    Then there exist unique $\alpha_1, \ldots, \alpha_g \in k$ such that, after replacing $f_P$ by $f_P - \left( \alpha_1 \frac{y}{x} + \ldots + \alpha_g \frac{y}{x^g}\right)$ for $P \in \pi^{-1}(\infty)$, we can find an $f \in K(X)$ which has a Laurent tail of $f_P$ at $P$ for all $P \in X$.
    \end{cor}
    \begin{proof}
    Since $f_P = 0$ for almost all $P \in X$ then $(f_P)_{P \in X} \in \bigoplus_{P \in X} K(X)/\cO_{X,P}$.
    From Lemma \ref{lemmaexactsequenceofhoneandhzero} we have the following exact sequence
        \begin{equation*}
        0 \ra H^0(X,\cO_X) \ra K(X) \ra \bigoplus_{P \in X} K(X)/\cO_{X,P} \rightarrow \hone \ra 0,
        \end{equation*}
    and we let $\delta$ denote the map $\bigoplus_{P \in X} K(X)/\cO_{X,P} \rightarrow \hone$.
    By Theorem \ref{theorembasisofhone} the residue classes $ \gamma_ 1= \left[ \frac{x}{y}\right], \ldots, \gamma_g = \left[\frac{x^g}{y}\right]$ form a basis of $\hone$, and it follows that there exist unique $\alpha_1, \ldots, \alpha_g \in k$ such that
        \[
        \delta\left( (f_P)_{P \in X} \right) - \left( \alpha_1\gamma_1 + \ldots + \alpha_g\gamma_g \right) = 0.
        \]
    We can derive the exact sequence \eqref{equationinitialdualitylesfunctions} by applying the snake lemma to the \cech complex of \eqref{equationshortexactsequencefunctionsdirectsum} over $\cU$, which is
        \begin{equation*}
        \xymatrix{\cO_X(U_0)\times\cO_X(U_\infty) \ar@{^{(}->}[r] \ar[d]^{d_1} & K(X) \times K(X) \ar[d]^{d_2} \ar@{->>}[r] & \bigoplus \limits_{P \in U_0} K(X)/\cO_{X,P} \times \bigoplus \limits_{P \in U_\infty} K(X)/\cO_{X,P} \ar[d]^{d_3} \\
        \cO_X(U_0 \cap U_\infty) \ar@{^{(}->}[r]  & K(X) \ar@{->>}[r] & \bigoplus \limits_{P\in U_0 \cap U_\infty} K(X)/\cO_{X,P} }
        \end{equation*}
        where the rows ar exact. \todo{recall reason for surjectivity of rows}
    %Analogously to \eqref{equationdualitycommutativediagramdifferentials}, the kernel of $d_3$ is $\bigoplus_{P \in X}K(X)/\cO_{X,P}$ and the cokernel of $d_1$ is $\hone$.
    Now $\delta$ is the differential map $\ker(d_3) = \bigoplus_{P \in X}K(X)/\cO_{X,P} \ra \operatorname{coker}(d_1) = \hone$ in the statement of the snake lemma \cite[Lem.\ 1.3.2]{weibel}.
    Hence we can perform a diagram chase to find the element in $\ker(d_3)$ which maps to $\left( \alpha_1\gamma_1 + \ldots + \alpha_g\gamma_g \right) \in \hone$ via this differential.
    Firstly, it is clear that $\alpha_1\gamma_1 + \ldots + \alpha_g\gamma_g$ pulls back to 
        \begin{equation}\label{immediate}
        \left( \left(\left( \alpha_1\frac{y}{x} + \ldots + \alpha_g\frac{y}{x^g} \right) \right)_{P \in U_0}, 0\right) \in \bigoplus_{P \in U_0}K(X)/\cO_{X,P} \times \bigoplus_{P \in U_\infty}K(X)/\cO_{X,P}.
        \end{equation}
    Since $\alpha_ix^i/y$ is regular on $U_\infty \cap U_0$, then \eqref{immediate} is equal to $\left( (g_P)_{P \in U_0}\, 0\right)$, where
        \[
        g_P =
            \begin{cases}
            \alpha_1\frac{y}{x} + \ldots + \alpha_g\frac{y}{x^g} & \quad \text{if}\ P \in \pi^{-1}(\infty), \\
            0 & \quad \text{else.}
            \end{cases}
        \]
    Clearly $\left( (g_P)_{P \in U_0},0\right) \in \ker (d_3) = \bigoplus_{P \in X} K(X)/\cO_{X,P}$, and $\delta \left( (g_P)_{P \in U_0}, 0 \right) = \gamma_1 + \ldots + \gamma_g$.
    Hence $\delta((f_P)_{P \in X} - (g_P)_{P \in X}) = 0$, and by the exactness of \eqref{equationinitialdualitylesfunctions} it follows that there exists an $f \in K(X)$ which has Laurent tail $f_P - g_P$ at each $P \in X$, as required in the statement of the corollary.
    \end{proof}

\section{Basis of $H^1_{\rm dR}(X/k)$}

In order to state a basis of $\derhamhone$, as well as to shorten the proof of the following theorem, we define the following polynomials. 
We suppose that $1 \leq i \leq g$.
Then when $p\neq 2$ we define
    \[
    s_i(x) := xf'(x) - 2if(x) \in k[x]
    \]
and when $p = 2$ we define
    \begin{equation}\label{equationSi}
    S_i(x,y) := xF'(x) + y(xH'(x) + iH(x))\in k[x]\oplus yk[x] \subseteq k(x,y).
    \end{equation}

We now decompose these polynomials into two parts, which will be used in the sequel.
Firstly, we write $s_i(x)$ as $s_i(x) = \phi_i(x) + \psi_i(x)$, where $\psi_i(x), \phi_i(x) \in k[x]$ are the unique polynomials such that the degree of $\psi_i (x)$ is at most $g+1$ and $x^{g+2}$ divides $\phi_i(x)$.
Secondly, we define $A_{j,i} \in k$ for $1 \leq j \leq 2g+2$, and $B_{k,i} \in k$ for $0\leq k \leq g+1$ by the equation
    \[
    S_i(x,y) = A_{2g+2,i}x^{2g+2} + \ldots + A_{1,i} x + y(B_{g+1,i} x^{g+1} + \ldots + B_{1,i} x + B_{0,i}).
    \]
Note that many of these coefficients may be zero.
In particular we remark that the $x^i$ term of $xH'(x) + iH(x)$ is always zero, since $B_{i,i}x^i = x \cdot iB_ix^{i-1} + iB_i x^i = 2iB_ix^i = 0$.
We now define the following polynomials:
    \begin{equation}\label{equationsplittingphiandpsi}
    \begin{split}
    \Phi_i^x(x) & =  A_{2g+2, i}x^{2g+2} + \ldots + A_{i+1, i}x^{i+1}, \\
    \Psi_i^x(x) & =  A_{i,i}x^i + \ldots + A_{1,i}x, \\
    \Phi_i^y(x) & =  B_{g,i}x^g + \ldots B_{i+1,i}x^{i+1}, \\
    \Psi_i^y(x) & =  B_{i-1,i}x^{i-1} + \ldots + B_{1,i}x + B_{0,i}.
    \end{split}
    \end{equation}
Finally, we define $\Phi_i(x,y) = \Phi_i^x(x) + y \Phi^y_i(x)$ and $\Psi_i(x,y) = \Psi_i^x(x) + y \Psi_i^y(x)$, so that $S_i(x,y) = \Phi_i(x,y) + \Psi_i(x,y)$.

Viewing $\derhamhone$ as the quotient of \eqref{equationderhamspace} by \eqref{equationderhamquotient}, we now give a $k$-vector space basis of $\derhamhone$.

    \begin{thm}\label{theorembasisofderham}
    If $p \neq 2$ then the residue classes 
        \begin{equation}\label{equationhonebasiselementofderhampnot2}
         \left[ \left( \left( \frac{\psi_i(x)}{2yx^{i+1}}\right) dx, \left(\frac{-\phi_i(x)}{2yx^{i+1}}\right) dx, x^{-i}y \right)\right] , i=1, \ldots ,g,
        \end{equation}
    along with the residue classes 
        \begin{equation}\label{equationhzerobasiselementofderhampnot2}
         \left[ \left( \frac{x^{i}}{y} dx , \frac{x^{i}}{y} dx, 0 \right)\right] , i = 0,\ldots ,g-1,
        \end{equation}
    form a $k$-basis of $\derhamhone$.
    
    On the other hand, if $p=2$ then the residue classes 
        \begin{equation}\label{equationhonebasiselementofderhampis2}
        \left[ \left( \left(\frac{\Psi_i(x,y)}{x^{i+1}H(x)}\right) dx, \left( \frac{\Phi_i(x,y)}{x^{i+1}H(x)} \right) dx, x^{-i}y \right)\right], i =1, \ldots , g,
        \end{equation}
    together with the residue classes 
        \begin{equation}\label{equationhzerobasiselementofderhampis2}
        \left[ \left( \frac{x^{i}}{H(x)} dx, \frac{x^{i}}{H(x)} dx, 0 \right)\right], i=0, \ldots, g-1,
        \end{equation}
    form a $k$-basis of $\derhamhone$.
    \end{thm}

Before proving this theorem we use it to prove the following corollary.

    \begin{cor}\label{corfaithfulactiononderhamhone}
    Let $G$ be a subgroup of the automorphism group $\aut(X)$.
    Then the action of $G$ on $\derhamhone$ is faithful unless $G$ contains a hyperelliptic involution and $p=2$, in which case the action of the hyperelliptic involution is trivial.
    \end{cor}

    \begin{proof}
    Recall from Proposition \ref{propshortexactsequence} that $H^0(X,\Omega_X)$ injects into $\derhamhone$.
    Then if $p \neq 2$ or $G$ does not contain a hyperelliptic involution it follows from Theorem \ref{theoremfaithfulaction} that $G$ acts faithfully on $H^0(X,\Omega_X)$, and hence $G$ acts faithfully on $\derhamhone$.
    
    We now suppose that $p=2$ and that $G$ contains a hyperelliptic involution, which we denote by $\sigma$.
    Again by Theorem \ref{theoremfaithfulaction}, we know that $\sigma$ acts trivially on $\hzero$.
    
    Since $\hzero$ is dual to $\hone$ then $\sigma$ also acts trivially on $\hone$.
    We can study exactly why this is from the view of \cech cohomology, and this will also help to determine the action of $\sigma$ on $\derhamhone$.
    If we fix a natural number $i\in \{1, \ldots ,g\}$ then $\sigma$ maps $\frac{y}{x^i}$ to $\frac{y}{x^i} + \frac{H(x)}{x^i}$. 
    Now we can write the rational function $\frac{H(x)}{x^i}$ as follows, 
        \begin{equation*}
        \frac{H(x)}{x^i} =  \frac{B_{i-1}x^{i-1} + \ldots + B_1x + B_0}{x^i} - \left( - \frac{x^{d_H} + B_{d_H-1}x^{d_H-1} + \ldots + B_ix^i}{x^i} \right),
        \end{equation*}
    where $B_j$ and $d_H$ are as in \eqref{equationcapitalh}.
    Since this is clearly the difference of an element of $\cO_X(U_0)$ and an element of $\cO_X(U_\infty)$ we see that $\frac{H(x)}{x^i}$ is zero in $\hone$.
    We let 
        \[
        H_{1,i}(x) = B_{i-1}x^{i-1} + \ldots + B_1x + B_0 \quad \text{ and } \quad H_{2,i}(x) = -( x^d + B_{d-1}x^{d-1} + \ldots + B_ix^i).
        \]
    
    We now consider the action of $\sigma$ on the entries in \eqref{equationhonebasiselementofderhampis2}.
    Firstly we see that
        \begin{align*}
        \sigma \left( \frac{-\Psi_i(x,y)}{x^{i+1}H(x)} dx\right) & = \frac{-\sigma(\Psi_i(x,y))}{x^{i+1} H(x)} dx \\
        & = \frac{-\Psi_i(x,y)}{x^{i+1}H(x)}dx + \frac{H(x)(xH_{1,i}'(x) + iH_{1,i}(x))}{x^{i+1}H(x)}dx\\
        & = \frac{-\Psi_i(x,y)}{x^{i+1}H(x)}dx + \frac{xH_{1,i}'(x) + iH_{1,i}(x)}{x^{i+1}}dx \\
        & = \frac{-\Psi_i(x,y)}{x^{i+1}H(x)}dx +  \frac{H_{1,i}'(x)}{x^i}dx + \frac{iH_{1,i}(x)}{x^{i+1}}dx \\
        & = \frac{-\Psi_i(x,y)}{x^{i+1}H(x)}dx +  \frac{1}{x^i}d\left( H_{1,i}(x) \right) + H_{1,i}(x) d \left( \frac{1}{x^i} \right) \\
        & = \frac{-\Psi_i(x,y)}{x^{i+1}H(x)}dx + d\left( \frac{H_{1,i}(x)}{x^i} \right),
        \end{align*}
    where the second equality follows from \eqref{equationSi} and the fact that $\sigma(y) = y + H(x)$.
    
    Similarly we can derive
        \begin{equation*}
        \sigma \left( \frac{\Phi_i(x,y)}{x^{i+1}H(x)} dx \right)  = \frac{\Phi_i(x,y)}{x^{i+1}H(x)} dx + d \left( \frac{H_{2,i}(x)}{x^i} \right).
        \end{equation*}
    Lastly, it is clear that $\sigma(x^{-i}y) = x^{-i}(y+H(x))$.
    
    
    We can now describe exactly how $\sigma$ acts on the elements of \eqref{equationhonebasiselementofderhampis2} using $H_{1,i}(x)$ and $H_{2,i}(x)$:
        \begin{multline*}
        \sigma \left( \left[ \left( \left(\frac{-\Psi_i(x,y)}{x^{i+1}H(x)}\right) dx, \left( \frac{\Phi_i(x,y)}{x^{i+1}H(x)} \right) dx, x^{-i}y \right)\right]\right) = \\
         \left[ \left( \left(\frac{-\Psi_i(x,y)}{x^{i+1}H(x)} \right) dx + d\left(\frac{H_{1,i}(x)}{x^i}\right),  \left( \frac{\Phi_i(x,y)}{x^{i+1}H(x)} \right) dx+ d\left(\frac{H_{2,i}(x)}{x^i} \right), \frac{y+H(x)}{x^i} \right) \right].
        \end{multline*}
    So the action of $\sigma$ on the basis elements in \eqref{equationhonebasiselementofderhampis2} amounts to adding the residue class 
        \[
        \left[ \left( d\left(\frac{H(x)_{1,i}}{x^i}\right), d\left(\frac{H(x)_{2,i}}{x^i}\right), \frac{H(x)}{x^i} \right)\right],
        \]
    which is clearly an element of \eqref{equationderhamquotient} and hence is zero.
    So the action of the involution $\sigma$ on $\derhamhone$ is trivial and hence the action of the group $G$ is not faithful.
    \end{proof}

    \begin{cor}
    Let $p \neq 2$.
    Then the action of the hyperelliptic involution on $\derhamhone$ is equivalent to multiplication by $-1$.
    \end{cor}
    \begin{proof}
    The hyperelliptic involution $\sigma$ acts by $(x,y) \mapsto (x,-y)$.
    Hence, if we let
        \[
        \gamma_i = \left[ \left( \left( \frac{\psi_i(x)}{2yx^{i+1}}\right) dx, \left(\frac{-\phi_i(x)}{2yx^{i+1}}\right) dx, x^{-i}y\right) \right],
        \]
    then clearly $\sigma(\gamma_i) = -\gamma_i$.
    Similarly, if 
        \[
        \lambda_i = \left[ \left( \frac{x^i}{y}dx, \frac{x^i}{y}dx, 0 \right) \right]
        \]
    then $\sigma(\lambda_i) = - \lambda_i$.
    Hence $\sigma$ acts by multiplication with $-1$ on $\derhamhone$.
    \end{proof}


We now prove Theorem \ref{theorembasisofderham}.

    \begin{proof}
    We make use of the fact that the short exact sequence in Proposition \ref{propshortexactsequence} splits as a sequence of vector spaces over $k$, and that we know bases of the outer two terms.
    
    It is clear that the elements in \eqref{equationhzerobasiselementofderhampnot2} and \eqref{equationhzerobasiselementofderhampis2} are elements of \eqref{equationderhamspace}. 
    In fact, it follows from Propositions \ref{prophyperellipticbasispnot2} and \ref{propbasishyperellipticp=2} that they are the image of a basis of $H^0(X,\Omega_X)$ in $\derhamhone$.
    
    Moreover, it is obvious that if the elements in \eqref{equationhonebasiselementofderhampnot2} and \eqref{equationhonebasiselementofderhampis2} are well defined elements of the space \eqref{equationderhamspace} then they will map to the basis of $\hone$ given in Theorem \ref{theorembasisofhone}.
    So we need only show that the terms in \eqref{equationhonebasiselementofderhampnot2} and \eqref{equationhonebasiselementofderhampis2} satisfy the conditions stated in \eqref{equationderhamspace}.
    For the rest of the proof we fix $i \in \{1, \ldots ,g\}$.
    
    
    We start with the case $p\neq 2$, and observe that
        \begin{align*}
        \left(  \frac{\psi_i(x)}{2yx^{i+1}}  - \frac{-\phi_i(x)}{2yx^{i+1}} \right) dx & =  \frac{s_i(x)}{2yx^{i+1}} dx \\
        & =  \frac{1}{2yx^i} \left( f(x)' - \frac{2if(x)}{x} \right) dx \\
        & =  \frac{x^i}{2y} \left( \frac{f(x)'}{x^{2i}}dx -\frac{2if(x)}{x^{2i+1}} dx \right) \\
        & =  \frac{x^i}{2y} \left( f(x)d\left(\frac{1}{x^{2i}}\right) + \frac{1}{x^{2i}}df(x) \right) \\
        & =  \frac{x^i}{2y}d(f(x)x^{-2i}) \\
        & =  \frac{x^i}{2y} d\left(\left(yx^{-i}\right)^2\right) \\
        & =  d(yx^{-i}),
        \end{align*}
    with the penultimate line following from the defining equation \eqref{equationdefiningequationpnot2}.
    This shows that the elements in \eqref{equationhonebasiselementofderhampnot2} satisfy $df_{0, \infty} = \omega_0 - \omega_\infty$, one of the conditions of \eqref{equationderhamspace}.
    Since we saw in the proof of Theorem \ref{theorembasisofhone} that $\frac{y}{x^i}$ is regular on $U_0\cap U_\infty$ it only remains to show that $\frac{\phi_i(x)}{2yx^{i+1}}dx$ and $\frac{-\psi_i(x)}{2yx^{i+1}}dx$ are regular on $U_\infty$ and $U_0$ respectively.
    
    
    In order to do this we fix $\alpha_{j,i} \in k$ for $0\leq j \leq 2g+2$ satisfying the equation
        \[
        s_i(x) = \alpha_{2g+2,i}x^{2g+2} + \ldots + \alpha_{0,i},
        \]
    so that
        \begin{align*}
        \phi_i(x)  & = \alpha_{2g+2,i}x^{2g+2} + \ldots + \alpha_{g+2,i}x^{g+2} \\
        \intertext{and }
        \psi_i(x)  & = \alpha_{g+1,i}x^{g+1} + \ldots + \alpha_{0,i}.
        \end{align*}
    Note that it is possible for any of $\alpha_{j,i}$ to be zero. In fact, it is possible for either $\phi_i(x)$ or $\psi_i(x)$ to be zero.
    Whenever they are non-zero we denote their degrees as polynomials in $x$ by $d_\phi$ and $d_\psi$ respectively. From the definition of $\phi_i(x)$ and $\psi_i(x)$ we know that $0 \leq d_\psi \leq g+1$ and $g+1 < d_\phi \leq 2g+2$.
    
    
    We now show that $\frac{-\phi_i(x)}{2yx^{i+1}}dx$ and $\frac{\psi_i(x)}{2yx^{i+1}}dx$ are regular on $U_\infty$ and $U_0$ respectively.
    We may assume that $\phi_i(x)$ and $\psi_i(x)$ are non-zero, since the zero function is regular everywhere.
    
    
    The divisor of $\frac{-\phi_i(x)}{2yx^{i+1}}dx$ is
        \begin{align*}
        \di\left( \frac{-\phi_i(x)}{2yx^{i+1}}dx \right) & =  \di(\phi_i(x)) -\di(y) - \di(x^{i+1}) + \di (dx) \\
        & =  \di(\phi_i(x)) - ( R - (g+1)D_\infty) - ((i+1)D_0 - (i+1)D_\infty) \\
        & \qquad + (R - 2D_\infty) \\
        & =  \left( \di_0\left( \frac{\phi_i(x)}{x^{g+2}}\right) + (g+2)D_0 - d_\phi D_\infty\right) - (i+1)D_0 + (g+i)D_\infty \\
        & \geq  \di_0\left( \frac{\phi_i(x)}{x^{g+2}}\right) + (g+2)D_0 - (2g+2)D_\infty - (i+1)D_0 + (g+i)D_\infty \\
        & =  \di_0\left( \frac{\phi_i(x)}{x^{g+2}} \right) + (i-g-2)D_\infty + (g-i+1)D_0,
        \end{align*}
    where the second equality makes use of \eqref{equationdivxpis2} and \eqref{equationdivisorofypnot2}.
    Since $i \leq g$ the differential $\frac{-\phi_i(x)}{2yx^{i+1}}dx$ is regular on $U_\infty = X\backslash \pi^{-1}(\infty)$.
    
    Similarly the divisor of $\frac{\psi_i(x)}{2yx^{i+1}}dx$ is
        \begin{align*}
        \di \left( \frac{\psi_i(x)}{2yx^{i+1}}dx\right) & =  \di(\psi_i(x)) - \di(y) - \di(x^{i+1}) + \di (dx) \\
        & =  \di (\psi_i(x) ) -(R - (g+1)D_\infty) - ((i+1)D_0 - (i+1)D_\infty) \\ 
        & \qquad + (R -2D_\infty) \\
        & =  \di(\psi_i(x)) + (g+i)D_\infty -(i+1)D_0 \\
        & =  (\di_0(\psi_i(x)) -d_\psi D_\infty) + (g+i)D_\infty -(i+1)D_0 \\
        & \geq \left( \di_0(\psi_i(x)) - (g+1)D_\infty \right) + (g+i)D_\infty -(i+1)D_0 \\
        & =  \di_0(\psi_i(x)) + (i-1)D_\infty - (i+1)D_0.
        \end{align*}
    Again, the second equality uses \eqref{equationdivxpis2} and \eqref{equationdivisorofypnot2}, and since $i\geq 1$ we conclude that $\frac{\psi_i(x)}{2yx^{i+1}}dx$ is regular on $U_0 = X \backslash \pi^{-1}(0)$, completing the $p\neq 2$ case.
    
    
    We now suppose that $p=2$.
    We remind the reader that this allows us to change signs between positive and negative as we wish.
    We see that
        \begin{align*}
        \left( \left( \frac{ \Psi_i(x,y)}{x^{i+1}H} \right) + \left( \frac{\Phi_i(x,y)}{x^{i+1}H} \right) \right) dx & =  \frac{S_i(x,y)}{x^{i+1}H(x)}dx \\
        & =  \left( \frac{F(x)'}{x^iH(x)} + \frac{yH(x)'}{x^iH(x)} + \frac{iy}{x^{i+1}} \right) dx \\
        & =  \frac{1}{x^i}\left( \frac{F(x)' + yH(x)'}{H(x)} \right) dx + \frac{iy}{x^{i+1}}dx \\
        & =  x^{-i}dy + yd \left( x^{-i}\right) \\
        & =  d\left( yx^{-i}\right),
        \end{align*}
    with the fourth equality following from \eqref{equationdivisorofdypis2}.
    We have also already seen in the proof of Theorem \ref{theorembasisofhone} that $\frac{y}{x^i}$ is regular on $U_0 \cap U_\infty$.
    So in order to prove that for $i\in \{1, \ldots, g\}$ the elements of \eqref{equationhonebasiselementofderhampis2} are satisfy the conditions of \eqref{equationderhamspace} it only remains to show that the differentials $\frac{\Phi_i(x,y)}{x^{i+1}H(x)}dx$ and $\frac{\Psi_i(x,y)}{x^{i+1}H(x)}dx$ are regular on $U_\infty$ and $U_0$ respectively.
    We denote the degrees of the polynomials defined in \eqref{equationsplittingphiandpsi} by $d_{\Phi}^x, d_{\Psi}^x, d_{\Phi}^y$ and $d_{\Psi}^y$.
    
    
    By \eqref{equationsplittingphiandpsi} we have $\Phi_i(x,y) = \Phi_i^x(x) + y\Phi_i^y(x)$ and $\Psi_i (x,y)= \Psi_i^x(x) + y\Psi_i^y(x)$, and we will use these splittings to show that $\frac{ \Phi_i(x,y) }{x^{i+1}H(x)}dx$ and $\frac{\Psi_i(x,y) }{x^{i+1}H(x)}dx$ are regular on $U_\infty$ and $U_0$ respectively.
    
    We start by computing the divisor of $\frac{1}{x^{i+1}H(x)}dx$, since it is a common component to all the differentials we will consider.
    This yields
        \begin{align*}
        \di \left( \frac{1}{x^{i+1}H(x)}dx \right) & = \di(dx) - \di (x^{i+1}) - \di (H(x)) \nonumber \\
        & = (R-2D_\infty) - ((i+1)D_0 - (i+1)D_\infty) - (R - (g+1)D_\infty) \nonumber \\
        & = (g+i)D_\infty - (i+1)D_0,
        \end{align*}
    using \eqref{equationdivxpis2}, \eqref{equationdifferentialdivisor}  and \eqref{equationdivisorofH}.
    We now use this along with Proposition \ref{propdivisorofypis2} and the polynomials \eqref{equationsplittingphiandpsi} to complete the proof.
    
    We begin by computing the divisors associated to $\Phi_i(x,y)$.
    Firstly,
        \begin{align*}
        \di \left( \frac{\Phi_i^x(x) }{x^{i+1} H(x)}dx \right)  = &  \di(\Phi_i^x(x)) -(i+1)D_0 + (g+i)D_\infty\\
         = & \left( \di_0(\Phi_i^x(x)) -d_\Phi^xD_\infty\right) -(i+1)D_0 + (g+i)D_\infty\\
         \geq & \di_0(\Phi_i^x(x)) - (2g+2)D_\infty - (i+1)D_0 + (g+i)D_\infty \\
         = &  \di_0(\Phi_i^x(x)) - (i+1)D_0 + (i-2-g)D_\infty \\
         =  & \di_0 \left( \frac{\Phi_i^x(x)}{x^{i+1}} \right) + (i-g-2)D_\infty.
        \end{align*}
    From this we see that the differential $\frac{\Phi_i^x(x)}{x^{i+1}H(x)}dx$ is clearly regular on $U_\infty = X \backslash \pi^{-1}(\infty)$.
    
    We now compute the divisor of the other half of $\frac{\Phi_i(x,y)}{x^{i+1}H(x)}dx$, namely
        \begin{align*}
        \di\left(\frac{y\Phi_i^y(x) dx}{x^{i+1}H(x)} \right)  = & \di(y) + \di(\Phi_i^y(x)) -(i+1)D_0 + (g+i)D_\infty\\
         = & \di(y) + \di_0(\Phi_i^y(x)) - d_\Phi^yD_\infty -(i+1)D_0 + (g+i)D_\infty \\
         \geq & \di(y) + \di_0(\Phi_i^y(x)) - (g+1)D_\infty - (i+1)D_0 + (g+i)D_\infty \\
         = & \di(y) + \di_0\left(\frac{\Phi_i^y(x)}{x^{i+1}} \right) + (i-1)D_\infty.
        \end{align*}
    From Proposition \ref{propdivisorofypis2} we see that $y$ only has poles at points in $\pi^{-1}(\infty)$, and hence $\frac{\Phi_i(x,y) }{x^{i+1}H(x)}dx$ is regular on $U_\infty = X \backslash \pi^{-1}(\infty)$.
    
    Now we complete the same computations on $\Psi_i(x,y)$, starting with $\Psi_i^x(x)$:
        \begin{align*}
        \di\left( \frac{\Psi_i^x(x) }{x^{i+1}H(x)}dx \right)  & =   \di(\Psi_i^x(x))  - (i+1)D_0 + (g+i)D_\infty \\
        & = (\di_0(\Psi_i^x(x)) -d_\Psi^xD_\infty) - (i+1)D_0 + (g+i)D_\infty \\
         & \geq   \di_0(\Psi_i^x(x) ) - iD_\infty - (i+1)D_0 + (g+i)D_\infty \\
         & =   \di_0(\Psi_i^x(x)) - (i+1)D_0 + gD_\infty,
        \end{align*}
    and it is clear that the divisor is positive on $U_0 = X \backslash \pi^{-1}(0)$.
    
    For the other half of the differential we need to consider separate cases.
    If we assume that $\infty$ is branch point then  using Proposition \ref{propdivisorofypis2} we see that
        \begin{align*}
        \di\left(\frac{y\Psi_i^y(x) }{x^{i+1}H(x)}dx \right)  =  & \di_0(y) - (2g+1)[P_\infty] + \di(\Psi_i^y(x)) - (i+1)D_0 + (g+i)[P_\infty] \\
         =  & \di_0(y) + \di(\Psi_i^y(x)) -(i+1)D_0 + (2i -1)[P_\infty] \\
         = &  \di_0(y) + \di_0(\Psi_i^y(x)) - d_\Psi^y[P_\infty] - (i+1)D_0 + (2i-1)[P_\infty] \\
         \geq &  \di_0(y) + \di_0(\Psi_i^y(x)) -(i-1)[P_\infty] -(i+1)D_0 + (2i-1)[P_\infty] \\
         =   &\di_0(y) + \di_0(\Psi_i^y(x)) -(i+1)D_0 + [P_\infty],
        \end{align*}
    which is clearly positive on $U_0$.
    On the other hand, if $\infty$ is not a branch point then we have
        \begin{align*}
        \di\left(\frac{y\Psi_i^y(x) }{x^{i+1}H(x)}dx \right)  =  & \di(y) + \di(\Psi_i^y(x)) - (i+1)D_0 + (g+i)D_\infty \\
        = & \di(y) + \di_0(\Psi_i^y(x)) - (i+1)D_0 + (g+i - d_\Psi^y)D_\infty \\
        \geq & \di(y) + \di_0(\Psi_i^y(x)) - (i+1)D_0 + (g+1)D_\infty. \\
        \end{align*}
    Since we know from Proposition \ref{propdivisorofypis2} that $y$ cannot have a pole of order greater $g+1$ at $P_\infty$ or $P_\infty'$, and only has poles at these points, it follows that the differential $\frac{y\Psi_i^y(x) }{x^{i+1}H(x)}dx$ is regular on $U_0 = X \backslash \pi^{-1}(0)$.
    Thus we have completed the proof.
    
    
    \end{proof}


%%%%%%%%%%%%%%%%%%%%% Section 4 %%%%%%%%%%%%%%%%%%%%

\section{Splitting of the short exact sequence}


We keep the assumptions of the previous section, and we now also assume that $\cha(k) = p \geq 3$.

In the previous section we found a basis for the de Rham cohomology of any hyperelliptic curve using \cech cohomology, with respect to the cover $\cU = \{ U_0 , U_\infty\}$ (Theorem \ref{theorembasisofderham}).
We let $\lambda_i$ and $\gamma_i$ denote the elements of this basis by defining
    \begin{align*}
    \lambda_i  = & \left[ \left( \frac{x^i}{y}dx, \frac{x^i}{y}dx, 0\right) \right] ,\quad i=0, \ldots, g-1 \\
    \intertext{and}
    \gamma_i = & \left[ \left ( \frac{\psi_i(x)}{2yx^{i+1}}dx, \frac{-\phi_i(x)}{2yx^{i+1}}dx, x^{-i}y \right) \right], \quad i=1,\ldots ,g.
    \end{align*}
In this section we further study the covers $\cU' = \{U_a, U_\infty\}$ and $\cU'' = \{U_0, U_a, U_\infty\}$ for some fixed $a \in \mathbb{P}_k^1\backslash \{0, \infty\}$.
Then $\derhamhone$ is isomorphic to the $k$-vector space 
    \begin{multline}\label{equationsixtupleconditions}
    \left\{ (\omega_0, \omega_a, \omega_\infty , f_{0a}, f_{0 \infty},f_{a \infty}) | \omega_i \in \Omega_X(U_i), f_{ij} \in \cO_X(U_i \cap U_j), \right. \\ \left. f_{0a} - f_{0\infty} + f_{a \infty} = 0, df_{ij} = \omega_i - \omega_j \right\}
    \end{multline}
quotiented by the subspace 
    \begin{equation}\label{equationsixtuplequotient}
    \left\{( df_0, df_a df_\infty, f_0- f_a, f_0 - f_\infty, f_a - f_\infty ) | f_i \in \cO_X(U_i)\right\}.
    \end{equation}

We introduce \cech cohomology notation for the different representations of $\derhamhone$ we have used, letting $\cechderhamhone(\cU)$ and $\cechderhamhone(\cU'')$ be the quotient of \eqref{equationderhamspace} by \eqref{equationderhamquotient} and \eqref{equationsixtupleconditions} by \eqref{equationsixtuplequotient} respectively.
Then we have a canonical isomorphism $\rho\colon \cechderhamhone(\cU'') \ra \cechderhamhone(\cU)$, given by the projection
    \begin{equation}\label{equationdefinitionofrho}
    \rho \colon (\omega_0, \omega_a, \omega_\infty , f_{0a}, f_{0 \infty},f_{a \infty}) \mapsto (\omega_0, \omega_\infty , f_{0 \infty}).
    \end{equation}

The next proposition explicitly describes the pre-image of the basis element $\gamma_i$ under $\rho$.
To this end, we define the following polynomials for $1 \leq i \leq g$:
    \[
    r_i(x) : = \sum_{k=0}^{i-1} (-1)^{g-k}\binom{g}{k} a^{g-k} x^k
    \]
and
    \[
    t_i(x) := \sum_{k=i}^{g} (-1)^{g-k}\binom{g}{k} a^{g-k} x^k.
    \]
These split the polynomial $(x-a)^g$ in to two parts.


    \begin{prop}\label{propbasisoftriplecoverderham}
    The pre-image $\rho^{-1}(\gamma_i)$ for $i \in \{1, \ldots, g\}$ is the residue class of
        \begin{multline*}
        \nu_i = \left(\frac{\psi_i(x)}{2yx^{i+1}}dx, \frac{(\psi_i(x)t_i(x) - \phi_i(x)r_i(x))(x-a) + 2if(x)(-1)^{g-i+1}\binom{g}{i} a^{g-i+1}x^i}{2yx^{i+1}(x-a)^{g+1}}dx,\right. \\\left. \frac{-\phi_i(x)}{2yx^{i+1}}dx,  \frac{r_i(x)y}{x^i(x-a)^g}, \frac{y}{x^i},  \frac{t_i(x)y}{x^i(x-a)^g} \right).
        \end{multline*}
    \end{prop}
    \begin{proof}
    In order to be able to refer to the entries in $\nu_i$ more succinctly we let
        \[
        \nu_i = \left( \omega_{0 i}, \omega_{a i}, \omega_{\infty i}, f_{0 a i}, f_{0 \infty i}, f_{a \infty i} \right).
        \]
    First, note that it follows from the proof of Theorem \ref{theorembasisofderham} that $d(f_{0 \infty i}) = \omega_{0 i} - \omega_{\infty i}$, and that $f_{0 \infty i}, \omega_{0 i}$ and $\omega_{\infty i}$ are regular on the appropriate open sets.
    
    Since $r_i(x)+t_i(x)$ is the binary expansion of $(x-a)^g$ then
        \begin{align*}
        f_{0 a i} - f_{0 \infty i}+ f_{a \infty i} & = \frac{r_i(x)y}{x^i(x-a)^g} - \frac{y}{x^i} + \frac{t_i(x)y}{x^i(x-a)^g} \\
        & = \frac{y(r_i(x) + t_i(x) - (x-a)^g)}{x^i(x-a)^g} \\
        & = 0.
        \end{align*}

    We now check that the differentials and functions in $\nu_i$ are regular on the appropriate open sets by computing the relevant divisors.
    Firstly, by \eqref{equationdivxpis2} and \eqref{equationdivisorofypnot2},
        \begin{align*}
        \di \left( f_{0 a i} \right) & = \di \left( \frac{r_i(x)y}{x^i(x-a)^g} \right) \\
        &  = \di(r_i(x)) + \di(y) - i\di(x) - g\di(x-a) \\
        & \geq \di_0(r_i(x)) - (i-1)D_\infty +R - (g+1)D_\infty - iD_0 + iD_\infty - gD_a + gD_\infty \\
        & = \di_0(r_i(x)) +R -iD_0 - gD_a,
        \end{align*}
    which is non-negative on $U_0 \cap U_a$.
    Note that the second and third line are not necessarily equal, since the coefficient of $x^{i-1}$ in $r_1(x)$ may be divisible by $p$, and hence zero in $k$.
    On the other hand, again by \eqref{equationdivxpis2} and \eqref{equationdivisorofypnot2},
        \begin{align*}
        \di \left( f_{a \infty i} \right) & = \di \left( \frac{t_i(x)y}{x^i(x-a)^g} \right) \\
        & = \di\left(\frac{t_i(x)}{x^i}\right) + \di(y) - g\di(x-a) \\
        & = \di_0 \left( \frac{t_i(x)}{x^i} \right) - (g-i)D_\infty +R - (g+1)D_\infty - gD_a + gD_\infty\\
        & = \di_0 \left( \frac{t_i(x)}{x^i} \right) +R - gD_a -(g-i+1)D_\infty,
        \end{align*}
    where the third equality holds because $t_i(x)/x^i$ is regular on $U_\infty$.
    We conclude that $f_{a \infty i}$ is regular on $U_a \cap U_\infty$.
    
    To show that
        \begin{equation}\label{equationexpandomegaai}
        \omega_{a i} =  \frac{(\psi_i(x)t_i(x) - \phi_i(x)r_i(x))(x-a) + 2if(x)(-1)^{g-i+1}\binom{g}{i} a^{g-i+1}x^i}{2yx^{i+1}(x-a)^{g+1}}dx
        \end{equation}
    is regular on $U_a$ we first compute the divisor
        \begin{align*}
        & \di\left( \frac{dx}{2yx^{i+1}(x-a)^{g+1}}\right)  = \di(dx) - \di(y) - (i+1)\di(x) - (g+1)\di(x-a) \\
        & = R - 2D_\infty - R + (g+1)D_\infty - (i+1)D_0 + (i+1)D_\infty  - (g+1)D_a + (g+1)D_\infty \\
        & = (2g+i+1)D_\infty -(i+1)D_0 - (g+1)D_a,
        \end{align*}
    using \eqref{equationdivxpis2}, \eqref{equationdifferentialdivisor} and \eqref{equationdivisorofypnot2}.
    We next show that the numerator of \eqref{equationexpandomegaai},
        \begin{equation}\label{equationnumerator}
        {(\psi_i(x)t_i(x) - \phi_i(x)r_i(x))(x-a) + 2if(x)(-1)^{g-i+1}\binom{g}{i} a^{g-i+1}x^i},
        \end{equation}
    has degree less than $2g+i+2$, from which it follows that \eqref{equationexpandomegaai} doesn't have a pole at the point(s) in $\pi^{-1}(\infty)$.
    The degree of $\psi_i(x)t_i(x)(x-a)$ is at most $2g+2$, which is less than $2g+2+i$ for all $i \geq 1$.
    If $\deg(f(x)) = 2g+1$, then clearly
        \[
        \deg\left( \phi_i(x)r_i(x)(x-a) \right) = \deg(\phi_i) + \deg(r_i(x)) + \deg(x-a) \leq 2g+1 + i-1 +1 = 2g+i+1
        \]
    and
        \[
        \deg \left( 2if(x)(-1)^{g-i+1}\binom{g}{i} a^{g-i+1}x^i \right)  \leq  2g+1+i .
        \]
    Lastly, if $\deg(f(x)) = 2g+2$ then the term of degree $2g+i+2$ in $-\phi_i(x)r_i(x)(x-a)$ is
        \begin{align*}
        -& ((2g+2)a_{2g+2}x^{2g+2}-2ia_{2g+2}x^{2g+2})\left( (-1)^{g-i+1}\binom{g}{i-1}a^{g-i+1}x^i\right) \\
         & \hskip 3em  = 2(-1)^{g-i+2}\left( (g-i+1)\binom{g}{i-1} \right) a_{2g+2}a^{g-i+1}x^{2g+i+2} \\
         & \hskip 3em = 2(-1)^{g-i} \left( \frac{g!}{(i-1)!(g-i)!} \right) a_{2g+2}a^{g-i+1}x^{2g+i+2} \\
         & \hskip 3em = 2i(-1)^{g-i}\binom{g}{i}a_{2g+2}a^{g-i+1}x^{2g+i+2},
        \end{align*}
    which cancels with the term of the same degree in $2if(x)(-1)^{g-i+1}\binom{g}{i}a^{g-i+1}x^i$.
    Since these terms cancel, we again have the that the degree of \eqref{equationnumerator} is at most $2g+i+1$, and \eqref{equationexpandomegaai} has no pole(s) at the point(s) in $\pi^{-1}(\infty)$.
    
    Finally, we show that \eqref{equationnumerator} is divisible by $x^{i+1}$.
    By definition $x^{g+2} | \phi_i(x)$, and since $i \leq g$ it follows that $x^{i+1}|\phi_i(x)r_i(x)(x-a)$.
    On the other hand, the lowest degree terms of $2if(x)(-1)^{g-i+1}\binom{g}{i}a^{g-i+1}x^i$ and $\psi_i(x)t_i(x)(x-a)$ which can be non-zero are, respectively,
        \[
         2ia_0(-1)^{g-i+1}\binom{g}{i}a^{g-i+1}x^i 
        \]
    and
        \[
         (-2ia_0)\left( (-1)^{g-i}\binom{g}{i}a^{g-i}x^i \right)(-a).
        \]
    When adding $\psi_i(x)t_i(x)(x-a)$ and $2if(x)(-1)^{g-i+1}\binom{g}{i}a^{g-i+1}x^i$ these two terms obviously cancel.
    Hence the numerator \eqref{equationnumerator} is divisible by $x^{i+1}$.
    
    
    It only remains to show that $\omega_{a i} = \omega_{0 i} -df_{0 a i}$.
    We begin this by computing $df_{0 a i}$, which is
        \begin{align*}
        df_{0 a i} & = d \left( \frac{y r_i(x)}{x^i(x-a)^g} \right) \\
        & = \frac{r_i(x)}{x^i(x-a)^g}dy + y d\left( \frac{r_i(x)}{x^i(x-a)^g} \right) \\
        & = \frac{f'(x)r_i(x)}{2yx^i(x-a)^g}dx + y\left( \frac{r_i'(x)}{x^i(x-a)^g} -\frac{i r_i(x)}{x^{i+1}(x-a)^g} - \frac{gr_i(x)}{x^i(x-a)^{g+1}}\right) dx \\
        & = \frac{xf'(x)r_i(x)(x-a) + 2f(xr_i'(x)(x-a) - i(x-a)r_i(x) - gxr_i(x))}{2yx^{i+1}(x-a)^{g+1}} dx.
        \end{align*}
    Hence $\omega_{0 i} - df_{0 a i}$ expands to
        \[
        \frac{\psi_i(x)(x-a)^{g+1} - xf'(x)r_i(x)(x-a) - 2f(x)\left(xr_i'(x)(x-a)-i(x-a)r_i(x)-gxr_i(x)\right)}{2yx^{i+1}(x-a)^{g+1}}dx.
        \]
    Now
        \[
        (x-a)^{g+1} = (x-a)^g(x-a)  = (r_i(x) + t_i(x))(x-a)
        \]
    and
        \begin{multline*}
        xf'(x)r_i(x)(x-a) - 2if(x)r_i(x)(x-a) = r_i(x)(x-a)(xf'(x)-2if(x)) \\
        = r_i(x)(x-a)(\psi_i(x) + \phi_i(x)).
        \end{multline*}
    So
        \[
        \psi_i(x)(x-a)^{g+1} - xf'(x)r_i(x)(x-a) + 2if(x)r_i(x)(x-a) = (\psi_i(x)t_i(x) - \phi_i(x) r_i(x))(x-a).
        \]
    
    
    We now compute $(x-a)r_i'(x)-gr_i(x)$.
    First, we note that
        \begin{align*}
        r_i'(x) & = \sum_{k=1}^{i-1} k (-1)^{g-k} \binom{g}{k} a^{g-k} x^{k-1} \\
        & = \sum_{k=0}^{i-2} (k+1) (-1)^{g-k-1} \binom{g}{k+1}a^{g-k-1} x^k.
        \end{align*}
    From this it follows that
        \begin{align*}
        r_i'(x)(x-a) & = x \sum_{k=1}^{i-1} k (-1)^{g-k} \binom{g}{k} a^{g-k} x^{k-1} - a \sum_{k=0}^{i-2} (k+1) (-1)^{g-k-1} \binom{g}{k+1}a^{g-k-1} x^k \\
        & = \sum_{k=1}^{i-1} k (-1)^{g-k} \binom{g}{k} a^{g-k} x^k  + \sum_{k=0}^{i-2} (k+1) (-1)^{g-k} \binom{g}{k+1}a^{g-k} x^k \\
        & = gr_i(x) + (-1)^{g-i+2}i \binom{g}{i}a^{g-i+1}x^{i-1},
        \end{align*}
    since
        \begin{align*}
        k\binom{g}{k} + (k+1)\binom{g}{k+1} & = k \left( \frac{g!}{k!(g-k)!} \right) + (k+1) \left( \frac{g!}{(k+1)!(g-k-1)!} \right) \\
        & = \frac{g!}{(k-1)!(g-k)!} + \frac{g!}{k!(g-k-1)!} \\
        & = \frac{g\cdot g!}{k!(g-k)!} \\
        & = g \binom{g}{k}.
        \end{align*}
    Hence $x(r_i'(x)(x-a)-gr_i(x))= (-1)^{g-i+2}i\binom{g}{i} a^{g-i+1}x^i$.
    
    Combining the above we conclude that
        \[
        \omega_{0 i } - df_{0a i} =  \frac{(\psi_i(x)t_i(x) - \phi_i(x)r_i(x))(x-a) + 2if(x)(-1)^{g-i+1}\binom{g}{i} a^{g-i+1}x^i}{2yx^{i+1}(x-a)^{g+1}}dx= \omega_{a i}.
        \]
    
    Note that the last relation ($df_{a \infty i} = \omega_{a i} - \omega_{\infty i}$) holds, since 
        \[
        df_{a \infty i} = df_{0 \infty i} - df_{0 a i} = \omega_{0 i} - \omega_{\infty i } - \omega_{0 i} + \omega_{a i} = \omega_{a i} - \omega_{\infty i}.
        \]
    \end{proof}
Recall that the hyperelliptic involution $\sigma$ is in the centre of $\aut(X)$ (see \cite[Cor.\ 7.4.31]{liu}).
Then, given any $\tau \in \aut(X)$, we have an induced map $\bar \tau \colon \mathbb{P}_k^1 \ra \mathbb{P}_k^1$, since $\mathbb {P}_k^1$ is the quotient of $X$ by the hyperelliptic involution.
Hence the following diagram commutes 
    \[
    \begin{array}{ccc}
    X & \xrightarrow[\tau] & X \\
    \downarrow\pi & & \downarrow\pi \\
    \mathbb P^1_k & \xrightarrow[\tau] & \mathbb P_k^1
    \end{array}
    \]
since 
%Hence, if $a = \sigma(0)$, the following diagram also commutes:
%    \[
%    \begin{array}{ccc}
%    \derhamhone \cong \cechderhamhone(\cU)  & \xleftarrow{\rho} & \cechderhamhone(\cU'')  \\
%            \sigma^*\downarrow & ~ & \rho'\downarrow  \\
%    \derhamhone \cong \cechderhamhone(\cU)  & \xleftarrow{\sigma^*} & \cechderhamhone(\cU')
%    \end{array}
%    \]
%Here $\rho$ is as defined earlier \eqref{equationdefinitionofrho}, and $\rho'$ is given by
%    \[
%    \rho' \colon 
%    \rho \colon (\omega_0, \omega_a, \omega_\infty , f_{0a}, f_{0 \infty},f_{a \infty}) \mapsto (\omega_a, \omega_\infty , f_{a \infty}).
%    \]

\begin{lem}\label{lemmatauactsbyplusminusoneony}
    Suppose there exists $\tau \in \aut(X)$ such that the induced automorphism $\bar \tau \colon \mathbb P_k^1 \ra \mathbb P_k^1$ is given by $x \mapsto x+ a$ for some $0 \neq a \in k$.
    Then the action of $\tau^*$ on $y$ is given by $\tau^*(y) = y$ or $\tau^*(y) = -y$ and moreover if such an automorphism of $X$ exists, then $p$ divides the degree of $f(x)$.
    \end{lem}
    \begin{proof}
    We first show that $\tau^*(y) = \pm y$.
    Since $y^2 \in k(x)$ then there exist $g_1(x), g_2(x) \in k(x)$ such that 
        \begin{equation*}
        \tau^*(y) = g_1(x)y + g_2(x) \in k(x,y).
        \end{equation*}
    Hence
        \begin{equation}\label{equationexpandingysquared}
        f(x+a) = \tau^*(y^2) = (\tau^*(y))^2 = g_1(x)^2f(x)+2g_1(x)g_2(x)y + g_2(x)^2.
        \end{equation}
    Firstly, note that if neither $g_1(x)$ nor $g_2(x)$ are zero then
        \[
        y = \frac{f(x+a) - g_1(x)^2f(x) - g_2(x)^2}{2g_1(x)g_2(x)},
        \]
    which contradicts the fact that $K(X)$ is a degree two extension of $k(x)$.
    Hence one of $g_1(x)$ or $g_2(x)$ must be zero.
    
    If $g_1(x) = 0$ then $\tau^*$ would not be an automorphism, since $y$ would not be in the image.
    Hence $\tau^*(y) = g_1(x)y$.
    Also, by comparing the degrees in \eqref{equationexpandingysquared} we see that $\deg(g_1(x)) = 0$, and then by comparing coefficients in the same equation we see that $g_1(x)^2 = 1$.
    Hence $\tau^*(y) = \pm y$.

    We now show that $d_f := \deg(f(x))$ is divisible by $p$, and we derived above that $f(x) = f(x+a)$.
    Comparing the terms of degree $d_f-1$ on each side we see that $d_fa - b_{d_f -1} = b_{d_f-1}$ (where $b_{d_f - 1}$ is as in \eqref{equationexpansionoff(x)}).
    It follows that $d_f = 0$ in $k$, and hence $p \mid d_f$.
    \end{proof}

%    \begin{rem}
%    If $\sigma^*(y) = -y$ we can, without loss of generality, replace $\sigma$ by $\sigma \circ j$, where $j$ denotes the hyperelliptic involution.
%    Hence we will assume throughout the rest of the paper that $\sigma^*(y) = y$.
%    \end{rem}

    Recall from Proposition \ref{propshortexactsequence} that we have a canonical short exact sequence
        \begin{equation}\label{equationderhamcohomologyshortexactseqeunce}
        0 \ra \hzero \ra \derhamhone \ra \hone \ra 0.
        \end{equation}
        
    \begin{thm}\label{theoremsplittingtheorem}
    Suppose there exists $\tau \in \aut(X)$ such that the induced automorphism $\bar \tau \colon \mathbb P_k^1 \ra \mathbb P_k^1$ is given by $x \mapsto x+a$ for some $0 \neq a \in k$.
    We let $G = \langle \tau \rangle$ be the subgroup of $\aut(X)$ generated by $\tau$.
    Then the sequence \eqref{equationderhamcohomologyshortexactseqeunce} does not split as a sequence of $k[G]$-modules.
    \end{thm}
    \begin{proof}
    By Lemma \ref{lemmatauactsbyplusminusoneony} we have $\tau^*(y) = y$ or $\tau^*(y) = -y$.
    Without loss of generality we can and will assume that $\tau^*(y) = y$ since, if $\tau^*(y) = -y$, we replace $\tau$ by $\tau \circ \sigma$ (where $\sigma$ is the hyperelliptic involution of $X$).
    Notice that the sequence \eqref{equationderhamcohomologyshortexactseqeunce} splits as a sequence of $K[G]$-modules if and only if it splits as a sequence of $K[\langle \tau \circ \sigma \rangle]$-modules (see remark after Corollary \ref{corfaithfulactiononderhamhone}).\todo{change reference to corollary}
    
    We now suppose that the sequence \eqref{equationderhamcohomologyshortexactseqeunce} does split, and that $s \colon \hone \ra \derhamhone$ is a splitting map.
    Then it follows that for all $\alpha \in \hone$ we have
        \begin{equation}\label{equationcommutativityoftauandsplittingmap}
        s(\tau^*(\alpha)) = \tau^*(s(\alpha)) \in \derhamhone.
        \end{equation}
    We will show that this equality gives rise to a contradiction when $\alpha$ is the residue class $\left[ \frac{y}{x^g}\right]$ in $\hone$ (see \eqref{equationcechhoneisomorphismfunctions} and Theorem \ref{theorembasisofhone}).  
    It will then follow that no splitting map can exist.

    We first compute the action of $\tau^*$ on the residue class $\left[\frac{y}{x^g}\right]$.
    In order to do this we consider the following obvious commutative diagram of isomorphisms:
        \[
        \begin{array}{cccc}
        \hone \cong \cechhone{\cU}  & \xleftarrow{\rho} & \cechhone{\cU''}  \\
                \tau^*\downarrow & ~ & \rho'\downarrow  \\
        \hone \cong \cechhone{\cU}  & \xleftarrow{\tau^*} & \cechhone{\cU'}
        \end{array}
        \]\todo{move arrow left}
    where $\rho$ and $\rho'$ are the canonical projections.
    From Proposition \ref{propbasisoftriplecoverderham} we know that $\rho^{-1} \left(\left[\frac{y}{x^g}\right]\right)$ is the residue class 
        \[
        \left[ \left( \frac{r_g(x)y}{x^g(x-a)^g}, \frac{y}{x^g}, \frac{t_g(x)y}{x^g(x-a)^g} \right)\right] = \left[ \left( \frac{((x-a)^g-x^g)y}{x^g(x-a)^g}, \frac{y}{x^g}, \frac{y}{(x-a)^g} \right) \right] \in \cechhone{\cU''}.
        \]  
    Therefore 
        \begin{align*}
        \tau^* \left( \left[ \frac{y}{x^g} \right] \right)  & = \tau^* \left( \rho' \left(\rho^{-1}\left(\left[\frac{y}{x^g}\right]\right)\right)\right) \\
        & = \tau^*\left( \rho' \left( \left[ \left( \frac{((x-a)^g-x^g)y}{x^g(x-a)^g}, \frac{y}{x^g}, \frac{y}{(x-a)^g} \right) \right] \right) \right) \\
        & = \tau^*\left( \left[ \frac{y}{(x-a)^g} \right] \right) \\
        & = \left[ \frac{y}{x^g} \right],
        \end{align*}
    \ie $\left[ \frac{y}{x^g} \right]$ in $\hone$ is fixed by $\tau^*$.

    Since the canonical projection $\derhamhone \ra \hone$ maps $\gamma_g$ to the residue class $\left[ \frac{y}{x^g} \right]$ it follows that
        \[
        \tau^*(\gamma_g) = \gamma_g + \sum_{i=0}^{g-1} c_i\lambda_i
        \]
    for some $c_0, \ldots, c_{g-1} \in k$.
    On the other hand, we also have
        \[
        s\left( \left[ \frac{y}{x^g} \right] \right)  = \gamma_g + \sum_{i=0}^{g-1}d_i \lambda_i
        \]
    for some $d_0, \ldots, d_{g-1} \in k$.
    Now the action of $\tau^*$ on $\lambda_i$ for $0 \leq i \leq g-1$ is easily seen to be given by
        \begin{align*}
        \tau^*(\lambda_i) & = \tau^*\left( \left[ \left( \frac{x^i}{y}dx, \frac{x^i}{y}dx, 0\right) \right] \right) \\
        & = \left[ \left( \frac{(x+a)^i}{y}dx, \frac{(x+a)^i}{y}dx, 0 \right) \right] \\
        & = \sum_{k=0}^i \binom{i}{k} a^{i-k} \lambda_i.
        \end{align*}
    Then, by \eqref{equationcommutativityoftauandsplittingmap}, it follows that
        \begin{align*}
        \gamma_g + \sum_{i=0}^{g-1} d_i\lambda_i & = s \left( \left[ \frac{y}{x^g} \right] \right) \\
        & = s \left( \tau^*\left( \left[ \frac{y}{x^g} \right] \right) \right) \\
        & = \tau^* \left( \gamma_g + \sum_{i=0}^{g-1} d_i \lambda_i \right) \\
        & = \left( \gamma_g + \sum_{i=0}^{g-1} c_i \lambda_i \right) + \sum_{i=0}^{g-1} d_i \left( \sum_{k = 0}^i \binom{i}{k} a^{i-k} \lambda_i \right).
        \end{align*}
    By comparing coefficients of the basis elements $\lambda_{g-1}$, we see that $c_{g-1} = 0$.
    We now show that we must have $c_{g-1} = a/4$ for the defining equation
        \[
        \tau^*(\gamma_g) = \gamma_g + \sum_{i=0}^{g-1} c_i \lambda_i
        \]
    to hold.
    Since we assumed that $a \neq 0$ this will give us the contradiction we desire.

    To compute $\tau^*(\gamma_g)$ we consider the following commutative diagram of isomorphisms
        \[
        \begin{array}{ccc}
        \derhamhone \cong \cechderhamhone(\cU)  & \xleftarrow{\rho} & \cechderhamhone(\cU'')  \\
                \tau^*\downarrow & ~ & \rho'\downarrow  \\
        \derhamhone \cong \cechderhamhone(\cU)  & \xleftarrow{\tau^*} & \cechderhamhone(\cU')
        \end{array}
        \]\todo{move arrow left}
        where $\rho$ is the canonical projection \eqref{equationdefinitionofrho} and $\rho'$ is given by
            \[
            \rho' \colon (\omega_0, \omega_a, \omega_\infty, f_{0 a}, f_{0 \infty}, f_{a, \infty}) \mapsto (\omega_a, \omega_\infty, f_{a \infty}).
            \]
        %Rewrite the below to just consider the second entry, using notation in comment on next five lines?
%    Recalling the notation
%        \[
%        \nu_g = \left( \omega_{0 g}, \omega_{a g}, \omega_{\infty g}, f_{0 a g}, f_{0 \infty g}, f_{a \infty g} \right).
%        \]
%    used in the proof of Proposition \ref{propbasisoftriplecoverderham}.
    Then 
        \begin{equation}\label{equationactionoftauongamma}
        \begin{split}
        \tau^*(\gamma_g) & = \tau^*(\rho'(\rho^{-1}(\gamma_g))) \\  
        & = \tau^*\left( \left[ \omega_{a g}, \frac{- \phi_g(x)}{2yx^{g+1}} dx, \frac{y}{(x-a)^g} \right) \right] \\
        & = \left[ \left( \tau^*(\omega_{a g}) , \frac{-\phi_g(x+a)}{2y(x+a)^{g+1}}dx, \frac{y}{x^g} \right) \right],
        \end{split}
        \end{equation}
    where $\omega_{a g}$ is the second entry in $\nu_g$, as in the proof of Proposition \ref{propbasisoftriplecoverderham}.
    On the other hand, we have 
        \begin{equation}\label{equationimageofgammaunders}
        \gamma_g + \sum_{i=0}^{g-1} c_i \lambda_i = \left[ \left( \frac{\psi_g(x)}{2yx^{g+1}}dx, \frac{-\phi_g(x)}{2yx^{g+1}}dx, \frac{y}{x^g} \right) \right] \\ + \sum_{i=0}^{g-1} c_i \left[ \left( \frac{x^i}{y}dx, \frac{x^i}{y}dx, 0 \right) \right].
        \end{equation}
    Note that the third entry in both \eqref{equationactionoftauongamma} and \eqref{equationimageofgammaunders} is $\frac{y}{x^g}$.
    Since any element of the form $(\omega_0, \omega_\infty, 0)$ in the subspace \eqref{equationderhamquotient} of the space \eqref{equationderhamspace} is in fact zero, we conclude, by comparing the second entries of \eqref{equationactionoftauongamma} and \eqref{equationimageofgammaunders}, that
        \[
        - \frac{\phi_g(x+a)}{2y(x+a)^g} dx = -\frac{\phi_g(x)}{2yx^{g+1}}dx + \sum_{i = 0}^{g-1} c_i \frac{x^i}{y}dx
        \]
    in $\Omega_{K(X)}$.
    
    Since $dx$ is a basis of $\Omega_{K(X)}$ considered as a $K(X)$-vector space, and as $K(X) = k(x) \oplus y^{-1}k(x)$, the equation above is equivalent to
        \[
        \frac{\phi_g(x+a)}{2(x+a)^{g+1}} = \frac{\phi_g(x)}{2x^{g+1}} - \sum_{i=0}^{g-1} c_i x^i,
        \]
    in $k[x]$, and this, in turn, is equivalent to
        \[
        \phi_g(x+a)x^{g+1} = \phi_g(x)(x+a)^{g+1} - 2(x+a)^{g+1}x^{g+1}\sum_{i=0}^{g-1}c_i x^i,
        \]
    also in $k[x]$.

    Recall that we assumed that the degree of $f(x)$ was odd, and hence by \todo{reference hyperelliptic curve section} the degree is precisely $2g+1$.\todo{haven't actually stated that the degree is odd, just that $p$ divides it}
    The terms of highest degree in $\phi_g(x)$ are the same as the terms of highest degree in 
        \[
        s_g(x) = xf'(x) - 2gf(x) = x^{2g+1} + 0\cdot x^{2g} + \ldots.
        \]
    We therefore obtain
        \begin{multline*}
        \left( (x+a)^{2g+1} + 0 \cdot (x+a)^{2g} + \ldots \right) x^{g+1}  \\ = (x^{2g+1} + 0 \cdot x^{2g} + \ldots )(x+a)^{g+1} - 2(x+a)^{g+1}x^{g+1}(c_{g-1}x^{g-1} + \ldots ),
        \end{multline*}
    and hence
        \[
        (2g+1)ax^{3g+1} = (g+1)ax^{3g+1} - 2c_{g-1}x^{3g+1}.
        \]
    Finally, since $2g + 1= \deg(f(x)) \equiv 0 \mod p$ (by Lemma \ref{lemmatauactsbyplusminusoneony}) then $g = -\frac{1}{2}$ in $k$.
    Hence we obtain
        \[
        c_{g-1} = \frac{((g+1) - (2g+1))a}{2} = \frac{a}{4},
        \]  
    as claimed above.
    This concludes the proof of theorem \ref{theoremsplittingtheorem}.
    \end{proof}

    \begin{ex}
    If $2g+1$ is an odd prime then exists an automorphism $\tau$ of the form $\tau\colon (x,y) \mapsto (x+a,y)$ if and only if $p=2g+1$ and $f(x) = x^p - a^{p-1}x + a_0$, for some $a_0 \in k$.
    In this case we may assume that $a_0 = 0$, since if it doesn't we can apply the automorphism $x \mapsto x+b, y\mapsto y$ to $K(X)$, where $b$ is a root of $f(x)$.
    Moreover, given an equation of the form $y^2 = x^p - a^{p-1}x$, we can apply automorphism of $K(X)$ to replace the coefficient $a^{p-1}$ by $1$, namely $x \mapsto ax, y \mapsto a^{\frac{p}{2}}y$.
    Hence we see that all such curves are isomorphic to those defined by $y^2 = x^p - x$.
    In particular, it follows from \cite[Thm.\ 3.1]{canonicalrepresentation} that the short exact sequence in Proposition \ref{propshortexactsequence} doesn't split when the above conditions hold.
    \end{ex}

We now give the general form of curves with an automorphism $\tau \colon (x,y) \mapsto (x+a, y)$ when $g_X = 4$.

    \begin{ex}
    It is easy to verify that
        \begin{equation*}
        f(x) = x^9 + a_6x^6 + a^2a_6x^4 + a_3x^3 + a^4a_6x^2 + 2(a^8 + a^2a_3)x + a_0
        \end{equation*}
    satisfies the condition $f(x+a) = f(x)$ for any $a, a_6, a_3, a_0 \in k$.
    Now we fix $a_6, a_3, a \in k$, such that both $a$ and either $a_6$ or $a_3 + a^6$ are non-zero.


    Then $f'(x) = a^2a_6x^3 + 2a^4a_6x + 2(a^8 + a^2a_3)$, and we let $\beta_1, \beta_2$ and $\beta_3$ denote the roots of $f'(x)$ (which may not be distinct, and are independent of the choice of $a_0$).
    Then we can define
        \[
        \beta_i' :=\beta_i^9 + a_6\beta_i^6 + a^2a_6\beta_i^4 + a_3\beta_i^3 + a^4a_6\beta_i^2 + 2(a^8 + a^2a_3)\beta_i.
        \]
    We now fix $a_0 \in k \backslash \{-\beta_1', -\beta_2', -\beta_3'\}$, and it is clear that $f'(x)$ and $f(x)$ do not share any roots, and hence $f(x)$ has no repeated roots.


    From this it follows that the equation $y^2  = f(x)$ defines a genus $4$ hyperelliptic curve over $k$, for which the exact sequence in Proposition \ref{propshortexactsequence} does not split, by Theorem \ref{theoremsplittingtheorem}.
    By considering the genus we can see that this is a new example, distinct from those in \cite{canonicalrepresentation}.
    \end{ex}

%We now describe the action of $\sigma^*$ on $\lambda_i$.
%    
%    \begin{lem}
%    For each $i \in \{ 0, \ldots, g-1\}$ then 
%        \[
%        \sigma^*(\lambda_i) = \sum_{k = 0}^i \binom{i}{k}a^{i-k}x^k.
%        \]
%    \end{lem}
%    \begin{proof}
%    Since $\sigma^*$ acts trivially on $y$, it follows that
%        \[
%        \sigma^*\left( \frac{x^i}{y} \right) = \sum_{k=0}^i \binom{i}{k}a^{i-k}\frac{x^k}{y}.
%        \]
%    The statement follows from this.
%    \end{proof}
%
%
%    \begin{thm}
%    Suppose that $\deg(f(x)) = p^n $ for some $n \in \NN$.
%    Then the short exact sequence of $k[G]$-modules
%        \[
%        0 \ra \hzero \ra \derhamhone \ra \hone \ra 0
%        \]
%    does not split.
%    \end{thm}
%    \begin{proof}
%    We suppose that the sequence does split, and that $s \colon \hone \ra \derhamhone$, given by $s \colon \bar\gamma_i \mapsto \gamma_i$, is the splitting map.
%    
%    We now examine the action of $\sigma^*$ on $\gamma_g$ and $\bar \gamma_g$.
%    We first look at the sixth entry in $\nu_g$ from Proposition \ref{propbasisoftriplecoverderham}.
%    This entry is
%        \[
%        \frac{t_g(x)y}{x^g(x-a)^g} = \frac{y}{(x-a)^g},
%        \]
%    and clearly $\sigma^*(y/(x-a)^g) = y/x^g$.
%    Hence $\sigma^*(\bar\gamma_g) = \bar \gamma_g$, and $\sigma^*(\gamma_g) = \gamma_g + \sum_{i =0}^{g-1}c_i\lambda_i$, for some $c_i \in k$.
%    We now compute $c_{g-1}$.
%    
%    We begin this by computing the lead term of the first entry of $\sigma^*(\gamma_g) \in \cechderhamhone(\cU)$.
%    This is equal to the lead term of $\omega_{\sigma g}$ in $\nu_g$, since applying $\sigma^*$ doesn't change the lead term of any polynomials.
%    As
%        \[
%        \deg(\psi_g(x)t_g(x)x) \leq g+1 + g+ 1 = 2g+1 < 3g+1 = \deg(\phi_g(x)r_g(x)x) = \deg(f(x)x^g)
%        \]
%    we need only compute the coefficient of $x^{3g+1}$ in $2gf(x)(-1)ax^g - \phi_g(x)r_g(x)x$ and show that it is non-zero to find the coefficient of the lead term.
%    Rearranging $2g+1 = p^n $ gives us the identity
%        \[
%        g = \frac{p^n - 1}{2}.
%        \]
%    From this we see that the lead coefficient of $2gf(x)(-1)ax^g$ is 
%        \[
%        2\left( \frac{p^n-1}{2} \right) (-1)a = a
%        \]
%    since $\cha(k) = p$.
%    On the other hand, the lead term of $-\phi_g(x)r_g(x)x$ is
%        \[
%        -(p^n-2g)(-1)\binom{g}{g-1}a = 2\left(\frac{p^n -1 }{2}\right) (-1)\left( \frac{p^n - 1}{2} \right)a = -\frac{a}{2}.
%        \]
%    Finally, it follows that the lead coefficient of the numerator in the second term of is
%        \[
%        a - \frac{a}{2} = \frac{a}{2}.
%        \]
%    
%    
%    Since the denominator of $\omega_{a g}$ is of degree $2g+2$, we see that overall the degree of the first entry of $\gamma_g$ is $g-1$ (as a polynomial in $x$).
%    Now the degree of $\frac{\psi_g(x)}{2yx^{g+1}}$ is less than this, as is the degree of the first entry in $\lambda_i$, unless $i=g-1$, in which case the degree of $\lambda_i$ is precisely $g-1$.
%    Hence, by comparing coefficients, we see that $c_{g-1} = \frac{a}{2}$.
%    
%    Now suppose that 
%        \[
%        s(\bar\gamma_i)  = s( \bar \gamma_g) = \gamma_g + \sum_{i=0}^{g-1}d_i \lambda_i
%        \]
%    for some $d_i \in k$.
%    Then, on the one hand,
%        \[
%        s(\sigma^*(\bar\gamma_g)) = \gamma_g + \sum_{i=0}^{g-1}d_i\lambda_i,
%        \]
%    whilst on the other hand
%        \begin{align}
%        \sigma^*(s(\bar\gamma_g)) & = \sigma^*(\gamma_g + \sum_{i=1}^{g-1} d_i\lambda_i ) \\
%        & = \gamma_g + \frac{a}{2}\lambda_{g-1} + \sum_{i=0}^{g-2} c_i \lambda_k + \sum_{i=0}^{g-1} d_i \sum_{k=0}^{i} \binom{i}{k}a^{i-k}\lambda_k.
%        \end{align}
%    Hence we see that for $s(\sigma^*(\gamma_g))$ to equal $\sigma^*(s(\gamma_g))$ we require $\frac{a}{2} + d_{g-1} = d_{g-1}$; \ie that $\frac{a}{2} = 0$. 
%    But then $a=0$, which is a contradiction.
%    \end{proof}
%
%
%
%
%
%
%
%
%
%
%
%
%
%
%
%
%
