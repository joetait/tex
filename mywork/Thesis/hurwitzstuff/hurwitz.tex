\section{The Riemann-Hurwitz Formula}\label{Hurwitzsection}

Let $X$ and $Y$ be two projective non-singular curves over an algebraically closed field $k$, with a map $\pi :X \rightarrow Y$ of degree $n$.
We let $g_X$ and $g_Y$ be the genera of $X$ and $Y$ respectively.
Let $K(X)$ and $K(Y)$ be the function fields of $X$ and $Y$ respectively.
Note that $K(X)$ is a degree $n$ extension of $K(Y)$, and recall that $\pi$ induces a map $\pi^*:K(Y) \rightarrow K(X)$.
As usual, for any $P \in X$ and $Q \in Y$ we denote the local ring of rational functions at $P$ and $Q$ by $\cO_P$ and $\cO_Q$ respectively.

We also recall some facts about Galois theory.
Any element $\alpha \in K(X)$ defines a $K(Y)$-linear map $\mu_{\alpha} : K(X) \rightarrow K(X)$, multiplication by $\alpha$.
We then define the {\em trace map}, denoted $\Tr_{X/Y}: K(X) \rightarrow K(Y)$, to be the trace of the matrix corresponding to $\mu_{\alpha}$.
Note that $\Tr_{X/Y}$ is an additive map, and that for $\alpha \in K(Y)$ we have $\Tr_{X/Y}(\alpha) = n \alpha$.
Also, since our extension is Galois (and hence, in particular, separable) then the trace map is non-zero (see, for example, \cite[Appendix A]{stichtenoth}).
Finally, if $\alpha \in K(X)$ has minimum polynomial 
\[
 f(X) = X^r + a_{r-1}X^{r-1} + \ldots +a_0 \in K(Y)[X]
\]
 and $s= [K(X):K(Y)(\alpha)]$, then $\Tr_{X/Y}(\alpha) = -sa_{r-1}$ as in \cite[Appendix A]{stichtenoth}.

Given a basis $\{z_1,\ldots,z_n\}$ of $K(X)$ over $K(Y)$, we now introduce its dual basis with respect to the trace.
We denote the dual space of $K(X)$ over $K(Y)$ by \[K(X)^*:=\{\lambda :K(X) \rightarrow K(Y)| \lambda\ \text{is}\ K(Y)\text{-linear}\}.\]
We make $K(X)^*$ in to a one-dimensional $K(X)$-vector space by defining $z \lambda(w):=\lambda(z w)$.
As $\Tr_{X/Y}$ is non-zero, then there exists a unique $z\in K(X)$ for every $\lambda \in K(X)^*$ such that $\lambda = z\cdot \Tr_{X/Y}$.
In particular, if we choose $\lambda_j\in K(X)^*$ such that $\lambda_j(z_i) = \delta_{ij}$ (the Kronecker symbol), then there exist $z_j^*$ such that $\lambda_j = z_j^*\cdot \Tr_{X/Y}$.
Hence
\[
 \Tr_{X/Y}(z_iz_j^*) = (z_j^*\cdot \Tr_{X/Y})(z_i) = \lambda_j(z_i) = \delta_{ij}.
\]
The elements $z_1^*, \ldots , z_n^*$ form the {\em dual basis} of $z_1, \ldots , z_n$.

We now briefly introduce differentials in terms of function fields.
Let $C$ be any projective, non-singular curve over $k$, with function field $K(C)$, and with genus $g_C$.
Then an {\em adele on $C$} is a map $\alpha: C \rightarrow K(C)$ such $\alpha(P) \in \cO_P$ for nearly all $P \in C$ (i.e. all but a finite number of points).
We can view $\alpha$ as an element of $\prod_{P\in C}K(C)$, and we write $\alpha_P := \alpha(P)$.
We let the {\em adele space} of $C$, denoted by $\cA_C$, be the space of all adeles on $C$.
There is a canonical injection $K(C) \hookrightarrow  \cA_C$, defined by sending $x\in K(C)$ to the adele $\alpha$ for which $\alpha_P =  x$ at each point $P\in C$.
The elements of $K(C)$ have natural valuations; for each $P\in C$ we choose a uniformising parameter $t\in \cO_P$, and we then define $v_P(x):=n$, where $n$ is the unique integer such that $x=ut^n$ for some unit $u\in \cO_P$.
We can we can then extend this valuation to elements of $\cA_C$ by defining $v_P(\alpha) := v_P(\alpha_P)$.

Recall that a divisor on $C$ is a finitely supported formal sum over $C$, with coefficients in $\mathbb Z$, which we write as 
\[
 D = \sum_{P\in C} n_P [P],
\]
and that the degree of such a divisor $D$ is 
\[
 \deg(D) := \sum_{P\in X} n_P.
\]
We then define the order of $D$ at $P$ to be $v_P(D) := n_P$.

We let $\cD_C$ be the space of divisors of $C$.
Note that for any $x\in K(C)$ we have a naturally associated divisor, which is
\[
 \di(x) := \sum_{P\in C} v_P(x) [P].
\]

For any $D\in \cD_C$ we define $v_P(D)$ to be the coefficient of $[P]$ in $D$, and we let
\[
 \cA_C(D):=\{\alpha \in \cA_C | v_P(\alpha) \geq -v_P(D)\ \text{for all} \ P\in C\}
\]
be the {\em adele space associated to $D$}.
Then we define a {\em differential} on $C$ to be a $k$-linear map $\omega: \cA_C \rightarrow k$ such that $\omega$ is zero on $\cA_C(D) +K(C)$ for some divisor $D\in \cD_C$.
Note that via the canonical embedding of $K(C)$ into $\cA_C$, we can view $\cA_C(D) +K(C)$ as a subspace of $\cA_C$.
We will denote the space of differentials on $C$ by $\Omega_{K(C)}$.

Now we define 
\[
 M(\omega) := \{ D\in \cD_C | \omega \ \text{is zero on }\ \cA_C(D) + K(C)\}
\]
for any non-zero $\omega \in \Omega_{K(C)}$, and state the following lemma.
\vskip1em

\begin{defnlem}
 For any non-zero differential $\omega$ on $C$ there is a unique maximal divisor $W$ in $M(\omega)$, which we call the {\rm canonical divisor associated to $\omega$} (we associate the zero divisor to $0\in \Omega_{K(C)}$).
 We denote the divisor associated to $\omega$ by $\di (\omega)$.
 We call a divisor a {\rm canonical divisor} if it is the canonical divisor associated to some $\omega \in \Omega_{K(C)}$.\\
 Given any two non-zero canonical divisors, $W$ and $W'$, there exists an $x\in K(C)$ such that $W = W' + \di(x)$.
 Conversely, if $W$ is a canonical divisor, and $x\in K(C)$, then $W + \di(x)$ is a canonical divisor.
\end{defnlem}
\begin{proof}
 See \cite[Lem. I.5.10 and Prop. I.5.13]{stichtenoth}.
\end{proof}



We can now define valuations on the differentials; namely, given some differential $\omega $ on $C$, and its associated divisor $W$, we set $v_P(\omega):=v_P(W)$ for $P\in C$.
The following proposition about the space of differentials is very useful, though we will not prove it here.
\vskip1em




\begin{prop}
 The space of differentials, $\Omega_{K(C)}$, is a one-dimensional vector space over $K(C)$.
\end{prop}
\begin{proof}
 See \cite[Prop. I.5.9]{stichtenoth}.
\end{proof}



It should be noted that in algebraic geometry differentials are normally defined in a different, but equivalent, manner.
Given a ring $K(C)$ and an $K(C)$-module $M$, we call any $K(C)$-linear map $D:K(C)\rightarrow M$ satisfying \[D(ab) = aD(b) + D(a)b\] for all $a,b\in K(C)$ a {\em derivation}.
There is a unique module, denoted $\Omega_{K(C)}$, with a map $d:K(C) \rightarrow \Omega_{K(C)}$, which every derivation must factor through; i.e. if $D:K(C)\rightarrow M$ is a derivation then there is a unique map $f:\Omega_{K(C)}\rightarrow M$ such that $D = f\circ d$.
Given any differential $\omega$, we can write it as $fdx$ for some $x$ and some $f$ in $K(C)$.
This is how we will consider differentials in the other sections of this report.
At the end of this section we will describe an isomorphism between $\Omega_C$ and $\Omega_{K(C)}$.

Before moving on from differentials, there is a particular class that we will be studying.
These are the {\em holomorphic differentials}.
A differential $\omega \in \Omega_{K(C)}$ is called holomorphic if its associated divisor, $W$, is non-negative (i.e. if $v_P(W) \geq 0$ for all $P\in C$).
We denote {\em the space of holomorphic differentials} by $H^0(C,\Omega_{K(C)})$.
Also, for any divisor $D$ on $C$ we define \[H^0(C,\cO(D)) := \{x\in K(C) | v_P(x) \geq -v_P(D)\}\cup \{0\}.\]
This is sheaf theoretic notation, which would not normally be used in function field theory, but we use it for consistency with the rest of the report.
We can now state the celebrated Riemann-Roch theorem.
\vskip1em

\begin{thm}[Riemann-Roch]\label{riemannroch}
Let $W$ be a canonical divisor on $C$.
Then for any divisor $D$ on $C$ 
\[
 \dim H^0(C,\cO(D)) = \deg(D) + 1 - g_C + \dim H^0(C,\cO(W-D)).
\]
\end{thm}
\begin{proof}
 See, for example, \cite[8.6]{fulton} or \cite[I.5.15]{stichtenoth}.
\end{proof}
\vskip1em


\begin{cor}\label{dim=gc}
 For any canonical divisor $W$ on $C$, we have 
 \[
  \deg(W) = 2g_C-2
 \]
and \[
     \dim H^0(C,\cO(W)) = g_C.
    \]
\end{cor}
\begin{proof}
Since $\dim H^0(C,\cO(0)) = 1$ (recall that the only functions with no poles are the constant functions), we have by Riemann-Roch, $ 1= \dim H^0(C,\cO(0)) = 0 + 1 -g_C + \dim H^0(C,\cO(W))$.
 Rearranging this gives the second statement.
 The first statement then follows by rearranging
\[
 g_C = \dim H^0(C,\cO(W)) = \deg(W) + 1 -g_C +  \dim H^0(C,\cO(W-W))= \deg(W) + 1 -g_C + 1.
\]
\end{proof}


There is an alternative way to map $K(C)$ in to $\cA_C$, which we will make use of shortly.
Given a point $P\in C$ we define $\iota_P:K(C) \rightarrow \cA_C$ by
\begin{equation}
 (\iota_P(x))_Q:= \begin{cases}
           x & \text{if }\ P=Q\\
           0 & \text{otherwise}.
           \end{cases}
\end{equation}
for $Q\in C$.
For $\omega \in \Omega_{K(C)}$ we then define $\omega_P:K(C) \rightarrow k$ to be the map $\omega_P(x) := \omega(\iota_P(x))$.
This is called the {\em local component} of $\omega$; we will use these definitions to prove the following proposition.\\

\begin{prop}\label{propertyofomega}
 Let $\omega \neq 0$ be a differential on $X$ and let $P\in X$. Then
 \[
  v_P(\omega) = \max \{r\in \mathbb{Z}|\omega_P(x) = 0\ \text{for all} \ x\in K(X) \ \text{with}\ v_P(x) \geq -r\}.
 \]
In particular $\omega_P \neq 0$.
\end{prop}
\begin{proof}
 Let $W$ be the divisor associated to $\omega$.
 Let $s:=v_P(\omega)$ be the order of $\omega$ at $P$.
 If $x\in K(X)$ and $v_P(x)\geq -s$, then $\iota_P(x) \in \cA_X(W)$, by definition.
 Hence $\omega_P(x) = 0$.
 On the other hand, suppose $\omega_P(x) = 0$ for any $x\in K(X)$ satisfying that $v_P(x) \geq -s-1$.
 Let $\alpha\in \cA_X(W+[P])$.
 Then we have
 \[
  \alpha = (\alpha-\iota_P(\alpha_P)) + \iota_P(\alpha_P).
 \]
Note that $\alpha - \iota_P(\alpha_P)\in \cA_X(W)$ and $v_P(\alpha_P) \geq -s-1$.
Hence
\[
 \omega(\alpha) = \omega(\alpha-\iota_P(\alpha_P))  + \omega_P(\alpha_P) = 0,
\]
and so $\omega$ is zero on $\cA_X(W+P)$.
But this contradicts the maximality of $W$ in its definition.
 
\end{proof}



We now give a small result regarding field extensions, after recalling some terminology.
Recall that if $S$ is a subring of $R$, then an element $x\in R$ is integral over $S$ if there is a monic polynomial with co-efficients in $S$ for which $x$ is a solution.
Then an integral basis of $R$ over $S$ is a basis of $R$ over $S$, for which each basis elements is integral.
 Recall that $\cO_Q$ is an integrally closed subring of $K(Y)$ for any $Q\in Y$, and that $K(Y)$ is the field of fractions of $\cO_Q$.
 \vskip1em


\begin{prop}
 For $z\in K(X)$ we let $\phi(T)\in K(Y)[T]$ be its minimal monic polynomial over $K(Y)$.
 Then $z$ is integral over $\cO_Q$ if and only $\phi (T)\in \cO_Q[T]$.
\end{prop}
\begin{proof}
 By definition $\phi(T)$ is the monic irreducible polynomial in $K(Y)[T]$ such that $\phi(z) = 0$. 
 Hence if $\phi (T)$ has co-efficients in $\cO_Q$, then $z$ is integral over $\cO_Q$ by definition.
 
 We now suppose that $z\in K(X)$ is integral over $\cO_Q$.
 Then we can choose some monic polynomial $f(T)\in \cO_Q[T]$ such that $f(z) = 0$.
 Since $\phi(T)$ is minimal over $K(Y)$, then there exists some $\psi(T)\in K(Y)[T]$ such that $f(T) = \phi(T)\cdot \psi(T)$.
 Let $F \supseteq K(X)$ be a finite extension of $K(X)$ containing all the roots of $\phi$, and let $R$ be the integral closure of $\cO_Q$ in $F$.
 Now the roots of $\phi(T)$ are also roots of $f(T)$, and hence are in $R$.
 Then the coefficients of $\phi(T)$ can be written as polynomials of the roots of $\phi(T)$, and as we just noted that these roots are in $R$ then $\phi(T) \in R[T]$.
 But then $\phi(T)\in K(Y)[T]$, and since $R$ is integrally closed it follows that $K(Y)\cap R = \cO_Q$, hence $\phi(T)\in \cO_Q[T]$.
\end{proof}

\begin{cor}\label{traceinclosure}
Let $Q$ be a point in $Y$ and let $x\in K(X)$ be integral over $\cO_Q$ under $\pi$.
 Then $\Tr_{X/Y}(x)\in \cO_Q$.
\end{cor}
\begin{proof}
 As noted earlier, if $\phi(T)=T^r+a_{r-1}T^{r-1} + \ldots + a_0\in K(Y)[T]$ is the minimal polynomial of $x$ over $K(Y)$, then $\Tr_{X/Y}(x)=-n_xa_{r-1}$, where $n_x : = [K(X):K(Y)(x)]$.
 Hence the corollary follows from the previous proposition.
\end{proof}


We wish to define the ramification divisor, as this is essential for the statement of the Riemann-Hurwitz formula.
In order to do this we first define the complementary module and the ramification index.
\vskip1em


\begin{defn}
 Consider a point $P\in X$ in the pre-image of $Q\in Y$ under $\pi$.
 By \cite[Prop. III.1.4]{stichtenoth} there is an integer $e_P$ such that for any $x\in K(X)$ the equality $v_P(x) = e_P\cdot v_Q(x)$ holds.
 This value is called the ramification index of $P$.
 If $e_P>1$ then we say that $\pi$ is ramified at $P$.
\end{defn}

Given this we can associate to each $Q\in Y$ the divisor
\[
 \pi^*([Q]) := \sum_{P\mapsto Q} e_P [P].
\]
This can then be extended from a single point to a divisor on $Y$, in which case for a divisor $D = \sum_{Q\in Y}n_Q [Q]$ we have
\[
 \pi^*(D) := \sum_{Q\in Y}n_Q \pi^*([Q]).
\]

\begin{rem}
It should be noted that in the literature regarding function field theory, what we have denoted by $\pi^*(D)$ is normally called the conorm of $D$ and is denoted ${\rm Con}_{X/Y}(D)$. 
We used the notation above, form algebraic geometry, to be consistent with the rest of this report.
\end{rem}
~

\begin{defn}
 For any $Q\in Y$, let $\cO_Q'$ be the integral closure of $\cO_Q$ in $K(X)$.
 We then define the complementary module over $\cO_Q$ to be
 \[
  C_Q :=\{z\in K(X) | \Tr_{X/Y}(z\cdot \cO_Q') \subseteq \cO_Q\}.
 \]
\end{defn}


We will list some properties of the complementary module in a proposition, but first we require the following lemmas.
\vskip1em

\begin{lem}[Approximation lemma]\label{approximationlemma}
Let $m$ be a positive integer. 
For each $i\in \{1,\ldots, m\}$ let $P_i\in X$, let $\mathcal{P}_i$ be the corresponding maximal ideal of $\cO_{P_i}$, let $x_i$ be an element of $K(X)$ and let $n_i$ be an integer.
Then there is an $x\in K(X)$ such that $v_{P_i}(x-x_i) \geq n_i$ for all $i$.
\end{lem}
\vskip1em

\begin{rem}
This result can be strengthened to also say that $v_P(x) \geq 0$ for any $P\notin \{P_1,\ldots ,P_m\}$.
We will not prove this here, for the sake of brevity, but the proof can be found in \cite[Ch. 1, \S 3]{localfields}.
\end{rem}
\begin{proof}
We let $R:= \cO_{P_i} \cap \ldots \cap \cO_{P_m}$, and we let $\mathcal{P}_i' := \mathcal{P}_i \cap R$.
To start we prove the lemma in the case that the $x_i$ are in $R$.
 We may increase the $n_i$ such that $n_i\geq 0$ for all $i$.
 By linearity we may assume that $x_2 = \ldots = x_m = 0$, since if we find an element for $x_1$ in this instance, we can similarly find an element for each $i$ and add them.
Let $I = {\mathcal{P}_1'}^{n_1} + {\mathcal{P}_2'}^{n_2}\cdots {\mathcal{P}_n'}^{n_m}$.
This is an ideal of $R$, and since it has elements whose valuation at any $P_i$ is zero, it is in fact equal to $R$.
Hence we can write $x_1 = x + y$, where $y \in {\mathcal{P}_1'}^{n_1}$ and $x\in {\mathcal{P}_2'}^{n_2}\cdots {\mathcal{P}_n'}^{n_m}$.
Since ${\mathcal{P}_1'}^{n_1} \subseteq {\mathcal{P}_1}^{n_1}$ and ${\mathcal{P}_2'}^{n_2}\cdots {\mathcal{P}_n'}^{n_m} \subseteq \mathcal{P}_2^{n_2}\cdots \mathcal{P}_n^{n_m}$, the $x$ above is as described in the lemma, and this finishes the proof in the case $x_i \in R$.

 
 In general one can write $x_i = \frac{a_i}{b}$ for $a_i\in R$ and $b\in R\backslash \{0\}$, and $x$ can be represented as $\frac{a}{s}$.
 Then we require that $v_{P_1}(a-a_i) \geq n_i + v_{P_i}(s)$ for all $i$ and that $v_P(a)_ \geq v_P(s)$ for all $P\notin \{P_1,\ldots ,P_m\}$.
 But after adding the points $P$ for which $v_P(s)$ is negative, this is precisely what we described above.
\end{proof}

\begin{lem}\label{pidlemma}
 Let $Q\in Y$, and let $\cO_Q'$ be the integral closure of $\cO_Q$ in $K(X)$.
 Then $\cO_Q'$ is a principal ideal domain.
\end{lem}
\begin{proof}
 By \cite[Cor. III.3.5]{stichtenoth}, we have $\cO_Q' = \{x\in K(X)|v_P(x) \geq 0 \ {\rm for all} \ P\in \pi^{-1}(Q)\}$.
 Let $I$ be an ideal of $\cO_Q'$.
 If we let $\{P_1,\ldots, P_l\} = \pi^{-1}(Q)$ then we can choose $x_i$ for $1\leq i \leq l$ such that $v_{P_i}(x_i) \leq v_{P_i}(y)$ for all $y\in I$.
 By the Approximation Lemma there exist $z_i$ such that $v_P (z_i) = 0$ if $P=P_i$ and $v_{P_j}(z_i) > v_{P_j}(z_j)$ for $j\neq i$.
 Now let $x = \sum_{i=1}^l x_iz_i \in I$.
 Clearly $v_{P_i}(x) = v_{P_i}(x_i)$ for all $1\leq i\leq l$.\\
 Now we show that $I \subseteq x\cO_Q'$.
 If $y\in I$ then we let $z = x^{-1}y$.
 Then $v_{P_i}(z) = v_{P_i}(y) - v_{P_i}(x_i) \geq 0$ for all $1\leq i\leq l$.
Hence $z\in \cO_Q'$ and so $y = xz \in x\cO_Q'$, completing the proof.
\end{proof}


\begin{prop}\label{factsaboutc'}
Let $Q\in Y$. 
 \begin{enumerate}
  \item $C_Q$ is an $\cO_Q'$-module, and $\cO_Q' \subseteq C_Q$.
  \item If $\{z_1,\ldots ,z_n\}$ is a (necessarily integral) basis of $\cO_Q'$ over $\cO_Q$, then 
  \[
   C_Q = \sum_{i=1}^n \cO_Q\cdot z_i^*,
  \]
where $z_i^*$ is the dual basis of $z_i$ for $1\leq i \leq n$.
  \item There is a $t\in K(X)$ such that $C_Q = t\cdot \cO_Q'$ (note that this $t$ depends on the choice of $Q$).
  Moreover, $v_P(t) \leq 0$ for all $P\in \pi^{-1}(Q)$, and if $t'\in K(X)$ then $C_Q=t'\cdot \cO_Q'$ if and only if $v_P(t) = v_P(t')$ for all $P\in \pi^{-1}(Q)$.

 \end{enumerate}
\end{prop}
\begin{proof}
 \begin{enumerate}
  \item It is clear that $C_Q$ is an $\cO_Q'$ module.
	Indeed, if $y\in \cO_Q'$, then $y\cdot \cO_Q' \subseteq \cO_Q'$ and hence $\Tr_{X/Y}(z y\cdot \cO_Q')\subseteq \Tr_{X/Y}(z\cdot \cO_Q') \subseteq \cO_Q$ for any $z\in C_Q$.
	The fact that $\cO_Q'$ is contained in $C_Q$ follows from Corollary \ref{traceinclosure}.
  \item We first show that $C_Q \subseteq \sum_{i=1}^n \cO_Q\cdot z_i^*$.
	Suppose $z\in C_Q$.
	Now $\{z_1^*, \ldots ,z_n^*\}$ is a basis of $K(X)$ over $K(Y)$, so there exist $x_1,\ldots , x_n\in K(Y)$ such that $z=\sum_{i=1}^n x_iz_i^*$.
	As $z\in C_Q$ and $z_1,\ldots ,z_n\in \cO_Q'$, it follows by definition of $C_Q$ that $\Tr_{X/Y}(zz_j)\in \cO_Q$ for $j\in \{1,\ldots ,n\}$.
	We know that 
	\[
	 \Tr_{X/Y}(zz_j) = \Tr_{X/Y}\left(\sum_{i=1}^nx_iz_iz_j^*\right) = \sum_{i=1}^nx_i \cdot \Tr_{X/Y}(z_iz_j^*) = x_j,
	\]
	since $z_j^*$ is dual to $z_j$.
	Hence each $x_j$ is in $\cO_Q$, and $z\in \sum_{i=1}^n\cO_Q\cdot z_i^*$.
	
	Now suppose that $z\in \sum_{i=1}^n\cO_Q\cdot z_i^*$ and $u\in \cO_Q'$.
	Then we need to show that $\Tr_{X/Y}(z u)\in \cO_Q$.
	We can find $x_i, y_j\in \cO_Q$ such that $z=\sum_{i=1}^n x_iz_i^*$ and $u=\sum_{i=1}^ny_jz_j$.
	Then
	\[
	 \Tr_{X/Y}(zu) = \Tr_{X/Y}\left(\sum_{i,j=1}^n \left(x_iy_jz_i^*z_j\right)\right) = \sum_{i,j=1}^n x_iy_j\cdot \Tr_{X/Y}(z_i^*z_j) = \sum_{i=1}^n x_iy_i.
	\]
	Since $\sum_{i=1}^n x_iy_i\in \cO_Q$, it follows that $z\in C_Q$.
  \item By the previous part, we can find $u_i\in K(X)$ such that $C_Q = \sum_{i=1}^n \cO_Q \cdot u_i$.
	Choose some $x\in K(Y)$ such that $v_Q(x)\geq 0$ and also $v_Q(x)\geq -v_P(u_i)$ for all $P\in \pi^{-1}(Q)$ and $i\in \{1,\ldots ,n\}$.
	By definition of $e_P$ it follows that
	\[ v_P(xu_i) = e_Pv_Q(x) + v_P(u_i) \geq 0\]
	for all $i\in \{1,\ldots, n\}$ and all $P\in \pi^{-1}(Q)$.
	Since $\cO'_Q = \cap_{P\mapsto Q} \cO_P$ (see, for example, \cite[Cor. III.3.5]{stichtenoth}), then $x\cdot C_Q \subseteq \cO_Q'$.
	Since $\cO_Q'$ is a principal ideal domain by Lemma \ref{pidlemma}, it follows that $x\cdot C_Q = y\cdot \cO_Q'$ for some $y\in \cO_Q'$.
	If we let $t=x^{-1}y$ then $C_Q =t\cdot \cO_Q'$, proving the first part of the statement.\\
	Since $\cO_Q'\subseteq C_Q$ then $v_P(t)\leq 0$ for all $P\in \pi^{-1}(Q)$.
	Now $t\cdot \cO_Q' = t'\cdot \cO_Q'$ if and only if both $tt'^{-1}$ and $t^{-1}t'$ are in $\cO_Q'$.
	But this is the case if and only $v_P(tt'^{-1}) \geq 0$ and $v_P(t^{-1}t')\geq 0$ for all $P\in \pi^{-1}(Q)$, which is equivalent to $v_P(t)=v_P(t')$ for all such $P$.
 \end{enumerate}
\end{proof}

\begin{prop}\label{almostallqiny}
 For almost all $Q\in Y$ we have $C_Q= \cO_Q'$.
\end{prop}
\begin{proof}
 We first show that, given a basis of $K(X)$ over $K(Y)$, it is almost always integral over $\cO_Q$ given $Q\in Y$.
 Recall that $\cO_Q$ is integrally closed in $K(Y)$, and its quotient field is $K(Y)$.
 As before, we denote by $\cO_Q'$ the integral closure of $\cO_Q$ in $K(X)$, and we consider a basis $\{z_1,\ldots ,z_n\}$ of $K(X)$ over $K(Y)$.
 Let $\{z_1^*,\ldots ,z_n^*\}$ be the dual basis.
 Now the minimal polynomials of $z_1,\ldots, z_n,z_1^*,\ldots z_n^*$ have finitely many coefficients in $K(Y)$.
 Hence if $S\subseteq Y$ is the set of poles of these co-efficients, then $S$ is finite and for $Q\notin S$ then we have
 \[
  z_1,\ldots,z_n,z_1^*,\ldots, z_n^*\in \cO_Q'.
 \]

 
 
Now we assume that $\{z_1,\ldots ,z_n,z_1^*,\ldots z_n^*\}\subseteq \cO_Q'$ and then we show that 
\[
  \cO_Q' \subseteq \sum_{i=1}^n \cO_Q\cdot z_i^*.
\]

 If $z\in K(X)$ then there are $e_1,\ldots, e_n\in K(Y)$ such that $z=e_1z_1^*+\ldots +e_nz_n^*$.
 If $z\in \cO_Q'$ then $zz_j\in \cO_Q'$ for $1\leq j\leq n$, and hence $\Tr_{X/Y}(zz_j)\in \cO_Q$, by Corollary \ref{traceinclosure}.
 As
 \[
  \Tr_{X/Y}(zz_j) = \Tr_{X/Y}\left(\sum_{i=1}^n e_iz_jz_i^*\right) = \sum_{i=1}^ne_i\cdot \Tr_{X/Y}(z_jz_i^*) = e_j.
 \]
  Hence $e_j\in \cO_Q$, and $\cO_Q'\subseteq \sum_{i=1}^n\cO_Q\cdot z_i^*$.
  Since $\{z_1,\ldots, z_n\}$ also forms a basis we can run the same argument again to show that $\cO_Q'\subseteq \sum_{i=1}^n\cO_Q\cdot z_i$.
  
  We then have the following set of inequalities:
  \[
   \sum_{i=1}^n \cO_Q\cdot z_i \subseteq \cO_Q' \subseteq \sum_{i=1}^n\cO_Q\cdot z_i^* \subseteq \cO_Q'\subseteq \sum_{i=1}^n \cO_Q\cdot z_i.
  \]
Since $C_Q = \sum_{i=1}^n \cO_Q\cdot z_i^*$ by part 2 of Proposition \ref{factsaboutc'}, the result follows.
\end{proof}

We can now define the ramification divisor.
\vskip1em


\begin{defn}
 For a point $Q\in X$ choose $t\in K(X)$ such that $C_Q = t\cdot \cO_Q'$, as in part 3 of Proposition \ref{factsaboutc'}.
 Then for any $P\in \pi^{-1}(Q)$ we define the different exponent to be $\delta_P := -v_P(t)$.
 By Proposition \ref{factsaboutc'} this is well defined, and by Proposition \ref{almostallqiny} it is almost always zero.
 Hence we can define the ramification divisor to be
 \[
  R := \sum_{P\in X} \delta_P [P].
 \]
\end{defn}

\begin{rem}
It should be noted that when considering the theory of function fields the ramification divisor is called the different, and denoted ${\rm Diff}_{X/Y}$.
\end{rem}

We now define the adele space of $X$ over $Y$.
\vskip1em

\begin{defn}
 We define $\mathcal {A}_{X/Y}$ as
 \[
  \mathcal{A}_{X/Y} := \{\alpha \in \mathcal{A}_X | \alpha_P =\alpha_{P'}\ \text{if}\ \cO_P\cap K(Y) = \cO_{P'}\cap K(Y)\}.
 \]
We extend the trace function $\Tr_{X/Y}:K(X)\rightarrow K(Y)$ to a map $\Tr_{X/Y}:\mathcal{A}_{X/Y} \rightarrow \mathcal{A}_Y$,  by letting
\[
 (\Tr_{X/Y}(\alpha))_Q := \Tr_{X/Y}(\alpha_P)
\]
for any $\alpha \in \mathcal{A}_{X/Y}$, $Q\in Y$ and $P\in \pi^{-1}(Q)$.

For any divisor $D$ over $X$ we define
\[
 \mathcal{A}_{X/Y}(D) := \mathcal{A}_X(D) \cap \mathcal{A}_{X/Y}.
\]

\end{defn}

~

\begin{defnthm}\label{detailedhurwitz}
 For every differential $\omega\in H^0(Y,\Omega_Y)$ there is a unique $\omega'\in H^0(X,\Omega_X)$ such that
 \[
  \omega'(\alpha) = \omega\left(\Tr_{X/Y}(\alpha)\right)
 \]
for all $\alpha \in {\mathcal A}_{X/Y}$.

This differential is called the {\rm pullback} of $\omega$, and is denoted $\pi^*(\omega)$. 
If $\omega\neq 0$ and $(\omega)$ is the associated divisor, then 
\[
 \di ( \pi^*(\omega)) = \pi^*(\di(\omega)) + R.
\]
\end{defnthm}
\begin{comment}
\begin{rem}
 In the language of algebraic geometry, where we would write differentials as $dx$ for some $x\in K(X)$, then this would be phrased differently.
 When we use the result in later sections, it will be written as
 \[
  \di (\pi^* (dx)) = \pi^*(\di (dx)) + R
 \]
for some holomorphic differential $dx$.

Also, it should be noted that in theory of function fields, what we have denoted by $\pi^*(\omega)$ is called the cotrace of $\omega$, and denoted ${\rm Cotr}_{X/Y}(\omega)$.
\end{rem}
\end{comment}
We first note that this immediately implies the Riemann-Hurwitz formula.
\vskip1em




\begin{cor}\label{hur}[Riemann-Hurwitz Formula]
 Given two non-singular projective curves $X$ and $Y$ of genera $g_X$ and $g_Y$ respectively, with a degree $n$ map $f:X \rightarrow Y$, then
 \[
  2g_X - 2 = n(2g_Y -2) + \deg(R),
 \]
where $R$ is the ramification divisor of $f$.
\end{cor}
\begin{proof}
 This follows from Theorem \ref{detailedhurwitz} and Corollary \ref{dim=gc}, after taking degrees.
\end{proof}


We first give a lemma that is necessary for the proof of the theorem.
\vskip1em


\begin{lem}\label{adelespacelemma}
 For any divisor $D$ over $X$ we have $\mathcal {A}_X = \mathcal{A}_{X/Y} + \mathcal{A}_X(D)$.
\end{lem}
\begin{proof}
 Let $\alpha \in \mathcal{A}_X$. Then by the approximation lemma there is a an element $x_Q\in K(X)$ for each $Q\in Y$ such that 
 \[
  v_P(\alpha_P - x_Q) \geq -v_P(D)
 \]
for each $P\in \pi^{-1}(Q)$. 
We then define the adele $\beta$ such that $\beta_P := x_Q$ for every $P\in \pi^{-1}(Q)$.
Then $\beta \in \mathcal{A}_{X/Y}$ and the difference $\alpha - \beta$ is in $\mathcal{A}_X(D)$ by definition of $\beta$.
Hence $\alpha = \beta + (\alpha - \beta) \in \mathcal{A}_{X/Y} + \mathcal{A}_X(D)$.
\end{proof}
\begin{comment}
\begin{lem}
 Let $V$ be a vector space over $K(X)$ and let $\mu:V\rightarrow K(Y)$ be a $K(Y)$ linear map.
 Then thee is a unique $K(X)$ linear map $\mu':V\rightarrow K(X)$ such that $\Tr_{X/Y}\circ \mu' = \mu$.
\end{lem}
\begin{proof}
 {\bf is prop III.3.3 needed?}
 Let $K(X)^:=\{ \lambda:K(X) \rightarrow K(Y)| \lambda L-\text{linear}\}$ be the space of linear forms, which forms a vector space over $K(X)$.
 It is one dimensional over $K(X)$ {\bf (check why)}, and so any $\lambda \in K(X)^*$ can be written as $z\cdot \Tr_{X/Y}$ for some $z\in K(X)$.
 
 If we fix a $v\in V$ then we can define an $L$-linear map $\lambda :K(X) \rightarrow K(Y)$ as $\lambda_v(a)\mapsto \mu(av)$.
 So there exists a unique $z_v\in K(X)$ such that $\lambda_v = z_v\cdot \mu$, and we define $\mu'(v) := z_v$.
 Hence
 \[
  \mu(av) = (\mu'(v)\cdot \Tr_{X/Y})(a) = \Tr_{X/Y}(a\cdot \mu'(diffv))
 \]
for every $a\in K(X)$ and $v\in V$.
Since the trace function and $\mu$ are both linear, it follows that $\mu'$ is also.
If we let $a=1$ then the equality $\mu = \Tr_{X/Y} \circ \mu'$, which proves existence.
If there were another such map, say $\mu^*$, then the difference $\mu'-\mu^*:V\rightarrow K(X)$ would be surjective, but $\Tr_{X/Y}\circ (\mu'-\mu^*) = 0$.
This implies that $\Tr_{X/Y}=0$, a contradiction, hence $\mu'$is unique.
\end{proof}
\end{comment}
We now prove Theorem \ref{detailedhurwitz}.

\begin{proof}
 We first construct a differential $\omega'\in H^0(X,\Omega_X)$ for every $\omega\in H^0(Y,\Omega_Y)$ such that $\omega'(\alpha) = \omega(\Tr_{X/Y}(\alpha))$ for every $\alpha \in \cA_{X/Y}$.
 If $\omega = 0$ then we can clearly let $\omega' = 0$, so we assume that $\omega \neq 0$.
 We will use the following divisor, $W' := \pi^*(\di (\omega)) + R$, throughout the proof.
 
 We first prove two assertions about $\omega_1:= \omega\circ \Tr_{X/Y} :\mathcal{A}_{X/Y} \rightarrow k$.
 Namely, we show that
 \begin{enumerate}[(i)]
  \item For any $\alpha \in \mathcal{A}_{X/Y}(W') + K(X)$ then $\omega_1(\alpha) = 0$;
  \item If $B'$ is a divisor on $X$ with $B' \nleq W'$ then there is a $\beta \in \mathcal{A}_{X/Y}(B')$ such that $\omega_1(\beta) \neq 0$.
 \end{enumerate}
To show (i) we start by noting that since $\Tr_{X/Y}$ and $\omega$ are $k$-linear, $\omega_1$ is too.
Also, note that since $\omega$ is zero on $K(Y)$ then $\omega_1$ is zero on $K(X)$.
To show that $\omega_1(\alpha) = 0$ for any $\alpha \in \mathcal{A}_{X/Y}(W')$ then it is sufficient to show that for any $Q\in Y$ and $P\in \pi^{-1}(Q)$ that
\[
 v_Q(\Tr_{X/Y}(\alpha_P)) \geq -v_Q(\omega).
\]

We choose $x\in K(Y)$ such that $v_Q(x) = v_Q(\omega)$.
Then
\begin{eqnarray}\label{remark}
 & v_P(x \alpha_P) = v_P(x) + v_P(\alpha_P) \geq e_P v_P(\omega) - v_P(W') \nonumber \\
 & = v_P(\pi^*((\omega)) - W') = -v_P(R) = -\delta_P. 
\end{eqnarray}
By the definition of $C_Q$ and the ramification divisor, it is clear that $z\in C_Q$ if and only if $v_P(z)\geq -\delta_P$ for all $P\in \pi^{-1}(Q)$.
It then follows from \eqref{remark} that $x \alpha_P$ is in $C_Q$ and hence that $v_Q(\Tr_{X/Y}(x \alpha_P)) \geq 0$, by definition of $C_Q$.
Since $\Tr_{X/Y}(x \alpha_P) = x\cdot \Tr_{X/Y}(\alpha_P)$ and $v_Q(x) = v_Q(\omega)$, we have shown the first claim.

To show (ii) we let $Q_0\in Y$ be a point such that there is a $P^*\in \pi^{-1}(Q_0)$ with $v_{P^*}(\pi^*((\omega)) - B') < -\delta_P$.
We know that such a $Q_0$ exists since $B' \nleq W'$.
As before we will denote by $\cO_{Q_0}'$ the integral closure of $\cO_{Q_0}$ in $K(X)$.
We let 
\[
 J := \{ z\in K(X) | v_{P^*}(z) \geq v_{P^*}(\pi^*(\di (\omega)) - B')\ \text{for all}\ P^*\in \pi^{-1}(Q_0)\}.
\]
By the approximation lemma, there exists a $u\in J$ such that 
\[
 v_{P^*}(u) = v_{P^*}(\pi^*(\di (\omega))-B')
 \]
 for all $P^*\in \pi^{-1}(Q_0)$, and hence $J\nsubseteq C_{Q_0}$.
(As noted earlier, $z\in \cO_Q$ if and only if $v_P(z) \geq -\delta_P$).
It is clear that $J\cdot \cO_{Q_0} \subseteq J$, and hence that $\Tr_{X/Y}(J) \nsubseteq \cO_Q$.
We let $t$ be an element of $K(Y)$ such that $v_Q(t) = 1$.
Then there is an $r\in \mathbb N$ such that 
$t^r\cdot J \subseteq \cO_Q,
 $
 and then \[ t^r\cdot \Tr_{X/Y}(J) = \Tr_{X/Y}(t^r\cdot J) \subseteq \cO_Q.\]
It is clear that $t^r\cdot \Tr_{X/Y} (J)$ is an ideal of $\cO_Q$, and so $\Tr_{X/Y}(J) = t^s\cdot \cO_Q$ for some negative integer $s$.
Hence 
\begin{equation}\label{traceinring}
 t^{-1}\cdot \cO_Q \subseteq \Tr_{X/Y}(J).
\end{equation}
By Proposition \ref{propertyofomega} we can find an $x\in K(Y)$ such that $v_Q(x) = -v_Q(\omega) - 1$ and $\omega_Q(x) \neq 0$.
If we choose $y\in K(Y)$ such that $v_Q(y) = v_Q(\omega)$, then $xy \in t^{-1}\cO_Q$.
Hence by \eqref{traceinring} there is a $z\in J$ such that $\Tr_{X/Y} (z) = xy$.
Let $\beta \in \mathcal{A}_{X/Y}$ be chosen  such that 
\begin{equation*}
 \beta_P = \begin{cases}
            y^{-1}z & \text{if}\ P\in \pi^{-1}(Q) \\
            0 & \text{otherwise}.
           \end{cases}
\end{equation*}
Then for any $P\in \pi^{-1}(Q)$ we have
\begin{eqnarray*}
 v_P(\beta) & = & -v_P(y) + v_P(z) \\
 & \geq & -v_P(\pi^*(\di (\omega))) + v_P(\pi^*(\di (\omega)) - \beta) \\
 & = & -v_P(B'),
\end{eqnarray*}
with the inequality following from the definition of $y$ and $J$.
Hence $\beta \in \mathcal{A}_{X/Y}(B')$.
Now $\omega_1(\beta) = \omega(\Tr_{X/Y}(\beta)) = \omega_Q(x) \neq 0$.
This shows (ii).

By Lemma \ref{adelespacelemma} for each $\alpha \in \mathcal{A}_X$ there exists some $\beta \in \mathcal{A}_{X/Y}$ and $\gamma \in \mathcal{A}_X(W')$ such that $\alpha = \beta + \gamma$.
We now define a new differential $\omega_2 : \mathcal{A}_X \rightarrow k$ by letting $\omega_2(\alpha) := \omega_1(\beta)$.
Suppose we have two representations of $\alpha$, say $\beta+ \gamma$ and $\beta' + \gamma'$, where $\beta, \beta' \in \mathcal{A}_{X/Y}$ and $\gamma, \gamma' \in \mathcal{A}_{X}(W')$.
Then 
\[
  \beta - \beta' = \gamma' - \gamma \in \mathcal{A}_{X/Y} \cap \mathcal{A}_X(W') = \mathcal{A}_{X/Y}(W').
\]
It then follows from (i) that 
\[
 \omega_1(\beta) - \omega_1(\beta') = \omega_1(\beta - \beta') = 0.
 \]
and hence $\omega_2$ is well defined.
It is also clear that $\omega_2$ is $k$-linear.
Also, by the first two points we proved, (i) and (ii), we have:
\begin{enumerate}[(i$'$)]
 \item $\omega_2(\alpha) = 0$ for all $\alpha \in \mathcal{A}_X(W') + K(X)$.
 \item If $B'$ is a divisor on $X$ such that $B'\nleq W'$ then there is a $\beta \in \mathcal{A}_X(B')$ with $\omega_2(\beta) \neq 0$.
\end{enumerate}


Now it follows that for $\alpha \in \mathcal{A}_{X/Y}$ we have $\omega_2(\alpha) = \omega_1(\alpha) = \omega(\Tr_{X/Y}(\alpha))$.
This means that we have found the $\omega'$ in the statement of the theorem; namely $\omega' = \omega_2$.

It is clear from (i$'$) and (ii$'$) that $\di(\omega') = W' = \pi^*(\di(\omega)) + R$.

We finally prove the uniqueness of $\omega'$.
Suppose that $\omega''$ also satisfies 
\[
 \omega''(\alpha) = \omega(\Tr_{X/Y}(\alpha))
\]
for all $\alpha\in \cA_{X/Y}$.
So if we let $\theta = \omega'' - \omega'$, then $\theta(\alpha) = 0$ for all $\alpha \in \cA_{X/Y}$.
But then if we choose a large enough divisor $D$ in Lemma \ref{adelespacelemma}, this implies that $\theta = 0$ and $\cA_X$, and hence $\omega' = \omega''$.
\begin{comment}
If $\omega = 0$ this is clear.
The order of the differential at any point can be determined by the equality of divisors,
\[
(\pi^*(\omega) ) = \pi^*((\omega)) + R.
\]
If differential is a unit at a point, it's precise value can be determined by the equality
\[
\omega'(\alpha) = \Tr_{X/Y}(\omega(\alpha)).
\]
{\bf check this last part}
\end{comment}
\end{proof}

%\bibliography{/home/jtait/files/Documents/Maths/Bibliography/biblio.bib}
%\bibliography{/home/joe/files/Documents/Maths/Bibliography/biblio.bib}

%\bibliographystyle{plain}
\begin{comment}
We now show the equivalence of our two definitions of differential.

Recall that for a curve $C$ a derivation of $K(C)$ is a $k$-linear map $d:K(C) \rightarrow M$ for some $K(C)$-module $M$ such that $d(ab) = ad(b) + d(a)b$ for all $a,b\in K(C)$.
Note that of course, as $K(C)$ is a field, $M$ is a vector space, but this does hold more generally for rings and modules.

By \cite[Prop. IV.1.4]{stichtenoth}, for each $x\in K(C)\backslash k$ there exists a unique derivation $d_x:K(C) \rightarrow K(C)$ such that $d_x(x) = 1$, which we call the {\em derivation with respect to $x$}.

If we let $Z:= \{(u,x)\in K(C)\times K(C) |x\notin k\}$, then we can define a relation on the elements of $Z$ by letting $(u,x) \sim (v,y)$ if $v = u\cdot d_y(x)$.
Then we denote the equivalence class of $(1,x)\in Z$ by $dx$.
It can then be shown that $Z$, when quotiented by the above relation, has the univerisal property of $\Omega_C$, and hence is isomorphic to $\Omega_C$ as a $K(C)$-module (see \cite[IV.1.8]{stichtenoth}).
\end{comment}

Let $C$ be a smooth, projective, connected algebraic curve over $k$.
We now show that our two different definitions of $\Omega_{K(C)}$ give rise to isomorphic differentials.
Recall that $K(x)$ is the field of rational functions of the projective line.
By \cite[Prop. I.7.4]{stichtenoth} there exists a unique differential $\omega \in \Omega_{K(x)}$ such that $\di (\omega) = -2[P_{\infty}]$ and $\omega(\iota_\infty(x^{-1})) = -1$, where $P_\infty \in \mathbb P_k^1$ is the point at infinity on the projective line.


Now for any $z\in K(C) \backslash k$ we have that $k(z)$ is isomorphic to $K(x)$.
Hence if we let $G$ be the Galois group of the extension $[K(C):k(z)]$ then the quotient curve $Y:=X/G$ is isomorphic to the projective line.
We let $f_z:C \rightarrow \mathbb P_k^1$ be the corresponding surjective map, and we denote the unique differential described above by $\omega_z$.
We then define $\delta:K(C) \rightarrow \Omega_{K(C)}$ to be the map such that if $z\in K(C)\backslash k$ then $\delta (z) := f_z^*(\omega_z)$ and if $y\in k$ then $\delta(y):=0$.
This then induces a map $\mu:\Omega_{K(C)} \rightarrow \Omega_{K(C)}$, defined by $z\cdot dx \mapsto z\cdot \delta(x)$·

\begin{thm}
The map $\mu:\Omega_{K(C)} \rightarrow \Omega_{K(C)}$ is an isomorphism.
\end{thm}
\begin{proof}
See \cite[Thm. IV.3.2]{stichtenoth}.
\end{proof}

We now define the order of a poly-differential at a point.
If we consider an element of the tensor product $\omega \in \Omega_X^{\otimes m}$ then it can be locally written as $y dx_1\otimes \ldots \otimes dx_m$, where $x_i \in K(X)$ for all $1 \leq i \leq m$.
Let $P$ be a point in $X$.
Since each $dx_i$ can be written as $y_i dt$ for some $y_i\in K(X)$ and some uniformising parameter $t$ at $P$, we can rewrite $\omega$ as $y' dt \otimes \ldots \otimes dt$, where $y' = y \cdot y_1 \cdots y_m$.
We then define the order of $\omega$ at $P$ to be $\ord_P(\omega ) := \ord_P(y')$.
In the particular case where $\omega = fdx \otimes \ldots \otimes fdx = f^m dx^{\otimes m}$, then we have $y_1 = \ldots = y_m = z$ for some $z$ when we change $x$ to a uniformising parameter.
Hence in this instance \[ \ord_P(\omega) = \ord_P(z^m) = m\ord_P(z) = m\ord_P(dx).\]
