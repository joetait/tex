This section gives an account of the Hurwitz formula for algebraic curves. It is the solution to an exercise in \cite[pg. 110]{fulton}, and the approach differs from that used in most textbooks about algebraic curves.

We consider two projective non-singular curves $X$ and $Y$, over an algebraically closed field $k$, with a surjective morphism $f:X\rightarrow Y$.
Let $g_X$ and $g_Y$ be the genus of $X$ and $Y$ respectively. 
We denote the function fields of $X$ and $Y$ by $K(X)$ and $K(Y)$.
Then $f$ induces a map $ f^*:K(Y)\rightarrow K(X)$.
We define the degree of $f$ to be the degree of the extension $n = [K(X):K(Y)]$. 
We denote by $\cO_Q(Y)$ the local ring of rational functions of $Y$ at $Q\in Y$.\\


\begin{defn}
	Let $t\in \cO_{Q}(Y)$ be a uniformising parameter for some $Q\in Y$.
	Then the ramification index of $f$ at $P\in f^{-1}(Q)$ is $e(P):= \ord_{P}(f^* (t))$.
\end{defn}

Note that this is independent of the choice of uniformising parameter; if $s$ is another uniformising parameter then $s=ut$ for some unit $u$. 
Then 
\[
\ord_{P}(f^*(s))=\ord_{P}(f^*(ut))=ord_{P}(f^*(u))+ord_{P}(f^*(t))=\ord_{P}(f^*(t)),
\]
as the order of any unit is zero.

Now we want to show the Hurwitz formula, which relates the genus of $X$ to the genus of $Y$ using the ramification indices (see theorem \ref{hur}).
The first step towards this is the next proposition.\\


\begin{prop}
	For each $Q\in Y$ we have $\sum_{P\mapsto Q}e(P)=n$.
\end{prop}
\begin{proof}
	Choose a uniformising parameter $t\in \cO_{Q}(Y)$ and let $m=[K(Y):k(t)]$.

	To start with we show that $\sum_{P\mapsto Q} e(P)\leq n$.
	Let $Z=\sum_{P\mapsto Q}e(P)[P]$, $q=\deg (Z)$ and $S=f^{-1}(Q)$. 
	For a divisor $D=\sum_{P\in X}n_P[P]$ we define 
		\[
			L^S(D):=\{f\in K(X)|\ord_P(f)\geq n_P\mbox{ for all }P\in S\}.
		\]
	Now by \citep[Lem 1, pg.193]{fulton} we can choose $v_{1},\ldots,v_{q}\in L^{S}(0)$ such that the residues $\bar{v}_{1},\ldots ,\bar{v}_{q}\in L^{S}(0)/L^{S}(-Z)$ form a basis over $k$. 
	It suffices to show that $v_{1},\ldots ,v_{q}$ are linearly independent over $K(Y)$.
	If not then there exist (after multiplying by a suitable power of $t$ if necessary) $g_{i}=\lambda_{i} +h_{i}\in K(Y)$ with $\ord_{Q}(h_{i})>0$,
	$\lambda_{i}\in k$ and at least one $\lambda_{i}\neq 0$, such that $\sum_{i}g_{i}v_{i}=0$.
	But then $\sum_{i}\lambda_{i}v_{i}=-\sum_{i}h_{i}v_{i}\in L^{S}(-Z)$. 
	Hence $\sum_{i}\lambda_{i}\bar{v}_{i}=0$, contradicting the choice of $\bar{v}_{1}, \ldots, \bar v_q$ as a basis, and so $\sum_{P\mapsto Q} e(P) \leq n$.

	Now let
		\[
			(t)_{0}^{Y}=\sum_{\textrm{ord}_{Q}(t)>0}m_{Q}[Q]
		\]
	be the divisor of zeros of $t$ in $Y$. Then
		\[
			(t)_{0}^{X}=\sum_{\textrm{ord}_{Q}(t)>0}m_{Q} \sum_{P\mapsto Q}e(P)[P]
		\]
	is obviously the divisor of zeroes of $t$ in $X$.

	Now
		\begin{align}
			\deg\left( (t)_{0}^{X} \right) & = \sum_{\textrm{ord}_{Q}(t)>0}m_{Q} \sum_{P\mapsto Q}e(P) \nonumber \\
			& \leq \sum_{\textrm{ord}_{Q}(t)>0}m_{Q} \cdot n \nonumber \\
			& = \deg\left((t)_{0}^{Y}\right)\cdot n \nonumber \\
			& = mn. %&& \cite[\text{Prop 4, pg.194}]{fulton} \label{eq1}
		\end{align}
		
		
	On the other hand
		\begin{equation}
			\deg\left((t)_{0}^{X}\right)=[K(X):k(t)]=m\cdot n \label{eq2}
		\end{equation}
	by \cite[Prop 4, pg.194]{fulton} and the tower law.
	

	Combining \eqref{eq1} and \eqref{eq2} we have that $\sum_{\textrm{ord}_{Q}(t)>0}m_{Q} \sum_{P\mapsto Q}e(P)=m\cdot n$.
	As we know $\sum_{\textrm{ord}_{Q}(t)>0}m_{Q}=\deg\left((t)_{0}^{Y}\right)=m$ by \cite[prop 4, pg. 194]{fulton} and that $\sum_{P\mapsto Q}e(P)\leq n$ for each $Q$ by the above, it follows that $\sum_{P\mapsto Q} e(P)=n$. 
\end{proof}

The following corollary uses the previous proposition to relate the order of an element of $K(Y)$ to the order of its image in $K(X)$. \\


\begin{cor}
	For any $h\in K(Y)$ and any $Q\in Y$ then
		\begin{equation*}
			\sum_{P\mapsto Q} \ord_{P} (f^*(h))=n\cdot \ord_{Q}(h).
		\end{equation*}
	In particular we have $\deg( \di(f^*(h)))=n\cdot \deg( \di (h))$.
\end{cor}
\begin{proof}
	Then $\ord_{Q}(h)=r$ for some $r\in \mathbb Z$ and $h=ut^{r}$ for some unit $u$ and uniformising parameter $t$ both in $\cO_{Q}(Y)$. 
	Now for each $P\in f^{-1}(Q)$ we have that 
		\begin{eqnarray*}
			\ord_{P}(f^*(h)) & = & \ord_{P}(f^*(u)f^*(t^{r})) \\
			&  = & \ord_{P}(f^*(u))+\ord_{P}(f^*(t^{r})) \\
			& = &  0 + r\cdot \ord_{P}(f^*(t)).  \\
		\end{eqnarray*}
	The last line follows as $u$ is a unit in $\cO_Q(Y)$, hence for any $P\in f^{-1}(Q)$ we have $f^*(u)(P)=u(f(P))\neq 0$ and $\ord_P(f^*(u))=0$.

	So
		\begin{align*}
			\sum_{P\mapsto Q}\ord_{P}(f^*(h)) & =  r\cdot \sum_{P\mapsto Q} \ord_{P}(f^*(t)) \\
			& = r\cdot \sum_{P\mapsto Q} e(P) \\
			& = r\cdot n && \mbox{[Prop. 1]} \\
			& = \ord_{Q}(h)\cdot n, \\
		\end{align*}
	thus completing the proof.
\end{proof}


\begin{lem}
	If $t$ is a uniformising parameter in $\cO_{Q}(Y)$ then $\ord_{P}(d(f^*(t)))=e(P)-1$ for each $P\in f^{-1}(Q)$ whenever $\cha(k)\nmid e(P)$.
\end{lem}
\begin{proof}
	By definition of $e(P)$ there is a unit $u$ and a uniformising parameter $s$ in $\cO_{P}(Y)$ such that $f^*(t)=us^{e(P)}$.
	So $df^*(t)=d(us^{e(P)})$.

	As $d$ is a derivation we have that $d(us^{e(P)})=ud(s^{e(P)})+s^{e(P)}d(u)$, and by a simple inductive argument we can also see that $d(s^{e(P)})=e(P)s^{e(P)-1}d(s)$.
	Combining these gives $d(us^{e(P)})=ue(P)s^{e(P)-1}d(s)+s^{e(P)}d(u)$.
	Now we have
		\begin{equation*}
			\ord_{P}\left(e(P)s^{e(P)-1}d(s)\right)= \ord_{P}(e(P)s^{e(P)-1})=e(P)-1.
		\end{equation*}
	The last equality holds as $\mbox{char}(k)\nmid e(P)$.
	The first holds by definition \cite[pg.107]{fulton}.

	Also
		\begin{equation*}
			\ord_{P}(s^{e(P)}d(u))\geq e(P).
		\end{equation*}
	Hence we see that
		\begin{align*}
			\ord_{P}(d(f^*(t))) & = \ord_{P}(ue(P)s^{e(P)-1}ds+s^{e(P)}d(u)) \\
			& =  \mbox{min}\{ \ord_{P}(ue(P)s^{e(P)-1}ds),s^{e(P)}d(u) \} && \mbox{\cite[pg. 48]{fulton}}\\
			& = e(P)-1. \\
		\end{align*}
\end{proof}

Note that we required that $\cha(k)\nmid e(P)$.
If $k$ is of finite characteristic and $\cha(k)\nmid e(P)$ for any $P\in X$ then we say that $f$ is tamely ramified.
Otherwise we say $f$ is wildly ramified.

We record the information about ramification in a divisor, which we call the ramification divisor, in order to relate the canonical divisors over $X$ and $Y$.
In the tamely ramified case we will define it to have co-efficients $e(P)-1$, but to extend this to the wildly ramified case we will need the following the definition of higher ramification groups.\\

\begin{defn}
	Let $G:=\gal(X/Y)$ and let $t$ be a uniformising parameter at $P\in X$.
	Then for $i\geq -1$ we define the $i^{th}$ ramification group at $P$, denoted $G_i$, to be the sub-group of $s\in G$ such that $\ord_P(s(t)-t)\geq i+1$.
	This is	independent of the choice of $t$, as shown in { \em \cite[$\S$1, Ch. IV]{localfields}}.
\end{defn}

Note that $G_{-1}=G$, if $i$ is sufficiently large then $G_i$ is trivial and $G_i\supseteq G_{i+1}$.
Also, $G_1$ is a $p$-group and $\ord(G_0/G_1)$ is co-prime to $p$.
In particular, $\pi$ is tamely ramified at $P$ if and only if $G_1$ is the trivial group.
More details can be found in \cite[Ch. IV]{localfields}.
We can now give the following definition.\\

\begin{defn}\label{ramdiv}[Ramification Divisor]
	For each $P\in X$ we define 
	\[
	\delta_P:=\sum_{i=0}^{\infty}(\ord(G_i)-1).
	\]
	Then we define the ramification divisor to be $R:=\sum_{P\in X}\delta_P[P]$.
\end{defn}

Note that $e_P = \ord (G_0(P))$ for any $P \in X$.
Also, for any $Q \in Y$ we write $\delta_Q$ for $\delta_P$ and $e_Q$ for $e_P$, where $P\ \in f^{-1}(Q)$.\\

\begin{thm}
	Suppose that $P_1,\ldots ,P_r$ are the ramification points of a map $\pi:X\rightarrow Y$.
	Suppose further that $\pi_i$ is a uniformising parameter in $\cO_{P_i}(X)$ for each  $i=1,\ldots ,r$.
	Then we define $i_G(s):=\ord_{P_i}(s(\pi_i)-\pi_i)$ for each $s\in G$.
	If $e\in G$ is the identity then Hilbert's formula says
	\[
	    \sum_{s\neq e}i_G(s)=\sum_{i=0}^{\infty}\left(\ord(G_i)-1\right).
	\]
	This can be of use in calculations involving the ramification divisor.
	This formula is true in all cases, both ramified and unramified.
\end{thm}
\begin{proof}
	For the sake of brevity we do not prove these statements here. See \cite[Prop 4, $\S$1, Ch IV]{localfields} for a proof of Hilbert's formula.
	For a proof of the Hurwitz formula, see \citep[Cor 2.4, Ch IV]{hart}.
\end{proof}



Recall that the push forward of a divisor $D = \sum_{P\in X}n_P[P]$ by $f$ is defined to be $f_*(D):= \sum_{Q\in Y} \sum_{P\mapsto Q} n_P [Q]$.
We are now in a position to state the main theorem of this section.\\

\begin{thm}\label{detailedhur}
Let $x\in K(X)$ be a non-zero element of the function field. 
Then 
\[
\di (f^* (dx)) = f_*(\di (dx)) + R.
\] 
\end{thm}
\begin{proof}
We will not prove this result here, though it is stated in \cite[Thm III.4.6]{stichtenoth}.
\end{proof}

The following theorem follows immediately, but is stated here as we will often use through out this report.\\

\begin{thm}\label{hur}[Hurwitz Formula]
	If $f$ is tamely ramified then we have
		\begin{equation*}
			2g_{X}-2=n(2g_{Y}-2)+\deg(R).
		\end{equation*}
\end{thm}

\begin{proof}
	This follows immediately by taking the degrees of the divisors in the previous theorem.


	Since we did not prove the previous theorem, we will prove this one in the tamely ramified case, where the statement simplifies to
		\[
			2g_X-2 = n(2g_Y-2) + \sum_{P\in X} e(P)-1.
		\]

	It suffices to show that the degree of a canonical divisor on $X$ is precisely 
		\begin{equation*}
			n\cdot (2g_{Y}-2)+\sum_{P\in X}(e(P)-1)
		\end{equation*}
	as, by \cite[Cor, pg.107]{fulton}, any canonical divisor $W$ on $X$ has degree $2g_X-2$.

	Suppose that $\omega_{Y}$ is a non-zero meromorphic differential on $Y$.
	Then $f^*(\omega_{Y})$ is a non-zero differential on $X$.

	
	Since
		\begin{equation*}
			\deg(\di(f^*(\omega_{Y}))) = \sum_{Q\in Y} \sum_{P\mapsto Q} \ord_{P}(f^*(\omega_{Y}))
		\end{equation*}
	and
		\begin{eqnarray*}
			\sum_{Q\in Y} \left(n\cdot \ord_{Q}(\omega_{Y}) +\sum_{P\mapsto Q}(e(P)-1)\right)
			& = & n\cdot \deg(\di(\omega_{Y})) +\sum_{P\in X} (e(P)-1) \\
			& = & n\cdot (2g_{Y}-2) + \sum_{P\in X}(e(P)-1)
		\end{eqnarray*}
	
	it suffices to show that for each $Q\in Y$ the following holds:
		\begin{equation*}
			\sum_{P\mapsto Q}\ord_{P}\left(f^*(\omega_{Y})\right)=n\cdot \ord_{Q}(\omega_{Y})+\sum_{P\mapsto Q}(e(P)-1).
		\end{equation*}
	Fix $Q\in Y$.
	Then for some uniformising parameter $t\in \cO_{Q}(Y)$ and some $h\in K(Y)$ we have that $\omega_{Y}=hdt$ by \cite[Prop 6, pg. 105]{fulton}.

	So now we have
		\begin{eqnarray}\label{once}
			\sum_{P\mapsto Q}\ord_{P}(f^*(\omega_{Y})) & = & \sum_{P\mapsto Q}\ord_{P}(f^*(hdt)) \nonumber \\
			& = & \sum_{P\mapsto Q} \ord_{P}(f^*(h)) + \sum_{P\mapsto Q} \ord_{P}(d(f^*(t))) \nonumber \\
			& = & n\cdot \ord_{Q}(h) + \sum_{P\mapsto Q} (e(P)-1) \\
		      & = & n\cdot \ord_{Q}(\omega_{Y}) + \sum_{P\mapsto Q} (e(P)-1) \nonumber
		\end{eqnarray}
	with line \eqref{once} following from the previous corollary and lemma, and the proof is complete.
\end{proof}



\begin{cor}
	If $Y=\mathbb{P}^{1}$ and $n>1$ then there exist ramification points (i.e. points  $P\in X$ such that $e(P)>1$).
\end{cor}
\begin{proof}
	As $Y=\mathbb{P}^{1}$ we have that $g_{Y}=0$.
	It follows that 
		\begin{equation*}
			2g_{X}-2=-2n+\deg(R)
		\end{equation*}
	 and hence
		\begin{equation*}
			2g_{X}=-2(n-1)+\deg(R).
		\end{equation*}
	As we know that $g_{X}\geq 0$ and that $n>1$, it follows that there exists a $P\in X$ such that $\delta_P\geq 1$, and $P$ is a ramification point.
\end{proof}


\newpage
