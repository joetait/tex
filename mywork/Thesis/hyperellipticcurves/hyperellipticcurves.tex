\chapter{Hyperelliptic curves} \label{chapterhyperellipticcurves}


In this section we recall the definition and basic details of hyperelliptic curves, and then go on to compute bases for the space of holomorphic differentials and polydifferentials.
The next chapter will then make use of this, working almost exclusively with hyperelliptic curves.
We will also use them to effectively demonstrate the underlying  \todo{fill in} ideas of the main theorem in the final chapter.

\section{Hyperelliptic curves, characteristic unequal to 2}\label{charneq2}
Let $X$ be a smooth, projective, connected, hyperelliptic algebraic curve of genus $g$ over an algebraically closed field $k$, with $\cha (k) \neq 2$.
Let $x:X\rightarrow \mathbb{P}_k^1$ be the corresponding holomorphic map of degree $2$.
This curve then has a corresponding degree two extension $k(x,y)$ of the rational function field in one variable over $k$, $K(\mathbb P_k^1) = k(x)$, where
   \begin{equation}\label{definingequation}
   y^2 = f(x)
   \end{equation}
for some $f(x)\in k[x]$ with no repeated roots (see, for example, \cite[Rem. 7.4.25]{liu}).
Also, the degree of $f(x)$ must be $2g+1$ or $2g+2$, as will be shown in the next paragraph.
In the first case then $\infty \in \mathbb P_k^1$ will be a branch point, where as in the second case it is not.

We now show some preliminary results about $x$. 
Firstly, $x$ is ramified at precisely $2g + 2$ branch points.
This follows from computing the degree of the ramification divisor $R$ of $x$.
By the Riemann-Hurwitz formula, Corollary \ref{corhurwitzformula},
    \[ 
    \deg(R) = 2g -2 +2\cdot 2 = 2g + 2.
    \]
Since $x$ is of degree $2$ and at most tamely ramified, the coefficients of the ramification divisor are $1$ (see the start of Section \ref{dimsection}), and hence $R = p_1 + \ldots + p_{2g+2}$, for distinct $p_i\in X$.
We let $a_i = x(p_i)$ and $D_i = \di_{a_i}(x)$.
Moreover, for a general point $a \in \mathbb P_k^1$ we let $D_a = \di_x(x)$.
We now show that the ramification points correspond to the zeroes of $f(x)$.
Since there are no repeated roots in $f(x)$, then \eqref{definingequation} defines a non-singular affine curve $X'$, with a degree two projection $x': X'\rightarrow \mathbb A_k^1$.
Clearly for any point $a\in \mathbb A_k^1$ which is not a solution to $f(x)$, then there are two points in the pre-image, namely $(a,\pm \sqrt{a})$, and the point is not ramified.
On the other hand, if $a\in \mathbb A_k^1$ is a solution to $f(x)$, then there is only one point in the pre-image and hence it is ramified.
Since there are $2g+2$ solutions of $f(x)$, and the degree of $R$ is $2g+2$, then we conclude that these are all the ramification points.

Next we will need to define precisely what the divisor of a poly-differential is.
If we consider an element of the tensor product $\omega \in \Omega_X^{\otimes m}$ then it can be locally written as $y dx_1\otimes \ldots \otimes dx_m$, where $x_i \in K(X)$ for all $1 \leq i \leq m$.
Let $P$ be a point in $X$.
Since each $dx_i$ can be written as $y_i dt$ for some $y_i\in K(X)$ and some uniformising parameter $t$ at $P$, we can rewrite $\omega$ as $y' dt \otimes \ldots \otimes dt$, where $y' = y \cdot y_1 \cdots y_m$.
We then define the order of $\omega$ at $P$ to be $\ord_P(\omega ) := \ord_P(y')$.
In the particular case where $\omega = fdx \otimes \ldots \otimes fdx = f^m dx^{\otimes m}$, then we have $y_1 = \ldots = y_m = z$ for some $z$ when we change $x$ to a uniformising parameter.
Hence in this instance 
    \[ 
    \ord_P(\omega) = \ord_P(z^m) = m\ord_P(z) = m\ord_P(dx).
    \]



    \begin{prop}\label{prophyperellipticbasispnot2}
    Let $m\geq 1$.
    Let $X$, $x$ and $y$ be as above, and let $\omega := \frac{dx^{\otimes m}}{y^m}$. 
    Then if $g\geq 2$, a basis of $H^0(X,\Omega_X^{\otimes m})$ is given by:
        \begin{itemize}
        \item $\omega, x\omega, \ldots , x^{g-1}\omega$ if $m=1$ 
        \item $\omega, x\omega, x^2\omega$  if $m=g=2$ 
        \item $\omega, x\omega, \ldots, x^{m(g-1)}\omega;\  y\omega, xy\omega, \ldots x^{(m-1)(g-1)-2}y\omega$ otherwise.
        \end{itemize}
    \end{prop}
    
    \begin{rem}
    Note that the case where $m=1$ is treated in \cite[Prop. 7.4.26]{liu} and \cite[Ch. IV, \S 4, Prop. 4.3]{griffiths}.
    \end{rem}
    
    \begin{proof}
    We first show that the elements are linearly independent over $k$.
    Since $\omega$ is fixed, it is equivalent to show that the coefficients are linearly independent over $k$ - \ie that $1,x,\ldots ,x^n, y, xy, x^ly$ are linearly independent over $k$ for any $n$ and $l$.
    But the only part of this which is not clear is when the coefficients are not just in $x$, but then it follows since the extension would otherwise be of degree 1.
    To show that they are indeed holomorphic differentials, we show that their divisors are greater than $0$.
    We first compute the divisors associated to $x$, $y$ and $dx^{\otimes m}$.
    
    Let $D_0 = [p']+[q']$ with $p',q' \in X$ (note that we could have $p' = q'$) be the divisor of zeroes of $x$. 
    Then $ i D_0$ is the divisor of zeroes of $x^i$. 
    If we also let $D_\infty = [p] + [q]$, with $p$ and $q$ defined as above, be the divisor of poles of $x$. 
    So overall $\di (x^i) = i D_0 - i D_\infty$.
    
    
    To compute $\di (dx^{\otimes m})$, we first note that it suffices to compute the divisor of $dx$, since $\di (dx^{\otimes m}) =m\di (dx)$, as noted above.
    Since $x$ can viewed as the projection of $X$ on to the projective line, or as a function on the projective line, we will use $\di_X (dx)$ and $\di_{\mathbb P^1} (dx)$ to differentiate these cases.
    We first recall that the Riemann-Hurwitz formula (Theorem \ref{theoremdetailedhurwitz}) tells us that
        \[
        \di_X (dx) = \pi^*( \di_{\mathbb P^1}(dx)) + R = R - 2D_{\infty},
        \]
    since $\di_{\mathbb P^1}(dx) = -2[\infty]$ and hence $\pi^* (\di_{\mathbb P^1}(dx)) = -2D_\infty$.
    
    Finally, we compute $\di (y)$.
    Since $\di (y^2) = \di (f(x))$ and hence $\di(y) = \frac{1}{2}\di(f(x))$, we need only compute the divisor of $f(x)$.
    As noted earlier, the solutions to $f$ correspond to the ramification points.
    So for any $P\notin x^{-1}(\infty)$ then $\ord_P(y) =  \frac{1}{2}\ord_P(f(x)) = 1$ if $P$ is a ramification point, and $\ord_P(y) = \frac{1}{2}\ord_P(f(x)) = 0$ otherwise.
    Finally, if $P\in x^{-1}(\infty)$, then $\ord_P(y)  = \frac{1}{2}\ord_P(f(x)) = - \frac{1}{2} (2g+2) = -(g+1)$.
    So, overall, we have 
        \[
        \di(y) = \sum_{i=1}^{2g+2} [p_i]- (g+1)D_\infty = R - (g+1)D_{\infty}.
        \]
    
    We now show that the differentials listed in Proposition \ref{prophyperellipticbasispnot2} are holomorphic.
    We consider the first two cases together.
    So if $m=1$ or $m\geq 2$, $g\geq 2$ then we have that
        \begin{align*}
        \di(x^i\omega) & =  \di \left( \frac{x^idx^{\otimes m}}{y^m} \right)\\ 
        & =  i(D_0 -D_\infty) + m(R-2D_\infty) -m(R-(g+1)D_\infty) \\
        & =  iD_0 + (mg -m -i)D_\infty \\
        & =  iD_0 + (m(g-1) -i)D_\infty,
        \end{align*}
    which is positive for $0\leq i \leq m(g-1)$.
    Note that if $m=g=2$ then there are three elements, and since $\dim_kH^0(X,\Omega_X^{\otimes 2})=3$ by Lemma \ref{dim3}, these elements form a basis.
    Also, if $m=1$ then by Riemann-Roch $\dim_k H^0(X,\Omega_X)=g$, and we have $g$ linearly independent elements, so they again must form a basis.
    
    
    
    Now we consider the case where $m\geq 2$ and $g \geq 2$ and at least one of these inequalities is strict.
        \begin{align*}
        \di (x^iy\omega) & =  \di(x^i\omega) + R -(g+1)D_\infty \\
        & =  iD_0 + R +((m-1)(g-1)-2-i)D_\infty,
        \end{align*}
    which is positive for $0\leq i \leq (m-1)(g-1)-2$.
    By Lemma \ref{dim3} we know that 
        \[
        \dim_kH^0(X,\Omega_X^{\otimes m}) = (2m-1)(g-1).
        \]
    Since the number of differentials listed in the last case of the proposition is precisely
        \[
        (m-1)(g-1)-1 + m(g-1) +1 = 2mg -2m -g + 1 = (2m-1)(g-1)
        \]
    it is clear that these elements form a basis.
    \end{proof}

We denote by $\sigma$ the automorphism of $X$ of order 2, which maps each point to its corresponding point in the pre-image of $x$.
Since $\sigma(y) = -y$, and $\sigma$ acts trivially on $x$, we can see that in the case where $m=g=2$ the action will be trivial, since the only power of $y$ is two, and of course $y^2 = (-y)^2$.
In any other case we have an odd power of $y$ in the basis, and hence the action of $\sigma$ is not trivial.
This proves the main theorem for hyperelliptic curves over a field of characteristic not equal to 2.




\section{Hyperelliptic curves, characteristic 2}
Let $X$ be a smooth, projective, connected hyperelliptic algebraic curve of genus $g$ over an algebraically closed field $k$ of characteristic two.
Let $\pi:X \rightarrow \mathbb P_k^1$ be the corresponding holomorphic map of degree two.
By \cite[Prop. 7.4.24]{liu} this curve has a corresponding degree two extension $k(x,y)$ of the function field of one variable over $k$, $k(x)$, where
    \begin{equation}\label{ext}
    y^2 - h(x)y = f(x)
    \end{equation}
for some polynomials $h(x), f(x)\in k[x]$, with maximum degrees of $g+1$ and $2g+2$ respectively.
We now show what conditions the smoothness of the curve imposes on $f(x)$ and $h(x)$.



    \begin{lem}\label{smoothness}
    Since the curve is smooth we have $h(x)$ and $h'(x)^2 f(x) + f'(x)$ have no common zeroes on $X$.
    \end{lem}
    \begin{proof}
    The Jacobian criterion (see, for example, \cite[Thm. 4.2.19]{liu}), states that if the derivatives of \eqref{ext} with respect to $x$ and with respect to $y$ are zero at a point $P\in X$ then the curve is not smooth, and otherwise it is.
    Clearly
        \[
        \frac{d}{dy} (y^2 -h(x)y -f(x)) = h(x)
        \]
    since the characteristic of $k$ is 2.
    On the other hand,
        \[
        \frac{d}{dx} (y^2 - h(x)y -f(x)) = h'(x)y - f'(x).
        \]
    Since this is zero if and only if its square is zero, $h(x)$ and $h'(x)^2 f(x) + f'(x)$ have a common zero if and only if
        \[
        (h'(x)y-f'(x))^2 = h'(x)^2y^2 -f'(x)^2 = h'(x)^2h(x)y + h'(x)^2f(x) - f'(x)^2
        \]
    and $h(x)$ are zero at some $P\in X$.
    But this is the case if and only if $h(x) = 0$ and $h'(x)^2f(x) + f'(x) = 0$ at $P$, which is what we wished to show.
    \end{proof}

By an automorphism of $\mathbb P_k^1$, similarly to the previous section, we can assume that $\infty\in \mathbb P_1^k$ is a branch point.
As will be shown, this means that the degree of $h(x)$ is strictly less than $g+1$.
By \cite[Prop. 7.4.24]{liu} this also forces the degree of $f(x)$ to be $2g+1$.
We will show that this also means that
Let $d$ be the degree of $h(x)$, and let $k< g+1$ be the number of distinct roots of $h(x)$ in $\mathbb A_k^1$.
We will denote by $\sigma$ the automorphism of order two on $X$, which sends each point to its corresponding point in the preimage of $\pi$. 
Note that we have $X/\langle \sigma \rangle \cong \mathbb P_k^1$.


We first describe the ramified points of $\pi$, in order to compute the ramification divisor.
By Lemma \ref{smoothness} if we consider the affine curve defined by this equation it will be smooth.
We denote this curve by $X'$.
Then $\pi$ restricts to a map $X'\rightarrow \mathbb A^1_k$, the projection on to the $x$ co-ordinate.
Let $a\in \mathbb A_k^1$.
Then if $(a,b)$ is a point in $\pi^{-1}(a)$, so is the point $(a,b+h(a))$, which is clearly distinct if and only if $h(a)\neq 0$.
Since the extension is degree two, this shows that the ramified points in the affine part correspond to the roots of $h(x)$.
We denote the zeroes of $h(x)$ by $a_i$ for $1\leq i \leq k$.
For each $a_i$ there is a corresponding $b_i$, which is the square root of $f(a_i)$.
We will also denote the corresponding ramification point by $P_i$.
If the point at infinity is also a branch point, the point that maps to infinity will be written $P_{\infty}$.

We will now compute the ramification divisor.


    \begin{lem}\label{char2ramification}
    Let $n_i$ be the order of $h(x)$ at $a_i\in \mathbb A_k^1$.
    Then the coefficient $\delta_P$ of the ramification divisor $R$ at $P\in X$ is given by
        \[
        \delta_P = \left\{
            \begin{array}{ll}
            2n_i & \text{if }\ P=P_i\ \text{ for some }\ i \in \{1,\ldots ,k\}, \\
            2(g+1-d) & \text{ if }\  P=P_\infty, \\
            0 & \text{ otherwise.} 
            \end{array}
        \right.
        \]
    \end{lem}
    \begin{proof}
    We first show that it will suffice to prove that the coefficient of $[P_i]$ is $2n_i$ for $1\leq i \leq k$.
    Note that by the Riemann-Hurwitz formula $\deg(R) = 2g+2$.
    Then the coefficient at $P_\infty$ is $\deg(R) - \sum_{i=1}^k2n_i$, which, by the Riemann-Hurwitz formula (Corollary \ref{corhurwitzformula}), is equal to $2g+2-2d = 2(g+1-d)$, as stated.
    
    Let $P=P_i$ for some $i\in \{1,\ldots , k\}$.
    Then $y-b_i$ is a local parameter at $P$.
    To see this, note that the maximal ideal $\mathfrak m_{P}$ of the local ring $\cO_{X,P}$ at $P$ is generated by $x-a_i$ and $y-b_i$.
    But $x-a_i\in \mathfrak m_{P}^2$ since $\pi$ is ramified at $P$ with ramification index 2.
    By Nakayama's lemma \cite[Prop. 2.6]{atiyahmacdonald},\todo{check citation} $y-b_i$ is therefore a local parameter at $P$.\todo{more specific citation}
    
    Using Hilbert's formula \cite[Chap. IV, \S 1, Prop. 4]{localfields} we obtain
        \begin{align*}
        \delta_P & =  \sum_{i\geq 0} \left(\ord(G_i(P))-1\right) \\
        & =  \max\left\{ i\in \NN | G_i(P)\neq \{1\}\right\} + 1 \\
        & =  \ord_{P}(\sigma(y-b_i) - (y-b_i)).
        \end{align*}
    
    By an argument similar to that used to show the correspondence between the solutions of $h(x)$ and the ramification points, it is clear $\sigma(y)=y+h(x)$.
    The following calculation then concludes the proof,
        \begin{align*}
        \delta_P & =  \ord_{P}(\sigma(y-b_i) - (y-b_i)) \\
        & =  \ord_{P}(y-b_i+h(x) - y + b_i) \\
        & =  2\ord_{a_i}(h(x)) \\
        & =  2n_i.
        \end{align*}
    \end{proof}



We will now compute the divisors associated to $h(x)$, $x$ and $y$ in $K(X)$, and also to $dx$.
If $a\in \mathbb P_k^1$ is a branch point, we denote the point in the pre-image by $P_a$, otherwise we denote the points in the pre-image by $P_a'$ and $P_a''$.
In order to simplify our calculations we let $D_a := \pi^*([a])$ for any $a\in \mathbb P_k^1$.
These will allow us to consider the case when 0 is or is not a branch point simultaneously. 
We now compute the divisors of all the elements required to form a basis of $H^0(X,\Omega_X^{\otimes m})$.

    \begin{lem}\label{xchar2}
    The divisor of $x\in K(X)$ is 
        \[
        \di (x)= D_0 - D_\infty.
        \]
    \end{lem}
    \begin{proof}
    Given our notations above, this is clear.
    \end{proof}


    \begin{lem}\label{dxchar2}
    Let $m\geq 1$.
    The divisor associated to the poly-differential $dx^{\otimes m}$ is 
        \[
        \di (dx^{\otimes m}) =mR + m(g-1-d)D_\infty
        \]
    \end{lem}
    \begin{proof}
    We first note that it suffices to compute $\di (dx)$, since $\di (dx^{\otimes m}) = m\di (dx)$, as described above.
    
    Now we compute the divisor of $dx$.
    We will need to consider the divisor of $dx$ both as a differential on $X$ and on $\mathbb P_k^1$. 
    We will use the notation of $\di_X(dx)$ and $\di_{\mathbb P^1}(dx)$ to differentiate between the two cases.
    Then the Riemann-Hurwitz formula (Theorem \ref{theoremdetailedhurwitz}) states that
        \[
        \di_X( dx) = \pi^*\di_{\mathbb P^1}(dx) + R.
        \]
    Now we have already computed $R$ in Lemma \ref{char2ramification}, and $\pi^*\di_{\mathbb P^1}(dx) = -2D_\infty$ hence we have
        \[
        \di_X( dx) = \sum_{i=1}^k 2n_iP_i + (g+1-d)D_\infty - 2D_\infty = \sum_{i=1}^k D_{a_i} + (g-1-d)D_\infty.
        \]
    Multiplying through by $m$ we obtain the desired result.
    \end{proof}



    \begin{lem}\label{h(x)char2}
    The divisor associated to $\frac{1}{h(x)}$ is
        \[
        \di \left(\frac{1}{h(x)}\right) = - \sum_{i=1}^k D_{a_i} + dD_\infty = dD_\infty - R
        \]
    \end{lem}
    \begin{proof}
    Since $\pi$ is ramified at infinity then $\ord_{P_{\infty}}\left(\frac{1}{h(x)}\right) = -\ord_{P_{\infty}}(h(x)) = 2d$.
    If it is not ramified, then $\ord_{P_{\infty}'}\left(\frac{1}{h(x)}\right) = \ord_{P_{\infty}''}\left(\frac{1}{h(x)}\right)=d$.
    For the ramified points $P_i$, $1\leq i \leq k$, then $\ord_{P_i}\left(\frac{1}{h(x)}\right) = -\ord_{P_i}(h(x))= -2n_i$.
    At any other point of $X$ the order of $\frac{1}{h(x)}$ is clearly zero.
    \end{proof}

We again consider the affine curve $X'$, and we let $\alpha_i \in \mathbb A_k^1$, for $1\leq i\leq l \leq 2g+1$, be the zeroes of $f(x)$.
If $\alpha_i$ corresponds to a branch point then we label the unique point in $X$ that maps to it $Q_i$.
Otherwise there are two points that map to it, and we denote these 
    \[  
    Q_i:=(\alpha_i,0)\ \text{ and}\ Q_i':=(\alpha_i,h(\alpha_i)).
    \]


    \begin{lem}\label{ychar2}
    Let $m_i$ be the order of $f(x)$ at $\alpha_i\in \mathbb A_k^1$.
    Then 
        \[
        \di(y) = \sum_{i=1}^l m_i[Q_i] - (2g+1)[P_\infty].
        \]
    \end{lem}
    \begin{proof}
    First note that if $P\in X'$ is not a zero of $f(x)$ (\ie $P\neq Q_i$ for any $i$), then it is clear that $y|_P \neq 0$.
    Indeed, if it were zero then $f(x)|_P = y^2 + h(x)y|_P = 0$, a contradiction.
    
    If $Q_i$ is not ramified then $h(\alpha_i) \neq 0$, and $y|_{Q_i} = 0$, hence $y+h(x)$ is a unit at $Q_i$.
    Since $y(y+h(x)) = f(x)$, we have
        \[
        \ord_{Q_i}(y) = \ord_{Q_i}\left(\frac{f(x)}{y+h(x)}\right) = \ord_{Q_i}(f(x)) -\ord_{Q_i}(y+h(x)) = m_i.
        \]
    
    This only leaves the case when $Q_i$ is ramified. 
    In this case we must have $m_i=1$.
    Otherwise we would also have $f'(x)|_{Q_i} = 0$, and since we also have $h(x)|_{Q_i} = f(x)|_{Q_i} = 0$ this would contradict Lemma \ref{smoothness}.
    So if we let $\tilde f(x) := \frac{f(x)}{x-\alpha_i}$ and $\tilde h(x) := \frac{h(x)}{x-\alpha_i}$ then $\tilde f(x)$ is a unit at $Q_i$.
    Then we have 
        \[
        y^2 = f(x) - yh(x) = (x-\alpha_i)(\tilde f(x) - \tilde h(x)y).
        \]
    Hence
        \[
        \ord_{Q_i}(y^2) = \ord_{Q_i}(x-\alpha_i) + \ord_{Q_i}(\tilde f(x) - \tilde h(x)y).
        \]
    But since $\tilde f(x) - \tilde h(x) y$ does not have an affine pole, and $\tilde f(x)$ is a unit at $Q_i$, and $Q_i$ is a zero of $\tilde h(x) y$, it follows that $\ord_{Q_i}(\tilde f(x) -  \tilde h(x) y) = 0$
    and hence $\ord_{Q_i}(y^2) = \ord_{Q_i}(x-\alpha_i) = 2$, and we have determined the zero divisor of $y$.
    
    Since there is only one point at which $y$ can have a pole, namely $P_\infty$, and since the degree of $\di (y)$ is zero, then we have the pole divisor must be $(2g+1)[P_\infty]$, and this completes the proof.
    \end{proof}



We now prove the following proposition, determining a basis of the space of global holomorphic poly-differentials as a vector space over $k$.

    \begin{prop}\label{propbasishyperellipticp=2}
    We assume that $g\geq 2$ and let $\omega:= \frac{dx^{\otimes m}}{h(x)^m}$. 
    Then if $g\geq 2$, a basis of $H^0(X,\Omega_X^{\otimes m})$ is given by
        $\begin{cases}
        \omega, x\omega, \ldots , x^{g-1}\omega &  \mbox{if}\ m=1 \\
        \omega, x\omega, x^2\omega & \mbox{if}\ m=g=2 \\
        \omega, x\omega, \ldots, x^{m(g-1)}\omega;\  y\omega, xy\omega, \ldots x^{(m-1)(g-1)-2}y\omega & \mbox{otherwise.}
        \end{cases}$
    \end{prop}
    
    \begin{rem}
    Note that the case where $m=1$ can be found in \cite[Prop. 7.4.26]{liu}.
    \end{rem}

    \begin{proof}
    We first assume that above elements are holomorphic poly-differentials, and show that they then form a basis.
    To show that the elements are linearly independent over $k$ we need only show that the coefficients are, since $\omega$ is fixed.
    The only case where this is not clear is when the coefficients contain both $x$ and $y$ terms.
    But since the $y$ terms are all linear, and the extension is of degree two, it must follow that coefficients are linearly independent.
    
    
    In the case that $m=1$ then we have that $\dim_k H^0(X,\Omega_X) =g$ by Lemma \ref{dim3}, and there are $g$ elements described in the statement of the proposition in this case, so they must form a basis.
    If $m \geq 2$ then $\dim_k H^0(X,\Omega_X^{\otimes m}) = (2m-1)(g-1)$.
    If $m=g=2$ then $(2m-1)(g-1) = 3$, and there are three elements listed in the proposition.
    On the other hand if $m\geq 2$ and $g > 2$ the proposition lists
        \[
        m(g-1)+1 + (g-1)(m-1)-2+1 = 2mg -2m -g +1 = (2m-1)(g-1)
        \]
    elements, and again they must form a basis.
    
    We now show that the listed poly-differentials are holomorphic, \ie that their divisors are non-negative.
    Firstly we have
        \begin{align*}
        \di(x^i\omega) & =  \di \left( \frac{x^i dx^{\otimes m}}{h(x)^m} \right)\\ 
        & =  i(D_0 - D_\infty) +2m\sum_{i=1}^k n_i[P_i] + m(g-1-d)D_\infty\\
        &  -2m\sum_{i=1}^k n_i [P_i] + dmD_\infty \\
        & =  iD_0 + (m(g-1) -i)D_\infty
        \end{align*}
    by Lemmas \ref{xchar2}, \ref{dxchar2} and \ref{h(x)char2}, and this is clearly non-negative for $0\leq i \leq m(g-1)$.
    
    Similarly the divisor 
        \begin{align*}
        \di(x^iy\omega) & =  \di \left( \frac{x^i ydx^{\otimes m}}{h(x)^m} \right)\\ 
        & =  iD_0 + (m(g-1) -i)D_\infty + \sum_{i=1}^l m_i[Q_i] - (2g+1)[P_\infty] \\
        & =  iD_0 +  \sum_{i=1}^l m_i[Q_i] + (2m(g-1) -(2g+1) -2i)[P_\infty] \\
        & =  iD_0 +  \sum_{i=1}^l m_i[Q_i] + (2((m-1)(g-1) -1 -i)-1)[P_\infty]
        \end{align*}
    is again clearly non-negative for $0 \leq i \leq (g-1)(m-1)-2$.
    
    \end{proof}

Since $\sigma$ acts trivially on $x$, it is clear that in both the $m=1$ case and the case where $m=g=2$ that the group action is trivial.
On other hand, since $\sigma(y) = y+h(x)$, we can see that in the other cases the action is not trivial, as there are $y$ coefficients of the basis elements.
Hence this proves the main theorem in the case of hyperelliptic curves in characteristic two.


