\chapter{Faithful actions on Riemann-Roch spaces} \label{Chapter:Faithfulactions}

In this section our main aim is to compute when a subgroup of the automorphism group of an algebraic curve acts faithfully on the space of holomorphic differentials and polydifferentials.
Our approach uses the fact that if any finite group $G$ does not act faithfully on $H^0(X,\Omega_X^{\otimes m})$ then there exists a subgroup of $G$ which fixes at least one element of this $k$ vector space.
We then use dimension formulae to see when this occurs.

To this end, we start by computing the dimension of $H^0(X,\Omega_X^{\otimes m})$, and the dimension of the fixed space $H^0(X,\Omega_X^{\otimes m})^G$.
These dimensions rely primarily on the genus of the quotient curve $Y:=X/G$, $m$ and the ramification divisor of $\pi \colon X \ra Y$.

Then we use these formulae to compute exactly when a cyclic group of prime order will act trivially on $H^0(X,\Omega_X^{\otimes m})$.
When we are considering holomorphic differentials (\ie when $m=1$), this depends solely on the characteristic of $k$, whilst for polydifferentials (\ie when $m \geq 2$) this is actually independent of $\cha (k)$, and is determined by the genus of $X$, $m$ and the order of the group.
In the same section we also extend these results to more general Riemann-Roch spaces.

We then move on to the main theorem, which answers the question of when $G$ acts faithfully on $H^0(X,\Omega_X^{\otimes m})$.
After proving this theorem we give examples which illustrate both when we do and do not have faithful actions.
In particular, we use results of Chapter 3 to explicitly show the result holds for hyperelliptic curves.

We close the chapter with an alternative proof of when a cyclic group of prime order acts faithfully on $\hzero$, by studying the $k[G]$-module structure of $\hzero$, which was determined in \cite{valmadan}.

\section{Dimension formulae}\label{dimsection}

Throughout this chapter, unless otherwise stated, we assume that $X$ is a connected, smooth, projective algebraic curve over an algebraically closed field $k$ of characteristic $p \geq 0$.
We furthermore assume that $G$ is a finite group of order $n$ that acts faithfully on $X$.
Note that $G$ also induces an action on the vector space $H^0(X,\Omega_X^{\otimes m})$ of global holomorphic poly-differentials of order $m$.
We let $Y$ denote the quotient curve $X/G$, and we let $\pi\colon X\rightarrow Y$ be the canonical projection.
Finally, we denote by $g_X$ and $g_Y$ the genus of $X$ and $Y$ respectively, and we let $K_X$ and $K_Y$ be canonical divisors on $X$ and $Y$.\todo{put comment in previous section regarding divisors}


In this section we compute the dimension of $H^0(X,\Omega_X^{\otimes m})$ and of $H^0(X,\Omega_X^{\otimes m})^G$, the subspace of $H^0(X,\Omega_X^{\otimes m})$ fixed by $G$.


By Corollary \ref{dim=gc} we have $\dim_kH^0(X,\Omega_X)=g_X$.
If $m\geq 2$, we have the following formula for $\dim_kH^0(X,\Omega_X^{\otimes m})$.

    \begin{lem}\label{dim3}
    Let $m\geq 2$. Then
        \begin{equation}
        \dim_kH^0(X,\Omega_X^{\otimes m}) =
            \begin{cases}
            0 & \mbox{if } g_X=0,\\
            1 & \mbox{if } g_X=1,\\
            (2m-1)(g_X-1) & \mbox{otherwise}.
            \end{cases}
        \end{equation}
    \end{lem}
    \begin{proof}
    The trivial cases of $g_X =0$ and $g_X=1$ are explicitly explained in examples (a) and (b) in section \ref{examplessection}.
    
    If $g_X\geq 2$ then $\deg(K_X)\geq1$, so $\deg(mK_X)>\deg(K_X)$.
    Since $H^0(X,\Omega_X^{\otimes m}) \cong H^0(X,\Omega_X(mK_X))$ it then follows from the Riemann-Roch theorem (Theorem \ref{theoremriemannroch}) that
        \[
        \dim_kH^0(X,\Omega_X^{\otimes m})=\deg(mK_X)+1-g_X=(2m-1)(g_X-1).
        \]
    \end{proof}

We now introduce some notations. 
Let $D=\sum_{P\in X}n_P[P]$ be a $G$-invariant divisor on $X$ (\ie $n_{\sigma(P)} = n_P$ for all $\sigma \in G$ and $P\in X$) and let $\cO_X(D)$ denote the corresponding equivariant invertible $\cO_X$-module. 
Furthermore, let $\pi_*^G(\cO_X(D))$ denote the sub-sheaf of the direct image $\pi_*(\cO_X(D))$ fixed by the obvious action of $G$ on $\pi_*(\cO_X(D))$.
We also let $\left\lfloor \frac{\pi_*(D)}{n} \right \rfloor$ denote the divisor on $Y$ obtained from the push-forward $\pi_*(D)$ by replacing the coefficient $m_Q$ of $Q$ in $\pi_*(D)$ with the integral part $\left \lfloor \frac{m_Q}{n} \right \rfloor$ of $\frac{m_Q}{n}$ for each $Q \in Y$. 
The function fields of $X$ and~$Y$ are denoted by $K(X)$ and $K(Y)$ respectively. 
Finally, for any $P \in X$ let $\ord_P$ and $\ord_Q$ denote the respective valuations of $K(X)$ and $K(Y)$ at $P$ and $Q:=\pi(P)$.



The next lemma is the main idea in the proof of our formula for $\dim_kH^0(X,\Omega_X^{\otimes m})^G$, see Proposition \ref{dim}. 



    \begin{lem}
    Let $D=\sum_{P\in X}n_P[P]$ be a $G$-invariant divisor on $X$.
    Then the sheaves $\pi_*^G(\cO_X(D))$ and $\cO_Y\left(\left\lfloor \frac{\pi_*(D)}{n}\right \rfloor\right)$ are equal as subsheaves of the constant sheaf $K(Y)$ on $Y$. 
    In particular, the sheaf $\pi_*^G(\cO_X(D))$ is an invertible $\cO_Y$-module.
    \end{lem}
    \begin{proof}
    For every open subset $V$ of $Y$ we have 
        \[
        \pi_*^G(\cO_X(D))(V) = \cO_X(D) (\pi^{-1}(V))^G \subseteq K(X)^G = K(Y).
        \]
    In particular, both sheaves are subsheaves of the constant sheaf $K(Y)$ as stated. 
    It therefore suffices to check that their stalks are equal. 
    For any $Q \in Y$ and $P \in \pi^{-1}(Q)$.
    We have
        \begin{align*}
        \lefteqn{\pi_*^G\left(\cO_X(D)\right)_Q = \cO_X(D)_P \cap K(Y)}\\
        &= \left\{f \in K(Y): \ord_P(f) \ge -n_P\right\}\\
        &= \left\{f \in K(Y): \ord_Q(f) \ge - \frac{n_P}{e_P}\right\}\\
        &= \left\{ f \in K(Y): \ord_Q(f) \ge - \left\lfloor\frac{n_P}{e_P} \right\rfloor \right\}\\
        &= \cO_Y\left(\left\lfloor \frac{\pi_*(D)}{n} \right\rfloor\right)_Q,
        \end{align*}
    as desired.
    \end{proof}

The following proposition contains the aforementioned formula for the dimension of the subspace of $H^0(X,\Omega_X^{\otimes m})$ fixed by $G$.
In particular we see that this dimension is completely determined by $m$, $g_Y$ and $\deg \left\lfloor \frac{m\pi_*(R)}{n} \right\rfloor$.

    \begin{prop}\label{dim}
    Let $m\geq 1$. Then the dimension of $H^0(X,\Omega_X^{\otimes m})^G$ is equal to
        \[
        \dim_k \left( H^0(X,\Omega_X^{\otimes m})^G \right) = (2m-1)(g_Y-1) + \deg\left\lfloor\frac{m\pi_*(R)}{n} \right\rfloor,
        \]  
    unless one of the following conditions holds:
        \begin{itemize}
        \item $m=1 \mbox{ and } \deg\left\lfloor\frac{m\pi_*(R)}{n}\right\rfloor = 0$ or
        \item $g_Y=1 \mbox{ and } \deg\left\lfloor\frac{m\pi_*(R)}{n}\right\rfloor = 0$ or
        \item  $g_Y=0 \mbox{ and } \deg\left\lfloor\frac{m\pi_*(R)}{n}\right\rfloor < 2m-1$,
        \end{itemize}
    in which case 
        \[
        \dim_k \left( H^0(X,\Omega_X^{\otimes m})^G \right) = g_Y.
        \]      
    \end{prop}
    \begin{proof}
    Let $E$ denote the divisor $\left\lfloor \frac{\pi_*(mK_X)}{n} \right\rfloor$ on $Y$. As $K_X=\pi^*(K_Y)+R$ we have
        \[ 
        E = 
        \left \lfloor \frac{\pi_*\pi^*(mK_Y) + \pi_*(mR)}{n} \right \rfloor =
        mK_Y + \left \lfloor \frac{m\pi_*(R)}{n} \right \rfloor.
        \]
    Using the previous lemma we conclude that $\pi_*^G(\Omega_X^{\otimes m}) \cong \cO_Y (E)$ and finally that
        \begin{equation*}
        \dim_k H^0(X,\Omega_X^{\otimes m})^G 
        = \dim_k H^0\left(Y, \pi_*^G(\Omega_X^{\otimes m})\right)
        = \dim_k H^0\left(Y, \cO_Y\left( E \right) \right).
        \end{equation*}
    
    
    In the first case of the proposition, \ie if $m=1$ and $\deg \left\lfloor\frac{m\pi_*(R)}{n} \right\rfloor=0$, then $\left\lfloor\frac{m\pi_*(R)}{n} \right\rfloor$ is the zero divisor and we conclude that 
        \begin{equation*}
        \dim_kH^0(X,\Omega_X)^G = \dim_kH^0(Y, \Omega_Y) = g_Y.
        \end{equation*}
    
    
    In the second case $\left\lfloor \frac{m\pi_*(R)}{n} \right\rfloor$ is again the zero divisor. 
    Furthermore, as $g_Y=1$, the divisor $K_Y$ is equivalent to the zero divisor, and hence $mK_Y$ is too. 
    This means that
        \begin{equation*}
        \dim_kH^0(X,\Omega_X^{\otimes m})^G = \dim_kH^0\left( Y,\cO_Y\left( E \right) \right) 
        = \dim_k  H^0\left( Y,\cO_Y\left( 0 \right) \right)
        = 1.
        \end{equation*}
    
    
    For the third case, by \cite[Chap. IV, ex. 1.3.4]{hart} it suffices to show that $\deg \left( E \right) < 0$.
    As $g_Y=0$ we have $\deg(K_Y)=-2$, so $\deg(mK_Y)=-2m$, and $\deg \left( E \right)$ is indeed negative.
    
    
    
    We will show below that in all other cases $\deg(E) > \deg(K_Y)$, and then the Riemann-Roch formula (Theorem \ref{theoremriemannroch}) will give 
        \begin{align*}
        \lefteqn{\dim_kH^0(X,\Omega_X^{\otimes m})^G = \dim_kH^0\left(Y,\cO_Y\left( E \right)\right)} \\
        & =  1-g_Y+\deg\left(mK_Y+\left\lfloor{\frac{m\pi_*(R)}{n}}\right\rfloor\right) \\
        & =  (2m-1)(g_Y-1)+\deg\left\lfloor{\frac{m\pi_*(R)}{n}}\right\rfloor,
        \end{align*}
    completing the proof for this case.
    
    
    All that remains is to show that $\deg(E)>\deg(K_Y)$ in all other cases.
    Firstly, if $g_Y=0$ and $\deg \left\lfloor\frac{m\pi_*(R)}{n} \right\rfloor \geq 2m-1$ then, since $\deg(mK_Y)=-2m$, we have 
        \[
        \deg \left( E \right) \geq -1 >-2 = \deg(K_Y).
        \]
    Similarly, if $g_Y=1$ and $\deg \left\lfloor\frac{m\pi_*(R)}{n} \right\rfloor >0$ then, as $\deg \left( mK_Y \right)=0$, we have $\deg \left( E \right) > 0 = \deg (K_Y)$.
    If $m=1$ and $\deg \left\lfloor\frac{m\pi_*(R)}{n} \right\rfloor >0$ then clearly $\deg \left( E \right) > \deg (K_Y)$.
    Lastly, if $m\geq 2$ and $g_Y\geq 2$ then $\deg (K_Y) > 0$ and we have 
        \begin{equation*}
        \deg \left( E \right) \geq \deg\left( mK_Y \right) > \deg (K_Y).
        \end{equation*}
    So in all other cases $\deg(E)>\deg(K_Y)$, and the proof is complete.
    \end{proof}


If $m=1$ we reformulate Proposition \ref{dim} in the following slightly more concrete way. 
Let $S$ denote the set of all points $Q\in Y$ such that $\pi$ is not tamely ramified at $Q$ and let $s$ denote the cardinality of $S$. 
Note that $s=0$ if $p$ does not divide $n$.


For the next corollary we recall the notations $e_Q$ and $\delta_Q$ for any $Q\in Y$ defined before Theorem \ref{hilbertsformula}.


    \begin{cor}\label{dim2}
    We have 
        \begin{equation*}
        \dim_kH^0(X,\Omega_X)^G = 
            \begin{cases}
            g_Y & \mbox{if } s=0, \\
            g_Y-1+\sum_{Q\in S}\left\lfloor \frac{\delta_Q}{e_Q} \right\rfloor & \mbox{otherwise}.
            \end{cases}
        \end{equation*}
    \end{cor}
    \begin{proof}
    We have
        \[
        \deg\left\lfloor\frac{\pi_*(R)}{n} \right\rfloor = \sum_{Q\in Y}\left\lfloor\sum_{P\mapsto Q} \frac{\delta_P}{n} \right\rfloor = \sum_{Q\in Y} \left\lfloor \frac{\delta_Q}{e_Q} \right\rfloor.
        \]
    Furthermore we have $\left\lfloor \frac{\delta_Q}{e_Q} \right\rfloor = 0$ if and only if $\delta_Q<e_Q$, \ie if and only if $Q\notin S$. 
    Thus Corollary \ref{dim2} follows from Proposition \ref{dim}.
    \end{proof}

    \begin{rem}
    Note that if $p>0$ and $G$ is cyclic then Corollary \ref{dim2} can be derived from Proposition $6$ in the recent pre-print \cite{kako} by Karanikolopoulos and Kontogeorgis.\todo{check to see if preprint is now published}
    \end{rem}



\section{Trivial action in the cyclic case}

In this section we will look at the case where $G$ is a cyclic group of prime order, or a power of a prime, and determine when $G$ acts trivially on $H^0(X,\Omega_X^{\otimes m})$.
Compared to arbitrary groups, it is considerably easier to compute when these groups act trivially, and we will later see that we can reduce to this case, regardless of what the structure of $G$ is.


Throughout this section, $P_1,\ldots ,P_r \in X$ denote the ramification points of $\pi$ and we write $e_i$ and $\delta_i$ for $e_{P_i}$ and $\delta_P{_i}$.
Also, for $i=1, \ldots, r$, we define $N_i \in \NN$ by $\ord_{P_i}(\sigma(\pi_i) - \pi_i) = N_i +1$, where $\pi_i$ is a local parameter at the ramification point $P_i$ and $\sigma$ is a generator of $G(P_i)$. 
We also assume that $g_X \geq 2$.


    \begin{prop}\label{m=1}
    Let $p  > 0$ and let $G$ be cyclic of order $p$.
    Furthermore, we assume that $g_Y=0$.
    Then $G$ acts trivially on $H^0(X,\Omega_X)$ if and only if $p=2$. 
    \end{prop}
    \begin{proof}
    From \cite[Lem. 1]{Naka} we know that $p$ does not divide $N_i$ for $i=1,\ldots ,r$, a fact we will use several times below. \todo{more specific citation}
    Let $N:= \sum_{i=1}^r N_i$. 
    Using the Riemann-Hurwitz formula, Corollary \ref{corhurwitzformula}, we obtain
        \begin{equation}\label{hur2}
        2g_X - 2 = -2p + (N+r)(p-1)
        \end{equation}
    and hence
        \[
        \dim_kH^0(X,\Omega_X) = g_X =\frac{(N+r-2)(p-1)}{2}.
        \] 
    Since $g_X \ge 0$ we obtain $r \ge 1$; that is, $\pi$ is not unramified. 
    As $\cha(k) = p = \ord(G)$, the morphism $\pi$ is not tamely ramified, and the cardinality $s$ defined before Corollary \ref{dim2} is not zero.
    Therefore we have 
        \[
        \deg \left\lfloor \frac{\pi_*(R)}{p} \right\rfloor =
        \sum_{i=1}^r \left\lfloor \frac{(N_i+1)(p-1)}{p}\right\rfloor 
        \ge \sum_{i=1}^r \left\lfloor \frac{2(p-1)}{p}\right\rfloor = r > 0.
        \] 
    From Corollary \ref{dim2} we then conclude that 
        \begin{align*}
        \dim_kH^0\left(X,\Omega_X\right)^G & =  g_Y - 1 + \sum_{i=1}^r\left\lfloor \frac{\delta_i}{e_i}\right\rfloor \\
        & =  -1 + N + r \sum_{i=1}^r\left\lfloor -\frac{N_i+1}{p}\right\rfloor.
        \end{align*}
    
    If $p=2$, the dimension of both $H^0(X,\Omega_X)$ and $H^0(X,\Omega_X)^G$ is therefore equal to $\frac{N+r-2}{2}$. 
    This shows the if-direction in Proposition \ref{m=1}.
    
    
    
    To prove the other direction we now assume that $G$ acts trivially on $H^0(X, \Omega_X)$.
    For each $i=1, \ldots, r$, we write $N_i = s_i p +t_i$ with $s_i \in \NN$ and $t_i \in \{1, \ldots, p-1\}$. 
    We furthermore put $S:=\sum_{i=1}^r s_i$ and $T:= \sum_{i=1}^r t_i \ge r$. 
    Then we have
        \[ 
        \frac{(N+r-2)(p-1)}{2} =\dim_kH^0(X,\Omega_X)  = \dim_k H^0(X,\Omega_X)^G = N-S-1 .
        \]
    Rearranging this equation we obtain
        \[
        (3-p)N - 2S = (r-2)(p-1) +2  
        \]
    and hence
        \[
        (-p^2 + 3p -2)S = (r-2)(p-1) +2 - (3-p)T.
        \]
    Assuming that $p \ge 3$ this equation implies that
        \[ 
        S = \frac{(r-2)(1-p)-2 + T (3-p)}{(p-1)(p-2)}. 
        \]
    since $-p^2+3p-2 = - (p-1)(p-2)$. 
    
    Because $S \geq 0$, the numerator of this fraction is non-negative, that is
        \begin{align*}
        \lefteqn{0 \le (r-2)(1-p) - 2 + T (3-p)}\\
        &\le  (r-2)(1-p) - 2 + r (3-p)\\
        &= 2 (r-1)(2-p).
        \end{align*}
    Hence we have that $r=1$ and that the numerator is $0$. 
    We conclude that $S=0$ and hence that $T=1$ or $p=3$. 
    If $T=1$ we also have $N=1$ and finally
        \[
        g_X = \frac{(N+r-2)(p-1)}{2} = 0,
        \]
    a contradiction.
    If $T \not=1$ and $p=3$ we obtain $N=T=2$ and finally 
        \[
        g_X = \frac{(N+r-2)(p-1)}{2} =1,
        \] 
    again a contradiction.
    \end{proof}

    \begin{prop}\label{triv}
    Let $m \geq 2$. 
    Suppose that $G$ is a cyclic group of prime order $l$ (which may or may not be equal to $p$) and that $g_Y=0$. 
    Then $G$ acts trivially on $H^0(X,\Omega_X^{\otimes m})$ if and only if $g_X=m=l=2$.
    \end{prop}
    \begin{proof}
    We have different proofs according to whether or not the order $l$ of the group is the same as the characteristic $p$ of the field.
    
    
    First we assume that $l=p$. 
    As in the proof of Proposition \ref{m=1}, we let $N=\sum_{i=1}^r N_i$, and we let $M=N+r$.
    Then due to (\ref{hur2}) we have
        \begin{equation}\label{simplehur}
        2g_X-2=-2p+M(p-1),
        \end{equation}
    and combining this with Lemma \ref{dim3} we can write
        \begin{equation}\label{altdim2}
        \dim_kH^0(X,\Omega_X^{\otimes m})=(2m-1)(g_X-1)=(2m-1)\left(\frac{M(p-1)-2p}{2}\right).
        \end{equation}
    
    Furthermore, we have
        \begin{equation}\label{altdim}
        \deg\left\lfloor \frac{m\pi_*(R)}{p} \right\rfloor = \sum_{i=1}^r\left\lfloor \frac{m(N_i+1)(p-1)}{p} \right\rfloor  = mM + \sum_{i=1}^r\left\lfloor \frac{-m(N_i+1)}{p} \right\rfloor.
        \end{equation}
    If we have $p=g_X=m=2$, then on the one hand we see that $\dim_kH^0(X,\Omega_X^{\otimes m}) =3$. 
    On the other hand, we first note that \eqref{simplehur} implies $M=6$.
    So 
        \begin{equation*}
        \deg\left\lfloor \frac{m\pi_*(R)}{p}\right\rfloor = 2M -M =6 > 3 = 2m-1.
        \end{equation*}  
    Then, by Proposition \ref{dim}, we obtain 
        \begin{equation*}
        \dim_kH^0(X,\Omega_X^{\otimes m})^G = (2m-1)(g_Y-1)+\deg\left\lfloor \frac{m\pi_*(R)}{p} \right\rfloor = -3 + 6 = 3.
        \end{equation*}
    So the two dimensions are equal and the action of $G$ on $H^0(X,\Omega_X^{\otimes m})$ is trivial. 
    This completes the if direction of the proof.
    
    Now we assume that the action is trivial. This first implies that 
    $\deg \left\lfloor\frac{m\pi_*(R)}{p}\right\rfloor \geq 2m-1$ because otherwise we would 
    have $\dim_kH^0(X,\Omega_X^{\otimes m})^G=0$ by Proposition \ref{dim}, but we know that 
    $\dim_kH^0(X,\Omega_X^{\otimes m})=(2m-1)(g_X-1)$ is strictly positive.\todo{rewrite sentence}
    So, using \eqref{altdim}, \eqref{altdim2} and Proposition \ref{dim} we see that
        \begin{align}\label{bound}
        \lefteqn{(2m-1)\frac{M(p-1)-2p}{2} = \dim_kH^0(X,\Omega_X^{\otimes m})} \nonumber\\
        & =  \dim_kH^0(X,\Omega_X^{\otimes m})^G \nonumber\\
        & =  1-2m+mM+\sum_{i=1}^r\left\lfloor\frac{-m(N_i+1)}{p}\right\rfloor\nonumber \\
        & \leq  1-2m+mM+\sum_{i=1}^r\frac{-m(N_i+1)}{p}\nonumber \\
        & =  1-2m+mM-\frac{mM}{p}.
        \end{align}
    
    After multiplying by $2p$ and rearranging we obtain
        \begin{align}\label{times2p}
        0 & \geq  (2mM-M-4m+2)p^2+(-4mM+M-2+4m)p+2mM \nonumber \\
            & =  (M-2)(2m-1)p^2-((M-2)(2m-1)+2mM)p+2mM \nonumber \\
        & =  (p-1)((M-2)(2m-1)p-2mM).
        \end{align}
    
    Furthermore from \eqref{hur2} we obtain that $-2p+M(p-1)=2g_X-2 \geq 2$ and hence that 
        \begin{equation}\label{greater2}
        M\geq \frac{2+2p}{p-1}=2+\frac{4}{p-1}>2.
        \end{equation}
    
    So from \eqref{times2p} and \eqref{greater2} we see that
        \begin{align}\label{plessthan4}
        p & \leq  \frac{2mM}{(M-2)(2m-1)}\nonumber\\
        & =  \frac{M}{M-2}\cdot\frac{2m}{2m-1}\nonumber\\
        & =  \left( 1+\frac{2}{M-2} \right) \left(1+\frac{1}{2m-1} \right)\\
        & \leq  4, \nonumber	
        \end{align}
    \ie $p=2$ or $p=3$. 
    
    Suppose that $p=3$. Then from \eqref{greater2} we have $M\geq 4$. However, from  \eqref{plessthan4} we also have that 
        \begin{align*}
        3 & \leq \left( 1+\frac{2}{M-2} \right) \left(1+\frac{1}{2m-1} \right)\\
        & \leq  \left( 1+\frac{2}{M-2} \right) \frac{4}{3}\\
        & \leq  \frac{8}{3},
        \end{align*}
    a contradiction.
    
    Lastly, we come to the case when $p=2$. From \eqref{plessthan4} we see that $2\leq \left(1+\frac{2}{M-2}\right)\frac{4}{3}$ 
    and hence $M\leq 6$. However, from \eqref{greater2} we know that $M\geq 6$, so $M=6$. Then from \eqref{bound}  we obtain that $2m-1=1-2m+6m-3m$
    and hence that $m=2$. Finally, (\ref{hur2}) gives us that $2g_X-2=-4+6=2$ and hence $g_X=2$. 
    This completes the only if direction of the proof when $l=p$.
    
    Now if $l\neq p$ then we know that all the coefficients $\delta_i$ of the ramification divisor are equal to $l-1$. 
    To show the if direction in this case, first note that $\dim_kH^0(X,\Omega_X^{\otimes m})=3$ by Lemma~\ref{dim3}. 
    On the other hand, the Riemann-Hurwitz formula (Corollary \ref{corhurwitzformula}) implies that $2 = 2g_X-2=-2l+\deg(R)=-2l+r(l-1)$, and hence that $r=6$. 
    Finally Proposition \ref{dim} gives us
        \begin{equation*}
        \dim_kH^0(X,\Omega_X^{\otimes m})^G = -(2m-1) + \sum_{i=1}^r \left\lfloor \frac{m\cdot \delta_i}{l} \right\rfloor
        = -3 +\sum_{i=1}^6 \left\lfloor \frac{m(l-1)}{l} \right\rfloor
        = 3,
        \end{equation*}
    since $m=l=2$.
    As the dimensions of $H^0(X,\Omega_X^{\otimes m})$ and $H^0(X,\Omega_X^{\otimes m})^G$ are equal, the action is trivial.
    
    
    Now, for the final section of the proof we suppose that $G$ acts trivially on the space $H^0(X,\Omega_X^{\otimes m})$.
    We then show that this implies that $g_X=l=m=2$.
    
    
    From Lemma \ref{dim3} and Proposition~\ref{dim} we obtain
        \begin{align*}
        \lefteqn{(2m-1)(g_X-1)=\dim_kH^0(X,\Omega_X^{\otimes m})} \\
        & =  \dim_kH^0(X,\Omega_X^{\otimes m})^G=-(2m-1)+\sum_{i=1}^r \left\lfloor \frac{m\cdot \delta_i}{l} \right\rfloor
        \end{align*}
    and hence
        \begin{equation*}
        (2m-1)g_X = \sum_{i=1}^r \left\lfloor \frac{m\cdot \delta_i}{l} \right\rfloor
        = \sum_{i=1}^r \left\lfloor \frac{m(l-1)}{l} \right\rfloor
        = r\left( m+\left\lfloor \frac{-m}{l} \right\rfloor \right).
        \end{equation*}
    By choosing $s\in \{1,\ldots ,l\}$ and $q\in \mathbb{N}$ such that $m=ql+s$ we can rewrite this as
        \begin{equation}\label{eq:mult}
        (2m-1)g_X=r(m-q-1).
        \end{equation}
    If we multiply (\ref{eq:mult}) by $l-1$ and then substitute in for the $r(l-1)$ term in the Riemann-Hurwitz formula (Corollary \ref{corhurwitzformula}) we get
        \begin{equation*}
        (2m-1)(l-1)g_X=(2g_X+2(l-1))(m-q-1).
        \end{equation*}
    By rearranging we are able to compute $g_X$ in terms of $m,l$ and $q$:
        \begin{align}\label{equationgxintermsofmandlandq}
        \lefteqn{g_X = \frac{2(l-1)(m-q-1)}{(2m-1)(l-1)-2(m-q-1)}} \nonumber \\
        & =  1 + \frac{2(m-q-1)-(2q+1)(l-1)}{(2m-1)(l-1)-2(m-q-1)}  \nonumber\\
        & =  1 + \frac{2s-1-l}{(2m-1)(l-1)-2(m-q-1)} \nonumber  \\
        & =  1 + \frac{2(s-1)+1-l}{(2m-1-2q)(l-1)-2(s-1)}. 
        \end{align}
    First, we show that if $l\geq 3$ the equation cannot hold whilst $g_X\geq 2$.
    Observe that the denominator is strictly greater than $l-1$, remembering that $m=ql+s$:
        \begin{align*}
        (2m-1-2q)(l-1)-2(s-1) & =  ((2q(l-1)+2s-1)(l-1)-2(s-1) \\
        & \geq  (2s-1)(l-1)-2(s-1) \\
        & \geq  (2s-1)(l-1)-2(s-1)(l-1) \\
        & =  l-1;
        \end{align*}
    here the two inequalities are equalities if and only if $q=0$ and $s=1$, respectively, and, as $m\geq 2$, not both inequalities can be equalities.
    Now the numerator is at most $l-1$, occurring when $s=l$. 
    Hence if $l\geq 3$ the fraction in \eqref{equationgxintermsofmandlandq} will be less than one and $g_X < 2$, contradicting our assumption.
    If $l=2$, then $s$ is either 1 or 2.
    If $s=1$ the fraction is negative, and $g_X<1$, which again contradicts our assumption.
    Finally, if $s=2$ then $g_X\leq 2$, with equality if and only if $q=0$, \ie~if and only if $m=2$.
    So if $g_X \geq 2$ then the action being trivial implies that $g_X=l=m=2$, and the proof is complete.    
    \end{proof}

For the rest of this section we assume that $p>0$ and that $G$ is a cyclic group of order $p^l$ for some $l \in \NN$.
What we are now going to do will not be used in the proof of the main theorem, but is included because it generalises the previous results.
More precisely, we do not restrict ourselves to looking at $H^0(X,\Omega_X^{\otimes m})$, but using a comparatively deep result from \cite{kako} we study $H^0(X,\cO(D))$ for any $G$-invariant divisor $D$ such that $\deg(D)>2g_X-2$.


We first introduce some notation.
Let $D = \sum_{P\in X} n_P[P]$ be a $G$-invariant divisor on $X$.
Then let $\langle a \rangle$ denote the fractional part of any $a\in \mathbb{R}$, \ie $\langle a \rangle = a - \lfloor a \rfloor$.
Also, for any $Q\in Y$ let $n_Q$ be equal to $n_P$ for any $P\in \pi^{-1}(Q)$.




    \begin{prop}\label{nakaj}
    Suppose $p>0$ and $G$ is a cyclic group of order $p^l$ for some $l\geq 1$.
    Let $D$ be a $G$-invariant divisor on $X$ such that $\deg(D)>2g_X-2$.
    Then the action of~$G$ on $H^0(X,\cO_X(D))$ is trivial if and only if
        \[ 
        (p^l-1)\deg(D)=p^l\left(g_X-g_Y-\sum_{Q\in Y}\left\langle \frac{n_Q}{e_Q} \right\rangle\right).
        \]
    \end{prop}
    \begin{proof}
    We first remind the reader of the notation in \cite{kako}.
    Let $\sigma$ be a generator of $G$.
    Let $V$ be the $k[G]$ module with $k$-basis $e_1,\ldots ,e_{p^l}$ and $G$-action defined by $\sigma( e_i)=e_i+e_{i-1}$, $1\leq i \leq p^l,\ e_0=0$.\todo{changed from $\sigma \cdot e_i$ to how is now. Check rest of work for issues}
    Then $V_j$, defined to be the subspace of $V$ spanned by $e_1,\ldots ,e_j$ over $k$, is also a $k[G]$ module.
    In fact, the modules $V_1,\ldots ,V_{p^l}$ form a complete set of representatives for the set of isomorphism classes of indecomposable $k[G]$-modules. For each $j=1,\ldots,p^l$ let $m_j$ denote the multiplicity of $V_j$ in the $k[G]$-module $H^0(X,\cO_x(D))$, \ie we have $H^0(X,\cO_x(D))\cong \oplus_{j=1}^{p^l}m_jV_j$.
    
    
    
    First note that the action of $G$ on $H^0(X,\cO_X(D))$ is trivial if and only if
        \begin{equation}\label{triva}
        \dim_k H^0(X,\cO_X(D))^G =\dim_k H^0(X,\cO_X(D)).
        \end{equation}
    
    It is clear that the $G$-invariant part of each submodule $V_j$ is spanned by $e_1$. 
    Hence $\dim_kH^0(X,\cO_X(D))^G = \sum_{j=1}^{p^l} m_j$.
    By \cite[Thm. 2.1]{quaddiffequi}, which relies on \cite{cohogsheaves}, we have
        \begin{align*}
        \sum_{j=1}^{p^l} m_j & =  1- g_Y +\sum_{Q\in Y} \left\lfloor \frac{n_Q}{e_Q}\right\rfloor\\
        & =  1- g_Y + \sum_{Q\in Y} \left( \frac{n_Q}{e_Q} - \left\langle \frac{n_Q}{e_Q}\right\rangle \right) \\
        & =  1 - g_Y + \frac{1}{p^l}\deg(D) - \sum_{Q\in Y} \left\langle \frac{n_Q}{e_Q} \right\rangle.
        \end{align*}
    
    Now as $\deg(D)>2g_X-2$ we have $\dim_kH^0(X,\cO_X(D)) =\deg(D)+1-g_X$ by the Riemann-Roch theorem. 
    So the action of $G$ on $H^0(X,\cO_X(D))$ is trivial if and only if
        \begin{equation*}
        \deg(D)+1-g_X  = 1 - g_Y + \frac{1}{p^l}\deg(D) - \sum_{Q\in Y}\left\langle \frac{n_Q}{e_Q} \right\rangle. \label{hi}
        \end{equation*}
    
    This then rearranges to $(p^l-1)\deg(D)=p^l\left(g_X-g_Y-\sum_{Q\in Y}\left\langle \frac{n_Q}{e_Q} \right\rangle\right)$, as desired.
    \end{proof}

    \begin{cor}\label{this}
    Suppose that $\deg(D)\geq 2g_X$. Then the action of $G$ on $H^0(X,\cO_X(D))$ is trivial if and 
    only if $g_Y = 0$, $e_Q | n_Q$ for all $Q\in Y$, $\deg(D)=2g_X$ and either $g_X=0$ or $p^l=2$.
    \end{cor}
    \begin{proof}
    The following inequalities always hold under the stated assumptions:
        \begin{multline}
        (p^l-1)\deg(D)\geq (p^l-1)2g_X \geq p^lg_X \geq p^lg_X-p^l\sum_{Q\in Y}\left\langle\frac{n_Q}{e_Q}\right\rangle \\ \geq p^l\left( g_X - g_Y -\sum_{Q\in Y}\left\langle \frac{n_Q}{e_Q} \right\rangle \right).
        \end{multline}
    Now the first inequality is an equality if and only if $\deg(D)=2g_X$. 
    The second is an equality if and only if either $g_X=0$ or $p^l=2$. 
    The third inequality is an equality if and only if $\sum_{Q\in Y}\left\langle\frac{n_Q}{e_Q}\right\rangle=0$, which is the case if and only if each $n_Q$ is divisible by~$e_Q$. 
    Lastly, the fourth inequality is an equality if and only if $g_Y = 0$.
    Given these observations, Proposition \ref{nakaj} implies Corollary~\ref{this}.
    \end{proof}

The following Corollary slightly strengthens the only if direction of the $l=p$ part of Proposition \ref{triv}
(from $\ord(G) = p$ to $\ord(G) = p^l$) and also provides a different proof for it;
note that this new proof relies on the comparatively deep result result in section 7 of \cite{cohogsheaves}.


    \begin{cor}
    Let $m \geq 2$ and let $G$ be a cyclic group of order $p^l$ for some $l$. 
    If $G$ acts trivially on $H^0(X,\Omega_X^{\otimes m})$, then $g_Y = 0$ and $p^l = g_X = m = 2$.
    \end{cor}
    \begin{proof}
    As $g_X \geq 2$ and $m\geq 2$ we have $\deg(mK_X) \geq 2g_X$. 
    So, if the action of $G$ on $H^0(X,\Omega_X^{\otimes m})$ is trivial, we obtain from Corollary \ref{this} that $\deg(mK_X) = 2g_X$, $p^l = 2$ and $g_y = 0$.
    Now $\deg (mK_X) = 2g_X$ implies that $m(2g_X -2 ) = 2g_X$, so $m(g_X -1) = g_X$ and hence $m=g_X=2$.
    \end{proof}

Similarly to the case $\deg(D)\geq 2g_X$ in Corollary \ref{this}, the following corollary derives necessary and sufficient conditions for trivial action from Proposition \ref{nakaj} in the case $\deg(D) =2g_X-1$.



    \begin{cor}
    Suppose that $\deg(D)= 2g_X-1$ and that $g_Y=0$. Then the action of $G$ on $H^0(X,\cO_X(D))$ is trivial if and only if one of the following conditions hold:
        \begin{itemize}
        \item  $p^l=2$ and $\sum_{Q\in Y}\left\langle\frac{n_Q}{e_Q}\right\rangle=\frac{1}{2}$;
        \item  $g_X=2$, $p^l=3$ and $e_Q\mid n_Q$ for all $Q\in Y$.
        \end{itemize}
    \end{cor}


    \begin{rem}
    It can easily be shown that in the last case the Riemann-Hurwitz formula implies that $r\leq 4$. 
    Furthermore, if $r=1$ then the conditions ``$\sum_{Q\in Y}\left\langle\frac{n_Q}{e_Q}\right\rangle=\frac{1}{p^l}$" and ``$e_Q\mid n_Q$ for all $Q\in Y$" are already implied by ``$\deg(D)=2g_X-1$".
    \end{rem}

    \begin{proof}
    Firstly, if $g_X=0$ then $\deg(D)=-1<0$, so $\dim_kH^0(X,\cO_X(D))=0$ and the action is trivial.
    
    Now note that, as $\deg(D)=2g_X-1$, we conclude from Proposition \ref{nakaj} that the action is trivial if and only if 
        \begin{equation*}
        (p^l-1)(2g_X-1)=p^l\left(g_X-\sum_{Q\in Y}\left\langle\frac{n_Q}{e_Q}\right\rangle\right).
        \end{equation*}
    If $p^l=2$ then this is equivalent to $2g_X-1=2g_X-2\sum_{Q\in Y}\left\langle\frac{n_Q}{e_Q}\right\rangle$ and hence to $\sum_{Q\in Y}\left\langle\frac{n_Q}{e_Q}\right\rangle=\frac{1}{2}$.
    
    If $g_X=1$ then this is equivalent to $p^l-1=p^l-p^l\sum_{Q\in Y}\left\langle\frac{n_Q}{e_Q}\right\rangle$ and hence is also equivalent to $\sum_{Q\in Y}\left\langle\frac{n_Q}{e_Q}\right\rangle=\frac{1}{p^l}$.
    
    Lastly, if $p^l\geq 3$ and $g_X\geq 2$ then we have that $g_X\geq \frac{p^l-1}{p^l-2}$ which is equivalent to the first inequality in the chain
        \begin{equation*}
        (p^l-1)(2g_X-1)\geq p^lg_X\geq p^lg_X-p^l\sum_{Q\in Y}\left\langle\frac{n_Q}{e_Q}\right\rangle \geq p^l\left( g_X - g_Y -\sum_{Q\in Y} \left\langle \frac{n_Q}{e_Q} \right\rangle \right).
        \end{equation*}
    Hence the action is trivial if and only if both inequalities are equalities, which is the case if and only if $p^l=3,\ g_X=2$, $e_Q\mid n_Q$ for all $Q\in Y$ and $g_Y = 0$.
    \end{proof}


\section{The main theorem}\label{maintheoremsection}
In this section we prove the main theorem of this chapter, describing exactly when $G$ will act faithfully on $H^0(X,\Omega_X^{\otimes m})$.


    \begin{thm}\label{theoremfaithfulaction}
    Suppose that $g_X\geq 2$ and let $m\geq1$. 
    Then $G$ does not act faithfully on $H^0(X,\Omega_X^{\otimes m})$ if and only if $G$ contains a hyperelliptic involution and one of the following two sets of conditions holds:
        \begin{itemize}
        \item $m=1$ and $p=2$;
        \item $m=2$ and $g_X=2$.
        \end{itemize}
    \end{thm}
    \begin{proof}
    We first show the if direction. 
    In the case when $m=1$, the hyperelliptic involution contained in $G$ generates a subgroup of order $2$.
    Since $p=2$, this acts trivially by Proposition \ref{m=1}, and hence $G$ does not act faithfully.
    In the case when $m=2$, then again looking at the subgroup generated by the hyperelliptic involution, we have a group of order $2$ acting on $H^0(X,\Omega_X^{\otimes m})$.
    So, by Proposition \ref{triv} and since $g_X=m=2$, the action of this subgroup is trivial, and again, this means that $G$ does not act faithfully.
    
    
    We now start the proof of the only if direction, supposing that $G$ does not act faithfully on $H^0(X,\Omega_X^{\otimes m})$. 
    By replacing $G$ with the (non-trivial) kernel $H$ if necessary, we may assume that $G$ is non-trivial and acts trivially on $H^0(X,\Omega_X^{\otimes m})$.
    
    
    We start the proof by showing that $g_Y=0$, which is shown separately for the cases when $m=1$ and when $m\geq 2$.
    In the case when $m=1$ we start by showing that $\deg  \left\lfloor \frac {\pi_*(R)}{n} \right\rfloor >0$ by contradiction.
    Suppose that $\deg\left\lfloor \frac{\pi_*(R)}{n} \right\rfloor =0$.
    As $G$ acts trivially it follows from Proposition~\ref{dim} that:
        \begin{equation*}
        g_X=\dim_k H^0(X,\Omega_X)=\dim_k H^0(X,\Omega_X)^G=g_Y.
        \end{equation*}
    Substituting this into the Riemann-Hurwitz formula (Corollary \ref{corhurwitzformula}) yields the desired contradiction because $g_X\geq 2, n\geq 2$ and $\deg(R)\geq 0$.
    
    Thus $\deg\left( \left\lfloor \frac{\pi_*(R)}{n} \right\rfloor \right) >0$. 
    Now Proposition~\ref{dim} gives us that
        \begin{equation*}
        g_X=\dim_k H^0(X,\Omega_X)=\dim_k H^0(X,\Omega_X)^G= g_Y-1+\deg\left\lfloor \frac{\pi_*(R)}{n} \right\rfloor.
        \end{equation*}
    Substituting this in to the Riemann-Hurwitz formula we see that
        \begin{equation*}
        2\left(g_Y - 1 + \deg\left \lfloor \frac{\pi_*(R)}{n} \right \rfloor -1 \right) = 2n (g_Y -1) + \deg(R).
        \end{equation*}
    For any $Q \in Y$ we let $\delta_Q$ denote the coefficient of the ramification divisor $R$ at any $P \in \pi^{-1}(Q)$ and let $e_Q := e_P$ for any $P \in \pi^{-1}(Q)$. 
    Rewriting the previous equation then yields
        \begin{align*}
        \lefteqn{(2n-2)g_Y = 2n-4 + 2 \,\deg\left \lfloor \frac{\pi_*(R)}{n}\right \rfloor - \deg(R)}\\
        &= 2 \left(n-2 + \sum_{Q \in Y} \left(\left\lfloor \frac{n}{e_Q} \frac{\delta_Q}{n} \right\rfloor - \frac{n}{e_Q} \frac{\delta_Q}{2}\right) \right)\\
        &= 2 \left(n-2 + \sum_{Q \in Y} \left( \left\lfloor \frac{\delta_Q}{e_Q} \right\rfloor - \frac{\delta_Q}{e_Q} \frac{n}{2} \right)\right)\\
        & \le  2(n-2),
        \end{align*}
    because $\frac{n}{2} \ge 1$ and $\left\lfloor \frac{\delta_Q}{e_Q}\right\rfloor \le \frac{\delta_Q}{e_Q}$ for all $Q \in Y$. 
    Hence we obtain $g_Y \le \frac{n-2}{n-1} < 1$ and therefore $g_Y =0$, as desired.
    
    We now show that $g_Y=0$ when $m\geq 2$. 
    Since $g_X\geq 2$ we have that $\deg(mK_X)=m(2g_X-2)>2g_X-2=\deg(K_X)$.
    By Lemma \ref{dim3}, and as both $m$ and $g_X$ are at least 2, then $\dim_kH^0(X,\Omega_X^{\otimes m})^G=\dim_kH^0(X,\Omega_X^{\otimes m})=(2m-1)(g_X-1)>1$.
    There is only one case in Proposition \ref{dim} such that $m\geq 2$ and $\dim_k H^0(X,\Omega_X^{\otimes m})^G>1$, which yields 
        \begin{equation*}
        (2m-1)(g_X-1)=(2m-1)(g_Y-1)+\deg\left(\left\lfloor \frac{m\pi_*(R)}{n} \right\rfloor \right).
        \end{equation*}
    Combining this with the Riemann-Hurwitz formula, Corollary \ref{corhurwitzformula}, we see that
        \begin{align*}
        2(2m-1)(g_Y-1)+2\deg\left(\left\lfloor\frac{m\pi_*(R)}{n}\right\rfloor\right) & =  2(2m-1)(g_X-1)\\
        & =  2n(2m-1)(g_Y-1)+(2m-1)\deg(R),
        \end{align*}
    which can be re-arranged as
        \begin{equation*}
        (2m-1)(2n-2)(g_Y-1)=2\deg\left(\left\lfloor\frac{m\pi_*(R)}{n}\right\rfloor\right)-(2m-1)\deg(R).
        \end{equation*}
    So if we can show that the right hand side of this equation is negative then we will have $g_Y-1<0$ and hence $g_Y=0$, as desired.
    
    Using the same notation as in the case when $m=1$, we calculate:
        \begin{align*}
        2\deg\left(\left\lfloor\frac{m\pi_*(R)}{n}\right\rfloor\right)-(2m-1)\deg(R) & = \sum_{Q \in Y} \left(2\left\lfloor m\cdot \frac{n}{e_Q}\frac{\delta_Q}{n}\right\rfloor -n(2m-1)\frac{\delta_Q}{e_Q}\right) \\
        & \leq   \sum_{Q\in Y}\left( 2m\cdot\frac{\delta_Q}{e_Q}-n(2m-1)\frac{\delta_Q}{e_Q}\right) \\
        & =  (2m-n(2m-1))\sum_{Q\in Y }\frac{\delta_Q}{e_Q}.
        \end{align*}
    
    Now as $n,m\geq 2$ then we have $2m-n(2m-1)\leq 2m-2(2m-1)=2(1-m)<0$ and we are done as $\sum_{Q\in Y}\frac{\delta_Q}{e_Q}$ is positive.
    
    So we have shown for all $m\geq 1$, if the group $G$ acts trivially  on $H^0(X,\Omega_X^{\otimes m})$ then $g_Y=0$.
    Now if $m\geq 2$ then first note that $G$ must contain a cyclic subgroup of prime order, say $H$, such that $H$ acts trivially on $H^0(X,\Omega_X^{\otimes m})$.
    Now Proposition \ref{triv} tells us that $m=g_X=2$, and that the order of $H$ must also be 2.
    Hence $X/H\cong \mathbb{P}_k^1$, and this completes the only if direction for $m\geq 2$.
    
    Similarly, the $m=1$ case of the only if direction will follow from Proposition \ref{m=1} after we show that $p>0$ and there is a cyclic subgroup of $G$ of order $p$. 
    This is true since $\pi$ cannot be tamely ramified.
    Indeed, if it were then $R=\sum_{P\in X} (e_P-1)[P]$ \cite[Chap. IV, Cor. 2.4]{hart}, and $\deg\left\lfloor \frac{\pi_*(R)}{n} \right\rfloor=0$, which we have already shown cannot be the case.
    Hence $p$ must be positive, and there is a cyclic subgroup of order $p$ which acts trivially.
    \end{proof}

    \begin{rem}
    Note that the existence of a hyperelliptic involution $\sigma$ in $G$ means not only that the genus of $X/\langle \sigma \rangle$, but also the genus of $Y=X/G$, is $0$ (by the Riemann-Hurwitz formula).
    If, moreover $p=2$, then the canonical projection $X\rightarrow X/\langle \sigma \rangle$ is not unramified (again by the Riemann-Hurwitz formula) and hence not tamely ramified; then $\pi$ cannot be tamely ramified either.
    \end{rem}


\section{Examples}
We will now give some examples of a finite group acting on a curve, and the consequent action on the holomorphic poly-differentials. 
We start with some examples in which $G$ acts trivially on $H^0(X,\Omega_X^{\otimes m})$.
We then follow this with the example of hyperelliptic curves, for which we compute an explicit basis of $H^0(X,\Omega_X^{\otimes m})$, allowing us to see when the action is trivial.


\subsection{Trivial Examples}\label{examplessection}


(a) Let $g_X = 0$, \ie $X\cong \mathbb P_k^1$.
Then $\deg(K_X) = -2$ and so $\deg(mK_X) < 0$ for $m~\geq~1$.
Hence $H^0(X,\Omega_X^{\otimes m}) =\{0\}$ by \cite[Lem. 2, pg. 295]{hart}\todo{check citation} and $G$ acts trivially on $H^0(X,\Omega_X^{\otimes m})$ for all $m\geq 1$.

(b) Let $g_X = 1$, \ie $X$ is an elliptic curve.
If $G$ is a finite subgroup of $X(k)$ acting on $X$ by translations, then $G$ leaves invariant any global non-vanishing holomorphic differential $\omega$ and hence $G$ acts trivially on $H^0(X,\Omega_X)$;
since $\omega^{\otimes m}$ is a basis of $H^0(X,\Omega_X^{\otimes m})$ this means that $G$ acts trivially on $H^0(X,\Omega_X^{\otimes m})$ for all $m\geq 1$.

If $p>0$ and $G$ is a $p$-group, then the multiplicative character $G\rightarrow k^*$ afforded by the one-dimensional representation $H^0(X,\Omega_X^{\otimes m})$ of $G$ has to be trivial because $k$ doesn't contain any $p^{\mbox{th}}$ roots of unity;
in particular the action of $G$ on $H^0(X,\Omega_X^{\otimes m})$ is trivial as well.
On the other hand, if $p\neq 2$ and $X$ is given by the Weierstrass equation of the form $y^2 = f(x)$, then the involution $\sigma \colon  (x,y) \rightarrow (x,-y)$ maps the invariant differential $\omega = \frac{dx}{y}$ to $-\omega$.




