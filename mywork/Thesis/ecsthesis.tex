%% ----------------------------------------------------------------
%% Thesis.tex
%% ---------------------------------------------------------------- 
\documentclass{ecsthesis}      % Use the Thesis Style
%\graphicspath{{../Figures/}}   % Location of your graphics files
%\usepackage{natbib}            % Use Natbib style for the refs.
\hypersetup{colorlinks=true}   % Set to false for black/white printing
\usepackage{verbatim} % adds environment for commenting out blocks of text & for better verbatim
\usepackage[all]{xy}
\usepackage{paralist} % very flexible & customisable lists (e.g. enumerate/itemize, etc.)
\usepackage{etex}
\usepackage{array} % for better arrays (e.g. matrices) in maths
\usepackage{amsmath}
\usepackage{makeidx}
\usepackage[colorinlistoftodos, textwidth=2.6cm, obeyDraft]{todonotes} % Allows todo notes to be added to the text 
\usepackage[T1]{fontenc}
\usepackage{kpfonts,baskervald}
\usepackage{epigraph}
\input{definitions}            % Include your abbreviations
\makeindex

%% ----------------------------------------------------------------
\begin{document}
\frontmatter
\title      {Group actions on differentials of curves and cohomology bases of hyperelliptic curves}
\authors    {\texorpdfstring
             {\href{mailto:Joe.Tait@soton.ac.uk}{Joseph J. Tait}}
             {Joseph J. Tait}
            }
\addresses  {\groupname\\\deptname\\\univname}
\date       {\today}
\subject    {}
\keywords   {}
\maketitle
\pagebreak
%\includegraphics[width = 400px]{soton-crest.png}
\pagebreak
\begin{abstract}
In this thesis we consider the natural action of a subgroup $G$ of the automorphism group of an algebraic curve on spaces of differentials and similar algebraic structures.
We focus on curves over an algebraically closed field $k$ of characteristic $p >0$, and in particular on cases where $p$ divides the order of the group $G$.
There is also an emphasis on explicit examples and concrete computations throughout the thesis.

After covering background material about smooth projective curves we remind the reader of the details of hyperelliptic curves.  %We cover the case $p = 2$ in more depth, as details of this are less readily available in the literature.
Given a hyperelliptic curve $X$, we present an explicit basis for $H^0(X,\Omega_X^{\otimes m})$, the space of global polydifferentials of degree $m$.

We apply our study of hyperelliptic curves by computing bases of $\hone$ and the first de Rham cohomology group of $X$, $\derhamhone$.
We make these computations via \cech cohomology, and use them to determine the action of a specific automorphism $\tau$ of order $p$ on $\derhamhone$.
We then show that the natural short exact sequence of $k[\langle \tau \rangle]$-modules
    \[
    0 \ra \hzero \ra H^1_{\text{dR}}(X) \ra \hone \ra 0
    \]
does not split if $X$ is ramified above $\infty$.
We also give a Mittag-Leffler style theorem for hyperelliptic curves.

We finally consider the question of when $G$ acts faithfully on the space $H^0(X,\Omega_X^{\otimes m})$, for any smooth projective curve $X$.
We give a complete and concise answer to this question, as well as extending the result to general Riemann-Roch spaces $H^0(X,\cO_X(D))$ where $D$ is a $G$-invariant divisor of degree at least $2g_X - 2$.
Lastly, we use our earlier work for hyperelliptic curves to elucidate the main theorem.
\end{abstract}
\tableofcontents
%\listoffigures
%\listoftables
%\lstlistoflistings
\pagebreak
\chapter*{\Large \textbf{Academic Thesis: Declaration of Authorship}} 
I, Joseph Tait, declare that this thesis and the work presented in it are my own and has been generated by me as the result of my own original research.

Title of thesis: Group actions on differentials of curves and cohomology bases of hyperelliptic curves.

I confirm that:
    \begin{enumerate}
    \item This work was done wholly or mainly while in candidature for a research degree at this University;
    \item Where any part of this thesis has been submitted for a degree or any other qualification at this University or any other institution, this has been clearly stated;
    \item Where I have consulted the published works of other, this has always been clearly attributed;
    \item Where I have quoted from the work of others, the source is always given. With the exception of such quotations, this thesis is entirely my own work;
    \item I have acknowledged all main sources of help;
    \item Where the thesis is based on work done by myself or jointly with others, I have made clear exactly what was done by others and what I have contributed myself;
    \item The majority of the work in Chapter 5, and parts of Chapter 2, have been published in \emph{Faithfulness of actions on Riemann-Roch spaces}, to appear in the Canadian Journal of Mathematics. None of the other work has been published before submission.
    \end{enumerate}

Signed:\\

Date: \\

\pagebreak
\acknowledgements{
%    I would first and foremost like to thank my supervisor, Bernhard K\"ock, for his great guidance and patience, both in matters of mathematics and style. He has always afforded the greatest amount of time and care in reading and annotating work, and especially in considering how mathematics may be extended and presentation improved. Lastly, I would like to thank Bernhard for never letting anyone else know that his understanding of my native language is clearly greater than mine. I would further like to thank both Gareth Jones and David Singerman for mathematical discussions.
%    My office mates have been a great source of inspiration, and have made the last four years incredibly interesting, enjoyable and in no way predictable. In particular, I would like to mention Ingrid , Alex Bailey, Matt Burfitt, Chris Cave, Charles Cox, Max Fennelly, Michal Ferov, Martin Finn-Sell, Martin Fluch, Robin Frankhuizen, Giovanni Gandini, Tom Harris, Pip Hiscock, Ana Khukhro, David Matthews, Raffaele Rainone, Yafet Sanchez-Sanchez, Rob Snocken, Simon St.\ John-Green, Ed Tye, Mike West, Michaele Zordan. I would also like to thank Vesna Perisic and Alex Stasinski
%    Family and Laur
%    EPSRC
    }
\listofsymbols{ll}{ 
                    $\left\lfloor D \right\rfloor$  & The divisor $D$ after applying the floor function applied to each coefficient \\
                    $\langle a \rangle$     & Fractional part of $a \in \RR$ \\
                    $\langle -, - \rangle$  & Serre duality pairing \\
                    $\mathbb A_k^1$         & Affine line over $k$ \\
                    $\aut(X)$               & Automorphism group of $X$ \\
                    $\check{d}$             & \cech cohomology differential \\
                    $D$                     & Divisor \\
                    $\di(f)$; $\di(\omega)$ & The divisor of a function or differential \\
                    $\Di(X)$                & Space of divisors on $X$ \\
                    $\deg(D)$               & Degree of the divisor $D$ \\
                    $\delta_P$              & Different exponent at the point $P$ \\
                    $e_P$                   & Ramification index at the point $P$ \\
                    $g_X$                   & Genus of the curve $X$ \\
                    $G$                     & Subgroup of the automorphism group of a curve \\
                    $G_i(P)$                & The $i^{\text {th}}$ ramification group at the point $P$ \\
                    $\hzero$                & Space of global holomorphic differentials on $X$ \\
                    $H^0(X,\Omega_X(D))$    & Space of differentials associated to the divisor $D$ on $X$ \\
                    $H^0(X,\cO_X(D))$       & Space of meromorphic functions associated to the divisor $D$ on $X$ \\
                    $\hzeropoly$            & Space of global holomorphic polydifferentials of degree $m$ on $X$ \\
                    $\hone$                 & First cohomology group of $\cO_X$ \\
                    $\check{H}^n(\cU)$      & The $n^{\text{th}}$ \cech cohomology group corresponding to the cover $\cU$ \\
                    $\derhamhone$           & The first algebraic de Rham cohomology group of $X$ \\
                    $k$                     & Algebraically closed field of characteristic $p \geq 0$\\
                    $K_X$                   & Canonical divisor on $X$ \\
                    $K(X)$                  & Function field of $X$ \\
                    $\underline{K}(X)$      & Constant sheaf of $K(X)$ \\
                    $M^G$                   & Subspace of $M$ fixed by $G$ \\
                    $\mathcal{M}_{X,P}$     & Maximal ideal of $\cO_{X,P}$ \\
                    $\ord_P(f)$; $\ord_P(\omega)$ & Order at the point $P$ of a function or differential \\
                    $\cO_X$                 & Sheaf of rational functions on $X$ \\
                    $\cO_{X,P}$             & Ring of functions that are regular at the point $P$ \\
                    $\Omega_X$              & Sheaf of differentials on $X$ \\
                    $\Omega_X^{\otimes m}$  & Sheaf of polydifferentials of degree $m$ \\
                    $\Omega_{K(X)}$         & Module of differentials of $K(X)$ \\
                    $\underline{\Omega}_{K(X)}$ & Constant sheaf of $\Omega_{K(X)}$ \\
                    $\omega$                & Differential \\
                    $P, P', P_a$            & Points on $X$ \\
                    $\mathbb P_k^1$         & Projective line over $k$ \\
                    $\pi$                   & Projection map from $X$ to the quotient curve $X/G$ \\
                    $\pi^*$                 & Associated map between function fields $K(Y)$ and $K(X)$ \\
                    $\pi_*$                 & Push forward map from $\Di(X)$ to $\Di(Y)$ \\
                    $Q, Q', Q_a$            & Points on $Y$ \\
                    $R$                     & Ramification divisor \\
                    $X$                     & Projective smooth curve over $k$
                    %$P_a, U_a$              & Maybe define these, not sure
                  }
\dedicatory{To no-one\ldots}
\mainmatter
%% ----------------------------------------------------------------
\listoftodos
\reversemarginpar

\chapter{Introduction} \label{Chapter:introduction}

\section{Introduction}

In topology, the number of holes in a compact, orientable surface is an important invariant, called the genus, and classifies compact, orientable surfaces up to homeomorphism.
In particular the genus is an important topological invariant of compact Riemann surfaces.
It is well known that for any compact Riemann surface the genus is also equal to the dimension of the space of global holomorphic differentials.
Furthermore there is a correspondence between compact Riemann surfaces and smooth projective algebraic curves over $\mathbb C$, and the notion of holomorphic differentials can be extended to such curves.
In fact we can extend this even further, by defining the genus of any curve over an algebraically closed field to the dimension of the space of global holomorphic differentials.
The above alone makes it obvious that the space of global holomorphic differentials is a fundamental object in the theory of algebraic curves.
The general motivation underlying this report is to study this space as a representation of a subgroup of the automorphism group of the given curve.

\begin{comment} It is a well known and fundamental fact of the theory of Riemann surfaces that Riemann surfaces can be classified as $n$-tori (including the sphere as the zero torus).
This means that the topological invariant of genus is an invariant in this area of study too, where it is called the analytic genus.
Equally as important is the correspondence between compact Riemann surfaces and connected, projective, smooth algebraic curves over the complex numbers.
So it is natural to ask how the genus arises in the case of these curves.
As a corollary to the Riemann-Roch theorem it can be seen that the dimension of the space of {\em holomorphic differentials} (see Section \ref{Hurwitzsection}) over $\mathbb C$, which we denote by $H^0(X,\Omega_X)$, is what corresponds to the analytic genus.
This invariant is called the arithmetic genus, and can be extended to any curve over an algebraically closed field.
For this reason, among others, the space of holomorphic differentials is a fundamental object in algebraic geometry, and widely studied.
\end{comment}
Let $X$ be a smooth connected projective curve over an algebraically closed field $k$.
Given a subgroup $G$ of the automorphism group of $X$ then a classic problem pertaining to $H^0(X,\Omega_X)$, the space of holomorphic differentials (see section \ref{Hurwitzsection}), is determining its $k[G]$-module structure.
This originally dates back to 1934, and a paper of Chevalley and Weil \cite{chev}.
They only considered the case when $k= \mathbb C$, but the complete structure has since been discovered in the case where the projection from $X$ to the quotient curve is tamely ramified.
This was done by Kani in 1986 \cite{Kani}.
Progress has also been made recently in the case where the projection is wildly ramified; in particular Karanikolopoulos and A. Kontogeorgis \cite{kako} have computed the $k[G]$-module structure for any cyclic group $G$.
Also, in 1986 Broughton \cite{Broughton} computed the $k[G]$-module structure of the space of global holomorphic poly-differentials, $H^0(X,\Omega_X^{\otimes m})$ (see Section \ref{charneq2}), in the case where $\cha(k) = 0$.

In this report we will not look directly at the $k[G]$-module structure, but rather at the related question of determining when the action of $G$ on $H^0(X,\Omega_X)$, and also on $H^0(X,\Omega_X^{\otimes m})$, is faithful.
The following is our main result:

  \begin{unnumthm}{\bf 1}\label{maintheorem}
    Suppose that $g_X\geq 2$ and let $m\geq1$. 
    Then $G$ does not act faithfully on $H^0(X,\Omega_X^{\otimes m})$ if and only if $G$ contains a hyperelliptic involution and one of the following two sets of conditions holds:
      \begin{itemize}
	\item $m=1$ and $p=2$;
	\item $m=2$ and $g_X=2$.
      \end{itemize}
  \end{unnumthm}

  The format of the report is now briefly outlined.
  
  In the first section we prove the strong form of the Riemann-Hurwitz formula (Theorem \ref{hur}).
  The Riemann-Hurwitz formula relates the genus of two curves when there is a surjective map from one to the other, via the degree of the map and the degree of the ramification divisor.
  However, this can obscure the fact that the canonical divisors (see Section \ref{Hurwitzsection}) themselves are related.
  The strong form of the theorem states that given two curves and a surjective map $\pi:X\rightarrow Y$ of degree $n$ between the curves, with ramification divisor $R$ (see Section \ref{Hurwitzsection}), we have
  \[
 \di (\pi^* (\omega)) = \pi^*(\di (\omega)) + R,
  \]
where $\omega$ is a non-zero differential on $Y$, and $\pi^*$ is the pull-back induced by $\pi$.
This section closely follows Stichtenoth's book, see \cite{stichtenoth}.

The second section looks at computing the dimension of various spaces, but the most significant is the dimension of the subspace of $H^0(X,\Omega_X^{\otimes m})$ fixed by $G$, where $m\geq 1$.
This result, along with two other results in section three, forms the heart of the proof of the main theorem.
The dimension itself is dependent, essentially, on the genus of the quotient curve, $Y=X/G$, the degree of the projection map $\pi:X\rightarrow Y$, the ramification divisor $R$ of $\pi$ and $m$.
By using the Riemann-Roch theorem we can easily compute the dimension of $H^0(X,\Omega_X^{\otimes m})$, and the comparison of these two dimensions is what will be used in the third section.


In the third section we consider when a group of prime power order acts trivially on $H^0(X,\Omega_X^{\otimes m})$.
By only considering cyclic groups of prime power order our computations are made considerably easier.
Initially we only consider groups of prime order.
In this case, if the characteristic of $k$ is different to $p$ then the projection map is tamely ramified, and hence we know that the coefficients of the ramification divisor are $p-1$.
This makes it considerably easier to compute the dimension of the fixed space, as of all the parameters it depends on, the ramification divisor is the most difficult to deal with.
When the characteristic of the field and the order of the group are the same (\ie when wild ramification could occur), then the computations are longer, but still made considerably easier by our assumptions.
At the end of the third section we use results of \cite{kako} to make the same computations for groups whose orders are powers of $p$.
The results we use are somewhat more technical, but they do extend the original results.
We also make some computations for a general divisor $D$, which in general should have degree greater than $2g-2$, and give criteria for when the action is trivial on the associated Riemann-Roch space $H^0(X,\cO_X(D))$.

In the fourth section we prove the main result.
This builds on sections two and three, by reducing from a group which does not act faithfully on $H^0(X,\Omega_X^{\otimes m})$, to a subgroup that acts trivially.
We consider the cases where $m=1$ and $m\geq 2$ separately; despite similar methods being employed, there are technical details that need to be changed according to which case is being considered.
These technical details show clearly in the statement: if $m=1$ then the characteristic of $k$ must be 2 for the action to not be faithful, but the genus is not relevant at all.
In contrast, if we consider $m\geq 2$, the characteristic now does not matter, but the genus must be 2 (as does $m$).

In the final section we consider examples to illuminate what has been done in the previous sections.
We start by considering the rather trivial cases where $g_X=0$ and $g_X= 1$.
We give explicit proofs of when the action is faithful, as these cases where not included in the proof of the main theorem.
We then go on to construct a basis for the space of holomorphic poly-differentials for any given hyperelliptic curve.
This serves to explicitly prove the main theorem for this class of curves, and helps to enlighten the reader by way of a concrete example.
In particular, it helps to show why we have a different type of result according to whether $m=1$ or $m\geq 2$.

\chapter{Background}\label{chapterbackground}

\todo{Throughout - change definitions to be roman, and the italicise the "definition" part}

In this section we will introduce the basic concepts, notation and terminology required for the study of algebraic curves.

Throughout this thesis $k$ will denote an algebraically closed field of characteristic $p \geq 0$.
It should be noted that while the majority of results in the thesis hold for all characteristics, including $p=0$, our main focus will be on the case $p > 0$.

When we refer to an \emph{algebraic curve} (or often just a \emph{curve}) we will mean a smooth, connected, projective variety of dimension one over $k$.
In particular, we let $\PP_k^1$ be the projective.
Similarly, when we refer to an \emph{affine curve} we mean a smooth, connected, affine variety of dimension one over $k$. % Add if needed
We recall that a morphism of affine curves $X$ and $Y$ is just polynomial map $\phi \colon X \ra Y$.
Then if $X$ and $Y$ are algebraic curves then a map $\phi \colon X \ra Y$ is a morphism if we can write $X = \cup X_i$ and $Y = \cup Y_i$ for open $X_i$ and $Y_i$ such that $\phi(X_i) \subseteq Y_i$ and $\phi|_{X_i}$ is a morphism for every $i$.



\section{Functions and differentials}

In this section we recall the basic results pertaining to functions and differentials on $X$.

A \emph{meromorphic function} on $X$ is any morphism $f \colon X \ra \PP_k^1$, other than the morphism mapping all points to infinity.
The collection of meromorphic functions on $X$ is denoted $K(X)$, and called the \emph{function field} of $X$.\todo{B: and non-constant morphisms}

We recall that the category of algebraic curves is actually equivalent to the category of function fields over $k$ (which can be defined independently of curves as fields of transcendence degree one over $k$).
An overview of this correspondence is given in \cite[Appendix B]{stichtenoth}.
Furthermore, when working over the complex numbers $\CC$ we actually have a triple equivalence of categories.
The category of function fields over $\CC$ and the category of algebraic curves are both equivalent to the category of compact Riemann surfaces. 
A short explanation of the correspondence between complex curves and Riemann surfaces is given in \cite[Chap.\ 1, \S 2]{griffiths}, whilst \cite{miranda} exhibits the connection between all three categories throughout.


Returning to our study of functions on $X$, we recall that a meromorphic function $f$ on $X$ is \emph{regular} on an open set $U \subseteq X$ if the image $f(U)$ lies in $k = \AA_k^1 \subset \PP_k^1$.
We let $H^0(U,\cO_X)$ denote the space of functions in $K(X)$ which are regular on $U$.
Moreover, if $f \in K(X)$ is regular on $X$ we say that $f$ is \emph{regular}, and then $H^0(X,\cO_X)$ is the space of regular functions.
Since $X$ is projective $H^0(X,\cO_X)$ is in fact isomorphic to $k$ --- \ie the only regular functions are constant functions.
The reader should note that we are using sheaf theoretic notation here. 
We will not give details of sheaves and sheaf cohomology (since it will rarely be needed), but we will consistently use the notation, in order to be with consistent with current work in the area.

Given $P \in X$ we say that a meromorphic function $f \in K(X)$ is \emph{regular at $P$} if $f(P) \in k \subset \PP_k^1$.
The collection of functions regular at $P$ form a ring, which we call $\cO_{X,P}$.

    \begin{lem}
    For any $P \in X$ the ring $\cO_{X,P}$ is a discrete valuation ring, with maximal ideal
        \[
        \mathcal {M}_{X,P} := \{ f \in \cO_{X,P} | f(P) = 0 \}.
        \]
    \end{lem}
    \begin{proof}
    See \cite[Chap.\ 1, \S 4]{fulton}.
    \end{proof}

The valuation on $\cO_{X,P}$ can be given as follows.
Let $t \in \cO_{X,P}$ be a generator of $\mathcal{M}_{X,P}$.
Now any $0 \neq f \in \cO_{X,P}$ can be written as $f = ut^n$ for some unique $n \in \ZZ$ and some unit $u \in \cO_{X,P}\backslash \mathcal{M}_{X,P}$.
We then define \emph{the order of $f$ at $P$} to be $\ord_P(f) := n$.
For any $f \in K(X)^*$ and $P \in X$ at least one $f$ or $1/f$ is an element of $ \cO_{X,P}$.
Hence we may extend the definition of $\ord_P$ to the whole $K(X)^*$, by letting $\ord_P(f) := - \ord_P(1/f)$ whenever $f \notin \cO_{X,P}$.
If $\ord_P(f) = n >0$ we say that $f$ has a \emph{zero of order $n$} at $P$, whilst if $\ord_P(f) = n < 0$ then we say that $f$ has a \emph{pole of order $n$} at $P$.
Clearly, for any $f, g \in K(X)$ and $P \in X$, it is true that $\ord_P(fg) = \ord_P(f) + \ord_P(g)$.
We call any element $t \in \cO_{X,P}$ which has order $1$ at $P$ a \emph{uniformising parameter at $P$}.

    \begin{prop}\label{propfinitelymanyzeroesandpoles}
    Any non-zero meromorphic function $f$ on $X$ has finitely many poles and zeroes.
    Moreover, the number of poles and zeroes of $f$ are equal, counting multiplicity; \ie 
        \[
        \sum_{P \in X} \ord_P(f) = 0.
        \]
    \end{prop}
    \begin{proof}
    See \cite[Chap.\ 8, \S 1, Prop.\ 1]{fulton}.
    \end{proof}


We now introduce the concept of a differential on the curve $X$.
Let $R$ be any commutative ring containing $k$ and let $M$ be an $R$-module.
Then a $k$-linear map $D \colon R \ra M$ satisfying $D(fg) = fD(g) + gD(f)$ is called a \emph{derivation} of $R$ in to $M$ over $k$.

There exists a unique module $\Omega_k(R)$, and a derivation $d \colon R \ra \Omega_k(R)$, through which all derivations of $R$ over $k$ must factor.
We can describe this more concretely as the free module generated by $[f]$ for all $f \in K(X)$, quotiented by the relations
    \begin{itemize}
    \item $[f]+[g] = [f+g]$,
    \item $[cf] = c[f]$,
    \item $[fg] = f[g] + g[f]$,
    \end{itemize}
where $f, g \in R$ and $c \in k$.

In particular, when $R = K(X)$ we let $\Omega_{K(X)} := \Omega_k(K(X))$, and we call the map $d \colon K(X) \ra \Omega_{K(X)}$ the \emph{differential map}.
We then call $\Omega_{K(X)}$ the \emph{module of differentials of $X$}, and we call any $ \omega \in \Omega_{K(X)}$ a \emph{meromorphic differential}.

    \begin{prop}\label{propdifferentialsareonedimensional}
    The module of differentials, $\Omega_{K(X)}$, is a one dimensional vector space over $K(X)$.
    \end{prop}
    \begin{proof}
    See \cite[Prop. 1.5.9]{stichtenoth}.
    \end{proof}

We suppose that $P \in X$ and we choose a uniformising parameter $t \in \cO_{X,P}$.
Then for any $ 0 \neq \omega \in \Omega_{K(X)}$ there exists a unique integer $n \in \ZZ$ and unit $u \in \cO_{X,P}$ such that $\omega = ut^ndt$, by Proposition \ref{propdifferentialsareonedimensional}.
Then we define the \emph{order of $\omega$ at $P$} to be $\ord_P(\omega) := n$.

Let $U$ be an open subset of $X$.
We call a differential $\omega$ \emph{holomorphic on $U$} if $\ord_P(\omega) \geq 0$ for all $P \in U$, and we let
    \[
    H^0(U, \Omega_X) := \{ \omega \in \Omega_{K(X)} | \ord_P(\omega) \geq 0\ \text{for all}\ P \in X\}
    \]
be \emph{the space of holomorphic differentials on $U$}.
If $\omega \in \Omega_{K(X)}$ is holomorphic on $X$ we say that $\omega$ is \emph{holomorphic}, and so $\hzero$ is the space of holomorphic differentials.

\section{The Riemann-Roch theorem}

We now recall the relevant facts and definitions needed to state the Riemann-Roch theorem.

We first recall that a \emph{divisor} on $X$ is a finitely supported formal sum 
    \[
    D = \sum_{P \in X} n_P[P],
    \]
with coefficients in $\ZZ$.
The set off all divisors on $X$ forms an additive group, denoted $\di (X)$.
The \emph{degree} of the divisor $D$ is $\deg(D) := \sum_{P \in X} n_P$.


Given any function $f \in K(X)$ we define the \emph{divisor associated to $f$} to be
    \[
    \di(f) := \sum_{P \in X} \ord_P(f) [P].
    \]
Note that by Proposition \ref{propfinitelymanyzeroesandpoles} this has finite support and degree zero.
We call any divisor $D$ which is equal to $\di(f)$ for some $f \in K(X)$ a \emph{principal divisor}.
Is is clear that for any $f, g \in K(X)$ we have $\di(fg) = \di(f) + \di(g)$.
Also, for any $f \in K(X)$ we define $\di_0(f)$ and $\di_\infty (f)$, the \emph{divisor of zeroes} and the \emph{divisor of poles} of $f$ respectively, as follows:
    \[
    \di_0(f) := \sum_{\ord_P(f) >0} \ord_P(f)[P]
    \]
and then
    \[
    \di_\infty(f) := \di_0(f) - \di(f).
    \]

Now for any differential $0 \neq \omega \in \Omega_{K(X)}$ we define the \emph{divisor associated to $\omega$} to be
    \[
    \di(\omega) := \sum_{P \in X} \ord_P(\omega)[P].
    \]
If $W$ is a divisor on $X$ and $W = \di(\omega)$ for some $ \omega \in \Omega_{K(X)}$ then we say that $W$ is a \emph{canonical divisor} on $X$.

The principal divisors of $X$ form a subgroup of $\di(X)$, and we say that two divisors $D, D' \in \di(X)$ are equivalent, denoted $D \sim D'$, if their image in the quotient of $\di(X)$ by the group of principal divisors is the same; \ie if there exists $f \in K(X)$ such that $ D = D' + \di(f)$.
By the following theorem, it makes sense to refer to the (unique) canonical divisor on $X$, up to equivalence, which we write as $K_X$.

    \begin{thm}
    Let $W$ and $W'$ be canonical divisors on $X$.
    Then there exists some $f \in K(X)$ such that $W = W' + \di(f)$.
    Moreover, the divisor $W + \di(f)$ is canonical for every $f \in K(X)$.
    \end{thm}
    \begin{proof}
    For any $f \in K(X)$ and differential $\omega \in \Omega_{K(X)}$ it is true that $\di(f \omega) = \di(f) + \di(\omega)$.
    Then the second statement follows from the fact that $\Omega_{K(X)}$ is a $K(X)$ vector space.


    To prove the first statement, suppose that $\omega, \omega' \in \Omega_{K(X)}$ are differentials such that $W = \di(\omega)$ and $W' = \di(\omega')$.
    Then, by Proposition \ref{propdifferentialsareonedimensional}, there exists an $f \in K(X)$ such that $\omega = f \omega'$.
    We conclude the proof by taking the divisors of both sides of this equation.
    \end{proof}

Given any divisor $D = \sum_{P \in X} n_P[P]$ we let
    \[
    H^0(X,\cO_X(D)) : = \{ f \in K(X) | \ord_P(f) \geq n_P\ \text{for all}\ P \in X\}
    \]
be the \emph{vector space of meromorphic functions associated to $D$}.
Similarly, we let 
    \[
    H^0(X,\Omega_X(D)) :  = \{ \Omega \in \omega_{K(X)} | \ord_P(\omega) \geq n_P \ \text{for all} \ P \in X\}
    \]
be the \emph{vector space of meromorphic differentials associated to $D$}.
In particular, when $D$ is the zero divisor we have $H^0(X,\Omega(0)) = \hzero$, and similarly $H^0(X,\cO_X(0)) = H^0(X,\cO_X)$.

We now state the celebrated Riemann-Roch theorem.

    \begin{thm}[Riemann-Roch theorem]\label{theoremriemannroch}
    Let $D$ be any divisor on $X$, and let $W$ be any canonical divisor on $X$.
    Then
        \[
        \dim_k(H^0(X,\cO_X(D))) = \deg(D) + 1 - g_X + \dim_k(H^0(X,\cO_X(W-D)), 
        \]
    for some constant $g_X \in \ZZ_{\geq 0}$, which is independent of the choice of $W$ and $D$.
    \end{thm}
    \begin{proof}
    See \cite[Chap.\ IV, \S 1, Thm.\ 1.3]{hart} or, for a more elementary approach, \cite[Chap.\ 8, \S 6]{fulton}.
    \end{proof}

    \begin{defn}
    The constant $g_X$ in the statement of the Riemann-Roch theorem is the genus of $X$.
    \end{defn}

The genus is an invariant of fundamental importance in the study of algebraic curves.
In particular, we remark that if $k = \CC$ then the genus of an algebraic curve (also called the arithmetic genus) is the same as the topological genus of the corresponding Riemann surface (the corresponding Riemann surface being found via the equivalence of categories mentioned earlier).

We now give a corollary to the Riemann-Roch theorem, which shows some of the properties of the genus.

    \begin{cor}\label{dim=gc}
    For any canonical divisor $W$ on $X$, we have 
        \[
        \deg(W) = 2g_X-2
        \]
    and 
        \[
        \dim H^0(X,\cO_X(W)) = g_X.
        \]
    \end{cor}
    \begin{proof}
    Since $\dim H^0(X,\cO_X(0)) = 1$ (the only functions with no poles are the constant functions), we have by the Riemann-Roch theorem, 
        \[
        1= \dim H^0(X,\cO_X(0)) = 0 + 1 -g_X + \dim H^0(X,\cO_X(W)).
        \]
    Rearranging this gives the second statement.
    The first statement then follows by rearranging
        \begin{multline*}
        g_X = \dim H^0(X,\cO_X(W)) = \deg(W) + 1 -g_X +  \dim H^0(X,\cO_X(W-W))\\ = \deg(W) + 1 -g_X + 1.
        \end{multline*}
    \end{proof}

\section{Ramification and the Riemann-Hurwitz formula}

In this section we will introduce the concept of ramification, and conclude by stating the Riemann-Hurwitz formula, which relates the canonical divisor of two curves via the ramification divisor.



Let $X$ and $Y$ be curves over $k$.
We begin by defining what it means for a map $\phi \colon X \ra Y$ to be a morphism of curves.
We first note that given a map of sets $\phi \colon X \ra Y$ we have an induced ring homomorphism on the function fields,
    \[
    \phi^* \colon K(Y) \ra K(X),
    \]
given by composition with $\phi$; \ie $\phi^*(f) = f \circ \phi$.
Moreover, it transpires that $\phi^*$ is an injection, and hence we can view $K(Y)$ as a subfield of $K(X)$.
We then define the \emph{degree} of $\phi$ to be the degree of the extension $K(X)/K(Y)$.

Then a \emph{morphism of curves} is a continuous map $\phi \colon X \ra Y$ such that for every open set $U \subseteq Y$ we have
    \[
    \phi^*(H^0(U,\cO_Y)) \subseteq H^0(\phi^{-1}(U),\cO_X).
    \]

Let $G$ be a subgroup of the automorphism group of $X$.
Then have a natural morphism $\pi \colon X \ra X/G$, which is a projection of $X$ on to the quotient of $X$ by the action of $G$.
The majority of the topics considered in this thesis will focussed on situations where we have such a projection.

We now return to assuming that $\phi \colon X \ra Y$ is an arbitrary morphism of curves.
    \begin{defn}
    Let $P \in X$ and choose a uniformising parameter $t \in \cO_{Y,f(P)}$.
    We define the ramification index $e_P$ of $\phi$ at $P$ to be
        \[
        e_P := \ord_P(\phi^*(t)).
        \]
    \end{defn}

Note that $e_P =1$ for almost all points $P \in X$.
We say that the point $Q \in Y$ is a \emph{branch point} of $\phi$ if there exists some $P \in \phi^{-1}(Q)$ for which $e_P >1$.
We say that $P \in X$ is a \emph{ramification point} of $X$ if $e_P >1$.

    
Suppose $P \in X$ is a ramification point.
Then if $p = \cha(k)$ divides $e_P$ we say that $P$ is \emph{wildly ramified}, and we also say $\phi$ is wildly ramified if this holds for any point in $X$.
If $p$ does not divide $e_P$ we say that $P$ is tamely ramified.

    \begin{defn}
    Let $D = \sum_{Q \in Y}n_Q [Q]$ be a divisor on $Y$.
    Then the pull back of $D$ with respect to $\phi$ is
        \[
        \phi^*(D) := \sum_{Q \in Y} \sum_{P \mapsto Q} e_P \cdot n_Q [P].
        \]
    \end{defn}

Note that $\phi^*$ defines a group homomorphism $\di(Y) \ra \di(X)$.


We also define the pullback of a differential $\omega = g\cdot df \in \Omega_{K(Y)}$ by $\phi$ to be
    \[
    \phi^*(\omega) := \phi^*(g)d\phi^*(f).
    \]
Clearly $\phi^*(\omega)$ is a differential on $X$.


Now we describe the different exponent, which we require to define the ramification divisor.
    \begin{defn}\label{definitiondifferent}
    For any $P\in X$ let $Q = \phi(P) \in Y$.
    Moreover, we choose uniformising parameters $s \in \cO_{X,P}$ and $t \in \cO_{Y,Q}$.
    Then by Proposition \ref{propdifferentialsareonedimensional} there exists a unique $f \in K(X)$ such that $\phi^*(dt) = f\cdot ds$.
    Then we define the different exponent at $P$ to be 
        \[
        \delta_P : = \ord_P(f).
        \]
    \end{defn}

Note that since $s$ is a uniformising parameter at $P$, and $\phi^*(t)$ is regular at $P$, it follows that $f$ is regular at $P$; in particular, $\delta_P$ is non-negative for all $P \in X$.
 
    \begin{defn}[Ramification divisor]\label{defnramificationdivisor}
    The ramification divisor of $\phi \colon X \ra Y$ is 
        \[
        R:= \sum_{P \in X} \delta_P [P].
        \]
    If $\phi$ is tamely ramified then 
        \[
        R: = \sum_{P \in X} (e_P - 1)[P].
        \]
    \end{defn}

The following theorem has the classical Riemann-Hurwitz formula as a corollary, but also goes further, actually relating the canonical divisors on $X$ and $Y$.
    \begin{thm}\label{theoremdetailedhurwitz}
    If $0 \neq \omega \in \Omega_{K(Y)}$ then
        \begin{equation}\label{equationstronghurwitzformula}
        \di(\phi^*(\omega)) = \phi^*(\di(\omega)) + R.
        \end{equation}
    In particular, we have
        \[
        K_X \sim \phi^*(K_Y) + R.
        \]
    \end{thm}
    \begin{proof}
    See \cite[Chap.\ IV, \S 2, Prop.\ 2.3]{hart} for a sheaf theoretic approach, or alternatively \cite[Thm. 3.4.6]{stichtenoth}, for a proof involving function fields.
    \end{proof}


As a corollary to this we have the Riemann-Hurwitz formula.
    \begin{cor}\label{corhurwitzformula}[Riemann-Hurwitz Formula]
    Given two non-singular projective curves $X$ and $Y$ of genera $g_X$ and $g_Y$ respectively, with a degree $n$ map $\phi\colon X \rightarrow Y$, then
        \[
        2g_X - 2 = n(2g_Y -2) + \deg(R),
        \]
    where $R$ is the ramification divisor of $\phi$.
    \end{cor}
    \begin{proof}
    This follows from Corollary \ref{dim=gc}, after taking degrees in \eqref{equationstronghurwitzformula}.
    \end{proof}


We will use the above statements in the subsequent sections.
In order to make full use of them however, we need to use Hilbert's formula, which gives an alternative description of $\delta_P$ which is easier to compute.
However, in order to state Hilbert's formula we need to introduce higher ramification groups.
   \begin{defn}
    Let $G:=\gal(X/Y)$ and let $t$ be a uniformising parameter at $P\in X$.
    Then for $i\geq -1$ we define the $i^{th}$ ramification group at $P$, denoted $G_i(P)$, to be the subgroup of $s\in G$ such that $i_G(s) := \ord_P(s(t)-t)$ is at least $i+1$.
    This is	independent of the choice of $t$, see \cite[Chap. IV, \S 1, pg. 62]{localfields}.
    \end{defn}

Note that for any $P\in X$ we have that $G_{-1}(P)=G$, if $i$ is sufficiently large then $G_i(P)$ is trivial and $G_i(P)\supseteq G_{i+1}(P)$.
Also, $G_1$ is a $p$-group and $\ord(G_0(P)/G_1(P))$ is coprime to $p$.
In particular, $\phi$ is tamely ramified at $P$ if and only if $G_1(P)$ is the trivial group.
More details can be found in \cite[Chap. IV, \S 1]{localfields}.%Citation is slightly ambiguous, but only references four pages, so is okay.

Note that $e_P = \ord (G_0(P))$ for any $P \in X$.
We now state Hilbert's formula, relating our presentation of the ramification divisor in Definition \ref{defnramificationdivisor} to the ramification groups, which will make future computations easier.


    \begin{thm}[Hilbert's Formula]\label{hilbertsformula}
    Suppose that $P\in X$ and that $t$ is a uniformising parameter in $\cO_{X,P}$.
    Then we have
        \begin{equation}
        \delta_P = \sum_{s\neq e}i_G(s)=\sum_{j=0}^{\infty}\left(\ord(G_i(P))-1\right),\label{ramdiv}
        \end{equation}
    where $e$ denotes the identity in $G$.
    \end{thm}
    \begin{proof}
    For the sake of brevity we do not prove these statements here. See \cite[Chap. IV, \S 1, Prop. 4]{localfields} for a proof of Hilbert's formula.
    \end{proof}

 
\section{Serre duality for curves}

In this section we give the details of Serre duality, in such a way that we will be able to perform explicit computations using Serre duality in later chapters.


We retain the notations of the previous sections.
We further let $\underline{\Omega}_{K(X)}$ and $\underline{K}(X)$ denote the constant sheaves of $\Omega_{K(X)}$ and $K(X)$ respectively, and $\Omega_X$ denote the sheaf of differentials on $X$.
The following lemma gives us useful descriptions of $\hone$ and $H^1(X,\Omega_X)$.
    \begin{lem}\label{lemmaexactsequenceofhoneandhzero}
    We have canonical exact sequences as follows:
        \begin{equation}\label{equationinitialdualitylesfunctions}
        0 \ra H^0(X,\cO_X) \ra K(X) \ra \bigoplus_{P \in X} K(X)/\cO_{X,P} \rightarrow \hone \ra  0;
        \end{equation}
        \begin{equation}\label{equationinitialdualitysesdifferentials}
        0 \rightarrow \hzero \ra \Omega_{K(X)} \ra \bigoplus_{P \in X}\Omega_{K(X)}/\Omega_{X,P} \rightarrow H^1(X,\Omega_X) \ra 0.
        \end{equation}
    \end{lem}
    \begin{rem}
    Note that a sketch of the proof below can be found in \cite[Pg.\ 248]{hart}.
        \end{rem}
    \begin{proof}
    The short exact sequence
        \begin{equation}\label{equationinitialdualitysesfunctions}
        0 \rightarrow \cO_X \rightarrow \underline{K}(X) \rightarrow \underline{K}(X)/\cO_X \rightarrow 0
        \end{equation}
    is a flasque resolution of $\cO_X$ (see \cite[Chap.\ II, ex.\ 1.16]{hart}).
    
    For each $P \in X$ we have a natural embedding $i\colon \{P\} \hookrightarrow X$, and we view the module $K(X)/\cO_{X,P}$ as a sheaf on the singleton $\{P\}$.
    Then for each $P\in X$ we have the induced sheaf $i_*\left( K(X)/\cO_{X,P} \right)$ on $X$.
    If we consider the direct sum of these induced sheaves over all points $P\in X$ we have the following isomorphism
        \begin{equation}\label{equationsheafisomorphism}
        \underline{K}(X)/\cO_X\cong \bigoplus_{P\in X} i_*\left(K(X)/\cO_{X,P}\right).
        \end{equation}
    
    
    To explain this isomorphism we first construct a map from $\underline{K}(X)/\cO_X$ in to the product $\prod_{P \in X} i_*\left({K(X)}/\cO_{X,P}\right)$, and then show that the image of each element under this map has finite support.
    
    Given $i\colon \{P\} \hookrightarrow X$ we have the following equalities
        \begin{align*}
        i^{-1}\left(\underline{K}(X)/\cO_X\right) & = \left(\underline{K}(X)/\cO_X\right)_P \\
        & = \underline{K}(X)_P/\cO_{X,P} \\
        & = K(X)/\cO_{X,P}.
        \end{align*}
    It follows that for any $P \in X$ we have the adjunction map $\underline{K}(X)/\cO_X \ra i_* \left( K(X)/\cO_{X,P} \right)$.
    Since we have an injection $i$ for every $P \in X$, we can actually produce a map in to the product $\prod_{P \in X} i_*\left(K(X)/\cO_{X,P}\right)$.
    Moreover, the stalk $i_*\left( K(X)/\cO_{X,P} \right)_Q$ is zero for $Q \neq P$ and is $K(X)/\cO_{X,P}$ when $Q = P$.
    Hence the product is actually isomorphic to the sum $\bigoplus_{P \in X} \left( i_*\left(K(X)/\cO_{X,P} \right)\right)$, and from this the isomorphism in \eqref{equationsheafisomorphism} follows.
    
    Replacing $\underline{K}(X)/\cO_X$ by $\bigoplus_{P\in X} i_*\left(K(X)/\cO_{X,P}\right)$ in \eqref{equationinitialdualitysesfunctions} yields
        \begin{equation}\label{equationshortexactsequencefunctionsdirectsum}
        0 \ra \cO_X \ra \underline{K}(X) \ra \bigoplus_{P \in X}i_* \left( K(X)/\cO_{X,P} \right) \ra 0.
        \end{equation}
    Taking cohomology we arrive at the exact sequence \eqref{equationinitialdualitylesfunctions}.
    
    We now perform a similar computation to produce the second exact sequence \eqref{equationinitialdualitysesdifferentials}.
    We start with the short exact sequence
        \[
        0 \ra \Omega_X \ra \underline{\Omega}_{K(X)} \ra \underline{\Omega}_{K(X)}/\Omega_X \ra 0,
        \]
    which is a flasque resolution of $\Omega_X$ (see \cite[Chap.\ II, ex.\ 1.16]{hart}).
    For each $P \in X$ we again have a natural injection $i \colon \{ P \} \hookrightarrow X$, giving rise to the induced sheaf $i_*\left( K(X)/\cO_{X,P} \right)$ on $X$.
    Then we have an isomorphism
        \begin{equation}\label{equationsheafisomorphismdifferentials}
        \underline{\Omega}_{K(X)}/\Omega_X \cong \bigoplus_{P \in X} i_* \left( \Omega_{K(X)}/\Omega_{X,P}\right),
        \end{equation}
    similar to that in \eqref{equationsheafisomorphism}.

    Hence we arrive at the short exact sequence
        \begin{equation}\label{equationshortexactsequencedifferentialsdirectsum}
        0 \ra \Omega_X \ra \underline{\Omega}_{K(X)} \ra \bigoplus_{P \in X} i_* \left( \Omega_{K(X)}/\Omega_{X,P} \right)  \ra 0.
        \end{equation}
    Taking cohomology of this then yields the second exact sequence \eqref{equationinitialdualitysesdifferentials}.
    \end{proof}

    \begin{rem}
        When considering elements of $H^1(X,\Omega_X)$ as elements in the cokernel of the map $\Omega_{K(X)} \ra \bigoplus_{P \in X}\Omega_{K(X)}/\Omega_{X,P}$ above, we will denote them by $\overline{(\omega_P)}_{P \in X}$, where ${(\omega_P)}_{P \in X}\in \bigoplus_{P \in X}\Omega_{K(X)}/\Omega_{X,P}$.
        Similarly, when considering elements of $\hone$ as elements of the cokernel of the map $K(X) \ra \bigoplus_{P \in X}K(X)/\cO_{X,P}$, we will denote them by $\overline{(f_P)}_{P \in X}$, where ${(f_P)}_{P \in X} \in \bigoplus_{P \in X}K(X)/\cO_{X,P}$.
    \end{rem}

The residue map $\res_P \colon \Omega_{K(X)} \ra k$ is of fundamental importance in the computations that follow.
We define the {\em residue map}, $\res_P$, to be the unique map identified in the following theorem.

    \begin{thm}\label{theoremresiduemap}
    For any $P\in X$ there exists a unique $k$-linear map $\res_P \colon \Omega_{K(X)} \ra k$ defined by the following properties:
        \begin{itemize}
            \item $\res_P(\omega) = 0$ for all $\omega \in \Omega_{X,P}$;
            \item $\res_P(f^ndf) = 0$ for all $f \in K(X)^*$ and all $n \neq -1$;
            \item $\res_P(f^{-1}df) = \ord_P(f)$, where $\ord_P(f)$ is the order of $f$ at $P$.
        \end{itemize}
    \end{thm}
    \begin{proof}
    See \cite[Chap.\ II, \S 7 and \S 11]{algebraicgroupsandclassfields} or \cite{residuesofdifferentialsoncurve}.
    \end{proof}


    \begin{thm}[Residue Theorem]\label{theoremresiduetheorem}
    Given any differential $\omega \in \Omega_{K(X)}$ on $X$ then $\sum_{P \in X}\res_P(\omega) = 0$.
    \end{thm}
    \begin{proof}
    See \cite[Chap.\ II, Prop.\ 6]{algebraicgroupsandclassfields} or \cite[Pg.\ 155]{residuesofdifferentialsoncurve}.
    \end{proof}

Since $\Omega_{X,P} \subseteq \ker (\res_P)$, it follows that $\res_P$ is a well defined function on the quotient $\Omega_{K(X)}/\Omega_{X,P}$.
Hence by the residue theorem the map
    \begin{equation*} 
    \bigoplus_{P \in X} \Omega_{K(X)}/\Omega_{X,P} \ra k, \quad (\omega_P)_{P \in X} \mapsto \sum_{P\in X} \res_P(\omega_P)
    \end{equation*} 
vanishes on the image of $\Omega_{K(X)}$, which allows us to make the following definition.
    
    \begin{defn}
    Let $\overline{(\omega_P)}_{P \in X} \in H^1(X,\Omega_X)$.
    Then we define the trace map to be 
        \[
    t \colon H^1\left(X, \Omega_X\right) \ra k, \qquad \overline{(\omega_P)}_{P \in X}  \mapsto \sum_{P \in X} \res_P(\omega_P).
        \]
    \end{defn}

We now use the trace map to define a pairing between the $k$-vector spaces $\hone$ and $\hzero$.
Since $\Omega_{K(X)}$ is a $K(X)$-module, we can define the product map 
    \begin{equation}\label{equationproductmap}
    \hzero \times \hone \ra H^1\left(X, \Omega_X\right), \ (\omega, \overline{(f_P)}_{P \in X}) \mapsto ( \overline {(f  \omega)_P})_{P \in X},
    \end{equation}
where $(f\omega)_P$ denotes the product of $f_P \in K(X)/\cO_{X,P}$ with the residue class of $\omega$ in $\Omega_{K(X)}/\Omega_{X,P}$.

We now combine the product map in \eqref{equationproductmap} with the trace map $t$ to get a map 
    \[
    \hzero \times \hone \ra k,\quad (\omega, \overline{(f_P)}_{P \in X}) \mapsto \langle \omega, \overline{(f_P)}_{P \in X} \rangle := t \left( \overline{(f \omega)}_P \right)_{P \in X}.
    \]

    \begin{thm}\label{theoremserreduality}
    Via the pairing $\langle\ ,\ \rangle$, the $k$-vector spaces $\hone$ and $\hzero$ are dual to each other.
    \end{thm}
    \begin{proof}
    This is a specialisation of \cite[Chap.\ II, Thm.\ 2]{algebraicgroupsandclassfields}.
    \end{proof}

More explicitly, this theorem means the following.
If we fix any $\omega \in \hzero$ we produce a map $\theta(\omega)\colon \hone \ra k$, given by $\theta(\omega)(f) = \langle \omega , f\rangle$.
Similarly, if we fix any $f \in \hone$ then we get a map $\psi(f) \colon \hzero \ra k$.
Then the maps 
    \[
    \psi \colon \hone \ra \hom(H^0(X,\Omega_X),k) \quad
    \text{and} \quad    
    \theta \colon H^0(X,\Omega_X) \ra \hom(\hone, k)
    \]
are isomorphisms.
In particular, given a $k$-basis $\omega_1, \ldots, \omega_g$ of $\hzero$, we can find a basis $f_1, \ldots , f_g$ of $\hone$ such that $\langle \omega_i, f_j \rangle = 1$ for all $1 \leq i \leq n$ and $\langle \omega_i, f)j \rangle = 0$ if $i \neq j$, and likewise, starting with a basis of $\hone$ we can find corresponding basis of $H^0(X,\Omega_X)$. 






















\chapter{Bases for the spaces of (poly)differentials on hyperelliptic curves}\label{chapterhyperellipticcurves}


In this chapter we recall the definition and basic details of hyperelliptic curves, and then go on to compute bases for the spaces of holomorphic differentials and polydifferentials, see Propositions \ref{prophyperellipticbasispnot2} and \ref{propbasishyperellipticp=2}.
The primary use of these concepts is to form a foundation for the next chapter.
Furthermore, we also use the bases computed to illustrate all the facets of our main theorem in Chapter 5.
The various attributes of hyperelliptic curves differ greatly according to whether the characteristic of the base field is two or not, and as such we split this chapter into two sections, considering these cases separately.


Before going in to the details of hyperelliptic curves we recall that a curve $X$ is hyperelliptic if there exists a finite separable morphism $\pi \colon X \ra \PP_k^1$ of degree two.\index{Hyperelliptic curve}
Every hyperelliptic curve has a hyperelliptic involution $\sigma$ which permutes the elements of $\pi^{-1}(a)$ for each $a \in \mathbb P_k^1$ (except for the finite number of points $a$ for which $\pi^{-1}(a)$ has order one), and the quotient curve $X /\langle \sigma \rangle$ is isomorphic to $\mathbb P_k^1$.
We let $X$ be a hyperelliptic curve of genus $g$ throughout the chapter, and we fix such a map $\pi$, which is unique up to an automorphism of $\PP_k^1$ \cite[Prop.\ 7.4.29]{liu}.
We also let $P_a$ and $P_a'$ denote the unique elements of $\pi^{-1}(a)$ for any point $a \in \mathbb P_k^1$ that is not a branch point.
If $a \in \mathbb P_k^1$ is a branch point we denote the unique point in $\pi^{-1}(a)$ by $P_a$.
We define $D_a$ to be the divisor $\pi^*\left([a]\right)$ for any $a \in \mathbb P_k^1$, and hence
    \begin{equation*}
    D_a= 
        \begin{cases}
         2[P_a] & \text{if $a$ is a branch point}, \\
         [P_a] + [P_a'] & \text{otherwise.}
        \end{cases}
    \end{equation*}

We also have for $x \in k(x)  = K(\mathbb P_k^1) \subseteq K(X)$, that
    \begin{equation}\label{equationdivxpis2}
    \di (x)  = D_0 - D_\infty,
    \end{equation}
regardless of characteristic.
Furthermore, the strong Riemann-Hurwitz formula (Theorem \ref{theoremdetailedhurwitz}) gives us
    \[
    \di_X (dx) = \pi^*( \di_{\mathbb P^1_k}(dx)) + R,
    \]
and since $\di_{\PP_k^1}(dx) = -2[\infty]$, it follows that $\pi^*(\di_{\PP_k^1}(dx)) = -2D_\infty$.
Hence we conclude that
    \begin{equation}\label{equationdifferentialdivisor}
    \di(dx) = R - 2D_\infty.
    \end{equation}


\section{Characteristic unequal to 2}\label{charneq2}

In this section we assume that $\cha(k) = p \neq 2$.
Then the extension $K(X)$ of $K(\mathbb P_k^1) = k(x)$ corresponding to $ \pi \colon X \ra \mathbb P_k^1$ will be $k(x,y)$, where $y$ satisfies
    \begin{equation}\label{equationdefiningequationpnot2}
    y^2 = f(x)
    \end{equation}
for some polynomial $f(x) \in k[x]$ which has no repeated roots and is of degree $2g+1$ or $2g+2$ \cite[Prop.\ 7.4.24]{liu}.
Moreover, by applying an automorphism of $\mathbb P_k^1$ if necessary, we can and will assume that $f(x)$ is monic.

If we let $d_f:=\deg(f(x))$ then 
    \begin{equation}\label{equationexpansionoff(x)}
    f(x) = \prod_{i=1}^{d_f} (x-a_i) = x^{d_f} + b_{d_f - 1}x^{d_f-1} + \ldots + b_0,
    \end{equation}
for some $a_i, b_i \in k$.
We now show that the $a_i \in \AA_k^1$, and possibly $\infty \in \PP_k^1$, are the branch points of $\pi$.

Firstly, observe that by the Riemann-Hurwitz formula, Corollary \ref{corhurwitzformula},
    \[ 
    \deg(R) = 2g -2 +2\cdot 2 = 2g + 2.
    \]
Since $\pi$ is of degree two and $\cha (k) \neq 2$ it is only tamely ramified, and it follows that the coefficient of each ramification point is $1$ in $R$.
From this we conclude that each branch point has precisely one corresponding ramification point, and that there are precisely $2g+2$ ramification points.
Also, since there are no repeated roots in $f(x)$, then \eqref{equationdefiningequationpnot2} defines a non-singular affine curve $X'$ with a degree two projection $\pi'\colon X'\rightarrow \mathbb A_k^1$.
For any point $a\in \mathbb A_k^1$ which is not a solution to $f(x)$ there are two points in the pre-image, namely $(a,\pm \sqrt{a})$, and the point is not a branch point.
On the other hand, if $a = a_i \in \mathbb A_k^1$ is a solution to $f(x)$, then there is only one point in the pre-image and hence it is a branch point.
We let $P_i = P_{a_i}$ denote the ramification point corresponding to $a_i$.
Since $\deg(R) = 2g+2$ we conclude that if $d_f = 2g+1$ then $\infty \in \mathbb P_k^1$ is also a branch point and we define $P_{2g+2} := P_\infty$ in this case.
Hence the ramification divisor $R$ of $\pi$ is
    \[
    R = \sum_{i=1}^{2g+2} [P_i] .
    \]


In the following lemma we compute the divisor of $y \in K(X)$.
    \begin{lem}\label{lemmadivisorofycharacteristicnot2}
    The divisor of $y \in K(X)$ is 
        \begin{equation}\label{equationdivisorofypnot2}
        \di(y)  = R - (g+1)D_\infty.
        \end{equation}
    \end{lem}
    \begin{proof}
    Since $\di (y^2) = \di (f(x))$ and hence $\di(y) = \frac{1}{2}\di(f(x))$, we need only compute the divisor of $f(x)$.
    As noted earlier, the solutions to $f(x)$ correspond to the ramification points.
    So for any $P\notin x^{-1}(\infty)$ then $\ord_P(y) =  \frac{1}{2}\ord_P(f(x)) = 1$ if $P$ is a ramification point, and $\ord_P(y) = \frac{1}{2}\ord_P(f(x)) = 0$ otherwise.

    We now consider the poles of $y$.
    By Proposition \ref{propfinitelymanyzeroesandpoles} we know that $\sum_{P \in X} \ord_P(f(x)) = 0$, and we also know that the poles of $f(x)$ can only lie in $\pi^{-1}(\infty)$.
    Hence if $\infty$ is a branch point then $\ord_{P_\infty}(f(x)) = -\sum_{i=1}^{2g+1}\ord_{P_i}(f(x)) = -2(2g+1)$, and $\ord_{P_\infty}(y) = -(2g+1)$.
    On the other hand, if $\infty$ is not a branch point we know that $\ord_{P_\infty}(f(x)) + \ord_{P_\infty'}(f(x)) = -2(2g+2)$.
    Recall that $\ord_{P}(\sigma(f(x))) = \ord_{\sigma(P)}(f(x))$ for any automorphism $\sigma \in \aut(X)$ and any point $P \in X$.
    In particular, if $\sigma$ is the hyperelliptic involution of $X$ then 
        \[
        \ord_{P_\infty}(f(x)) = \ord_{P_\infty}(\sigma(f(x))) = \ord_{\sigma(P_\infty)}(f(x)) = \ord_{P_\infty'}(f(x)).
        \]
    Hence $\ord_{P_\infty}(y) = \ord_{P_\infty'}(y) = -(g+1)$
    Overall, we conclude that
        \[
        \di(y) = \sum_{i=1}^{2g+2} [P_i]- (g+1)D_\infty = R - (g+1)D_{\infty}.
        \]
    \end{proof}


    \begin{prop}\label{prophyperellipticbasispnot2}
    Let $m\geq 1$.
    Let $X$, $x$ and $y$ be as above, and let $\omega := \frac{dx^{\otimes m}}{y^m}$. 
    Then if $g\geq 2$, a basis of $H^0(X,\Omega_X^{\otimes m})$ is given by\par
        {\centering 
        \begin{tabular}{c c}
        $\omega, x\omega, \ldots , x^{g-1}\omega$ &  if $m=1$, \\
        $\omega, x\omega, x^2\omega$  &  if $m=g=2$, \\
        $\omega, x\omega, \ldots, x^{m(g-1)}\omega;\  y\omega, xy\omega, \ldots, x^{(m-1)(g-1)-2}y\omega$ &  otherwise.
        \end{tabular}\par
        }
    \end{prop}

    \begin{rem}
    Note that the case where $m=1$ is treated in \cite[Prop. 7.4.26]{liu} and \cite[Ch. IV, \S 4, Prop. 4.3]{griffiths}.
    \end{rem}

    \begin{proof}
    We first show that the elements are linearly independent over $k$.
    Since $\omega$ is fixed, it is equivalent to show that the coefficients are linearly independent over $k$ - \ie that $1,x,\ldots ,x^n, y, xy, \ldots, x^ly$ are linearly independent over $k$ for any $n$ and $l$ in $\NN$.
    It is immediate that $1, x, \ldots, x^n$ are linearly independent, and similarly that $y, yx, \ldots, yx^l$ are linearly independent.
    Finally, the two sets of elements are linearly independent of each other, otherwise the extension $K(X)/k(x)$ would be degree 1.

    To show that the differentials in the statement of the lemma are indeed holomorphic differentials, we show that their divisors are greater than $0$.
    Recall that $\di (dx^{\otimes m}) =m\di (dx)$, as noted in the previous chapter.
    We now show that the differentials listed in Proposition \ref{prophyperellipticbasispnot2} are holomorphic.
    We have that
        \begin{equation}\label{eqbasedifferentialdivisor}
        \begin{split}
        \di(x^i\omega) & =  \di \left( \frac{x^idx^{\otimes m}}{y^m} \right)\\ 
        & =  i(D_0 -D_\infty) + m(R-2D_\infty) -m(R-(g+1)D_\infty) \\
        & =  iD_0 + (mg -m -i)D_\infty \\
        & =  iD_0 + (m(g-1) -i)D_\infty,
        \end{split}
        \end{equation}
    by Lemma \ref{lemmadivisorofycharacteristicnot2}, \eqref{equationdivxpis2} and \eqref{equationdifferentialdivisor}, which is positive for $0\leq i \leq m(g-1)$.
    Hence all the polydifferentials in the first two cases and the first $m(g-1)+1$ differentials in the third case are holomorphic.
    Note that if $m=g=2$ then there are three elements, and since $\dim_kH^0(X,\Omega_X^{\otimes 2})=3$ by Corollary \ref{dim3}, these elements form a basis.
    Also, if $m=1$ then by Definition \ref{definitiongenus} $\dim_k H^0(X,\Omega_X)=g$, and we have $g$ linearly independent elements, so they again must form a basis.
    
    
    
    We now consider the final $(m-1)(g-1) - 1$ differentials in the third case.
    The divisor of one of these elements is
        \begin{align*}
        \di (x^iy\omega) & =  \di(x^i\omega) + R -(g+1)D_\infty \\
        & =  iD_0 + R +((m-1)(g-1)-2-i)D_\infty,
        \end{align*}
    by Lemma \ref{lemmadivisorofycharacteristicnot2} and \eqref{eqbasedifferentialdivisor}, which is positive for $0\leq i \leq (m-1)(g-1)-2$.
    By Corollary \ref{dim3} we know that 
        \[
        \dim_kH^0(X,\Omega_X^{\otimes m}) = (2m-1)(g-1).
        \]
    Since the number of differentials listed in the last case of the proposition is precisely
        \[
        (m-1)(g-1)-1 + m(g-1) +1 = 2mg -2m -g + 1 = (2m-1)(g-1),
        \]
    it is clear that these elements form a basis.
    \end{proof}


\section{Characteristic 2}

In this section we assume that $\cha(k) = p  =2$.   
In this case the function field $K(X)$ is $k(x,y)$, a degree two extension of the function field of one variable over $k$, $k(x) = k(\PP_k^1)$, where
    \begin{equation}\label{ext}
    y^2 - H(x)y = F(x)
    \end{equation}
for some polynomials $H(x), F(x)\in k[x]$, such that $H(x)$ and $H'(x)F(x) + F'(x)^2$ have no common roots in $k$ \cite[Prop.\ 7.4.24]{liu}.
We have that $\deg(H(x)) \leq g+1$, with equality if and only if $\infty$ is not a branch point, and that $\deg(F(x)) \leq 2g+2$ with $\deg(F(x)) = 2g+1$ if $\infty$ is a branch point  \cite[Prop.\ 7.4.24]{liu}.

    \begin{lem}\label{smoothness}
    The affine plane curve $X'$ given by \eqref{ext} is smooth if and only if $H(x)$ and $H'(x)^2 F(x) + F'(x)^2$ have no common zeroes in $k$.
    \end{lem}
    \begin{proof}
    The Jacobian criterion (see, for example, \cite[Thm. 4.2.19]{liu}), states that if the derivatives of \eqref{ext} with respect to $x$ and with respect to $y$ are zero at a point $P\in X'$ then the curve is not smooth at $P$, and otherwise it is.
    Clearly
        \begin{equation}\label{eqsmoothnessone}
        \frac{d}{dy} (y^2 -H(x)y -F(x)) = H(x)
        \end{equation}
    since the characteristic of $k$ is 2.
    On the other hand,
        \begin{equation}\label{eqsmoothnesstwo}
        \frac{d}{dx} (y^2 - H(x)y -F(x)) = H'(x)y - F'(x).
        \end{equation}
    The affine plane curve given by \eqref{ext} is smooth at $P\in X'$ if and only if at least one of \eqref{eqsmoothnessone} and \eqref{eqsmoothnesstwo} is non-zero at $P$.
    Of course, \eqref{eqsmoothnesstwo} is zero if and only its square
        \begin{equation}\label{eqsmoothnessthree}
        (H'(x)y-F'(x))^2 = H'(x)^2y^2 -F'(x)^2 = H'(x)^2H(x)y + H'(x)^2F(x) - F'(x)^2
        \end{equation}
    is zero.
    Finally, if $H(a) = 0$ for some $a \in k$, then \eqref{eqsmoothnessthree} evaluated at $a$ is $H'(a)^2F(a) - F'(a)^2$.
    Hence the curve is smooth if and only if $H'(x)^2F(x) - F'(x)^2$ and $H(x)$ share no roots in $k$.
    \end{proof}

We first describe the ramified points of $\pi$, in order to compute the ramification divisor.
By Lemma \ref{smoothness} if we consider the affine curve defined by this equation it will be smooth.
We denote this curve by $X'$.
Then $\pi$ restricts to a map $X'\rightarrow \mathbb A^1_k$, the projection on to the $x$ co-ordinate.
Let $a\in \mathbb A_k^1$.
Then if $(a,b)$ is a point in $\pi^{-1}(a)$, so is the point $(a,b+H(a))$, which is clearly distinct if and only if $H(a)\neq 0$.
Since the extension is degree two, this shows that the ramified points in the affine part correspond to the roots of $H(x)$.
We let $k$ be the number of distinct roots that $H(x)$ has and then 
    \begin{equation}\label{equationcapitalh}
    H(x) = \prod_{i=1}^{k} (x-A_i)^{n_i} = x^{d_H} + B_{{d_H}-1}x^{{d_H}-1} + \ldots + B_1x + B_0
    \end{equation}
for some $A_i, B_i \in  k$ and $n_i \in \mathbb N$.
As above, the $A_i$ are branch points of $\pi$ and we let $P_i \in X$ be the corresponding ramification points, and $D_i = D_{P_i}$.
Note that for each $A_i$ there is a corresponding $K_i$, which is the square root of $F(A_i)$.

We now compute the ramification divisor of $\pi$.


    \begin{lem}\label{char2ramification}
    Let $n_i$ be the order of $H(x)$ at $a_i\in \mathbb A_k^1$.
    Then the coefficient $\delta_P$ of the ramification divisor $R$ at $P\in X$ is given by
        \[
        \delta_P = \left\{
            \begin{array}{ll}
            2n_i & \text{if }\ P\in \{P_1,\ldots ,P_k\}, \\
            2(g+1-d_H) & \text{ if }\  P \in \pi^{-1}(\infty), \\
            0 & \text{ otherwise.} 
            \end{array}
        \right.
        \]
    \end{lem}
    \begin{proof}
    We first show that it will suffice to prove that the coefficient of $[P_i]$ is $2n_i$ for $1\leq i \leq k$.
    Note that by the Riemann-Hurwitz formula $\deg(R) = 2g+2$.
    If $\infty$ is not a branch point of $\pi$ then $\delta_P = 0 = 2(g+1-d_H)$, as stated.
    If $\infty$ is a branch point then the coefficient at $P_\infty$ is $\deg(R) - \sum_{i=1}^k2n_i = 2g+2-2d_H = 2(g+1-d_H)$, again as stated.
    
    Let $P=P_i$ for some $i\in \{1,\ldots , k\}$.
    Then $y-b_i$ is a local parameter at $P$.
    To see this, note that the maximal ideal $\mathfrak m_{P}$ of the local ring $\cO_{X,P}$ at $P$ is generated by $x-a_i$ and $y-b_i$.
    But $x-a_i\in \mathfrak m_{P}^2$ since $\pi$ is ramified at $P$ with ramification index 2.
    By Nakayama's lemma \cite[Prop. 2.6]{atiyahmacdonald}, $y-b_i$ is therefore a local parameter at $P$.
    \todo[inline]{change notations to capital a and check definition of $b_i$}
    
    Using Hilbert's formula, Theorem \ref{hilbertsformula}, we obtain
        \begin{align*}
        \delta_P & =  \sum_{i\geq 0} \left(\ord(G_i(P))-1\right) \\
        & =  \max\left\{ i\in \NN | G_i(P)\neq \{1\}\right\} + 1 \\
        & =  \ord_{P}(\sigma(y-b_i) - (y-b_i)).
        \end{align*}
    
    From the defining equation \eqref{ext}, it is clear that the hyperelliptic involution is given by $\sigma(y)=y+H(x)$.
    The following calculation then concludes the proof,
        \begin{align*}
        \delta_P & =  \ord_{P}(\sigma(y-b_i) - (y-b_i)) \\
        & =  \ord_{P}(y-b_i+H(x) - y + b_i) \\
        & =  2\ord_{a_i}(H(x)) \\
        & =  2n_i.\qedhere
        \end{align*}
    \end{proof}


The divisors of $x$ and $dx$ are the same as when $\cha(k) \neq 2$, see \eqref{equationdivxpis2} and \eqref{equationdifferentialdivisor}.
We also note that since $\cha(k) = 2$ we have
    \[
    dF(x) = d(y^2 + yH(x)) = d(yH(x)) = H(x)dy + ydH(x)
    \]
    and hence
    \begin{equation}\label{equationdivisorofdypis2}
    dy = \frac{F'(x) + yH'(x)}{H(x)}dx
    \end{equation}
We now compute the divisor of $H(x)$ too.



    \begin{lem}\label{h(x)char2}
    The divisor associated to $H(x)$ is
        \[
        \di (H(x)) = \sum_{i=1}^k n_i D_i - d_HD_\infty = R - (g+1)D_\infty.
        \]
    \end{lem}
    \begin{proof}
    If $\pi$ is ramified at infinity then $\ord_{P_{\infty}}(H(x)) = - 2d_H$.
    If it is not ramified, then $\ord_{P_{\infty}}(H(x)) = \ord_{P_{\infty}'}(H(x))=-d_H= -(g+1)$.
    For the ramified points $P_i$, $1\leq i \leq k$, then $ \ord_{P_i}(H(x))= 2n_i$.
    At any other point of $X$ the order of $H(x)$ is clearly zero, and the first equality follows.

    The second equality follows from Lemma \ref{char2ramification}.
    \end{proof}
    
Finally, we describe the divisor of $y$.
In order to do this we need to distinguish the zeroes of $F(x)$.
Suppose that $F(x)$ has $l \leq \deg(F(x))$ distinct zeroes, and let $\gamma_1, \ldots, \gamma_l \in k \subseteq \mathbb P_k^1$ be these zeroes.
Then if $\gamma_i$ is a branch point let $Q_i = (\gamma_i, 0)$ be the unique point in the pre-image $\pi^{-1}(\gamma_i)$.
If $\gamma_i$ is not a branch point then let $Q_i = (\gamma_i, 0)$ and $Q_i' = (\gamma_i, H(\gamma_i))$ be the unique points that form the pre-image $\pi^{-1}(\gamma_i)$.
Also, we denote the order of the zero of $F(x)$ at $\gamma_i \in k$ by $m_i \in \NN$.


    \begin{prop}\label{propdivisorofypis2}
    If $\infty$ is a branch point, the divisor of $y$ is
        \begin{equation*}
        \di(y) = 
         {\displaystyle \sum_{i=1}^l} m_i[Q_i] -(2g+1)[P_\infty].
        \end{equation*}
    If $\infty$ is not a branch point then, after possibly swapping the notations for the two points $P_\infty$ and $P_\infty'$ in $\pi^{-1}(\infty)$, we have
        \begin{equation*}
        \di(y) = {\displaystyle \sum_{i=1}^l} m_i[Q_i] +(g+1-\deg(F(x)))[P_\infty] - (g+1)[P_\infty'].
        \end{equation*}
    \end{prop}
    \begin{proof}
    We first show that the divisor of $y$ on the affine part of $X$, $U_\infty := X \backslash {\pi^{-1}(\infty)}$, is $\sum_{i=1}^l m_i [Q_i]$.
    Suppose $P\in U_\infty$.
    If $\left. F \right|_P \neq 0$ then it follows that $y|_P \neq 0$, since $F(x) = y (y + H(x))$ (and similarly $y$ does not have a pole at $P$).
    Hence $\di(y)$ has a coefficient  of zero for any point in $U_\infty\backslash \{Q_1, \ldots, Q_l \}$.
    
    Suppose that $P= Q_i = (\gamma_i, 0)$ is an unramified point in $U_\infty$.
    Then $H(\gamma_i) \neq 0$ and $\left. y \right|_P = 0$, so $y + H(x)$ is a unit at $P$.
    Since $y(y+H(x)) = F(x)$ we find that
        \begin{equation*}
        \ord_P(y) = \ord_P\left( \frac{F(x)}{y + H(x)} \right) = \ord_P(F(x)) = m_i.
        \end{equation*}
    
    We now look at when $P = Q_i = (\gamma_i, 0)$ is a ramification point.
    Since $H(x)$ and $H'(x)^2F(x) + F'(x)^2$ cannot share roots it follows that $m_i = 1$.
    Hence the function $\tilde F(x) := (x- \gamma_i)^{-1}F(x)$ is a unit at $P$.
    We let $\tilde H(x) = (x- \gamma_i)^{-1}H(x)$.
    
    
    Now 
        \[
        y^2 = F(x) - y H(x) = (x- \gamma_i) \left(\tilde F(x) - y \tilde H(x)\right),
        \]
    and hence
        \[
        \ord_P(y^2 ) = \ord_P(x-\gamma_i) + \ord_P(\tilde F(x) - y \tilde H(x)).
        \]
    Since $\ord_P(x-\gamma_i) = 2$ and $\ord_P\left(\tilde F(x) - y \tilde H(x)\right) \geq 0$ we know that $\ord_P(y) \geq 1$.
    Hence $(y \tilde H(x)) \big|_P = 0$, and since $\tilde F(x)$ is a unit at $P$, we conclude that $\tilde F(x) - y \tilde H(x)$ is a unit at $P$.
    Hence $\ord_P(y^2) = 2$, and so $\ord_P(y) = 1 = m_i$.
    It follows that the divisor of $y$ restricted to $U_\infty$ is $\sum_{i=1}^l m_i [Q_i]$.
    
    We now consider the coefficients in $\di(y)$ of the points in $\pi^{-1}(\infty)$.
    If $\infty$ is a branch point then $\deg(F(x)) = 2g+1$ and hence $\sum_{i=1}^l m_i = 2g+1$.
    Since $y$ can only have a pole at $P_\infty$, we conclude that the order of this pole is $2g+1$, and hence
        \[
        \di(y) = \sum_{i=1}^l m_i [Q_i] - (2g+1)[P_\infty].
        \]
    
    If $\infty$ is not a branch point then there are two points at which $y$ may have a pole, namely $P_\infty$ and $P_\infty'$.
    The hyperelliptic involution $\sigma$ switches these two points.
    Furthermore, since $\sigma \colon y \mapsto y+H(x)$ it follows that $\ord_{P_\infty'}(y) = \ord_{P_ \infty}(y+H(x))$, a fact we use below.
    
    
    We now consider three cases, firstly supposing that $\ord_{P_\infty}(y) < -(g+1)$.
    Then $\ord_{P_\infty}(y) < \ord_{P_\infty}(H(x))$ and hence $ \ord_{P_\infty}(y) = \ord_{P_\infty}(y+H(x))$.
    But this contradicts $\ord_{P_\infty}(y) + \ord_{P_\infty}(y+H(x)) = \ord_{P_\infty}(F(x))$, since the left hand side is less than $-2(g+1)$, which is the minimum value of the right hand side.
    
    We now suppose that $\ord_{P_\infty} (y) = -(g+1)$. Since $y(y+H(x)) = F(x)$ it follows that $-(g+1) + \ord_{P_\infty}(y+H(x)) = \ord_{P_\infty}(F(x))$, and hence $\ord_{P_\infty'}(y) = \ord_{P_\infty}(y+H(x)) = -\deg(F(x)) + g + 1$.
    
    We now consider the case in which $\ord_{P_\infty}(y) > -(g+1)$.
    Then, since $\ord_{P_\infty}(H(x)) = -(g+1)$, it follows that $\ord_{P_\infty'}(y) = \ord_{P_\infty} (y+H(x)) = -(g+1)$.
    It now follows from a computation similar to that in the previous paragraph that $\ord_{P_\infty}(y) = -\deg(F(x)) + g +1$, completing the proof.
    \end{proof}

The following proposition determines a basis of the $k$ vector space of global holomorphic polydifferentials.
The case where $m=1$ can again be found in \cite[Prop. 7.4.26]{liu}.
    
    \begin{prop}\label{propbasishyperellipticp=2}
    We assume that $g\geq 2$ and let $\omega:= \frac{dx^{\otimes m}}{H(x)^m}$. 
    Then if $g\geq 2$, a basis of $H^0(X,\Omega_X^{\otimes m})$ is given by\par
        {\centering
        \begin{tabular}{c c}
        $\omega, x\omega, \ldots , x^{g-1}\omega$ &  if $m=1$, \\
        $\omega, x\omega, x^2\omega$ & if $m=g=2$, \\
        $\omega, x\omega, \ldots, x^{m(g-1)}\omega;\  y\omega, xy\omega, \ldots, x^{(m-1)(g-1)-2}y\omega$ & otherwise.
        \end{tabular}\par
        }
    \end{prop}

    \begin{proof}
    We first assume that above elements are holomorphic polydifferentials, and show that they then form a basis.
    To show that the elements are linearly independent over $k$ we need only show that the coefficients of $\omega$ are, since $\omega$ is fixed.
    The only case where this is not clear is when the coefficients contain both $x$ and $y$ terms.
    But since the $y$ terms are all linear, and the extension is of degree two, it must follow that coefficients are linearly independent.
    
    
    In the case that $m=1$ then we have that $\dim_k H^0(X,\Omega_X) =g$ by Definition \ref{definitiongenus}, and there are $g$ elements described in the statement of the proposition in this case, so they must form a basis.
    If $m \geq 2$ then $\dim_k H^0(X,\Omega_X^{\otimes m}) = (2m-1)(g-1)$.
    If $m=g=2$ then $(2m-1)(g-1) = 3$, and there are three elements listed in the proposition.
    On the other hand if $m\geq 2$ and $g > 2$ the proposition lists
        \[
        m(g-1)+1 + (g-1)(m-1)-2+1 = 2mg -2m -g +1 = (2m-1)(g-1)
        \]
    elements, and again they must form a basis.
    
    We now show that the listed polydifferentials are holomorphic, \ie that their divisors are non-negative.
    Firstly we have
        \begin{align*}
        & \di(x^i\omega)  =  \di \left( \frac{x^i dx^{\otimes m}}{H(x)^m} \right)\\ 
        & =  i(D_0 - D_\infty) +m(R-2D_\infty) - m(R - (g+1)D_\infty) \\
        & =  iD_0 + (m(g-1) -i)D_\infty
        \end{align*}
    by \eqref{equationdivxpis2}, \eqref{equationdifferentialdivisor} and Lemma \ref{h(x)char2}, and this is clearly non-negative for $0\leq i \leq m(g-1)$.
    
    Similarly, if $\infty$ is a branch point, we have
        \begin{align*}
        \di(x^iy\omega) & =  \di (x^i\omega) + \di(y) \\
        & =  iD_0 + (m(g-1) -i)D_\infty + \sum_{i=1}^l m_i[Q_i] - (2g+1)[P_\infty] \\
        & =  iD_0 +  \sum_{i=1}^l m_i[Q_i] + (2m(g-1) -2g-1 -2i)[P_\infty] \\
        & =  iD_0 +  \sum_{i=1}^l m_i[Q_i] + (2((m-1)(g-1) -1 -i)-1)[P_\infty],
        \end{align*}
    by \eqref{equationdivxpis2}, \eqref{equationdifferentialdivisor} and Lemma \ref{h(x)char2} and Proposition \ref{propdivisorofypis2}, which is again clearly non-negative for $0 \leq i \leq (g-1)(m-1)-2$.

    Finally, if $\infty$ is not a branch then, after possibly switching $P_\infty$ and $P_\infty'$, we have
        \begin{align*}
        \di(x^iy\omega) &  = \di (x^i\omega) + \di(y) \\
        & \hspace{-3em} = iD_0 + (m(g-1)-1)D_\infty + {\displaystyle \sum_{i=1}^l} m_i[Q_i] + (g+1-\deg(F(x)))[P_\infty] - (g+1)[P_\infty'] \\
        & \hspace{-2em} - mR + m(g+1)D_\infty \\
        & \hspace{-3em} =  iD_0 + {\displaystyle \sum_{i=1}^l} m_i[Q_i] + (mg - i - m - g - 1)[P_\infty'] + (mg - i - m + g + 1 - \deg(F(x)))[P_\infty] \\
        & \hspace{-3em} =  iD_0 + {\displaystyle \sum_{i=1}^l} m_i[Q_i] + ((m-1)(g-1) - 2 - i)[P_\infty'] + (mg - i - m + g + 1 - \deg(F(x)))[P_\infty],
        \end{align*}
    by Proposition \ref{propdivisorofypis2}, \eqref{equationdivxpis2}, \eqref{equationdifferentialdivisor} and Lemma \ref{h(x)char2}.
    Since $0 \leq i \leq (g-1)(m-1)- 2$ then the coefficient of $[P_\infty']$ is clearly non-negative.
    Finally, since $\deg(F(x)) \leq 2g+2$, the coefficient of $[P_\infty]$ is greater than or equal to that of $[P_\infty']$, and we conclude that the above divisor is non-negative, completing the proof.
    \end{proof}























\chapter{Group actions on algebraic de-Rham cohomology} \label{Chapter:De-Rham cohomology}

\todo[inline]{change references to preprint when in thesis}

\section{Serre duality for curves}


We begin by recalling some of the details of Serre duality.
We take $X$ to be a smooth, projective, connected curve over an algebraically closed field $k$ of characteristic $p \geq 0$.
Furthermore we let $K(X)$ denote the function field of $X$ and we define $\Omega_{K(X)}$ to be the module of differentials of $K(X)$ over $k$.
We also let $\underline{\Omega}_{K(X)}$ and $\underline{K}(X)$ denote the constant sheaves of $\Omega_{K(X)}$ and $K(X)$ respectively, and $\Omega_X$ denotes the sheaf of differentials on $X$.
The following lemma gives us a useful description of $H^1(X,\Omega_X)$.
    \begin{lem}\label{exactsequencelemma}
    We have canonical exact sequences as follows:
        \begin{equation}\label{dualitysesfunctions}
        0 \ra H^0(X,\cO_X) \ra K(X) \ra \bigoplus_{P \in X} K(X)/\cO_{X,P} \rightarrow \hone \ra  0;
        \end{equation}
        \begin{equation}\label{dualitysesdifferentials}
        0 \rightarrow \hzero \ra \Omega_{K(X)} \ra \bigoplus_{P \in X}\Omega_{K(X)}/\Omega_{X,P} \rightarrow H^1(X,\Omega_X) \ra 0.
        \end{equation}
    \end{lem}
    \begin{rem}
        Note that a sketch of the proof below can be found in \cite[pg. 248]{hart}.
        \end{rem}
    \begin{proof}
    The short exact sequence
        \begin{equation}\label{serredualitysesfunctions}
        0 \rightarrow \cO_X \rightarrow \underline{K}(X) \rightarrow \underline{K}(X)/\cO_X \rightarrow 0
        \end{equation}
    is a flasque resolution of $\cO_X$ (see \cite[Chap. II, ex. 1.16]{hart}).
    
    For each $P \in X$, which has a natural embedding $i\colon \{P\} \rightarrow X$, we view the module $K(X)/\cO_{X,P}$ as a sheaf on the singleton $\{P\}$.
    Hence for each $P\in X$ we have the induced sheaf $i_*\left( K(X)/\cO_{X,P} \right)$ on $X$.
    If we consider the direct sum of these induced sheaves over all points $P\in X$ we have the following isomorphism
        \begin{equation}\label{sheafisomorphism}
        \underline{K}(X)/\cO_X\cong \bigoplus_{P\in X} i_*\left(K(X)/\cO_{X,P}\right).
        \end{equation}
    
    
    To explain the isomorphism in \eqref{sheafisomorphism} we first construct a map from $\underline{K}(X)/\cO_{X,P}$ in to the product $\prod_{P \in X} i_*\left({K(X)}/\cO_{X,P}\right)$, and then show that this map has finite support.\todo{B: Not the whole map}
    
    Given $i\colon \{P\} \hookrightarrow X$ we have the following equalities
        \begin{align*}
        i^{-1}\left(\underline{K}(X)/\cO_X\right) & = \left(\underline{K}(X)/\cO_X\right)_P \\
        & = \underline{K(X)}_P/\cO_{X,P} \\
        & = K(X)/\cO_{X,P}.
        \end{align*}
    It follows that for any $P \in X$ we can map $f \in \underline{K}(X)/\cO_X$ to $i_*i^{-1}(f) \in i_* \left( K(X)/\cO_{X,P} \right)$.\todo{B: Notation (of $I_*i^{-1}(f)$}
    Recall that for any $f \in K(X)$then $f$ lies in $\cO_{X,P}$ for all but a finite number of points $P \in X$.
    Hence the image of $f$ in $\prod_{P \in X} i_*\left(K(X)/\cO_{X,P}\right)$ is zero in almost all factors.
    From this the isomorphism in \eqref{sheafisomorphism} follows.\todo{Add more details - mention adjoint}
    
    Replacing $\underline{K}(X)/\Omega_X$ by $\bigoplus_{P\in X} i_*\left(K(X)/\cO_{X,P}\right)$ in \eqref{serredualitysesfunctions} yields
        \begin{equation}\label{resolutionofox}
        0 \ra \cO_X \ra \underline{K}(X) \ra \bigoplus_{P \in X}i_* \left( K(X)/\cO_{X,P} \right)
        \end{equation}
    Taking cohomology then yields the exact sequence \eqref{dualitysesfunctions}.
    
    We can also tensor \eqref{resolutionofox} with $\Omega_X$ to get\todo{replace tensoring with repeating above argument again}
        \begin{equation}\label{serredualitydifferentials}
        0 \ra \Omega_X \ra \underline{\Omega}_{K(X)} \ra \bigoplus_{P \in X} i_* \left( \Omega_{K(X)}/\Omega_{X,P} \right).
        \end{equation}
    Taking cohomology of this then yields the second exact sequence \eqref{dualitysesdifferentials}.
    \end{proof}

    \begin{rem}
        When considering elements of $H^1(X,\Omega_X)$ as elements in the cokernel of the map $\Omega_{K(X)} \ra \bigoplus_{P \in X}\Omega_{K(X)}/\Omega_{X,P}$ above, we will denote them by $\overline{(\omega_P)}_{P \in X}$, where ${(\omega_P)}_{P \in X}\in \bigoplus_{P \in X}\Omega_{K(X)}/\Omega_{X,P}$.
        Similarly, when considering elements of $\hone$ as elements of the cokernel of the map $K(X) \ra \bigoplus_{P \in X}K(X)/\cO_{X,P}$, we will denote them by $\overline{(f_P)}_{P \in X}$, where ${(f_P)}_{P \in X} \in \bigoplus_{P \in X}K(X)/\cO_{X,P}$.
    \end{rem}


    \begin{thm}\label{theoremresiduemap}
    For any $P\in X$ there exists a unique $k$-linear map $\res_P \colon \Omega_{K(X)} \ra k$ defined by the following properties:
        \begin{itemize}
            \item $\res_P(\omega) = 0$ for all $\omega \in \Omega_{X,P}$;
            \item $\res_P(f^ndf) = 0$ for all $f \in K(X)^*$ and all $n \neq 1$;
            \item $\res_P(f^{-1}df) = \ord_P(f)$, where $\ord_P(f)$ is the order of $f$ at $P$.
        \end{itemize}
    \end{thm}
    \begin{proof}
    See \todo{find reference}
    \end{proof}

We define the {\em residue map at $P$} to be the map $\res_P$ described above.

    \begin{thm}[Residue Theorem]\label{theoremresiduetheorem}
    Given any differential $\omega \in \Omega_{K(X)}$ then $\sum_{P \in X}\res_P(\omega) = 0$.
    \end{thm}
    \begin{proof}
    See \todo{find citations, see Hartshorne to find them}
    \end{proof}

Since $\Omega_{X,P} \subseteq \ker (\res_P)$, it follows that $\res_P$ is a well defined function on the quotient $\Omega_{K(X)}/\Omega_{X,P}$.
By the residue theorem the map
    \begin{equation*} 
    \bigoplus_{P \in X} \Omega_{K(X)}/\Omega_{X,P} \ra k, \quad (\omega_P)_{P \in X} \mapsto \sum_{P\in X} \res_P(\omega_P)
    \end{equation*} 
vanishes on the image of $\Omega_{K(X)}$, which means that the following is well defined.
    
    \begin{defn}
    Let $\overline{(\omega_P)}_{P \in X} \in H^1(X,\Omega_X)$.
    Then we define the trace map $t \colon H^1\left(X, \Omega_X\right) \ra k$ by 
        \[
        t \colon \overline{(\omega_P)}_{P \in X}  \mapsto \sum_{P \in X} \res_P(\omega_P).
        \]
    \end{defn}

We now use the trace map to define a pairing between the $k$-vector spaces $\hone$ and $\hzero$.
Since $\Omega_{K(X)}$ is a $K(X)$-module, we can define a map 
    \begin{equation}\label{productmap}
    \hzero \times \hone \ra H^1\left(X, \Omega_X\right), \ (\omega, \overline{(f_P)}_{P \in X}) \mapsto ( \overline {(f  \omega)_P})_{P \in X},
    \end{equation}
where $(f\omega)_P$ denotes the product of the residue class of $\omega$ in $\Omega_{K(X)}/\Omega_{X,P}$ with $f_P \in K(X)$.

We now combine the product map in \eqref{productmap} with the trace map $t$ to get a map 
    \[
     \hzero \times \hone \ra k,\quad (\omega, \overline{(f_P)}_{P \in X}) \mapsto \langle \omega, \overline{(f_P)}_{P \in X} \rangle := t \left( \overline{(f \omega)_P} \right)_{P \in X}.
    \]

    \begin{thm}\label{serredualitytheorem}
    Via the pairing $\langle\ ,\ \rangle$, the $k$-vector spaces $\hone$ and $\hzero$ are dual to each other.
    \end{thm}
    \begin{proof}
    This is a specialisation of \cite[Chap. II, Thm. 2]{algebraicgroupsandclassfields}.
    \end{proof}

More explicitly, this theorem means the following.
If we fix any $\omega \in \hzero$ we produce a map $\theta(\omega)\colon \hone \ra k$, given by $\theta(\omega)(f) = \langle \omega , f\rangle$.
Similarly, if we fix any $f \in \hone$ then we get a map $\psi(f) \colon \hzero \ra k$.
Then the maps 
    \[
    \psi \colon \hone \ra \hom(H^0(X,\Omega_X),k) \quad
    \text{and} \quad    
    \theta \colon H^0(X,\Omega_X) \ra \hom(\hone, k)
    \]
    are isomorphisms and are dual to each other.
    In particular, given a basis $\omega_1, \ldots, \omega_n$ of $\hzero$, we can find a basis $f_1, \ldots , f_n$ of $\hone$ such that $\theta(\omega_i)(f_i) = 1$ for all $1 \leq i \leq n$ and $\theta(\omega_i)(f_j) = 0$ if $i \neq j$ (and similarly for $\psi$).


\section{\cech cohomology and de Rham cohomology for\\ hyperelliptic curves}

In addition to the assumptions in section 1, we now assume that $X$ is hyperelliptic of genus $g \geq 2$.\todo{Do we require the genus to be two?}
We recall that a curve is hyperelliptic if there exists a finite, separable morphism of degree two from the curve to $\mathbb P_k^1$.
We  fix a map $\pi \colon X \rightarrow \mathbb P_k^1$ of degree two, which is unique up to an automorphism of $\mathbb P_k^1$ (see \cite[Rem. 7.4.30]{liu}).
In this section we describe $\hone$ and $\hzero$ concretely for such an $X$, using \cech cohomology.

By Leray's theorem \cite[Thm. 5.2.12]{liu} and Serre's affineness criterion \cite[Thm. 5.2.23]{liu} we know that, if we use an affine cover, the first \cech cohomology group of $\cO_X$ will be isomorphic to $\hone$.
We define $U_a = X \backslash \pi^{-1}(a)$ for any $a \in \mathbb P_k^1$ and we let ${\cal U}$ be the affine cover of $X$ formed by $U_0$ and $U_\infty$.
Given any sheaf $\cal F$ on $X$ we have the \cech differential $\check{d}\colon {\cal F}(U_0) \times {\cal F} (U_\infty) \rightarrow {\cal F}(U_0 \cap U_\infty)$, defined by $(f_0,f_\infty) \mapsto f_0|_{U_0 \cap U_\infty} - f_\infty|_{U_0 \cap U_\infty}$.
In general we will suppress the notation denoting the restriction map.
Via this differential we have the following cochain complex
    \begin{equation*}
    0 \rightarrow \cO_X(U_0)\times \cO_X(U_\infty) \xrightarrow{\check{d}} \cO_X(U_0 \cap U_\infty) \rightarrow 0.
    \end{equation*}
The first cohomology group of this chain complex is $\cechhone = \frac{\cO_X(U_0 \cap U_\infty)}{\Ima(\check{d})}$ and hence
    \begin{equation}\label{cechhone}
    \hone \cong \frac{\cO_X(U_0 \cap U_\infty)}{\Ima(\check{d})}  
     = \frac{\cO_X(U_0 \cap U_\infty)}{\{f_0 - f_\infty | f_i \in \cO_X(U_i) \}}.
    \end{equation}

If we replace $\cO_X$ by $\Omega_X$ in the previous paragraph then everything still holds, and we conclude that
    \begin{equation}\label{equationcechcohomologyH1}
    H^1(X,\Omega_X) \cong \frac{\Omega_X(U_0 \cap U_\infty)}{\Ima(\check{d})} = \frac{\Omega_X(U_0 \cap U_\infty)}{\{\omega_0 - \omega_ \infty | \omega_i \in \Omega_X(U_i)\}}.
    \end{equation}

\todo[inline]{Change residue classes to square brackets}
We now describe how the trace map acts on $H^1(X,\Omega_X)$ via the presentation \eqref{equationcechcohomologyH1}.
    \begin{lem}\label{lemmatracemaplemma}
    Let $ \omega \in \Omega_X(U_0 \cap U_\infty)$ and let $[\omega]$ denote its residue class in $\hone$ via the isomorphism \eqref{equationcechcohomologyH1}.
    Then we have
        \[
        t(\bar \omega) = \sum_{P \in \pi^{-1}(\infty)}\res_P(\omega).
        \]
    Note that when computing $\res_P(\omega)$, we consider $\omega$ as an element of $\Omega_{K(X)}$ via the canonical injection $\Omega_X(U_0 \cap U_\infty) \hookrightarrow \Omega_{K(X)}$.
    \end{lem}
    \begin{proof}
    We take the \cech complex of \eqref{serredualitydifferentials} over the cover $\cU$, yielding the following bicomplex
        \begin{equation}\label{dualitydiagram2}
        \xymatrix{\Omega_X(U_0)\times\Omega_X(U_\infty) \ar@{^{(}->}[r] \ar[d]^{d_1} & \Omega_{K(X)} \times \Omega_{K(X)} \ar[d]^{d_2} \ar@{->>}[r] & \bigoplus \limits_{P \in U_0} \Omega_{K(X)}/\Omega_{X,P} \times \bigoplus \limits_{P \in U_\infty} \Omega_{K(X)}/\Omega_{X,P} \ar[d]^{d_3} \\
        \Omega_X(U_0 \cap U_\infty) \ar@{^{(}->}[r]  & \Omega_{K(X)} \ar@{->>}[r] & \bigoplus \limits_{P\in U_0 \cap U_\infty} \Omega_{K(X)}/\Omega_{X,P} }
        \end{equation}
    We can now apply the snake lemma to this diagram, giving a long exact sequence.
    In fact, this sequence is isomorphic to the sequence \eqref{dualitysesdifferentials} found in Lemma \ref{exactsequencelemma}.
    However, we will only exhibit isomorphisms between $\ker(d_3)$ and $\operatorname{coker}(d_1)$ and, respectively, the third and fourth terms of \eqref{dualitysesdifferentials}, since this is all we need to examine the trace map.
    The fact that $H^1(X,\Omega_X) \cong {\rm coker}(d_1)$ follows from the above discussion of \cech cohomology.
    To show the isomorphism $\ker(d_3) \cong \bigoplus_{P \in X} \Omega_{K(X)}/\Omega_{X,P}$ we first recall that $d_3$ is defined by $((\omega_P)_{P \in U_0}, (\omega_P')_{P \in U_\infty}) \mapsto \omega_P - \omega_P')_{P_ \in U_0 \cap U_\infty}$.
    Then the kernel of $d_3$ is formed of pairs $ ((\omega_P)_{P \in U_0}, (\omega_P')_{P \in U_\infty})\in \left( \bigoplus_{P \in U_0} \Omega_{K(X)}/\Omega_{X,P} \right) \times \left( \bigoplus_{P \in  U_\infty} \Omega_{K(X)}/\Omega_{X,P} \right)$ such that $\omega_P = \omega_P'$ for $ P \in U_0 \cap U_\infty$.
    It follows that the map 
        \[
        \bigoplus_{P \in X} \Omega_{K(X)}/\Omega_{X,P}\ra\ker(d_3), \quad  (\omega_P)_{P \in X} \to \left( (\omega_P)_{ P \in U_0}, (\omega_P)_{P \in U_\infty}) \right)
        \]
    is an isomorphism.
    
    The proof now follows from a diagram chase on \eqref{dualitydiagram2}.
     We start with the residue class $\bar \omega \in \hone$ with representative $\omega \in \Omega_X(U_0 \cap U_\infty)$.
    This then injects in to $\Omega_{K(X)}$, and since $d_2$ is surjective we can choose an element of $\Omega_{K(X)} \times \Omega_{K(X)}$ mapping to $\omega$.
    In particular, we choose $(\omega,0)$.
    This then maps to 
        \[
        \psi = ((\bar{\omega}_P)_{P\in U_0}, 0) \in \left( \bigoplus_{P \in U_0} \Omega_{K(X)}/\Omega_{X,P}\right) \times \left( \bigoplus_{P \in U_\infty} \Omega_{K(X)}/\Omega_{X,P} \right).
        \]
    By commutativity of the diagram $\psi \in \ker(d_3)$.
    In particular, this means that $\bar \omega_P$, and hence also $\psi$, is zero for any $P \in U_0 \cap U_\infty$.
    Since $\psi$ is also zero for $P \in U_0$ it follows that 
        \[
        t(\bar \omega) = \sum_{P \in X}\res_P(\psi) = \sum_{P \in \pi^{-1}(\infty)} \res_P(\omega).
        \]
    \end{proof}

We now recall how to compute the algebraic de Rham cohomology of $X$ via \cech cohomology.
Since $X$ is a curve any differentials of degree greater than one on $X$ are zero.
Hence the de Rham complex of $X$ is 
    \begin{equation}\label{res}
    0 \rightarrow \cO_X \xrightarrow{d} \Omega_X \rightarrow 0.
    \end{equation}
Here $d$ denotes the differential map $f \mapsto df$, as defined in \cite[Chap. II, pg. 172]{hart}.
We then recall from \cite[Pg. 351]{grothendiecklettertoatiyah} that the de Rham cohomology of $X$ is precisely the hypercohomology of \eqref{res}.

We use the cover $\cal U$ and the \cech differentials defined earlier to give us the \cech bicomplex of \eqref{res}, which is
    \begin{equation}\label{bicomplex}
    \xymatrix{ & 0 \ar[d] & 0 \ar[d] & \\
    0 \ar[r] & \cO_X(U_0) \times \cO_X(U_\infty) \ar[d] \ar[r] & \Omega_X(U_0) \times \Omega_X(U_\infty) \ar[d] \ar[r] & 0 \\
    0 \ar[r] & \cO_X(U_0\cap U_\infty) \ar[d] \ar[r] & \Omega_X(U_0 \cap U_\infty) \ar[r] \ar[d] & 0 \\
    & 0 & 0 &}
    \end{equation}
By a generalisation of Leray's theorem \cite[Cor. 12.4.7]{EGA0III} we know that the $\derhamhone$ is isomorphic to the first cohomology of the total complex of \eqref{bicomplex}.
Note that this requires ${\check H}^p(U_\sigma, \cO_X)$ and ${\check H}^p(U_\sigma, \Omega_X)$ to be zero for any $\sigma$ in the nerve of $\cU$ and any $p \geq 1$ ---
since $U_0$ and $U_\infty$ are affine, this follows from Serre's affineness criterion \cite[Thm. 5.2.23]{liu}.


Therefore $\derhamhone$ is isomorphic to the space
    \begin{equation}\label{derhamconditions}
    \left\{(\omega_0, \omega_\infty, f_{0,\infty}) | \omega_i\in \Omega_X(U_i), f_{0,\infty}\in \cO_X(U_0 \cap U_\infty), df_{0,\infty} = \omega_0|_{U_0\cap U_\infty} - \omega_\infty|_{U_0\cap U_\infty} \right\}
    \end{equation}
quotiented by the subspace
    \begin{equation}\label{quotient}
    \left\{  (df_0, df_\infty, f_0|_{U_0\cap U_\infty} -f_\infty|_{U_0\cap U_\infty} )|f_i \in \cO_X(U_i)\right\}.
    \end{equation}

Via the isomorphism \eqref{cechhone}, as well as the isomorphism above, we can define the maps
    \begin{equation}\label{equationinjectionofhzerointoderham} 
    i \colon \hzero \ra \derhamhone, \qquad \omega \mapsto (\omega, \omega, 0)
    \end{equation}
    and 
    \begin{equation}\label{surjectionofderhamontohone}
    p \colon \derhamhone \ra \hone, \qquad (\omega_0, \omega_\infty, f_{0 \infty}) \mapsto f_{0 \infty}.
    \end{equation}
The following lemma shows that $\derhamhone$ fits in to a short exact sequence.
    
    \begin{prop}\label{ses}
    We have a canonical short exact sequence as follows:
        \begin{equation}\label{equationses}
        0 \ra H^0(X,\Omega_X) \xrightarrow{i} \derhamhone \xrightarrow{p} H^1(X,\cO_X) \ra 0.
        \end{equation}
    \end{prop}
    \begin{proof}
    Let $T$ be the total complex of \eqref{bicomplex}.
    Moreover, we let $\cO$ and $\Omega$ be the complexes formed from the first and second (non-trivial) columns of \eqref{bicomplex} respectively.
    Then let $\Omega[1]$ denote the complex obtained from shifting $\Omega$ by one, \ie~$\Omega[1]^{n+1} = \Omega^n$.
    From this we obtain the following short exact sequence of complexes 
        \[
        \Omega[1] \hookrightarrow T \twoheadrightarrow \cO,
        \]
    giving rise to the following long exact sequence
        \begin{align} \label{longexactsequence}
        \begin{split}
        0 \ra & H^0_{\text {dR}}(X/k) \ra   H^0(X,\cO_X) \ra \\ 
        H^0(X,\Omega_X) \ra & \derhamhone \ra   \hone \ra  \\
        H^1(X,\Omega_X) \ra & H^2_{\text {dR}}(X/k) \ra   0, 
        \end{split}
        \end{align} 
    where the maps in the middle line are the maps $i$ \eqref{equationinjectionofhzerointoderham} and $p$ \eqref{surjectionofderhamontohone}.


    The map $H^0(X,\cO_X) \ra \hzero$ is the map $f \mapsto df$.
    Since the only globally holomorphic functions on $X$ are constant functions, it follows that this is the zero map, and hence $\hzero \ra \derhamhone$ is injective.
    
    Since \eqref{longexactsequence} is exact, $p$ is surjective if and only if $\alpha \colon H^1(X,\Omega_X) \ra H^2_{\text {dR}}(X/k)$ is injective.
    Now $H^1(X,\Omega_X)$ is isomorphic to $k$ via the residue map \cite[Chap. III, Thm. 7.14.1]{hart}, and if we can show that this isomorphism factors through $\alpha$ it will follow that $\alpha$ is injective.
    Considering the \cech cohomology constructions of $H^1(X,\Omega_X)$ and $H^2_{\text {dR}}(X/k)$, it suffices to show that the trace map is zero on $\Ima \left( d \colon \cO_X(U_0 \cap U_\infty) \ra \Omega_X(U_0 \cap U_\infty) \right)$.
    This follows from Theorem \ref{theoremresiduemap}, which says that given any $f \in K(X)$ then $\res_P(df)=0$ for any $P \in X$, and in particular for any $P \in \pi^{-1}(\infty)$.
    Hence $t(\bar(df)) = 0$ by Lemma \ref{lemmatracemaplemma}, and so the residue isomorphism factors through $\alpha$, and $p$ is surjective.
    \end{proof}

\section{Bases of $\hone$ and $\derhamhone$}

We keep the assumptions of the previous section.
In this section we will give an explicit basis of $\hone$ and $\derhamhone$, using the results of the previous sections along with methods similar to those found in \cite{canonicalrepresentation}.

Since $X$ is a hyperelliptic curve, it follows that if $p \neq 2$ the extension $K(X)$ of $K(\mathbb P_k^1) = k(x)$, corresponding to $ \pi \colon X \ra \mathbb P_k^1$, will be $k(x,y)$ where $y$ satisfies
    \begin{equation}\label{definingequationpnot2}
    y^2 = f(x)
    \end{equation}
for some polynomial $f(x) \in k[x]$ which has no repeated roots and is of degree $2g+1$ or $2g+2$ \cite[Prop. 7.4.24]{liu}.
Moreover, we can and will assume, by applying an automorphism of $\mathbb P_k^1$ if necessary, that $f(x)$ is monic.

On the other hand, if $p=2$, then the extension $K(X)$ of $k(x)$ will be $k(x,y)$, this time with $y$ satisfying the equation
    \begin{equation}\label{definep=2}
    y^2 - H(x)y = F(x)
    \end{equation}
for some $H(x),F(x) \in k[x]$, such that $H(x)$ and $H'(x)^2F(x) + F'(x)^2$ share no roots.
We require that $\deg(H(x)) \leq g+1$, with equality if and only if $\infty$ is not a branch point, and that $\deg(F(x)) \leq 2g+2$ with $\deg(F(x)) = 2g+1$ if $\infty$ is a branch point  \cite[Prop. 7.4.24]{liu}.
Finally, we can assume, again applying an automorphism of $\mathbb P_k^1$, that $H(x)$ is monic.

We recall that the assumptions made about the polynomials, $f(x)$ and $F(x)$ and $H(x)$ mean that the affine plane curve defined by \eqref{definingequationpnot2} and \eqref{definep=2}, respectively, are smooth.
We now recall the details of some divisors discussed in \cite{faithfulaction}, which we will need later.
Henceforth we let $P_a$ and $P_a'$ denote the unique elements of $\pi^{-1}(a)$ for any point $a \in \mathbb P_k^1$ that is not a branch point.
If $a \in \mathbb P_k^1$ is a branch point we denote the unique point in $\pi^{-1}(a)$ by $P_a$.
We also define $D_a$ to be the divisor $\pi^*\left([a]\right)$ for any $a \in \mathbb P_k^1$, and hence
    \begin{equation*}
    D_a= 
        \begin{cases}
         2[P_a] & \text{if $a$ is a branch point}, \\
         [P_a] + [P_a'] & \text{otherwise.}
        \end{cases}
    \end{equation*}

If $p \neq 2$ then the branch points of $\pi$ are the zeroes of $f(x)$, and also $\infty$ if $2 \nmid \deg(f(x))$.
We suppose that 
    \[
    f(x) = \prod_{i=1}^{d_f} (x-a_i) = x^{d_f} + b_{d_f - 1}x^{d_f-1} + \ldots + b_0,
    \]
for some $a_i, b_i \in k$ and with $d_f := \deg(f(x)) $.
In this case, the $a_i \in \mathbb A_k^1 \subset \mathbb P_k^1$ are branch points of $\pi$, and we let $P_i = P_{a_i}$ denote the corresponding ramification points.
When $d_f = 2g+1$ then $\infty \in \mathbb P_k^1$ is a branch point as well and we define $P_{2g+2} := P_\infty$.
Then the ramification divisor $R$ of $\pi$ is
    \[
    R = \sum_{i=1}^{2g+2} [P_i] .
    \]

If $p=2$ then we suppose that
\begin{equation}\label{capitalh}
H(x) = \prod_{i=1}^{d_H} (x-A_i)^{n_i} = x^{d_H} + B_{{d_H}-1}x^{{d_H}-1} + \ldots + B_1x + B_0
\end{equation}
for some $A_i, B_i \in  k$, $n_i \in \mathbb N$ and with  ${d_H} := \deg(H(x)) \leq g+1$.
Then, as above, the $A_i$ are the branch points of $\pi$ and we let $P_i \in X$ be the corresponding ramification points.
Given this, we can write the ramification divisor as
    \[
    R = \sum_{i=1}^{d_H} 2n_i[P_i] + (g+1-{d_H})D_\infty.
    \]


We also have
    \begin{equation}\label{divxp=2}
    \di (x)  = D_0 - D_\infty
    \end{equation}
and
    \begin{equation}\label{differentialdivisor}
    \di(dx) = R - 2D_\infty,
    \end{equation}
regardless of characteristic.

If $p \neq 2$ then 
    \begin{equation}\label{pnot2divisors}
    \di(y)  = R - (g+1)D_\infty,
    \end{equation}
whilst if $p=2$ then the divisor of $H(x)$ is
    \begin{equation}\label{divisorofH}
    \di (H(x))  = R - (g+1)D_\infty. 
    \end{equation}\todo{reference in other part of thesis}



We also compute the divisor of $dy$ when $p=2$.
We start by noting that when we take the differential of \eqref{definep=2} we obtain
    \[
    dF(x) = d\left(y^2 + yH(x) \right) = d(yH(x)) = H(x)dy + ydH(x)
    \]
and from this it follows that
    \begin{equation}\label{divdyp=2}
    dy = \frac{F(x)'+yH(x)'}{H(x)}dx.
    \end{equation}

Finally, we describe the divisor of $y$ when $p=2$.
In order to do this we need to distinguish the zeroes of $F(x)$.
Suppose that $F(x)$ has $l \leq \deg(F(x))$ distinct zeroes, and let $\gamma_1, \ldots, \gamma_l \in k \subseteq \mathbb P_k^1$ be these zeroes.
Then if $\gamma_i$ is a branch point let $Q_i = (\gamma_i, 0)$ be the unique point in the pre-image $\pi^{-1}(\gamma_i)$.
If $\gamma_i$ is not a branch point then let $Q_i = (\gamma_i, 0)$ and $Q_i' = (\gamma_i, H(\gamma_i))$ be the unique points that form the pre-image $\pi^{-1}(\gamma_i)$.
Also, we denote the order of the zero of $F(x)$ at $\gamma_i \in k$ by $m_i \in \NN$.


    \begin{prop}\label{divyp=2}
    Suppose that $p=2$.
    Then, if $\infty$ is a branch point, the divisor of $y$ is
        \begin{equation*}
        \di(y) = 
         {\displaystyle \sum_{i=1}^l} m_i[Q_i] -(2g+1)[P_\infty].
        \end{equation*}
    If $\infty$ is not a branch point, then, after possibly swapping the notations for the two points $P_\infty$ and $P_\infty'$ in $\pi^{-1}(\infty)$, we have
        \begin{equation*}
        \di(y) = {\displaystyle \sum_{i=1}^l} m_i[Q_i] +(g+1-\deg(F(x)))[P_\infty] - (g+1)[P_\infty'].
        \end{equation*}
    \end{prop}
    \begin{proof}
    We first show that $\di_0(y)$, the divisor of the zeroes of $y$, is $\sum_{i=1}^l m_i [Q_i]$.
    
    It is clear that the zeroes of $y$ can only occur in the affine part of the curve $X$ defined by \eqref{definep=2} \ie~in $U_\infty$.
    Suppose $P\in U_\infty$.
    If $\left. F \right|_P \neq 0$ then it follows that $y|_P \neq 0$, since $F(x) = y (y + H(x))$.
    Hence $\di_0(y)$ has zero coefficients for any point in $U_\infty\backslash \{Q_1, \ldots, Q_l \}$.
    
    Suppose that $P= Q_i = (\gamma_i, 0)$ is an unramified point in $U_\infty$.
    Then $H(\gamma_i) \neq 0$ and $\left. y \right|_P = 0$, so $y + H(x)$ is a unit at $P$.
    Since $y(y+H(x)) = F(x)$ we find that
        \begin{equation*}
        \ord_P(y) = \ord_P\left( \frac{F(x)}{y + H(x)} \right) = \ord_P(F(x)) = m_i.
        \end{equation*}
    
    We now look at when $P = Q_i = (\gamma_i, 0)$ is a ramification point.
    Since $H(x)$ and $H'(x)^2F(x) + F'(x)^2$ cannot share roots it follows that $m_i = 1$.
    Hence the function $\tilde F(x) := (x- \gamma_i)^{-1}F(x)$ is a unit at $P$.
    We let $\tilde H(x) = (x- \gamma_i)^{-1}H(x)$.
    
    
    Now 
        \[
        y^2 = F(x) - y H(x) = (x- \gamma_i) \left(\tilde F(x) - y \tilde H(x)\right),
        \]
    and hence
        \[
        \ord_P(y^2 ) = \ord_P(x-\gamma_i) + \ord_P(\tilde F(x) - y \tilde H(x)).
        \]
    Since $\ord_P(x-\gamma_i) = 2$ and $\ord_P\left(\tilde F(x) - y \tilde H(x)\right) \geq 0$ we know that $\ord_P(y) \geq 1$.
    Hence $(y \tilde H(x)) \big|_P = 0$, and since $\tilde F(x)$ is a unit at $P$, we conclude that $\tilde F(x) - y \tilde H(x)$ is a unit at $P$.
    Hence $\ord_P(y^2) = 2$, and so $\ord_P(y) = 1 = m_i$.
    It follows that $\di_0(y) =  \sum_{i=1}^l m_i [Q_i]$.
    
    We now consider the poles of $y$.
    If $\infty$ is a branch point then $\deg(F(x)) = 2g+1$ and hence $\sum_{i=1}^l m_i = 2g+1$.
    Since $y$ can only have a pole at $P_\infty$, we conclude that the degree of this pole is $2g+1$, and hence
        \[
        \di(y) = \sum_{i=1}^l m_i [Q_i] - (2g+1)[P_\infty].
        \]
    
    If $\infty$ is not a branch point then there are two points at which $y$ may have a pole, namely $P_\infty$ and $P_\infty'$.
    We consider three cases, noting that $\ord_{P_\infty}(y + H(x)) = \ord_{P_\infty'}(y)$, since $y \mapsto y+H(x)$ is an automorphism.
    
    
    Firstly, we suppose that $\ord_{P_\infty}(y) < -(g+1)$.
    Then $\ord_{P_\infty}(y) < \ord_{P_\infty}(H(x))$ and hence $ \ord_{P_\infty}(y) = \ord_{P_\infty}(y+H(x))$.
    But this contradicts $\ord_{P_\infty}(y) + \ord_{P_\infty}(y+H(x)) = \ord_{P_\infty}(F(x))$, since the left hand side is less than $-2(g+1)$, which is the minimum value of the right hand side.
    
    We now suppose that $\ord_{P_\infty} (y) = -(g+1)$. Since $y(y+H(x)) = F(x)$ it follows that $-(g+1) + \ord_{P_\infty}(y+H(x)) = \ord_{P_\infty}(F(x))$, and hence $\ord_{P_\infty'}(y) = \ord_{P_\infty}(y+H(x)) = -\deg(F(x)) + g + 1$.
    
    Finally, if $\ord_{P_\infty}(y) > -(g+1)$, then since $\ord_{P_\infty}(H(x)) = -(g+1)$ it follows that $\ord_{P_\infty'}(y) = \ord_{P_\infty} (y+H(x)) = -(g+1)$.
    Then, from a computation similar to that in the previous paragraph we see that $\ord_{P_\infty}(y) = -\deg(F(x)) + g +1$, completing the proof.
    \end{proof}


    \begin{thm}\label{basish1}
     Via the isomorphism \eqref{cechhone} the residue classes of $\frac{y}{x}, \ldots , \frac{y}{x^g} \in K(X)$, restricted to $U_0 \cap U_\infty$, form a basis of $H^1(X,\cO_X)$.
    \end{thm}
    \begin{proof}
    We start by considering the case $p \neq 2$ and first check that the functions $\frac{y}{x}, \ldots, \frac{y}{x^g}$ are indeed regular on $U_0 \cap U_\infty$ (as required by \eqref{cechhone}) by computing their divisors.
    From \eqref{divxp=2} and \eqref{pnot2divisors} we see that
        \begin{align}\label{divisorofyoverx}
        \di \left( \frac{y}{x^i} \right) & = \di (y) - \di ( x^i) \nonumber \\
        & = R - (g+1)D_\infty - iD_0 + iD_\infty \nonumber \\
        & = R - iD_0 - (g+1 - i)D_\infty.
        \end{align}
    Since $R$ is a positive divisor this is non-negative on $U_0 \cap U_\infty$ if $i\in \{0, \ldots, g-1\}$.
    
    
    Recall that the differentials $\omega_0 = y^{-1}dx, \ldots, \omega_{g-1} = x^{g-1}y^{-1}dx$ form a basis of $\hzero$ (see, for example, \cite[Prop. 7.4.26]{liu}).
    By Lemma \ref{tracemaplemma} we know that $\langle x^iy^{-1}dx, yx^{-j} \rangle = \sum_{P \in \pi^{-1}(\infty)}\res_P(x^{i-j}dx)$.
    It follows immediately from Theorem \ref{theoremresiduemap} that $\sum_{P \in \pi^{-1}(\infty)}\res_P(x^{i-j}dx) = -2$ if $i-j=-1$ and is zero otherwise (regardless of whether $\infty$ is a branch point).
    It then follows from Theorem \ref{serredualitytheorem} that the residue classes of $ yx^{-1}|_{U_0\cap U_\infty},\ldots,yx^{-g}|_{U_0\cap U_\infty}$ form a basis of $\hone$.
    
    
    
    We now suppose that $p=2$, and again start by checking that for $i \in \{1, \ldots , g\}$ the function $yx^{-i}$ is regular on $U_0 \cap U_\infty$.
    This follows once we compute the divisor of $yx^{-i}$, which is
        \begin{align*}
        \di \left( \frac{y}{x^i} \right)  & =  \di(y) - i\di(x) \\
        & = {\displaystyle \sum_{i=1}^l} m_i[Q_i] -iD_0 -(2g+1 - 2i)[P_\infty]
        \end{align*}
    if $\infty$ is a branch point and
        \begin{align*}
        \di \left( \frac{y}{x^i} \right)  & =  \di(y) - i\di(x) \\  
        & = {\displaystyle \sum_{i=1}^l} m_i[Q_i] - iD_0 +(g+1-\deg(F(x)) + i)[P_\infty] - (g+1-i)[P_\infty']
        \end{align*}
    otherwise.
    These equalities follow from Proposition \ref{divyp=2} and \eqref{divxp=2}.
    The divisors are clearly positive on $U_0 \cap U_\infty$.
    
    Next we recall from \cite[Prop. 7.4.26]{liu} that if $p=2$ a basis of $\hzero$ is given by $\frac{1}{H(x)}dx, \ldots, \frac{x^{g-1}}{H(x)}dx$.\todo{change reference in thesis}
    We then deduce from Lemma \ref{lemmatracemaplemma} that when $\infty$ is not a branch point
        \[
        \left \langle \frac{x^i}{H(x)}dx, \frac{y}{x^j} \right \rangle = \res_{P_\infty} \left( \frac{yx^{i-j}}{H(x)}dx \right) + \res_{P_\infty'}\left( \frac{yx^{i-j}}{H(x)} dx \right).
        \]
    Then recall that in characteristic two we have an involution $\sigma \colon X \ra X$ given by $(x,y) \mapsto (x, y + H(x))$, and that $\res_P(\sigma^*(\omega)) = \res_{\sigma(P)}(\omega)$ for any $P \in X$ and $\omega\in \hzero$.
    Then it follows that
        \begin{align*}
        \left \langle \frac{x^i}{H(x)}dx, \frac{y}{x^j} \right \rangle & = \res_{P_ \infty} \left( \frac{yx^{i-j}}{H(x)}dx \right) + \res_{P_ \infty} \left( \frac{(y+H(x))x^{i-j}}{H(x)}dx \right) \\
        & = 2\res_{P_\infty}\left( \frac{yx^{i-j}}{H(x)}dx \right) + \res_{P \infty}(x^{i-j}dx) \\
        & = \res_{P_\infty}(x^{i-j}dx),
        \end{align*}\todo{check this makes sense}
    since we are assuming that $\cha(k) = 2$.
    As in the previous case, it follows from the definition of $\res_P$ that $\res_{P_\infty}(x^{i-j}dx) = -1$ if $i-j = -1$ and is zero otherwise.
    Hence it follows from Theorem \ref{serredualitytheorem} that the residue classes of $\frac{y}{x}|_{U_0 \cap U_\infty}, \ldots, \frac{y}{x^g}|_{U_0 \cap U_\infty}$ form a basis of $\hone$.
    
    
    
    If $P_\infty$ is a branch point then we compute the divisor of $ \frac{y}{x^j} \cdot \frac{x^i}{H(x)}dx$, using \eqref{divxp=2}, \eqref{differentialdivisor}, \eqref{divisorofH} and Proposition \ref{divyp=2}:
        \begin{align*}
        \di\left( \frac{yx^{i-j}}{H(x)}dx \right) & = \di(y) + \di(x^{i-j}) + \di( dx) - \di(H(x)) \\
        & = \sum_{i=1}^l m_i[Q_i] - (2g+ 1 )[P_\infty] + (i-j)D_0 - (i-j)D_\infty + R - 2D_\infty \\
        & \qquad - R + (g+1)D_\infty\\
        & = \sum_{i=1}^l m_i[Q_i] + (2j-3-2i)[P_\infty] + (i-j)D_0.
        \end{align*}
    We see that there is a pole of order one at $P_\infty$ if $2j - 3 - 2i = -1$, or equivalently if $j = i+1$.
    Hence $\left\langle \frac{x^i}{H(x)}dx, \frac{y}{x^j} \right\rangle = \res_{P_ \infty}\left( \frac{yx^{i-j}}{H(x)}dx\right)  \neq 0$ in this case.\todo{what is the residue?}
    
    We also check that if $j \neq i+1$ then $\left \langle \frac{x^i}{H(x)}dx, \frac{y}{x^j} \right \rangle = 0$.
    Indeed, if $j-i \geq 2$ then clearly $\frac{yx^{i-j}}{H(x)}dx$ does not have a pole at $P_\infty$.
    On the other hand, if $j-i \leq 0$ then the differential $\frac{yx^{i-j}}{H(x)}dx$ is regular on $U_\infty$, and hence the residue on this set is zero.
    Since $X \backslash U_\infty = \{P_\infty\}$ it follows from the Theorem \ref{theoremresiduetheorem} the residue of $\frac{yx^{i-j}}{H(x)}dx$ at $P_\infty$ is also zero, and hence the residue classes of the elements $\frac{y}{x}|_{U_0 \cap U_\infty}, \ldots, \frac{y}{x^g}|_{U_0 \cap U_g}$ form a basis of $\hone$.\todo{rephrase this bit due to repetitiveness}
    \end{proof}

%Mittag-Leffler style corollary

We now give a corollary to Theorem \ref{basish1}, which is of the same style as the Mittag-Leffler theorem, but for hyperelliptic curves.

    \begin{cor}
    For each $P \in X$ we fix $f_P \in K(X)/\cO_{X,P}$, such that $f_P = 0$ for almost all $P \in X$.
    Then there exist unique $\alpha_1, \ldots, \alpha_g \in k$ such that, after replacing $f_P$ by $f_P - \left( \frac{\alpha_1 x}{y} + \ldots + \frac{\alpha_g x^g}{y}\right)$ for $P \in \pi^{-1}(\infty)$, we can find an $f \in K(X)$ which has a Laurent tail of $f_P$ at $P$ for all $P \in X$.
    \end{cor}
    \begin{proof}
    Since $f_P = 0$ for almost all $P \in X$ then $(f_P)_{P \in X} \in \bigoplus_{P \in X} K(X)/\cO_{X,P}$.
    From Lemma \ref{exactsequencelemma} we have the following exact sequence
        \begin{equation*}
        0 \ra H^0(X,\cO_X) \ra K(X) \ra \bigoplus_{P \in X} K(X)/\cO_{X,P} \rightarrow \hone \ra 0.
        \end{equation*}\todo{introduce final map as $\delta$}
    By Theorem \ref{basish1} the residue classes $ \gamma_ 1= \left[ \frac{x}{y}\right], \ldots, \gamma_n = \left[\frac{x^g}{y}\right]$ form a basis of $\hone$, it follows that there exist unique $\alpha_1, \ldots, \alpha_g$ such that
        \[
        \delta\left( (f_P)_{P \in X} \right) - \left( \alpha_1\gamma_1 + \ldots + \alpha_n\gamma_n \right) = 0.
        \]
    We can derive the exact sequence \eqref{dualitysesfunctions} by applying the snake lemma to the \cech complex of \eqref{serredualitysesfunctions} over $\cU$, which is
        \begin{equation}
        \xymatrix{\cO_X(U_0)\times\cO_X(U_\infty) \ar@{^{(}->}[r] \ar[d]^{d_1} & K(X) \times K(X) \ar[d]^{d_2} \ar@{->>}[r] & \bigoplus \limits_{P \in U_0} K(X)/\cO_{X,P} \times \bigoplus \limits_{P \in U_\infty} K(X)/\cO_{X,P} \ar[d]^{d_3} \\
        \cO_X(U_0 \cap U_\infty) \ar@{^{(}->}[r]  & K(X) \ar@{->>}[r] & \bigoplus \limits_{P\in U_0 \cap U_\infty} K(X)/\cO_{X,P} }
        \end{equation}
    Analogously to \eqref{dualitydiagram2}, the kernel of $d_3$ is $\bigoplus_{P \in X}K(X)/\cO_{X,P}$ and the cokernel of $d_1$ is $\hone$.
    Via a diagram chase on $\left( \gamma_1 + \ldots + \gamma_g \right) \in \hone$ we can find a corresponding element \todo{B: not precise enough, rather write $\delta$ maps ... to ...}
        \begin{equation}\label{immediate}
        \left( \left(\left( \gamma_1 + \ldots + \gamma_g \right) \right)_{P \in U_0}, 0\right) \in \bigoplus_{P \in U_0}K(X)/\cO_{X,P} \times \bigoplus_{P \in U_\infty}K(X)/\cO_{X,P}.
        \end{equation}
    Since $(\alpha_ix^i/y$ is regular on $U_\infty \cap U_0$, \eqref{immediate} is equal to $\left( (g_P)_{P \in U_\infty}\, 0\right)$, where
        \[
        g_P =
            \begin{cases}
            \left( \gamma_1 + \ldots + \gamma_g \right) & \quad \text{if}\ P \in \pi^{-1}(\infty) \\
            0 & \quad \text{else.}
            \end{cases}
        \]
    Clearly $\left( (g_P)_{P \in U_0},0\right) \in \ker (d_3) = \bigoplus_{P \in X} K(X)/\cO_{X,P}$, and $\delta((f_P)_{P \in X} - (g_P)_{P \in X}) = 0$, by choice of $(g_P)_{P \in X}$.
    By the exactness of \eqref{dualitysesfunctions} it follows that there exists an $f \in K(X)$ which has Laurent tail $f_P - g_P$ at each $P \in X$, as required in the statement of the corollary.
    \end{proof}

\todo[inline]{possible new section here?}

In order to state a basis of $\derhamhone$, as well as to shorten the proof of the following theorem, we define the following polynomials. 
We suppose that $1 \leq i \leq g$.
Then when $p\neq 2$ we define
    \[
    s_i(x) := xf'(x) - 2if(x) \in k[x]
    \]
and when $p = 2$ we define
    \begin{equation}\label{capitals}
    S_i(x,y) := xF'(x) + y(xH'(x) + iH(x))\in k[x]\oplus yk[x] \subseteq k(x,y).
    \end{equation}

We now decompose these polynomials into two parts, which will be used in the sequel.
Firstly, we write $s_i(x)$ as $s_i(x) = \phi_i(x) + \psi_i(x)$, where $\psi_i(x)\in k(x)$ and $\phi_i(x) \in k[x]$ are the unique polynomials such that the degree of $\psi_i (x)$ is at most $g+1$ and $x^{g+2}$ divides $\phi_i(x)$.
Secondly we define $A_{j,i} \in k$ for $1 \leq j \leq 2g+2$, and $B_{k,i} \in k$ for $0\leq k \leq g+1$ by the equation
    \[
    S_i(x,y) = A_{2g+2,i}x^{2g+2} + \ldots + A_{1,i} x + y(B_{g+1,i} x^{g+1} + \ldots + B_{1,i} x + B_{0,i}).
    \]
Note that many of these coefficients may be zero.
In particular we remark that the $x^i$ term of $xH'(x) + iH(x)$ is always zero, since $B_{i,i}x^i = x \cdot iB_ix^{i-1} + iB_i x^i = 2iB_{i,i}x^i = 0$.


We now define the following polynomials:
    \begin{equation}\label{Split}
    \begin{split}
    \Phi_i^x(x) & =  A_{2g+2, i}x^{2g+2} + \ldots + A_{i+1, i}x^{i+1} \\
    \Psi_i^x(x) & =  A_{i,i}x^i + \ldots + A_{1,i}x \\
    \Phi_i^y(x) & =  B_{g,i}x^g + \ldots B_{i+1,i}x^{i+1} \\
    \Psi_i^y(x) & =  B_{i-1,i}x^{i-1} + \ldots + B_{1,i}x + B_{0,i}.
    \end{split}
    \end{equation}
Finally, we define $\Phi_i(x,y) = \Phi_i^x(x) + y \Phi^y_i(x)$ and $\Psi_i(x,y) = \Psi_i^x(x) + y \Psi_i^y(x)$, so that $S_i(x,y) = \Phi_i(x,y) + \Psi_i(x,y)$.

Viewing $\derhamhone$ as a quotient of \eqref{derhamconditions}, we now give a $k$-vector space basis of $\derhamhone$.

    \begin{thm}\label{basis}
    If $p \neq 2$ then the residue classes of 
        \begin{equation}\label{one}
         \left( \left( \frac{\psi_i(x)}{2yx^{i+1}}\right) dx, \left(\frac{-\phi_i(x)}{2yx^{i+1}}\right) dx, x^{-i}y\right), i=1, \ldots ,g,
        \end{equation}
    along with the residue classes of 
        \begin{equation}\label{two}
         \left( \frac{x^{i}}{y} dx , \frac{x^{i}}{y} dx, 0 \right), i = 0,\ldots ,g-1,
        \end{equation}
    form a basis of $\derhamhone$.
    
    On the other hand, if $p=2$ then the residue classes of the elements 
        \begin{equation}\label{three}
        \left( \left(\frac{\Psi_i(x,y)}{x^{i+1}H(x)}\right) dx, \left( \frac{\Phi_i(x,y)}{x^{i+1}H(x)} \right) dx, x^{-i}y \right), i =1, \ldots , g,
        \end{equation}
    together with the residue classes of 
        \begin{equation}\label{four}
        \left( \frac{x^{i}}{H(x)} dx, \frac{x^{i}}{H(x)} dx, 0 \right), i=0, \ldots, g-1,
        \end{equation}
    form a basis of $\derhamhone$.
    \end{thm}

Before proving this theorem we use it to prove the following corollary.

    \begin{cor}
    The action of $G := \aut(X)$ on $\derhamhone$ is faithful unless $G$ contains a hyperelliptic involution and $p=2$, in which case the action of the hyperelliptic involution is trivial.
    \end{cor}

    \begin{proof}
    Recall from Proposition \ref{ses} that $H^0(X,\Omega_X)$ injects into $\derhamhone$.
    Then if $p \neq 2$ or $G$ does not contain a hyperelliptic involution it follows from \cite[Thm. 4.2]{faithfulaction}\todo{reference in thesis} that $G$ acts faithfully on $H^0(X,\Omega_X)$, and hence $G$ acts faithfully on $\derhamhone$.
    
    We now suppose that $p=2$ and that $G$ contains a hyperelliptic involution, which we denote by $\sigma$.
    By the same theorem from \cite{faithfulaction}\todo{reference in thesis} we know that $\sigma$ acts trivially on $\hzero$.
    
    Since $\hzero$ is dual to $\hone$ then $\sigma$ also acts trivially on $\hone$.
    We can study exactly why this is from the view of \cech cohomology, and this will also help to determine the action of $\sigma$ on $\derhamhone$.
    If we fix a natural number $i\in \{1, \ldots ,g\}$ then $\sigma$ maps $\frac{y}{x^i}$ to $\frac{y}{x^i} + \frac{H(x)}{x^i}$. 
    Now we can split $\frac{H(x)}{x^i}$ as follows, 
        \begin{equation*}
        \frac{H(x)}{x^i} =  \frac{B_{i-1}x^{i-1} + \ldots + B_1x + B_0}{x^i} - \left( - \frac{x^d + B_{d-1}x^{d-1} + \ldots + B_ix^i}{x^i} \right),
        \end{equation*}
    where $B_j$ are as in \eqref{capitalh}.
    Since this is clearly the difference of an element of $\cO_X(U_0)$ and an element of $\cO_X(U_\infty)$ we see that $\frac{H(x)}{x^i}$ is zero in $\hone$.
    We let 
        \[
        H_{1,i}(x) = B_{i-1}x^{i-1} + \ldots + B_1x + B_0 \quad \text{ and } \quad H_{2,i}(x) = -( x^d + B_{d-1}x^{d-1} + \ldots + B_ix^i).
        \]
    
    We now consider the action of $\sigma$ on the entries in \eqref{three}.
    Firstly we see that
        \begin{align*}
        \sigma \left( \frac{-\Psi_i(x,y)}{x^{i+1}H(x)} dx\right) & = \frac{-\sigma(\Psi_i(x,y))}{x^{i+1} H(x)} dx \\
        & = \frac{-\Psi_i(x,y)}{x^{i+1}H(x)}dx + \frac{H(x)(xH_{1,i}'(x) + iH_{1,i}(x))}{x^{i+1}H(x)}dx\\
        & = \frac{-\Psi_i(x,y)}{x^{i+1}H(x)}dx + \frac{xH_{1,i}'(x) + iH_{1,i}(x)}{x^{i+1}}dx \\
        & = \frac{-\Psi_i(x,y)}{x^{i+1}H(x)}dx +  \frac{H_{1,i}'(x)}{x^i}dx + \frac{iH_{1,i}(x)}{x^{i+1}}dx \\
        & = \frac{-\Psi_i(x,y)}{x^{i+1}H(x)}dx +  \frac{1}{x^i}d\left( H_{1,i}(x) \right) + H_{1,i}(x) d \left( \frac{1}{x^i} \right) \\
        & = \frac{-\Psi_i(x,y)}{x^{i+1}H(x)}dx + d\left( \frac{H_{1,i}(x)}{x^i} \right),
        \end{align*}
    where the second equality follows from \eqref{capitals} and the fact that $\sigma(y) = y + H(x)$.
    
    Similarly we can derive
        \begin{equation*}
        \sigma \left( \frac{\Phi_i(x,y)}{x^{i+1}H(x)} dx \right)  = \frac{\Phi_i(x,y)}{x^{i+1}H(x)} dx + d \left( \frac{H_{2,i}(x)}{x^i} \right).
        \end{equation*}
    Lastly, it is clear that $\sigma(x^{-i}y) = x^{-i}(y+H(x))$.
    
    
    We can now describe exactly how sigma acts on the elements of \eqref{three} using $H_{1,i}(x)$ and $H_{2,i}(x)$:
        \begin{multline*}
        \sigma \left( \left( \left(\frac{-\Psi_i(x,y)}{x^{i+1}H(x)}\right) dx, \left( \frac{\Phi_i(x,y)}{x^{i+1}H(x)} \right) dx, x^{-i}y \right)\right) = \\
         \left( \left(\frac{-\Psi_i(x,y)}{x^{i+1}H(x)} \right) dx + d\left(\frac{H_{1,i}(x)}{x^i}\right),  \left( \frac{\Phi_i(x,y)}{x^{i+1}H(x)} \right) dx+ d\left(\frac{H_{2,i}(x)}{x^i} \right), \frac{y+H(x)}{x^i} \right).
        \end{multline*}
    So the action of $\sigma$ on the basis elements in \eqref{three} amounts to adding 
        \[
        \left( d\left(\frac{H(x)_{1,i}}{x^i}\right), d\left(\frac{H(x)_{2,i}}{x^i}\right), \frac{H(x)}{x^i} \right),
        \]
    which clearly satisfies the conditions of \eqref{quotient} and hence is zero.
    So the action of the involution $\sigma$ on $\derhamhone$ is trivial and hence the action of the group $G$ is not faithful.
    \end{proof}

    \begin{rem}
    We briefly study the action of $\sigma$ on the elements \eqref{one} (when $p\neq 2$).
    When $p \neq 2$ then $\sigma$ acts by $(x,y) \mapsto (x,-y)$.
    If we let
        \[
        \nu_i = \left( \left( \frac{\psi_i(x)}{2yx^{i+1}}\right) dx, \left(\frac{-\phi_i(x)}{2yx^{i+1}}\right) dx, x^{-i}y\right)
        \]
    then 
        \begin{equation*}
        \sigma(\nu_i) = -\nu_i.
        \end{equation*}
    Similarly, if 
        \[
        \eta_i = \left( \frac{x^i}{y}dx, \frac{x^i}{y}dx, 0 \right)
        \]
    then 
        \[
        \sigma(\eta_i) = - \eta_i.
        \]
    Hence $\sigma$ acts by multiplication with $-1$ on $\derhamhone$.
    \end{rem}


We now prove Theorem \ref{basis}.

    \begin{proof}
    We make use of the fact that the short exact sequence in Lemma \ref{ses} splits as a sequence of vector spaces over $k$, and that we know bases of the outer two terms.
    
    It is clear that the elements in \eqref{two} and \eqref{four} are elements of the space \eqref{derhamconditions}. 
    In fact, it follows from \cite[Thm. 6.1]{faithfulaction}\todo{reference in thesis} that they are the image of a basis of $H^0(X,\Omega_X)$ in $\derhamhone$.
    
    Moreover, it is obvious that if the elements in \eqref{one} and \eqref{three} are well defined elements of the space \eqref{derhamconditions} then they will map to the basis of $\hone$ given in Theorem \ref{basish1}.
    So we need only show that the terms in \eqref{one} and \eqref{three} satisfy the conditions stated in \eqref{derhamconditions}.
    For the rest of the proof we fix $i \in \{1, \ldots ,g\}$.
    
    
    We start with the case $p\neq 2$, and observe that
        \begin{align*}
        \left(  \frac{\psi_i(x)}{2yx^{i+1}}  - \frac{-\phi_i(x)}{2yx^{i+1}} \right) dx & =  \frac{s_i(x)}{2yx^{i+1}} dx \\
        & =  \frac{1}{2yx^i} \left( f(x)' - \frac{2if(x)}{x} \right) dx \\
        & =  \frac{x^i}{2y} \left( \frac{f(x)'}{x^{2i}}dx -\frac{2if(x)}{x^{2i+1}} dx \right) \\
        & =  \frac{x^i}{2y} \left( f(x)d\left(\frac{1}{x^{2i}}\right) + \frac{1}{x^{2i}}df(x) \right) \\
        & =  \frac{x^i}{2y}d(f(x)x^{-2i}) \\
        & =  \frac{x^i}{2y} d\left(\left(yx^{-i}\right)^2\right) \\
        & =  d(yx^{-i}),
        \end{align*}
    with the penultimate line following from the defining equation \eqref{definingequationpnot2}.
    This shows that the elements in \eqref{one} satisfy $df_{0, \infty} = \omega_0 - \omega_\infty$, one of the conditions of \eqref{derhamconditions}.
    Since we saw in the proof of Theorem \ref{basish1} that $\frac{y}{x^i}$ is regular on $U_0\cap U_\infty$ it only remains to show that $\frac{\phi_i(x)}{2yx^{i+1}}dx$ and $\frac{-\psi_i(x)}{2yx^{i+1}}dx$ are regular on $U_\infty$ and $U_0$ respectively.
    
    
    In order to do this we define $\alpha_{j,i} \in k$ for $0\leq j \leq 2g+2$ to satisfy the equation
        \[
        s_i(x) = \alpha_{2g+2,i}x^{2g+2} + \ldots + \alpha_{0,i},
        \]
    so that
        \begin{align*}
        \phi_i(x) = \alpha_{2g+2,i}x^{2g+2} + \ldots + \alpha_{g+2,i}x^{g+2} \\
        \intertext{and }
        \psi_i(x) = \alpha_{g+1,i}x^{g+1} + \ldots + \alpha_{0,i}.
        \end{align*}
    Note that it is possible for any of $\alpha_{j,i}$ to be zero. In fact, it is possible for either $\phi_i(x)$ or $\psi_i(x)$ to be zero.
    Whenever they are non-zero we denote their degrees as polynomials in $x$ by $d_\phi$ and $d_\psi$ respectively. From the definition of $\phi_i(x)$ and $\psi_i(x)$ we know that $0 \leq d_\psi \leq g+1$ and $g+1 < d_\phi \leq 2g+2$.
    
    
    We now show that $\frac{-\phi_i(x)}{2yx^{i+1}}dx$ and $\frac{\psi_i(x)}{2yx^{i+1}}dx$ are regular on $U_\infty$ and $U_0$ respectively.
    We may assume that $\phi_i(x)$ and $\psi_i(x)$ are non-zero, since the zero function is regular everywhere.
    
    
    The divisor of $\frac{-\phi_i(x)}{2yx^{i+1}}dx$ is
        \begin{align*}
        \di\left( \frac{-\phi_i(x)}{2yx^{i+1}}dx \right) & =  \di(\phi_i(x)) -\di(y) - \di(x^{i+1}) + \di (dx) \\
        & =  \di(\phi_i(x)) - ( R - (g+1)D_\infty) - ((i+1)D_0 - (i+1)D_\infty) \\
        & \qquad + (R - 2D_\infty) \\
        & =  \left( \di_0\left( \frac{\phi_i(x)}{x^{g+2}}\right) + (g+2)D_0 - d_\phi D_\infty\right) - (i+1)D_0 + (g+i)D_\infty \\
        & \geq  \di_0\left( \frac{\phi_i(x)}{x^{g+2}}\right) + (g+2)D_0 - (2g+2)D_\infty - (i+1)D_0 + (g+i)D_\infty \\
        & =  \di_0\left( \frac{\phi_i(x)}{x^{g+2}} \right) + (i-g-2)D_\infty + (g-i+1)D_0,
        \end{align*}
    where the second equality makes use of \eqref{divxp=2} and \eqref{pnot2divisors}.
    Since $i \leq g$ the differential $\frac{-\phi_i(x)}{2yx^{i+1}}dx$ is regular on $U_\infty = X\backslash \pi^{-1}(\infty)$.
    
    Similarly the divisor of $\frac{\psi_i(x)}{2yx^{i+1}}dx$ is
    
        \begin{align*}
        \di \left( \frac{\psi_i(x)}{2yx^{i+1}}dx\right) & =  \di(\psi_i(x)) - \di(y) - \di(x^{i+1}) + \di (dx) \\
        & =  \di (\psi_i(x) ) -(R - (g+1)D_\infty) - ((i+1)D_0 - (i+1)D_\infty) \\ 
        & \qquad + (R -2D_\infty) \\
        & =  \di(\psi_i(x)) + (g+i)D_\infty -(i+1)D_0 \\
        & =  (\di_0(\psi_i(x)) -d_\psi D_\infty) + (g+i)D_\infty -(i+1)D_0 \\
        & \geq \left( \di_0(\psi_i(x)) - (g+1)D_\infty \right) + (g+i)D_\infty -(i+1)D_0 \\
        & =  \di_0(\psi_i(x)) + (i-1)D_\infty - (i+1)D_0.
        \end{align*}
    Again, the second equality uses \eqref{divxp=2} and \eqref{pnot2divisors}, and since $i\geq 1$ we conclude that $\frac{\psi_i(x)}{2yx^{i+1}}dx$ is regular on $U_0 = X \backslash \pi^{-1}(0)$, completing the $p\neq 2$ case.
    
    
    We now suppose that $p=2$.
    We remind the reader that this allows us to change signs between positive and negative as we wish.
    We see that
        \begin{align*}
        \left( \left( \frac{ \Psi_i(x,y)}{x^{i+1}H} \right) + \left( \frac{\Phi_i(x,y)}{x^{i+1}H} \right) \right) dx & =  \frac{S_i(x,y)}{x^{i+1}H(x)}dx \\
        & =  \left( \frac{F(x)'}{x^iH(x)} + \frac{yH(x)'}{x^iH(x)} + \frac{iy}{x^{i+1}} \right) dx \\
        & =  \frac{1}{x^i}\left( \frac{F(x)' + yH(x)'}{H(x)} \right) dx + \frac{iy}{x^{i+1}}dx \\
        & =  x^{-i}dy + yd \left( x^{-i}\right) \\
        & =  d\left( yx^{-i}\right),
        \end{align*}
    with the fourth equality following from \eqref{divdyp=2}.
    We have also already seen in the proof of Theorem \ref{basish1} that $\frac{y}{x^i}$ is regular on $U_0 \cap U_\infty$.
    So in order to prove that for $i\in \{1, \ldots, g\}$ the elements of \eqref{three} are satisfy the conditions of \eqref{derhamconditions} it only remains to show that the differentials $\frac{\Phi_i(x,y)}{x^{i+1}H(x)}dx$ and $\frac{\Psi_i(x,y)}{x^{i+1}H(x)}dx$ are regular on $U_\infty$ and $U_0$ respectively.
    We denote the degrees of the polynomials defined in \eqref{Split} by $d_{\Phi}^x, d_{\Psi}^x, d_{\Phi}^y$ and $d_{\Psi}^y$.
    
    
    By \eqref{Split} $\Phi_i(x,y) = \Phi_i^x(x) + y\Phi_i^y(x)$ and $\Psi_i (x,y)= \Psi_i^x(x) + y\Psi_i^y(x)$, and we will use these splittings to show that $\frac{ \Phi_i(x,y) }{x^{i+1}H(x)}dx$ and $\frac{\Psi_i(x,y) }{x^{i+1}H(x)}dx$ are regular on $U_\infty$ and $U_0$ respectively.\todo{note that some stuff was removed before this para}
    
    We start by computing the divisor of $\frac{1}{x^{i+1}H(x)}dx$, since it is a common component to all the differentials we need to look at.
    This yields
        \begin{align*}
        \di \left( \frac{1}{x^{i+1}H(x)}dx \right) & = \di(dx) - \di (x^{i+1}) - \di (H(x)) \nonumber \\
        & = (R-2D_\infty) - ((i+1)D_0 - (i+1)D_\infty) - (R - (g+1)D_\infty) \nonumber \\
        & = (g+i)D_\infty - (i+1)D_0,
        \end{align*}
    using \eqref{differentialdivisor}, \eqref{divisorofH} and \eqref{divxp=2}.
    We now use this along with Proposition \ref{divyp=2} and the polynomials \eqref{Split} to complete the proof.
    
    We begin by computing the divisors associated to $\Phi_i(x,y)$.
    Firstly,
        \begin{align*}
        \di \left( \frac{\Phi_i^x(x) }{x^{i+1} H(x)}dx \right)  = &  \di(\Phi_i^x(x)) -(i+1)D_0 + (g+i)D_\infty\\
         = & \left( \di_0(\Phi_i^x(x)) -d_\Phi^xD_\infty\right) -(i+1)D_0 + (g+i)D_\infty\\
         \geq & \di_0(\Phi_i^x(x)) - (2g+2)D_\infty - (i+1)D_0 + (g+i)D_\infty \\
         = &  \di_0(\Phi_i^x(x)) - (i+1)D_0 + (i-2-g)D_\infty \\
         =  & \di_0 \left( \frac{\Phi_i^x(x)}{x^{i+1}} \right) + (i-g-2)D_\infty.
        \end{align*}
    From this we see that the differential $\frac{\Phi_i^x(x)}{x^{i+1}H(x)}dx$ is clearly regular on $U_\infty = X \backslash \pi^{-1}(\infty)$.
    
    We now compute the divisor of the other half of $\frac{\Phi_i(x,y)}{x^{i+1}H(x)}dx$, namely
        \begin{align*}
        \di\left(\frac{y\Phi_i^y(x) dx}{x^{i+1}H(x)} \right)  = & \di(y) + \di(\Phi_i^y(x)) -(i+1)D_0 + (g+i)D_\infty\\
         = & \di(y) + \di_0(\Phi_i^y(x)) - d_\Phi^yD_\infty -(i+1)D_0 + (g+i)D_\infty \\
         \geq & \di(y) + \di_0(\Phi_i^y(x)) - (g+1)D_\infty - (i+1)D_0 + (g+i)D_\infty \\
         = & \di(y) + \di_0\left(\frac{\Phi_i^y(x)}{x^{i+1}} \right) + (i-1)D_\infty.
        \end{align*}
    From Proposition \ref{divyp=2} we see that $y$ only has poles at points in $\pi^{-1}(\infty)$, and hence this completes the proof that $\frac{\Phi_i(x,y) }{x^{i+1}H(x)}dx$ is regular on $U_\infty = X \backslash \pi^{-1}(\infty)$.
    
    Now we complete the same computations on $\Psi_i(x,y)$, starting with $\Psi_i^x(x)$:
        \begin{align*}
        \di\left( \frac{\Psi_i^x(x) }{x^{i+1}H(x)}dx \right)  & =   \di(\Psi_i^x(x))  - (i+1)D_0 + (g+i)D_\infty \\
        & = (\di_0(\Psi_i^x(x)) -d_\Psi^xD_\infty) - (i+1)D_0 + (g+i)D_\infty \\
         & \geq   \di_0(\Psi_i^x(x) ) - iD_\infty - (i+1)D_0 + (g+i)D_\infty \\
         & =   \di_0(\Psi_i^x(x)) - (i+1)D_0 + gD_\infty,
        \end{align*}
    and it is clear that the divisor is positive on $U_0 = X \backslash \pi^{-1}(0)$.
    
    For the other half of the differential we need to consider separate cases.
    If we assume that $\infty$ is branch point then  using Proposition \ref{divyp=2} we see that
        \begin{align*}
        \di\left(\frac{y\Psi_i^y(x) }{x^{i+1}H(x)}dx \right)  =  & \di_0(y) - (2g+1)[P_\infty] + \di(\Psi_i^y(x)) - (i+1)D_0 + (g+i)[P_\infty] \\
         =  & \di_0(y) + \di(\Psi_i^y(x)) -(i+1)D_0 + (2i -1)[P_\infty] \\
         = &  \di_0(y) + \di_0(\Psi_i^y(x)) - d_\Psi^y[P_\infty] - (i+1)D_0 + (2i-1)[P_\infty] \\
         \geq &  \di_0(y) + \di_0(\Psi_i^y(x)) -(i-1)[P_\infty] -(i+1)D_0 + (2i-1)[P_\infty] \\
         =   &\di_0(y) + \di_0(\Psi_i^y(x)) -(i+1)D_0 + [P_\infty],
        \end{align*}
    which is clearly positive on $U_0$.
    On the other hand, if $\infty$ is not a branch point then we have
        \begin{align*}
        \di\left(\frac{y\Psi_i^y(x) }{x^{i+1}H(x)}dx \right)  =  & \di(y) + \di(\Psi_i^y(x)) - (i+1)D_0 + (g+i)D_\infty \\
        = & \di(y) + \di_0(\Psi_i^y(x)) - (i+1)D_0 + (g+i - d_\Psi^y)D_\infty \\
        \geq & \di(y) + \di_0(\Psi_i^y(x)) - (i+1)D_0 + (g+1)D_\infty. \\
        \end{align*}
    Since we know from Proposition \ref{divyp=2} that $y$ cannot have a pole of order greater $g+1$ at $P_\infty$ or $P_\infty'$, and only has poles at these points, it follows that the differential $\frac{y\Psi_i^y(x) }{x^{i+1}H(x)}dx$ is regular on $U_0 = X \backslash \pi^{-1}(0)$.
    Thus we have completed the proof.
    
    
    \end{proof}


%%%%%%%%%%%%%%%%%%%%% Section 4 %%%%%%%%%%%%%%%%%%%%

\section{Splitting of the short exact sequence}


We keep the assumptions of the previous section, except that we now also assume that $\cha(k) = p$ is odd.

In the previous section (Theorem \ref{basis}) we found a basis for the de Rham cohomology of any hyperelliptic curve using \cech cohomology, with respect to the cover $\cU = \{ U_0 , U_\infty\}$.
We let $\lambda_i$ and $\gamma_i$ denote the elements of this basis by defining
    \begin{align*}
    \lambda_i  = & \left[ \left( \frac{x^i}{y}dx, \frac{x^i}{y}dx, 0\right) \right] ,\quad i=0, \ldots, g-1 \\
    \intertext{and}
    \gamma_i = & \left[ \left ( \frac{\psi_i(x)}{2yx^{i+1}}dx, \frac{-\phi_i(x)}{2yx^{i+1}}dx, x^{-i}y \right), \right] \quad i=1,\ldots ,g,
    \end{align*}
where $[z]$ denotes the residue class of $z$.
In this section we moreover consider the covers $\cU' = \{U_a, U_\infty\}$ and $\cU' = \{U_0, U_a, U_\infty\}$ for some fixed $a \in \mathbb{P}_k^1\backslash \{0, \infty\}$.
Then $\cechderhamhone(\cU'')$ is isomorphic to the $k$-vector space 
    \begin{multline}\label{sixtupleconditions}
    \left\{ (\omega_0, \omega_a, \omega_\infty , f_{0a}, f_{0 \infty},f_{a \infty}) | \omega_i \in \Omega_X(U_i), f_{ij} \in \cO_X(U_i \cap U_j), \right. \\ \left. f_{0a} - f_{0\infty} + f_{a \infty} = 0, df_{ij} = \omega_i - \omega_j \right\}
    \end{multline}
quotiented by the subspace 
    \[
    \left\{ df_0, df_a df_\infty, f_0- f_a, f_0 - f_\infty, f_a - f_\infty | f_i \in \cO_X(U_i)\right\}.
    \]
We have a canonical projection $\rho\colon \cechderhamhone(\cU'') \ra \cechderhamhone(cU)$, given by 
    \[
    \rho \colon (\omega_0, \omega_a, \omega_\infty , f_{0a}, f_{0 \infty},f_{a \infty}) \mapsto (\omega_0, \omega_\infty , f_{0 \infty}).
    \]

We also define the following polynomials for $1 \leq i \leq g$
    \[
    r_i(x) : = \sum_{k=0}^{i-1} (-1)^{g-k}\binom{g}{k} a^{g-k} x^k
    \]
and
    \[
    t_i(x) := \sum_{k=i}^{g} (-1)^{g-k}\binom{g}{k} a^{g-k} x^k,
    \]
splitting the polynomial $(x-a)^g$ in to two parts.


    \begin{prop}\label{basis22}
    The pre-image $\rho^{-1}(\gamma_i)$ for $i \in \{1, \ldots, g\}$ is the residue class of
        \begin{multline*}
        \nu_i = \left(\frac{\psi_i(x)}{2yx^{i+1}}dx, \frac{(\psi_i(x)t_i(x) - \phi_i(x)r_i(x))(x-a) + 2if(x)(-1)^{g-i+1}\binom{g}{i} a^{g-i+1}x^i}{2yx^{i+1}(x-a)^{g+1}}dx,\right. \\\left. \frac{-\phi_i(x)}{2yx^{i+1}}dx,  \frac{r_i(x)y}{x^i(x-a)^g}, \frac{y}{x^i},  \frac{t_i(x)y}{x^i(x-a)^g} \right).
        \end{multline*}
    \end{prop}
    \begin{proof}
    In order to be able to refer to the entries in $\nu_i$ we let
        \[
        \nu_i = \left( \omega_{0 i}, \omega_{a i}, \omega_{\infty i}, f_{0 a i}, f_{0 \infty i}, f_{a \infty i} \right).
        \]
    First, note that it follows from the proof of Theorem \ref{basis} that $d(f_{0 \infty i}) = \omega_{0 i} - \omega_{\infty i}$, and that $f_{0 \infty i}, \omega_{0 i}$ and $\omega_{\infty i}$ are regular on the appropriate open sets.
    
    Since $r_i(x)+t_i(x)$ is the binary expansion of $(x-a)^g$ then
        \begin{align*}
        f_{0 a i} - f_{0 \infty i}+ f_{a \infty i} & = \frac{r_i(x)y}{x^i(x-a)^g} - \frac{y}{x^i} + \frac{t_i(x)y}{x^i(x-a)^g} \\
        & = \frac{y(r_i(x) + t_i(x) - (x-a)^g)}{x^i(x-a)^g} \\
        & = 0.
        \end{align*}
    
    
    
    We now check that differentials and functions in $\nu_i$ are regular on the appropriate open sets by computing the relevant divisors.
    Firstly, by \eqref{divxp=2} and \eqref{differentialdivisor},
        \begin{align*}
        \di \left( f_{0 a i} \right) & = \di \left( \frac{r_i(x)y}{x^i(x-a)^g} \right) \\
        &  = \di(r_i(x)) + \di(y) - i\di(x) - g\di(x-a) \\
        & \geq \di_0(r_i(x)) - (i-1)D_\infty +R - (g+1)D_\infty - iD_0 + iD_\infty - gD_a + gD_\infty \\
        & = \di_0(r_i(x)) +R -iD_0 - gD_a,
        \end{align*}
    which is non-negative on $U_0 \cap U_a$.
    Note that the second and third line are not necessarily equal, since the coefficient of $x^{i-1}$ may be divisible by $p$.
    On the other hand, by \eqref{divxp=2} and \eqref{differentialdivisor},
        \begin{align*}
        \di \left( f_{a \infty i} \right) & = \di \left( \frac{t_i(x)y}{x^i(x-a)^g} \right) \\
        & = \di\left(\frac{t_i(x)}{x^i}\right) + \di(y) - g\di(x-a) \\
        & = \di_0 \left( \frac{t_i(x)}{x^i} \right) - (g-i)D_\infty +R - (g+1)D_\infty - gD_a + gD_\infty\\
        & = \di_0 \left( \frac{t_i(x)}{x^i} \right) +R - gD_a -(g-i+1)D_\infty,
        \end{align*}
    where the third equality holds because $t_i(x)/x^i$ is regular on $U_\infty$.
    We conclude that $f_{0 infty i}$ is regular on $U_a \cap U_\infty$.
    
    To show that
        \begin{equation}\label{longequation}
        \omega_{a i} =  \frac{(\psi_i(x)t_i(x) - \phi_i(x)r_i(x))(x-a) + 2if(x)(-1)^{g-i+1}\binom{g}{i} a^{g-i+1}x^i}{2yx^{i+1}(x-a)^{g+1}}dx
        \end{equation}
    is regular on $U_a$ we first compute the divisor
        \begin{align*}
        \di\left( \frac{dx}{2yx^{i+1}(x-a)^{g+1}}\right) & = \di(dx) - \di(y) - (i+1)\di(x) - (g+1)\di(x-a) \\
        & = R - 2D_\infty - R + (g+1)D_\infty - (i+1)D_0 + (i+1)D_\infty \\
        & \quad - (g+1)D_a + (g+1)D_\infty \\
        & = (2g+i+1)D_\infty -(i+1)D_0 - (g+1)D_a,
        \end{align*}
    using \eqref{divxp=2}, \eqref{differentialdivisor} and \eqref{pnot2divisors}.
    We next show that the numerator of \eqref{longequation},
        \begin{equation}\label{numerator}
        {(\psi_i(x)t_i(x) - \phi_i(x)r_i(x))(x-a) + 2if(x)(-1)^{g-i+1}\binom{g}{i} a^{g-i+1}x^i},
        \end{equation}
    has degree less than $2g+i+2$, from which it follows that \eqref{longequation} doesn't have a pole at the point(s) in $\pi^{-1}(\infty)$.
    The degree of $\psi_i(x)t_i(x)(x-a)$ is at most $2g+2$, which is less than $2g+2+i$ for all $i \geq 1$.
    If $\deg(f) = 2g+1$, then clearly
        \[
        \deg\left( \phi_i(x)r_i(x)(x-a) \right) = \deg(\phi_i) + \deg(r_i(x)) + \deg(x-a) \leq 2g+1 + i-1 +1 = 2g+i+1
        \]
    and
        \[
        \deg \left( 2if(x)(-1)^{g-i+1}\binom{g}{i} a^{g-i+1}x^i \right)  \leq  2g+1+i .
        \]
    Lastly, if $\deg(f(x)) = 2g+2$ then the term of degree $2g+i+2$ in $-\phi_i(x)r_i(x)(x-a)$ is
        \begin{align*}
        -((2g+2)a_{2g+2}x^{2g+2}-2ia_{2g+2}x^{2g+2})&\left( (-1)^{g-i+1}\binom{g}{i-1}a^{g-i+1}x^i\right) \\
        &  = 2(-1)^{g-i+2}\left( (g-i+1)\binom{g}{i-1} \right) a_{2g+2}a^{g-i+1}x^{2g+i+2} \\
        & = 2(-1)^{g-i} \left( \frac{g!}{(i-1)!(g-i)!} \right) a_{2g+2}a^{g-i+1}x^{2g+i+2} \\
        & = 2i(-1)^{g-i}\binom{g}{i}a_{2g+2}a^{2g+2}x^{2g+i+2},
        \end{align*}
    which cancels with the term of the same degree in $2if(x)(-1)^{g-i+1}\binom{g}{i}a^{g-i+1}x^i$.
    Since these terms cancel, we again have the that the degree of \eqref{numerator} is at most $2g+i+1$, and \eqref{longequation} has no pole(s) at the point(s) in $\pi^{-1}(\infty)$.
    
    Finally, we show that \eqref{numerator} is divisible by $x^{i+1}$.
    By definition $x^{g+2} | \phi_i(x)$, and since $i \leq g$ it follows that $x^{i+1}|\phi_i(x)r_i(x)(x-a)$.
    On the other hand, the lowest degree terms of $2if(x)(-1)^{g-i+1}\binom{g}{i}a^{g-i+1}x^i$ and $\psi_i(x)t_i(x)(x-a)$ which can be non-zero are, respectively,
        \[
         2ia_0(-1)^{g-i+1}\binom{g}{i}a^{g-i+1}x^i 
        \]
    and
        \[
         (-2ia_0)\left( (-1)^{g-i}\binom{g}{i}a^{g-i}x^i \right)(-a).
        \]
    When adding $\psi_i(x)t_i(x)(x-a)$ and $2if(x)(-1)^{g-i+1}\binom{g}{i}a^{g-i+1}x^i$ these two terms obviously cancel.
    Hence the numerator \eqref{numerator} is divisible by $x^{i+1}$.
    
    
    It only remains to show that $\omega_{a i} = \omega_{0 i} -df_{0 a i}$.
    We begin this by computing $df_{0 a i}$, which is
        \begin{align*}
        df_{0 a i} & = d \left( \frac{y r_i(x)}{x^i(x-a)^g} \right) \\
        & = \frac{r_i(x)}{x^i(x-a)^g}dy + y d\left( \frac{r_i(x)}{x^i(x-a)^g} \right) \\
        & = \frac{f'(x)r_i(x)}{2yx^i(x-a)^g}dx + y\left( \frac{r_i(x)}{x^i(x-a)^g} -\frac{i r_i(x)}{x^{i+1}(x-a)^g} - \frac{gr_i(x)}{x^i(x-a)^{g+1}}\right) dx \\
        & = \frac{xf'(x)r_i(x)(x-a) + 2f(xr_i'(x)(x-a) - i(x-a)r_i(x) - gxr_i(x))}{2yx^{i+1}(x-a)^{g+1}} dx.
        \end{align*}
    Hence $\omega_{0 i} - df_{0 a i}$ expands to
        \[
        \frac{\psi_i(x)(x-a)^{g+1} - xf'(x)r_i(x)(x-a) - 2f(x)\left(xr_i'(x)(x-a)-i(x-a)r_i(x)-gxr_i(x)\right)}{2yx^{i+1}(x-a)^{g+1}}dx.
        \]
    Now
        \[
        (x-a)^{g+1} = (x-a)^g(x-a)  = (r_i(x) + t_i(x))(x-a)
        \]
    and
        \begin{multline*}
        xf'(x)r_i(x)(x-a) - 2if(x)r_i(x)(x-a) = r_i(x)(x-a)(xf'(x)-2if(x)) \\
        = r_i(x)(x-a)(\psi_i(x) + \phi_i(x)).
        \end{multline*}
    So
        \[
        \psi_i(x)(x-a)^{g+1} - xf'(x)r_i(x)(x-a) + 2if(x)r_i(x)(x-a) = (\psi_i(x)t_i(x) - \phi_i(x) r_i(x))(x-a).
        \]
    
    
    We now compute $(x-a)r_i'(x)-gr_i(x)$.
    First, we note that
        \begin{align*}
        r_i'(x) & = \sum_{k=1}^{i-1} k (-1)^{g-k} \binom{g}{k} a^{g-k} x^{k-1} \\
        & = \sum_{k=0}^{i-2} (k+1) (-1)^{g-k-1} \binom{g}{k+1}a^{g-k-1} x^k.
        \end{align*}
    From this it follows that
        \begin{align*}
        r_i'(x)(x-a) & = x \sum_{k=1}^{i-1} k (-1)^{g-k} \binom{g}{k} a^{g-k} x^{k-1} - a \sum_{k=0}^{i-2} (k+1) (-1)^{g-k-1} \binom{g}{k+1}a^{g-k-1} x^k \\
        & = \sum_{k=1}^{i-1} k (-1)^{g-k} \binom{g}{k} a^{g-k} x^k  + \sum_{k=0}^{i-2} (k+1) (-1)^{g-k} \binom{g}{k+1}a^{g-k} x^k \\
        & = gr_i(x) + (-1)^{g-i+2}i \binom{g}{i}a^{g-i+1}x^i,
        \end{align*}
    since
        \begin{align*}
        k\binom{g}{k} + (k+1)\binom{g}{k+1} & = k \left( \frac{g!}{k!(g-k)!} \right) + (k+1) \left( \frac{g!}{(k+1)!(g-k-1)!} \right) \\
        & = \frac{g!}{(k-1)!(g-k)!} + \frac{g!}{k!(g-k-1)!} \\
        & = \frac{g\cdot g!}{k!(g-k)!} \\
        & = g \binom{g}{k}.
        \end{align*}
    Hence $r_i'(x)-gr_i(x)= (-1)^{g-i+2}i\binom{g}{i} a^{g-i+1}x^i$.\todo{check if a factor of $x$ is missing}
    
    Combining the above we conclude that
        \[
        \omega_{0 i } - df_{0a i} =  \frac{(\psi_i(x)t_i(x) - \phi_i(x)r_i(x))(x-a) + 2if(x)(-1)^{g-i+1}\binom{g}{i} a^{g-i+1}x^i}{2yx^{i+1}(x-a)^{g+1}}dx= \omega_{a i}.
        \]
    
    Note that the last relation ($df_{a \infty i} = \omega_{a i} - \omega_{\infty i}$) holds, since 
        \[
        df_{a \infty i} = df_{0 \infty i} - df_{0 a i} = \omega_{0 i} - \omega_{\infty i } - \omega_{0 i} + \omega_{a i} = \omega_{a i} - \omega_{\infty i}.
        \]
    \end{proof}

We note that for any $\sigma \in \aut(X)$ the following diagram commutes,
    \[
    \begin{array}{ccc}
    X & \xrightarrow[\sigma] & X \\
    \downarrow\pi & & \downarrow\pi \\
    \mathbb P^1_k & \xrightarrow[\sigma] & \mathbb P_k^1
    \end{array}
    \]\todo{change to $\tau$?}
since the hyperelliptic involution $j$ is in the centre of $\aut(X)$.
Hence, if $a = \sigma(0)$, the following diagram also commutes:
    \[
    \begin{array}{ccc}
    \derhamhone \cong \cechderhamhone(\cU)  & \xleftarrow{\rho} & \cechderhamhone(\cU'')  \\
            \sigma^*\downarrow & ~ & \rho'\downarrow  \\
    \derhamhone \cong \cechderhamhone(\cU)  & \xleftarrow{\sigma^*} & \cechderhamhone(\cU')
    \end{array}
    \]
Here $\rho$ and $\rho'$ are the projections on the first, third and fifth coordinates and on the second, third and sixth coordinates respectively.


    \begin{lem}
    Suppose that $\tau \in \aut(X)$ is such that $\tau^*\colon x \mapsto x+a$.
    Then the action of $\sigma^*$ on $y$ is given by $\sigma^*(y) = y$ or $\sigma^*(y) = -y$.
    Moreover, if such an automorphism of $X$ exists, then $p$ divides the degree of$f(x)$.
    \end{lem}
    \begin{proof}
    We first show that $\tau^*(y) = \pm y$.
    Since $y^2 \in k(x)$ then there must exist $g_1(x), g_2(x) \in k(x)$ such that 
        \begin{equation*}
        \sigma^*(y) = g_1(x)y + g_2(x) \in k(x,y).
        \end{equation*}
    Hence
        \begin{equation}\label{easylemma}
        f(x+a) = \sigma^*(y^2) = (\sigma^*(y))^2 = g_1(x)^2f(x)+2g_1(x)g_2(x)y + g_2(x)^2.
        \end{equation}
    Firstly, note that if neither $g_1(x)$ nor $g_2(x)$ are zero then
        \[
        y = \frac{f(x+a) - g_1(x)^2f(x) - g_2(x)^2}{2g_1(x)g_2(x)},
        \]
    which contradicts the fact $K(X)$ is a degree two extension of $k(x)$.
    Hence one of $g_1(x)$ or $g_2(x)$ must be zero.
    
    If $g_1(x) = 0$ then $\sigma^*$ would not be an automorphism, since $y$ would not be in the image.
    Hence $\sigma^*(y) = g_1(x)y$.
    Also, by comparing the degrees in \eqref{easylemma} we see that $\deg(g_1(x)) = 0$, and then by comparing coefficients in the same equation we see that $g_1(x)^2 = 1$.
    Hence $\tau^*(y) = \pm y$.

    We now show that $d_f := \deg(f(x))$ is divisible by $p$.
    Clearly 
        \[
        f(x) = y^2 = (\tau^*(y))^2 = \tau(f(x)) = f(x+a).  
        \]  
    Comparing the terms of degree $d_f-1$ on each side we see that $d_f - a_{d_f -1} = a_{d_f-1}$.
    Hence $d_f = 0$ in $k$, and since $\deg(f(x)) > 0$ it follows that $p \mid d_f$.
    \end{proof}

    \begin{rem}
    If $\sigma^*(y) = -y$ we can, without loss of generality, replace $\sigma$ by $\sigma \circ j$, where $j$ denotes the hyperelliptic involution.
    Hence we will assume throughout the rest of the paper that $\sigma^*(y) = y$.
    \end{rem}

    \begin{ex}
    If $2g+1$ is an odd prime then exists an automorphism $\sigma$ of the form $\sigma\colon (x,y) \mapsto (x+a,y)$ if and only if $p=2g+1$ and $f(x) = x^p - a^{p-1}x + a_0$, for some $a_0 \in k$.
    In this case we may assume that $a_0 = 0$, since if it doesn't we can apply the automorphism $x \mapsto x+b, y\mapsto y$ to $K(X)$, where $b$ is a root of $f(x)$.
    Moreover, given an equation of the form $y^2 = x^p - a^{p-1}x$, we can apply automorphism of $K(X)$ to replace the coefficient $a^{p-1}$ by $1$, namely $x \mapsto ax, y \mapsto a^{\frac{p}{2}}y$.
    Hence we see that all such curves are isomorphic to those defined by $y^2 = x^p - x$.
    In particular, it follows from \cite{canonicalrepresentation} that the short exact sequence in Proposition \ref{ses} doesn't split when the above conditions hold.
    \end{ex}


We now describe the action of $\sigma^*$ on $\lambda_i$.
    
    \begin{lem}
    For each $i \in \{ 0, \ldots, g-1\}$ then 
        \[
        \sigma^*(\lambda_i) = \sum_{k = 0}^i \binom{i}{k}a^{i-k}x^k.
        \]
    \end{lem}
    \begin{proof}
    Since $\sigma^*$ acts trivially on $y$, it follows that
        \[
        \sigma^*\left( \frac{x^i}{y} \right) = \sum_{k=0}^i \binom{i}{k}a^{i-k}\frac{x^k}{y}.
        \]
    The statement follows from this.
    \end{proof}


    \begin{thm}
    Suppose that $\deg(f(x)) = p^n $ for some $n \in \NN$.
    Then the short exact sequence of $k[G]$-modules
        \[
        0 \ra \hzero \ra \derhamhone \ra \hone \ra 0
        \]
    does not split.
    \end{thm}
    \begin{proof}
    We suppose that the sequence does split, and that $s \colon \hone \ra \derhamhone$, given by $s \colon \bar\gamma_i \mapsto \gamma_i$, is the splitting map.
    
    We now examine the action of $\sigma^*$ on $\gamma_g$ and $\bar \gamma_g$.
    We first look at the sixth entry in $\nu_g$ from Proposition \ref{basis22}.
    This entry is
        \[
        \frac{t_g(x)y}{x^g(x-a)^g} = \frac{y}{(x-a)^g},
        \]
    and clearly $\sigma^*(y/(x-a)^g) = y/x^g$.
    Hence $\sigma^*(\bar\gamma_g) = \bar \gamma_g$, and $\sigma^*(\gamma_g) = \gamma_g + \sum_{i =0}^{g-1}c_i\lambda_i$, for some $c_i \in k$.
    We now compute $c_{g-1}$.
    
    We begin this by computing the lead term of the first entry of $\sigma^*(\gamma_g) \in \cechderhamhone(\cU)$.
    This is equal to the lead term of $\omega_{\sigma g}$ in $\nu_g$, since applying $\sigma^*$ doesn't change the lead term of any polynomials.
    As
        \[
        \deg(\psi_g(x)t_g(x)x) \leq g+1 + g+ 1 = 2g+1 < 3g+1 = \deg(\phi_g(x)r_g(x)x) = \deg(f(x)x^g)
        \]
    we need only compute the coefficient of $x^{3g+1}$ in $2gf(x)(-1)ax^g - \phi_g(x)r_g(x)x$ and show that it is non-zero to find the coefficient of the lead term.
    Rearranging $2g+1 = p^n $ gives us the identity
        \[
        g = \frac{p^n - 1}{2}.
        \]
    From this we see that the lead coefficient of $2gf(x)(-1)ax^g$ is 
        \[
        2\left( \frac{p^n-1}{2} \right) (-1)a = a
        \]
    since $\cha(k) = p$.
    On the other hand, the lead term of $-\phi_g(x)r_g(x)x$ is
        \[
        -(p^n-2g)(-1)\binom{g}{g-1}a = 2\left(\frac{p^n -1 }{2}\right) (-1)\left( \frac{p^n - 1}{2} \right)a = -\frac{a}{2}.
        \]
    Finally, it follows that the lead coefficient of the numerator in the second term of is
        \[
        a - \frac{a}{2} = \frac{a}{2}.
        \]
    
    
    Since the denominator of $\omega_{a g}$ is of degree $2g+2$, we see that overall the degree of the first entry of $\gamma_g$ is $g-1$ (as a polynomial in $x$).
    Now the degree of $\frac{\psi_g(x)}{2yx^{g+1}}$ is less than this, as is the degree of the first entry in $\lambda_i$, unless $i=g-1$, in which case the degree of $\lambda_i$ is precisely $g-1$.
    Hence, by comparing coefficients, we see that $c_{g-1} = \frac{a}{2}$.
    
    Now suppose that 
        \[
        s(\bar\gamma_i)  = s( \bar \gamma_g) = \gamma_g + \sum_{i=0}^{g-1}d_i \lambda_i
        \]
    for some $d_i \in k$.
    Then, on the one hand,
        \[
        s(\sigma^*(\bar\gamma_g)) = \gamma_g + \sum_{i=0}^{g-1}d_i\lambda_i,
        \]
    whilst on the other hand
        \begin{align}
        \sigma^*(s(\bar\gamma_g)) & = \sigma^*(\gamma_g + \sum_{i=1}^{g-1} d_i\lambda_i ) \\
        & = \gamma_g + \frac{a}{2}\lambda_{g-1} + \sum_{i=0}^{g-2} c_i \lambda_k + \sum_{i=0}^{g-1} d_i \sum_{k=0}^{i} \binom{i}{k}a^{i-k}\lambda_k.
        \end{align}
    Hence we see that for $s(\sigma^*(\gamma_g))$ to equal $\sigma^*(s(\gamma_g))$ we require $\frac{a}{2} + d_{g-1} = d_{g-1}$; \ie that $\frac{a}{2} = 0$. 
    But then $a=0$, which is a contradiction.
    \end{proof}


















\chapter{Faithful actions on Riemann-Roch spaces} \label{Chapter:Faithfulactions}

In this section our main aim is to compute when a subgroup of the automorphism group of an algebraic curve acts faithfully on the space of holomorphic differentials and polydifferentials.
Our approach uses the fact that if any finite group $G$ does not act faithfully on $H^0(X,\Omega_X^{\otimes m})$ then there exists a subgroup of $G$ which fixes at least one element of this $k$ vector space, and the dimension of the space fixed by this subgroup will be positive.

To this end, we start by computing the dimension of $H^0(X,\Omega_X^{\otimes m})$, and the dimension of the fixed space $H^0(X,\Omega_X^{\otimes m})^G$.
These dimensions rely primarily on the genus of the quotient curve $Y:=X/G$, $m$ and the ramification divisor of $\pi \colon X \ra Y$.

Then we use these formulae to compute exactly when a cyclic group of prime order will act trivially on $H^0(X,\Omega_X^{\otimes m})$.
In this case of holomorphic differentials (when $m=1$), this depends solely on the characteristic of $k$, whilst for polydifferentials (\ie when $m \geq 2$) this is independent of $\cha (k)$.
In the same section we also give similar results for more general Riemann-Roch spaces.

We then move on to the main theorem, which answers the question of when $G$ acts faithfully on $H^0(X,\Omega_X^{\otimes m})$.
After giving the main theorem we give examples which illustrate both when we do and do not have faithful actions.
In particular, we use results of Chapter 3 to explicitly show the result holds for hyperelliptic curves.

We close the chapter with an alternative proof of when a cyclic group of prime order acts faithfully on $\hzero$, by studying the $k[G]$-module structure, which was determined in \cite{valmadan}.

\section{Dimension formulae}\label{dimsection}

Throughout this chapter, unless otherwise stated, we assume that $X$ is a connected, smooth, projective algebraic curve over an algebraically closed field $k$ of characteristic $p \geq 0$.
We furthermore assume that $G$ is a finite group of order $n$ that acts faithfully on $X$.
Note that $G$ also induces an action on the vector space $H^0(X,\Omega_X^{\otimes m})$ of global holomorphic poly-differentials of order $m$.
We let $Y$ denote the quotient curve $X/G$, and we let $\pi:X\rightarrow Y$ be the canonical projection.
Finally, we denote by $g_X$ and $g_Y$ the genus of $X$ and $Y$ respectively, and we let $K_X$ and $K_Y$ be canonical divisors on $X$ and $Y$.\todo{put comment in previous section regarding divisors}


In this section we compute the dimension of $H^0(X,\Omega_X^{\otimes m})$ and of $H^0(X,\Omega_X^{\otimes m})^G$, the subspace of $H^0(X,\Omega_X^{\otimes m})$ fixed by $G$.


By Corollary \ref{dim=gc} we have $\dim_kH^0(X,\Omega_X)=g_X$.
If $m\geq 2$, we have the following formula for $\dim_kH^0(X,\Omega_X^{\otimes m})$.

    \begin{lem}\label{dim3}
    Let $m\geq 2$. Then
        \begin{equation}
        \dim_kH^0(X,\Omega_X^{\otimes m}) =
            \begin{cases}
            0 & \mbox{if } g_X=0,\\
            1 & \mbox{if } g_X=1,\\
            (2m-1)(g_X-1) & \mbox{otherwise}.
            \end{cases}
        \end{equation}
    \end{lem}
    \begin{proof}
    The trivial cases of $g_X =0$ and $g_X=1$ are explicitly explained in examples (a) and (b) in section \ref{examplessection}.
    
    If $g_X\geq 2$ then $\deg(K_X)\geq1$, so $\deg(mK_X)>\deg(K_X)$.
    Since $H^0(X,\Omega_X^{\otimes m}) \cong H^0(X,\Omega_X(mK_X))$ it then follows from the Riemann-Roch theorem (Theorem \ref{theoremriemannroch}) that
        \[
        \dim_kH^0(X,\Omega_X^{\otimes m})=\deg(mK_X)+1-g_X=(2m-1)(g_X-1).
        \]
    \end{proof}

We now introduce some notations. 
Let $D=\sum_{P\in X}n_P[P]$ be a $G$-invariant divisor on $X$ (\ie $n_{\sigma(P)} = n_P$ for all $\sigma \in G$ and $P\in X$) and let $\cO_X(D)$ denote the corresponding equivariant invertible $\cO_X$-module. 
Furthermore, let $\pi_*^G(\cO_X(D))$ denote the sub-sheaf of the direct image $\pi_*(\cO_X(D))$ fixed by the obvious action of $G$ on $\pi_*(\cO_X(D))$.
We also let $\left\lfloor \frac{\pi_*(D)}{n} \right \rfloor$ denote the divisor on $Y$ obtained from the push-forward $\pi_*(D)$ by replacing the coefficient $m_Q$ of $Q$ in $\pi_*(D)$ with the integral part $\left \lfloor \frac{m_Q}{n} \right \rfloor$ of $\frac{m_Q}{n}$ for each $Q \in Y$. 
The function fields of $X$ and~$Y$ are denoted by $K(X)$ and $K(Y)$ respectively. 
Finally, for any $P \in X$ let $\ord_P$ and $\ord_Q$ denote the respective valuations of $K(X)$ and $K(Y)$ at $P$ and $Q:=\pi(P)$.



The next lemma is the main idea in the proof of our formula for $\dim_kH^0(X,\Omega_X^{\otimes m})^G$, see Proposition \ref{dim}. 



    \begin{lem}
    Let $D=\sum_{P\in X}n_P[P]$ be a $G$-invariant divisor on $X$.
    Then the sheaves $\pi_*^G(\cO_X(D))$ and $\cO_Y\left(\left\lfloor \frac{\pi_*(D)}{n}\right \rfloor\right)$ are equal as subsheaves of the constant sheaf $K(Y)$ on $Y$. 
    In particular, the sheaf $\pi_*^G(\cO_X(D))$ is an invertible $\cO_Y$-module.
    \end{lem}
    \begin{proof}
    For every open subset $V$ of $Y$ we have 
        \[
        \pi_*^G(\cO_X(D))(V) = \cO_X(D) (\pi^{-1}(V))^G \subseteq K(X)^G = K(Y).
        \]
    In particular, both sheaves are subsheaves of the constant sheaf $K(Y)$ as stated. 
    It therefore suffices to check that their stalks are equal. 
    For any $Q \in Y$ and $P \in \pi^{-1}(Q)$.
    We have
        \begin{align*}
        \lefteqn{\pi_*^G\left(\cO_X(D)\right)_Q = \cO_X(D)_P \cap K(Y)}\\
        &= \left\{f \in K(Y): \ord_P(f) \ge -n_P\right\}\\
        &= \left\{f \in K(Y): \ord_Q(f) \ge - \frac{n_P}{e_P}\right\}\\
        &= \left\{ f \in K(Y): \ord_Q(f) \ge - \left\lfloor\frac{n_P}{e_P} \right\rfloor \right\}\\
        &= \cO_Y\left(\left\lfloor \frac{\pi_*(D)}{n} \right\rfloor\right)_Q,
        \end{align*}
    as desired.
    \end{proof}

The following proposition contains the aforementioned formula for the dimension of the subspace of $H^0(X,\Omega_X^{\otimes m})$ fixed by $G$.
In particular we see that this dimension is completely determined by $m$, $g_Y$ and $\deg \left\lfloor \frac{m\pi_*(R)}{n} \right\rfloor$.


    \begin{prop}\label{dim}
    Let $m\geq 1$. Then the dimension of $H^0(X,\Omega_X^{\otimes m})^G$ is equal to
        \[
        \dim_k \left( H^0(X,\Omega_X^{\otimes m})^G \right) = (2m-1)(g_Y-1) + \deg\left\lfloor\frac{m\pi_*(R)}{n} \right\rfloor,
        \]  
    unless one of the following conditions holds:
        \begin{itemize}
        \item $m=1 \mbox{ and } \deg\left\lfloor\frac{m\pi_*(R)}{n}\right\rfloor = 0$ or
        \item $g_Y=1 \mbox{ and } \deg\left\lfloor\frac{m\pi_*(R)}{n}\right\rfloor = 0$ or
        \item  $g_Y=0 \mbox{ and } \deg\left\lfloor\frac{m\pi_*(R)}{n}\right\rfloor < 2m-1$,
        \end{itemize}
    in which case 
        \[
        \dim_k \left( H^0(X,\Omega_X^{\otimes m})^G \right) = g_Y.
        \]      
    \end{prop}
    \begin{proof}
    Let $E$ denote the divisor $\left\lfloor \frac{\pi_*(mK_X)}{n} \right\rfloor$ on $Y$. As $K_X=\pi^*(K_Y)+R$ we have
        \[ 
        E = 
        \left \lfloor \frac{\pi_*\pi^*(mK_Y) + \pi_*(mR)}{n} \right \rfloor =
        mK_Y + \left \lfloor \frac{m\pi_*(R)}{n} \right \rfloor.
        \]
    Using the previous lemma we conclude that $\pi_*^G(\Omega_X^{\otimes m}) \cong \cO_Y (E)$ and finally that
        \begin{equation*}
        \dim_k H^0(X,\Omega_X^{\otimes m})^G 
        = \dim_k H^0\left(Y, \pi_*^G(\Omega_X^{\otimes m})\right)
        = \dim_k H^0\left(Y, \cO_Y\left( E \right) \right).
        \end{equation*}
    
    
    In the first case of the proposition, \ie if $m=1$ and $\deg \left\lfloor\frac{m\pi_*(R)}{n} \right\rfloor=0$, then $\left\lfloor\frac{m\pi_*(R)}{n} \right\rfloor$ is the zero divisor and we conclude that 
        \begin{equation*}
        \dim_kH^0(X,\Omega_X)^G = \dim_kH^0(Y, \Omega_Y) = g_Y.
        \end{equation*}
    
    
    In the second case $\left\lfloor \frac{m\pi_*(R)}{n} \right\rfloor$ is again the zero divisor. 
    Furthermore, as $g_Y=1$, the divisor $K_Y$ is equivalent to the zero divisor, and hence $mK_Y$ is too. 
    This means that
        \begin{equation*}
        \dim_kH^0(X,\Omega_X^{\otimes m})^G = \dim_kH^0\left( Y,\cO_Y\left( E \right) \right) 
        = \dim_k  H^0\left( Y,\cO_Y\left( 0 \right) \right)
        = 1.
        \end{equation*}
    
    
    For the third case, by \cite[Chap. IV, ex. 1.3.4]{hart} it suffices to show that $\deg \left( E \right) < 0$.
    As $g_Y=0$ we have $\deg(K_Y)=-2$, so $\deg(mK_Y)=-2m$, and $\deg \left( E \right)$ is indeed negative.
    
    
    
    We will show below that in all other cases $\deg(E) > \deg(K_Y)$, and then the Riemann-Roch formula (Theorem \ref{theoremriemannroch}) will give 
        \begin{align*}
        \lefteqn{\dim_kH^0(X,\Omega_X^{\otimes m})^G = \dim_kH^0\left(Y,\cO_Y\left( E \right)\right)} \\
        & =  1-g_Y+\deg\left(mK_Y+\left\lfloor{\frac{m\pi_*(R)}{n}}\right\rfloor\right) \\
        & =  (2m-1)(g_Y-1)+\deg\left\lfloor{\frac{m\pi_*(R)}{n}}\right\rfloor,
        \end{align*}
    completing the proof for this case.
    
    
    All that remains is to show that $\deg(E)>\deg(K_Y)$ in all other cases.
    Firstly, if $g_Y=0$ and $\deg \left\lfloor\frac{m\pi_*(R)}{n} \right\rfloor \geq 2m-1$ then, since $\deg(mK_Y)=-2m$, we have 
        \[
        \deg \left( E \right) \geq -1 >-2 = \deg(K_Y).
        \]
    Similarly, if $g_Y=1$ and $\deg \left\lfloor\frac{m\pi_*(R)}{n} \right\rfloor >0$ then, as $\deg \left( mK_Y \right)=0$, we have $\deg \left( E \right) > 0 = \deg (K_Y)$.
    If $m=1$ and $\deg \left\lfloor\frac{m\pi_*(R)}{n} \right\rfloor >0$ then clearly $\deg \left( E \right) > \deg (K_Y)$.
    Lastly, if $m\geq 2$ and $g_Y\geq 2$ then $\deg (K_Y) > 0$ and we have 
        \begin{equation*}
        \deg \left( E \right) \geq \deg\left( mK_Y \right) > \deg (K_Y).
        \end{equation*}
    So in all other cases $\deg(E)>\deg(K_Y)$, and the proof is complete.
    \end{proof}


If $m=1$ we reformulate Proposition \ref{dim} in the following slightly more concrete way. 
Let $S$ denote the set of all points $Q\in Y$ such that $\pi$ is not tamely ramified at $Q$ and let $s$ denote the cardinality of $S$. 
Note that $s=0$ if $p$ does not divide $n$.


For the next corollary we recall the notations $e_Q$ and $\delta_Q$ for any $Q\in Y$ defined before Theorem \ref{hilbertsformula}.


    \begin{cor}\label{dim2}
    We have 
        \begin{equation*}
        \dim_kH^0(X,\Omega_X)^G = 
            \begin{cases}
            g_Y & \mbox{if } s=0, \\
            g_Y-1+\sum_{Q\in S}\left\lfloor \frac{\delta_Q}{e_Q} \right\rfloor & \mbox{otherwise}.
            \end{cases}
        \end{equation*}
    \end{cor}
    \begin{proof}
    We have
        \[
        \deg\left\lfloor\frac{\pi_*(R)}{n} \right\rfloor = \sum_{Q\in Y}\left\lfloor\sum_{P\mapsto Q} \frac{\delta_P}{n} \right\rfloor = \sum_{Q\in Y} \left\lfloor \frac{\delta_Q}{e_Q} \right\rfloor.
        \]
    Furthermore we have $\left\lfloor \frac{\delta_Q}{e_Q} \right\rfloor = 0$ if and only if $\delta_Q<e_Q$, \ie if and only if $Q\notin S$. 
    Thus Corollary \ref{dim2} follows from Proposition \ref{dim}.
    \end{proof}

    \begin{rem}
    Note that if $p>0$ and $G$ is cyclic then Corollary \ref{dim2} can be derived from Proposition $6$ in the recent pre-print \cite{kako} by Karanikolopoulos and Kontogeorgis.\todo{check to see if preprint is now published}
    \end{rem}



\section{Trivial action in the cyclic case}

In this section we will look at the case where $G$ is a cyclic group of prime order, or a power of a prime, and determine when $G$ acts trivially on $H^0(X,\Omega_X^{\otimes m})$.
Compared to arbitrary groups, it is considerably easier to compute when these groups act trivially, and we will later see that we can reduce to this case, regardless of what the structure of $G$ is.


Throughout this section, $P_1,\ldots ,P_r \in X$ denote the ramification points of $\pi$ and we write $e_i$ and $\delta_i$ for $e_{P_i}$ and $\delta_P{_i}$.
Also, for $i=1, \ldots, r$, we define $N_i \in \NN$ by $\ord_{P_i}(\sigma(\pi_i) - \pi_i) = N_i +1$, where $\pi_i$ is a local parameter at the ramification point $P_i$ and $\sigma$ is a generator of $G(P_i)$. 
We also assume that $g_X \geq 2$.


    \begin{prop}\label{m=1}
    Let $p  > 0$ and let $G$ be cyclic of order $p$.
    Furthermore, we assume that $g_Y=0$.
    Then $G$ acts trivially on $H^0(X,\Omega_X)$ if and only if $p=2$. 
    \end{prop}
    \begin{proof}
    From \cite[Lem. 1]{Naka} we know that $p$ does not divide $N_i$ for $i=1,\ldots ,r$, a fact we will use several times below. \todo{more specific citation}
    Let $N:= \sum_{i=1}^r N_i$. 
    Using the Riemann-Hurwitz formula, Corollary \ref{corhurwitzformula}, we obtain
        \begin{equation}\label{hur2}
        2g_X - 2 = -2p + (N+r)(p-1)
        \end{equation}
    and hence
        \[
        \dim_kH^0(X,\Omega_X) = g_X =\frac{(N+r-2)(p-1)}{2}.
        \] 
    Since $g_X \ge 0$ we obtain $r \ge 1$; that is, $\pi$ is not unramified. 
    As $\cha(k) = p = \ord(G)$, the morphism $\pi$ is not tamely ramified, and the cardinality $s$ defined before Corollary \ref{dim2} is not zero.
    Therefore we have 
        \[
        \deg \left\lfloor \frac{\pi_*(R)}{p} \right\rfloor =
        \sum_{i=1}^r \left\lfloor \frac{(N_i+1)(p-1)}{p}\right\rfloor 
        \ge \sum_{i=1}^r \left\lfloor \frac{2(p-1)}{p}\right\rfloor = r > 0.
        \] 
    From Corollary \ref{dim2} we then conclude that 
        \begin{align*}
        \dim_kH^0\left(X,\Omega_X\right)^G & =  g_Y - 1 + \sum_{i=1}^r\left\lfloor \frac{\delta_i}{e_i}\right\rfloor \\
        & =  -1 + N + r \sum_{i=1}^r\left\lfloor -\frac{N_i+1}{p}\right\rfloor.
        \end{align*}
    
    If $p=2$, the dimension of both $H^0(X,\Omega_X)$ and $H^0(X,\Omega_X)^G$ is therefore equal to $\frac{N+r-2}{2}$. 
    This shows the if-direction in Proposition \ref{m=1}.
    
    
    
    To prove the other direction we now assume that $G$ acts trivially on $H^0(X, \Omega_X)$.
    For each $i=1, \ldots, r$, we write $N_i = s_i p +t_i$ with $s_i \in \NN$ and $t_i \in \{1, \ldots, p-1\}$. 
    We furthermore put $S:=\sum_{i=1}^r s_i$ and $T:= \sum_{i=1}^r t_i \ge r$. 
    Then we have
        \[ 
        \frac{(N+r-2)(p-1)}{2} =\dim_kH^0(X,\Omega_X)  = \dim_k H^0(X,\Omega_X)^G = N-S-1 .
        \]
    Rearranging this equation we obtain
        \[
        (3-p)N - 2S = (r-2)(p-1) +2  
        \]
    and hence
        \[
        (-p^2 + 3p -2)S = (r-2)(p-1) +2 - (3-p)T.
        \]
    Assuming that $p \ge 3$ this equation implies that
        \[ 
        S = \frac{(r-2)(1-p)-2 + T (3-p)}{(p-1)(p-2)}. 
        \]
    since $-p^2+3p-2 = - (p-1)(p-2)$. 
    
    Because $S \geq 0$, the numerator of this fraction is non-negative, that is
        \begin{align*}
        \lefteqn{0 \le (r-2)(1-p) - 2 + T (3-p)}\\
        &\le  (r-2)(1-p) - 2 + r (3-p)\\
        &= 2 (r-1)(2-p).
        \end{align*}
    Hence we have that $r=1$ and that the numerator is $0$. 
    We conclude that $S=0$ and hence that $T=1$ or $p=3$. 
    If $T=1$ we also have $N=1$ and finally
        \[
        g_X = \frac{(N+r-2)(p-1)}{2} = 0,
        \]
    a contradiction.
    If $T \not=1$ and $p=3$ we obtain $N=T=2$ and finally 
        \[
        g_X = \frac{(N+r-2)(p-1)}{2} =1,
        \] 
    again a contradiction.
    \end{proof}

    \begin{prop}\label{triv}
    Let $m \geq 2$. 
    Suppose that $G$ is a cyclic group of prime order $l$ (which may or may not be equal to $p$) and that $g_Y=0$. 
    Then $G$ acts trivially on $H^0(X,\Omega_X^{\otimes m})$ if and only if $g_X=m=l=2$.
    \end{prop}
    \begin{proof}
    We have different proofs according to whether or not the order $l$ of the group is the same as the characteristic $p$ of the field.
    
    
    First we assume that $l=p$. 
    As in the proof of Proposition \ref{m=1}, we let $N=\sum_{i=1}^r N_i$, and we let $M=N+r$.
    Then due to (\ref{hur2}) we have
        \begin{equation}\label{simplehur}
        2g_X-2=-2p+M(p-1),
        \end{equation}
    and combining this with Lemma \ref{dim3} we can write
        \begin{equation}\label{altdim2}
        \dim_kH^0(X,\Omega_X^{\otimes m})=(2m-1)(g_X-1)=(2m-1)\left(\frac{M(p-1)-2p}{2}\right).
        \end{equation}
    
    Furthermore, we have
        \begin{equation}\label{altdim}
        \deg\left\lfloor \frac{m\pi_*(R)}{p} \right\rfloor = \sum_{i=1}^r\left\lfloor \frac{m(N_i+1)(p-1)}{p} \right\rfloor  = mM + \sum_{i=1}^r\left\lfloor \frac{-m(N_i+1)}{p} \right\rfloor.
        \end{equation}
    If we have $p=g_X=m=2$, then on the one hand we see that $\dim_kH^0(X,\Omega_X^{\otimes m}) =3$. 
    On the other hand, we first note that \eqref{simplehur} implies $M=6$.
    So 
        \begin{equation*}
        \deg\left\lfloor \frac{m\pi_*(R)}{p}\right\rfloor = 2M -M =6 > 3 = 2m-1.
        \end{equation*}  
    Then, by Proposition \ref{dim}, we obtain 
        \begin{equation*}
        \dim_kH^0(X,\Omega_X^{\otimes m})^G = (2m-1)(g_Y-1)+\deg\left\lfloor \frac{m\pi_*(R)}{p} \right\rfloor = -3 + 6 = 3.
        \end{equation*}
    So the two dimensions are equal and the action of $G$ on $H^0(X,\Omega_X^{\otimes m})$ is trivial. 
    This completes the if direction of the proof.
    
    Now we assume that the action is trivial. This first implies that 
    $\deg \left\lfloor\frac{m\pi_*(R)}{p}\right\rfloor \geq 2m-1$ because otherwise we would 
    have $\dim_kH^0(X,\Omega_X^{\otimes m})^G=0$ by Proposition \ref{dim}, but we know that 
    $\dim_kH^0(X,\Omega_X^{\otimes m})=(2m-1)(g_X-1)$ is strictly positive.\todo{rewrite sentence}
    So, using \eqref{altdim}, \eqref{altdim2} and Proposition \ref{dim} we see that
        \begin{align}\label{bound}
        \lefteqn{(2m-1)\frac{M(p-1)-2p}{2} = \dim_kH^0(X,\Omega_X^{\otimes m})} \nonumber\\
        & =  \dim_kH^0(X,\Omega_X^{\otimes m})^G \nonumber\\
        & =  1-2m+mM+\sum_{i=1}^r\left\lfloor\frac{-m(N_i+1)}{p}\right\rfloor\nonumber \\
        & \leq  1-2m+mM+\sum_{i=1}^r\frac{-m(N_i+1)}{p}\nonumber \\
        & =  1-2m+mM-\frac{mM}{p}.
        \end{align}
    
    After multiplying by $2p$ and rearranging we obtain
        \begin{align}\label{times2p}
        0 & \geq  (2mM-M-4m+2)p^2+(-4mM+M-2+4m)p+2mM \nonumber \\
            & =  (M-2)(2m-1)p^2-((M-2)(2m-1)+2mM)p+2mM \nonumber \\
        & =  (p-1)((M-2)(2m-1)p-2mM).
        \end{align}
    
    Furthermore from \eqref{hur2} we obtain that $-2p+M(p-1)=2g_X-2 \geq 2$ and hence that 
        \begin{equation}\label{greater2}
        M\geq \frac{2+2p}{p-1}=2+\frac{4}{p-1}>2.
        \end{equation}
    
    So from \eqref{times2p} and \eqref{greater2} we see that
        \begin{align}\label{plessthan4}
        p & \leq  \frac{2mM}{(M-2)(2m-1)}\nonumber\\
        & =  \frac{M}{M-2}\cdot\frac{2m}{2m-1}\nonumber\\
        & =  \left( 1+\frac{2}{M-2} \right) \left(1+\frac{1}{2m-1} \right)\\
        & \leq  4, \nonumber	
        \end{align}
    \ie $p=2$ or $p=3$. 
    
    Suppose that $p=3$. Then from \eqref{greater2} we have $M\geq 4$. However, from  \eqref{plessthan4} we also have that 
        \begin{align*}
        3 & \leq \left( 1+\frac{2}{M-2} \right) \left(1+\frac{1}{2m-1} \right)\\
        & \leq  \left( 1+\frac{2}{M-2} \right) \frac{4}{3}\\
        & \leq  \frac{8}{3},
        \end{align*}
    a contradiction.
    
    Lastly, we come to the case when $p=2$. From \eqref{plessthan4} we see that $2\leq \left(1+\frac{2}{M-2}\right)\frac{4}{3}$ 
    and hence $M\leq 6$. However, from \eqref{greater2} we know that $M\geq 6$, so $M=6$. Then from \eqref{bound}  we obtain that $2m-1=1-2m+6m-3m$
    and hence that $m=2$. Finally, (\ref{hur2}) gives us that $2g_X-2=-4+6=2$ and hence $g_X=2$. 
    This completes the only if direction of the proof when $l=p$.
    
    Now if $l\neq p$ then we know that all the coefficients $\delta_i$ of the ramification divisor are equal to $l-1$. 
    To show the if direction in this case, first note that $\dim_kH^0(X,\Omega_X^{\otimes m})=3$ by Lemma~\ref{dim3}. 
    On the other hand, the Riemann-Hurwitz formula (Corollary \ref{corhurwitzformula}) implies that $2 = 2g_X-2=-2l+\deg(R)=-2l+r(l-1)$, and hence that $r=6$. 
    Finally Proposition \ref{dim} gives us
        \begin{equation*}
        \dim_kH^0(X,\Omega_X^{\otimes m})^G = -(2m-1) + \sum_{i=1}^r \left\lfloor \frac{m\cdot \delta_i}{l} \right\rfloor
        = -3 +\sum_{i=1}^6 \left\lfloor \frac{m(l-1)}{l} \right\rfloor
        = 3,
        \end{equation*}
    since $m=l=2$.
    As the dimensions of $H^0(X,\Omega_X^{\otimes m})$ and $H^0(X,\Omega_X^{\otimes m})^G$ are equal, the action is trivial.
    
    
    Now, for the final section of the proof we suppose that $G$ acts trivially on the space $H^0(X,\Omega_X^{\otimes m})$.
    We then show that this implies that $g_X=l=m=2$.
    
    
    From Lemma \ref{dim3} and Proposition~\ref{dim} we obtain
        \begin{align*}
        \lefteqn{(2m-1)(g_X-1)=\dim_kH^0(X,\Omega_X^{\otimes m})} \\
        & =  \dim_kH^0(X,\Omega_X^{\otimes m})^G=-(2m-1)+\sum_{i=1}^r \left\lfloor \frac{m\cdot \delta_i}{l} \right\rfloor
        \end{align*}
    and hence
        \begin{equation*}
        (2m-1)g_X = \sum_{i=1}^r \left\lfloor \frac{m\cdot \delta_i}{l} \right\rfloor
        = \sum_{i=1}^r \left\lfloor \frac{m(l-1)}{l} \right\rfloor
        = r\left( m+\left\lfloor \frac{-m}{l} \right\rfloor \right).
        \end{equation*}
    By choosing $s\in \{1,\ldots ,l\}$ and $q\in \mathbb{N}$ such that $m=ql+s$ we can rewrite this as
        \begin{equation}\label{eq:mult}
        (2m-1)g_X=r(m-q-1).
        \end{equation}
    If we multiply (\ref{eq:mult}) by $l-1$ and then substitute in for the $r(l-1)$ term in the Riemann-Hurwitz formula (Corollary \ref{corhurwitzformula}) we get
        \begin{equation*}
        (2m-1)(l-1)g_X=(2g_X+2(l-1))(m-q-1).
        \end{equation*}
    By rearranging we are able to compute $g_X$ in terms of $m,l$ and $q$:
        \begin{align}\label{equationgxintermsofmandlandq}
        \lefteqn{g_X = \frac{2(l-1)(m-q-1)}{(2m-1)(l-1)-2(m-q-1)}} \nonumber \\
        & =  1 + \frac{2(m-q-1)-(2q+1)(l-1)}{(2m-1)(l-1)-2(m-q-1)}  \nonumber\\
        & =  1 + \frac{2s-1-l}{(2m-1)(l-1)-2(m-q-1)} \nonumber  \\
        & =  1 + \frac{2(s-1)+1-l}{(2m-1-2q)(l-1)-2(s-1)}. 
        \end{align}
    First, we show that if $l\geq 3$ the equation cannot hold whilst $g_X\geq 2$.
    Observe that the denominator is strictly greater than $l-1$, remembering that $m=ql+s$:
        \begin{align*}
        (2m-1-2q)(l-1)-2(s-1) & =  ((2q(l-1)+2s-1)(l-1)-2(s-1) \\
        & \geq  (2s-1)(l-1)-2(s-1) \\
        & \geq  (2s-1)(l-1)-2(s-1)(l-1) \\
        & =  l-1;
        \end{align*}
    here the two inequalities are equalities if and only if $q=0$ and $s=1$, respectively, and, as $m\geq 2$, not both inequalities can be equalities.
    Now the numerator is at most $l-1$, occurring when $s=l$. 
    Hence if $l\geq 3$ the fraction in \eqref{equationgxintermsofmandlandq} will be less than one and $g_X < 2$, contradicting our assumption.
    If $l=2$, then $s$ is either 1 or 2.
    If $s=1$ the fraction is negative, and $g_X<1$, which again contradicts our assumption.
    Finally, if $s=2$ then $g_X\leq 2$, with equality if and only if $q=0$, \ie~if and only if $m=2$.
    So if $g_X \geq 2$ then the action being trivial implies that $g_X=l=m=2$, and the proof is complete.    
    \end{proof}

For the rest of this section we assume that $p>0$ and that $G$ is a cyclic group of order $p^l$ for some $l \in \NN$.
What we are now going to do will not be used in the proof of the main theorem, but is included because it generalises the previous results.
More precisely, we do not restrict ourselves to looking at $H^0(X,\Omega_X^{\otimes m})$, but using a comparatively deep result from \cite{kako} we study $H^0(X,\cO(D))$ for any $G$-invariant divisor $D$ such that $\deg(D)>2g_X-2$.


We first introduce some notation.
Let $D = \sum_{P\in X} n_P[P]$ be a $G$-invariant divisor on $X$.
Then let $\langle a \rangle$ denote the fractional part of any $a\in \mathbb{R}$, \ie $\langle a \rangle = a - \lfloor a \rfloor$.
Also, for any $Q\in Y$ let $n_Q$ be equal to $n_P$ for any $P\in \pi^{-1}(Q)$.




    \begin{prop}\label{nakaj}
    Suppose $p>0$ and $G$ is a cyclic group of order $p^l$ for some $l\geq 1$.
    Let $D$ be a $G$-invariant divisor on $X$ such that $\deg(D)>2g_X-2$.
    Then the action of~$G$ on $H^0(X,\cO_X(D))$ is trivial if and only if
        \[ 
        (p^l-1)\deg(D)=p^l\left(g_X-g_Y-\sum_{Q\in Y}\left\langle \frac{n_Q}{e_Q} \right\rangle\right).
        \]
    \end{prop}
    \begin{proof}
    We first remind the reader of the notation in \cite{kako}.
    Let $\sigma$ be a generator of $G$.
    Let $V$ be the $k[G]$ module with $k$-basis $e_1,\ldots ,e_{p^l}$ and $G$-action defined by $\sigma( e_i)=e_i+e_{i-1}$, $1\leq i \leq p^l,\ e_0=0$.\todo{changed from $\sigma \cdot e_i$ to how is now. Check rest of work for issues}
    Then $V_j$, defined to be the subspace of $V$ spanned by $e_1,\ldots ,e_j$ over $k$, is also a $k[G]$ module.
    In fact, the modules $V_1,\ldots ,V_{p^l}$ form a complete set of representatives for the set of isomorphism classes of indecomposable $k[G]$-modules. For each $j=1,\ldots,p^l$ let $m_j$ denote the multiplicity of $V_j$ in the $k[G]$-module $H^0(X,\cO_x(D))$, \ie we have $H^0(X,\cO_x(D))\cong \oplus_{j=1}^{p^l}m_jV_j$.
    
    
    
    First note that the action of $G$ on $H^0(X,\cO_X(D))$ is trivial if and only if
        \begin{equation}\label{triva}
        \dim_k H^0(X,\cO_X(D))^G =\dim_k H^0(X,\cO_X(D)).
        \end{equation}
    
    It is clear that the $G$-invariant part of each submodule $V_j$ is spanned by $e_1$. 
    Hence $\dim_kH^0(X,\cO_X(D))^G = \sum_{j=1}^{p^l} m_j$.
    By \cite[Thm. 2.1]{quaddiffequi}, which relies on \cite{cohogsheaves}, we have
        \begin{align*}
        \sum_{j=1}^{p^l} m_j & =  1- g_Y +\sum_{Q\in Y} \left\lfloor \frac{n_Q}{e_Q}\right\rfloor\\
        & =  1- g_Y + \sum_{Q\in Y} \left( \frac{n_Q}{e_Q} - \left\langle \frac{n_Q}{e_Q}\right\rangle \right) \\
        & =  1 - g_Y + \frac{1}{p^l}\deg(D) - \sum_{Q\in Y} \left\langle \frac{n_Q}{e_Q} \right\rangle.
        \end{align*}
    
    Now as $\deg(D)>2g_X-2$ we have $\dim_kH^0(X,\cO_X(D)) =\deg(D)+1-g_X$ by the Riemann-Roch theorem. 
    So the action of $G$ on $H^0(X,\cO_X(D))$ is trivial if and only if
        \begin{equation*}
        \deg(D)+1-g_X  = 1 - g_Y + \frac{1}{p^l}\deg(D) - \sum_{Q\in Y}\left\langle \frac{n_Q}{e_Q} \right\rangle. \label{hi}
        \end{equation*}
    
    This then rearranges to $(p^l-1)\deg(D)=p^l\left(g_X-g_Y-\sum_{Q\in Y}\left\langle \frac{n_Q}{e_Q} \right\rangle\right)$, as desired.
    \end{proof}

    \begin{cor}\label{this}
    Suppose that $\deg(D)\geq 2g_X$. Then the action of $G$ on $H^0(X,\cO_X(D))$ is trivial if and 
    only if $g_Y = 0$, $e_Q | n_Q$ for all $Q\in Y$, $\deg(D)=2g_X$ and either $g_X=0$ or $p^l=2$.
    \end{cor}
    \begin{proof}
    The following inequalities always hold under the stated assumptions:
        \begin{multline}
        (p^l-1)\deg(D)\geq (p^l-1)2g_X \geq p^lg_X \geq p^lg_X-p^l\sum_{Q\in Y}\left\langle\frac{n_Q}{e_Q}\right\rangle \\ \geq p^l\left( g_X - g_Y -\sum_{Q\in Y}\left\langle \frac{n_Q}{e_Q} \right\rangle \right).
        \end{multline}
    Now the first inequality is an equality if and only if $\deg(D)=2g_X$. 
    The second is an equality if and only if either $g_X=0$ or $p^l=2$. 
    The third inequality is an equality if and only if $\sum_{Q\in Y}\left\langle\frac{n_Q}{e_Q}\right\rangle=0$, which is the case if and only if each $n_Q$ is divisible by~$e_Q$. 
    Lastly, the fourth inequality is an equality if and only if $g_Y = 0$.
    Given these observations, Proposition \ref{nakaj} implies Corollary~\ref{this}.
    \end{proof}

The following Corollary slightly strengthens the only if direction of the $l=p$ part of Proposition \ref{triv}
(from $\ord(G) = p$ to $\ord(G) = p^l$) and also provides a different proof for it;
note that this new proof relies on the comparatively deep result result in section 7 of \cite{cohogsheaves}.


    \begin{cor}
    Let $m \geq 2$ and let $G$ be a cyclic group of order $p^l$ for some $l$. 
    If $G$ acts trivially on $H^0(X,\Omega_X^{\otimes m})$, then $g_Y = 0$ and $p^l = g_X = m = 2$.
    \end{cor}
    \begin{proof}
    As $g_X \geq 2$ and $m\geq 2$ we have $\deg(mK_X) \geq 2g_X$. 
    So, if the action of $G$ on $H^0(X,\Omega_X^{\otimes m})$ is trivial, we obtain from Corollary \ref{this} that $\deg(mK_X) = 2g_X$, $p^l = 2$ and $g_y = 0$.
    Now $\deg (mK_X) = 2g_X$ implies that $m(2g_X -2 ) = 2g_X$, so $m(g_X -1) = g_X$ and hence $m=g_X=2$.
    \end{proof}

Similarly to the case $\deg(D)\geq 2g_X$ in Corollary \ref{this}, the following corollary derives necessary and sufficient conditions for trivial action from Proposition \ref{nakaj} in the case $\deg(D) =2g_X-1$.



    \begin{cor}
    Suppose that $\deg(D)= 2g_X-1$ and that $g_Y=0$. Then the action of $G$ on $H^0(X,\cO_X(D))$ is trivial if and only if one of the following conditions hold:
        \begin{itemize}
        \item  $p^l=2$ and $\sum_{Q\in Y}\left\langle\frac{n_Q}{e_Q}\right\rangle=\frac{1}{2}$;
        \item  $g_X=2$, $p^l=3$ and $e_Q\mid n_Q$ for all $Q\in Y$.
        \end{itemize}
    \end{cor}


    \begin{rem}
    It can easily be shown that in the last case the Riemann-Hurwitz formula implies that $r\leq 4$. 
    Furthermore, if $r=1$ then the conditions ``$\sum_{Q\in Y}\left\langle\frac{n_Q}{e_Q}\right\rangle=\frac{1}{p^l}$" and ``$e_Q\mid n_Q$ for all $Q\in Y$" are already implied by ``$\deg(D)=2g_X-1$".
    \end{rem}

    \begin{proof}
    Firstly, if $g_X=0$ then $\deg(D)=-1<0$, so $\dim_kH^0(X,\cO_X(D))=0$ and the action is trivial.
    
    Now note that, as $\deg(D)=2g_X-1$, we conclude from Proposition \ref{nakaj} that the action is trivial if and only if 
        \begin{equation*}
        (p^l-1)(2g_X-1)=p^l\left(g_X-\sum_{Q\in Y}\left\langle\frac{n_Q}{e_Q}\right\rangle\right).
        \end{equation*}
    If $p^l=2$ then this is equivalent to $2g_X-1=2g_X-2\sum_{Q\in Y}\left\langle\frac{n_Q}{e_Q}\right\rangle$ and hence to $\sum_{Q\in Y}\left\langle\frac{n_Q}{e_Q}\right\rangle=\frac{1}{2}$.
    
    If $g_X=1$ then this is equivalent to $p^l-1=p^l-p^l\sum_{Q\in Y}\left\langle\frac{n_Q}{e_Q}\right\rangle$ and hence is also equivalent to $\sum_{Q\in Y}\left\langle\frac{n_Q}{e_Q}\right\rangle=\frac{1}{p^l}$.
    
    Lastly, if $p^l\geq 3$ and $g_X\geq 2$ then we have that $g_X\geq \frac{p^l-1}{p^l-2}$ which is equivalent to the first inequality in the chain
        \begin{equation*}
        (p^l-1)(2g_X-1)\geq p^lg_X\geq p^lg_X-p^l\sum_{Q\in Y}\left\langle\frac{n_Q}{e_Q}\right\rangle \geq p^l\left( g_X - g_Y -\sum_{Q\in Y} \left\langle \frac{n_Q}{e_Q} \right\rangle \right).
        \end{equation*}
    Hence the action is trivial if and only if both inequalities are equalities, which is the case if and only if $p^l=3,\ g_X=2$, $e_Q\mid n_Q$ for all $Q\in Y$ and $g_Y = 0$.
    \end{proof}


\section{The main theorem}\label{maintheoremsection}
In this section we prove the main theorem of this chapter, describing exactly when $G$ will act faithfully on $H^0(X,\Omega_X^{\otimes m})$.


    \begin{thm}\label{theoremfaithfulaction}
    Suppose that $g_X\geq 2$ and let $m\geq1$. 
    Then $G$ does not act faithfully on $H^0(X,\Omega_X^{\otimes m})$ if and only if $G$ contains a hyperelliptic involution and one of the following two sets of conditions holds:
        \begin{itemize}
        \item $m=1$ and $p=2$;
        \item $m=2$ and $g_X=2$.
        \end{itemize}
    \end{thm}
    \begin{proof}
    We first show the if direction. 
    In the case when $m=1$, the hyperelliptic involution contained in $G$ generates a subgroup of order $2$.
    Since $p=2$, this acts trivially by Proposition \ref{m=1}, and hence $G$ does not act faithfully.
    In the case when $m=2$, then again looking at the subgroup generated by the hyperelliptic involution, we have a group of order $2$ acting on $H^0(X,\Omega_X^{\otimes m})$.
    So, by Proposition \ref{triv} and since $g_X=m=2$, the action of this subgroup is trivial, and again, this means that $G$ does not act faithfully.
    
    
    We now start the proof of the only if direction, supposing that $G$ does not act faithfully on $H^0(X,\Omega_X^{\otimes m})$. 
    By replacing $G$ with the (non-trivial) kernel $H$ if necessary, we may assume that $G$ is non-trivial and acts trivially on $H^0(X,\Omega_X^{\otimes m})$.
    
    
    We start the proof by showing that $g_Y=0$, which is shown separately for the cases when $m=1$ and when $m\geq 2$.
    In the case when $m=1$ we start by showing that $\deg  \left\lfloor \frac {\pi_*(R)}{n} \right\rfloor >0$ by contradiction.
    Suppose that $\deg\left\lfloor \frac{\pi_*(R)}{n} \right\rfloor =0$.
    As $G$ acts trivially it follows from Proposition~\ref{dim} that:
        \begin{equation*}
        g_X=\dim_k H^0(X,\Omega_X)=\dim_k H^0(X,\Omega_X)^G=g_Y.
        \end{equation*}
    Substituting this into the Riemann-Hurwitz formula (Corollary \ref{corhurwitzformula}) yields the desired contradiction because $g_X\geq 2, n\geq 2$ and $\deg(R)\geq 0$.
    
    Thus $\deg\left( \left\lfloor \frac{\pi_*(R)}{n} \right\rfloor \right) >0$. 
    Now Proposition~\ref{dim} gives us that
        \begin{equation*}
        g_X=\dim_k H^0(X,\Omega_X)=\dim_k H^0(X,\Omega_X)^G= g_Y-1+\deg\left\lfloor \frac{\pi_*(R)}{n} \right\rfloor.
        \end{equation*}
    Substituting this in to the Riemann-Hurwitz formula we see that
        \begin{equation*}
        2\left(g_Y - 1 + \deg\left \lfloor \frac{\pi_*(R)}{n} \right \rfloor -1 \right) = 2n (g_Y -1) + \deg(R).
        \end{equation*}
    For any $Q \in Y$ we let $\delta_Q$ denote the coefficient of the ramification divisor $R$ at any $P \in \pi^{-1}(Q)$ and let $e_Q := e_P$ for any $P \in \pi^{-1}(Q)$. 
    Rewriting the previous equation then yields
        \begin{align*}
        \lefteqn{(2n-2)g_Y = 2n-4 + 2 \,\deg\left \lfloor \frac{\pi_*(R)}{n}\right \rfloor - \deg(R)}\\
        &= 2 \left(n-2 + \sum_{Q \in Y} \left(\left\lfloor \frac{n}{e_Q} \frac{\delta_Q}{n} \right\rfloor - \frac{n}{e_Q} \frac{\delta_Q}{2}\right) \right)\\
        &= 2 \left(n-2 + \sum_{Q \in Y} \left( \left\lfloor \frac{\delta_Q}{e_Q} \right\rfloor - \frac{\delta_Q}{e_Q} \frac{n}{2} \right)\right)\\
        & \le  2(n-2),
        \end{align*}
    because $\frac{n}{2} \ge 1$ and $\left\lfloor \frac{\delta_Q}{e_Q}\right\rfloor \le \frac{\delta_Q}{e_Q}$ for all $Q \in Y$. 
    Hence we obtain $g_Y \le \frac{n-2}{n-1} < 1$ and therefore $g_Y =0$, as desired.
    
    We now show that $g_Y=0$ when $m\geq 2$. 
    Since $g_X\geq 2$ we have that $\deg(mK_X)=m(2g_X-2)>2g_X-2=\deg(K_X)$.
    By Lemma \ref{dim3}, and as both $m$ and $g_X$ are at least 2, then $\dim_kH^0(X,\Omega_X^{\otimes m})^G=\dim_kH^0(X,\Omega_X^{\otimes m})=(2m-1)(g_X-1)>1$.
    There is only one case in Proposition \ref{dim} such that $m\geq 2$ and $\dim_k H^0(X,\Omega_X^{\otimes m})^G>1$, which yields 
        \begin{equation*}
        (2m-1)(g_X-1)=(2m-1)(g_Y-1)+\deg\left(\left\lfloor \frac{m\pi_*(R)}{n} \right\rfloor \right).
        \end{equation*}
    Combining this with the Riemann-Hurwitz formula, Corollary \ref{corhurwitzformula}, we see that
        \begin{align*}
        2(2m-1)(g_Y-1)+2\deg\left(\left\lfloor\frac{m\pi_*(R)}{n}\right\rfloor\right) & =  2(2m-1)(g_X-1)\\
        & =  2n(2m-1)(g_Y-1)+(2m-1)\deg(R),
        \end{align*}
    which can be re-arranged as
        \begin{equation*}
        (2m-1)(2n-2)(g_Y-1)=2\deg\left(\left\lfloor\frac{m\pi_*(R)}{n}\right\rfloor\right)-(2m-1)\deg(R).
        \end{equation*}
    So if we can show that the right hand side of this equation is negative then we will have $g_Y-1<0$ and hence $g_Y=0$, as desired.
    
    Using the same notation as in the case when $m=1$, we calculate:
        \begin{align*}
        2\deg\left(\left\lfloor\frac{m\pi_*(R)}{n}\right\rfloor\right)-(2m-1)\deg(R) & = \sum_{Q \in Y} \left(2\left\lfloor m\cdot \frac{n}{e_Q}\frac{\delta_Q}{n}\right\rfloor -n(2m-1)\frac{\delta_Q}{e_Q}\right) \\
        & \leq   \sum_{Q\in Y}\left( 2m\cdot\frac{\delta_Q}{e_Q}-n(2m-1)\frac{\delta_Q}{e_Q}\right) \\
        & =  (2m-n(2m-1))\sum_{Q\in Y }\frac{\delta_Q}{e_Q}.
        \end{align*}
    
    Now as $n,m\geq 2$ then we have $2m-n(2m-1)\leq 2m-2(2m-1)=2(1-m)<0$ and we are done as $\sum_{Q\in Y}\frac{\delta_Q}{e_Q}$ is positive.
    
    So we have shown for all $m\geq 1$, if the group $G$ acts trivially  on $H^0(X,\Omega_X^{\otimes m})$ then $g_Y=0$.
    Now if $m\geq 2$ then first note that $G$ must contain a cyclic subgroup of prime order, say $H$, such that $H$ acts trivially on $H^0(X,\Omega_X^{\otimes m})$.
    Now Proposition \ref{triv} tells us that $m=g_X=2$, and that the order of $H$ must also be 2.
    Hence $X/H\cong \mathbb{P}_k^1$, and this completes the only if direction for $m\geq 2$.
    
    Similarly, the $m=1$ case of the only if direction will follow from Proposition \ref{m=1} after we show that $p>0$ and there is a cyclic subgroup of $G$ of order $p$. 
    This is true since $\pi$ cannot be tamely ramified.
    Indeed, if it were then $R=\sum_{P\in X} (e_P-1)[P]$ \cite[Chap. IV, Cor. 2.4]{hart}, and $\deg\left\lfloor \frac{\pi_*(R)}{n} \right\rfloor=0$, which we have already shown cannot be the case.
    Hence $p$ must be positive, and there is a cyclic subgroup of order $p$ which acts trivially.
    \end{proof}

    \begin{rem}
    Note that the existence of a hyperelliptic involution $\sigma$ in $G$ means not only that the genus of $X/\langle \sigma \rangle$, but also the genus of $Y=X/G$, is $0$ (by the Riemann-Hurwitz formula).
    If, moreover $p=2$, then the canonical projection $X\rightarrow X/\langle \sigma \rangle$ is not unramified (again by the Riemann-Hurwitz formula) and hence not tamely ramified; then $\pi$ cannot be tamely ramified either.
    \end{rem}


\section{Examples}
We will now give some examples of a finite group acting on a curve, and the consequent action on the holomorphic poly-differentials. 
We start with some examples in which $G$ acts trivially on $H^0(X,\Omega_X^{\otimes m})$.
We then follow this with the example of hyperelliptic curves, for which we compute an explicit basis of $H^0(X,\Omega_X^{\otimes m})$, allowing us to see when the action is trivial.


\subsection{Trivial Examples}\label{examplessection}


(a) Let $g_X = 0$, \ie $X\cong \mathbb P_k^1$.
Then $\deg(K_X) = -2$ and so $\deg(mK_X) < 0$ for $m~\geq~1$.
Hence $H^0(X,\Omega_X^{\otimes m}) =\{0\}$ by \cite[Lem. 2, pg. 295]{hart}\todo{check citation} and $G$ acts trivially on $H^0(X,\Omega_X^{\otimes m})$ for all $m\geq 1$.

(b) Let $g_X = 1$, \ie $X$ is an elliptic curve.
If $G$ is a finite subgroup of $X(k)$ acting on $X$ by translations, then $G$ leaves invariant any global non-vanishing holomorphic differential $\omega$ and hence $G$ acts trivially on $H^0(X,\Omega_X)$;
since $\omega^{\otimes m}$ is a basis of $H^0(X,\Omega_X^{\otimes m})$ this means that $G$ acts trivially on $H^0(X,\Omega_X^{\otimes m})$ for all $m\geq 1$.

If $p>0$ and $G$ is a $p$-group, then the multiplicative character $G\rightarrow k^*$ afforded by the one-dimensional representation $H^0(X,\Omega_X^{\otimes m})$ of $G$ has to be trivial because $k$ doesn't contain any $p^{\mbox{th}}$ roots of unity;
in particular the action of $G$ on $H^0(X,\Omega_X^{\otimes m})$ is trivial as well.
On the other hand, if $p\neq 2$ and $X$ is given by the Weierstrass equation of the form $y^2 = f(x)$, then the involution $\sigma : (x,y) \rightarrow (x,-y)$ maps the invariant differential $\omega = \frac{dx}{y}$ to $-\omega$.





%% !TEX TS-program = pdflatex
% !TEX encoding = UTF-8 Unicode

% This is a simple template for a LaTeX document using the "article" class.
% See "book", "report", "letter" for other types of document.

\documentclass[11pt]{article} % use larger type; default would be 10pt

\usepackage[utf8]{inputenc} % set input encoding (not needed with XeLaTeX)

%%% Examples of Article customizations
% These packages are optional, depending whether you want the features they provide.
% See the LaTeX Companion or other references for full information.

%%% PAGE DIMENSIONS
\usepackage{geometry} % to change the page dimensions
\geometry{a4paper} % or letterpaper (US) or a5paper or....
% \geometry{margins=2in} % for example, change the margins to 2 inches all round
% \geometry{landscape} % set up the page for landscape
%   read geometry.pdf for detailed page layout information

\usepackage{graphicx} % support the \includegraphics command and options

\usepackage[parfill]{parskip} % Activate to begin paragraphs with an empty line rather than an indent

%%% PACKAGES
\usepackage{booktabs} % for much better looking tables
\usepackage{array} % for better arrays (eg matrices) in maths
\usepackage{paralist} % very flexible & customisable lists (eg. enumerate/itemize, etc.)
\usepackage{verbatim} % adds environment for commenting out blocks of text & for better verbatim
\usepackage{subfig} % make it possible to include more than one captioned figure/table in a single float
% These packages are all incorporated in the memoir class to one degree or another...

%%% HEADERS & FOOTERS
\usepackage{fancyhdr} % This should be set AFTER setting up the page geometry
\pagestyle{fancy} % options: empty , plain , fancy
\renewcommand{\headrulewidth}{0pt} % customise the layout...
\lhead{}\chead{}\rhead{}
\lfoot{}\cfoot{\thepage}\rfoot{}

%%% SECTION TITLE APPEARANCE
\usepackage{sectsty}
\allsectionsfont{\sffamily\mdseries\upshape} % (See the fntguide.pdf for font help)
\usepackage{amsmath}
\usepackage{amsthm}
\usepackage{amsfonts}
\usepackage{mathrsfs}
\usepackage{amsopn}
\usepackage{amssymb}
\usepackage{natbib}
% (This matches ConTeXt defaults)

%%% ToC (table of contents) APPEARANCE
\usepackage[nottoc,notlof,notlot]{tocbibind} % Put the bibliography in the ToC
\usepackage[titles,subfigure]{tocloft} % Alter the style of the Table of Contents
\renewcommand{\cftsecfont}{\rmfamily\mdseries\upshape}
\renewcommand{\cftsecpagefont}{\rmfamily\mdseries\upshape} % No bold!

%Theorems and stuff
\newtheorem{defn}{Definition}
\newtheorem{thm}{Theorem}
\newtheorem{cor}{Corollary}
\newtheorem{lem}{Lemma}
\newtheorem{prop}{Proposition}
\theoremstyle{remark}\newtheorem*{rem}{Remark}

\newcommand{\cO}{{\cal O}}
\newcommand{\ra}{\rightarrow}
\newcommand{\NN}{{\mathbb N}}
\newcommand{\PP}{{\mathbb P}}
\newcommand{\ZZ}{{\mathbb Z}}
\newcommand{\cL}{{\cal L}}

\DeclareMathOperator{\ord}{ord}
\DeclareMathOperator{\di}{div}
\DeclareMathOperator{\cha}{char}
\DeclareMathOperator{\gal}{Gal}


%%% END Article customizations

%%% The "real" document content comes below...

\title{Valentini and Madan}
\author{J Tait}
%\date{} % Activate to display a given date or no date (if empty),
         % otherwise the current date is printed 

\begin{document}
\maketitle
Let $X$ be a smooth connected projective algebraic curve over an algebraically closed field $k$ of characteristic $p>0$, on which the cyclic group $G$ of prime order $p$ acts faithfully.
This induces an action of $G$ on $H^0(X,\Omega_X)$, and we will compute when this action is trivial.
This can be done with the paper of Valentini and Madan when the genus of $X$ is at least $2$.

We start by phrasing the question in the language of the paper; rather than the curve $X$ we will refer to the corresponding function field $F$, and we will write $\Omega_F$ instead of $H^0(X,\Omega_X)$.
The subspace of $F$ fixed by $G$ (equivalently, the quotient of $X$ by the action of $G$) will be denoted $E$, and it's space of holomorphic differentials $\Omega_E$.
Finally, we will denote the genus of $E$ and $F$ by $g_E$ and $g_F$ respectively, and assume that $g_E=0$.

Note that the paper assumes $|G|=p^n$ for some $n \geq 1$, so where appropriate the value of $n$ will be substituted with $1$, after specifying that the substitution has been made.
 
Let $\sigma$ be a generator of $G$. 
There are $p$ unique indecomposable representations of $G$, which can be written $K[G]/(\sigma - 1)^k$, for $k\in \{1,\ldots, p\}$ \textbf{Find citation}.
The trivial action corresponds to $k=1$ and the regular representation corresponds to $k=p$.
For a decomposition of the $G$-representation of $\Omega_F$ in to indecomposable representations, we will denote by $d_k$ the number of times the representation of degree $k$ occurs.
So if $\Omega_F = \sum_{m=1}^t \oplus \Omega_m$ for some $t\leq g_F$ is a decomposition of $\Omega_F$ in to a direct sum of indecomposable sub-modules $\Omega_m$, then $d_k$ is the number of components of the sum isomorphic to $K[G]/(\sigma -1)^k.$
Note that if the action is trivial then the trivial representation will be the only indecomposable sub-module. 
As such, showing that $G$ acts trivially is equivalent to showing that $d_1 = g_F$, and $d_k= 0 $ for $k \geq 1$.
{\em
We start by defining the following subspaces of $\Omega_F$,
\begin{equation*}
	\Omega_F^i = \{\omega\in \Omega_F | (\sigma - 1)^i\omega=0\}\ \mbox{ for } 0\leq i \leq p.
\end{equation*}
This is an increasing sequence of subspaces, from $0$ to $\Omega_F$, and we can see that $\dim_k\Omega_F^i = \sum_{m=1}^t\dim(\Omega_F^i \cap \Omega_m)$, where $\Omega_m$ are the same indecomposable sub-modules as earlier.
Now note that for the indecomposable module $k[G]/(\sigma - 1)^k$ it is true that 
\[
 \dim_k\{v \in k[G]/(\sigma - 1)^k | (\sigma - 1)^iv = 0\} = \left\{ \begin{array}{ll}
                                                                      i & i\leq k \\
\\
								      k & i > k
                                                                     \end{array} \right. \mbox{\textbf{prove this!}}
\]

Therefore $\dim_k\Omega_F^i = \sum _{k=1}^{i-1} kd_k + \sum_{k=i}^p id_k$, and it follows that 
\[
dim_k(\Omega_F^{i+1}/\Omega_F^i) = \sum_{k=i+1}^p d_k.
\]

As we wish to find $d_k$, we rewrite this as 
\begin{eqnarray*}
d_p & = & \dim_k(\Omega_F^p/\Omega_F^{p-1}) \\
d_k & = & \dim_k(\Omega_F^k/\Omega_F^{k-1}) - \dim_k(\Omega_F^{k-1}/\Omega_F^{k-1}).
\end{eqnarray*}
Now to compute $\dim_k(\Omega_F^i/\Omega_F^{i-1})$ we look at the extension $F/E$.

}

The extension $F/E$ is a degree $p$ Artin extension, so it has generation of the form 
\[
F=E(y), \mbox{	} y^p-y = b, \mbox{	} b\in E.
\]

Note that in the paper, there are $n$ intermediary fields, labelled $E_j$, each of which is an Artin-Schreier extension with equivalently labelled $b_j$ and $y_j$.
The $b$ and $y$ we are using correspond to $b_1$ and $y_1$ respectively, and $F$ and $E$ correspond to $E_1$ and $E_0$ respectively.


We can choose $b$ and $y$ to satisfy the conditions of Lemma 5.1 of appendix 5 of \cite{quaddiffequi}, which states:\newline


\begin{lem}\label{koeck}
Let $L/K$ be a totally ramified Galois extension of degree $p$.
Then there exists an element $y \in L$ whose valuation is coprime to $p$ and
negative, say $-m$, such that $y^p - y \in K$ and $L = K(y)$. The greatest
integer $M$ such that the higher ramification group $G_M$ of $L/K$ does not vanish
is then equal to $m$.
\end{lem}


At this point in the paper, for $1\leq k \leq p^n-1$, $a_i^k$ is defined to be the co-efficient of $p^i$ in the p-adic expansion of $k$. 
Since we have $n=1$, we only have $a_0^k = k$ in this expansion, a fact we will later use.

Let $\bar{P_1},\dots , \bar{P_s}$ be the primes of $E$ that ramify, and let $P(i,1,1)$ denote the prime ideal above $\bar{P_i}$, in $F$.
(Note that we are using this notation to be consistent with the paper. 
There, the authors used $P(i,j,m)$, where $i$ denoted the prime ideal in $E$ it was above, $j$ denoted which extension the ideal was in, and $m$ was used to differentiate between these.
We set $j=1$ as $n=1$, as already commented.
Also, since the order of $G$ is prime, we may only have one prime above each $\bar{P_i}$, so $m=1$ too.)


Now for each $\bar{P_i}$ the normalised valuation determined by $\bar{P_i}$ applied to $z\in E$ is denoted $v(i,1,1,z)$.
(Again, note that in the paper the authors use $v(i,j,m,z)$ to denote the valuation of a $z\in E_j$ determined by $P(i,j,m).$
Here however we will only require the valuation in $E$.)
We let $e_i$ be the ramification index of $\bar{P_i}$ and let $r=n-\max{e_i}$.
Since $n=1$, it is clear that $e_i=1$ for all $i$, unless the extension is unramified, and hence $r=0$.
Again, this is not necessarily the case if $n\geq 2$, and but we will use the notation of the paper.

From the proof of Lemma 2 in Valentini and Madan, if we set $\Phi(i,j) = \Phi(i,1) = -v(i,1,1,b)$, then we have the following formula to determine the exponent of the different at $P(i,1,m)$:

\begin{eqnarray}\label{eq}
\delta_i & = & (p-1)\sum_{j=n-e_i+1}^n (\Phi(i,j) + 1)p^{n-j} \\
	    & = & (p-1)(\Phi(i,1) + 1).
\end{eqnarray}

To determine $d_k$ in terms of this, then we set 
\[
v_{ik} = \left\lfloor \frac{\delta_i - k\Phi(i,1)}{p} \right\rfloor,
\]

for $0\leq k \leq p-1$, where $\lfloor c \rfloor$ denotes the largest integer not exceeding $c$ for any $c\in \mathbb{R}$.

For each $k$ we then denote the sum of these values of all the ramification points by $ \Gamma_k = \sum_{i=1}^s v_{ik}.$

Now we can apply Theorem 1 in Valentini and Madan, which states:\\
\begin{thm}
Let $G$ be a cyclic group of automorphisms of $F$ of order $p^n$. 
Let $E$ be the fixed field of $G$ with $g_E$ its genus.
The regular representation of $G$ occurs $g_E-1+\alpha$ times in the representation of $G$ on $\Omega_F$, with $\alpha = 1$ if $r=0$ and $\alpha = 0$ otherwise.
For $k=1,\ldots p^n-1$, the indecomposable representation of degree $k$ occurs $\Gamma_{k-1}-\Gamma_k + \alpha_k$ times, with $\alpha_k = 1$ if $k= p^n-p^r +1$, $\alpha_k = -1$ if $k=p^n-p^r$ and $\alpha_k = 0$ otherwise.
\end{thm}

Now note that $g_E=0$ by assumption, and $r=0$ as commented earlier, so the degree of the regular representation is zero.

Now suppose that $p>2$ and that $g_X\geq 2$.
Suppose also that the action is trivial.


We first observe that $\Gamma_{p-1} = 0$. Indeed
\[
 v_{i(p-1)} = \delta_i - (p-1)\Phi(i,1) = (p-1)(\Phi(i,1) + 1) - (p-1)\Phi(i,1) = p-1,
\]
and $\left \lfloor \frac{p-1}{p} \right\rfloor = 0$. This gives $\Gamma_{p-1} = \sum_{i=1}^s 0 = 0$, as desired.

Now if we were to assume that the action is trivial, then for all $k\neq 1$ the representation of degree $k$ should not occur and it should be true that $d_k = 0$.
So if $k= p-1$ we have $\Gamma_{p-2} - \Gamma_{p-1} - 1 = \Gamma_{p-2} -1 = 0$, and hence $\Gamma_{p-2} = 1$.
Inductively for $2\leq k \leq p-2$, by the relation $\Gamma_{k-1} - \Gamma_k =0$, then $\Gamma_k = 1$.

Finally, we show a contradiction occurs when $k = 1$.
Since $p| \delta_i$, we can write
\begin{equation}\label{round}
 \Gamma_0 - \Gamma_1 = \left\lfloor \frac{\delta_i}{p} \right\rfloor -\left\lfloor \frac{\delta_i - \Phi(i,1)}{p} \right\rfloor = -\left\lfloor \frac{-\Phi(i,1)}{p} \right\rfloor.
\end{equation}

As 
\[
 1 = \Gamma_{p-2} = \sum_{i=1}^s\left\lfloor \frac{\Phi(i,1) + p -1}{p} \right\rfloor,
\]
 and as the sum is made of non-negative terms (by lemma \ref{koeck}, $-\Phi(i,1) \geq 0$) , we see that $p \leq \Phi(i,1) \leq p-1$ for one $i$, and $\Phi(i,1) = 0$ otherwise.
Without loss of generality we can assume the $\Phi(1,i)$ is the only non-zero term.
Combining this with \ref{round} and the action being trivial implies that $1 = -\left\lfloor \frac{-\Phi(1,1)}{p} \right\rfloor = \Gamma_0 - \Gamma_1 = g_F.$
Since $g_F \geq 2$ this is a contradiction and we are done.


\newpage


Putting $k=1$, we get 
\[
\Gamma_0 - \Gamma_1 = \sum_{i=1}^s\left(\left\lfloor \frac{\delta_i}{p} \right\rfloor \left\lfloor \frac{\delta_i - \Phi(i,1)}{p} \right\rfloor \right) = \sum_{i=1}^s - \left\lfloor \frac{\phi(i)}{p} \right\rfloor = g_F.
\]

Now since $\Gamma_{k-1}-\Gamma_k+\alpha_k = \Gamma_{k-1}-\Gamma_k = 0$ for $k=1,\ldots ,p-2$, it follows that $\Gamma_k=g_F$ for all $k\neq p-1$.

We now show that $\Gamma_{p-1} = 0$, which will contradict
\[
 \Gamma_{p-2} - \Gamma_{p-1} - 1 = g_F - \Gamma_{p-1} - 1 = 0.
\]


Indeed, for $k = p-1$, we have by \ref{eq}
\begin{eqnarray*}
v_{ik} & = & \left \lfloor \frac{\delta_i - k\Phi(i,1)}{p} \right\rfloor \\
	& = & \left\lfloor \frac{(p-1)(\Phi(i,1) + 1) - (p-1)\Phi(i,1)}{p} \right\rfloor\\
	& = & 0.
\end{eqnarray*}



\bibliography{/home/jtait/files/Documents/Maths/Bibliography/biblio.bib}
\bibliographystyle{plain}

\end{document}


\appendix
%\include{AppendixA}
\backmatter
\bibliographystyle{amsalpha}
\bibliography{biblio}
\printindex
\end{document}
%% ----------------------------------------------------------------
