\chapter{Background}\label{chapterbackground}

In this section we will introduce the basic concepts, notation and terminology required in the study of algebraic curves.

Throughout this thesis $k$ will denote an algebraically closed field of characteristic $p \geq 0$.
It should be noted that while the majority of results in the thesis hold for all characterisitcs, including $p=0$, our main focus will be on the case $p > 0$.

When we refer to a \emph{curve} we will mean a smooth, connected, projective variety of dimension one over $k$.
% Similarly, when we refer to an \emph{affine curve} we mean a smooth, connected, affine variety of dimension one over $k$. % Add if needed


A \emph{meromophirc function} on $X$ is a morphism $f \colon X \ra \PP_k^1$.
The collection of meromorphic functions on $X$ is denoted $K(X)$, and called the \emph{function field} of $X$.

We recall that the category of algebraic curves up to birational  equivalence is actually equivalent to the category of function fields over $k$ (which can be defined independantly of curves as fields of transcendence degree one over $k$).
Furthermore, if $k = \CC$ then we have a triple equivalence of categories, between function fields over $\CC$, algebraic curves (up to biratioanl equivalence) and the category of compact Riemann surfaces.\todo{References. Maybe Miranda, Stichtenoth and/or Hartshorne.}

We say that a meromorphic function on $X$ is \emph{regular} on an open set $U \subseteq X$ if the image $f(U)$ lies in $k = \AA_k^1 \subset \PP_k^1$.
We let $H^0(U,\cO_X)$ denote the space of functions in $K(X)$ which are regular on $U$.
Moreover, if $f \in K(X)$ is regular on the whole curve we say that $f$ is \emph{regular}, and then $H^1(X,\cO_X)$ is the space of regular functions.

For any $P \in X$ a meromorphic function $f \in K(X)$ is \emph{regular at $P$} if $f(P) \in k \subset \PP_k^1$.
The collection of functions regular at $P$ form a ring, which we call $\cO_{X,P}$.

    \begin{lem}
    For any $P \in X$ the ring $\cO_{X,P}$ is a discrete valuation ring, with maximal ideal
        \[
        \mathcal {M}_{X,P} := \{ f \in \cO_X,P} | f(P) = 0 \}.
        \]
    \end{lem}
    \begin{proof}
    \todo[inline]{Proof or reference here}
    \end{proof}

The valuation on $\cO_{X,P}$ can be given as follows.
Let $t \in \cO_{X,P}$ be a generator of $\mathcal{M}_{X,P}$.
Now any $f \in \cO_{X,P}$ can be written as $f = ut^n$ for some unique $n \in \ZZ_{\geq 0}$ and some unit $u \in \cO_{X,P}\backslash \mathcal{M}_{X,P}$.
We then define \emph{the order of $f$ at $P$} to be $v_P(f) := n$.
We call any element $t \in \cO_{X,P}$ with order 1 at $P$ a \emph{uniformising parameter at $P$}.

\section{Differentials}

We now introduce the concept of a differential on the xurve $X$.

Let $R$ be any commutaitive ring containing $k$ and let $M$ be an $R$-module.
Then a $k$-linear map $D \colon R \ra M$ satisfying $D(fg) = fD(g) + gD(f)$ is called a \emph{derivation} of $R$ in to $M$ over $k$.

There exists a unique module $\Omega_k(R)$, and a derivation $d \colon R \ra \Omega_k(R)$, through which all derivations must factor.
We can describe this more concretely as the free module generated by $[f]$ for all $f \in K(X)$, quotiented by the relations
    \begin{itemize}
    \item $[f]+[g] = [f+g]$;
    \item $[cf] = c[f]$;
    \item $[fg] = f[g] + g[f]$;
    \end{itemize}
where $f, g \in R$ and $c \in k$.

In particular, when $R = K(X)$ we let $\Omega_{K(X)} := \Omega_k(K(X))$.
We call $\Omega_{K(X)}$ the module of differentials of $X$, and we call any $ \omega \in \Omega_{K(X)}$ a meromorphic differential.

    \begin{prop}
    The module of differentials, $\Omega_{K(X)}$, is a one dimensional vector space over $K(X)$.
    \end{prop}







