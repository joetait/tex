\chapter{Background}\label{chapterbackground}

In this section we will introduce the basic concepts, notation and terminology required for the study of algebraic curves.

Throughout this thesis $k$ will denote an algebraically closed field of characteristic $p \geq 0$.
It should be noted that while the majority of results in the thesis hold for all characteristics, including $p=0$, our main focus will be on the case $p > 0$.

When we refer to a \emph{curve} we will mean a smooth, connected, projective variety of dimension one over $k$.
% Similarly, when we refer to an \emph{affine curve} we mean a smooth, connected, affine variety of dimension one over $k$. % Add if needed

\section{Functions and differentials}

In this section we define functions and differentials on a curve $X$ and give basic results pertaining to these objects.


A \emph{meromorphic function} on $X$ is a morphism $f \colon X \ra \PP_k^1$.
The collection of meromorphic functions on $X$ is denoted $K(X)$, and called the \emph{function field} of $X$.

We recall that the category of algebraic curves, up to birational, equivalence is actually equivalent to the category of function fields over $k$ (which can be defined independently of curves as fields of transcendence degree one over $k$).
An overview of this correspondence is given in \cite[Appendix B]{stichtenoth}.
Furthermore, if $k = \CC$ then we have a triple equivalence of categories, between function fields over $\CC$, algebraic curves (up to birational equivalence) and the category of compact Riemann surfaces. 
A short explanation of the correspondence between complex curves and Riemann surfaces is given in \cite[Chap.\ 1, \S 2]{griffiths}, whilst \cite{miranda} exhibits the connection between the two structures throughout.


We say that a meromorphic function on $X$ is \emph{regular} on an open set $U \subseteq X$ if the image $f(U)$ lies in $k = \AA_k^1 \subset \PP_k^1$.
We let $H^0(U,\cO_X)$ denote the space of functions in $K(X)$ which are regular on $U$.
Moreover, if $f \in K(X)$ is regular on $X$ we say that $f$ is \emph{regular}, and then $H^0(X,\cO_X)$ is the space of regular functions.

Given $P \in X$ we say that a meromorphic function $f \in K(X)$ is \emph{regular at $P$} if $f(P) \in k \subset \PP_k^1$.
The collection of functions regular at $P$ form a ring, which we call $\cO_{X,P}$.

    \begin{lem}
    For any $P \in X$ the ring $\cO_{X,P}$ is a discrete valuation ring, with maximal ideal
        \[
        \mathcal {M}_{X,P} := \{ f \in \cO_{X,P} | f(P) = 0 \}.
        \]
    \end{lem}
    \begin{proof}
    See \cite[Chap.\ 1, \S 4]{fulton}.
    \end{proof}

The valuation on $\cO_{X,P}$ can be given as follows.
Let $t \in \cO_{X,P}$ be a generator of $\mathcal{M}_{X,P}$.
Now any $f \in \cO_{X,P}$ can be written as $f = ut^n$ for some unique $n \in \ZZ$ and some unit $u \in \cO_{X,P}\backslash \mathcal{M}_{X,P}$.
We then define \emph{the order of $f$ at $P$} to be $\ord_P(f) := n$.
If $\ord_P(f) = n >0$ we say that $f$ has a \emph{zero of order $n$} at $P$, whilst if $\ord_P(f) = n < 0$ then we say that $f$ has a \emph{pole of order $n$} at $P$.
Clearly, for any $f, g \in K(X)$ and $P \in X$ it is true that $\ord_P(fg) = \ord_P(f) + \ord_P(g)$.
We call any element $t \in \cO_{X,P}$ for which we have $\ord_P(t)=1$ a \emph{uniformising parameter at $P$}.

    \begin{prop}\label{propfinitelymanyzeroesandpoles}
    Any non-zero meromorphic function $f$ on $X$ has finitely many poles and zeroes.
    Moreover, the number of poles and zeroes of $f$ are equal, counting multiplicity; \ie 
        \[
        \sum_{P \in X} \ord_P(f) = 0.
        \]
    \end{prop}
    \begin{proof}
    See \cite[Chap.\ 8, \S 1, Prop.\ 1]{fulton}.
    \end{proof}


We now introduce the concept of a differential on the curve $X$.
Let $R$ be any commutative ring containing $k$ and let $M$ be an $R$-module.
Then a $k$-linear map $D \colon R \ra M$ satisfying $D(fg) = fD(g) + gD(f)$ is called a \emph{derivation} of $R$ in to $M$ over $k$.

There exists a unique module $\Omega_k(R)$, and a derivation $d \colon R \ra \Omega_k(R)$, through which all derivations of $R$ over $k$ must factor.
We can describe this more concretely as the free module generated by $[f]$ for all $f \in K(X)$, quotiented by the relations
    \begin{itemize}
    \item $[f]+[g] = [f+g]$,
    \item $[cf] = c[f]$,
    \item $[fg] = f[g] + g[f]$,
    \end{itemize}
where $f, g \in R$ and $c \in k$.

In particular, when $R = K(X)$ we let $\Omega_{K(X)} := \Omega_k(K(X))$, and we call the map $d \colon K(X) \ra \Omega_{K(X)}$ the \emph{differential map}.
We then call $\Omega_{K(X)}$ the \emph{module of differentials of $X$}, and we call any $ \omega \in \Omega_{K(X)}$ a \emph{meromorphic differential}.

    \begin{prop}\label{propdifferentialsareonedimensional}
    The module of differentials, $\Omega_{K(X)}$, is a one dimensional vector space over $K(X)$.
    \end{prop}
    \begin{proof}
    See \cite[Prop. 1.5.9]{stichtenoth}.
    \end{proof}

We suppose that $P \in X$ and we choose a uniformising parameter $t \in \cO_{X,P}$.
Then for any $ 0 \neq \omega \in \Omega_{K(X)}$ there exists a unique integer $n \in \ZZ$ and unit $u \in \cO_{X,P}$ such that $\omega = ut^ndt$, by Proposition \ref{propdifferentialsareonedimensional}.
Then we define the \emph{order of $\omega$ at $P$} to be $\ord_P(\omega) := n$.

Let $U$ be an open subset of $X$.
We call a differential $\omega$ \emph{holomorphic on $U$} if $\ord_P(\omega) \geq 0$ for all $P \in U$, and we let
    \[
    H^0(U, \Omega_X) := \{ \omega \in \Omega_{K(X)} | \ord_P(\omega) \geq 0\ \text{for all}\ P \in X\}
    \]
be \emph{the space of holomorphic differentials on $U$}.
If $\omega \in \Omega_{K(X)}$ is holomorphic on $X$ we say that $\omega$ is \emph{holomorphic}, and so $\hzero$ is the space of holomorphic differentials.

\section{The Riemann-Roch theorem}

We now recall the relevant facts and definitions needed to state the Riemann-Roch theorem.

We first recall that a \emph{divisor} on $X$ is a finitely supported formal sum 
    \[
    D = \sum_{P \in X} n_P[P],
    \]
with coefficients in $\ZZ$.
The set off all divisors on $X$ forms an additive group, denoted $\di (X)$.
The \emph{degree} of the divisor $D$ is $\deg(D) := \sum_{P \in X} n_P$.


Given any function $f \in K(X)$ we define the \emph{divisor associated to $f$} to be
    \[
    \di(f) := \sum_{P \in X} \ord_P(f) [P].
    \]
Note that by Proposition \ref{propfinitelymanyzeroesandpoles} this has finite support.
We call any divisor $D$ which is equal to $\di(f)$ for some $f \in K(X)$ a \emph{principal divisor}.
Is is clear that for any $f, g \in K(X)$ we have $\di(fg) = \di(f) + \di(g)$.
Also, for any $f \in K(X)$ we define $\di_0(f)$ and $\di_\infty (f)$, the \emph{divisor of zeroes} and the \emph{divisor of poles} of $f$ respectively, as follows:
    \[
    \di_0(f) := \sum_{\ord_P(f) >0} \ord_P(f)[P]
    \]
and then
    \[
    \di_\infty(f) := \di_0(f) - \di(f).
    \]

Now for any differential $0 \neq \omega \in \Omega_{K(X)}$ we define the \emph{divisor associated to $\omega$} to be
    \[
    \di(\omega) := \sum_{P \in X} \ord_P(\omega)[P].
    \]
If $W$ is a divisor on $X$ and $W = \di(\omega)$ for some $ \omega \in \Omega_{K(X)}$ then we say that $W$ is a \emph{canonical divisor} on $X$.

The principal divisors of $X$ form a subgroup of $\di(X)$, and we say that two divisors $D, D' \in \di(X)$ are equivalent, denoted $D \sim D'$, if their image in the quotient of $\di(X)$ by the group of principal divisors is the same; \ie if there exists $f \in K(X)$ such that $ D = D' + \di(f)$.
By the following theorem, it makes sense to refer to the (unique) canonical divisor on $X$, up to equivalence, which we write as $K_X$.

    \begin{thm}
    Let $W$ and $W'$ be canonical divisors on $X$.
    Then there exists some $f \in K(X)$ such that $W = W' + \di(f)$.
    Moreover, the divisor $W + \di(f)$ is canonical for every $f \in K(X)$.
    \end{thm}
    \begin{proof}
    The second statement follows from the fact that $\Omega_{K(X)}$ is a $K(X)$ vector space, and from the definition of canonical divisor.


    To prove the first statement, suppose that $\omega, \omega' \in \Omega_{K(X)}$ are differentials such that $W = \di(\omega)$ and $W' = \di(\omega')$.
    Then, by Proposition \ref{propdifferentialsareonedimensional}, there exists an $f \in K(X)$ such that $\omega = f \omega'$.
    We conclude the proof by taking the divisors of both sides of this equation.
    \end{proof}

Given any divisor $D = \sum_{P \in X} n_P[P]$ we let
    \[
    H^0(X,\cO_X(D)) : = \{ f \in K(X) | \ord_P(f) \geq n_P\ \text{for all}\ P \in X\}
    \]
be the \emph{vector space of meromorphic functions associated to $D$}.
Similarly, we let 
    \[
    H^0(X,\Omega_X(D)) :  = \{ \Omega \in \omega_{K(X)} | \ord_P(\omega) \geq n_P \ \text{for all} \ P \in X\}
    \]
be the \emph{vector space of meromorphic differentials associated to $D$}.
In particular, when $D$ is the zero divisor we have $H^0(X,\Omega(0)) = \hzero$, and similarly $H^0(X,\cO_X(0)) = H^0(X,\cO_X)$.

We now state the celebrated Riemann-Roch theorem.

    \begin{thm}[Riemann-Roch theorem]\label{theoremriemannroch}
    Let $D$ be any divisor on $X$, and let $W$ be any canonical divisor on $X$.
    Then
        \[
        \dim_k(H^0(X,\cO_X(D))) = \deg(D) + 1 - g_X + \dim_k(H^0(X,\cO_X(W-D)), 
        \]
    for some constant $g_X \in \ZZ_{\geq 0}$, which is independent of the choice of $W$ and $D$.
    \end{thm}
    \begin{proof}
    See \cite[Chap.\ IV, \S 1, Thm.\ 1.3]{hart} or, for a more elementary approach, \cite[Chap.\ 8, \S 6]{fulton}.
    \end{proof}

    \begin{defn}
    The constant $g_X$ in the statement of the Riemann-Roch theorem is the genus of $X$.
    \end{defn}

The genus is an invariant of fundamental importance in the study of algebraic curves.
In particular, we remark that if $k = \CC$ then the genus of an algebraic curve (also called the arithmetic genus) is the same as the topological genus of the corresponding Riemann surface.

We now give a corollary to the Riemann-Roch theorem, which shows some of the properties of the genus.

    \begin{cor}\label{dim=gc}
    For any canonical divisor $W$ on $X$, we have 
        \[
        \deg(W) = 2g_X-2
        \]
    and 
        \[
        \dim H^0(X,\cO_X(W)) = g_X.
        \]
    \end{cor}
    \begin{proof}
    Since $\dim H^0(X,\cO_X(0)) = 1$ (the only functions with no poles are the constant functions), we have by the Riemann-Roch theorem, 
        \[
        1= \dim H^0(X,\cO_X(0)) = 0 + 1 -g_X + \dim H^0(X,\cO_X(W)).
        \]
    Rearranging this gives the second statement.
    The first statement then follows by rearranging
        \begin{multline*}
        g_X = \dim H^0(X,\cO_X(W)) = \deg(W) + 1 -g_X +  \dim H^0(X,\cO_X(W-W))\\ = \deg(W) + 1 -g_X + 1.
        \end{multline*}
    \end{proof}

\section{Ramification and the Hurwitz formula}

In this section we will introduce the concept of ramification, and conclude by stating the Hurwitz formula, which relates the genera of two curves via the ramification divisor.



Let $X$ and $Y$ be curves over $k$.
We begin by defining what it means for a map $\phi \colon X \ra Y$ to be a morphism of curves.
We first note that given a map of sets $\phi \colon X \ra Y$ we have an induced ring homomorphism on the function fields,
    \[
    \phi^* \colon K(Y) \ra K(X),
    \]
given by composition with $\phi$; \ie $\phi^*(f) = f \circ \phi$.
Moreover, it transpires that $\phi^*$ is an injection, and hence we can view $K(Y)$ as a subfield of $K(X)$.
We then define the \emph{degree} of $\phi$ to be the degree of the extension $K(X)/K(Y)$.

Then a \emph{morphism of curves} is a continuous map $\phi \colon X \ra Y$ such that for every open set $U \subseteq Y$ we have
    \[
    \phi^*(H^0(U,\cO_Y)) \subseteq H^0(\phi^{-1}(U),\cO_X).
    \]

When considering the quotient $X/G$ of $X$ by some subgroup $G$ of the automorphism group $\aut(X)$ we have a morphism $\phi \colon X \ra X/G$, which is a projection.
The majority of the topics considered in this thesis will focussed on situations where we have such a projection.

We now return to assuming that $\phi \colon X \ra Y$ is an arbitrary morphism of curves.
    \begin{defn}
    Let $P \in X$ and choose a uniformising parameter $t \in \cO_{Y,f(P)}$.
    We define the ramification index $e_P$ of $\phi$ at $P$ to be
        \[
        e_P := \ord_P(\phi^*(t)).
        \]
    \end{defn}

Note that $e_P =1$ for almost all points $P \in X$.
We say that the point $Q \in Y$ is a \emph{branch point} of $\phi$ if there exists some $P \in \phi^{-1}(Q)$ for which $e_P >1$.
We say that $P \in X$ is a \emph{ramification point} of $X$ if $e_P >1$.

    
Suppose $P \in X$ is a ramification point.
Then if $p = \cha(k)$ divides $e_P$ we say that $P$ is \emph{wildly ramified}, and we also say $\phi$ is wildly ramified if this holds for any point in $X$.
If $p$ does not divide $e_P$ we say that $P$ is tamely ramified.

    \begin{defn}
    Let $D = \sum_{Q \in Y}n_Q [Q]$ be a divisor on $Y$.
    Then the pull back of $D$ with respect to $\phi$ is
        \[
        \phi^*(D) := \sum_{Q \in Y} \sum_{P \mapsto Q} e_P \cdot n_Q [P].
        \]
    \end{defn}

Note that $\phi^*$ defines a group homomorphism $\di(Y) \ra \di(X)$.
Now we define the different exponent, which we require to define the ramification divisor.
    
    \begin{defn}\label{definitiondifferent}
    For any $P\in X$ let $Q = \phi(P) \in Y$.
    Moreover, we choose uniformising parameters $s \in \cO_{X,P}$ and $t \in \cO_{Y,Q}$.
    Then by Proposition \ref{propdifferentialsareonedimensional} there exists a unique $f \in K(X)$ such that $\phi^*(dt) = f\cdot ds$.
    Then we define the different exponent at $P$ to be 
        \[
        \delta_P : = \ord_P(f).
        \]
    \end{defn}

Note that since $s$ is a uniformising parameter at $P$, and $\phi^*(t)$ is regular at $P$, it follows that $f$ is regular at $P$; in particular, $\delta_P$ is non-negative for all $P \in X$.
 
    \begin{defn}[Ramification divisor]\label{defnramificationdivisor}
    The ramification divisor of $\phi \colon X \ra Y$ is 
        \[
        R:= \sum_{P \in X} \delta_P [P].
        \]
    If $\phi$ is tamely ramified then 
        \[
        R: = \sum_{P \in X} (e_P - 1)[P].
        \]
    \end{defn}

Hilbert's formula, stated below, gives an alternative presentation of the ramification divisor, which will ease computations later in the thesis.
However, in order to do this we need to introduce higher ramification groups.

   \begin{defn}
    Let $G:=\gal(X/Y)$ and let $t$ be a uniformising parameter at $P\in X$.
    Then for $i\geq -1$ we define the $i^{th}$ ramification group at $P$, denoted $G_i(P)$, to be the subgroup of $s\in G$ such that $i_G(s) := \ord_P(s(t)-t)$ is at least $i+1$.
    This is	independent of the choice of $t$, see \cite[Chap. IV, \S 1, pg. 62]{localfields}.
    \end{defn}

Note that for any $P\in X$ we have that $G_{-1}(P)=G$, if $i$ is sufficiently large then $G_i(P)$ is trivial and $G_i(P)\supseteq G_{i+1}(P)$.
Also, $G_1$ is a $p$-group and $\ord(G_0(P)/G_1(P))$ is coprime to $p$.
In particular, $\phi$ is tamely ramified at $P$ if and only if $G_1(P)$ is the trivial group.
More details can be found in \cite[Chap. IV, \S 1]{localfields}.%Citation is slightly ambiguous, but only references four pages, so is okay.

Note that $e_P = \ord (G_0(P))$ for any $P \in X$.
We now state Hilbert's formula, relating our presentation of the ramification divisor in Definition \ref{defnramificationdivisor} to the ramification groups, which will make future computations easier.


    \begin{thm}[Hilbert's Formula]\label{hilbertsformula}
    Suppose that $P\in X$ and that $t$ is a uniformising parameter in $\cO_{X,P}$.
    Then we have
        \begin{equation}
        \delta_P = \sum_{s\neq e}i_G(s)=\sum_{j=0}^{\infty}\left(\ord(G_i(P))-1\right),\label{ramdiv}
        \end{equation}
    where $e$ denotes the identity in $G$.
    \end{thm}
    \begin{proof}
    For the sake of brevity we do not prove these statements here. See \cite[Chap. IV, \S 1, Prop. 4]{localfields} for a proof of Hilbert's formula.
    \end{proof}

 
Given any differential $ \omega = g\cdot df \in \Omega_{K(Y)}$ we define the pullback of $\omega$ by $\phi$ to be
    \[
    \phi^*(\omega) := \phi^*(g)d\phi^*(f).
    \]
Clearly $\phi^*(\omega)$ is a differential on $X$.

    \begin{thm}\label{theoremdetailedhurwitz}
    If $0 \neq \omega \in \Omega_{K(Y)}$ then
        \begin{equation}\label{equationstronghurwitzformula}
        \di(\phi^*(\omega)) = \phi^*(\di(\omega)) + R.
        \end{equation}
    In particular, we have
        \[
        K_X \sim \phi^*(K_Y) + R.
        \]
    \end{thm}
    \begin{proof}
    See \cite[Chap.\ IV, \S 2, Prop.\ 2.3]{hart} for a sheaf theoretic approach, or alternatively \cite[Thm. 3.4.6]{stichtenoth}, for a proof involving function fields.
    \end{proof}


As a corollary to this we have the Riemann-Hurwitz formula.

    \begin{cor}\label{corhurwitzformula}[Riemann-Hurwitz Formula]
    Given two non-singular projective curves $X$ and $Y$ of genera $g_X$ and $g_Y$ respectively, with a degree $n$ map $\phi\colon X \rightarrow Y$, then
        \[
        2g_X - 2 = n(2g_Y -2) + \deg(R),
        \]
    where $R$ is the ramification divisor of $\phi$.
    \end{cor}
    \begin{proof}
    This follows from Corollary \ref{dim=gc}, after taking degrees in \eqref{equationstronghurwitzformula}.
    \end{proof}




















