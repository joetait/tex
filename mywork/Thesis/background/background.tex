\chapter{Background}\label{chapterbackground}

In this section we give the basic definitions and results that will be used throughout the thesis.
The vast majority results apply to smooth, connected, projective curves over any algebraically closed fields, with no further assumptions, though occasionally we do specialise slightly more than this.
All definitions and results should be available in elementary texts on algebraic curves or algebraic geometry in general, such as \cite{fulton} or \cite{hart} .

We start by defining precisely what we mean by a curve, and by functions and differentials on a curve.
We then go on to give some basic results about these objects, and finally define the genus of a curve.

In the next section we define divisors, canonical divisors and the spaces associated to divisors.
We then reach the highlight of the section with the statement of the Riemann-Roch theorem.
We give corollaries to this, which show some of its applications.

In the penultimate section of this chapter we consider ramification.
We define ramification and branch points, and subsequently the ramification divisor.
We then use this to state a strong version of the Riemann-Hurwitz formula, at the level of divisors.
The section concludes by looking at group actions on curves, and defining higher ramification to state Hilbert's formula.

The chapter finishes by discussing Serre duality, which will be of particular use in the fourth chapter of the thesis.
In particular the dual maps are described explicitly, using \cech cohomology to do so.


\section{Set up}

Throughout this thesis $k$ will denote an algebraically closed field of characteristic $p \geq 0$.
It should be noted that while the majority of results in the thesis hold for all characteristics, including $p=0$, our main focus will be on the case $p > 0$.

When we refer to an \emph{algebraic curve} (or often just a \emph{curve}) we will mean a smooth, connected, projective variety of dimension one over $k$.
In particular, we let $\PP_k^1$ be the projective.
Similarly, when we refer to an \emph{affine curve} we mean a smooth, connected, affine variety of dimension one over $k$. % Add if needed
We recall that a morphism of affine curves $X$ and $Y$ is just polynomial map $\phi \colon X \ra Y$.
Then if $X$ and $Y$ are algebraic curves a map $\phi \colon X \ra Y$ is a morphism if we can write $X = \cup X_i$ and $Y = \cup Y_i$ for open, affine $X_i$ and $Y_i$, such that $\phi(X_i) \subseteq Y_i$ and $\phi|_{X_i}$ is a morphism for every $i$.



\section{Functions and differentials}

In this section we recall the basic results pertaining to functions and differentials on $X$.

A \emph{meromorphic function} on $X$ is any morphism $f \colon X \ra \PP_k^1$, other than the morphism mapping all points to infinity.
The collection of meromorphic functions on $X$ is denoted $K(X)$, and called the \emph{function field} of $X$.

We recall that the category of algebraic curves and non-constant morphisms is actually equivalent to the category of function fields over $k$ (which can be defined independently of curves as fields of transcendence degree one over $k$).
An overview of this correspondence is given in \cite[Appendix B]{stichtenoth}.
Furthermore, when working over the complex numbers $\CC$ we actually have a triple equivalence of categories.
The category of function fields over $\CC$ and the category of algebraic curves are both equivalent to the category of compact Riemann surfaces. 
A short explanation of the correspondence between complex curves and Riemann surfaces is given in \cite[Chap.\ 1, \S 2]{griffiths}, whilst \cite{miranda} exhibits the connection between all three categories throughout.


Returning to our study of functions on $X$, we recall that a meromorphic function $f$ on $X$ is \emph{regular} on an open set $U \subseteq X$ if the image $f(U)$ lies in $k = \AA_k^1 \subset \PP_k^1$.
We let $H^0(U,\cO_X)$ denote the space of functions in $K(X)$ which are regular on $U$.
Moreover, if $f \in K(X)$ is regular on $X$ we say that $f$ is \emph{regular}, and then $H^0(X,\cO_X)$ is the space of regular functions.
Since $X$ is projective $H^0(X,\cO_X)$ is in fact isomorphic to $k$ --- \ie the only regular functions are constant functions.
The reader should note that we are using sheaf theoretic notation here. 
We will not give details of sheaves and sheaf cohomology (since it will rarely be needed), but we will consistently use the notation, in order to be with consistent with current work in the area.

Given $P \in X$ we say that a meromorphic function $f \in K(X)$ is \emph{regular at $P$} if $f(P) \in k \subset \PP_k^1$.
The collection of functions regular at $P$ form a ring, which we call $\cO_{X,P}$.

    \begin{lem}
    For any $P \in X$ the ring $\cO_{X,P}$ is a discrete valuation ring, with maximal ideal
        \[
        \mathcal {M}_{X,P} := \{ f \in \cO_{X,P} | f(P) = 0 \}.
        \]
    \end{lem}
    \begin{proof}
    See \cite[Chap.\ 1, \S 4]{fulton}.
    \end{proof}

The valuation on $\cO_{X,P}$ can be given as follows.
Let $t \in \cO_{X,P}$ be a generator of $\mathcal{M}_{X,P}$.
Now any $0 \neq f \in \cO_{X,P}$ can be written as $f = ut^n$ for some unique $n \in \ZZ$ and some unit $u \in \cO_{X,P}\backslash \mathcal{M}_{X,P}$.
We then define \emph{the order of $f$ at $P$} to be $\ord_P(f) := n$.
For any $f \in K(X)^*$ and $P \in X$ at least one $f$ or $1/f$ is an element of $ \cO_{X,P}$.
Hence we may extend the definition of $\ord_P$ to the whole $K(X)^*$, by letting $\ord_P(f) := - \ord_P(1/f)$ whenever $f \notin \cO_{X,P}$.
If $\ord_P(f) = n >0$ we say that $f$ has a \emph{zero of order $n$} at $P$, whilst if $\ord_P(f) = n < 0$ then we say that $f$ has a \emph{pole of order $n$} at $P$.
Clearly, for any $f, g \in K(X)^*$ and $P \in X$, it is true that $\ord_P(fg) = \ord_P(f) + \ord_P(g)$.
We call any element $t \in \cO_{X,P}$ which has order $1$ at $P$ a \emph{uniformising parameter at $P$}.

    \begin{prop}\label{propfinitelymanyzeroesandpoles}
    Any non-zero meromorphic function $f$ on $X$ has finitely many poles and zeroes.
    Moreover, the number of poles and zeroes of $f$ are equal, counting multiplicity; \ie 
        \[
        \sum_{P \in X} \ord_P(f) = 0.
        \]
    \end{prop}
    \begin{proof}
    See \cite[Chap.\ 8, \S 1, Prop.\ 1]{fulton}.
    \end{proof}


We now introduce the concept of a differential on the curve $X$.
Let $R$ be any commutative ring containing $k$ and let $M$ be an $R$-module.
Then a $k$-linear map $D \colon R \ra M$ satisfying $D(fg) = fD(g) + gD(f)$ is called a \emph{derivation} of $R$ in to $M$ over $k$.

There exists a unique module $\Omega_k(R)$, called the \emph{module of differentials of $R$ over $k$}, and a derivation $d \colon R \ra \Omega_k(R)$ through which all derivations of $R$ over $k$ must factor.
We can describe $\Omega_k(R)$ more concretely as the free module generated by $[f]$ for all $f \in R$, quotiented by the relations
    \begin{itemize}
    \item $[f]+[g] = [f+g]$,
    \item $[cf] = c[f]$,
    \item $[fg] = f[g] + g[f]$,
    \end{itemize}
where $f, g \in R$ and $c \in k$.
Then $d(f)$ is the image of $[f]$ in this quotient.

In particular, when $R = K(X)$ we let $\Omega_{K(X)} := \Omega_k(K(X))$.
Then we call the map $d \colon K(X) \ra \Omega_{K(X)}$ the \emph{differential map} and we let $df := d(f)$.
We say that $\omega \in \Omega_{K(X)}$ is a \emph{meromorphic differential}.

    \begin{prop}\label{propdifferentialsareonedimensional}
    The module of differentials, $\Omega_{K(X)}$, is a one dimensional vector space over $K(X)$.
    Moreover, if $t \in K(X)$ is a uniformising parameter for any point in $X$ then $dt$ is a basis of $\Omega_{K(X)}$.
    \end{prop}
    \begin{proof}
    See \cite[Prop. 1.5.9]{stichtenoth}.
    \end{proof}

We suppose that $P \in X$ and we choose a uniformising parameter $t \in \cO_{X,P}$.
Then for any $ 0 \neq \omega \in \Omega_{K(X)}$ there exists a unique integer $n \in \ZZ$ and unit $u \in \cO_{X,P}$ such that $\omega = ut^ndt$, by Proposition \ref{propdifferentialsareonedimensional}.
We define the \emph{order of $\omega$ at $P$} to be $\ord_P(\omega) := n$.
The set of differentials regular at $P$ form a module over $\cO_{X,P}$, which we call the \emph{module of differentials regular at $P$}, and denote by $P$.

    \begin{lem}
    Fix $P \in X$ and let $t \in \cO_{X,P}$ be a uniformising parameter.
    Then $\Omega_{X,P}$ is generated as a $\cO_{X,P}$-module by $dt$.
    \end{lem}
    \begin{proof}
    As stated in \cite[Pg.\ 300]{hart}, this follows from \cite[Chap.\ II, \S 8, Prop.\ 8.7 and Thm.\ 8.8]{hart}.
    \end{proof}

Using the previous lemma we now show that if $f \in K(X)$ is regular at $P \in X$ then so is $df$.
Firstly, if $f$ is a unit at $P$ then we let $a:= f(P) \in k$.
Then $\ord_{P}(f - a) \geq 1$, and also $df = d(f-a)$, since the image of $d$ on $k$ is zero.
So it suffices to show our claim for $f \in \mathcal{M}_{X,P}$.
Any such $f$ is equal to $ut^n$ for some unit $u$, uniformising parameter $t$ and $n \in \NN$.
Then $d(ut^n) = utd(t^{n-1}) + t^{n-1}d(ut) = (n-1)ut^{n-1}dt + t^{n-1}d(ut)$.
Since $ut$ is also a uniformising parameter, both terms are regular at $P$.
Finally, as $\ord_P(\omega - \omega') \geq \max\{\ord_P(\omega), \ord_P(\omega')\}$ (see \todo{citation}), it follows that $\ord_P(f) \geq 0$.

Let $U$ be an open subset of $X$.
We call a differential $\omega$ \emph{holomorphic on $U$} if $\ord_P(\omega) \geq 0$ for all $P \in U$, and we let
    \[
    H^0(U, \Omega_X) := \{ \omega \in \Omega_{K(X)} | \ord_P(\omega) \geq 0\ \text{for all}\ P \in X\} \cup \{ 0 \}
    \]
be \emph{the space of holomorphic differentials on $U$}.
If $\omega \in \Omega_{K(X)}$ is holomorphic on $X$ we say that $\omega$ is \emph{holomorphic}, and so $\hzero$ is the space of holomorphic differentials.

    \begin{defn}\label{definitiongenus}
    We define the \emph{genus} of $X$ to be
        \[
        g_X := \dim_k \left( H^0(X,\Omega_X)\right).
        \]
    \end{defn}

The genus is an invariant of fundamental importance in the study of algebraic curves.
In particular, we remark that if $k = \CC$ then the genus of an algebraic curve (also called the arithmetic genus) is the same as the topological genus of the corresponding Riemann surface (the corresponding Riemann surface being found via the equivalence of categories mentioned earlier).

We now briefly recall the notion of a polydifferential.
If we consider an element of the tensor product $\omega \in \Omega_X^{\otimes m}$ then it can be locally written as $f dx_1\otimes \ldots \otimes dx_m$, where $f, x_i \in K(X)$ for all $1 \leq i \leq m$.
Let $P$ be a point in $X$.
Since each $dx_i$ can be written as $f_i dt$ for some $f_i\in K(X)$ and some uniformising parameter $t$ at $P$, we can rewrite $\omega$ as $f' dt \otimes \ldots \otimes dt$, where $f' = f \cdot f_1 \cdots f_m$.
We then define the order of $\omega$ at $P$ to be $\ord_P(\omega ) := \ord_P(y')$.
In the particular case where $\omega = fdx \otimes \ldots \otimes fdx = f^m dx^{\otimes m}$, then we have $f_1 = \ldots = f_m = h$ for some $h \in K(X)$ when we change $x$ to a uniformising parameter.
Hence in this instance 
    \[ 
    \ord_P(\omega) = \ord_P(h^m) = m\ord_P(h) = m\ord_P(dx).
    \]


\section{The Riemann-Roch theorem}

We now recall the relevant facts and definitions needed to state the Riemann-Roch theorem.

We first recall that a \emph{divisor} on $X$ is a finitely supported formal sum 
    \[
    D = \sum_{P \in X} n_P[P],
    \]
with coefficients in $\ZZ$.
The set off all divisors on $X$ forms an additive group, denoted $\Di (X)$.
The \emph{degree} of the divisor $D$ is $\deg(D) := \sum_{P \in X} n_P \in \ZZ$.


Given any function $f \in K(X)$ we define the \emph{divisor associated to $f$} to be
    \[
    \di(f) := \sum_{P \in X} \ord_P(f) [P].
    \]
Note that by Proposition \ref{propfinitelymanyzeroesandpoles} $\di(f)$ has finite support and degree zero.
We call any divisor $D$ which is equal to $\di(f)$ for some $f \in K(X)$ a \emph{principal divisor}.
Is is clear that for any $f, g \in K(X)$ we have $\di(fg) = \di(f) + \di(g)$.
Also, for any $f \in K(X)$ we define $\di_0(f)$ and $\di_\infty (f)$, the \emph{divisor of zeroes} and the \emph{divisor of poles} of $f$ respectively, as follows:
    \[
    \di_0(f) := \sum_{\ord_P(f) >0} \ord_P(f)[P]
    \]
and then
    \[
    \di_\infty(f) := \di_0(f) - \di(f).
    \]

Now for any differential $0 \neq \omega \in \Omega_{K(X)}$ we define the \emph{divisor associated to $\omega$} to be
    \[
    \di(\omega) := \sum_{P \in X} \ord_P(\omega)[P].
    \]
To show that does indeed have finite support we start by recalling for any $P \in X$ and $f \in K(X)$ then $\ord_P(df) = 0$ if $f$ is a unit in $\cO_{x,P}$.
Then, since any differential $\omega$ can be written as $gdf$ for some $f,g \in K(X)$, and $\ord_P(\omega) = \ord_P(g) + \ord_P(df)$, it follows that $\omega$ has only finitely many zeroes and poles --- \ie $\di(\omega)$ has finite support.
If $W$ is a divisor on $X$ and $W = \di(\omega)$ for some $ 0 \neq \omega \in \Omega_{K(X)}$ then we say that $W$ is a \emph{canonical divisor} on $X$.

The principal divisors of $X$ form a subgroup of $\di(X)$, and we say that two divisors $D, D' \in \di(X)$ are \emph{equivalent}, denoted $D \sim D'$, if their image in the quotient of $\di(X)$ by the group of principal divisors is the same; \ie if there exists $f \in K(X)$ such that $ D = D' + \di(f)$.
By the following corollary, it makes sense to refer to the (unique) canonical divisor on $X$, up to equivalence, which we write as $K_X$.

    \begin{cor}
    The canonical divisors on $X$ form precisely one equivalence class on $X$ with respect to the relation $\sim$.
    \end{cor}
    \begin{proof}
    Let $W$ be the canonical divisor associated to $\omega \in \Omega_{K(X)}$ and suppose that $D \in \Di(X)$ is equivalent to $W$.
    Then $D = W + \di(f) = \di(f \omega)$ is also a canonical divisor.

    On the other hand, suppose that $W$ and $W'$ are the canonical divisors associated to $\omega, \omega' \in \Omega_{K(X)}$ respectively.
    Then we can find a meromorphic function such that $\omega = f\omega'$, by Proposition \ref{propdifferentialsareonedimensional}.
    Then $W = W' + \di(f)$, and the divisors are equivalent.
    \end{proof}

Given any divisor $D = \sum_{P \in X} n_P[P]$ we let
    \[
    H^0(X,\cO_X(D)) : = \{ f \in K(X) | \ord_P(f) \geq -n_P\ \text{for all}\ P \in X\}
    \]
be the \emph{vector space of meromorphic functions associated to $D$}.
Similarly, we let 
    \[
    H^0(X,\Omega_X(D)) :  = \{ \omega \in \Omega_{K(X)} | \ord_P(\omega) \geq -n_P \ \text{for all} \ P \in X\}
    \]
be the \emph{vector space of meromorphic differentials associated to $D$}.
In particular, when $D$ is the zero divisor we have $H^0(X,\Omega(0)) = \hzero$, and similarly $H^0(X,\cO_X(0)) = H^0(X,\cO_X)$.
Also, it follows immediately from Proposition \ref{propfinitelymanyzeroesandpoles} that if $D \in Di(X)$ is a divisor with negative degree then $H^0(X,\cO_X(D)) = \{0\}$.

    \begin{lem}
    Given any divisor $D$ on $X$ we have the following isomorphism,
        \[
        H^0(X,\cO_X(D)) \cong H^0(X,\Omega_X(D - W))
        \]
    where $W$ is any canonical divisor on $X$.
    \end{lem}
    \begin{proof}
    Choose $\omega \in \Omega_{K(X)}$ and let $W = \di(\omega)$.
    Since $\di(f \omega) = \di(f) + \di(\omega)$, it follows that if $f \in H^0(X,\cO_X(D))$ if and only if $f \omega \in H^0(X,\Omega_X(D - W))$.
    Since $\Omega_{K(X)}$ is a one dimensional vector space over $K(X)$ we can find a unique $f \in K(X)$ for every differential $\omega'$ in $H^0(X,\Omega_X(D - W)$ such that $\omega' = f \omega$.
    Hence the map $f \mapsto f\omega$ is an isomorphism.
    \end{proof}

It follows from the above lemma and the definition of the genus of $X$ that $\dim_k(H^0(X,\cO_X(W))) = g_X$ for any canonical divisor $W$.

We now state the celebrated Riemann-Roch theorem.

    \begin{thm}[Riemann-Roch theorem]\label{theoremriemannroch}
    Let $g_X$ be the genus of $X$.
    Furthermore, let $D$ be any divisor on $X$, and let $W$ be any canonical divisor on $X$.
    Then
        \[
        \dim_k(H^0(X,\cO_X(D))) = \deg(D) + 1 - g_X + \dim_k(H^0(X,\cO_X(W-D)).
        \]
    \end{thm}
    \begin{proof}
    See \cite[Chap.\ IV, \S 1, Thm.\ 1.3]{hart} or, for a more elementary approach, \cite[Chap.\ 8, \S 6]{fulton}.
    \end{proof}




We now give two corollaries to the Riemann-Roch theorem.

    \begin{cor}\label{cordegreeofcanonicaldivisor}
    For any canonical divisor $W$ on $X$, we have 
        \[
        \deg(W) = 2g_X-2.
        \]
    \end{cor}
    \begin{proof}
    The statement follows by rearranging
        \begin{align*}
        g_X & = \dim H^0(X,\cO_X(W))  \\ & = \deg(W) + 1 -g_X +  \dim_k H^0(X,\cO_X(W-W))\\ & = \deg(W) + 1 -g_X + 1,
        \end{align*}
    where the first equality is Definition \ref{definitiongenus}, and the second equality follows from the Riemann-Roch theorem.
    \end{proof}

    \begin{cor}
    If $D$ is a divisor degree greater than $2g_X-2$ and $P$ is any point in $X$ then
        \[
        \dim_k(H^0(X,\cO_X(D +[P]))) = \dim_k(H^0(X,\cO_X(D))) + 1.
        \]
    \end{cor}
    \begin{proof}
    Since $\deg(D) > 2g-2$, it follows from the previous lemma that $\deg(W-D) <0$.
    Then $\dim_k(H^0(,\cO_X(W - D)))  = 0$.
    We then apply the Riemann-Roch theorem and see that
        \[
        \dim_k(H^0(X,\cO_X(D + [P]))) = \deg(D +[P]) + 1 - g_X = \deg(D) + 1 + 1 - g_X = \dim_k(H^0(X,\cO_X(D)) + 1.
        \]
    \end{proof}

Using the Riemann-Roch theorem and Corollary \ref{cordegreeofcanonicaldivisor} we can compute the dimension of the space of \emph{holomorphic polydifferentials} of order $m$, denoted $H^0(X,\Omega_X^{\otimes m})$, where $m \in \ZZ_{> 0}$.

    \begin{lem}\label{dim3}
    Let $g_X, m\geq 2$. Then
        \begin{equation}
        \dim_kH^0(X,\Omega_X^{\otimes m}) = (2m-1)(g_X-1) 
        \end{equation}
    \end{lem}
    \todo[inline]{removed cases $g_X = 0,1$, mention in chapter 5}
    \begin{proof}
    Since $g_X\geq 2$ it follows from Corollary \ref{cordegreeofcanonicaldivisor} that $\deg(W)\geq1$, and hence $\deg(mW)>\deg(W)$.
    Since $H^0(X,\Omega_X^{\otimes m}) \cong H^0(X,\Omega_X(mW))$ it then follows from the Riemann-Roch theorem (Theorem \ref{theoremriemannroch}) that
        \[
        \dim_kH^0(X,\Omega_X^{\otimes m})=\deg(mW)+1-g_X=(2m-1)(g_X-1).
        \]
    \end{proof}


\section{Ramification and the Riemann-Hurwitz formula}

In this section we will introduce the concept of ramification, and conclude by stating the Riemann-Hurwitz formula, which relates the canonical divisor of two curves which have a morphism between them, via the ramification divisor.



Let $X$ and $Y$ be curves over $k$.
We first note that given a non-constant morphism $\phi \colon X \ra Y$ we have an induced ring homomorphism on the function fields,
    \[
    \phi^* \colon K(Y) \ra K(X),
    \]
given by composition with $\phi$; \ie $\phi^*(f) = f \circ \phi$.
Moreover, it transpires that $\phi^*$ is an injection, and hence we can view $K(Y)$ as a subfield of $K(X)$.
We then define the \emph{degree} of $\phi$, denoted $\deg(\phi)$, to be the degree of the extension $K(X)/K(Y)$, which is always finite.

We henceforth assume that $\phi \colon X \ra Y$ is an arbitrary non-constant morphism of curves.
Recall that we have
    \[
    \phi^*(H^0(U,\cO_Y)) \subseteq H^0(\phi^{-1}(U),\cO_X).
    \]

    \begin{defn}
    Let $P \in X$ and choose a uniformising parameter $t \in \cO_{Y,\phi(P)}$.
    We define the \emph{ramification index} $e_P$ of $\phi$ at $P$ to be
        \[
        e_P := \ord_P(\phi^*(t)).
        \]
    \end{defn}

Note that $e_P =1$ for almost all points $P \in X$.
We say that the point $Q \in Y$ is a \emph{branch point} of $\phi$ if there exists some $P \in \phi^{-1}(Q)$ for which $e_P >1$.
We say that $P \in X$ is a \emph{ramification point} of $X$ if $e_P >1$.

The following theorem asserts that the degree of $\phi$ is the same as the number of points in the pre-image $\phi^{-1}(Q)$ for any $Q \in Y$, if we count multiplicities correctly.

    \begin{thm}
    Let $n := \deg(\phi)$.
    Then, for any $Q \in Y$, we have 
        \[
        \sum_{P \mapsto Q} e_P = n.
        \]
    \end{thm}
    \begin{proof}
    See, for example, \cite[Pg.\ 290]{liu}.
    \end{proof}

    
Suppose $P \in X$ is a ramification point.
Then if $p = \cha(k)$ divides $e_P$ we say that $P$ is \emph{wildly ramified}, and we also say $\phi$ is wildly ramified if this holds for any point in $X$.
If $p$ does not divide $e_P$ we say that $P$ is tamely ramified.

    \begin{defn}
    Let $D = \sum_{Q \in Y}n_Q [Q]$ be a divisor on $Y$.
    Then the \emph{pull back} of $D$ with respect to $\phi$ is
        \[
        \phi^*(D) := \sum_{Q \in Y} \sum_{P \mapsto Q} e_P \cdot n_Q [P].
        \]
    \end{defn}

Note that $\phi^*$ defines a group homomorphism $\Di(Y) \ra \Di(X)$.


We also define the pullback of a differential $\omega = g\cdot df \in \Omega_{K(Y)}$ by $\phi$ to be
    \[
    \phi^*(\omega) := \phi^*(g)d\phi^*(f).
    \]
Clearly $\phi^*(\omega)$ is a differential on $X$.


Now we describe the different exponent, which we require to define the ramification divisor.
    \begin{defn}\label{definitiondifferent}
    For any $P\in X$ we choose a uniformising parameter $t \in \cO_{Y,\phi(P)}$.
    Then we define the \emph{different exponent} at $P$ to be 
        \[
        \delta_P : = \ord_P(\phi^*(dt)).
        \]
    \end{defn}

Note that since $\phi^*(t)$ is regular at $P$ it follows that $\delta_P$ is non-negative for all $P \in X$.
 
    \begin{defn}[Ramification divisor]\label{defnramificationdivisor}
    The \emph{ramification divisor} of $\phi \colon X \ra Y$ is 
        \[
        R:= \sum_{P \in X} \delta_P [P].
        \]
    \end{defn}

The following theorem has the classical Riemann-Hurwitz formula as a corollary, but also goes further, actually relating the canonical divisors on $X$ and $Y$.
    \begin{thm}\label{theoremdetailedhurwitz}
    If $0 \neq \omega \in \Omega_{K(Y)}$ then
        \begin{equation}\label{equationstronghurwitzformula}
        \di(\phi^*(\omega)) = \phi^*(\di(\omega)) + R.
        \end{equation}
    In particular, we have
        \[
        K_X \sim \phi^*(K_Y) + R.
        \]
    \end{thm}
    \begin{proof}
    See \cite[Chap.\ IV, \S 2, Prop.\ 2.3]{hart} for a sheaf theoretic approach, or alternatively \cite[Thm. 3.4.6]{stichtenoth}, for a proof involving function fields.
    \end{proof}


As a corollary to this we have the Riemann-Hurwitz formula.
    \begin{cor}[Riemann-Hurwitz Formula]\label{corhurwitzformula}
    Given two non-singular projective curves $X$ and $Y$ of genera $g_X$ and $g_Y$ respectively, with a degree $n$ map $\phi\colon X \rightarrow Y$, then
        \[
        2g_X - 2 = n(2g_Y -2) + \deg(R),
        \]
    where $R$ is the ramification divisor of $\phi$.
    \end{cor}
    \begin{proof}
    This follows from Corollary \ref{cordegreeofcanonicaldivisor} and Theorem \ref{theoremdetailedhurwitz}, by taking degrees in \eqref{equationstronghurwitzformula}.
    \end{proof}

Let $G$ be a finite subgroup of the automorphism group of $X$ (recall that if $g_X \geq 2$ then the automorphism group itself is finite).
The $G$ has a natural action on the function field of $X$, given by $g\cdot f(P) := f(g \cdot P)$ for every $P \in X$.
Then the quotient $Y := X/G$ of $X$ by the action of $G$ is again a curve (see \cite[Chap.\ 2, \S 1.7, Ex.\ 8]{pretendshafarevichalggeom1}), and the function field of the quotient curve is the subfield of $K(X)$ fixed by this action, which we denote $K(X)^G$.
We let $\pi \colon X \ra Y$ be the projection of $X$ on to the quotient.
Note that $G$ acts transitively on the fibres of $\pi$ (\textit{ibid.}).
We also recall that the stabiliser of a point $P \in X$ is the subgroup $G(P) : = \{ g \in G | g \cdot P = P\}$ of $G$.
The majority of the topics considered in this thesis will focus on situations where we have such a projection.


We now introduce the higher ramification groups, which we will use to state Hilbert's formula. 
This formula will then be used in subsequent chapters to compute the ramification divisor.

   \begin{defn}
    Let $G$ be finite subgroup of $\aut(X)$ and let $t$ be a uniformising parameter at $P\in X$.
    Then for $i\geq -1$ we define the \emph{$i^{th}$ ramification group} at $P$, denoted $G_i(P)$, to be the subgroup of $s\in G$ such that $i_G(s) := \ord_P(s(t)-t)$ is at least $i+1$.
    This is	independent of the choice of $t$, see \cite[Chap. IV, \S 1, pg. 62]{localfields}.
    \end{defn}

Note that for any $P\in X$ we have that $G_{-1}(P)=G$, $G_0(P)$ is the stabiliser of $P$ and $G_i(P)\supseteq G_{i+1}(P)$.
Also, $e_P = \ord (G_0(P))$ for any $P \in X$.
Less obviously, we have that $G_i(P)$ is trivial if $i$ is sufficiently large, that $G_1$ is a $p$-group and that $\ord(G_0(P)/G_1(P))$ is coprime to $p$ ---  see \cite[Chap. IV, \S 1]{localfields} for details.
In particular, $\phi$ is tamely ramified at $P$ if and only if $G_1(P)$ is the trivial group.



    \begin{thm}[Hilbert's Formula]\label{hilbertsformula}
    For every $P\in X$ we have
        \begin{equation*}
        \delta_P = \sum_{s\neq e}i_G(s)=\sum_{j=0}^{\infty}\left(\ord(G_i(P))-1\right),\label{ramdiv}
        \end{equation*}
    where $e$ denotes the identity in $G$.
    In particular, if $P$ is tamely ramified then $\delta_P = e_P -1$.
    \end{thm}
    \begin{proof}
    See \cite[Chap. IV, \S 1, Prop. 4]{localfields} for a proof of Hilbert's formula.
    \end{proof}

 
\section{Serre duality for curves}

In this section we give the details of Serre duality, in such a way that we will be able to perform explicit computations using Serre duality in later chapters.


We retain the notations of the previous sections.
We further let $\underline{\Omega}_{K(X)}$ and $\underline{K}(X)$ denote the constant sheaves of $\Omega_{K(X)}$ and $K(X)$ respectively, and $\Omega_X$ denote the sheaf of differentials on $X$.
The following lemma gives us useful descriptions of $\hone$ and $H^1(X,\Omega_X)$.
    \begin{lem}\label{lemmaexactsequenceofhoneandhzero}
    We have canonical exact sequences as follows:
        \begin{equation}\label{equationinitialdualitylesfunctions}
        0 \ra H^0(X,\cO_X) \ra K(X) \ra \bigoplus_{P \in X} K(X)/\cO_{X,P} \rightarrow \hone \ra  0;
        \end{equation}
        \begin{equation}\label{equationinitialdualitysesdifferentials}
        0 \rightarrow \hzero \ra \Omega_{K(X)} \ra \bigoplus_{P \in X}\Omega_{K(X)}/\Omega_{X,P} \rightarrow H^1(X,\Omega_X) \ra 0.
        \end{equation}
    \end{lem}
    \begin{rem}
    Note that a sketch of the proof below can be found in \cite[Pg.\ 248]{hart}.
        \end{rem}
    \begin{proof}
    The short exact sequence
        \begin{equation}\label{equationinitialdualitysesfunctions}
        0 \rightarrow \cO_X \rightarrow \underline{K}(X) \rightarrow \underline{K}(X)/\cO_X \rightarrow 0
        \end{equation}
    is a flasque resolution of $\cO_X$ (see \cite[Chap.\ II, ex.\ 1.16]{hart}).
    
    For each $P \in X$ we have a natural embedding $i\colon \{P\} \hookrightarrow X$, and we view the module $K(X)/\cO_{X,P}$ as a sheaf on the singleton $\{P\}$.
    Then for each $P\in X$ we have the induced sheaf $i_*\left( K(X)/\cO_{X,P} \right)$ on $X$.
    If we consider the direct sum of these induced sheaves over all points $P\in X$ we have the following isomorphism
        \begin{equation}\label{equationsheafisomorphism}
        \underline{K}(X)/\cO_X\cong \bigoplus_{P\in X} i_*\left(K(X)/\cO_{X,P}\right).
        \end{equation}
    
    
    To explain this isomorphism we first construct a map from $\underline{K}(X)/\cO_X$ in to the product $\prod_{P \in X} i_*\left({K(X)}/\cO_{X,P}\right)$, and then show that the image of each element under this map has finite support.
    
    Given $i\colon \{P\} \hookrightarrow X$ we have the following equalities
        \begin{align*}
        i^{-1}\left(\underline{K}(X)/\cO_X\right) & = \left(\underline{K}(X)/\cO_X\right)_P \\
        & = \underline{K}(X)_P/\cO_{X,P} \\
        & = K(X)/\cO_{X,P}.
        \end{align*}
    It follows that for any $P \in X$ we have the adjunction map $\underline{K}(X)/\cO_X \ra i_* \left( K(X)/\cO_{X,P} \right)$.
    Since we have an injection $i$ for every $P \in X$, we can actually produce a map in to the product $\prod_{P \in X} i_*\left(K(X)/\cO_{X,P}\right)$.
    Moreover, the stalk $i_*\left( K(X)/\cO_{X,P} \right)_Q$ is zero for $Q \neq P$ and is $K(X)/\cO_{X,P}$ when $Q = P$.
    Hence the product is actually isomorphic to the sum $\bigoplus_{P \in X} \left( i_*\left(K(X)/\cO_{X,P} \right)\right)$, and from this the isomorphism in \eqref{equationsheafisomorphism} follows.
    
    Replacing $\underline{K}(X)/\cO_X$ by $\bigoplus_{P\in X} i_*\left(K(X)/\cO_{X,P}\right)$ in \eqref{equationinitialdualitysesfunctions} yields
        \begin{equation}\label{equationshortexactsequencefunctionsdirectsum}
        0 \ra \cO_X \ra \underline{K}(X) \ra \bigoplus_{P \in X}i_* \left( K(X)/\cO_{X,P} \right) \ra 0.
        \end{equation}
    Taking cohomology we arrive at the exact sequence \eqref{equationinitialdualitylesfunctions}.
    
    We now perform a similar computation to produce the second exact sequence \eqref{equationinitialdualitysesdifferentials}.
    We start with the short exact sequence
        \[
        0 \ra \Omega_X \ra \underline{\Omega}_{K(X)} \ra \underline{\Omega}_{K(X)}/\Omega_X \ra 0,
        \]
    which is a flasque resolution of $\Omega_X$ (see \cite[Chap.\ II, ex.\ 1.16]{hart}).
    For each $P \in X$ we again have a natural injection $i \colon \{ P \} \hookrightarrow X$, giving rise to the induced sheaf $i_*\left( K(X)/\cO_{X,P} \right)$ on $X$.
    Then we have an isomorphism
        \begin{equation*}%\label{equationsheafisomorphismdifferentials}
        \underline{\Omega}_{K(X)}/\Omega_X \cong \bigoplus_{P \in X} i_* \left( \Omega_{K(X)}/\Omega_{X,P}\right),
        \end{equation*}
    similar to that in \eqref{equationsheafisomorphism}.

    Hence we arrive at the short exact sequence
        \begin{equation}\label{equationshortexactsequencedifferentialsdirectsum}
        0 \ra \Omega_X \ra \underline{\Omega}_{K(X)} \ra \bigoplus_{P \in X} i_* \left( \Omega_{K(X)}/\Omega_{X,P} \right)  \ra 0.
        \end{equation}
    Taking cohomology of this then yields the second exact sequence \eqref{equationinitialdualitysesdifferentials}.
    \end{proof}

    \begin{rem}
        When considering elements of $H^1(X,\Omega_X)$ as elements in the cokernel of the map $\Omega_{K(X)} \ra \bigoplus_{P \in X}\Omega_{K(X)}/\Omega_{X,P}$ above, we will denote them by $\overline{(\omega_P)}_{P \in X}$, where ${(\omega_P)}_{P \in X}\in \bigoplus_{P \in X}\Omega_{K(X)}/\Omega_{X,P}$.
        Similarly, when considering elements of $\hone$ as elements of the cokernel of the map $K(X) \ra \bigoplus_{P \in X}K(X)/\cO_{X,P}$, we will denote them by $\overline{(f_P)}_{P \in X}$, where ${(f_P)}_{P \in X} \in \bigoplus_{P \in X}K(X)/\cO_{X,P}$.
    \end{rem}

The residue map $\res_P \colon \Omega_{K(X)} \ra k$ is of fundamental importance in the computations that follow.
We define the {\em residue map}, $\res_P$, to be the unique map identified in the following theorem.

    \begin{thm}\label{theoremresiduemap}
    For any $P\in X$ there exists a unique $k$-linear map $\res_P \colon \Omega_{K(X)} \ra k$ defined by the following properties:
        \begin{itemize}
            \item $\res_P(\omega) = 0$ for all $\omega \in \Omega_{X,P}$;
            \item $\res_P(f^ndf) = 0$ for all $f \in K(X)^*$ and all $n \neq -1$;
            \item $\res_P(f^{-1}df) = \ord_P(f)$, where $\ord_P(f)$ is the order of $f$ at $P$.
        \end{itemize}
    \end{thm}
    \begin{proof}
    See \cite[Chap.\ II, \S 7 and \S 11]{algebraicgroupsandclassfields} or \cite{residuesofdifferentialsoncurve}.
    \end{proof}


    \begin{thm}[Residue Theorem]\label{theoremresiduetheorem}
    Given any differential $\omega \in \Omega_{K(X)}$ on $X$ then $\sum_{P \in X}\res_P(\omega) = 0$.
    \end{thm}
    \begin{proof}
    See \cite[Chap.\ II, Prop.\ 6]{algebraicgroupsandclassfields} or \cite[Pg.\ 155]{residuesofdifferentialsoncurve}.
    \end{proof}

Since $\Omega_{X,P} \subseteq \ker (\res_P)$, it follows that $\res_P$ is a well defined function on the quotient $\Omega_{K(X)}/\Omega_{X,P}$.
Hence by the residue theorem the map
    \begin{equation*} 
    \bigoplus_{P \in X} \Omega_{K(X)}/\Omega_{X,P} \ra k, \quad (\omega_P)_{P \in X} \mapsto \sum_{P\in X} \res_P(\omega_P)
    \end{equation*} 
vanishes on the image of $\Omega_{K(X)}$, which allows us to make the following definition.
    
    \begin{defn}
    Let $\overline{(\omega_P)}_{P \in X} \in H^1(X,\Omega_X)$.
    Then we define the \emph{trace map} to be 
        \[
    t \colon H^1\left(X, \Omega_X\right) \ra k, \qquad \overline{(\omega_P)}_{P \in X}  \mapsto \sum_{P \in X} \res_P(\omega_P).
        \]
    \end{defn}

We now use the trace map to define a pairing between the $k$-vector spaces $\hone$ and $\hzero$.
Since $\Omega_{K(X)}$ is a $K(X)$-module, we can define the product map 
    \begin{equation}\label{equationproductmap}
    \hzero \times \hone \ra H^1\left(X, \Omega_X\right), \ (\omega, \overline{(f_P)}_{P \in X}) \mapsto ( \overline {(f  \omega)_P})_{P \in X},
    \end{equation}
where $(f\omega)_P$ denotes the product of $f_P \in K(X)/\cO_{X,P}$ with the residue class of $\omega$ in $\Omega_{K(X)}/\Omega_{X,P}$.

We now combine the product map in \eqref{equationproductmap} with the trace map $t$ to get a map 
    \[
    \hzero \times \hone \ra k,\quad (\omega, \overline{(f_P)}_{P \in X}) \mapsto \langle \omega, \overline{(f_P)}_{P \in X} \rangle := t \left( \overline{(f \omega)}_P \right)_{P \in X}.
    \]

    \begin{thm}\label{theoremserreduality}
    Via the pairing $\langle\ ,\ \rangle$, the $k$-vector spaces $\hone$ and $\hzero$ are dual to each other.
    \end{thm}
    \begin{proof}
    This is a specialisation of \cite[Chap.\ II, Thm.\ 2]{algebraicgroupsandclassfields}.
    \end{proof}

More explicitly, this theorem means the following.
If we fix any $\omega \in \hzero$ we produce a map $\theta(\omega)\colon \hone \ra k$, given by $\theta(\omega)(f) = \langle \omega , f\rangle$.
Similarly, if we fix any $f \in \hone$ then we get a map $\psi(f) \colon \hzero \ra k$.
Then the maps 
    \[
    \psi \colon \hone \ra \hom(H^0(X,\Omega_X),k) \quad
    \text{and} \quad    
    \theta \colon H^0(X,\Omega_X) \ra \hom(\hone, k)
    \]
are isomorphisms.
In particular, given a $k$-basis $\omega_1, \ldots, \omega_g$ of $\hzero$, we can find a basis $f_1, \ldots , f_g$ of $\hone$ such that $\langle \omega_i, f_j \rangle = 1$ for all $1 \leq i \leq n$ and $\langle \omega_i, f)j \rangle = 0$ if $i \neq j$, and likewise, starting with a basis of $\hone$ we can find corresponding basis of $H^0(X,\Omega_X)$. 





















