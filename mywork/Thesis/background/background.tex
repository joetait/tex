\chapter{Background}\label{chapterbackground}

In this section we will introduce the basic concepts, notation and terminology required in the study of algebraic curves.

Throughout this thesis $k$ will denote an algebraically closed field of characteristic $p \geq 0$.
It should be noted that while the majority of results in the thesis hold for all characterisitcs, including $p=0$, our main focus will be on the case $p > 0$.

When we refer to a \emph{curve} we will mean a smooth, connected, projective variety of dimension one over $k$.
% Similarly, when we refer to an \emph{affine curve} we mean a smooth, connected, affine variety of dimension one over $k$. % Add if needed


A \emph{meromophirc function} on $X$ is a morphism $f \colon X \ra \PP_k^1$.
The collection of meromorphic functions on $X$ is denoted $K(X)$, and called the \emph{function field} of $X$.

We recall that the category of algebraic curves up to birational  equivalence is actually equivalent to the category of function fields over $k$ (which can be defined independantly of curves as fields of transcendence degree one over $k$).
Furthermore, if $k = \CC$ then we have a triple equivalence of categories, between function fields over $\CC$, algebraic curves (up to biratioanl equivalence) and the category of compact Riemann surfaces.\todo{References. Maybe Miranda, Stichtenoth and/or Hartshorne.}

We say that a meromorphic function on $X$ is \emph{regular} on an open set $U \subseteq X$ if the image $f(U)$ lies in $k = \AA_k^1 \subset \PP_k^1$.
We let $H^0(U,\cO_X)$ denote the space of functions in $K(X)$ which are regular on $U$.
Moreover, if $f \in K(X)$ is regular on the whole curve we say that $f$ is \emph{regular}, and then $H^1(X,\cO_X)$ is the space of regular functions.

For any $P \in X$ a meromorphic function $f \in K(X)$ is \emph{regular at $P$} if $f(P) \in k \subset \PP_k^1$.
The collection of functions regular at $P$ form a ring, which we call $\cO_{X,P}$.

    \begin{lem}
    For any $P \in X$ the ring $\cO_{X,P}$ is a discrete valuation ring, with maximal ideal
        \[
        \mathcal {M}_{X,P} := \{ f \in \cO_X,P} | f(P) = 0 \}.
        \]
    \end{lem}
    \begin{proof}
    \todo[inline]{Proof or reference here}
    \end{proof}

The valuation on $\cO_{X,P}$ can be given as follows.
Let $t \in \cO_{X,P}$ be a generator of $\mathcal{M}_{X,P}$.
Now any $f \in \cO_{X,P}$ can be written as $f = ut^n$ for some unique $n \in \ZZ$ and some unit $u \in \cO_{X,P}\backslash \mathcal{M}_{X,P}$.
We then define \emph{the order of $f$ at $P$} to be $v_P(f) := n$.
We call any element $t \in \cO_{X,P}$ with order 1 at $P$ a \emph{uniformising parameter at $P$}.

\section{Differentials}

We now introduce the concept of a differential on the xurve $X$.

Let $R$ be any commutaitive ring containing $k$ and let $M$ be an $R$-module.
Then a $k$-linear map $D \colon R \ra M$ satisfying $D(fg) = fD(g) + gD(f)$ is called a \emph{derivation} of $R$ in to $M$ over $k$.

There exists a unique module $\Omega_k(R)$, and a derivation $d \colon R \ra \Omega_k(R)$, through which all derivations must factor.
We can describe this more concretely as the free module generated by $[f]$ for all $f \in K(X)$, quotiented by the relations
    \begin{itemize}
    \item $[f]+[g] = [f+g]$;
    \item $[cf] = c[f]$;
    \item $[fg] = f[g] + g[f]$;
    \end{itemize}
where $f, g \in R$ and $c \in k$.

In particular, when $R = K(X)$ we let $\Omega_{K(X)} := \Omega_k(K(X))$, and we call the map $d \colon K(X) \ra \Omega_{K(X)}$ the \emph{differential map}.
We then call $\Omega_{K(X)}$ the \emph{module of differentials of $X$}, and we call any $ \omega \in \Omega_{K(X)}$ a \emph{meromorphic differential}.

    \begin{prop}\label{propdifferentialsareonedimensional}
    The module of differentials, $\Omega_{K(X)}$, is a one dimensional vector space over $K(X)$.
    \end{prop}
    \begin{proof}
    See \cite[Prop. 1.5.9]{stichtenoth}.\todo{check citation}
    \end{proof}

We suppose that $P \in X$ we choose a uniformising parameter $t \in \cO_{X,P}$.
Then for any $ 0 \neq \omega \in \Omega_{K(X)}$ there exists a unique integer $n \in \ZZ$ and unit $u \in \cO_{X,P}$ such that $\omega = ut^ndt$.
Then we define the \emph{order of $\omega$ at $P$} to be $\ord_P(\omega) := n$.
Note that this is well defined by Proposition \ref{propdifferentialsareonedimensional}.

We call differential on an open subset $U \subset X$ \emph{holomorphic on $U$} if $\ord_P(\omega) \geq 0$ for all $P \in U$, and we let
    \[
    H^0(U, \Omega_X) := \{ \omega \in \Omega_{K(X)} | \ord_P(\omega) \geq 0\ \text{for all}\ P \in X\}
    \]\todo{check spacing of ''for all"}
be \emph{the space of holomorphic differentials on $U$}.
If $\omega \in \Omega_{K(X)}$ is homolorphic on $X$ we say that $\omega$ is \emph{holomorphic}, and we let $\hzero$ be \emph{the space of holomorphic differentials}.

\section{The Riemann-Roch theorem}

We now recall the relevant facts and definitions needed to state the Riemann-Roch theorem.

We first recall that a \emph{divisor} on $X$ is a finitely supported formal sum 
    \[
    D = \sum_{P \in X} n_P[P].
    \]
The set off all divisors forms an additive group, $\di (X)$.
The \emph{degree} of the divisor $D$ is $\deg(D) = \sum_{P \in X} n_P$.
Given any function $f \in K(X)$ we have the \emph{divisor associated to $f$}, which is
    \[
    \di(f) := \sum_{P \in X} \ord_P(f) [P].
    \]
Note that this is well defined by Lemma. \todo{write and cite lemma about finitely many poles and zeroes}
We call any divisor $D$ which is equal to $\di(f)$ for some $f \in K(X)$ a \emph{principal divisor}.
Is is clear that for any $f, g \in K(X)$ we have $\di(fg) = \di(f) + \di(g)$.\todo{make same remark about orders}
Also, for any $f \in K(X)$ we define $\di_0(f)$ and $\di_\infty (f)$, the \emph{divisor of zeroes} and the \emph{divisor of poles} of $f$ respectively, as follows:
    \[
    \di_0(f) := \sum_{\ord_P(f) >0} \ord_P(f)[P]
    \]
and then
    \[
    \di_\infty(f) := \di_0(f) - \di(f).
    \]

Now for any differential $0 \neq \omega \in \Omega_{K(X)}$ we define the \emph{divisor associated to $\omega$} to be
    \[
    \di(\omega) := \sum_{P \in X} \ord_P(\omega).
    \]
If $W$ is a divisor on $X$ and $W = \di(\omega)$ for some $ \omega \in \Omega_{K(X)}$ then we say that $W$ is a \emph{canonical divisor} on $X$.

The principal divisors of $X$ form a subgroup of $\di(X)$, and we say that two divisors $D, D' \in \di(X)$ are equivalent, denoted $D \tilde D'$, if their image in the quptiemt of $\di(X)$ by the group of principal divisors is the same; \ie if there exists $f \in K(X)$ such that $ D = D' + \di(f)$.
By the following, it makes sense to refer to \emph{the} canonical divisor on $X$, up to equivalnce, which we write as $K_X$.

    \begin{thm}
    Let $W$ and $W'$ be canonical divisors on $X$.
    Then there exists some $f \in K(X)$ such that $W = W' + \di(f)$.
    Moreover, the divisor $W + \di(f)$ is canonical for every $f \in K(X)$.
    \end{thm}
    \begin{proof}
    The first statement follows from the fact that $\Omega_{K(X)}$ is a $K(X)$ and from the definition of canonical divisor.
    Suppose that $\omega, \omega' \in \Omega_{K(X)}$ are the differentials such that $W = \di(\omega)$ and $W' = \di(\omega')$.
    Then, by Proposition \todo{reference one dim prop} there exists an $f \in K(X)$ such that $\omega = f \omega'$.
    Then the first statement follows once we take the divisor of both sides.
    \end{proof}

Given any divisor $D = \sum_{P \in X} n_P[P]$ we let
    \[
    H^0(X,\cO_X(D)) : = \{ f \in K(X) | \ord_P(f) \geq n_P\ \text{for all}\ P \in X\}
    \]
be the \emph{vector space of meromorphic functions associated to $D$}.
Similary, we let 
    \[
    H^0(X,\Omega_X(D)) :  = \{ \omega \in \omega_{K(X)} | \ord_P(\omega) \geq n_P \ \text{for all} \ P \in X\}
    \]
be the \emph{vector space of meromorphic differentials associated to $D$}.\todo{check spacing of ''for all" in sets}
In particular, when $D$ is the zero divisor we have $H^0(X,\Omega(0)) = \hzero$, and similarly $H^0(X,\cO_X(0)) = H^0(X,\cO_X)$.

We now state the celebrated Riemann-Roch theorem.

    \begin{thm}[Riemann-Roch theorem]
    Let $D$ be any divisor on $X$, and let $W$ be any canonical divisor on $X$.
    Then
        \[
        \dim_k(H^0(X,\Omega_X(D))) = \deg(D) + 1 - g_X + \dim_k(H^0(X,\Omega)X(W-D)), 
        \]
    for some constant $g_X \in \ZZ_{\geq 0}$, which is independent of the choice of $W$ and $D$.
    \end{thm}
    \begin{proof}
    See, for example, \todo{reference hartshorne and fulton}
    \end{proof}

    \begin{defn}
    We call the constant $g_X$ in the statement of the Riemann-Roch theorem the genus of $X$.
    \end{defn}

The genus is an invariant of fundamental importance in the study of algebraic curves.
In particular, we remark that if $k = \CC$ then the genus of an algebraic curve (also called the arithmetic genus) is the same as the topological genus of the corresponding Riemann surface.

We now give a corollary to the Riemann-Roch theorem, which shows some of the properties of the genus.

    \begin{cor}\label{dim=gc}
    For any canonical divisor $W$ on $X$, we have 
        \[
        \deg(W) = 2g_X-2
        \]
    and 
        \[
        \dim H^0(X,\cO_X(W)) = g_X.
        \]
    \end{cor}
    \begin{proof}
    Since $\dim H^0(X,\cO_X(0)) = 1$ (the only functions with no poles are the constant functions), we have by the Riemann-Roch theorem, 
        \[
        1= \dim H^0(X,\cO_X(0)) = 0 + 1 -g_X + \dim H^0(X,\cO_X(W)).
        \]
    Rearranging this gives the second statement.
    The first statement then follows by rearranging
        \begin{multline*}
        g_X = \dim H^0(X,\cO_X(W)) = \deg(W) + 1 -g_X +  \dim H^0(X,\cO_X(W-W))\\ = \deg(W) + 1 -g_X + 1.
        \end{multline*}
    \end{proof}

\section{Ramification and the Hurwitz formula}

In this section we will introduce the concept of ramification, and conclude by stating the Hurwitz formula, relating the genera of two curves with a map between via the ramification divisor.



Let $X$ and $Y$ be curves over $k$.
We start doing this by defining what it means for a map $\phi \colon X \ra Y$ to be a morphism of curves.
We first note that given an map of sets $\phi \colon X \ra Y$ we have an induced ring homomorphism on the function fields,
    \[
    \phi^* \colon K(Y) \ra K(X),
    \]
given by composition with $\phi$.\todo{define $\phi^*$ in words}

Then a \emph{morphism of curves} is a continuous map $\phi \colon X \ra Y$ such that for every open set $U \subseteq Y$ we have
    \[
    \phi^*(H^0(U,\cO_Y)) \subseteq H^0(\phi^{-1}(U),\cO_X).
    \]

A particular manner in which morphisms often arise comes from considering the quotient of $X$ by some subgroup $G$ of the automorphism group $\aut(X)$.
Then we have a natural projection $\pi \colon \ra X/G$, which is a morphism of curves.
The majority of the topics considered in this thesis will 






















