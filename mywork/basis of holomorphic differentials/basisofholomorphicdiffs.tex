% !TEX TS-program = pdflatex
% !TEX encoding = UTF-8 Unicode

% This is a simple template for a LaTeX document using the "article" class.
% See "book", "report", "letter" for other types of document.

\documentclass[11pt]{article} % use larger type; default would be 10pt

\usepackage[utf8]{inputenc} % set input encoding (not needed with XeLaTeX)

%%% Examples of Article customizations
% These packages are optional, depending whether you want the features they provide.
% See the LaTeX Companion or other references for full information.

%%% PAGE DIMENSIONS
\usepackage{geometry} % to change the page dimensions
\geometry{a4paper} % or letterpaper (US) or a5paper or....
% \geometry{margins=2in} % for example, change the margins to 2 inches all round
% \geometry{landscape} % set up the page for landscape
%   read geometry.pdf for detailed page layout information

 \usepackage{graphicx} % support the \includegraphics command and options

\parindent0ex % Activate to begin paragraphs with an empty line rather than an indent

%%% PACKAGES
\usepackage{mathtools}
\usepackage{booktabs} % for much better looking tables
\usepackage{array} % for better arrays (eg matrices) in maths
\usepackage{paralist} % very flexible & customisable lists (eg. enumerate/itemize, etc.)
\usepackage{verbatim} % adds environment for commenting out blocks of text & for better verbatim
\usepackage{subfig} % make it possible to include more than one captioned figure/table in a single float
% These packages are all incorporated in the memoir class to one degree or another...

%%% HEADERS & FOOTERS
\usepackage{fancyhdr} % This should be set AFTER setting up the page geometry
\pagestyle{fancy} % options: empty , plain , fancy
\renewcommand{\headrulewidth}{0pt} % customise the layout...
\lhead{}\chead{}\rhead{}
\lfoot{}\cfoot{\thepage}\rfoot{}

%%% SECTION TITLE APPEARANCE
\usepackage{sectsty}
\allsectionsfont{\sffamily\mdseries\upshape} % (See the fntguide.pdf for font help)
\usepackage{amsmath}
\usepackage{amsthm}
\usepackage{amsfonts}
\usepackage{mathrsfs}
\usepackage{amsopn}
\usepackage{amssymb}
\usepackage{natbib}
% (This matches ConTeXt defaults)

%%% ToC (table of contents) APPEARANCE
\usepackage[nottoc,notlof,notlot]{tocbibind} % Put the bibliography in the ToC
\usepackage[titles,subfigure]{tocloft} % Alter the style of the Table of Contents
\renewcommand{\cftsecfont}{\rmfamily\mdseries\upshape}
\renewcommand{\cftsecpagefont}{\rmfamily\mdseries\upshape} % No bold!

%Theorems and stuff
\theoremstyle{plain}
\newtheorem{defn}{Definition}[section]
\newtheorem{thm}[defn]{Theorem}
\newtheorem{cor}[defn]{Corollary}
\newtheorem{lem}[defn]{Lemma}
\newtheorem{prop}[defn]{Proposition}
\newtheorem{ex}[defn]{Example}
\newtheorem*{unnumthm}{Theorem}
\newtheorem{defnlem}[defn]{Definition/Lemma}
\newtheorem{defnthm}[defn]{Theorem/Definition}
\theoremstyle{remark}
\newtheorem*{rem}{Remark}


\newcommand{\cO}{{\cal O}}
\newcommand{\ra}{\rightarrow}
\newcommand{\NN}{{\mathbb N}}
\newcommand{\PP}{{\mathbb P}}
\newcommand{\ZZ}{{\mathbb Z}}
\newcommand{\cL}{{\mathcal L}}
\newcommand{\cA}{{\mathcal A}}
\newcommand{\cD}{{\mathcal D}}


\DeclareMathOperator{\aut}{Aut}
\DeclareMathOperator{\ord}{ord}
\DeclareMathOperator{\di}{div}
\DeclareMathOperator{\cha}{char}
\DeclareMathOperator{\gal}{Gal}
\DeclareMathOperator{\Tr}{Tr}

%%% END Article customizations

%%% The "real" document content comes below...

\title{More hyperelliptic curves}
\author{J Tait}
%\date{} % Activate to display a given date or no date (if empty),
         % otherwise the current date is printed 

\begin{document}
\maketitle

  \subsection{Hyperelliptic curves, characteristic unequal to 2}\label{charneq2}
  Let $X$ be a smooth, projective, connected hyperelliptic algebraic curve of genus $g$ over an algebraically closed field $k$, with $\cha (k) \neq 2$.
Let $x:X\rightarrow \mathbb{P}_k^1$ be the corresponding holomorphic map of degree $2$.
This curve then has a corresponding degree two extension $k(x,y)$ of the rational function field in one variable over $k$, $k(x)$, where
\begin{equation}\label{definingequation}
 y^2 = f(x)
\end{equation}
for some $f(x)\in k[x]$ with no repeated roots (see, for example, \cite[7.4.3]{liu}).
Also, the degree of $f(x)$ must be $2g+1$ or $2g+2$, as will be shown in the next paragraph.
In the first case then $\infty \in \mathbb P_k^1$ will be a branch point, where as in the second case it is not.
However, since there are only a finite number of branch points, if $\infty$ is ramified we can find an automorphism of $\mathbb P_k^1$ which will map $\infty$ to a different point, and not map any of the other branch points to $\infty$.
So we can assume that $\infty$ is not a branch point, and that the degree of $f(x)$ is $2g+2$.

We now show some preliminary results about $x$. 
Firstly, $x$ is ramified at precisely $2g + 2$ ramification points.
This follows from computing the degree of the ramification divisor $R$ of $x$.
By the Riemann-Hurwitz formula, Corollary \ref{hur},
\[ 
\deg(R) = 2g -2 +2\cdot 2 = 2g + 2.
\]
Since $x$ is of degree $2$ and at most tamely ramified, the coefficients of the ramification divisor are $1$ (see the start of Section \ref{dimsection}), and hence $R = p_1 + \ldots + p_{2g+2}$, for distinct $p_i\in X$.
We let $a_i = x(p_i)$.
Since $\infty \notin \{a_1,\ldots ,a_{2g+2}\}$ then the divisor of poles of $x$ is $\di_{\infty}(x) = p + q$ for some $p \neq q $ in $X$.
We now show that the ramification points correspond to the zeroes of $f(x)$.
Since there are no repeated roots in $f(x)$, then \eqref{definingequation} defines a non-singular affine curve $X'$, with a degree two projection $x': X'\rightarrow \mathbb A_k^1$.
Clearly for any point $a\in \mathbb A_k^1$ which is not a solution to $f(x)$, then there are two points in the pre-image, namely $(a,\pm \sqrt{a})$, and the point is not ramified.
On the other hand, if $a\in \mathbb A_k^1$ is a solution to $f(x)$, then there is only one point in the pre-image and hence it is ramified.
Since there are $2g+2$ solutions of $f(x)$, and the degree of $R$ is $2g+2$, then we conclude that these are all the ramification points.
\begin{comment}
We can define $j\in \aut(X)$ to be the map defined $q_1 \mapsto q_2$ for $q_1, q_2 \in x^{-1}(a)$. 
Obviously this is the identity on the ramification points, and since $x$ is of degree $2$, $j^2$ is the identity map.
We will now use this to show the existence of one more function, $y$, which is necessary to define the basis of $H^0(X,\Omega_X^{\otimes m})$.

If we let $D$ be the divisor $(g+1)p + (g+1)q$ on $k$, then there exists a $y \in \cL(D)$ (unique up to a factor) such that $j^* (y) = -y$.
To start, we have by the Riemann-Roch theorem that $l(D) = \dim\cL(D) = (2g+2)-g+1 = g+3$.
So $\cL(D) \cong k^{g+3}$.
Since $x(p) = x(q) = \infty$ then $j^*$ defines a linear map $\cL(D) \rightarrow \cL(D)$.
Since $j^{*2} = 1$, the eigenvalues must be $\pm 1$, and as such we can decompose $\cL(D)$ in two subspaces, $\cL(D)^+$ and $\cL(D)^-$, corresponding to the eigenspaces of eigenvalues $1$ and $-1$ respectively.
Note that $\cL(D) = \cL(D)^+ \oplus \cL(D)^-$.

If $f\in \cL(D)^+$ then this means that $j^*(f(q))=f(j(q))=f(q)$ for all $q \in X$.
But $x$ maps $q$ and $j(q)$, and only these two points, to the same point in $\mathbb P_k^1$ for all $q\in X$.
So any meromorphic function in $\cL(D)^+$ can be written as a composition of $x$ and a meromorphic function on $\mathbb P_k^1$, such that the composition only has poles at $p$ and $q$ (i.e. the meromorphic function on $\mathbb{P}_k^1$ only has poles at infinity.)
The order of the pole at $p$ or $q$ cannot exceed $g+1$, hence $1,x,\ldots ,x^{g+1}$ forms a basis of $\cL(D)^+$.
As this implies that $\dim\cL(D)^+ = g+2$, and as $\dim\cL(D) = g+3$, we see that $\dim\cL(D)^- = 1$, and so there is a non-trivial meromorphic function $y \in \cL(D)$ such that $j^*(y) = -y$.
This is the $y$ we will use in the following proposition.\\
\end{comment}


Next we will need to define precisely what the divisor of a poly-differential is.
If we consider an element of the tensor product $\omega \in \Omega_X^{\otimes m}$ then it can be locally written as $y dx_1\otimes \ldots \otimes dx_m$, where $x_i \in K(X)$ for all $1 \leq i \leq m$.
Let $P$ be a point in $X$.
Since each $dx_i$ can be written as $y_i dt$ for some $y_i\in K(X)$ and some uniformising parameter $t$ at $P$, we can rewrite $\omega$ as $y' dt \otimes \ldots \otimes dt$, where $y' = y \cdot y_1 \cdots y_m$.
We then define the order of $\omega$ at $P$ to be $\ord_P(\omega ) := \ord_P(y')$.
In the particular case where $\omega = fdx \otimes \ldots \otimes fdx = f^m dx^{\otimes m}$, then we have $y_1 = \ldots = y_m = z$ for some $z$ when we change $x$ to a uniformising parameter.
Hence in this instance \[ \ord_P(\omega) = \ord_P(z^m) = m\ord_P(z) = m\ord_P(dx).\]
\vskip1em



\begin{prop}\label{propneq2}
Let $m\geq 1$.
Let $X$, $x$ and $y$ be as above, and let $\omega := \frac{dx^{\otimes m}}{y^m}$. 
Then if $g\geq 2$, a basis of $H^0(X,\Omega_X^{\otimes m})$ is given by


$\begin{cases}
 \omega, x\omega, \ldots , x^{g-1}\omega &  \mbox{if}\ m=1 \\
 \omega, x\omega, x^2\omega & \mbox{if}\ m=g=2 \\
 \omega, x\omega, \ldots, x^{m(g-1)}\omega;\  y\omega, xy\omega, \ldots x^{(m-1)(g-1)-2}y\omega & \mbox{otherwise.}
 \end{cases}
$

\end{prop}

\vskip1em

\begin{rem}
 Note that the case where $m=1$ is treated in \cite[7.4.3]{liu} and \cite[$\S$4, Ch. IV]{griffiths}.
\end{rem}

\begin{proof}
We first show that the elements are linearly independent over $k$.
Since $\omega$ is fixed, it is equivalent to show that the coefficients are linearly independent over $k$ - i.e. that $1,x,\ldots ,x^n, y, xy, x^ly$ are linearly independent over $k$ for any $n$ and $l$.
But the only part of this which is not clear is when the coefficients are not just in $x$, but then it follows since the extension would otherwise be of degree 1.
To show that they are indeed holomorphic differentials, we show that their divisors are greater than $0$.
We first compute the divisors associated to $x$, $y$ and $dx^{\otimes m}$.
\begin{comment}
For any $f\in K(X)$ we will denote by $(f)_0$ and $(f)_\infty$ the divisor of zeroes and divisor of poles of $f$ respectively.
In particular, 
\[
          (f)_0 := \sum_{\{P\in X|\ord_P(f)>0\}}\ord_P(f)[P],      
 \]
 and 
 \[
 (f)_\infty = \sum_{\{P\in X |\ord_P(f)<0\}} -\ord_P(f)[P].
 \]
\end{comment}

Let $D_0 = [p']+[q']$ with $p',q' \in X$ (note that we could have $p' = q'$) be the divisor of zeroes of $x$. 
Then $ i D_0$ is the divisor of zeroes of $x^i$. 
If we also let $D_\infty = [p] + [q]$, with $p$ and $q$ defined as above, be the divisor of poles of $x$. 
So overall $\di (x^i) = i D_0 - i D_\infty$.


To compute $\di (dx^{\otimes m})$, we first note that it suffices to compute the divisor of $dx$, since $\di (dx^{\otimes m}) =m\di (dx)$, as noted above.
Since $x$ can viewed as the projection of $X$ on to the projective line, or as a function on the projective line, we will use $\di_X (dx)$ and $\di_{\mathbb P^1} (dx)$ to differentiate these cases.
We first recall that the Riemann-Hurwitz formula (Theorem \ref{detailedhurwitz}) tells us that
\[
 \di_X (dx) = \pi^*( \di_{\mathbb P^1}(dx)) + R = R - 2D_{\infty},
\]
since $\di_{\mathbb P^1}(dx) = -2[\infty]$ and hence $\pi^* (\di_{\mathbb P^1}(dx)) = -2D_\infty$.

Finally, we compute $\di (y)$.
Since $\di (y^2) = \di (f(x))$ and hence $\di(y) = \frac{1}{2}\di(f(x))$, we need only compute the divisor of $f(x)$.
As noted earlier, the solutions to $f$ correspond to the ramification points.
So for any $P\notin x^{-1}(\infty)$ then $\ord_P(y) =  \frac{1}{2}\ord_P(f(x)) = 1$ if $P$ is a ramification point, and $\ord_P(y) = \frac{1}{2}\ord_P(f(x)) = 0$ otherwise.
Finally, if $P\in x^{-1}(\infty)$, then $\ord_P(y)  = \frac{1}{2}\ord_P(f(x)) = - \frac{1}{2} (2g+2) = -(g+1)$.
So, overall, we have 
\[
 \di(y) = \sum_{i=1}^{2g+2} [p_i]- (g+1)D_\infty = R - (g+1)D_{\infty}.
 \]
\begin{comment}
Firstly, since $y(p_i)= y(j(p_i)) = j^*(y)(p_i)= -y(p_i)$ for all $i$ we see that $y(p_i)=0$.
Therefore $\deg(y)_0 \geq \deg\left(\sum_{i = 1}^{2g +2} [p_i] \right) = 2g+2$.
But since $y\in \cL(D)$, we know that $\deg(y)_{\infty} \leq \deg((g+1)D) = 2g + 2$, and as $\deg(y) = \deg(y)_0 - \deg(y)_{\infty} = 0$ then $\deg(y)_0 = \deg(y)_{\infty} = 2g+2$.
So
\[
 (y)_0 = \sum_{i=1}^{2g+2} [p_i], \ (y)_{\infty} = (g+1)D_\infty.
\]
\end{comment}

We now show that the differentials listed in Proposition \ref{propneq2} are holomorphic.
We consider the first two cases together.
So if $m=1$ or $m\geq 2$, $g\geq 2$ then we have that
\begin{eqnarray*}
 \di(x^i\omega) & = & \di \left( \frac{x^idx^{\otimes m}}{y^m} \right)\\ & = & i(D_0 -D_\infty) + m(R-2D_\infty) -m(R-(g+1)D_\infty) \\
 & = & iD_0 + (mg -m -i)D_\infty \\
 & = & iD_0 + (m(g-1) -i)D_\infty,
\end{eqnarray*}
which is positive for $0\leq i \leq m(g-1)$.
Note that if $m=g=2$ then there are three elements, and since $\dim_kH^0(X,\Omega_X^{\otimes 2})=3$ by Lemma \ref{dim3}, these elements form a basis.
Also, if $m=1$ then by Riemann-Roch $\dim_k H^0(X,\Omega_X)=g$, and we have $g$ linearly independent elements, so they again must form a basis.

\begin{comment}
If $m=g=2$ then we have that
\begin{eqnarray*}
 \di\left(\frac{x^idx^{\otimes 2}}{y^2}\right) & = & i(D_0-D_\infty) + 2(R-2D_\infty) - 2(R-(g+1)D_\infty) \\
 & = & iD_0+ ((2g-2)-i)D_\infty \\
 & = & iD_0 +(2-i)D_\infty,
\end{eqnarray*}
which is positive for $0\leq 2$.
By Lemma \ref{dim3} then $\dim_kH^0(X,\Omega_X^{\otimes 2})=3$, so again, we have three linearly independent elements, so they must form a basis.
\end{comment}


Now we consider the case where $m\geq 2$ and $g \geq 2$ and at least one of these inequalities is strict.
\begin{eqnarray*}
 \di (x^iy\omega) & = & \di(x^i\omega) + R -(g+1)D_\infty \\
 & = & iD_0 + R +((m-1)(g-1)-2-i)D_\infty,
\end{eqnarray*}
which is positive for $0\leq i \leq (m-1)(g-1)-2$.
By Lemma \ref{dim3} we know that 
\[
\dim_kH^0(X,\Omega_X^{\otimes m}) = (2m-1)(g-1).
\]
Since the number of differentials listed in the last case of the proposition is precisely
\[
 (m-1)(g-1)-1 + m(g-1) +1 = 2mg -2m -g + 1 = (2m-1)(g-1)
\]
it is clear that these elements form a basis.
\end{proof}

We denote by $\sigma$ the automorphism of $X$ of order 2, which maps each point to its corresponding point in the pre-image of $x$.
Since $\sigma(y) = -y$, and $\sigma$ acts trivially on $x$, we can see that in the case where $m=g=2$ the action will be trivial, since the only power of $y$ is two, and of course $y^2 = (-y)^2$.
In any other case we have an odd power of $y$ in the basis, and hence the action of $\sigma$ is not trivial.
This proves the main theorem for hyperelliptic curves over a field of characteristic not equal to 2.




\subsection{Hyperelliptic curves, characteristic 2}
Let $X$ be a smooth, projective, connected hyperelliptic algebraic curve of genus $g$ over an algebraically closed field $k$ of characteristic two.
Let $\pi:X \rightarrow \mathbb P_k^1$ be the corresponding holomorphic map of degree two.
By \cite[\S 7.4.3]{liu} this curve has a corresponding degree two extension $k(x,y)$ of the function field of one variable over $k$, $k(x)$, where
\begin{equation}\label{ext}
  y^2 - h(x)y = f(x)
\end{equation}
for some polynomials $h(x), f(x)\in k[x]$, with maximum degrees of $g+1$ and $2g+2$ respectively.
We now show what conditions the smoothness of the curve imposes on $f(x)$ and $h(x)$.
\vskip1em



\begin{lem}\label{smoothness}
Since the curve is smooth we have $h(x)$ and $h'(x)^2 f(x) + f'(x)$ have no common zeroes on $X$.
\end{lem}
\begin{proof}
 The Jacobian criterion (see, for example, \cite[Thm. 2.19]{liu}), states that if the partial derivatives of \eqref{ext} with respect to $x$ and with respect to $y$ are zero at a point $P\in X$ then the curve is not smooth, and otherwise it is.
 Clearly \[
          \frac{\delta}{\delta y} (y^2 -h(x)y -f(x)) = h(x)
         \]
 since the characteristic of $k$ is 2.
 On the other hand,
 \begin{equation}\label{xderivative}
  \frac{\delta}{\delta x} (y^2 - h(x)y -f(x)) = h'(x)y - f'(x).
 \end{equation}
 Since \eqref{xderivative} is zero if and only the square of \eqref{xderivative} is zero, then we need only consider the common zeroes of $h(x)$ and
\[
 (h'(x)y-f'(x))^2 = h'(x)^2y^2 -f'(x)^2 = h'(x)^2h(x)y + h'(x)^2f(x) - f'(x)^2.
 \]

 It then follows from the Jacobian criterion and that $X$ is smooth that $h(x)$ and $h'(x)^2f(x) + f'(x)$ can have no common zeroes on $X$. 
\end{proof}


We now introduce some notations. 
Let $a\in \mathbb P_k^1$.
If $a$ is a branch point then we denote the point in the pre-image by $P_a$, and otherwise we denote the two points int he pre-image by $P_a'$ and $P_a''$.
To simplify future calculations we also define $D_a := \pi^*([a])$.
We let $\sigma$ be the order two automorphism of $X$ which fixes $P_a$ if $a$ is a branch point, and otherwise permutes $P_a'$ and $P_a''$.
Note that $X/\langle \sigma \rangle \cong \mathbb P_k^1$.
We let $d$ be the degree of $h(x)$, and denote the number of distinct roots in $\mathbb A_k^1$ of $f(x)$ and $h(x)$ by $l$ and $k$ respectively.



We now describe the ramified points of $\pi$, in order to compute the ramification divisor.
The affine curve defined by \eqref{ext}, which we will denote by $X'$, will be smooth by Lemma \ref{smoothness}.
Then $\pi$ restricts to a map $X'\rightarrow \mathbb A^1_k$, the projection on to the $x$ co-ordinate.
For any $a\in \mathbb A_k^1$, if $(a,b)$ is in $\pi^{-1}(a)$ then so is $(a,b+h(a))$.
It is clear that $a,\b)$ and $a,b+h(a))$ are distinct if and only if $h(a)\neq 0$.
Since the extension is degree two, this shows that the ramified points in the affine part correspond to the roots of $h(x)$.
We denote the zeroes of $h(x)$ by $a_i$ for $1\leq i \leq k$.
For each $a_i$ there is a corresponding $b_i$, which is the square root of $f(a_i)$.
We will also denote the corresponding ramification point by $P_i := P_{a_i}$, and the divisors $D_{a_i}:=D_i$.
{\bf Removed the definition of $P_\infty$, since $P_a$ has been defined - I assume it is reasonable to think that this is clear}

We will now compute the ramification divisor.
\vskip1em


\begin{lem}\label{char2ramification}
 Let $n_i$ be the order of $h(x)$ at $a_i\in \mathbb A_k^1$.
Then the coefficient $\delta_P$ of the ramification divisor $R$ at $P\in X$ is given by
\[
 \delta_P = \left\{
 \begin{array}{ll}
 2n_i & {\rm if }\ P=P_i\ {\rm for\ some }\ i \in \{1,\ldots ,k\}, \\
 2(g+1-d) & {\rm if }\  P=P_\infty, \\
 0 & {\rm otherwise.} 
 \end{array}
\right.
\]
\end{lem}
\begin{proof}
 We first show that it will suffice to prove that the coefficient of $[P_i]$ is $2n_i$ for $1\leq i \leq k$.
 Note that by the Riemann-Hurwitz formula $\deg(R) = 2g+2$.
 Then the coefficient at $P_\infty$ is $\deg(R) - \sum_{i=1}^k2n_i$, which, by the Riemann-Hurwitz formula (Corollary \ref{hur}), is equal to $2g+2-2d = 2(g+1-d)$, as stated.
 
 Let $P=P_i$ for some $i\in \{1,\ldots , k\}$.
 Then $y-b_i$ is a local parameter at $P$.
 To see this, note that the maximal ideal $\mathfrak m_{P}$ of the local ring $\cO_{X,P}$ at $P$ is generated by $x-a_i$ and $y-b_i$.
 But $x-a_i\in \mathfrak m_{P}^2$ since $\pi$ is ramified at $P$ with ramification index 2.
 By Nakayama's lemma \cite[Prop. 2.6]{atiyahmacdonald}, $y-b_i$ is therefore a local parameter at $P$.
 
 Using Hilbert's formula \cite[Prop 4, \S 1, Ch IV]{localfields} we obtain
 \begin{eqnarray*}
  \delta_P & = & \sum_{i\geq 0} \left(\ord(G_i(P))-1\right) \\
  & = & {\rm max\ }\left\{ i\in \NN | G_i(P)\neq \{1\}\right\} + 1 \\
  & = & \ord_{P}(\sigma(y-b_i) - (y-b_i)).
  \end{eqnarray*}
  
 By an argument similar to that used to show the correspondence between the solutions of $h(x)$ and the ramification points, it is clear $\sigma(y)=y+h(x)$.
 The following calculation then concludes the proof,
  \begin{eqnarray*}
  \delta_P & = & \ord_{P}(\sigma(y-b_i) - (y-b_i)) \\
  & = & \ord_{P}(y-b_i+h(x) - y + b_i) \\
  & = & 2\ord_{a_i}(h(x)) \\
  & = & 2n_i.
 \end{eqnarray*}
\end{proof}

This lemma shows that $\infty \in \mathbb P_k^1$ is a branch point if and only if the degree of $h(x)$ is less than $g+1$.
In the case when $\infty$ is branch point, by \cite[\S 7.4.3]{liu} we have that the degree of $f(x)$ must be precisely $2g+1$.
On the other hand, if the degree of $h(x)$ is $g+1$, and hence $\infty$ is not a branch point, the degree of $f(x)$ can be anything from 1 to $2g+2$.

We will now compute the divisors associated to $h(x)$, $x$ and $y$ in $K(X)$, and also to $dx$.
These are all the elements required to form a basis of $H^0(X,\Omega_X^{\otimes m})$.
\vskip1em

\begin{lem}\label{xchar2}
 The divisor of $x\in K(X)$ is 
 \[
  \di (x)= D_0 - D_\infty.
\]
\end{lem}
\begin{proof}
 Given our notations above, this is clear.
\end{proof}


\begin{lem}\label{dxchar2}
  Let $m\geq 1$.
 The divisor associated to the poly-differential $dx^{\otimes m}$ is 
 \[
  \di (dx^{\otimes m}) =m\sum_{i=1}^kD_i + m(g-1-d)D_\infty
 \]
\end{lem}
\begin{proof}
 We first note that it suffices to compute $\di (dx)$, since $\di (dx^{\otimes m}) = m\di (dx)$, as described earlier.

 Now we compute the divisor of $dx$.
We will need to consider the divisor of $dx$ both as a differential on $X$ and on $\mathbb P_k^1$. 
We will use the notation of $\di_X(dx)$ and $\di_{\mathbb P^1}(dx)$ to differentiate between the two cases.
Then the Riemann-Hurwitz formula (Theorem \ref{detailedhurwitz}) states that
\[
 \di_X( dx) = \pi^*\di_{\mathbb P^1}(dx) + R.
\]
We already computed $R$ in Lemma \ref{char2ramification}, and $\pi^*\di_{\mathbb P^1}(dx) = -2D_\infty$.
Hence we have
\[
 \di_X( dx) = \sum_{i=1}^k 2n_iP_i + (g+1-d)D_\infty - 2D_\infty = \sum_{i=1}^k D_{a_i} + (g-1-d)D_\infty.
\]
Multiplying through by $m$ we obtain the desired result.
 \end{proof}

 \vskip1em

 
 \begin{lem}\label{h(x)char2}
  The divisor associated to $\frac{1}{h(x)}$ is
  \[
  \di \left(\frac{1}{h(x)}\right) = dD_\infty - \sum_{i=1}^k D_{a_i} 
 \]
 \end{lem}
\begin{proof}
If $\pi$ is ramified at infinity then $\ord_{P_{\infty}}\left(\frac{1}{h(x)}\right) = -\ord_{P_{\infty}}(h(x)) = 2d$.
If $\pi$ is not ramified at infinity then $\ord_{P_{\infty}'}\left(\frac{1}{h(x)}\right) = \ord_{P_{\infty}''}\left(\frac{1}{h(x)}\right)=d$.
For the ramified points $P_i$, $1\leq i \leq k$, then $\ord_{P_i}\left(\frac{1}{h(x)}\right) = -\ord_{P_i}(h(x))= -2n_i$.
At any other point of $X$ the order of $\frac{1}{h(x)}$ is clearly zero.
\end{proof}

We again consider the affine curve $X'$, and we let $b_j \in \mathbb A_k^1$, for $1\leq j\leq l \leq 2g+1$, be the zeroes of $f(x)$.
If $b_j$ corresponds to a branch point then we denote the unique point in the pre-image of $b_j$ by $Q_j$.
Otherwise there are two points in the pre-image, and we denote these \[ Q_j:=(b_j,0)\ {\rm and}\ Q_j':=(b_j,h(b_j)).\]


~


\begin{lem}\label{ychar2}
 Let $m_j$ be the order of $f(x)$ at $b_j\in \mathbb A_k^1$.
 Then if $\infty$ is a branch point we have
 \[
  \di(y) = \sum_{j=1}^l m_j[Q_j] - (2g+1)[P_\infty].
 \]

 Otherwise we have
 \begin{equation*}
  \di(y) = \sum_{j=1}^l m_j[Q_j] + (g + 1-\deg(f(x)))[P_\infty'] -(g+1)[P_\infty''].
 \end{equation*}
\end{lem}
\begin{rem}
 In the proof of this lemma we will show that when infinity is not a branch point one of $\ord_{P_\infty'}(y)$ and $\ord_{P\infty'}(\simga(y))$ is $g + 1-\deg(f(x))$, and the other is $-(g+1)$.
 Since $\sigma(P_\infty') = P_\infty''$, we will assume for the rest of the report that, 
 
\end{rem}

\begin{proof}
 First note that if $P\in X'$ is not a zero of $f(x)$ (i.e. $P\neq Q_j$ for any $j$), then it is clear that $y|_P \neq 0$.
 Indeed, if it were zero then $f(x)|_P = y^2 + h(x)y|_P = 0$, a contradiction.
 
 If $Q_i$ is not ramified then $h(\alpha_i) \neq 0$, and $y|_{Q_i} = 0$, hence $y+h(x)$ is a unit at $Q_i$.
 Since $y(y+h(x)) = f(x)$, we have
 \begin{equation*}\label{zeroesofy}
  \ord_{Q_i}(y) = \ord_{Q_i}\left(\frac{f(x)}{y+h(x)}\right) = \ord_{Q_i}(f(x)) -\ord_{Q_i}(y+h(x)) = m_i.
 \end{equation*}

 This only leaves the case when $Q_i$ is ramified. 
 In this case we must have $m_i=1$.
 Otherwise we would also have $f'(x)|_{Q_i} = 0$, and since we also have $h(x)|_{Q_i} = f(x)|_{Q_i} = 0$ this would contradict Lemma \ref{smoothness}.
 So if we let $\tilde f(x) := \frac{f(x)}{x-\alpha_i}$ and $\tilde h(x) := \frac{h(x)}{x-\alpha_i}$ then $\tilde f(x)$ is a unit at $Q_i$.
 Then we have \[y^2 = f(x) - yh(x) = (x-\alpha_i)(\tilde f(x) - \tilde h(x)y).\]
 Hence
 \begin{equation*}
   \ord_{Q_i}(y^2) = \ord_{Q_i}(x-\alpha_i) + \ord_{Q_i}(\tilde f(x) - \tilde h(x)y).
  \end{equation*}
But since $\tilde f(x) - \tilde h(x) y$ does not have an affine pole, and $\tilde f(x)$ is a unit at $Q_i$, and $Q_i$ is a zero of $\tilde h(x) y$, it follows that $\ord_{Q_i}(\tilde f(x) -  \tilde h(x) y) = 0$
and hence $\ord_{Q_i}(y^2) = \ord_{Q_i}(x-\alpha_i) = 2$, and we have determined the zero divisor of $y$.

Since there is only one point at which $y$ can have a pole, namely $P_\infty$, and since the degree of $\di (y)$ is zero, then we have the pole divisor must be
$
 (2g+1)[P_\infty],
$
and this completes the proof.

{\em If $\infty$ is not a branch point then there are two points, $P_\infty$ and $P_\infty'$ that both map to $\infty$.
Now we consider \eqref{zeroesofy} for three cases of $\ord_{P_\infty}(y)$.
These cases are when $\ord_{P_\infty}(y)$ is greater than, less than or equal to $-(g+1)$.
First we note that we cannot have $\ord_{P_\infty}(y) < -(g+1)$, else $\ord_{P_\infty}(y+h(x)) = \ord_{P_\infty}(y)< -(g+1)$ (recall that if $\infty$ is not a branch point, then $\deg(h(x)) = g+1$), and so $\ord_{P_\infty}(y+h(x)) + \ord_{P_\infty}(y) < -2(g+1) \leq \ord_{P_infty}$, contradicting \eqref{zeroesofy}.
If $\ord_{P_\infty} (y) = -(g+1)$, then by \eqref{zerzeroesofy} we see that $\ord_{P_\infty}(y+h(x)) = -\deg(f) +g+1$.
Note that this implies that $\ord_{P_\infty}(y+h(x)) \geq -(g+1)$, which is consistent with $\ord_{P_\infty}(y+h(x)) > \min\{\ord_{P_\infty}(y), \ord_{P_\infty}(h(x))\}$.
Lastly, if $\ord_{P_\infty}(y) < -(g+1)$, then $\ord_{P_\infty}(y+h(x)) = -(g+1)$, and hence $\ord_{P_\infty} (y) = -\deg(f) +g+1$ by \eqref{zeroesofy}.

Since $\sigma(y) = y+h(x)$, and $\sigma (P_\infty) = P_\infty'$, it follows that $\ord_{P_\infty}(y) = \ord_{P_\infty'}(y+h(x))$.
Hence we may assume that $\ord_{P_\infty}(y) = -\ord_{P_\infty}(f(x)) +g + 1$, and that $\ord_{P_\infty'}(y) = -(g+1)$.}

\end{proof}



We now prove the following proposition, determining a basis of the space of global holomorphic poly-differentials as a vector space over $k$.
\vskip1em

\begin{prop}
We assume that $g\geq 2$ and let $\omega:= \frac{dx^{\otimes m}}{h(x)^m}$. 
Then if $g\geq 2$, a basis of $H^0(X,\Omega_X^{\otimes m})$ is given by


$\begin{cases}
 \omega, x\omega, \ldots , x^{g-1}\omega &  \mbox{if}\ m=1 \\
 \omega, x\omega, x^2\omega & \mbox{if}\ m=g=2 \\
 \omega, x\omega, \ldots, x^{m(g-1)}\omega;\  y\omega, xy\omega, \ldots x^{(m-1)(g-1)-2}y\omega & \mbox{otherwise.}
 \end{cases}
$
\end{prop}
\begin{rem}
 Note that the case where $m=1$ can be found in \cite[\S 7.4.3]{liu}.
\end{rem}

\begin{proof}
 We first assume that above elements are holomorphic poly-differentials, and show that they then form a basis.
To show that the elements are linearly independent over $k$ we need only show that the coefficients are, since $\omega$ is fixed.
The only case where this is not clear is when the coefficients contain both $x$ and $y$ terms.
But since the $y$ terms are all linear, and the extension is of degree two, it must follow that coefficients are linearly independent.
 
 
 In the case that $m=1$ then we have that $\dim_k H^0(X,\Omega_X) =g$ by Lemma \ref{dim3}, and there are $g$ elements described in the statement of the proposition in this case, so they must form a basis.
 If $m \geq 2$ then $\dim_k H^0(X,\Omega_X^{\otimes m}) = (2m-1)(g-1)$.
 If $m=g=2$ then $(2m-1)(g-1) = 3$, and there are three elements listed in the proposition.
 On the other hand if $m\geq 2$ and $g > 2$ the proposition lists
 \[
  m(g-1)+1 + (g-1)(m-1)-2+1 = 2mg -2m -g +1 = (2m-1)(g-1)
 \]
 elements, and again they must form a basis.
 
 We now show that the listed poly-differentials are holomorphic, i.e. that their divisors are non-negative.
 Firstly we have
 \begin{eqnarray}\label{nonydifferentials}
  \di(x^i\omega) & = & \di \left( \frac{x^i dx^{\otimes m}}{h(x)^m} \right)\\ & = & i(D_0 - D_\infty) +2m\sum_{i=1}^k n_i[P_i] + m(g-1-d)D_\infty\\
  & & -2m\sum_{i=1}^k n_i [P_i] + dmD_\infty \\
  & = & iD_0 + (m(g-1) -i)D_\infty
 \end{eqnarray}
  by Lemmas \ref{xchar2}, \ref{dxchar2} and \ref{h(x)char2}, and this is clearly non-negative for $0\leq i \leq m(g-1)$.
  
  Similarly the divisor 
  \begin{eqnarray}\ref{unramifiedydifferential}
   \di(x^iy\omega) & = & \di \left( \frac{x^i ydx^{\otimes m}}{h(x)^m} \right)\\ & = & iD_0 + (m(g-1) -i)D_\infty + \sum_{i=1}^l m_i[Q_i] - (2g+1)[P_\infty] \\
    & = & iD_0 +  \sum_{i=1}^l m_i[Q_i] + (2m(g-1) -(2g+1) -2i)[P_\infty] \\
   & = & iD_0 +  \sum_{i=1}^l m_i[Q_i] + (2((m-1)(g-1) -1 -i)-1)[P_\infty]
  \end{eqnarray}
 is again clearly non-negative for $0 \leq i \leq (g-1)(m-1)-2$.

 
 {\em We now consider the case where $\infty$ is not a branch point.
 Then 
 \begin{eqnarray*}
  \di(y\omega x^i) & = & iD_0 -iD_\infty + (m(g-1)-i)D_\infty \\ & & 
  +\sum_{i=1}^l m_i[Q_i] +(g+1-\deg(f(x)))[P_\infty] -(g+1)[P_\infty'] \\
 & = & iD_0 + \sum_{i=1}^l m_i[Q_i] + ((m+1)(g-1) + 2 - \deg(f(x)) -i)[P_\infty]+((m-1)(g-1)-2-i)[P_\infty'].
 \end{eqnarray*}
 
For this to be holomorphic we require that $i\leq (m+1)(g-1)-2$ and that $i \leq (m-1)(g+1) - \deg(f(x)) + 2$.
Recalling that $1 \leq \deg(f(x)) \leq 2g+2$, then we see that 
\begin{eqnarray*}
 (m+1)(g-1)+2 - \deg(f(x)) & \geq & (m+1)(g-1) + 2 - 2g-2 \\
 & = & mg + g - m - 1 - 2g \\
 & = & mg - g -m -1 \\
 &= & (m-1)(g-1) - 2.
\end{eqnarray*}
Hence the differentials in \eqref{unramifiedydifferential} are holomorphic if $0 \leq i \leq (m-1)(g-1)-2$.
From \eqref{nonydifferentials} we have $m(g-1)-1$ basis elements of the form $x^i\omega$.
Now $m(g-1)+1 + (m-1)(g-1) -1 = 2mg -g-2m+1 = (2m-1)(g-1)$, and hence the differentials form a basis.

  \end{proof}

  Since $\sigma$ acts trivially on $x$, it is clear that in both the $m=1$ case and the case where $m=g=2$ that the group action is trivial.
  On other hand, since $\sigma(y) = y+h(x)$, we can see that in the other cases the action is not trivial, as there are $y$ coefficients of the basis elements.
  Hence this proves the main theorem in the case of hyperelliptic curves in characteristic two.


  Suppose that $\infty$ is not a branch point, so there are two points above it, say $P_\infty$ and $P_\infty'$.
  When working out the divisor of $y$, we 
  

\bibliography{/home/jtait/files/Documents/Maths/Bibliography/biblio.bib}
%\bibliography{/home/joe/files/Documents/Maths/Bibliography/biblio.bib}
\bibliographystyle{plain}


\end{document}