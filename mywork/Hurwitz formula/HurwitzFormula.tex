% !TEX TS-program = pdflatex
% !TEX encoding = UTF-8 Unicode

% This is a simple template for a LaTeX document using the "article" class.
% See "book", "report", "letter" for other types of document.

\documentclass[11pt]{article} % use larger type; default would be 10pt

\usepackage[utf8]{inputenc} % set input encoding (not needed with XeLaTeX)

%%% Examples of Article customizations
% These packages are optional, depending whether you want the features they provide.
% See the LaTeX Companion or other references for full information.

%%% PAGE DIMENSIONS
\usepackage{geometry} % to change the page dimensions
\geometry{a4paper} % or letterpaper (US) or a5paper or....
% \geometry{margins=2in} % for example, change the margins to 2 inches all round
% \geometry{landscape} % set up the page for landscape
%   read geometry.pdf for detailed page layout information

\usepackage{graphicx} % support the \includegraphics command and options

\usepackage[parfill]{parskip} % Activate to begin paragraphs with an empty line rather than an indent

%%% PACKAGES
\usepackage{booktabs} % for much better looking tables
\usepackage{array} % for better arrays (eg matrices) in maths
\usepackage{paralist} % very flexible & customisable lists (eg. enumerate/itemize, etc.)
\usepackage{verbatim} % adds environment for commenting out blocks of text & for better verbatim
\usepackage{subfig} % make it possible to include more than one captioned figure/table in a single float
% These packages are all incorporated in the memoir class to one degree or another...

%%% HEADERS & FOOTERS
\usepackage{fancyhdr} % This should be set AFTER setting up the page geometry
\pagestyle{fancy} % options: empty , plain , fancy
\renewcommand{\headrulewidth}{0pt} % customise the layout...
\lhead{}\chead{}\rhead{}
\lfoot{}\cfoot{\thepage}\rfoot{}

%%% SECTION TITLE APPEARANCE
\usepackage{sectsty}
\allsectionsfont{\sffamily\mdseries\upshape} % (See the fntguide.pdf for font help)
\usepackage{amsmath}
\usepackage{amsthm}
\usepackage{amsfonts}
\usepackage{mathrsfs}
\usepackage{MnSymbol}
\usepackage{amsopn}
% (This matches ConTeXt defaults)

%%% ToC (table of contents) APPEARANCE
\usepackage[nottoc,notlof,notlot]{tocbibind} % Put the bibliography in the ToC
\usepackage[titles,subfigure]{tocloft} % Alter the style of the Table of Contents
\renewcommand{\cftsecfont}{\rmfamily\mdseries\upshape}
\renewcommand{\cftsecpagefont}{\rmfamily\mdseries\upshape} % No bold!

%Theorems and stuff
\newtheorem{defn}{Definition}
\newtheorem{thm}{Theorem}
\newtheorem{cor}{Corollary}
\newtheorem{lem}{Lemma}


\DeclareMathOperator{\di}{div}
\DeclareMathOperator{\ord}{ord}

%%% END Article customizations

%%% The "real" document content comes below...

\title{Hurwitz Formula}
\author{J Tait}
%\date{} % Activate to display a given date or no date (if empty),
         % otherwise the current date is printed 

\begin{document}
\maketitle


Let $X$ \ ~\ ~\ ~\ ~\ ~\ and $Y$ be projective non-singular curves, with a surjective morhpism $f:X\rightarrow Y$. We denote the induced map $K(Y)\rightarrow K( X)$ by $\tilde{f}$. Let $n=[K(X):K(Y)]$ be the degree of $f$.


\begin{defn}
	Let $t\in \mathscr{O}_{Q}(Y)$ be a uniformising parameter for some $Q\in Y$. Then the ramification index of $f$ 	at $P\in f^{-1}(Q)$ is $e(P)\coloneq  \textrm{ord}_{P}(\tilde{f} (t))$.
\end{defn}

Note that this is independent of the choice of uniformising parameter; if $s$ is another uniformising parameter then $s=ut$ for some unit $u$. Then 
	\[
		\textrm{ord}_{P}(\tilde{f}(s))=\textrm{ord}_{P}(\tilde{f}(ut))=\textrm{ord}_{P}(\tilde{f}					(u))+\textrm{ord}_{P}(\tilde{f}(t))=0+\textrm{ord}_{P}(\tilde{f}(t))
	\]
as the order of any unit is zero.

\begin{thm}
	For each $Q\in Y$ we have $\sum_{P\mapsto Q}e(P)[P]=n$.
\end{thm}

\begin{proof}

	Choose a uniformising parameter $t\in \mathscr{O}_{Q}(Y)$ and let $m=[K(Y):k(t)]$.

	To start with we show that $\sum_{P\mapsto Q} e(P)\leq n$. Let $Z=\sum_{P\mapsto Q}e(P)[P]$, $q=\deg (Z)$ 		and $S=f^{-1}(Q)$. Now by (Lemma 1, p.193) we can choose $v_{1},\ldots ,v_{q}\in L^{S}(0)$ such that the 			residues $\bar{v}_{1},\ldots ,\bar{v}_{q}\in L^{S}(0)/L^{S}(-Z)$ form a basis. We will show that $v_{1},			\ldots v_{q}$ are linearly independent over $K(Y)$. If not then there exist (after multiplying by a suitable power 		of $t$ if necessary) $g_{i}=\lambda_{i} +h_{i}\in K(Y)$ with $\ord_{Q}(h_{i})>0$, $\lambda_{i}\in k$ and 			at least one $\lambda_{i}\nequal 0$, such that $\sum_{i}g_{i}v_{i}=0$. But then 							$\sum_{i}\lambda_{i}v_{i}=-\sum_{i}h_{i}v_{i}\in L^{S}(-Z)$. Hence $\sum_{i}\lambda_{i}\bar{v}_{i}=0$, 			contradicting the choice of $\bar{v}_{i}$'s as a basis.
	
	

	Now let

		\[
			(t)_{0}^{Y}=\sum_{\textrm{ord}_{Q}(t)>0}m_{Q}[Q]
		\]

	 the divisor of zeros of $t$ in $Y$. Then

		\[
			(t)_{0}^{X}=\sum_{\textrm{ord}_{Q}(t)>0}m_{Q} \sum_{P\mapsto Q}e(P)[P]
		\]

	is obviously the divisor of zeroes of $t$ in $X$.

	Now

		\begin{align}
			\deg(t)_{0}^{X} & = \sum_{\textrm{ord}_{Q}(t)>0}m_{Q} \sum_{P\mapsto Q}e(P) \nonumber \\
			& \leq \sum_{\textrm{ord}_{Q}(t)>0}m_{Q} \cdot n \nonumber \\
			& = \deg(t)_{0}^{Y}\cdot n \nonumber \\
			& = mn. && \mbox{[prop 4., p.194]}
		\end{align}

	On the other hand

		\begin{equation}
			\deg(t)_{0}^{X}=[K(X):k(t)]=m\cdot n
		\end{equation}

	by (prop 4., p.194) and the tower law.

	Combining (1) and (2) we have that $\sum_{\textrm{ord}_{Q}(t)>0}m_{Q} \sum_{P\mapsto Q}e(P)=m\cdot n$.		As we know $\sum_{\textrm{ord}_{Q}(t)>0}m_{Q}=\deg(t)_{0}^{Y}=m$ by prop.4 page 194 and that 			$\sum_{P\mapsto Q}e(P)\leq n$ for each $Q$ by the above, it follows that $\sum_{P\mapsto Q} e(P)=n$. 

\end{proof}

\begin{cor}
	For any $h\in K(Y)$ and any $Q\in Y$ we have

		\begin{equation*}
			\sum_{P\mapsto Q} \ord_{P} (\tilde{f}(h))=n\cdot \ord_{Q}(h).
		\end{equation*}

	In particular we have $\deg( \di(\tilde{f}(h)))=n\cdot \deg( \di (h))$.
\end{cor}

\begin{proof}
	Assume that $Q$ is not a pole of $h$. Then if $h(Q)\neq 0$ then  then $\ord_{Q}(h)=0$ and $n\cdot \ord_{Q}			(h)=0$. Also, for any $P\in f^{-1}(Q)$ then $\tilde{f}(h)(P)=h(f(P))\neq 0$, so $\ord_{P}(\tilde{f}(h))=0$ too. 			So

		\begin{equation*}
			\sum_{P\mapsto Q} \ord_{P}(\tilde{f}(h))=0=n\cdot \ord_{Q}(h).
		\end{equation*}

	If $h(Q)=0$ then $\ord_{Q}(h)=r>0$ and $h=ut^{r}$ for some unit $u$ and uniformising parameter $t$ both in 		$\mathscr{O}_{Q}(Y)$. Now for each $P\in f^{-1}(Q)$ we have that 

		\begin{eqnarray*}
			\ord_{P}(\tilde{f}(h)) & = & \ord_{P}(\tilde{f}(u)\tilde{f}(t^{r})) \nonumber \\
			&  = & \ord_{P}(\tilde{f}(u))+\ord_{P}(\tilde{f}(t^{r})) \nonumber\\
			& = &  0 + r\cdot \ord_{P}(\tilde{f}(t)). \nonumber \\
		\end{eqnarray*}

	So
		\begin{align*}
			\sum_{P\mapsto Q}\ord_{P}(\tilde{f}(h)) & =  r\cdot \sum_{P\mapsto Q} \ord_{P}(\tilde{f}(t)) \\
			& = r\cdot \sum_{P\mapsto Q} e(P) \\
			& = r\cdot n && \mbox{[Theorem 1]} \\
			& = \ord_{Q}(h)\cdot n \\
		\end{align*}

	Finally, if $Q$ is a pole of $h$ then we can take inverses to make it a zero (of $h^{-1}$) and the result follows 			from the above.
\end{proof}

\begin{lem}
	If $t$ is a uniformising parameter in $\mathscr{O}_{Q}(Y)$ then $\ord_{P}(d(\tilde{f}(t)))=e(P)-1$ for each $P\in 			f^{-1}(Q)$ whenever $char(k)\ndivides e(P)$.
\end{lem}

\begin{proof}
	By definition of $e(P)$ there is a unit $u$ and a uniformising parameter $s$ in $\mathscr{O}_{P}(Y)$ such that 			$\tilde{f}(t)=us^{e(P)}$. So $d\tilde{f}(t)=d(us^{e(P)})$.

	As $d$ is a derivation we have that $d(us^{e(P)})=ud(s^{e(P)})+s^{e(P)}d(u)$, and by a simple inductive 			argument we can also see that $d(s^{e(P)})=e(P)s^{e(P)-1}d(s)$. Combining these gives 					$d(us^{e(P)})=ue(P)s^{e(P)-1}d(s)+s^{e(P)}d(u)$. Now we have

		\begin{equation}
			\ord_{P}(e(P)s^{e(P)-1}d(s)):= \ord_{P}(e(P)s^{e(P)-1})=e(P)-1,
		\end{equation}

	the last equality holding as $char(k)\ndivides e(P)$, and the definition being on page 207.

	Also

		\begin{equation}
			\ord_{P}(s^{e(P)}d(u))\geq e(P)
		\end{equation}

	

	Hence we see that
		\begin{align*}
			\ord_{P}(d\tilde({f}(t))) & = \ord_{P}(ue(P)s^{e(P)-1}ds+s^{e(P)}d(u)) \\
			& =  min\{ \ord_{P}(ue(P)s^{e(P)-1}ds),s^{e(P)}d(u) \} && \mbox{[ex.2-29, p.48]}\\
			& = e(P)-1 \\
		\end{align*}

\end{proof}

\begin{thm}[Hurwitz Formula]
	Suppose that $char(k)\ndivides e(P)$ for all $P\in X$ such that $e(P)\neq 0$ and let $g_{X},\ g_{Y}$ denote the 		genus of $X$ and $Y$ respectively. Then we have
		\begin{equation*}
			2g_{X}-2=n(2g_{Y}-2)+\sum_{P\in X}(e(P)-1).
		\end{equation*}
\end{thm}

\begin{proof}
	It suffices to show that the degree of a canonical divisor on $X$ is precisely 
		\begin{equation*}
			n\cdot (2g_{Y}-2)+\sum_{P\in X}(e(P)-1)
		\end{equation*}

	as, by corollary on page 209, for any canonical divisor $W$ it is true that $\deg(W)=2g-2$.

	Suppose that $\omega_{Y}\in \Omega_{k}(K(Y))$ is a non-zero differential on $Y$. Then $\tilde{f}				(\omega_{Y})$ is a non-zero differential on $X$.

	It suffices to show that for each $Q\in Y$ the following holds:
		\begin{equation*}
			\sum_{P\mapsto Q}\ord_{P}((\tilde{f}(\omega_{Y})))=n\cdot \ord_{Q}								(\omega_{Y})+\sum_{P\mapsto Q}(e(P)-1)
		\end{equation*}

	Fix $Q\in Y$. Then for some uniformising parameter $t\in \mathscr{O}_{Q}(Y)$ and some $h\in K(Y)$ we have 			that $\omega_{Y}=hdt$ by proposition 6 page 205.

	So now we have
		\begin{eqnarray}
			\sum_{P\mapsto Q}\ord_{P}(\tilde{f}(\omega_{Y})) & = & \sum_{P\mapsto Q}\ord_{P}(\tilde{f}					(hdt)) \\
			& = & \sum_{P\mapsto Q} \ord_{P}(\tilde{f}(h)) + \sum_{P\mapsto Q} \ord_{P}(d(\tilde{f}(t)) \\
			& = & n\cdot \ord_{Q}(h) + \sum_{P\mapsto Q} (e(P)-1) \\
			& = & n\cdot \ord_{Q}(\omega_{Y}) + \sum_{P\mapsto Q} (e(P)-1) 
		\end{eqnarray}

	with line(7) following from the previous corollary and lemma.

	Hence

		\begin{eqnarray*}
			2g_{X}-2 & = & \deg(\tilde{f}(\omega_{Y})) \\
			& = & \sum_{Q\in Y} \sum_{P\mapsto Q} \ord_{P}(\tilde{f}(\omega_{Y})) \\
			& = & \sum_{Q\in Y} n\cdot \ord_{Q}(\omega_{Y}) + \sum_{Q}\sum_{P\mapsto Q} (e(P)-1) \\
			& = & n\cdot \deg(\di(\omega_{Y})) +\sum_{P\in X} (e(P)-1) \\
			& = & n\cdot (2g_{Y}-2) + \sum_{P\in X}(e(P)-1).
		\end{eqnarray*}
\end{proof}

\begin{cor}
	If $Y=\mathbb{P}^{1}$ and $n>1$ then there exist ramification points (i.e. points  $P\in X$ such that 				$e(P)>1$).
\end{cor}

\begin{proof}
	As $Y=\mathbb{P}^{1}$ we have that $g_{Y}=0$. It follows that 
		\begin{equation*}
			2g_{X}-2=-2n+\sum_{P\in X}(e(P)-1)
		\end{equation*}
	
	 and hence
		\begin{equation*}
			2g_{X}=-2(n-1)+\sum_{P\in X} (e(P)-1).
		\end{equation*}

	So, as we know that $g_{X}\geq 0$ and that $n>1$, it follows that there exists a $P\in X$ such that 				$e(P)-1\geq 1$, and $P$ is a ramification point.
\end{proof}

\end{document}
