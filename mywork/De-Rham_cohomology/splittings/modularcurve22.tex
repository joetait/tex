% !TEX TS-program = pdflatex
% !TEX encoding = UTF-8 Unicode

% This is a simple template for a LaTeX document using the "article" class.
% See "book", "report", "letter" for other types of document.

\documentclass[draft, 11pt]{article} % use larger type; default would be 10pt

\usepackage[utf8]{inputenc} % set input encoding (not needed with XeLaTeX)

%%% Examples of Article customizations
% These packages are optional, depending whether you want the features they provide.
% See the LaTeX Companion or other references for full information.

%%% PAGE DIMENSIONS
\usepackage{geometry} % to change the page dimensions
\geometry{a4paper} % or letterpaper (US) or a5paper or....
% \geometry{margins=2in} % for example, change the margins to 2 inches all round
% \geometry{landscape} % set up the page for landscape
%   read geometry.pdf for detailed page layout information

\usepackage{graphicx} % support the \includegraphics command and options

%\usepackage[parfill]{parskip} % Activate to begin paragraphs with an empty line rather than an indent

%%% PACKAGES
\usepackage{mathtools}
\usepackage{booktabs} % for much better looking tables
\usepackage{array} % for better arrays (eg matrices) in maths
\usepackage{paralist} % very flexible & customisable lists (eg. enumerate/itemize, etc.)
\usepackage{verbatim} % adds environment for commenting out blocks of text & for better verbatim
\usepackage{subfig} % make it possible to include more than one captioned figure/table in a single float
\usepackage[obeyDraft]{todonotes}
% These packages are all incorporated in the memoir class to one degree or another...

%%% HEADERS & FOOTERS
\usepackage{fancyhdr} % This should be set AFTER setting up the page geometry
\pagestyle{fancy} % options: empty , plain , fancy
\renewcommand{\headrulewidth}{0pt} % customise the layout...
\lhead{}\chead{}\rhead{}
\lfoot{}\cfoot{\thepage}\rfoot{}

%%% SECTION TITLE APPEARANCE
\usepackage{sectsty}
\allsectionsfont{\sffamily\mdseries\upshape} % (See the fntguide.pdf for font help)
\usepackage{amsmath}
\usepackage{amsthm}
\usepackage{amsfonts}
\usepackage{mathrsfs}
\usepackage{amsopn}
\usepackage{amssymb}
%\usepackage{natbib}
% (This matches ConTeXt defaults)

%%% ToC (table of contents) APPEARANCE
\usepackage[nottoc,notlof,notlot]{tocbibind} % Put the bibliography in the ToC
\usepackage[titles,subfigure]{tocloft} % Alter the style of the Table of Contents
\renewcommand{\cftsecfont}{\rmfamily\mdseries\upshape}
\renewcommand{\cftsecpagefont}{\rmfamily\mdseries\upshape} % No bold!

%Theorems and stuff
\theoremstyle{plain}
\newtheorem{defn}{Definition}[section]
\newtheorem{thm}[defn]{Theorem}
\newtheorem{cor}[defn]{Corollary}
\newtheorem{lem}[defn]{Lemma}
\newtheorem{prop}[defn]{Proposition}
\newtheorem{ex}[defn]{Example}
\newtheorem*{unnumthm}{Theorem}
\newtheorem{defnlem}[defn]{Definition/Lemma}
\newtheorem{defnthm}[defn]{Theorem/Definition}
\theoremstyle{remark}
\newtheorem*{rem}{Remark}


\newcommand{\cO}{{\cal O}}
\newcommand{\ra}{\rightarrow}
\newcommand{\NN}{{\mathbb N}}
\newcommand{\PP}{{\mathbb P}}
\newcommand{\ZZ}{{\mathbb Z}}
\newcommand{\cL}{{\mathcal L}}
\newcommand{\cA}{{\mathcal A}}
\newcommand{\cD}{{\mathcal D}}
\newcommand{\cU}{{\mathcal U}}
\newcommand{\cech}{\v{C}ech }
\newcommand{\hzero}{{H^0(X,\Omega_X)}}
\newcommand{\hone}{H^1(X,\mathcal{O}_X)}
\newcommand{\cechhone}{\check{H}^1(\mathcal U,\mathcal O_X)}
\newcommand{\derhamhone}{H_{\text {dR}}^1(X/k)}
\newcommand{\cechderhamhone}{\check{H}_{\text {dR}}^1(X/k)}


\DeclareMathOperator{\aut}{Aut}
\DeclareMathOperator{\ord}{ord}
\DeclareMathOperator{\di}{div}
\DeclareMathOperator{\cha}{char}
\DeclareMathOperator{\gal}{Gal}
\DeclareMathOperator{\Tr}{Tr}

%%% END Article customizations

%%% The "real" document content comes below...

\title{Group actions on de Rham cohomology of $X_0(22)$}
\author{J. Tait}
%\date{} % Activate to display a given date or no date (if empty),
         % otherwise the current date is printed 

\begin{document}
\maketitle
\listoftodos
\bigskip

\section{Note on de-Rham cohomology}
\todo[inline]{make edits to cech cohomology section. In particular, switch order of entries so infinity is at the end}
We let $X$ be the hyperelliptic curve over an algebraically closed field $k$ (characteristic not 2) defined by $y^2 = f(x)$, where $f(x) \in k[x]$ has no repeated roots.
We fix a degree two map $\pi \colon X \ra \mathbb P^1_k$, which is unique up to automorphism of $\mathbb P_k^1$.



In previous work we found a basis for the de Rham cohomology of any hyperelliptic curve using \cech cohomology.
In particular, we let $\cU = \{ U_0 , U_\infty\}$ (recall that $U_a = X\backslash \pi^{-1}(a)$) and a basis of $\cechderhamhone(\cU)$ is given in \cite{derhamactions}.
However, when we are computing actions on particular curves we will often find that the cover $\cU$ is not preserved under the action of particular elements of the automorphism group.
To accommodate for this we will need to add at least one more open set to our cover.


The elements of $G= \aut(X)$ we will be considering will always fix $U_\infty$, so if $\sigma$ is such an element we will only need to add $U_\sigma := \sigma^{-1}(U_0)$.
We call the cover with this additional element $\cU'' = \{U_0, U_\sigma, U_\infty \}$, and we let $\cU' = \{U_\sigma, U_\infty\}$.
We then have the following commutative diagram
\[
\begin{array}{ccc}
\derhamhone \cong \cechderhamhone(\cU)  & \xleftarrow{\rho} & \cechderhamhone(\cU'')  \\
\sigma\uparrow & ~ & \rho'\downarrow  \\
\derhamhone \cong \cechderhamhone(\cU)  & \xrightarrow{\sigma} & \cechderhamhone(\cU')
\end{array}
\]
where $\rho$ and $\rho'$ are the canonical projections.
To define these explicitly we recall that $\cechderhamhone(\cU'')$ can be defined is the $k$-vector space 
\begin{multline}\label{sixtupleconditions}
\left\{ (\omega_0, \omega_\sigma, \omega_\infty , f_{0\sigma}, f_{0 \infty},f_{\sigma \infty}) | \omega_i \in \Omega_X(U_i), f_{ij} \in \cO_X(U_i \cap U_j), \right. \\ \left. f_{0\sigma} - f_{0\infty} + f_{\sigma \infty} = 0, df_{ij} = \omega_i - \omega_j \right\}
\end{multline}
quotiented by the subspace 
\[
\left\{ df_0, df_\sigma, df_\infty, f_0- f_\sigma, f_0 - f_\infty, f_\sigma - f_\infty | f_i \in \cO_X(U_i)\right\}.
\]
Then $\rho$ projects the first, second and fourth coordinates whilst $\rho'$ projects the second, third and sixth coordinates.

Suppose that $v = (\omega_\sigma , \omega_\infty, f_{\sigma \infty})$ is an element of $\cechderhamhone(\cU')$.
Then we note let
\begin{equation*}
\sigma(v) := (\sigma(\omega_\sigma), \sigma(\omega_\infty), \sigma(f_{\sigma \infty})) \in \Omega_X(U_0) \times \Omega_X(U_\infty) \times \cO_X(U_\infty \cap U_0).
\end{equation*}
Since $\sigma(\omega_\sigma)-\sigma(\omega_\infty) = \sigma(\omega_\sigma - \omega_\infty) = \sigma(df_{\sigma \infty})$ we see that $\sigma(v)$ is an element of $\cechderhamhone(\cU)$.



\section{Splitting of short exact sequence for genus two curves}

We suppose that $k$ is an algebraically closed field of characteristic $p \geq 3$ and that $X$ is a genus two curve.
Any genus two curve is hyperelliptic, and hence $X$ is isomorphic to a curve defined by an equation of the form
\begin{equation}\label{definingequation}
y^2 = f(x) = a_6x^6 + a_5x^5 + a_4x^4 + a_3x^3 + a_2x^2 + a_1x + a_0;
\end{equation}
hence we assume that $X$ is the curve defined by \eqref{definingequation}.
Moreover, we suppose that $X$ has an automorphism of the form $\sigma \colon X \ra X$, such that $\sigma \colon x \to x+a$ for some $a\in k$.
We now show that this fully determines the automorphism $\sigma$.\todo{define $\sigma$ better - comment that it is a map $K(X) \ra K(X)$}

\begin{lem}
The action of $\sigma$ on $y$, possibly after composing with the hyperelliptic involution, is trivial.
\end{lem}
\begin{proof}
Since $y^2 \in k(x)$ then there must exist $g_1(x), g_2(x) \in k(x)$ such that 
\begin{equation*}
\sigma(y) = g_1(x)y + g_2(x).
\end{equation*}
Hence
\begin{equation}\label{easylemma}
f(x+a) = \sigma(y^2) = (\sigma(y))^2 = g_1(x)^2f(x)+2g_1(x)g_2(x)y + g_2(x)^2.
\end{equation}
Firstly, note that if neither $g_1(x)$ nor $g_2(x)$ are zero we get
\[
y = \frac{f(x+a) - g_1(x)f(x) - g_2(x)^2}{2g_1(x)g_2(x)},
\]
which contradicts the fact $K(X)$ is a degree two extension of $k(x)$.
Hence one of $g_1(x)$ or $g_2(x)$ must be zero.

If $g_1(x) = 0$ then $\sigma$ would not be an automorphism, since $y$ would not be in the image.
Hence $\sigma(y) = g_1(x)y$.
Also, by comparing the degrees in \eqref{easylemma} we see that $\deg(g_1(x) = 0$, and then by comparing coefficients in the same equation we see that $g_1(x)^2 = $.
Hence $\sigma(y) = \pm y$, and of course if $\sigma(y) = -y$ we can just compose $\sigma$ with the hyperelliptic involution, and we conclude that $\sigma(y) = y$.
\end{proof}

Given an automorphism $\sigma\colon (x,y) \mapsto (x+a,y)$ of $X$, we obtain some restrictions on the defining equation \eqref{definingequation}, which we list in the lemma below.

\begin{lem}
Given the situation above, it follows that the characteristic of $k$ is 3 or 5.
\end{lem}
\begin{proof}
Since $\sigma(y) = y$, it follows that $f(x+a) = f(x)$.
Expanding the left hand side we see that the coefficient of $x^5$ is $6aa_6 + a_5$.
Since the coefficient of $x^5$ in $f(x)$ is $a_5$.
We conclude that $6aa_6 = 0$, and hence either $6=0$ or $a_6=0$.
If $6=0$ then it follows that $\cha (k)=3$.

On the other hand, if $a_6=0$ then we compare the coefficients of $x^4$ in $f(x+a)$ and $f(x)$, which are $5aa_5+a_4$ and $a_4$ respectively.
We conclude that $5aa_5=0$. Since the degree of $f(x)$ is either $2g+2=6$ or $2g+1=5$ it follows that at least one of $a_6$ and $a_5$ must be non-zero.
Hence if $a_6=0$ it follows that $\cha(k) = 5$.
\end{proof}



The following proposition allows us to compute how $\sigma$ acts on $\cechderhamhone(\cU)$, as described in the first section.

\begin{prop}\label{basis22}
The residue classes of
\begin{multline*}
\nu_1 = \left(\frac{a_3x^3-a_1x-2a_0}{2yx^2}dx, \frac{\aleph_1}{2y(x-a)^3}dx, \frac{-(4a_6x^4+3a_5x^3+2a_4x^2)}{2y}dx,  \right. \\ \left. \frac{a^2y}{x(x-a)^2}, \frac{y}{x},  \frac{y(x-2a)}{(x-a)^2} \right)
\end{multline*}
and
\begin{multline*}
\nu_2 = \left( \frac{-(a_3x^3+2a_2x^2+3a_1x+4a_0)}{2yx^3},  \frac{\aleph_2}{2y(x-a)^3}dx, \frac{-(2a_6x^3+a_5x^2)}{2y}dx, \right. \\ \left.  \frac{y(a^2-2ax)}{x^2(x-a)^2},\frac{y}{x^2}, \frac{y}{(x-a)^2} \right),
\end{multline*}
where
\begin{multline*}
\aleph_1  = (4a^3a_6+a^2a_5+a_3)x^4 + (3a^3a_5+2a^2a_4-3aa_3)x^3 \\
 + (2a^3a_4 + 6a^2a_3 - a_1)x^2 + (4a^2a_2 +3aa_1-2a_0)x+(2a^2a_1+6aa_0)
\end{multline*}
and
\begin{multline*}
\aleph_2  = (-6a^2a_6-2aa_5)x^4 + (2a^3a_6-3a^2a_5-4aa_4-a_3)x^3 \\ + (a^3a_5-3aa_3-2a_2)x^2 + (-2a_2a-3a_1)x +(-aa_1-4a_0),
\end{multline*}
are well defined elements of $\cechderhamhone(\cU'')$.
Moreover, the canonical projections on to $\cechderhamhone(\cU)$ are the first two basis elements described in \cite[Thm. 2.3]{derhamactions}.
\end{prop}
\begin{proof}
In order to refer to the entries in $\nu_1$ and $\nu_2$ we define $f_{ijk}$ and $\omega_{ijk}$ by letting
\[
\nu_i = \left( \omega_{0 i}, \omega_{\sigma i}, \omega_{\infty i}, f_{0 \sigma i}, f_{0 \infty i}, f_{\sigma \infty i} \right).
\]\todo{clean this up, including the sentence before.}
First, note that it follows from \todo{cite previous work} that $d(f_{0 \infty i}) = \omega_{0 i} - \omega_{\infty i}$.

We check that $f_{0 \sigma i} - f_{0 \infty i}+ f_{\sigma \infty i} = 0$.
When $i=1$ we have
\begin{align*}
f_{0 \sigma i} - f_{0 \infty i}+ f_{\sigma \infty i} & = \frac{y}{x^2} - \frac{y(a^2-2ax)}{x^2(x-a)^2} - \frac{y}{(x-a)^2} \\
& = \frac{y(x-a)^2 - y(a^2-2ax) - x^2y}{x^2(x-a)^2} \\
& = 0
\end{align*}
after expanding the second line out.

When $i=2$ we have
\begin{align}
f_{0 \sigma i} - f_{0 \infty i}+ f_{\sigma \infty i} & = \frac{y}{x} - \frac{y}{x(x-a)^2} - \frac{y(x-2a)}{(x-a)^2} \\
& = \frac{y(x-a)^2 - a^2y -xy(x-2a)}{x(x-a)^2} \\
& = \frac{y(x-a)^2 - y(x-a)^2}{x(x-a)^2} \\
& = 0.
\end{align}

We now check that differentials and functions are regular on the appropriate open sets by computing the relevant divisors.
Firstly, the divisor of $f_{0 \sigma 1}$ is
\begin{align*}
\di(f_{0 \sigma 1}) & = \di\left( \frac{a^2y}{x(x-a)^2} \right) \\ 
& = \di(y) - \di(x) - 2\di(x-a) \\
& = R - 3D_\infty - D_0 + D_\infty -2D_a +2D_\infty \\
& = R - D_0 - 2D_a,
\end{align*}
and the divisor of $f_{0 \sigma 2}$ is
\begin{align*}
\di(f_{0 \sigma 2} ) & = \di \left( \frac{y(a^2 - 2ax)}{x^2(x-a)^2} \right) \\
& = \di(y) + \di(a-2x) - 2\di(x) - 2\di(x-a) \\
& = R - 3D_\infty +D_{\frac{a}{2}} - D_\infty -2D_0 + 2D_\infty -2D_a + 2D_\infty \\
& = R + D_{\frac{a}{2}} - 2D_0 - 2D_a,
\end{align*}
both of which are regular on $U_0 \cap U_\sigma$.
Secondly, the divisor of $f_{0 \infty i}$ is
\begin{align*}
\di(f_{0 \infty i}) & = \di\left( \frac{y}{x^i} \right) \\
& = \di(y) - \di(x^{-i}) \\
& = R - 3D_\infty -iD_0 + iD_\infty \\
& = R - (3-i)D_\infty -iD_0.
\end{align*}
When $i = 1,2$ this is clearly positive everywhere on $U_0 \cap U_\infty$.
Lastly we compute the divisor of  $f_{\sigma \infty i}$, which is
\begin{align*}
\di(f_{\sigma \infty 1} ) & = \di \left( \frac{-y(x-2a)}{(x-a)^2} \right) \\
& = \di(y) + \di(x-2a) - 2\di(x-a) \\
& = R - 3D_\infty + D_{2a} - D_\infty - 2D_a + 2D_\infty \\
& = R + D_{2a} -2D_\infty - 2D_a
\end{align*}
when $i=1$ and
\begin{align*}
\di( f_{\sigma \infty 2} ) & = \di \left( \frac{-y}{(x-a)^2} \right) \\
& = \di (y)  - 2\di(x-a) \\
& = R - 3D_\infty  - 2D_a + 2D_\infty \\
& = R - 2D_a - D_\infty 
\end{align*}
when $i=2$, and in both cases the divisor is clearly positive on $U_\sigma \cap U_\infty$.

Now we show that for $i = 1, 2$ the differential $\omega_{\sigma i}$ is regular on $U_\sigma$.
In both cases it is clear that there can only be poles at $\pi^{-1}(a)$ and $\pi^{-1}(\infty)$.
Now $\pi^{-1}(a) \cap U_\sigma$ is empty, hence we need only check that there are on poles in $\pi^{-1}(\infty)$, or equivalently that the coefficient of $D_\infty$ in $\di( \omega_{\sigma i})$ is non-negative.
By \todo{reference divisors from other paper} when $i=1$ this coefficient is $-4 - 4 + 3 + 3 = 0$ and when $i=2$ the coefficient is $-4 -2 +3 + 3=0$.

It only remains to show that $df_{0 \omega i} = \omega_{0 i} - \omega_{\sigma i}$ for $i \in \{ 1, 2\}$.
When $i=1$ the left hand side is 
\begin{align*}
df_{0 \omega 1} & = d \left( \frac{a^2y}{x(x-a)^2} \right)  \\
& = a^2\left( \frac{1}{x(x-a)^2}dy + y d \frac{1}{x(x-a)^2} \right) \\
 & = a^2 \left( \frac{f'}{2yx(x-a)^2} + \frac{y(a-3x)}{x^2(x-a)^3} \right) dx \\
& = \frac{a^2}{2yx^2(x-a)^3} \left( xf'(x-a) + 2f(a-3x) \right) dx,
\end{align*}
and the bracketed expression in the final line expands to 
\begin{equation}\label{i=1expansion}
-(4aa_6+a_5)x^6-(3aa_5+2a_4)x^5-(2aa_4+3a_3)x^4-(aa_3+4a_2)x^3-5a_1x^2+(aa_1-6a_0)x+2a_0a
\end{equation}
On the other hand, the right hand side is
\begin{equation*}
\omega_{0 1} - \omega_{\sigma 1}  = \frac{1}{2yx^2(x-a)^3} \left( (a_3x^3 -a_1x-2a_0)(x-a)^3 - x^2\aleph_1\right )dx,
\end{equation*}
and the bracketed expression here also expands to \eqref{i=1expansion}.
Hence $df_{0 \sigma 1} = \omega_{0 1} - \omega_{\sigma 1}$.

When $i=2$ then 
\begin{align*}
df_{0 \sigma 2} & = d \left( \frac{y(a^2 -2ax)}{x^2(x-a)^2} \right) \\
& = \frac{a^2-2ax}{x^2(x-a)^2} dy + y d \frac{a^2-2ax}{x^2(x-a)^2} \\
& = \left( \frac{f'(a^2-2ax)}{2yx^2(x-a)^2} + \frac{2af(a^2-3ax+3x^2}{x^3(x-a)^3} \right) dx \\
& = \frac{a}{2yx^3(x-a)^3} \left( f'(a-2x)x(x-a) + 4f(a^2-3ax+3x^2 \right) dx,
\end{align*}
and expanding the brackets we gives
\begin{multline}\label{i=2expansion}
(6aa_6+2a_5)x^7+(4a_4+3a_5a-2a_6a)x^6+(6a_3-aa_5)x^5+(8a_2-3a_3a)x^4 \\ + (10a_1 -6a_2a+a_3a^2)x^3 
(12a_0-9a_1a+2a_2a^2)x^2 + (3a_1a^2-12aa_0)x + 4a_0a^2.
\end{multline}
On the other hand
\begin{equation*}
\omega_{0 2} - \omega_{\sigma 2} = \frac{1}{2yx^3)(x-a)^3}\left( -(a_3x^3+2a_2x^2+3a_1x+4a_0)(x-a)^3 + x^3\aleph_2 \right) dx,
\end{equation*}
and the bracketed expression here also expands to \eqref{i=2expansion}.
\end{proof}

We now use this proposition to compute the action of $\sigma $ on $\derhamhone$.
Recall that $\sigma$ fixes $y$ and maps $x$ to $x+a$. 
Note that this means that $\sigma(dx) = d(\sigma(x)) = d(x+a) = dx$.
The action of $\sigma$ on $\nu_i$ is given by
\begin{equation*}
\sigma(\nu_1) = \left( \sigma( \omega_{\sigma}), \sigma(\omega_{\infty}), \sigma( f_{\sigma \infty}\right).
\end{equation*}
We will consider this action entry by entry, due to the amount space the expressions will take.

We consider $\sigma(\nu_1)$ first.
\begin{align*}
\sigma( \omega_{\sigma}) & = \sigma \left( \frac{\aleph_1}{2y(x-a)^3} dx \right) \\
& = \frac{\sigma(\aleph_1)}{2yx^3}dx.
\end{align*}
Now we compute $\sigma(\aleph_1)$, which is
\begin{align*}
\sigma(\aleph_1) & = (4a^3a_6+a^2a_5+a_3)x^4 + (16a^4a_6+7a^3a_5+2a^2a_4+aa_3)x^3 \\
& + (24a^5a_6+15a^4a^5+8a^3a_4+3a^2a_3-a_1)x^2 \\
& + (16a^6a_6+13a^5a_5+10a^4a_4+7a^3a_3+4a^2a_2+aa_1-2a_0)x \\
& + a(4a^6a_6+4a^5a_5+4a^4a_4+4a^3a_3+4a^2a_2+4aa_1+4a_0).
\end{align*}
We use the fact $f(x+a) = f(x)$, and equating the coefficients on both side, simplify the above expression.
We list the identities we get from $f(x) = f(x+a)$ here:
\begin{enumerate}
\item \label{1} $6aa_6=0$
\item \label{2} $15a^2a_6+5aa_5 = 0$
\item \label{3} $20a^3a_6+10a^2a_5+4aa_4 = 0$
\item \label{4} $15a^4a_6 + 10a^3a_5 + 6a^2a_4+3aa_3= 0$
\item \label{5} $6a^5a_6 + 5a^4a_5+4a^3a_4+3a^2a_3+2aa_2= 0$
\item \label{6} $a^6a_6+a^5a_5+a^4a_4+a^3a_3+a^2a_2+aa_1 = 0$
\end{enumerate}
When $\cha (k) =3$ using the above we see that the coefficients of $\sigma(\aleph_1)$ simplify as follows
\begin{equation*}
 a(4a^6a_6+4a^5a_5+4a^4a_4+4a^3a_3+4a^2a_2+4aa_1+4a_0) =  4aa_0
\end{equation*}
by \ref{6};
\[
16a^6a_6+13a^5a_5+10a^4a_4+7a^3a_3+4a^2a_2+aa_1-2a_0 = 3aa_1 -2a_0
\]
by \ref{5} and \ref{6};
\[
24a^5a_6+15a^4a^5+8a^3a_4+3a^2a_3-a_1 = 2a_2a^2-a_1
\]
by \ref{4} and \ref{5};
\[
16a^4a_6+7a^3a_5+2a^2a_4+aa_3 = 2a^3a_5 +aa_3
\]
by \ref{1} and \ref{3}.
Hence
\begin{multline*}
\sigma(\aleph_1) = \\ \frac{(4a^3a_6+a^2a_5+a_3)x^4+(6a^4a_6+2a^3a_5+aa_3)x^3+(2a^2a_2-a_1)x^2+(3aa_1-2a_0)x+4aa_0}{2yx^3}dx.
\end{multline*}
We also have
\begin{align*}
\sigma(f_{\sigma \infty 1}) & = \sigma\left(\frac{y(x-2a)}{(x-a)^2} \right) \\
& = \frac{y(x-a)}{x^2}  \\
& = \frac{y}{x} - \frac{ay}{x^2}.
\end{align*}
From this we can see that
\begin{equation}\label{firstidentity}
\sigma(\gamma_3) = \gamma_3 - a\gamma_4 + c_1\gamma_1 + c_2 \gamma_2
\end{equation}
for some $c_1,c_2 \in k$.
Finally we have
\begin{align*}
\sigma (\omega_{\infty 1}) & = \sigma\left( \frac{-(4a_6x^4+3a_5x^3+2a_4x^2)}{2y} dx \right) \\
& = \frac{-4a_6x^4-(16aa_6+3a_5)x^3-(24a^2a_6+9aa_5+2a_4)x^2-(16a^3a_6+9a^2a_5+4aa_4)x-4a^4a_6-3a^3a_5-2a^2a_4}{2y}dx.
\end{align*}
Using the above we can compute the $c_1$ and $c_2$ mentioned above.
We start by noting that
\begin{align*}
\sigma(\omega_{\infty 1}) + a\omega_{\infty 2} - \omega_{\infty 1} & = \frac{-18aa_6x^3 -(24a^2a_6+10aa_5)x^2-(16a^3a_6+9a^2a_5+4aa_4)x-4a^4a_6-3a^3a_5-2a^2a_4}{2y}dx \\
& = \frac{(4a^3a_6+a^2a_5)x +2a^3a_5}{2y}dx.
\end{align*}
From this we see that $c_1= a^3a_5$ and that $c_2 = \frac{4a^3a_6+a^2a_5}{2}$.\todo{Comment that this has been checked with the the $\omega_{0 1}$ entry too?}



Now when $i=2$ we have
\begin{align*}
\sigma(\aleph_2) & = -(6a^2a_6+2aa_5)x^4 - (22a^3a_6+11a^2a_5+4aa_4+a_3)x^3 \\
& - (30a^4a_6+20a^3a_5+12a^2a_4+6aa_3+2a_2)x^2 \\
& - (18a^5a_6+15a^4a_5+12a^3a_4+9a^2a_3+6aa_2+3a_1)x \\
& -4(a^6a_6+a^5a_5+a^4a_4+a^3a_3+a^2a_2+aa_1+a_0).
\end{align*}
Again, using the relations \ref{1} through \ref{6} we can simplify these coefficients:
by \ref{6} 
\[
-4(a^6a_6+a^5a_5+a^4a_4+a^3a_3+a^2a_2+aa_1+a_0) = -4a_0;
\]
by \ref{5}
\[
- (18a^5a_6+15a^4a_5+12a^3a_4+9a^2a_3+6aa_2+3a_1) = -3a_1;
\]
by \ref{4}
\[
- (30a^4a_6+20a^3a_5+12a^2a_4+6aa_3+2a_2) = -2a_2;
\]
by \ref{5} 
\[
- (22a^3a_6+11a^2a_5+4aa_4+a_3) = -2a^3a_5-a^2a_5-a_3;
\]
and by \ref{6}
Next we compute the action of sigma on $f_{\sigma \infty 2}$, as follows:
\begin{align*}
\sigma( f_{\sigma \infty 2}) & = \sigma \left( \frac{y}{(x-a)^2} \right) \\
& = \frac{y}{x^2}.
\end{align*}
Hence we see that
\[
\sigma( \gamma_4) = \gamma_4 + d_1\gamma_1 + d_2 \gamma_2
\]
for some $d_1, d_2 \in k$.
Lastly, we compute the action on $\omega_{\infty 2}$, which is
\begin{align*}
\sigma ( \omega_{\infty 2}) & = \sigma \left( \frac{-(2a_6x^3+a_5x^2)}{2y}dx \right) \\
& = \frac{-2a_6x^3-(6aa_6+a_5)x^2-(6a^2a_6+2aa_5)x-2a^3a_6-a^2a_5}{2y}dx \\
& = \frac{-2a_6x^3-a_5x^2 -2aa_5x -2a^3a_6 - a^2a_5}{2y}dx.
\end{align*}
We now use this to compute the $d_1$ and $d_2$ above, as follows:
\begin{align*}
\sigma( \omega_{\infty 1}) - \gamma_4  & = \frac{-2a_6x^3-a_5x^2 -2aa_5x -2a^3a_6 - a^2a_5 + 2a_6x^3+a_5x^2}{2y}dx \\
& = \frac{-(2aa_5x +(2a^3a_6+a^2a_5))}{2y}dx.
\end{align*}
Hence $d_2 = -aa_5x$ and $d_1 = \frac{2a^3a_6+a^2a_5}{2}$.

We wish to write the above elements in terms of the $k$-vector space basis elements of $\cechderhamhone(\cU)$.
Recall from \cite{derhamactions} that this basis is formed by
\begin{align*}
\gamma_1  = & \left( \frac{1}{y}dx, \frac{1}{y}dx, 0\right) \\
\gamma_2 = & \left(\frac{x}{y}dx, \frac{x}{y}dx, 0\right) \\
\gamma_3 = & \left( \frac{a_3x^3-a_1x-2a_0}{2yx^2}dx, \frac{-(4a_6x^4+3a_5x^3+2a_4x^2)}{2y}dx, \frac{y}{x} \right)\\
\gamma_4 = & \left(\frac{-(a_3x^3+2a_2x^2+3a_1x+4a_0)}{2yx^3},\frac{-(2a_6x^3+a_5x^2)}{2y}dx , \frac{y}{x^2} \right).
\end{align*}
Hence
\[
\sigma(\rho'(\nu_1)) = 2 \gamma_2 + 2\gamma_4 + \gamma_3
\]
and similarly
\[
\sigma(\rho'(\nu_2)) = \gamma_4 + 2 \gamma_1.
\]
It follows that $\sigma(\gamma_3) = \gamma_3 + 2\gamma_2 + 2 \gamma_4$ and $\sigma(\gamma_4) = \gamma_4 + 2\gamma_1$.
It is also easy to see that
\[
\sigma(\gamma_1) = \gamma_1 \ {\rm and} \ \sigma(\gamma_2) = \gamma_1 + \gamma_2.
\]
Using these facts we prove the following proposition.

\begin{prop}
The short exact sequence of $k[G]$-modules
\begin{equation*}
0 \ra \hzero \ra \derhamhone \xrightarrow{p} \hone \ra 0
\end{equation*}
where $X = X_0(22)$ and $G = \aut(X)$, does not split when $\cha(k) = 3$.
\end{prop}
\begin{proof}
Suppose there is a $k[G]$-module homomorphism $s\colon \hone \ra \derhamhone$ such that $p \circ s  = {\rm id} \colon \hone \ra \hone$.
Hence $s(\bar \gamma_3) = \gamma_3 + a_1 \gamma_1 + a_2 \gamma_2$ and $s(\bar \gamma_4) = \gamma_4 + b_1 \gamma_1 + b_2 \gamma_2$, for some $a_i, b_i \in k$.
Since $s$ is a $k[G]$-module homomorphism it follows that
\[
\gamma_4 + b_1\gamma_1 + b_2(\gamma_1 + \gamma_2) = \sigma (s(\bar \gamma_4)) = s ( \sigma(\bar \gamma_4)) =  s(\bar \gamma_4) = \gamma_4 + b_1\gamma_1.
\]
Clearly $b_2 = 0$ in this case, so $s(\gamma_4 ) = \gamma_4 + b_1 \gamma_1$.

We also require that $\sigma(s(\bar \gamma_3) = s (\sigma(\bar \gamma_3))$.
On the one hand we have
\[
\sigma(s(\bar \gamma_3)) = \gamma_3 + 2\gamma_4 + 2\gamma_2 + a_1\gamma_1 + a_2(\gamma_2 + \gamma_1) = (a_1 +a_2)\gamma_1 + (a_2 + 2)\gamma_2 + \gamma_3 + 2\gamma_4
\]
whilst on the other hand we have
\[
s(\sigma( \bar \gamma_3)) = s(\bar \gamma_3 + 2 \bar \gamma_4) = \gamma_3 + a_1 \gamma_1 + a_2 \gamma_2 + 2\gamma_4 + 2b_1\gamma_1 = (a_1 + 2b_1)\gamma_1 + a_2\gamma_2 + \gamma_3 + 2\gamma_4.
\]
So for $\sigma(s(\bar \gamma_3) = s (\sigma(\bar \gamma_3))$ to hold we would require $(a_2 + 2)\gamma_2 = a_2\gamma_2$.
Since $\cha(k) \neq 2$ this is not possible, and we arrive at a contradiction.
Hence $s$ cannot exist, and the short exact sequence in the proposition does not split.
\end{proof}



\section{Modular curve $X_0(50)$}
We now consider the modular curve $X_0(50)$.
When $\cha(k) = 3$ we see from \cite[Table 1]{automorphismshyperellipticmodular} that the automorphism group is $D_6$, the dihedral group with 12 elements.
We see from \cite[Table 2]{automorphismshyperellipticmodular} that this has 
\[
y^2 = f(x) = x^6 + 2x^5 + 2x^3 + 2 + 1
\]
as a defining equation.

However, as in the last equation we need to write this in the form $\sum_i=0^3 a_{2i}x^{2i}$ for some $a_i \in k$.
\todo[inline]{Not sure what automorphism can be used to remove the $x^5$ term}
We can deduce this, up to a constant, from the \cite[Table 7]{automorphismshyperellipticmodular}; namely $f(x) = 2x^6 + x^4 +2x^2$.
We can also deduce that the automorphism of order three is $x \mapsto x+i$, where $i^2 = -1$.
Indeed, we can easily verify that $2x^6 + x^4 + 2x^2 = 2(x+i)^6 + (x+i)^4 + 2(x+i)^2$.
However, without being sure of what the constant term is we cannot compute the basis.

\section{Modular curve $X_0(28)$}
Let $k$ be an algebraically closed field of characteristic three.
We now consider the modular curve $X_0(28)$, which by \cite[Table 1]{automorphismshyperellipticmodular} has defining equation 
\begin{align*}
y^2 &  = (x^2 + 1)(x^2 + x+2)(x^2+2x+2) \\
& = x^6 + x^4 + x^2 + 1.
\end{align*}
We also see from \cite[Table 1]{automorphismshyperellipticmodular} that the automorphism group when $\cha(k) = 3$ is $GL_2(\mathbb F_3)$.

We start by computing explicitly what this action.
\todo[inline]{There appear to be some issues in the case of the generic automorphism group. In Table 1 the group is listed as $D_6$, yet in 3.1 it is claimed that the reduced group is $\ZZ_2$. Also, there are three generators listed in Table 2, but they are all of order 2. Check this with Bernhard}
\todo[inline]{Given that the explicit elements on the generic fibre are all of order two, and that on page 154 it is mentioned that the reduced group on the generic fibre is $D_4$, I am guessing that the automorphism group is meant to be $D_8$ (or $D_4$?)}

Although the automorphism group is said to be $GL_2(\mathbb F_3)$, we need only find one element that acts in the correct way to show that the sequence in \cite{derhamactions} does not split.
We can construct at least one such element in the same way that we did for $X_0(22)$.
We already have our defining equation in the form
\[
y^2 = \sum_{i=0}^3 a_{2i}x^{2i}.
\]
When we compute $f(x+c) - f(x)$ we get $(2c^3+c)x^3 + (c^3 + 2c)x + (c^6 + c^4 + c^2)$.
If we set $c=1$ or $c=2$ then all the coefficients are clearly zero.
So the map $\sigma \colon x \mapsto x-1$ is an isomorphism of our curve.

We now compute what the basis elements of $\derhamhone$ are, using \cite[Thm. 2.3]{derhamactions}.
We first note that if $f(x) = x^6 + x^4 + x^2 +1$ then $f'(x) = x^3 + 2x$.
Hence
\begin{align}
s_1 & = xf'(x) -2f(x) = x^6 + 2x^4 + 1 \\
s_2 & = xf'(x) - f(x) = 2x^6 + x^2 + 2.
\end{align}
Then it follows that the first two basis elements in \cite[Thm. 2.3]{derhamactions} are
\begin{align}
\left( \frac{1}{2yx^2}dx, \frac{2(x^6 + 2x^4)}{2yx^2}dx, \frac{y}{x} \right) & = \left( \frac{1}{2yx^2}dx, \frac{x^4+2x^2}{y}dx, \frac{y}{x} \right) \label{28basisone} \\
\left( \frac{x^2 + 2}{2yx^3}dx, \frac{2(2x^6)}{2yx^3}dx, \frac{y}{x^2} \right) & = \left( \frac{x^2 + 2}{2yx^3}dx, \frac{2x^3}{y}dx, \frac{y}{x^2} \right). \label{28basistwo}
\end{align}

As in the $X_0(22)$ case, we need to refine our cover, since the open set $U_0$ is not closed under the action of $\sigma$.
In particular, we let $\cU' = \{U_\infty, U_{1} \}$ and $\cU'' = \cU \cup \cU'$.
Then we wish to find elements of $\cechderhamhone(\cU'')$ that surject on to \eqref{28basisone} and \eqref{28basistwo}.

\todo[inline]{Rough work from here - here be dragons}
We first posit that
\[
\nu_1 = \left( \frac{1}{2yx^2}dx, \frac{x^4+2x^2}{y}dx, \frac{x^4 + 2x^3 + 2x^2 + 2x}{2y(x-1)^3}dx, \frac{y}{x}, \frac{y}{x(x-1)^2}, \frac{-y(x+1)}{(x-1)^2} \right)
\]
is such an element.
We start by checking that the alternating sum of the functions is zero:
\begin{equation*}
\frac{y}{x} - \frac{y}{x(x-1)^2} - \frac{y(x+1)}{(x-1)^2}  = \frac{y(x-1)^2 - y - yx(x+1)}{x(x-1)^2} = 0
\end{equation*}

We next check that the third differential is regular on the correct open set, as follows:
\begin{align*}
\di \left( \frac{x^4 + 2x^3 + 2x^2 + 2x}{2y(x-1)^3}dx \right) = & 3D_\infty -R +3D_\infty -3D_1 +\di_0(x^4+2x^3+2x^2+2x) \\
&  -4D_\infty +R -2D_\infty \\
= & \di_0(x^4 + 2x^3 + 2x^2 + 2x) - 3D_1.
\end{align*}
Clearly this is positive on $U_1$, as desired.

We now verify the other relations, starting with
\begin{align*}
df_{0 1 1} & = d \left( \frac{y}{x(x-1)^2} \right) \\
& = \frac{1}{x(x-1)^2}dy + y d \left( \frac{1}{x(x-1)^2} \right) \\
& = \frac{f'}{2yx(x-1)^2}dx + \frac{y}{(x-1)^3x^2}dx \\
& = \frac{1}{2yx^2(x-1)^3}\left( f'x(x-1) +2f\right)dx \\
& = \frac{1}{2yx^2(x-1)^3}(2x^6+x^5+x^4+2x^3+2)dx \\
& = \omega_{0 1} - \omega_{1 1}.
\end{align*}
Similarly
\begin{align*}
df_{\infty 1} & = d \left( \frac{2y(x+1)}{(x-1)^2} \right) \\
& = 2yd\left( \frac{x+1}{(x-1)^2} \right) + \frac{2(x+1)}{(x-1)^2}dy \\
& = \frac{yx}{(x-1)^3} dx + \frac{f'(x+1)}{y(x-1)^2}dx \\
& = \frac{1}{y(x-1)^3}(fx + f'(x^2-1))dx \\
& = \frac{1}{y(x-1)^3} (x^7 + 2x^5 + 2x^3 + 2x)dx \\
& = \frac{(x^7 + 2x^5 + 2x^4 +x^2) + (x^4 + 2x^3 + 2x^2 + 2x)}{y(x-1)^3} dx \\
& = \frac{(x^4+2x^2)(x-1)^3}{y(x-1)^3}dx - \frac{x^4+2x^3+2x^2+2x}{2y(x-1)^3}dx \\
& = \omega_2 - \omega_3,
\end{align*}
concluding our proof that $\nu_1$ is an element of $\cechderhamhone(\cU'')$.

We next show that
\[
\nu_2 = \left( \frac{x^2+2}{2yx^3}dx, \frac{2x^3}{y}dx, \frac{x^3+x^2+x+2}{2y(x-1)^3}dx, \frac{y}{x^2}, \frac{y(x+1)}{x^2(x-1)^2}, \frac{2y}{(x-1)^2} \right) 
\]
is also an element of $\cechderhamhone(\cU')$.
Note that $\nu_1$ and $\nu_2$ are linearly independent.

Firstly
\[
s_2(x) = xf'(x) - f(x) = 2x^6 + x^2 +2,
\]
and hence
\[
\psi_2 = x^2 + 2 \ {\rm and} \ \phi_2 = 2x^6.
\]
We then conclude
\[
\frac{\psi_2(x)}{2yx^3}dx = \frac{x^2+2}{2yx^3}dx \ {\rm and} \ \frac{-\psi_2}{2yx^3}dx = \frac{2x^3}{y}dx.
\]

We now check that $df_{0 1 2} = \omega_{0 2} - \omega_{1 2}$ as follows
\begin{align*}
df_{0 1 2} & = d\left(\frac{y(x+1)}{x^2(x-1)^2} \right) \\
& = \frac{x+1}{x^2(x-1)^2}dy + y d\left( \frac{x+1}{x^2(x-1)^2} \right) \\
& = \frac{f'(x)(x+1)}{2yx^2(x-1)^2} dx - \frac{y}{x^3(x-1)^3}dx \\
& = \frac{f'(x)x(x^2-1) + f(x)}{2yx^3(x-1)^3}dx \\
& = \frac{2x^6 + 2x^4 + 2x^2 +1}{2yx^3(x-1)^3} dx \\
& = \frac{(x^5 + 2x^3 + 2x^2 + 1) + 2x^3(x^3 + x^2 + x + 2)}{2yx^3(x-1)^3}dx \\
& = \omega_1 - \omega_3.
\end{align*}

Similarly we see that $df_{\infty 1 2} = \omega_{\infty 2} - \omega_{1 2}$:
\begin{align*}
df_{\infty 1 2} & = d \left( \frac{2y}{(x-1)^2} \right) \\
& = 2y d\left( \frac{1}{(x-1)^2} \right) + \frac{2}{(x-1)^2} dy \\
& = \frac{2y}{(x-1)^3}dx + \frac{f'(x)}{y(x-1)^2}dx \\
& = \frac{2f(x) + (x-1)f'(x)}{y(x-1)^3} dx \\
& = \frac{2x^6 + 2x^3 + x^2 + x + 2}{y(x-1)^3} dx \\
& = \frac{2x^3(x-1)^3 + (x^3 + x^2 + x + 2)}{y(x-1)^3}dx \\
& = \omega_{\infty 2} - \omega_{12}.
\end{align*}

These project down to 
\[
\eta_1 = \left( \frac{x^4 + 2x^2}{y}dx , \frac{x^4 + 2x^3 + 2x^2 + 2x}{2y(x-1)^3}dx, \frac{2y(x+1)}{(x-1)^2} \right)
\]
and
\[
\eta_2 = \left(\frac{2x^3}{y}dx, \frac{x^3+x^2+x+2}{2y(x-1)^3}dx, \frac{2y}{(x-1)^2} \right).
\]

The basis of $\cechderhamhone (\cU)$ as given in \cite{derhamactions} can be written more explicitly as 
\begin{align*}
\alpha_1 & = \left( \frac{1}{y}dx, \frac{1}{y}dx, 0 \right) \\
\alpha_2 & = \left( \frac{x}{y}dx, \frac{x}{y}dx, 0 \right) \\
\alpha_3 & = \left( \frac{1}{2yx^2}dx, \frac{x^4 + 2x^2}{y}dx, \frac{y}{x} \right) \\
\alpha_4 & = \left( \frac{x^2 + 2}{2yx^3}dx, \frac{2x^3}{y}dx, \frac{y}{x^2} \right)
\end{align*}

We first compute the action of $\sigma$ on $\eta_1$ and $\eta_2$, which is as follows:
\[
\sigma(\eta_1) = \left( \frac{x^4 + 2x^2 + x +1}{2yx^3}dx, \frac{x^4 + x^3 + 2x^2 + 2x}{y}dx, \frac{y(x+2)}{x^2} \right) 
\]
and
\[
\sigma(\eta_2) = \left( \frac{x^3 + x^2 +2}{2yx^3}dx, \frac{2x^3 + 2}{y}dx, \frac{y}{x^2} \right):
\]
recall that we change the sign of the third time.

Then we have
\begin{align*}
\sigma(\eta_1) &= 2\alpha_4 + \alpha_3 + 2\alpha_2\\
\intertext{and}\\
\sigma(\eta_2) & = \alpha_4 + 2\alpha_1
\end{align*}
The action here is the same as for $X_0(22)$ and hence the short exact sequence here does not split.



\bibliography{biblio}
\bibliographystyle{amsalpha}


\end{document}
