% !TEX TS-program = pdflatex
% !TEX encoding = UTF-8 Unicode

% This is a simple template for a LaTeX document using the "article" class.
% See "book", "report", "letter" for other types of document.

\documentclass[draft, 11pt]{article} % use larger type; default would be 10pt

\usepackage[utf8]{inputenc} % set input encoding (not needed with XeLaTeX)

%%% Examples of Article customizations
% These packages are optional, depending whether you want the features they provide.
% See the LaTeX Companion or other references for full information.

%%% PAGE DIMENSIONS
\usepackage{geometry} % to change the page dimensions
\geometry{a4paper} % or letterpaper (US) or a5paper or....
% \geometry{margins=2in} % for example, change the margins to 2 inches all round
% \geometry{landscape} % set up the page for landscape
%   read geometry.pdf for detailed page layout information

\usepackage{graphicx} % support the \includegraphics command and options

%\usepackage[parfill]{parskip} % Activate to begin paragraphs with an empty line rather than an indent

%%% PACKAGES
\usepackage{mathtools}
\usepackage{booktabs} % for much better looking tables
\usepackage{array} % for better arrays (eg matrices) in maths
\usepackage{paralist} % very flexible & customisable lists (eg. enumerate/itemize, etc.)
\usepackage{verbatim} % adds environment for commenting out blocks of text & for better verbatim
\usepackage{subfig} % make it possible to include more than one captioned figure/table in a single float
\usepackage[obeyDraft]{todonotes}
% These packages are all incorporated in the memoir class to one degree or another...

%%% HEADERS & FOOTERS
\usepackage{fancyhdr} % This should be set AFTER setting up the page geometry
\pagestyle{fancy} % options: empty , plain , fancy
\renewcommand{\headrulewidth}{0pt} % customise the layout...
\lhead{}\chead{}\rhead{}
\lfoot{}\cfoot{\thepage}\rfoot{}

%%% SECTION TITLE APPEARANCE
\usepackage{sectsty}
\allsectionsfont{\sffamily\mdseries\upshape} % (See the fntguide.pdf for font help)
\usepackage{amsmath}
\usepackage{amsthm}
\usepackage{amsfonts}
\usepackage{mathrsfs}
\usepackage{amsopn}
\usepackage{amssymb}
%\usepackage{natbib}
% (This matches ConTeXt defaults)

%%% ToC (table of contents) APPEARANCE
\usepackage[nottoc,notlof,notlot]{tocbibind} % Put the bibliography in the ToC
\usepackage[titles,subfigure]{tocloft} % Alter the style of the Table of Contents
\renewcommand{\cftsecfont}{\rmfamily\mdseries\upshape}
\renewcommand{\cftsecpagefont}{\rmfamily\mdseries\upshape} % No bold!

%Theorems and stuff
\theoremstyle{plain}
\newtheorem{defn}{Definition}[section]
\newtheorem{thm}[defn]{Theorem}
\newtheorem{cor}[defn]{Corollary}
\newtheorem{lem}[defn]{Lemma}
\newtheorem{prop}[defn]{Proposition}
\newtheorem{ex}[defn]{Example}
\newtheorem*{unnumthm}{Theorem}
\newtheorem{defnlem}[defn]{Definition/Lemma}
\newtheorem{defnthm}[defn]{Theorem/Definition}
\theoremstyle{remark}
\newtheorem*{rem}{Remark}


\newcommand{\cO}{{\cal O}}
\newcommand{\ra}{\rightarrow}
\newcommand{\NN}{{\mathbb N}}
\newcommand{\PP}{{\mathbb P}}
\newcommand{\ZZ}{{\mathbb Z}}
\newcommand{\cL}{{\mathcal L}}
\newcommand{\cA}{{\mathcal A}}
\newcommand{\cD}{{\mathcal D}}
\newcommand{\cU}{{\mathcal U}}
\newcommand{\cech}{\v{C}ech }
\newcommand{\hzero}{{H^0(X,\Omega_X)}}
\newcommand{\hone}{H^1(X,\mathcal{O}_X)}
\newcommand{\cechhone}{\check{H}^1(\mathcal U,\mathcal O_X)}
\newcommand{\derhamhone}{H_{\text {dR}}^1(X/k)}
\newcommand{\cechderhamhone}{\check{H}_{\text {dR}}^1(X/k)}
\newcommand{\ie}{i.e.\ }


\DeclareMathOperator{\aut}{Aut}
\DeclareMathOperator{\ord}{ord}
\DeclareMathOperator{\di}{div}
\DeclareMathOperator{\cha}{char}
\DeclareMathOperator{\gal}{Gal}
\DeclareMathOperator{\Tr}{Tr}

%%% END Article customizations

%%% The "real" document content comes below...

\title{Group actions on de Rham cohomology of hyperelliptic~curves}
\author{J. Tait}
%\date{} % Activate to display a given date or no date (if empty),
         % otherwise the current date is printed 

\begin{document}
\maketitle
\listoftodos
\bigskip

\todo[inline]{replace $\sigma$ subscripts with $a$ throughout}

We let $X$ be a hyperelliptic curve of genus $g$ over an algebraically closed field $k$ characteristic $p \neq 2$.
We fix a degree two map $\pi \colon X \ra \mathbb{P}^1_k$, which is unique up to isomorphism of $\mathbb{P}_k^1$.
Then the extension $K(X)$ of $k(x) = K(\mathbb{P}_k^1)$ is $k(x,y)$ where $y$ satisfies
\begin{equation}\label{definingequation}
y^2 = f(x) \in k[x],
\end{equation}
for some $f(x)$ which has no repeated roots and degree $2g+1$ or $2g+2$.
We can furthermore assume, without loss of generality, that $f(x)$ is monic.


In previous work we found a basis for the de Rham cohomology of any hyperelliptic curve using \cech cohomology.
In particular, if we let $\cU = \{ U_0 , U_\infty\}$ (recall that $U_z~=~X\backslash \pi^{-1}(z)$ for any $z \in \mathbb P_k^1$), then a $k$-vector space basis of $\cechderhamhone(\cU)$ is formed by
\begin{align*}
\lambda_i  = & \left( \frac{x^i}{y}dx, \frac{x^i}{y}dx, 0\right) ,\quad i=0, \ldots, g-1 \\
\intertext{and}
\gamma_i = & \left ( \frac{\psi_i(x)}{2yx^{i+1}}dx, \frac{-\phi_i(x)}{2yx^{i+1}}dx, x^{-i}y \right), \quad i=1,\ldots ,g.
\end{align*}
See \cite{derhamactions} for details.\todo{recall $\psi$ and $\phi$}
We fix some $a \in \mathbb P_k^1\backslash \{0, \infty\}$ and define $U_a : = X \backslash \pi^{-1}(a)$.
We then let $\cU''~=~\{U_0,U_a, U_\infty\}$.
We now recall that $\cechderhamhone(\cU'')$ can be defined as the $k$-vector space 
\begin{multline}\label{sixtupleconditions}
\left\{ (\omega_0, \omega_\sigma, \omega_\infty , f_{0\sigma}, f_{0 \infty},f_{\sigma \infty}) | \omega_i \in \Omega_X(U_i), f_{ij} \in \cO_X(U_i \cap U_j), \right. \\ \left. f_{0\sigma} - f_{0\infty} + f_{\sigma \infty} = 0, df_{ij} = \omega_i - \omega_j \right\}
\end{multline}
quotiented by the subspace 
\[
\left\{ df_0, df_\sigma, df_\infty, f_0- f_\sigma, f_0 - f_\infty, f_\sigma - f_\infty | f_i \in \cO_X(U_i)\right\}.
\]

We also define the following polynomials for $1 \leq i \leq g$
\[
r_i(x) : = \sum_{k=0}^{i-1} (-1)^{g-k}\binom{g}{k} a^{g-k} x^k
\]
and
\[
t_i(x) := \sum_{k=i}^{g} (-1)^{g-k}\binom{g}{k} a^{g-k} x^k,
\]
splitting the polynomial $(x-a)^g$ in to two parts.
The following proposition allows computes a basis of $\cechderhamhone(\cU'')$.


\begin{prop}\label{basis22}
The pre-image of $\rho^{-1}(\gamma_i)$ for $i \in \{1, \ldots, g\}$ is the residue class of
\begin{multline*}
\nu_i = \left(\frac{\psi_i(x)}{2yx^{i+1}}dx, \frac{(\psi_i(x)t_i(x) - \phi_i(x)r_i(x))(x-a) + 2if(-1)^{g-i+1}\binom{g}{i} a^{g-i+1}x^i}{2yx^{i+1}(x-a)^{g+1}}dx,\right. \\\left. \frac{\phi_i(x)}{2yx^{i+1}}dx,  \frac{r_i(x)y}{x^i(x-a)^g}, \frac{y}{x^i},  \frac{t_i(x)y}{x^i(x-a)^g} \right).
\end{multline*}
\end{prop}
\begin{rem}
Note that the fourth term, $\frac{\phi_i(x)}{2yx^{i+1}} dx$, has a different sign than the equivalent term in $\gamma_i$.
This is due the increased cover and moving it from the second to third position.
\end{rem}
\begin{proof}
In order to be able to refer to the entries in $\nu_i$ we let
\[
\nu_i = \left( \omega_{0 i}, \omega_{\sigma i}, \omega_{\infty i}, f_{0 \sigma i}, f_{0 \infty i}, f_{\sigma \infty i} \right).
\]
First, note that it follows from \cite{derhamactions} that $d(f_{0 \infty i}) = \omega_{0 i} + \omega_{\infty i}$, and that $f_{0 \infty i}, \omega_{0 i}$ and $\omega_{\infty i}$ are regular on the appropriate open sets.

Since $r_i(x)+t_i(x)$ is the binary expansion of $(x-a)^g$ then
\begin{align*}
f_{0 \sigma i} - f_{0 \infty i}+ f_{\sigma \infty i} & = \frac{r_i(x)y}{x^i(x-a)^g} - \frac{y}{x^i} + \frac{t_i(x)y}{x^i(x-a)^g} \\
& = \frac{y(r_i(x) + t_i(x) - (x-a)^g)}{x^i(x-a)^g} \\
& = 0.
\end{align*}



We now check that differentials and functions are regular on the appropriate open sets by computing the relevant divisors.
Firstly
\begin{align*}
\di \left( f_{0 \sigma i} \right) & = \di \left( \frac{r_i(x)y}{x^i(x-a)^g} \right) \\
&  = \di(r_i(x)) + \di(y) - i\di(x) - g\di(x-a) \\
& \geq \di_0(r_i(x)) - (i-1)D_\infty +R - (g+1)D_\infty - iD_0 + iD_\infty - gD_a + gD_\infty \\
& = \di_0(r_i(x)) +R -iD_0 - gD_a,
\end{align*}
which is non-negative on $U_0 \cap U_a$.
On the other hand,
\begin{align*}
\di \left( f_{\sigma \infty i} \right) & = \di \left( \frac{t_i(x)y}{x^i(x-a)^g} \right) \\
& = \di\left(\frac{t_i(x)}{x^i}\right) + \di(y) - g\di(x-a) \\
& = \di_0 \left( \frac{t_i(x)}{x^i} \right) - (g-i)D_\infty +R - (g+1)D_\infty - gD_a + gD_\infty\\
& = \di_0 \left( \frac{t_i(x)}{x^i} \right) +R - gD_a -(g-i+1)D_\infty,
\end{align*}
where the third equality holds because $t_i(x)/x^i$ is regular on $U_\infty$.
Since $y$ is also regular on $U_\infty$  we conclude that $({t_i(x)y})/({x^i(x-a)^g})$ is regular on $U_a \cap U_\infty$.

To show that
\begin{equation}\label{longequation}
\omega_{\sigma i} =  \frac{(\psi_i(x)t_i(x) - \phi_i(x)r_i(x))(x-a) + 2if(-1)^{g-i+1}\binom{g}{i} a^{g-i+1}x^i}{2yx^{i+1}(x-a)^{g+1}}dx
\end{equation}
is regular on $U_a$ we first compute the divisor
\begin{align*}
\di\left( \frac{dx}{2yx^{i+1}(x-a)^{g+1}}\right) & = \di(dx) - \di(y) - (i+1)\di(x) - (g+1)\di(x-a) \\
& = R - 2D_\infty - R + (g+1)D_\infty - (i+1)D_0 + (i+1)D_\infty \\
& \quad - (g+1)D_a + (g+1)D_\infty \\
& = (2g+i+1)D_\infty -(i+1)D_0 - (g+1)D_a.
\end{align*}
We next show that the numerator of \eqref{longequation},
\begin{equation}\label{numerator}
{(\psi_i(x)t_i(x) - \phi_i(x)r_i(x))(x-a) + 2if(-1)^{g-i+1}\binom{g}{i} a^{g-i+1}x^i},
\end{equation}
has degree less then $2g+i+2$, from which it follows that \eqref{longequation} doesn't have a pole at the point(s) in $\pi^{-1}(\infty)$.
The degree of $\psi_i(x)t_i(x)(x-a)$ is at most $2g+2$, which is less than $2g+2+i$ for all $i \geq 1$.
If $\deg(f) = 2g+1$, then clearly\todo{check that it should be $2g$ and not $g$}
\[
\deg\left( \phi_i(x)r_i(x)(x-a) \right) = \deg(\phi_i) + \deg(r_i(x)) + \deg(x-a) \leq 2g+1 + i-1 +1 = 2g+i+1
\]
and
\[
\deg \left( 2if(-1)^{g-i+1}\binom{g}{i} a^{g-i+1}x^i \right)  \leq  2g+1+i .
\]
Lastly, if $\deg(f) = 2g+2$ then the leading term \todo{rephrase - don't use "leading term", in case coefficient is zero}of $-\phi_i(x)r_i(x)(x-a)$ is
\begin{align*}
-((2g+2)a_{2g+2}x^{2g+2}-2ia_{2g+2}x^{g+2})&\left( (-1)^{g-i+1}\binom{g}{i-1}a^{g-i+1}x^i\right) \\
&  = 2(-1)^{g-i+2}\left( (g-i+1)\binom{g}{i-1} \right) a_{2g+2}x^{2g+i+2} \\
& = 2(-1)^{g-i+2} \left( \frac{g!}{(i-1)!(g-i)!} \right) a_{2g+2}x^{2g+i+2} \\
& = 2i(-1)^{g-i+2}\binom{g}{i}a_{2g+2}x^{2g+i+2},
\end{align*}
which clearly cancels with the leading term of $2if(-1)^{g-i+1}\binom{g}{i}a^{g+i-1}x^i$.
Since these terms cancel, we again have the that the degree of \eqref{numerator} is at most $2g+i+1$, and \eqref{longequation} has no pole at the point(s) in $\pi^{-1}(\infty)$.

Finally, we show that \eqref{numerator} is divisible by $x^{i+1}$.
By definition $x^{g+1} | \phi_i(x)$, and since $i \leq g$ it follows that $x^{i+1}|\phi_i(x)r_i(x)(x-a)$.
Also, both
\[
2if(-1)^{g-i+1}\binom{g}{i}a^{g-i+1}x^i - 2ia_0(-1)^{g-i+1}\binom{g}{i}a^{g-i+1}x^i 
\]
and
\[
\psi_i(x)t_i(x)(x-a) - (-a)(-2ia_0)\left( (-1)^{g-i}\binom{g}{i}a^{g-i}x^i \right)
\]
are divisible by $x^{i+1}$.
Moreover, 
\[
(-a)(-2ia_0)\left( (-1)^{g-i}\binom{g}{i}a^{g-i}x^i \right) = 2ia_0(-1)^{g-i+2}\binom{g}{i}a^{g-i+1}x^i,
\]
so these terms cancel when adding $\psi_i(x)t_i(x)(x-a)$ and $2if(-1)^{g-i+1}\binom{g}{i}a^{g-i+1}x^i$ and the numerator, \eqref{numerator}, is divisible by $x^{i+1}$.


It only remains to show that $\omega_{\sigma i} = \omega_{0 i} -df_{0 \sigma i}$.
We begin this by computing $df_{0 \sigma i}$, which is
\begin{align*}
df_{0 \sigma i} & = d \left( \frac{y r_i(x)}{x^i(x-a)^g} \right) \\
& = \frac{r_i(x)}{x^i(x-a)^g}dy + y d\left( \frac{r_i(x)}{x^i(x-a)^g} \right) \\
& = \frac{f(x)'r_i(x)}{2yx^i(x-a)^g}dx + \left( \frac{yr_i(x)'}{x^i(x-a)^g} -\frac{iy r_i(x)}{x^{i+1}(x-a)^g} - \frac{gyr_i(x)}{x^i(x-a)^{g+1}}\right) dx \\
& = \frac{xf(x)'r_i(x)(x-a) + 2f(xr_i(x)'(x-a) - i(x-a)r_i(x) - gxr_i(x))}{2yx^{i+1}(x-a)^{g+1}} dx.
\end{align*}
Hence $\omega_{0 i} - df_{o \sigma i}$ expands to
\[
\frac{\psi_i(x)(x-a)^{g+1} - xf(x)'r_i(x)(x-a) + 2f(x)(xr_i(x)'(x-a)-i(x-a)r_i(x)-gxr_i(x))}{2yx^{i+1}(x-a)^{g+1}}dx.
\]
Now
\[
(x-a)^{g+1} = (x-a)^g(x-a)  = (r_i(x) + t_i(x))(x-a)
\]
and
\[
xf(x)'r_i(x)(x-a) - 2if(x)r_i(x)(x-a) = r_i(x)(x-a)(xf(x)'-2if(x)) = r_i(x)(x-a)(\psi_i(x) + \phi_i(x)).
\]
So
\[
\psi_i(x)(x-a)^{g+1} - xf'r_i(x)(x-a) - 2ifr_i(x)(x-a) = (\psi_i(x)t_i(x) - \phi_i(x) r_i(x))(x-a).
\]


We now compute $(x-a)r_i(x)'-gr_i(x)$.
Since 
\begin{align*}
r_i(x)' & = \sum_{k=0}^{i-1} k (-1)^{g-k} \binom{g}{k} a^{g-k} x^{k-1} \\
& = \sum_{k=0}^{i-2} (k+1) (-1)^{g-k-1} \binom{g}{k+1}a^{g-k-1} x^k
\end{align*}
it follows that
\begin{align*}
r_i(x)'(x-a) & = x \sum_{k=0}^i k (-1)^{g-k} \binom{g}{k} a^{g-k} x^{k-1} - a \sum_{k=0}^{i-1} (k+1) (-1)^{g-k-1} \binom{g}{k+1}a^{g-k-1} x^k \\
& = \sum_{k=0}^i k (-1)^{g-k} \binom{g}{k} a^{g-k} x^k  + \sum_{k=0}^{i-1} (k+1) (-1)^{g-k} \binom{g}{k+1}a^{g-k} x^k,
\end{align*}
and so the coefficient of $x^k$ in $r_i(x)'(x-a)$ is $(-1)^{g-k}\left(k\binom{g}{k} + (k+1)\binom{g}{k+1}\right) a^{g-k}$ for $0 \leq k \leq i-1$.
On the other hand, the coefficient of $x^k$ in $gr_i(x)$ is $(-1)^{g-k}g\binom{g}{k}a^{g-k}$.
Since 
\begin{align*}
k\binom{g}{k} + (k+1)\binom{g}{k+1} & = k \left( \frac{g!}{k!(g-k)!} \right) + (k+1) \left( \frac{g!}{(k+1)!(g-k-1)!} \right) \\
& = \frac{g!}{(k-1)!(g-k)!} + \frac{g!}{k!(g-k-1)!} \\
& = \frac{g\cdot g!}{k!(g-k)!} \\
& = g \binom{g}{k}
\end{align*}
it follows that $r_i(x)'-gr_i(x)$ has only one term, which is of order $i$.
This term is
\begin{align*}
 (-1)^{g-i+1}\left( (i-1-g)\frac{g!}{(i-1)!(g-i+1)!} \right) & a^{g-i+1}x^i \\
&  = (-1)^{g-i+2} \frac{g!}{(i-1)!(g-i)!} a^{g-i+1}x^i \\
& = (-1)^{g-i+2}i\binom{g}{i} a^{g-i+1}x^i.
\end{align*}

Combining the above we conclude that
\[
\omega_{0 i } - df_{0\sigma i} =  \frac{(\psi_i(x)t_i(x) - \phi_i(x)r_i(x))(x-a) + 2if(-1)^{g-i+1}\binom{g}{i} a^{g-i+1}x^i}{2yx^{i+1}(x-a)^{g+1}}dx.
\]

Note that the last relation ($df_{\sigma \infty i} = \omega_{\sigma i} - \omega_{\infty i}$) holds, since 
\[
df_{\sigma \infty i} = df_{0 \infty i} - df_{0 \sigma i} = \omega_{0 i} - \omega_{\infty i } - \omega_{0 i} + \omega_{\sigma i} = \omega_{\sigma i} - \sigma_{\infty i}.
\]
\end{proof}

Now we suppose that $\sigma \in \aut(X)$ such that the following diagram commutes:
\[
\begin{array}{ccc}
X & \xrightarrow[\sigma] & X \\
\downarrow\pi & & \downarrow\pi \\
\mathbb P^1_k & \xrightarrow[\sigma] & \mathbb P_k^1
\end{array}
\]
Hence, if $a = \sigma(0)$, the following diagram commutes:
\[
\begin{array}{ccc}
\derhamhone \cong \cechderhamhone(\cU)  & \xleftarrow{\rho} & \cechderhamhone(\cU'')  \\
\sigma^*\downarrow & ~ & \rho'\downarrow  \\
\derhamhone \cong \cechderhamhone(\cU)  & \xleftarrow{\sigma^*} & \cechderhamhone(\cU')
\end{array}
\]
Here $\rho$ and $\rho'$ are the projections on the first, third and fifth coordinates and on the second, third and sixth coordinates respectively.


Clearly if $\sigma^* x \mapsto x+a$ then $a = \sigma(0)$: we assume in what follows that $\sigma$ is such an automorphism.
\begin{lem}
The action of $\sigma^*$ on $y$ is given by $\sigma^*(y) = y$ or $\sigma^*(y) = -y$.
\end{lem}
\begin{proof}
Since $y^2 \in k(x)$ then there must exist $g_1(x), g_2(x) \in k(x)$ such that 
\begin{equation*}
\sigma^*(y) = g_1(x)y + g_2(x) \in k(x,y).
\end{equation*}
Hence
\begin{equation}\label{easylemma}
f(x+a) = \sigma^*(y^2) = (\sigma^*(y))^2 = g_1(x)^2f(x)+2g_1(x)g_2(x)y + g_2(x)^2.
\end{equation}
Firstly, note that if neither $g_1(x)$ nor $g_2(x)$ are zero then
\[
y = \frac{f(x+a) - g_1(x)f(x) - g_2(x)^2}{2g_1(x)g_2(x)},
\]
which contradicts the fact $K(X)$ is a degree two extension of $k(x)$.
Hence one of $g_1(x)$ or $g_2(x)$ must be zero.

If $g_1(x) = 0$ then $\sigma^*$ would not be an automorphism, since $y$ would not be in the image.
Hence $\sigma^*(y) = g_1(x)y$.
Also, by comparing the degrees in \eqref{easylemma} we see that $\deg(g_1(x)) = 0$, and then by comparing coefficients in the same equation we see that $g_1(x)^2 = 1$.
Hence $\sigma^*(y) = \pm y$.
\end{proof}
\begin{rem}
If $\sigma^*(y) = -y$ we can, without loss of generality, replace $\sigma$ by $\sigma \circ j$, where $j$ denotes the hyperelliptic involution.
Hence we will assume throughout the rest of the paper that $\sigma^*(y) = y$.
\end{rem}


We now describe the action of $\sigma^*$ on $\lambda_i$.
\begin{lem}
For each $i \in \{ 0, \ldots, g-1\}$ then 
\[
\sigma^*(\lambda_i) = \sum_{k = 0}^i \binom{i}{k}a^{i-k}x^k.
\]
\end{lem}
\begin{proof}
Since $\sigma^*$ acts trivially on $y$, it follows that
\[
\sigma^*\left( \frac{x^i}{y} \right) = \sum_{k=0}^i \binom{i}{k}a^{i-k}x^k.
\]
The statement follows from this.
\end{proof}


\begin{thm}
Suppose that $\deg(f(x)) = p^n $ for some $n \in \NN$.
Then the short exact sequence of $k[G]$-modules
\[
0 \ra \hzero \ra \derhamhone \ra \hone \ra 0
\]
does not split.
\end{thm}
\begin{proof}
We suppose that the sequence does split, and that $s \colon \hone \ra \derhamhone$, given by $s \colon \bar\gamma_i \mapsto \gamma_i$ is the splitting map.

We now examine the action of $\sigma^*$ on $\gamma_g$ and $\bar \gamma_g$.
We first look at the sixth entry in $\nu_g$ from Proposition \ref{basis22}, which $\sigma^*$ maps to $f_{0\infty}$.
This entry is
\[
\frac{t_g(x)y}{x^g(x-a)^g} = \frac{y}{(x-a)^g},
\]
and clearly $\sigma^*(y/(x-a)^g) = y/x^g$.
Hence $\sigma^*(\bar\gamma_g) = \bar \gamma_g$, and $\sigma^*(\gamma_g) = \gamma_g + \sum_{i =0}^{g-1}c_i\lambda_i$, for some $c_i \in k$.
We now compute $c_{g-1}$.

We begin this by computing the lead term of the first entry of $\sigma^*(\gamma_g)$.
This is equal to the lead term of $\omega_{\sigma g}$ in $\nu_g$, since applying $\sigma^*$ doesn't change the lead term of any polynomials.
As
\[
\deg(\psi_g(x)t_g(x)x) \leq g+1 + g+ 1 = 2g+1 < 3g+1 = \deg(\phi_g(x)r_g(x)x) = \deg(f(x)x^g
\]
we need only compute the coefficient of $x^{3g+1}$ in $2gf(x)(-1)ax^g - \phi_g(x)r_g(x)x$ and show that it is non-zero.
Rearranging $2g+1 = p^n - 1$ gives us the identity
\[
g = \frac{p^n - 1}{2}.
\]
From this we see that the lead coefficient of $2gf(x)(-1)ax^g$ is 
\[
2\left( \frac{p^n-1}{2} \right) (-1)a = a
\]
since $\cha(k) = p$.
On the other hand the lead term of $-\phi_g(x)r_g(x)x$ is
\[
-(-2g)(-1)\binom{g}{g-1}a = 2\left(\frac{p^n -1 }{2}\right) (-1)\left( \frac{p^n - 1}{2} \right)a = \frac{-a}{2}.
\]
Finally, it follows that the lead coefficient of the numerator in the second term of is
\[
a - \frac{a}{2} = \frac{a}{2}.
\]


Since the denominator of $\omega_{\sigma g}$ is of degree $2g+2$, we see that overall the degree of the first entry of $\gamma_g$ is $g-1$ (as a polynomial in $x$).
Now the degree of $\frac{\psi_g(x)}{2yx^{g+1}}$ is less than this, as is the degree of $\lambda_i$, unless $i=g-1$, when the degree of $\lambda_i$ is precisely $g-1$.
Hence, by comparing coefficients, we see that $c_{g-1} = \frac{a}{2}$.

Now suppose that 
\[
s(\bar\gamma_i) = \gamma_g + \sum_{i=0}^{g-1}d_i \lambda_i
\]
for some $d_i \in k$.
Then, on the one hand,
\[
s(\sigma^*(\bar\gamma_g)) = \gamma_g + \sum_{i=0}^{g-1}d_i\lambda_i,
\]
whilst on the other hand
\begin{align}
\sigma^*(s(\bar\gamma_g)) & = \sigma^*(\gamma_g + \sum_{i=1}^{g-1} d_i\lambda_i ) \\
& = \gamma_g + \frac{a}{2}\lambda_{g-1} + \sum_{i=0}^{g-2} c_i x^k + \sum_{i=0}^{g-1} d_i \sum_{k=0}^{i} \binom{i}{k}a^{i-k}x^k.
\end{align}
Hence we see that for $s(\sigma^*(\gamma_g))$ to equal $\sigma^*(s(\gamma_g))$ we require $\frac{a}{2} + d_{g-1} = d_{g-1}$; \ie that $\frac{a}{2} = 0$. 
But then $a=0$, which is a contradiction.

\end{proof}











\begin{comment}
We now suppose that $X$ is of genus 2, and that
\[
f(x) = a_6x^6 + a_5x^5 + a_4x^4 + a_3x^3 + a_2x^2 + a_1x + a_0.
\]
The existence of $\sigma^*$ places restrictions on the characteristic of $k$ and the coefficients of $f(x)$, as stated in the following lemma.

\begin{lem}
The characteristic of $k$ is 3 or 5. 
Moreover, if $\cha(k)=5$ then the coefficients $a_6, a_4, a_3$ and $a_2$ in \eqref{definingequation} are zero.
\end{lem}
\begin{proof}
Since $\sigma^*(y) = y$, it follows that $f(x+a) = f(x)$.
Expanding the left hand side we see that the coefficient of $x^5$ is $6aa_6 + a_5$.
Since the coefficient of $x^5$ in $f(x)$ is $a_5$.
We conclude that $6aa_6 = 0$, and hence either $6=0$ or $a_6=0$.
If $6=0$ then it follows that $\cha (k)=3$.

On the other hand, if $a_6=0$ then we compare the coefficients of $x^4$ in $f(x+a)$ and $f(x)$, which are $5aa_5+a_4$ and $a_4$ respectively.
We conclude that $5aa_5=0$. Since the degree of $f(x)$ is either $2g+2=6$ or $2g+1=5$ it follows that at least one of $a_6$ and $a_5$ must be non-zero.
Hence if $a_6=0$ it follows that $\cha(k) = 5$.

The final statement follows again by comparing coefficients.
\end{proof}


Recall that $\sigma^*$ fixes $y$ and maps $x$ to $x+a$. 
Note that this means that $\sigma^*(dx) = d(\sigma^*(x)) = d(x+a) = dx$.
The action of $\sigma^*$ on $\nu_i$ is given by
\begin{equation*}
\sigma^*(\nu_i) = \left( \sigma^*( \omega_{\sigma i}), \sigma^*(\omega_{\infty i}), \sigma^*( f_{\sigma \infty i})\right).
\end{equation*}

Note that in this specific case the basis of $\cechderhamhone(\cU)$ is formed by
\begin{align*}
\gamma_1  = & \left( \frac{1}{y}dx, \frac{1}{y}dx, 0\right) \\
\gamma_2 = & \left(\frac{x}{y}dx, \frac{x}{y}dx, 0\right) \\
\gamma_3 = & \left( \frac{a_3x^3-a_1x-2a_0}{2yx^2}dx, \frac{-(4a_6x^4+3a_5x^3+2a_4x^2)}{2y}dx, \frac{y}{x} \right)\\
\gamma_4 = & \left(\frac{-(a_3x^3+2a_2x^2+3a_1x+4a_0)}{2yx^3},\frac{-(2a_6x^3+a_5x^2)}{2y}dx , \frac{y}{x^2} \right).
\end{align*}
Also, the pre-image $\rho^{-1}(\tau_i)$ computed in the previous proposition can be more explicitly written as 
\begin{multline*}
\nu_1 = \left(\frac{a_3x^3-a_1x-2a_0}{2yx^2}dx, \frac{\lambda_1}{2y(x-a)^3}dx, \frac{-(4a_6x^4+3a_5x^3+2a_4x^2)}{2y}dx,  \right. \\ \left. \frac{a^2y}{x(x-a)^2}, \frac{y}{x},  \frac{y(x-2a)}{(x-a)^2} \right)
\end{multline*}
and
\begin{multline*}
\nu_2 = \left( \frac{-(a_3x^3+2a_2x^2+3a_1x+4a_0)}{2yx^3},  \frac{\lambda_2}{2y(x-a)^3}dx, \frac{-(2a_6x^3+a_5x^2)}{2y}dx, \right. \\ \left.  \frac{y(a^2-2ax)}{x^2(x-a)^2},\frac{y}{x^2}, \frac{y}{(x-a)^2} \right),
\end{multline*}
where
\begin{multline*}
\lambda_1  := (4a^3a_6+a^2a_5+a_3)x^4 + (3a^3a_5+2a^2a_4-3aa_3)x^3 \\
 + (2a^3a_4 + 6a^2a_3 - a_1)x^2 + (4a^2a_2 +3aa_1-2a_0)x+(2a^2a_1+6aa_0)
\end{multline*}
and
\begin{multline*}
\lambda_2  := (-6a^2a_6-2aa_5)x^4 + (2a^3a_6-3a^2a_5-4aa_4-a_3)x^3 \\ + (a^3a_5-3aa_3-2a_2)x^2 + (-2a_2a-3a_1)x +(-aa_1-4a_0).
\end{multline*}

We use this to prove the following proposition about the action of $\sigma^*$ on $\cechderhamhone(\cU)$.

\begin{prop}
The action of $\sigma^*$ on the basis elements of $\cechderhamhone(\cU)$ is
\begin{align*}
\sigma^*(\gamma_1) & = \gamma_1 \\
\sigma^*(\gamma_2) & = \gamma_2 + a\gamma_1 \\
\sigma^*(\gamma_3) & = \gamma_3 - a\gamma_4 + \frac{4a^3a_6 + a^2a_5}{2}\gamma_2 + a^3a_5\gamma_1 \\
\sigma^*(\gamma_4) & =  \gamma_4 - \frac{2a^3a_6+a^2a_5}{2} \gamma_2 - aa_5\gamma_1.
\end{align*}
\end{prop}
\begin{proof}
The action on $\gamma_1$ and $\gamma_2$ is clear.

For $\gamma_3$ and $\gamma_4$ we will consider this action entry by entry, starting with $\sigma^*(\omega_{\sigma 1})$, which is
\begin{align*}
\sigma^*( \omega_{\sigma}) & = \sigma^* \left( \frac{\lambda_1}{2y(x-a)^3} dx \right) \\
& = \frac{\sigma^*(\lambda_1)}{2yx^3}dx.
\end{align*}
Now we compute $\sigma^*(\lambda_1)$, which is
\begin{align*}
\sigma^*(\lambda_1) & = (4a^3a_6+a^2a_5+a_3)x^4 + (16a^4a_6+7a^3a_5+2a^2a_4+aa_3)x^3 \\
& + (24a^5a_6+15a^4a^5+8a^3a_4+3a^2a_3-a_1)x^2 \\
& + (16a^6a_6+13a^5a_5+10a^4a_4+7a^3a_3+4a^2a_2+aa_1-2a_0)x \\
& + a(4a^6a_6+4a^5a_5+4a^4a_4+4a^3a_3+4a^2a_2+4aa_1+4a_0).
\end{align*}
We use the fact $f(x+a) = f(x)$ to simplify the above expression, by equating the coefficients on both side.
We list the identities we get from doing this here:
\begin{enumerate}
\item \label{1} $6aa_6=0$;
\item \label{2} $15a^2a_6+5aa_5 = 0$;
\item \label{3} $20a^3a_6+10a^2a_5+4aa_4 = 0$;
\item \label{4} $15a^4a_6 + 10a^3a_5 + 6a^2a_4+3aa_3= 0$;
\item \label{5} $6a^5a_6 + 5a^4a_5+4a^3a_4+3a^2a_3+2aa_2= 0$;
\item \label{6} $a^6a_6+a^5a_5+a^4a_4+a^3a_3+a^2a_2+aa_1 = 0$.
\end{enumerate}
Hence the constant coefficient of $\sigma^* (\lambda_1)$ simplifies to
\begin{equation*}
 a(4a^6a_6+4a^5a_5+4a^4a_4+4a^3a_3+4a^2a_2+4aa_1+4a_0) =  4aa_0
\end{equation*}
by \ref{6}; the coefficient of $x$ simplifies to
\[
16a^6a_6+13a^5a_5+10a^4a_4+7a^3a_3+4a^2a_2+aa_1-2a_0 = 3aa_1 -2a_0
\]
by \ref{5} and \ref{6}; the coefficient of $x^2$ simplifies to 
\[
24a^5a_6+15a^4a^5+8a^3a_4+3a^2a_3-a_1 = 2a_2a^2-a_1
\]
by \ref{4} and \ref{5}; the coefficient of $x^3$ simplifies to 

\[
16a^4a_6+7a^3a_5+2a^2a_4+aa_3 = 2a^3a_5 +aa_3
\]
by \ref{1} and \ref{3}.
Hence
\begin{multline*}
\sigma^*(\lambda_1) = \\ \frac{(4a^3a_6+a^2a_5+a_3)x^4+(2a^3a_5+aa_3)x^3+(2a^2a_2-a_1)x^2+(3aa_1-2a_0)x+4aa_0}{2yx^3}dx.
\end{multline*}
We also have
\begin{align}\label{i=1sigmaaction}
\sigma^*(f_{\sigma \infty 1}) & = \sigma^*\left(\frac{y(x-2a)}{(x-a)^2} \right) \nonumber\\
& = \frac{y(x-a)}{x^2}  \
 = \frac{y}{x} - \frac{ay}{x^2}.
\end{align}
From \eqref{i=1sigmaaction} we see that
\begin{equation}\label{firstidentity}
\sigma^*(\gamma_3) = \gamma_3 - a\gamma_4 + c_1\gamma_1 + c_2 \gamma_2
\end{equation}
for some $c_1,c_2 \in k$.
Finally we have
\begin{align*}
\sigma^* (\omega_{\infty 1}) & = \sigma^*\left( \frac{-(4a_6x^4+3a_5x^3+2a_4x^2)}{2y} dx \right) \\
& = - \left( \frac{4a_6x^4+(16aa_6+3a_5)x^3+(24a^2a_6+9aa_5+2a_4)x^2}{2y}\right. \\ 
& \left. +\frac{(16a^3a_6+9a^2a_5+4aa_4)x+4a^4a_6+3a^3a_5+2a^2a_4}{2y}\right) dx.
\end{align*}
Using the above we can compute the $c_1$ and $c_2$ mentioned above.
We compute the coefficient of $\omega_{\infty}$ (using the notation of \eqref{sixtupleconditions}) in $\sigma^*(\rho'(\nu_1)) - \gamma_3 + a\gamma_4$, which is
\begin{multline*}
 -\left( \frac{18aa_6x^3 +(24a^2a_6+10aa_5)x^2+(16a^3a_6+9a^2a_5+4aa_4)x}{2y} \right. \\
+ \left. \frac{4a^4a_6+3a^3a_5+2a^2a_4}{2y}\right) dx  = \frac{(4a^3a_6+a^2a_5)x +2a^3a_5}{2y}dx.
\end{multline*}
From this we see that $c_1= a^3a_5$ and that $c_2 = \frac{4a^3a_6+a^2a_5}{2}$.


Now when $i=2$ we have
\begin{align*}
\sigma^*(\lambda_2) & = -(6a^2a_6+2aa_5)x^4 - (22a^3a_6+11a^2a_5+4aa_4+a_3)x^3 \\
& - (30a^4a_6+20a^3a_5+12a^2a_4+6aa_3+2a_2)x^2 \\
& - (18a^5a_6+15a^4a_5+12a^3a_4+9a^2a_3+6aa_2+3a_1)x \\
& -4(a^6a_6+a^5a_5+a^4a_4+a^3a_3+a^2a_2+aa_1+a_0).
\end{align*}
Again, using the relations \ref{1} through \ref{6} we can simplify these coefficients:
by \ref{6} the constant coefficient of $\sigma^*(\lambda_2)$ is
\[
-4(a^6a_6+a^5a_5+a^4a_4+a^3a_3+a^2a_2+aa_1+a_0) = -4a_0;
\]
by \ref{5} the coefficient of $x$ is
\[
- (18a^5a_6+15a^4a_5+12a^3a_4+9a^2a_3+6aa_2+3a_1) = -3a_1;
\]
by \ref{4} the coefficient of $x^2$ is 
\[
- (30a^4a_6+20a^3a_5+12a^2a_4+6aa_3+2a_2) = -2a_2;
\]
by \ref{3} the coefficient of $x^3$ is
\[
- (22a^3a_6+11a^2a_5+4aa_4+a_3) = -2a^3a_5-a^2a_5-a_3;
\]
and by \ref{1} the coefficient of $x^4$ is
\[
-(6a^2a_6+2aa_5) = -2aa_5.
\]

Next we compute the action of $\sigma^*$ on $f_{\sigma \infty 2}$, as follows:
\begin{align*}
\sigma^*( f_{\sigma \infty 2}) & = \sigma^* \left( \frac{y}{(x-a)^2} \right) \\
& = \frac{y}{x^2}.
\end{align*}
Hence we see that
\[
\sigma^*( \gamma_4) = \gamma_4 + d_1\gamma_1 + d_2 \gamma_2
\]
for some $d_1, d_2 \in k$.
Lastly, we compute the action on $\omega_{\infty 2}$, which is
\begin{align*}
\sigma^* ( \omega_{\infty 2}) & = \sigma^* \left( \frac{-(2a_6x^3+a_5x^2)}{2y}dx \right) \\
& = \frac{-2a_6x^3-(6aa_6+a_5)x^2-(6a^2a_6+2aa_5)x-2a^3a_6-a^2a_5}{2y}dx \\
& = \frac{-2a_6x^3-a_5x^2 -2aa_5x -2a^3a_6 - a^2a_5}{2y}dx,
\end{align*}
using \ref{6} for the last equality.
We now use this to compute the $d_1$ and $d_2$ above, by computing the coefficient of $\omega_\infty$ of $\sigma^*(\rho'(\nu_1)) - \gamma_4$, as follows
\begin{multline*}
 \frac{-2a_6x^3-a_5x^2 -2aa_5x -2a^3a_6 - a^2a_5 + 2a_6x^3+a_5x^2}{2y}dx \\
 = \frac{-(2aa_5x +(2a^3a_6+a^2a_5))}{2y}dx.
\end{multline*}
Hence $d_2 = -aa_5$ and $d_1 = -\frac{2a^3a_6+a^2a_5}{2}$.
\end{proof}

Having computed the action we can now prove the following proposition regarding the splitting of the short exact sequence described in \cite{derhamactions}.

\begin{prop}
The short exact sequence of $k[G]$-modules
\begin{equation*}
0 \ra \hzero \ra \derhamhone \xrightarrow{p} \hone \ra 0
\end{equation*}
does not split if $\deg(f) = 5$.
\end{prop}
\begin{rem}
Note that $\deg(f)=5$ is equivalent to $\cha(k)=5$.
\end{rem}
\begin{proof}
Suppose there is a $k[G]$-module homomorphism $s\colon \hone \ra \derhamhone$ such that $p \circ s  = {\rm id} \colon \hone \ra \hone$.
Hence $s(\bar \gamma_3) = \gamma_3 + b_{31} \gamma_1 + b_{32} \gamma_2$ and $s(\bar \gamma_4) = \gamma_4 + b_{41} \gamma_1 + b_{42} \gamma_2$, for some $ b_{ij} \in k$.
Since $s$ is a $k[G]$-module homomorphism it follows that
\[
s(\sigma^* ( \bar \gamma_4)) = s ( \bar \gamma_4) = \gamma_4 + b_{41}\gamma_1 + b_{42} \gamma_2
\]
and also that
\begin{align*}
\sigma^*(s(\bar \gamma_4))  & = \sigma^* (\gamma_4 + b_{14}\gamma_1 b_{24}\gamma_2 ) \\
& = \gamma_4 -aa_5\gamma_2 - \left( \frac{2a^3a_6+a^2a_5}{2} \right) \gamma_1 + b_{14}\gamma_1 + b_{24}\gamma_2 + ab_{24}\gamma_1 \\
& = \gamma_4 + (b_{42}-aa_5)\gamma_2 + \left( \frac{2b_{14}+2ab_{24}-2a^3a_6-a^2a_5}{2} \right) \gamma_1
\end{align*}
are equal.
In particular we must have $b_{24}-aa_5 = b_{24}$, which implies that $aa_5 =0$.
If $a=0$ then $\sigma^*$ acts trivially. 
Hence $a \neq 0$ and it follows that $a_5=0$.
But if $\deg(f) = 5$ this is a contraction.
\end{proof}
\begin{rem}
Note that if $\deg(f) = 6$ then there is no contradiction with either term, and the short exact sequence can split. 
However, because we are not considering the entire automorphism group we cannot say that it does split.
\end{rem}
\todo[inline]{say what restrictions on $s$ we get from the above?}
\todo[inline]{state which exact sequences split from kontogeorgis paper}

The following is for the case where $g = 4$.
In this case our curve must have 
\[
y^2 = x^9 + a_6x^6+ a^2a_6x^4+a_3x^3a^4a_6x^2 +2(a^8 + a^2a_3)x +a_0
\]
as its defining equation, where we also have the relations
\begin{itemize}
\item $2a^6a_6 + a^4a_4 = 0$;
\item $a^a_4 + 2a^2a_2 = 0$;
\item $a^9 + a^3a_3 + aa_1 = 0$.
\end{itemize}
Also, the characteristic of $k$ is necessarily three.
As before, $sigma \colon x \to x+a$.
We denote the basis elements that come from $\hzero$ by $\lambda_i$, as $i$ ranges from $0$ to $g-1$, and the basis elements that map tpo $\hone$ by $\gamma_i$ (same range for $i$, but plus one).
Then we have the following lemma.
\begin{lem}
For $i = 0, \ldots , g-1$ then
\[
\sigma^*(\lambda_i) = \sum_{j=0}^1 \binom{i}{j} a^{i-j} \lambda_j.
\]
\end{lem}
\begin{proof}
Clear
\end{proof}

Next we explicilty write out the elements $\gamma_1, \ldots, \gamma_4$.
\begin{align*}
\gamma_1 & = \left( \frac{a^2a_6x^4 + 2a_3x^3 + 2(a^8 + a^2a_3)x + 2a_0}{x^2y}, \frac{x^7 +a_6x^3}{y}, \frac{y}{x} \right)\\
\gamma_2 & = \left(\frac{a^4a_6x^2 +2a_3x^3 +2a_0}{2x^3y} , \frac{2(x^6 +a_6x^3)}{y}, \frac{y}{ x^2 }\right)\\
\gamma_3 & = \left(\frac{a^2a_6x^4 + 2a^4a_6x^2 + 2(a^8 + a^2a_3)x }{2x^4y } , 0, \frac{y }{x^3 } \right)\\
\gamma_4 & = \left(\frac{a^2a_6x^4 +2a_3x^3 + 2(a^8 +a^2a_3)x + 2a_0 }{x^5y } , \frac{x^4+a_6x }{ y}, \frac{y }{ x^4} \right)
\end{align*}



We now compute the action of $\sigma^*$ on the above elements, computed via the sixtuple elements given earlier:
\begin{multline}
\left( \frac{a^4x^8 + 2a^2a_6x^7 + (2a^6 + a^3a_6 + a_3)x^6 + (2a^7 + aa_3)x^5 + (a^8  + a^a_3)x^4 + 2(a^6a_6+2a^3a_3+2a_0)x^3 + 2(a^{10}+a^4a_3+2aa_0)x^2 + 2(a^{11}+a^5a_3)x + 2a^3a_0}{2yx^5} \right. \\ \left. \frac{x^7 + a^6 + (2a^3+a_6)x^4 + (2a^4+aa_6)x^3 + (a^6 + a^3a_6)x + a^4a_6 + a^7}{y}, \frac{y(x^3-ax^2+a^2x-a^3)}{x^4} \right).
\end{multline}
From this we see that
\[
\sigma^*(\gamma_1) = \gamma_1 - a\gamma_2 + a^2\gamma_3 - a^3\gamma_4 + a^4\lambda_4 + (2a^6 + a^3a_6)\lambda_2 + (2a^7 + 2a^4a_6)\lambda_1.
\]
\todo[inline]{there are sign errors in the above, when comparing the second entry. Happens in lower order terms}

\begin{multline}
\sigma^*(\gamma_2) = \left( \frac{2a^3x^8 + (a^6 + 2a^3a_6 + a_3)x^5 + a^4a_6x^4 + 2a^5a_6x^3 + 2(a^9 + a^3+a_0)x^2}{2yx^5} \right., \\
\left. \frac{2(x^6 + (2a^3+a_6)x^3 + (a^6 + a^3a_6))}{y} , \frac{y(x^2 + ax}{x^4} \right).
\end{multline}
 

From this we deduce that
\[
\sigma^*(\gamma_2) = \gamma_2 + a \gamma_3 + 2a^3 \lambda_3 + (a^6 + a^3a_3 + 2a_3)\lambda_0.
\]





Next we compute $\sigma^*(\gamma_3)$, which is
\begin{equation*}
\left( \frac{a^2a_6x^5 + 2a^4a_6x^3 + (2a^8 + 2a^2a_3)x^2}{2yx^5}, 0, \frac{y}{x^3} \right)
\end{equation*}


Hence we conclude that $\sigma^*(\gamma_3) = \gamma_3)$.

Lastly, we compute $\sigma^*(\gamma_4)$, which is
\begin{multline*}
\left( \frac{2ax^8 + 2a^3x^6 + (2a^4 +aa_6)x^5 + 2a^2a_6x^4 (a^3a_6 + a_3)x^3 + 2a^4a_6x^2 + (a^8 + a^2a_3)x + a^6a_6 +a_0 }{2yx^5},\right. \\ \left. \frac{x^4 +ax^3 +(a^3+a_6)x + a^4+aa_6}{y}, \frac{y}{x^4} \right)
\end{multline*}

Then the action of $\sigma^*(\gamma_4)$ is 
\[
\sigma^*(\gamma_4) = \gamma_4 + 2a\lambda_3 + 2a^3\lambda_1 + 2(a^4 + aa_0)\lambda_0.
\]

%--------
%Could be useful to describe why the action of sigma changes covers
Suppose that $v = (\omega_\sigma , \omega_\infty, f_{\sigma \infty})$ is an element of $\cechderhamhone(\cU')$.
Then
\begin{equation*}
\sigma^*(v) := (\sigma^*(\omega_\sigma), \sigma^*(\omega_\infty), \sigma^*(f_{\sigma \infty})) \in \Omega_X(U_0) \times \Omega_X(U_\infty) \times \cO_X(U_\infty \cap U_0).
\end{equation*}
Since $\sigma^*(\omega_\sigma)-\sigma^*(\omega_\infty) = \sigma^*(\omega_\sigma - \omega_\infty) = \sigma^*(df_{\sigma \infty})$ we see that $\sigma^*(v)$ is an element of $\cechderhamhone(\cU)$.


\end{comment}



\bibliography{biblio}
\bibliographystyle{amsalpha}


\end{document}
