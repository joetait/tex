% !TEX TS-program = pdflatex
% !TEX encoding = UTF-8 Unicode

% This is a simple template for a LaTeX document using the "article" class.
% See "book", "report", "letter" for other types of document.

\documentclass[11pt]{article} % use larger type; default would be 10pt

\usepackage[utf8]{inputenc} % set input encoding (not needed with XeLaTeX)

%%% Examples of Article customizations
% These packages are optional, depending whether you want the features they provide.
% See the LaTeX Companion or other references for full information.

%%% PAGE DIMENSIONS
\usepackage{geometry} % to change the page dimensions
\geometry{a4paper} % or letterpaper (US) or a5paper or....
% \geometry{landscape} % set up the page for landscape
% read geometry.pdf for detailed page layout information

%\usepackage{graphicx} % support the \includegraphics command and options
\usepackage[obeyDraft]{todonotes}

%\usepackage[parfill]{parskip} % Activate to begin paragraphs with an empty line rather than an indent

%%% PACKAGES
\usepackage[all]{xy}
\usepackage{mathtools}
%\usepackage{booktabs} % for much better looking tables
\usepackage{array} % for better arrays (eg matrices) in maths
%\usepackage{paralist} % very flexible & customisable lists (eg. enumerate/itemize, etc.)
\usepackage{verbatim} % adds environment for commenting out blocks of text & for better verbatim
\usepackage{subfig} % make it possible to include more than one captioned figure/table in a single float
%\usepackage{hyperref}
% These packages are all incorporated in the memoir class to one degree or another...

%\usepackage[activate={true,nocompatibility},final,tracking=true,kerning=true,spacing=true,factor=1100,stretch=10,shrink=10]{microtype}
%\microtypecontext{spacing=nonfrench}
% activate={true,nocompatibility} - activate protrusion and expansion
% final - enable microtype; use "draft" to disable
% tracking=true, kerning=true, spacing=true - activate these techniques
% factor=1100 - add 10% to the protrusion amount (default is 1000)
% stretch=10, shrink=10 - reduce stretchability/shrinkability (default is 20/20)

%%% HEADERS & FOOTERS
%\usepackage{fancyhdr} % This should be set AFTER setting up the page geometry
%\pagestyle{fancy} % options: empty , plain , fancy
%\renewcommand{\headrulewidth}{0pt} % customise the layout...
%\lhead{}\chead{}\rhead{}
%\lfoot{}\cfoot{\thepage}\rfoot{}

%%% SECTION TITLE APPEARANCE
%\usepackage{sectsty}
%\allsectionsfont{\sffamily\mdseries\upshape} % (See the fntguide.pdf for font help)
\usepackage{amsmath}
\usepackage{amsthm}
\usepackage{amsfonts}
%\usepackage{mathrsfs}
%\usepackage{amsopn}
\usepackage{amssymb}
%\usepackage{etex}
%\usepackage{natbib}
% (This matches ConTeXt defaults)

%%% ToC (table of contents) APPEARANCE
\usepackage[nottoc,notlof,notlot]{tocbibind} % Put the bibliography in the ToC
\usepackage[titles,subfigure]{tocloft} % Alter the style of the Table of Contents
%\renewcommand{\cftsecfont}{\rmfamily\mdseries\upshape}
%\renewcommand{\cftsecpagefont}{\rmfamily\mdseries\upshape} % No bold!
%\renewcommand{\familydefault}{\sfdefault}
%\usepackage{cabin}
%\usepackage{libertine}
%\usepackage[T1]{fontenc}

%%% END Article customizations
%%Theorems and stuff
\theoremstyle{plain}
\newtheorem{defn}{Definition}[section]
\newtheorem{thm}[defn]{Theorem}
\newtheorem{cor}[defn]{Corollary}
\newtheorem{lem}[defn]{Lemma}
\newtheorem{prop}[defn]{Proposition}
\newtheorem{ex}[defn]{Example}
\newtheorem*{unnumthm}{Theorem}
\newtheorem{defnlem}[defn]{Definition/Lemma}
\newtheorem{defnthm}[defn]{Theorem/Definition}
\theoremstyle{remark}
\newtheorem*{rem}{Remark}
\newtheorem*{note}{Note}


\newcommand{\cO}{{\cal O}}
\newcommand{\ra}{\rightarrow}
\newcommand{\NN}{{\mathbb N}}
\newcommand{\PP}{{\mathbb P}}
\newcommand{\ZZ}{{\mathbb Z}}
\newcommand{\cL}{{\mathcal L}}
\newcommand{\cA}{{\mathcal A}}
\newcommand{\cD}{{\mathcal D}}
\newcommand{\cU}{{\mathcal U}}
\newcommand{\cech}{\v{C}ech }
\newcommand{\hzero}{{H^0(X,\Omega_X)}}
\newcommand{\hzeropoly}{{H^0(X,\Omega_X^{\otimes m})}}
\newcommand{\hone}{H^1(X,\mathcal{O}_X)}
\newcommand{\cechhone}{\check{H}^1(\mathcal U,\mathcal O_X)}
\newcommand{\derhamhone}{H_{\text {dR}}^1(X/k)}
\newcommand{\cechhzero}{{\check{H}^0(X,\Omega_X)}}
\newcommand{\ubar}{\underset{\bar{}}}
\newcommand{\cechderhamhone}{\check{H}_{\text {dR}}^1(X/k)}
\newcommand{\ie}{i.e.\ }
\newcommand{\adele}{ad\`ele}


\DeclareMathOperator{\aut}{Aut}
\DeclareMathOperator{\res}{Res}
\DeclareMathOperator{\ord}{ord}
\DeclareMathOperator{\di}{div}
\DeclareMathOperator{\cha}{char}
\DeclareMathOperator{\gal}{Gal}
\DeclareMathOperator{\Tr}{Tr}
\DeclareMathOperator{\Ima}{Im}




%%% The "real" document content comes below...

\title{Group actions on the de Rham cohomology of hyperelliptic~curves}
\author{}
%\date{} % Activate to display a given date or no date (if empty),
         % otherwise the current date is printed

\begin{document}
\maketitle

\listoftodos

\todo[inline]{change references to preprint when in thesis}

\section{Serre duality}


We begin by recalling some of the details of Serre duality.
We take $X$ to be a smooth, projective, connected curve over an algebraically closed field $k$ of characteristic $p \geq 0$.
Furthermore, we let $G$ denote the automorphism group of $X$ and we let $K(X)$ denote the function field of $X$.
We then define $\Omega_{K(X)}$ to be the module of differentials of $K(X)$ over $k$.
We also let $\underline{\Omega}_{K(X)}$ and $\underline{K}(X)$ denote the constant sheaves of $\Omega_{K(X)}$ and $K(X)$ respectively.
The following lemma gives us a useful description of $H^1(X,\Omega_X)$ and $\hone$.
\begin{lem}\label{exactsequencelemma}
The following sequences are exact:
\begin{equation}\label{dualitysesfunctions}
0 \ra H^0(X,\cO_X) \ra K(X) \ra \bigoplus_{P \in X} K(X)/\cO_{X,P} \xrightarrow{\delta} \hone \ra  0;
\end{equation}
\begin{equation}\label{dualitysesdifferentials}
0 \rightarrow \hzero \ra \Omega_{K(X)} \ra \bigoplus_{P \in X}\Omega_{K(X)}/\Omega_{X,P} \xrightarrow{\delta'} H^1(X,\Omega_X) \ra 0.
\end{equation}
\end{lem}
\begin{rem}
Note that a sketch of the proof below can be found in \cite[Pg. 248]{hart}.
\end{rem}
\begin{proof}
The short exact sequence
\begin{equation}\label{serredualitysesfunctions}
0 \rightarrow \cO_X \rightarrow \underline{K}(X) \rightarrow \underline{K}(X)/\cO_X \rightarrow 0
\end{equation}
is a flasque resolution of $\cO_X$ (see \cite[Chap II, ex. 1.16 (a)]{hart}).

We view the module $K(X)/\cO_{X,P}$ as a sheaf on the singleton $\{P\}$, which has a natural embedding $i\colon \{P\} \rightarrow X$.
Hence for each $P\in X$ we have the induced sheaf $i_*\left( K(X)/\cO_{X,P} \right)$ on $X$.
If we consider the direct sum of these induced sheaves over all points $P\in X$ we have the following isomorphism
\begin{equation}\label{sheafisomorphism}
\underline{K}(X)/\cO_X\cong \bigoplus_{P\in X} i_*\left(K(X)/\cO_{X,P}\right).
\end{equation}


To explain this isomorphism we first construct a map from $\underline{K}(X)/\cO_{X,P}$ in to the product $\prod_{P \in X} i_*\left({K(X)}/\cO_{X,P}\right)$, and then show that this map has finite support.

Given $i\colon \{P\} \hookrightarrow X$ we have the following equalities
\begin{align*}
i^{-1}\left(\underline{K}(X)/\cO_X\right) & = \left(\underline{K}(X)/\cO_X\right)_P \\
& = \underline{K(X),P}/\cO_{X,P} \\
& = K(X)/\cO_{X,P}.
\end{align*}
It follows that for any $P \in X$ we can map $f \in \underline{K}(X)/\cO_X$ to $i_*i^{-1}(f) \in i_* \left( K(X)/\cO_{X,P} \right)$.
Recall that for any $f \in K(X)$ then $f$ lies in $\cO_{X,P}$ for all but a finite number of points $P \in X$.
Hence the image of $f$ in $\prod_{P \in X} i_*\left(K(X)/\cO_{X,P}\right)$ is zero in almost all factors.
From this the isomorphism in \eqref{sheafisomorphism} follows.

Replacing $\underline{K}(X)/\cO_{X,P}$ by $\bigoplus_{P\in X} i_*\left(K(X)/\cO_{X,P}\right)$ in \eqref{serredualitysesfunctions} yields
\begin{equation}\label{resolutionofox}
0 \ra \cO_X \ra \underline{K}(X) \ra \bigoplus_{P \in X}i_* \left( K(X)/\cO_{X,P} \right) \ra 0
\end{equation}
Taking cohomology then yields the exact sequence \eqref{dualitysesfunctions}.

We can also tensor \eqref{resolutionofox} with $\Omega_X$ to get
\begin{equation}\label{serredualitydifferentials}
0 \ra \Omega_X \ra \underline{\Omega}_{K(X)} \ra \bigoplus_{P \in X} i_* \left( \Omega_{K(X)}/\Omega_{X,P} \right) \ra 0.
\end{equation}
Taking cohomology of this then yields the second exact sequence, \eqref{dualitysesdifferentials}.
\end{proof}

\begin{rem}
When considering elements of $H^1(X,\Omega_X)$ as elements of the cokernel of the map $\Omega_{K(X)} \ra \bigoplus_{P \in X}\Omega_{K(X)}/\Omega_{X,P}$ above, we will denote them by $\overline{(\omega_P)}_{P \in X}$.
Similarly, when we are considering elements of $\hone$ as elements of the cokernel of the map $K(X) \ra \bigoplus_{P \in X}K(X)/\cO_{X,P}$, we will denote them by $\overline{(f_P)}_{P \in X}$.
\end{rem}

For any $P\in X$ the residue map $\res_P \colon \Omega_{K(X)} \ra k$ is defined by the following properties:
\begin{itemize}
\item $\res_P(\omega) = 0$ for all $\omega \in \Omega_{P}$;
\item $\res_P(f^ndf) = 0$ for all $f \in K(X)^*$ and all $n \neq 1$;
\item $\res_P(f^{-1}df) = \ord_P(f)$, where $\ord_P(f)$ is the order of $f$ at $P$.
\end{itemize}
See \cite[Chap III, Thm. 7.14.1]{hart} for details.



Since $\Omega_{X,P} \subseteq \ker (\res_P)$, it follows that $\res_P$ is a well defined function on the quotient $\Omega_{K(X)}/\Omega_{X,P}$.
The residue theorem \cite[Chap. III, Thm. 7.14.2]{hart} states that $\sum_{P\in X} \res_P(\omega) = 0$ for any $\omega \in \Omega_{K(X)}$, so the map 
\[
\bigoplus_{P \in X} \Omega_{K(X)}/\Omega_{X,P} \ra k, \quad (\omega_P)_{P \in X} \mapsto \sum_{P\in X} \res_P(\omega_P)
\]
vanishes on the image of $\Omega_{K(X)}$.
It follows that given some $\overline{(\omega_P)}_{P \in X} \in H^1(X,\Omega_X)$, mapped to by $(\omega_P)_{P \in X} \in \bigoplus_{P \in X} \Omega_K(X)/\Omega_{X,P}$, we can define the trace map $t$ to be:
\[
t \colon H^1\left(X, \Omega_X\right) \ra k,\quad \overline{(\omega_P)}_{P \in X}  \mapsto \sum_{P \in X} \res_P(\omega_P).
\]


We now use the trace map to define a pairing between the $k$-vector spaces $\hone$ and $\hzero$.
Since $\Omega_X$ is a $\underline{K}(X)$-module, we can define a map 
\begin{equation}\label{productmap}
\hzero \times \hone \ra H^1\left(X, \Omega_X\right), \ (\omega, \overline{(f_P)}_{P \in X}) \mapsto ( \overline {(f  \omega)_P})_{P \in X},
\end{equation}
where $(f\omega)_P$ denotes $f_p\omega_p$, the product of the germ of $\omega$ in $\Omega_{X,P}$ with the germ of $f$ in $\underline{K}(X)_{P}$.

We now combine the product map in \eqref{productmap} with the trace map $t$ to get a map 
\[
 \hzero \times \hone \ra k,\quad \left(\omega, \overline{(f_P)}_{P \in X}\right) \mapsto \left\langle \omega, \overline{(f_P)}_{P \in X} \right\rangle := t \left( \overline{(f \omega)_P} \right)_{P \in X}.
\]

\begin{thm}\label{serredualitytheorem}
Via the pairing $\langle\ ,\ \rangle$, the $k$-vector spaces $\hone$ and $\hzero$ are dual to each other.
\end{thm}
\begin{proof}
This is a specialisation of \cite[Thm. 2, Chap. II]{algebraicgroupsandclassfields}.
\end{proof}

We briefly describe the maps Serre uses to prove the above theorem.
If we fix any $\omega \in \hzero$ we produce a map $\theta(\omega)\colon \hone \ra k$, given by $\theta(\omega)(f) = \langle \omega , f\rangle$.
Similarly, if we fix any $f \in \hone$ then we get a map $\psi(f) \colon \hzero \ra k$.
The maps $\psi$ and $\theta$ are isomorphisms and are dual to each other: in particular, given a basis $\omega_1, \ldots, \omega_n$ of $\hzero$, we can find a basis $f_1, \ldots , f_n$ of $\hone$ such that $\theta(\omega_i)(f_i) = 1$ for all $1 \leq i \leq n$ and $\theta(\omega_i)(f_j) = 0$ if $i \neq j$ (and similarly for $\psi$).


\section{\cech cohomology}
We now assume that $X$ is a smooth, projective, connected hyperelliptic curve of genus $g \geq 2$.
We  fix a map $\pi \colon X \rightarrow \mathbb P_k^1$ of degree two, which is unique up to an automorphism of $\mathbb P_k^1$.
In this section we describe $\hone$ and $\hzero$ concretely for such an $X$, using \cech cohomology.

By Leray's theorem \cite[Thm 5.2.12]{liu} and Serre's affineness criterion \cite[Thm 5.2.23]{liu} we know that the first \cech cohomology group of $\cO_X$ will, if we use an affine cover, be isomorphic to $\hone$.
We define $U_a := X \backslash \pi^{-1}(a)$ for any $a \in \mathbb P_k^1$ and we let ${\cal U}$ be the affine cover of $X$ formed by $U_0$ and $U_\infty$.
Given any sheaf $\cal F$ on $X$ we have the \cech differential $\check{d}\colon {\cal F}(U_0) \times {\cal F} (U_\infty) \rightarrow {\cal F}(U_0 \cap U_\infty)$, defined by $(f_0,f_\infty) \mapsto f_0|_{U_0 \cap U_\infty} - f_\infty|_{U_0 \cap U_\infty}$.
In general we will suppress the notation denoting the restriction map.
Via this differential we have the following cochain complex
\begin{equation*}
0 \rightarrow \cO_X(U_0)\times \cO_X(U_\infty) \xrightarrow{\check{d}} \cO_X(U_0 \cap U_\infty) \rightarrow 0.
\end{equation*}
The first cohomology group of this chain is $\cechhone = \frac{\cO_X(U_0 \cap U_\infty)}{\Ima(\check{d})}$ and hence
\begin{equation}\label{cechhone}
\hone \cong \frac{\cO_X(U_0 \cap U_\infty)}{\Ima(\check{d})}  
 = \frac{\cO_X(U_0 \cap U_\infty)}{\{f_0 - f_\infty | f_i \in \cO_X(U_i) \}}.
\end{equation}

If we replace $\cO_X$ by $\Omega_X$ in the previous paragraph then everything still holds, and we can conclude that
\[
H^1(X,\Omega_X) \cong \frac{\Omega_X(U_0 \cap U_\infty)}{\Ima(\check{d})} = \frac{\Omega_X(U_0 \cap U_\infty)}{\{\omega_0 - \omega_ \infty | \omega_i \in \Omega_X(U_i)\}}.
\]

We now describe how the trace map acts on $\check{H}^1(X,\Omega_X)$.
\begin{lem}\label{tracemaplemma}
For any $\bar \omega \in \check{H}^1\left (\cU, \Omega_X\right )$, which has $\omega$ as a representative in $\Omega_X(U_0 \cap U_\infty)$,\todo{Should this be rephrased, since $\omega$ doesn't have a residue at $P \in \pi^{-1}(\infty)$} we have the following equality:
\[
t(\bar \omega) = \sum_{P \in \pi^{-1}(\infty)}\res_P(\omega).
\]
\end{lem}
\begin{proof}
We take the \cech complex of \eqref{serredualitydifferentials} over the cover $\cU$, yielding the following bicomplex
\begin{equation}\label{dualitydiagram2}
\xymatrix{\Omega_X(U_0)\times\Omega_X(U_\infty) \ar@{^{(}->}[r] \ar[d]^{d_1} & \underline{\Omega}_{K(X)} \times \underline{\Omega}_{K(X)} \ar[d]^{d_2} \ar@{->>}[r] & \bigoplus \limits_{P \in U_0} \Omega_{K(X)}/\Omega_{X,P} \times \bigoplus \limits_{P \in U_\infty} \Omega_{K(X)}/\Omega_{X,P} \ar[d]^{d_3} \\
\Omega_X(U_0 \cap U_\infty) \ar@{^{(}->}[r]  & \underline{\Omega}_{K(X)} \ar@{->>}[r] & \bigoplus \limits_{P\in U_0 \cap U_\infty} \Omega_{K(X)}/\Omega_{X,P} }
\end{equation}
We can now apply the snake lemma to this diagram, giving a long exact sequence.
In fact, this sequence is isomorphic to that found in Lemma \ref{exactsequencelemma}, but we will only exhibit the isomorphisms for the third and fourth terms, since this is all we need to examine the trace map.
The fact that $H^1(X,\Omega_X) \cong {\rm coker}(d_1)$ follows from the above discussion of \cech cohomology.
To show the isomorphism $\ker(d_3) \cong \bigoplus_{P \in X} \Omega_{K(X)}/\Omega_{X,P}$ we first recall that $d_3$ is defined by $(\omega_0, \omega_\infty) \mapsto \omega_0|_{U_0 \cap U_\infty} - \omega_\infty|_{U_0 \cap U_\infty}$.
Then the kernel of $d_3$ is formed of pairs $(\omega_0, \omega_\infty) \in \bigoplus_{P \in U_0} \Omega_{K(X)}/\Omega_{X,P} \times \bigoplus_{P \in  U_\infty} \Omega_{K(X)}/\Omega_{X,P}$ such that $\omega_0$ and $\omega_\infty$ agree on $U_0 \cap U_\infty$.
It follows that the map 
\[
\bigoplus_{P \in X} \Omega_{K(X)}/\Omega_{X,P}\ra\ker(d_3), \quad  (\omega_P)_{P \in X} \to \left( (\omega_P)_{ P \in U_0}, (\omega_P)_{P \in U_\infty}) \right)
\]
is an isomorphism.

The proof now follows from a diagram chase on \eqref{dualitydiagram2}.
 We have a cocycle in $\bar \omega \in \hone$ with representative $\omega \in \Omega_X(U_0 \cap U_\infty)$.
This then injects in to $\underline{\Omega}_{K(X)}$, and since $d_2$ is surjective we can choose an element of $\underline{\Omega}_{K(X)} \times \underline{\Omega}_{K(X)}$ mapping to $\omega$.
In particular, we could choose $(\omega,0)$.
This then maps to 
\[
\psi = ((\bar{\omega}_P)_{P\in U_0}, 0) \in \left( \bigoplus_{P \in U_0} \Omega_{K(X)}/\Omega_{X,P}\right) \times \left( \bigoplus_{P \in U_\infty} \Omega_{K(X)}/\Omega_{X,P} \right).
\]
By commutativity of the diagram $\psi \in \ker(d_3)$.
Since $\res_P(\omega)=0$ for all $P \in U_0 \cap U_\infty$, and similarly $\res_P(0)=0$ for all $P \in U_\infty$, we conclude that 
\[
t(\bar \omega) = \sum_{P \in X}\res_P(\bar \omega) = \sum_{P \in \pi^{-1}(\infty)} \res_P(\bar \omega) = \sum_{P \in \pi^{-1}(\infty)} \res_P(\omega).
\]
\end{proof}

We now recall how to compute the algebraic de Rham cohomology of $X$ via \cech cohomology.
Since $X$ is a curve any differentials of degree greater than one on $X$ are zero.
Hence the de Rham complex of $X$ is 
\begin{equation}\label{res}
0 \rightarrow \cO_X \xrightarrow{d} \Omega_X \rightarrow 0.
\end{equation}
Here $d$ denotes the differential map $f \mapsto df$, as defined in \cite[Chap II, \S 8, Pg. 172]{hart}.

We use the cover $\cal U$ and the \cech differentials defined earlier to give us the \cech bicomplex of \eqref{res}, which is
\begin{equation}\label{bicomplex}
\xymatrix{ & 0 \ar[d] & 0 \ar[d] & \\
0 \ar[r] & \cO_X(U_0) \times \cO_X(U_\infty) \ar[d] \ar[r] & \Omega_X(U_0) \times \Omega_X(U_\infty) \ar[d] \ar[r] & 0 \\
0 \ar[r] & \cO_X(U_0\cap U_\infty) \ar[d] \ar[r] & \Omega_X(U_0 \cap U_\infty) \ar[r] \ar[d] & 0 \\
& 0 & 0 &}
\end{equation}
By a generalisation of Leray's theorem \cite[Cor 12.4.7]{EGA0III} we know that the $\derhamhone$ is isomorphic to the first cohomology of the total complex of \eqref{bicomplex}.
Note that this requires ${\check H}^p(U_\sigma, \cO_X)$ and ${\check H}^p(U_\sigma, \Omega_X)$ to be zero for any $\sigma$ in the nerve of $\cU$ and any $p \geq 1$ ---
since $U_0$ and $U_\infty$ are affine, this follows from Serre's affineness criterion \cite[Thm 5.2.23]{liu}.



After computing the first cohomology group of the total complex it is clear that $\derhamhone$ is isomorphic to the space
\begin{equation}\label{derhamconditions}
\left\{(\omega_0, \omega_\infty, f_{0,\infty}) | \omega_i\in \Omega_{X/k}(U_i), f_{0,\infty}\in \cO_X(U_0 \cap U_\infty), df_{0,\infty} = \omega_0|_{U_0\cap U_\infty} - \omega_\infty|_{U_0\cap U_\infty} \right\}
\end{equation}
quotiented by the subspace
\begin{equation}\label{quotient}
\left\{  (df_0, df_\infty, f_0|_{U_0\cap U_\infty} -f_\infty|_{U_0\cap U_\infty} )|f_i \in \cO_X(U_i)\right\}.
\end{equation}

We wish to compute a $k$-basis of $\derhamhone$ in order to study its $k[G]$-module structure.
The following lemma shows that $\derhamhone$ fits in to a short exact sequence with $\hone$ and $\hzero$ which we will use to compute the $k$-basis.
\begin{prop}\label{ses}
The following is a short exact sequence
\begin{equation}\label{equationses}
0 \ra H^0(X,\Omega_X) \rightarrow \derhamhone \rightarrow H^1(X,\cO_X) \ra 0.
\end{equation}
\end{prop}
\begin{proof}
Let $T$ be the total complex of \eqref{bicomplex}.
Moreover, we let $\cO$ and $\Omega$ be the complexes formed from the first and second (non-trivial) columns of \eqref{bicomplex} respectively.
Then let $\Omega[1]$ denote the complex obtained from shifting $\Omega$ by one, i.e.~$\Omega[1]^{n+1} = \Omega^n$.
From this we obtain the following short exact sequence of complexes 
\[
\Omega[1] \hookrightarrow T \twoheadrightarrow \cO,
\]
giving rise to the following long exact sequence
\begin{equation*}
0 \ra H^0_{\text {dR}}(X/k) \ra H^0(X,\cO_X) \ra \\
\end{equation*}
\begin{equation}\label{longexactsequence}
 H^0(X,\Omega_X) \ra \derhamhone \ra \hone \ra \\
\end{equation}
\begin{equation*}
 H^1(X,\Omega_X) \ra H^2_{\text {dR}}(X/k) \ra 0.
\end{equation*}
The map $H^0(X,\cO_X) \ra \hzero$ is the map $f \mapsto df$.
Since the only globally holomorphic functions on $X$ are constant functions, it follows that this is the zero map, and hence $\hzero \ra \derhamhone$ is injective.

Since \eqref{longexactsequence} is exact, $p \colon \cechderhamhone(\cU) \ra \cechhone(\cU)$ is surjective if and only if $\alpha \colon H^1(X,\Omega_X) \ra H^2_{\text {dR}}(X/k)$ is injective.
Now $H^1(X,\Omega_X)$ is isomorphic to $k$ via the residue map \cite[Chap. III, Thm. 7.14.1]{hart}, and if we can show that this isomorphism factors through $\alpha$ it will follow that $\alpha$ is injective.
Considering the \cech cohomology constructions of $H^1(X,\Omega_X)$ and $H^2_{\text {dR}}(X/k)$, it suffices to show that the residue map is zero on $\Ima \left( d \colon \cO_X(U_0 \cap U_\infty) \ra \Omega_X(U_0 \cap U_\infty) \right)$.
This follows from \cite[Chap. III, Thm. 7.14.1 (b)]{hart}, which says that if $f \in \cO_X(U_0 \cap U_\infty)$ then $\res_{P}(df) =0$ for any $P \in _0 \cap U_\infty$.
Hence the residue isomorphism factors through $\alpha$, and $p$ is surjective.
\end{proof}
\begin{rem}
If we replace the terms in \eqref{equationses} by their \cech cohomology representatives then we can describe the maps concretely as
\[
i\colon \check{H}^0(\cU,\Omega_X) \ra \cechderhamhone(\cU), \quad \omega \mapsto (\omega, \omega, 0)
\]
and
\[
p \colon \cechderhamhone(\cU) \ra \cechhone\quad (\omega_0, \omega_\infty, f_{0 \infty}) \mapsto f_{0, \infty}.
\]  
\end{rem}

\section{Basis of $\hone$ and $\derhamhone$}

In this section we will give an explicit basis of $\hone$ and $\derhamhone$, using the results of the previous sections along with methods similar to those found in \cite{canonicalrepresentation}.\todo{remove this and mention in introduction/overview section(s)}

Since $X$ is a hyperelliptic curve, it follows that if $p \neq 2$ the extension $K(X)$ of $K(\mathbb P_k^1) = k(x)$ will be $k(x,y)$ where $y$ satisfies
\begin{equation}\label{definingequationpnot2}
y^2 = f(x)
\end{equation}
for some polynomial $f(x) \in k[x]$ which has no repeated roots and is of degree $2g+1$ or $2g+2$ \cite[Prop 7.4.24]{liu}.
Moreover, we can assume, by applying an automorphism of $\mathbb P_k^1$ if necessary, that $f(x)$ is monic.

On the other hand, if $p=2$, then the extension $K(X)$ of $k(x)$ will be $k(x,y)$, this time with $y$ satisfying the equation
\begin{equation}\label{definep=2}
y^2 - H(x)y = F(x)
\end{equation}
for some $H(x),F(x) \in k[x]$, such that $H(x)$ and $H'(x)^2F(x) + F'(x)^2$ share no roots.
We require that $\deg(H(x)) \leq g+1$, with equality if and only if $\infty$ is not a branch point, and that $\deg(F(x)) \leq 2g+2$ with $\deg(F(x)) = 2g+1$ if $\infty$ is a branch point  \cite[Prop 7.4.24]{liu}.
Finally, we can assume, again applying an automorphism of $\mathbb P_k^1$, that $H(x)$ is monic.

We now recall the details of some divisors discussed in \cite{faithfulaction}, which we will need later.
Henceforth we let $P_a$ and $P_a'$ denote the unique elements of $\pi^{-1}(a)$ for any point $a \in \mathbb P_k^1$ that is not a branch point.
If $a \in \mathbb P_k^1$ is a branch point we denote the unique point in $\pi^{-1}(a)$ by $P_a$.
We also define $D_a$ to be the divisor $\pi^*\left([a]\right)$ for any $a \in \mathbb P_k^1$, and hence
\begin{equation*}
D_a= 
\begin{cases}
 2[P_a] & \text{if $a$ is a branch point}, \\
 [P_a] + [P_a'] & \text{otherwise.}
\end{cases}
\end{equation*}

If $p \neq 2$ then the ramification points of $\pi$ are the zeroes of $f(x)$.
We suppose that 
\[
f(x) = \prod_{i=1}^{d_f} (x-a_i) = x^{d_f} + b_{d_f - 1}x^{d_f-1} \ldots + b_0,
\]
for some $a_i, b_i \in k$ and with $d_f := \deg(f) $.
In this case, the $a_i \in \mathbb A_k^1 \subset \mathbb P_k^1$ are branch points of $\pi$, and we let $P_i = P_{a_i}$ denote the corresponding ramification points.
When $d_f = 2g+1$ then $\infty \in \mathbb P_k^1$ is a branch point and we define $P_{2g+2} := P_\infty$.
Then the ramification divisor $R$ of $\pi$ is
\[
R = \sum_{i=1}^{2g+2} [P_i] .
\]

If $p=2$ then we suppose that
\begin{equation}\label{capitalh}
H(x) = \prod_{i=1}^{d_H} (x-A_i)^{n_i} = x^{d_H} + B_{{d_H}-1}x^{{d_H}-1} + \ldots + B_1x + B_0
\end{equation}
for some $A_i, B_i \in  k$, $n_i \in \mathbb N$ and with  ${d_H} := \deg(H(x)) \leq g+1$.
Then, as above, the $A_i$ are the branch points of $\pi$ and we let $P_i \in X$ be the corresponding ramification points.
Given this, we can write the ramification divisor as
\[
R = \sum_{i=1}^{d_H} 2n_i[P_i] + (g+1-{d_H})D_\infty.
\]


We also have
\begin{equation}\label{divxp=2}
\di (x)  = D_0 - D_\infty
\end{equation}
and
\begin{equation}\label{differentialdivisor}
\di(dx) = R - 2D_\infty,
\end{equation}
regardless of characteristic.

If $p \neq 2$ then 
\begin{equation}\label{pnot2divisors}
\di(y)  = R - (g+1)D_\infty,
\end{equation}
whilst if $p=2$ then the divisor of $H(x)$ is
\begin{equation}\label{divisorofH}
\di (H(x))  = R - (g+1)D_\infty. 
\end{equation}
We also compute $dy$ in terms of $x$ and $dx$ when $p=2$.
We start by noting that when we take the differential of \eqref{definep=2} we obtain
\[
dF(x) = d\left(y^2 + yH(x) \right) = d(yH(x)) = H(x)dy + ydH(x)
\]
and from this it follows that
\begin{equation}\label{divdyp=2}
dy = \frac{F(x)'-yH(x)'}{H(x)}dx.
\end{equation}

Finally, we describe the divisor of $y$ when $p=2$.
In order to do this we need to distinguish the zeroes of $F(x)$.
Suppose that $F(x)$ has $l \leq \deg(F(x))$ distinct zeroes, and let $\gamma_1, \ldots, \gamma_l \in \mathbb{A}_k^1 \subseteq \mathbb P_k^1$ be these zeroes.
Then if $\gamma_i$ is a branch point let $Q_i = (\gamma_i, 0)$ be the unique point in the pre-image $\pi^{-1}(\gamma_i)$.
If $\gamma_i$ is not a branch point then let $Q_i = (\gamma_i, 0)$ and $Q_i' = (\gamma_i, H(\gamma_i))$ be the unique points that form the pre-image $\pi^{-1}(\gamma_i)$.
Also, we denote the order of the zero of $F(x)$ at $\gamma_i \in k$ by $m_i \in \NN$.


\begin{prop}\label{divyp=2}
Suppose that $p=2$.
Then, if $\infty$ is a branch point, the divisor of $y$ is
\begin{equation*}
\di(y) = 
 {\displaystyle \sum_{i=1}^l} m_i[Q_i] -(2g+1)[P_\infty].
\end{equation*}
If $\infty$ is not a branch point, then, after possibly swapping the notations for the two points $P_\infty$ and $P_\infty'$ in $\pi^{-1}(\infty)$, we have
\begin{equation*}
 \di(y) = {\displaystyle \sum_{i=1}^l} m_i[Q_i] +(g+1-\deg(F(x)))[P_\infty] - (g+1)[P_\infty'].
\end{equation*}
\end{prop}
\begin{proof}
We first show that $\di_0(y)$, the divisor of the zeroes of $y$, is $\sum_{i=1}^l m_i [Q_i]$.

It is clear that the zeroes of $y$ can only occur in the affine part of the curve $X$ defined by \eqref{definep=2} (\ie~in $U_\infty$).
Suppose $P\in U_\infty$.
If $\left. F \right|_P \neq 0$ then it follows that $y|_P \neq 0$, since $F(x) = y (y + H(x))$.
Hence the coefficient of $\di_0(y)$ for any point in $U_\infty\backslash \{Q_1, \ldots, Q_l \}$ is zero.

Suppose that $P= Q_i = (\gamma_i, 0)$ is an unramified point in $U_\infty$.
Then $H(\gamma_i) \neq 0$ and $\left. y \right|_P = 0$, so $y + H(x)$ is a unit at $P$.
Since $y(y+H(x)) = F(x)$ we find that
\begin{equation*}
\ord_P(y) = \ord_P\left( \frac{F(x)}{y + H(x)} \right) = \ord_P(F(x)) = m_i.
\end{equation*}

We now look at when $P = Q_i = (\gamma_i, 0)$ is a ramification point.
Since $H(x)$ and $H'(x)^2F(x) + F'(x)^2$ cannot share roots it follows that $m_i = 1$.
Hence the function $\tilde F(x) := (x- \gamma_i)^{-1}F(x)$ is a unit at $P$.
We let $\tilde H(x) = (x- \gamma_i)^{-1}H(x)$.


Now 
\[
y^2 = F(x) - y H(x) = (x- \gamma_i) \left(\tilde F(x) - y \tilde H(x)\right),
\]
and hence
\[
\ord_P(y^2 ) = \ord_P(x-\gamma_i) + \ord_P(\tilde F(x) - y \tilde H(x)).
\]
Since $\ord_P(x-\gamma_i) = 2$ and $\ord_P\left(\tilde F(x) - y \tilde H(x)\right) \geq 0$ we know that $\ord_P(y) \geq 1$.
Hence $(y \tilde H(x)) \big|_P = 0$, and since $\tilde F(x)$ is a unit at $P$, we conclude that $\tilde F(x) - y \tilde H(x)$ is a unit at $P$.
Hence $\ord_P(y^2) = 2$, and so $\ord_P(y) = 1 = m_i$.
It follows that $\di_0(y) =  \sum_{i=1}^l m_i [Q_i]$.

We now consider the poles of $y$.
If $\infty$ is a branch point then $\deg(F(x)) = 2g+1$ and hence $\sum_{i=1}^l m_i = 2g+1$.
Since $y$ can only have a pole at $P_\infty$, we conclude that the degree of this pole is $2g+1$, and hence
\[
\di(y) = \sum_{i=1}^l m_i [Q_i] - (2g+1)[P_\infty].
\]

If $\infty$ is not a branch point then there are two points at which $y$ may have a pole, namely $P_\infty$ and $P_\infty'$.
We consider three cases, noting that $\ord_{P_\infty}(y + H(x)) = \ord_{P_\infty'}(y)$, since the hyperelliptic involution of $X$, denoted $\sigma$, acts on $y$ by $y \mapsto y+H(X)$ and $\sigma(P_\infty) = P_\infty'$.  


Firstly, we suppose that $\ord_{P_\infty}(y) < -(g+1)$.
Then $\ord_{P_\infty}(y) < \ord_{P_\infty}(H(x))$ and hence $ \ord_{P_\infty}(y) = \ord_{P_\infty}(y+H(x))$.
But this contradicts $\ord_{P_\infty}(y) + \ord_{P_\infty}(y+H(x)) = \ord_{P_\infty}(F(x))$, since the left hand side is less than $-2(g+1)$, which is the minimum value of the right hand side.

We now suppose that $\ord_{P_\infty} (y) = -(g+1)$. Since $y(y+H(x)) = F(x)$ it follows that $-(g+1) + \ord_{P_\infty}(y+H(x)) = \ord_{P_\infty}(F(x))$, and hence $\ord_{P_\infty'}(y) = \ord_{P_\infty}(y+H(x)) = -\deg(F(x)) + g + 1$.

Finally, if $\ord_{P_\infty}(y) > -(g+1)$, then since $\ord_{P_\infty}(H(x)) = -(g+1)$ it follows that $\ord_{P_\infty'}(y) = \ord_{P_\infty} (y+H(x)) = -(g+1)$.
Then, from a computation similar to that in the previous paragraph we see that $\ord_{P_\infty}(y) = -\deg(F(x)) + g +1$, completing the proof.
\end{proof}


\begin{prop}\label{basish1}
 Via the isomorphism \eqref{cechhone} the residue classes of $\frac{y}{x}, \ldots , \frac{y}{x^g} \in K(X)$, restricted to $U_0 \cap U_\infty$, form a basis of $H^1(X,\cO_X)$.
\end{prop}
\begin{proof}
We start by considering the case $p \neq 2$ and first check that the functions $\frac{y}{x}, \ldots, \frac{y}{x^g}$ are indeed regular on $U_0 \cap U_\infty$ (as required by \eqref{cechhone}) by computing their divisors.
From \eqref{divxp=2} and \eqref{pnot2divisors} we see that
\begin{align}\label{divisorofyoverx}
\di \left( \frac{y}{x^i} \right) & = \di (y) - \di ( x^i) \nonumber \\
& = R - (g+1)D_\infty - iD_0 + iD_\infty \nonumber \\
& = R - iD_0 - (g+1 - i)D_\infty.
\end{align}
Since $R$ is a positive divisor this is non-negative on $U_0 \cap U_\infty$ if $i\in \{0, \ldots, g-1\}$.


Recall that the differentials $\omega_0 = y^{-1}dx, \ldots, \omega_{g-1} = x^{g-1}y^{-1}dx$ form a basis of $\hzero$ (see, for example, \cite[Chap 7, Prop. 4.26]{liu}).
By Lemma \ref{tracemaplemma} we know that $\langle x^iy^{-1}dx, yx^{-j} \rangle = \sum_{P \in \pi^{-1}(\infty)}\res_P(x^{i-j}dx)$.
It follows immediately from \cite[Chap. III, Thm. 7.14.1(b)]{hart} that $\sum_{P \in \pi^{-1}(\infty)}\res_P(x^{i-j}dx) = -2$ if $i-j=-1$ and is zero otherwise.
It then follows from Theorem \ref{serredualitytheorem} that the collection of elements $\{ yx^{-j}|_{U_0\cap U_\infty}\}_{ 1 \leq j \leq g}$ form a basis of $\hone$.



We now suppose that $p=2$, and again start by checking that for $i \in \{1, \ldots , g\}$ the function $yx^{-i}$ is regular on $U_0 \cap U_\infty$.
This follows once we compute the divisor of $yx^{-i}$, which is
\begin{equation*}
\di \left( \frac{y}{x^i} \right)  =  
{\displaystyle \sum_{i=1}^l} m_i[Q_i] -iD_0 -(2g+1 - 2i)[P_\infty]
\end{equation*}
if $\infty$ is a branch point and
\begin{equation*}
\di \left( \frac{y}{x^i} \right)  =  
{\displaystyle \sum_{i=1}^l} m_i[Q_i] - iD_0 +(g+1-\deg(F(x)) + i)[P_\infty] - (g+1-i)[P_\infty']
\end{equation*}
otherwise.
These equalities follow from Proposition \ref{divyp=2} and \eqref{divxp=2}.
The divisors are clearly positive on $U_0 \cap U_\infty$.

Next we recall from \cite[Chap 7, Prop. 4.26]{liu} that if $p=2$ a basis of $\hzero$ is given by $\frac{1}{H(x)}dx, \ldots, \frac{x^{g-1}}{H(x)}dx$.
We then deduce from Lemma \ref{tracemaplemma} that
\[
\left \langle \frac{x^i}{H(x)}dx, \frac{y}{x^j} \right \rangle = \res_{P_\infty} \left( \frac{yx^{i-j}}{H(x)}dx \right) + \res_{P_\infty'}\left( \frac{yx^{i-j}}{H(x)} dx \right).
\]
Then recall that in characteristic two the hyperelliptic involution $\sigma \colon X \ra X$ is given by $(x,y) \mapsto (x, y + H(x))$, and that $\res_P(\sigma^*(\omega)) = \res_{\sigma(P)}(\omega)$ for any $P \in X$ and $\omega\in \hzero$.
Then it follows that
\begin{align*}
\sum_{P \in \pi^{-1}(\infty)} \res_P \left( \frac{yx^{i-j}}{H(x)}dx \right) & = \res_{P_\infty} \left( \frac{yx^{i-j}}{H(x)} dx \right) + \res_{P_\infty'}\left( \frac{yx^{i-j}}{H(x)} dx\right) \\
& = \res_{P_ \infty} \left( \frac{yx^{i-j}}{H(x)}dx \right) + \res_{P_ \infty} \left( \frac{(y+H(x))x^{i-j}}{H(x)}dx \right) \\
& = \res_{P_\infty}(x^{i-j}dx),
\end{align*}
since we are assuming that $\cha(k) = 2$.
As in the previous case, it then follows from the definition of $\res_P$ that $\res_{P_\infty}(x^{i-j}dx) = -1$ if $i-j = -1$ and is zero otherwise.



If $P_\infty$ is a branch point then we start by computing the divisor of $ \frac{y}{x^j} \cdot \frac{x^i}{H(x)}dx$, using \eqref{divxp=2}, \eqref{differentialdivisor}, \eqref{divisorofH} and Proposition \ref{divyp=2}:
\begin{align*}
\di\left( \frac{yx^{i-j}}{H(x)}dx \right) & = \di(y) + \di(x^{i-j}) + \di( dx) - \di(H(x)) \\
& = \sum_{i=1}^l m_i[Q_i] - (2g+ 1 )[P_\infty] + (i-j)D_0 - (i-j)D_\infty + R - 2D_\infty \\
& \qquad - R + (g+1)D_\infty\\
& = \sum_{i=1}^l m_i[Q_i] + (2j-3-2i)[P_\infty] + (i-j)D_0.
\end{align*}
We see that there is a pole of order one at $P_\infty$ if $2j - 3 - 2i = -1$, or equivalently if $j = i+1$.
Hence $\left\langle \frac{x^i}{H(x)}dx, \frac{y}{x^j} \right\rangle \neq 0$ in this case.\todo{what is the residue?}

We also need to check that if $j \neq i+1$ then $\left \langle \frac{x^i}{H(x)}dx, \frac{y}{x^j} \right \rangle = 0$.
Indeed, if $j-i \geq 2$ then clearly $\frac{yx^{i-j}}{H(x)}dx$ does not have a pole at $P_\infty$.
On the other hand, if $j-i \leq 0$ then the differential $\frac{yx^{i-j}}{H(x)}dx$ is regular on $U_\infty$, and hence the residue on this set is zero.
Since $\{P_\infty\} = X \backslash U_\infty $ it follows from the residue theorem the residue of $\frac{yx^{i-j}}{H(x)}dx$ at $P_\infty$ is also zero.
\end{proof}

%Mittag-Leffler style corollary

\begin{cor}
For each $P \in X$ we fix $f_P \in K(X)/\cO_{X,P}$, such that $f_P = 0$ for almost all $P \in X$.
Then there exist unique $\alpha_1, \ldots, \alpha_g \in k$ such that after replacing $f_P$ by $f_P - \left( \frac{\alpha_1 x}{y} + \ldots + \frac{\alpha_g x^g}{y}\right)$ for $P \in \pi^{-1}(\infty)$, we can find an $f \in K(X)$ such that the laurent tail of $f$ at $P$ in $X$ is $f_P$.
\end{cor}
\begin{proof}
Since $f_P = 0$ for almost all $P \in X$ then $(f_P)_{P \in X} \in \bigoplus_{P \in X} K(X)/\cO_{X,P}$.
From Lemma \ref{exactsequencelemma} we have the following exact sequence
\begin{equation*}
0 \ra H^0(X,\cO_X) \ra K(X) \ra \bigoplus_{P \in X} K(X)/\cO_{X,P} \xrightarrow{\delta} \hone \ra 0.
\end{equation*}
Since $\frac{x}{y}, \ldots, \frac{x^g}{y}$ form a basis of $\hone$, it follows that there exist unique $\alpha_1, \ldots, \alpha_g$ such that
\[
\delta\left( (f_P)_{P \in X} \right) - \left( \frac{\alpha_1 x}{y} + \ldots + \frac{\alpha_g x^g}{y}\right) = 0.
\]
We can derive \eqref{dualitysesfunctions} by applying the snake lemma to the \cech complex of \eqref{serredualitysesfunctions} over $\cU$, which is
\begin{equation}\label{dualitydiagram2}
\xymatrix{\cO_X(U_0)\times\cO_X(U_\infty) \ar@{^{(}->}[r] \ar[d]^{d_1} & \underline{K}(X) \times \underline{K}(X) \ar[d]^{d_2} \ar@{->>}[r] & \bigoplus \limits_{P \in U_0} K(X)/\cO_{X,P} \times \bigoplus \limits_{P \in U_\infty} K(X)/\cO_{X,P} \ar[d]^{d_3} \\
\cO_X(U_0 \cap U_\infty) \ar@{^{(}->}[r]  & \underline{K}(X) \ar@{->>}[r] & \bigoplus \limits_{P\in U_0 \cap U_\infty} K(X)/\cO_{X,P} }
\end{equation}
Analagously to \eqref{dualitydiagram2}, the kernel of $d_3$ is $\bigoplus_{P \in X}K(X)/\cO_{X,P}$ and the cokernel of $d_1$ is $\hone$.
Via a diagram chase on $- \left( \frac{\alpha_1 x}{y} + \ldots + \frac{\alpha_g x^g}{y}\right) \in \hone$ we can find a corresponding element 
\begin{equation}\label{immediate}
\left(0, \left(- \left( \frac{\alpha_1 x}{y} + \ldots + \frac{\alpha_g x^g}{y}\right) \right)_P\right) \in \bigoplus_{P \in U_0}K(X)/\cO_{X,P} \times \bigoplus_{P \in U_\infty}K(X)/\cO_{X,P}.
\end{equation}
Since $\alpha_ix^i/y$ is regular on $U_\infty \cap U_0$, \eqref{immediate} is equivalent to $\left(0, (g_P)_{P \in U_\infty}\right)$, where
\[
g_P = \begin{cases}
    - \left( \frac{\alpha_1 x}{y} + \ldots + \frac{\alpha_g x^g}{y}\right) & \quad \text{if}\ P \in \pi^{-1}(\infty) \\
    0 & \quad \text{else.}
\end{cases}
\]
Clearly $(0, g_P) \in \ker (d_3) = \bigoplus_{P \in X} K(X)/\cO_{X,P}$, and $\delta((f_P)_{P \in X} - (g_P)_{P \in X}) = 0$, by choice of $(g_P)_{P \in X}$.
By the exactness of \eqref{dualitysesfunctions} it follows that there exists an $f \in K(X)$ which has laurent tail $f_P - g_P$ at each $P \in X$.
\end{proof}


In order to state a basis of $\derhamhone$, as well as to shorten the proof of the following theorem, we fix some notation. 
We suppose that $1 \leq i \leq g$.
Then when $p\neq 2$ we define
\[
s_i(x) := xf'(x) - 2if(x) \in k[x]
\]
and when $p = 2$ we define
\begin{equation}\label{capitals}
S_i(x,y) := xF'(x) + y(xH'(x) + iH(x))\in k[x]\oplus yk[x] \subseteq k(x,y).
\end{equation}

We now decompose these polynomials into two parts, which will be used in the sequel.
Firstly, we write $s_i(x)$ as $s_i(x) = \phi_i(x) + \psi_i(x)$, where $\psi_i(x), \phi_i(x) \in k[x]$ are the unique polynomials such that the degree of $\psi_i (x)$ is at most $g+1$ and $x^{g+2}$ divides $\phi_i(x)$.

For the second case we define $A_{j,i} \in k$ for $1 \leq j \leq 2g+2$, and $B_{k,i} \in k$ for $0\leq k \leq g+1$ by the equation
\[
S_i(x,y) = A_{2g+2,i}x^{2g+2} + \ldots + A_{1,i} x + y(B_{g+1,i} x^{g+1} + \ldots + B_{1,i} x + B_{0,i}).
\]
Note that many of these coefficients may be zero.
In particular we remark that the $x^i$ term of $xH'(x) + iH(x)$ is always zero, since $B_{i,i}x^i = x \cdot B_ix^{i-1} + B_i x^i = 2B_{i,i}x^i = 0$.


We now define the following polynomials:
\begin{equation}\label{Split}
\begin{split}
\Phi_i^x(x) & =  A_{2g+2, i}x^{2g+2} + \ldots + A_{i+1, i}x^{i+1} \\
\Psi_i^x(x) & =  A_{i,i}x^i + \ldots + A_{1,i}x \\
\Phi_i^y(x) & =  B_{g,i}x^g + \ldots B_{i+1,i}x^{i+1} \\
\Psi_i^y(x) & =  B_{i-1,i}x^{i-1} + \ldots + B_{1,i}x + B_{0,i}.
\end{split}
\end{equation}
Finally, we define $\Phi_i(x,y) = \Phi_i^x(x) + y \Phi^y_i(x)$ and $\Psi_i(x,y) = \Psi_i^x(x) + y \Psi_i^y(x)$, so that $S_i(x,y) = \Phi_i(x,y) + \Psi_i(x,y)$.

Viewing $\derhamhone$ as a quotient of \eqref{derhamconditions}, we now give a $k$-vector space basis of $\derhamhone$.
\begin{thm}\label{basis}

If $p \neq 2$ then the residue classes of 
\begin{equation}\label{one}
 \left( \left( \frac{\psi_i(x)}{2yx^{i+1}}\right) dx, \left(\frac{-\phi_i(x)}{2yx^{i+1}}\right) dx, x^{-i}y\right), i=1, \ldots ,g,
\end{equation}
along with the residue classes of 
\begin{equation}\label{two}
 \left( \frac{x^{i}}{y} dx , \frac{x^{i}}{y} dx, 0 \right), i = 0,\ldots ,g-1,
\end{equation}
form a basis of $\derhamhone$.

On the other hand, if $p=2$ then the residue classes of the elements 
\begin{equation}\label{three}
\left( \left(\frac{\Psi_i(x,y)}{x^{i+1}H(x)}\right) dx, \left( \frac{\Phi_i(x,y)}{x^{i+1}H(x)} \right) dx, x^{-i}y \right), i =1, \ldots , g,
\end{equation}
together with the residue classes of 
\begin{equation}\label{four}
\left( \frac{x^{i}}{H(x)} dx, \frac{x^{i}}{H(x)} dx, 0 \right), i=0, \ldots, g-1,
\end{equation}
form a basis of $\derhamhone$.
\end{thm}

Before proving this theorem we use it to prove the following corollary.

\begin{cor}
The action of $G$ on $\derhamhone$ is faithful unless $G$ contains a hyperelliptic involution and $p=2$, in which case the action of the hyperelliptic involution is trivial.
\end{cor}

\begin{proof}
Recall from Proposition \ref{ses} that $H^0(X,\Omega_X)$ injects into $\derhamhone$.
Then if $p \neq 2$ or $G$ does not contain a hyperelliptic involution it follows from \cite[Thm. 4.2]{faithfulaction} that $G$ acts faithfully on $H^0(X,\Omega_X)$, and hence $G$ acts faithfully on $\derhamhone$.

We now suppose that $p=2$ and that $G$ contains a hyperelliptic involution, which we denote by $\sigma$.
By the same theorem from \cite{faithfulaction} we know that $\sigma$ acts trivially on $\hzero$.

Since $\hzero$ is dual to $\hone$ then $\sigma$ also acts trivially on $\hone$.
We can study exactly why this is from the view of \cech cohomology, and this will also help to determine the action of $\sigma$ on $\derhamhone$.
If we fix a natural number $i\in \{1, \ldots ,g\}$ then $\sigma$ maps $\frac{y}{x^i}$ to $\frac{y}{x^i} + \frac{H(x)}{x^i}$. 
Now we can split $\frac{H(x)}{x^i}$ as follows, 
\begin{equation*}
\frac{H(x)}{x^i} =  \frac{B_{i-1}x^{i-1} + \ldots + B_1x + B_0}{x^i} - \left( - \frac{x^d + B_{d-1}x^{d-1} + \ldots + B_ix^i}{x^i} \right),
\end{equation*}
where $B_j$ are as in \eqref{capitalh}.
Since this is clearly the difference of an element of $\cO_X(U_0)$ and an element of $\cO_X(U_\infty)$ we see that $\frac{H(x)}{x^i}$ is zero in $\hone$.
We let 
\[
H_{1,i}(x) = B_{i-1}x^{i-1} + \ldots + B_1x + B_0 \quad \text{ and } \quad H_{2,i}(x) = -( x^d + B_{d-1}x^{d-1} + \ldots + B_ix^i).
\]

We now consider the action of $\sigma$ on the entries in \eqref{three}.
Firstly we see that
\begin{align*}
\sigma \left( \frac{-\Psi_i(x,y)}{x^{i+1}H(x)} dx\right) & = \frac{-\sigma(\Psi_i(x,y))}{x^{i+1} H(x)} dx \\
& = \frac{-\Psi_i(x,y)}{x^{i+1}H(x)}dx + \frac{H(x)(xH_{1,i}'(x) + iH_{1,i}(x))}{x^{i+1}H(x)}dx\\
& = \frac{-\Psi_i(x,y)}{x^{i+1}H(x)}dx + \frac{xH_{1,i}'(x) + iH_{1,i}(x)}{x^{i+1}}dx \\
& = \frac{-\Psi_i(x,y)}{x^{i+1}H(x)}dx +  \frac{H_{1,i}'(x)}{x^i}dx + \frac{iH_{1,i}(x)}{x^{i+1}}dx \\
& = \frac{-\Psi_i(x,y)}{x^{i+1}H(x)}dx +  \frac{1}{x^i}d\left( H_{1,i}(x) \right) + H_{1,i}(x) d \left( \frac{1}{x^i} \right) \\
& = \frac{-\Psi_i(x,y)}{x^{i+1}H(x)}dx + d\left( \frac{H_{1,i}(x)}{x^i} \right),
\end{align*}
where the second equality follows from \eqref{capitals} and the fact that $\sigma(y) = y + H(x)$.

Similarly we can derive
\begin{equation*}
\sigma \left( \frac{\Phi_i(x,y)}{x^{i+1}H(x)} dx \right)  = \frac{\Phi_i(x,y)}{x^{i+1}H(x)} dx + d \left( \frac{H_{2,i}(x)}{x^i} \right).
\end{equation*}
Lastly, it is clear that $\sigma(x^{-i}y) = x^{-i}(y+H(x))$.


We can now describe exactly how sigma acts on the elements of \eqref{three} using $H_{1,i}(x)$ and $H_{2,i}(x)$:
\begin{multline*}
\sigma \left( \left( \left(\frac{-\Psi_i(x,y)}{x^{i+1}H(x)}\right) dx, \left( \frac{\Phi_i(x,y)}{x^{i+1}H(x)} \right) dx, x^{-i}y \right)\right) = \\
 \left( \left(\frac{-\Psi_i(x,y)}{x^{i+1}H(x)} \right) dx + d\left(\frac{H_{1,i}(x)}{x^i}\right),  \left( \frac{\Phi_i(x,y)}{x^{i+1}H(x)} \right) dx+ d\left(\frac{H_{2,i}(x)}{x^i} \right), \frac{y+H(x)}{x^i} \right).
\end{multline*}
So the action of $\sigma$ on the basis elements in \eqref{three} amounts to adding 
\[
\left( d\left(\frac{H(x)_{1,i}}{x^i}\right), d\left(\frac{H(x)_{2,i}}{x^i}\right), \frac{H(x)}{x^i} \right),
\]
which clearly satisfies the conditions of \eqref{quotient} and hence is zero.
So the action of the involution $\sigma$ on $\derhamhone$ is trivial and hence the action of the group $G$ is not faithful.
\end{proof}

\begin{rem}
We briefly study the action of $\sigma$ on the elements \eqref{one} (when $p\neq 2$).
When $p \neq 2$ then $\sigma$ acts by $(x,y) \mapsto (x,-y)$.
If we let
\[
\gamma_i = \left( \left( \frac{\psi_i(x)}{2yx^{i+1}}\right) dx, \left(\frac{-\phi_i(x)}{2yx^{i+1}}\right) dx, x^{-i}y\right)
\]
then 
\begin{equation*}
\sigma(\gamma_i) = -\gamma_i.
\end{equation*}
Similarly, if 
\[
\lambda_i = \left( \frac{x^i}{y}dx, \frac{x^i}{y}dx, 0 \right)
\]
then 
\[
\sigma(\lambda_i) = - \lambda_i.
\]
Hence $\sigma$ acts by multiplication with $-1$ on $\derhamhone$.
\end{rem}


We now prove Theorem \ref{basis}.

\begin{proof}
We make use of the fact that the short exact sequence in Lemma \ref{ses} splits as a sequence of vector spaces over $k$, and that we know bases of the outer two terms.

It is clear that the elements in \eqref{two} and \eqref{four} are elements of the space \eqref{derhamconditions}. 
In fact, it follows from \cite[Thm 6.1]{faithfulaction} that they are the image of a basis of $H^0(X,\Omega_X)$ in $\derhamhone$.

Moreover, it is obvious that if the elements in \eqref{one} and \eqref{three} are well defined elements of the space \eqref{derhamconditions} then they will map to the basis of $\hone$ given in Lemma \ref{basish1}.
So we need only show that the terms in \eqref{one} and \eqref{three} satisfy the conditions stated in \eqref{derhamconditions}.
For the rest of the proof we fix $i \in \{1, \ldots ,g\}$.


We start with the case $p\neq 2$, and observe that
\begin{align*}
\left(  \frac{\psi_i(x)}{2yx^{i+1}}  - \frac{-\phi_i(x)}{2yx^{i+1}} \right) dx & =  \frac{s_i(x)}{2yx^{i+1}} dx \\
& =  \frac{1}{2yx^i} \left( f(x)' - \frac{2if(x)}{x} \right) dx \\
& =  \frac{x^i}{2y} \left( \frac{f(x)'}{x^{2i}}dx -\frac{2if(x)}{x^{2i+1}} dx \right) \\
& =  \frac{x^i}{2y} \left( f(x)d\left(\frac{1}{x^{2i}}\right) + \frac{1}{x^{2i}}df(x) \right) \\
& =  \frac{x^i}{2y}d(f(x)x^{-2i}) \\
& =  \frac{x^i}{2y} d\left(\left(yx^{-i}\right)^2\right) \\
& =  d(yx^{-i}),
\end{align*}
with the penultimate line following from the defining equation \eqref{definingequationpnot2}.
This shows that the elements in \eqref{one} satisfy $df_{0, \infty} = \omega_0 - \omega_\infty$, one of the conditions of \eqref{derhamconditions}.
Since we saw in the proof of Lemma \ref{basish1} that $\frac{y}{x^i}$ is regular on $U_0\cap U_\infty$ it only remains to show that $\frac{\phi_i(x)}{2yx^{i+1}}dx$ and $\frac{-\psi_i(x)}{2yx^{i+1}}dx$ are regular on $U_\infty$ and $U_0$ respectively.


In order to do this we define $\alpha_{j,i} \in k$ for $0\leq j \leq 2g+2$ to satisfy the equation
\[
s_i(x) = \alpha_{2g+2,i}x^{2g+2} + \ldots + \alpha_{0,i},
\]
so that
\begin{align*}
\phi_i(x) = \alpha_{2g+2,i}x^{2g+2} + \ldots + \alpha_{g+2,i}x^{g+2} \\
\intertext{and }
\psi_i(x) = \alpha_{g+1,i}x^{g+1} + \ldots + \alpha_{0,i}.
\end{align*}
Note that it is possible for any of $\alpha_{j,i}$ to be zero. In fact, it is possible for either $\phi_i(x)$ or $\psi_i(x)$ to be zero.
Whenever they are non-zero we denote their degrees as polynomials in $x$ by $d_\phi$ and $d_\psi$ respectively. From the definition of $\phi_i(x)$ and $\psi_i(x)$ we know that $0 \leq d_\psi \leq g+1$ and $g+1 < d_\phi \leq 2g+2$.


We now show that $\frac{-\phi_i(x)}{2yx^{i+1}}dx$ and $\frac{\psi_i(x)}{2yx^{i+1}}dx$ are regular on $U_\infty$ and $U_0$ respectively.
We may assume that $\phi_i(x)$ and $\psi_i(x)$ are non-zero, since the zero function is regular everywhere.


The divisor of $\frac{-\phi_i(x)}{2yx^{i+1}}dx$ is
\begin{align*}
\di\left( \frac{-\phi_i(x)}{2yx^{i+1}}dx \right) & =  \di(\phi_i(x)) -\di(y) - \di(x^{i+1}) + \di (dx) \\
& =  \di(\phi_i(x)) - ( R - (g+1)D_\infty) - ((i+1)D_0 - (i+1)D_\infty) \\
& \qquad + (R - 2D_\infty) \\
& =  \left( \di_0\left( \frac{\phi_i(x)}{x^{g+2}}\right) + (g+2)D_0 - d_\phi D_\infty\right) - (i+1)D_0 + (g+i)D_\infty \\
& \geq  \di_0\left( \frac{\phi_i(x)}{x^{g+2}}\right) + (g+2)D_0 - (2g+2)D_\infty - (i+1)D_0 + (g+i)D_\infty \\
& =  \di_0\left( \frac{\phi_i(x)}{x^{g+2}} \right) + (i-g-2)D_\infty + (g-i+1)D_0,
\end{align*}
where the second equality makes use of \eqref{divxp=2} and \eqref{pnot2divisors}.
Since $i \leq g$ the differential $\frac{-\phi_i(x)}{2yx^{i+1}}dx$ is regular on $U_\infty = X\backslash \pi^{-1}(\infty)$.

Similarly the divisor of $\frac{\psi_i(x)}{2yx^{i+1}}dx$ is

\begin{align*}
\di \left( \frac{\psi_i(x)}{2yx^{i+1}}dx\right) & =  \di(\psi_i(x)) - \di(y) - \di(x^{i+1}) + \di (dx) \\
& =  \di (\psi_i(x) ) -(R - (g+1)D_\infty) - ((i+1)D_0 - (i+1)D_\infty) \\ 
& \qquad + (R -2D_\infty) \\
& =  \di(\psi_i(x)) + (g+i)D_\infty -(i+1)D_0 \\
& =  (\di_0(\psi_i(x)) -d_\psi D_\infty) + (g+i)D_\infty -(i+1)D_0 \\
& \geq \left( \di_0(\psi_i(x)) - (g+1)D_\infty \right) + (g+i)D_\infty -(i+1)D_0 \\
& =  \di_0(\psi_i(x)) + (i-1)D_\infty - (i+1)D_0.
\end{align*}
Again, the second equality uses \eqref{divxp=2} and \eqref{pnot2divisors}, and since $i\geq 1$ we conclude that $\frac{\psi_i(x)}{2yx^{i+1}}dx$ is regular on $U_0 = X \backslash \pi^{-1}(0)$, completing the $p\neq 2$ case.


We now suppose that $p=2$.
We remind the reader that this allows us to change signs between positive and negative as we wish.
We see that
\begin{align*}
\left( \left( \frac{ \Psi_i(x,y)}{x^{i+1}H} \right) + \left( \frac{\Phi_i(x,y)}{x^{i+1}H} \right) \right) dx & =  \frac{S_i(x,y)}{x^{i+1}H(x)}dx \\
& =  \left( \frac{F(x)'}{x^iH(x)} + \frac{yH(x)'}{x^iH(x)} + \frac{iy}{x^{i+1}} \right) dx \\
& =  \frac{1}{x^i}\left( \frac{F(x)' + yH(x)'}{H(x)} \right) dx + \frac{iy}{x^{i+1}}dx \\
& =  x^{-i}dy + yd \left( x^{-i}\right) \\
& =  d\left( yx^{-i}\right),
\end{align*}
with the fourth equality following from \eqref{divdyp=2}.
We have also already seen in the proof of Lemma \ref{basish1} that $\frac{y}{x^i}$ is regular on $U_0 \cap U_\infty$.
So in order to prove that for $i\in \{1, \ldots, g\}$ the elements of \eqref{three} are satisfy the conditions of \eqref{derhamconditions} it only remains to show that the differentials $\frac{\Phi_i(x,y)}{x^{i+1}H(x)}dx$ and $\frac{\Psi_i(x,y)}{x^{i+1}H(x)}dx$ are regular on $U_\infty$ and $U_0$ respectively.
We denote the degrees of the polynomials defined in \eqref{Split} by $d_{\Phi}^x, d_{\Psi}^x, d_{\Phi}^y$ and $d_{\Psi}^y$.


By \eqref{Split} $\Phi_i(x,y) = \Phi_i^x(x) + y\Phi_i^y(x)$ and $\Psi_i (x,y)= \Psi_i^x(x) + y\Psi_i^y(x)$, and we will use these splittings to show that $\frac{ \Phi_i(x,y) }{x^{i+1}H(x)}dx$ and $\frac{\Psi_i(x,y) }{x^{i+1}H(x)}dx$ are regular on $U_\infty$ and $U_0$ respectively.\todo{note that some stuff was removed before this para}

We start by computing the divisor of $\frac{1}{x^{i+1}H(x)}dx$, since it is a common component to all the differentials we need to look at.
This yields
\begin{align*}
\di \left( \frac{1}{x^{i+1}H(x)}dx \right) & = \di(dx) - \di (x^{i+1}) - \di (H(x)) \nonumber \\
& = (R-2D_\infty) - ((i+1)D_0 - (i+1)D_\infty) - (R - (g+1)D_\infty) \nonumber \\
& = (g+i)D_\infty - (i+1)D_0,
\end{align*}
using \eqref{differentialdivisor}, \eqref{divisorofH} and \eqref{divxp=2}.
We now use this along with Proposition \ref{divyp=2} and the polynomials \eqref{Split} to complete the proof.

We begin by computing the divisors associated to $\Phi_i(x,y)$.
Firstly,
\begin{align*}
\di \left( \frac{\Phi_i^x(x) }{x^{i+1} H(x)}dx \right)  = &  \di(\Phi_i^x(x)) -(i+1)D_0 + (g+i)D_\infty\\
 = & \left( \di_0(\Phi_i^x(x)) -d_\Phi^xD_\infty\right) -(i+1)D_0 + (g+i)D_\infty\\
 \geq & \di_0(\Phi_i^x(x)) - (2g+2)D_\infty - (i+1)D_0 + (g+i)D_\infty \\
 = &  \di_0(\Phi_i^x(x)) - (i+1)D_0 + (i-2-g)D_\infty \\
 =  & \di_0 \left( \frac{\Phi_i^x(x)}{x^{i+1}} \right) + (i-g-2)D_\infty.
\end{align*}
From this we see that the differential $\frac{\Phi_i^x(x)}{x^{i+1}H(x)}dx$ is clearly regular on $U_\infty = X \backslash \pi^{-1}(\infty)$.

We now compute the divisor of the other half of $\frac{\Phi_i(x,y)}{x^{i+1}H(x)}dx$, namely
\begin{align*}
\di\left(\frac{y\Phi_i^y(x) dx}{x^{i+1}H(x)} \right)  = & \di(y) + \di(\Phi_i^y(x)) -(i+1)D_0 + (g+i)D_\infty\\
 = & \di(y) + \di_0(\Phi_i^y(x)) - d_\Phi^yD_\infty -(i+1)D_0 + (g+i)D_\infty \\
 \geq & \di(y) + \di_0(\Phi_i^y(x)) - (g+1)D_\infty - (i+1)D_0 + (g+i)D_\infty \\
 = & \di(y) + \di_0\left(\frac{\Phi_i^y(x)}{x^{i+1}} \right) + (i-1)D_\infty.
\end{align*}
From Proposition \ref{divyp=2} we see that $y$ only has poles at points in $\pi^{-1}(\infty)$, and hence this completes the proof that $\frac{\Phi_i(x,y) }{x^{i+1}H(x)}dx$ is regular on $U_\infty = X \backslash \pi^{-1}(\infty)$.

Now we complete the same computations on $\Psi_i(x,y)$, starting with $\Psi_i^x(x)$:
\begin{align*}
\di\left( \frac{\Psi_i^x(x) }{x^{i+1}H(x)}dx \right)  & =   \di(\Psi_i^x(x))  - (i+1)D_0 + (g+i)D_\infty \\
& = (\di_0(\Psi_i^x(x)) -d_\Psi^xD_\infty) - (i+1)D_0 + (g+i)D_\infty \\
 & \geq   \di_0(\Psi_i^x(x) ) - iD_\infty - (i+1)D_0 + (g+i)D_\infty \\
 & =   \di_0(\Psi_i^x(x)) - (i+1)D_0 + gD_\infty,
\end{align*}
and it is clear that the divisor is positive on $U_0 = X \backslash \pi^{-1}(0)$.

For the other half of the differential we need to consider separate cases.
If we assume that $\infty$ is branch point then  using Proposition \ref{divyp=2} we see that
\begin{align*}
\di\left(\frac{y\Psi_i^y(x) }{x^{i+1}H(x)}dx \right)  =  & \di_0(y) - (2g+1)[P_\infty] + \di(\Psi_i^y(x)) - (i+1)D_0 + (g+i)[P_\infty] \\
 =  & \di_0(y) + \di(\Psi_i^y(x)) -(i+1)D_0 + (2i -1)[P_\infty] \\
 = &  \di_0(y) + \di_0(\Psi_i^y(x)) - d_\Psi^y[P_\infty] - (i+1)D_0 + (2i-1)[P_\infty] \\
 \geq &  \di_0(y) + \di_0(\Psi_i^y(x)) -(i-1)[P_\infty] -(i+1)D_0 + (2i-1)[P_\infty] \\
 =   &\di_0(y) + \di_0(\Psi_i^y(x)) -(i+1)D_0 + [P_\infty],
\end{align*}
which is clearly positive on $U_0$.
On the other hand, if $\infty$ is not a branch point then we have
\begin{align*}
\di\left(\frac{y\Psi_i^y(x) }{x^{i+1}H(x)}dx \right)  =  & \di(y) + \di(\Psi_i^y(x)) - (i+1)D_0 + (g+i)D_\infty \\
= & \di(y) + \di_0(\Psi_i^y(x)) - (i+1)D_0 + (g+i - d_\Psi^y)D_\infty \\
\geq & \di(y) + \di_0(\Psi_i^y(x)) - (i+1)D_0 + (g+1)D_\infty. \\
\end{align*}
Since we know from Proposition \ref{divyp=2} that $y$ cannot have a pole of order greater $g+1$ at $P_\infty$ or $P_\infty'$, and only has poles at these points, it follows that the differential $\frac{y\Psi_i^y(x) }{x^{i+1}H(x)}dx$ is regular on $U_0 = X \backslash \pi^{-1}(0)$.
Thus we have completed the proof.


\end{proof}


%%%%%%%%%%%%%%%%%%%%% Section 4 %%%%%%%%%%%%%%%%%%%%

\section{Splitting of the short exact sequence}


As in the previous section, we assume $X$ is a hyperelliptic curve, but now with genus $g \geq 2$.
Furthermore, we now assume that $\cha(k) \neq2$.

In the previous section (Theorem \ref{basis}) we found a basis for the de Rham cohomology of any hyperelliptic curve using \cech cohomology, with respect to the cover $\cU = \{ U_0 , U_\infty\}$.
We let $\lambda_i$ and $\gamma_i$ denote the elements of this basis by defining
\begin{align*}
\lambda_i  = & \left( \frac{x^i}{y}dx, \frac{x^i}{y}dx, 0\right) ,\quad i=0, \ldots, g-1 \\
\intertext{and}
\gamma_i = & \left ( \frac{\psi_i(x)}{2yx^{i+1}}dx, \frac{-\phi_i(x)}{2yx^{i+1}}dx, x^{-i}y \right), \quad i=1,\ldots ,g.
\end{align*}
We now fix some $a \in \mathbb P_k^1\backslash \{0, \infty\}$ and define $U_a : = X \backslash \pi^{-1}(a)$ and $\cU''~:=~\{U_0,U_a, U_\infty\}$.
Then $\cechderhamhone(\cU'')$ can be defined as the $k$-vector space 
\begin{multline}\label{sixtupleconditions}
\left\{ (\omega_0, \omega_a, \omega_\infty , f_{0a}, f_{0 \infty},f_{a \infty}) | \omega_i \in \Omega_X(U_i), f_{ij} \in \cO_X(U_i \cap U_j), \right. \\ \left. f_{0a} - f_{0\infty} + f_{a \infty} = 0, df_{ij} = \omega_i - \omega_j \right\}
\end{multline}
quotiented by the subspace 
\[
\left\{ df_0, df_a df_\infty, f_0- f_a, f_0 - f_\infty, f_a - f_\infty | f_i \in \cO_X(U_i)\right\}.
\]

We also define the following polynomials for $1 \leq i \leq g$
\[
r_i(x) : = \sum_{k=0}^{i-1} (-1)^{g-k}\binom{g}{k} a^{g-k} x^k
\]
and
\[
t_i(x) := \sum_{k=i}^{g} (-1)^{g-k}\binom{g}{k} a^{g-k} x^k,
\]
splitting the polynomial $(x-a)^g$ in to two parts.
The following proposition allows us to compute how some elements of $G$ ct on $\derhamhone$.


\begin{prop}\label{basis22}
The pre-image of $\rho^{-1}(\gamma_i)$ for $i \in \{1, \ldots, g\}$ is the residue class of
\begin{multline*}
\nu_i = \left(\frac{\psi_i(x)}{2yx^{i+1}}dx, \frac{(\psi_i(x)t_i(x) - \phi_i(x)r_i(x))(x-a) + 2if(x)(-1)^{g-i+1}\binom{g}{i} a^{g-i+1}x^i}{2yx^{i+1}(x-a)^{g+1}}dx,\right. \\\left. \frac{\phi_i(x)}{2yx^{i+1}}dx,  \frac{r_i(x)y}{x^i(x-a)^g}, \frac{y}{x^i},  \frac{t_i(x)y}{x^i(x-a)^g} \right).
\end{multline*}
\end{prop}
\begin{rem}
Note that the third term, $\frac{\phi_i(x)}{2yx^{i+1}} dx$, has a different sign than the equivalent term in $\gamma_i$.
This is due the increased number of open sets in the cover moving it from the second to third position.
\end{rem}
\begin{proof}
In order to be able to refer to the entries in $\nu_i$ we let
\[
\nu_i = \left( \omega_{0 i}, \omega_{a i}, \omega_{\infty i}, f_{0 a i}, f_{0 \infty i}, f_{a \infty i} \right).
\]
First, note that it follows from the proof of Theorem \ref{basis} that $d(f_{0 \infty i}) = \omega_{0 i} + \omega_{\infty i}$, and that $f_{0 \infty i}, \omega_{0 i}$ and $\omega_{\infty i}$ are regular on the appropriate open sets.

Since $r_i(x)+t_i(x)$ is the binary expansion of $(x-a)^g$ then
\begin{align*}
f_{0 a i} - f_{0 \infty i}+ f_{a \infty i} & = \frac{r_i(x)y}{x^i(x-a)^g} - \frac{y}{x^i} + \frac{t_i(x)y}{x^i(x-a)^g} \\
& = \frac{y(r_i(x) + t_i(x) - (x-a)^g)}{x^i(x-a)^g} \\
& = 0.
\end{align*}



We now check that differentials and functions are regular on the appropriate open sets by computing the relevant divisors.
Firstly, by \eqref{divxp=2} and \eqref{differentialdivisor},
\begin{align*}
\di \left( f_{0 a i} \right) & = \di \left( \frac{r_i(x)y}{x^i(x-a)^g} \right) \\
&  = \di(r_i(x)) + \di(y) - i\di(x) - g\di(x-a) \\
& \geq \di_0(r_i(x)) - (i-1)D_\infty +R - (g+1)D_\infty - iD_0 + iD_\infty - gD_a + gD_\infty \\
& = \di_0(r_i(x)) +R -iD_0 - gD_a,
\end{align*}
which is non-negative on $U_0 \cap U_a$.
On the other hand, also by \eqref{divxp=2} and \eqref{differentialdivisor},
\begin{align*}
\di \left( f_{a \infty i} \right) & = \di \left( \frac{t_i(x)y}{x^i(x-a)^g} \right) \\
& = \di\left(\frac{t_i(x)}{x^i}\right) + \di(y) - g\di(x-a) \\
& = \di_0 \left( \frac{t_i(x)}{x^i} \right) - (g-i)D_\infty +R - (g+1)D_\infty - gD_a + gD_\infty\\
& = \di_0 \left( \frac{t_i(x)}{x^i} \right) +R - gD_a -(g-i+1)D_\infty,
\end{align*}
where the third equality holds because $t_i(x)/x^i$ is regular on $U_\infty$.
Since $y$ is also regular on $U_\infty$  we conclude that $({t_i(x)y})/({x^i(x-a)^g})$ is regular on $U_a \cap U_\infty$.

To show that
\begin{equation}\label{longequation}
\omega_{a i} =  \frac{(\psi_i(x)t_i(x) - \phi_i(x)r_i(x))(x-a) + 2if(x)(-1)^{g-i+1}\binom{g}{i} a^{g-i+1}x^i}{2yx^{i+1}(x-a)^{g+1}}dx
\end{equation}
is regular on $U_a$ we first compute the divisor
\begin{align*}
\di\left( \frac{dx}{2yx^{i+1}(x-a)^{g+1}}\right) & = \di(dx) - \di(y) - (i+1)\di(x) - (g+1)\di(x-a) \\
& = R - 2D_\infty - R + (g+1)D_\infty - (i+1)D_0 + (i+1)D_\infty \\
& \quad - (g+1)D_a + (g+1)D_\infty \\
& = (2g+i+1)D_\infty -(i+1)D_0 - (g+1)D_a,
\end{align*}
using \eqref{divxp=2}, \eqref{differentialdivisor} and \eqref{pnot2divisors}.
We next show that the numerator of \eqref{longequation},
\begin{equation}\label{numerator}
{(\psi_i(x)t_i(x) - \phi_i(x)r_i(x))(x-a) + 2if(x)(-1)^{g-i+1}\binom{g}{i} a^{g-i+1}x^i},
\end{equation}
has degree less then $2g+i+2$, from which it follows that \eqref{longequation} doesn't have a pole at the point(s) in $\pi^{-1}(\infty)$.
The degree of $\psi_i(x)t_i(x)(x-a)$ is at most $2g+2$, which is less than $2g+2+i$ for all $i \geq 1$.
If $\deg(f) = 2g+1$, then clearly
\[
\deg\left( \phi_i(x)r_i(x)(x-a) \right) = \deg(\phi_i) + \deg(r_i(x)) + \deg(x-a) \leq 2g+1 + i-1 +1 = 2g+i+1
\]
and
\[
\deg \left( 2if(x)(-1)^{g-i+1}\binom{g}{i} a^{g-i+1}x^i \right)  \leq  2g+1+i .
\]
Lastly, if $\deg(f(x)) = 2g+2$ then the term of degree $2g+i+2$ in $-\phi_i(x)r_i(x)(x-a)$ is
\begin{align*}
-\left((2g+2)a_{2g+2}x^{2g+2}-\right.&\left.2ia_{2g+2}x^{g+2}\right)\left( (-1)^{g-i+1}\binom{g}{i-1}a^{g-i+1}x^i\right) \\
&  = 2(-1)^{g-i+2}\left( (g-i+1)\binom{g}{i-1} \right) a_{2g+2}a^{g-i+1}x^{2g+i+2} \\
& = 2(-1)^{g-i} \left( \frac{g!}{(i-1)!(g-i)!} \right) a_{2g+2}a^{g-i+1}x^{2g+i+2} \\
& = 2i(-1)^{g-i}\binom{g}{i}a_{2g+2}x^{2g+i+2},
\end{align*}
which cancels with the term of the same degree in $2if(x)(-1)^{g-i+1}\binom{g}{i}a^{g-i+1}x^i$.
Since these terms cancel, we again have the that the degree of \eqref{numerator} is at most $2g+i+1$, and \eqref{longequation} has no pole(s) at the point(s) in $\pi^{-1}(\infty)$.

We now show that \eqref{numerator} is divisible by $x^{i+1}$.
By definition $x^{g+2} | \phi_i(x)$, and hence for $i \leq g$ it follows that $x^{i+1}|\phi_i(x)r_i(x)(x-a)$.
On the other hand, the lowest degree terms of $2if(x)(-1)^{g-i+1}\binom{g}{i}a^{g-i+1}x^i$ and $\psi_i(x)t_i(x)(x-a)$ which can be non-zero are, respectively,
\[
 2ia_0(-1)^{g-i+1}\binom{g}{i}a^{g-i+1}x^i 
\]
and
\[
 (-a)(-2ia_0)\left( (-1)^{g-i}\binom{g}{i}a^{g-i}x^i \right).
\]
Since
\[
(-a)(-2ia_0)\left( (-1)^{g-i}\binom{g}{i}a^{g-i}x^i \right) = 2ia_0(-1)^{g-i+2}\binom{g}{i}a^{g-i+1}x^i,
\]
it follows that when adding $\psi_i(x)t_i(x)(x-a)$ and $2if(x)(-1)^{g-i+1}\binom{g}{i}a^{g-i+1}x^i$ these terms will cancel.
Hence the numerator \eqref{numerator} is divisible by $x^{i+1}$.


It only remains to show that $\omega_{a i} = \omega_{0 i} -df_{0 a i}$.
We begin this by computing $df_{0 a i}$, which is
\begin{align*}
df_{0 a i} & = d \left( \frac{y r_i(x)}{x^i(x-a)^g} \right) \\
& = \frac{r_i(x)}{x^i(x-a)^g}dy + y d\left( \frac{r_i(x)}{x^i(x-a)^g} \right) \\
& = \frac{f(x)'r_i(x)}{2yx^i(x-a)^g}dx + y\left( \frac{r_i(x)'}{x^i(x-a)^g} -\frac{i r_i(x)}{x^{i+1}(x-a)^g} - \frac{gr_i(x)}{x^i(x-a)^{g+1}}\right) dx \\
& = \frac{xf(x)'r_i(x)(x-a) + 2f(xr_i(x)'(x-a) - i(x-a)r_i(x) - gxr_i(x))}{2yx^{i+1}(x-a)^{g+1}} dx.
\end{align*}
Hence $\omega_{0 i} - df_{0 a i}$ expands to
\[
\frac{\psi_i(x)(x-a)^{g+1} - xf(x)'r_i(x)(x-a) - 2f(x)\left(xr_i(x)'(x-a)-i(x-a)r_i(x)-gxr_i(x)\right)}{2yx^{i+1}(x-a)^{g+1}}dx.
\]
Now
\[
(x-a)^{g+1} = (x-a)^g(x-a)  = (r_i(x) + t_i(x))(x-a)
\]
and
\begin{multline*}
xf(x)'r_i(x)(x-a) - 2if(x)r_i(x)(x-a) = r_i(x)(x-a)(xf(x)'-2if(x)) \\
= r_i(x)(x-a)(\psi_i(x) + \phi_i(x)).
\end{multline*}
So
\[
\psi_i(x)(x-a)^{g+1} - xf(x)'r_i(x)(x-a) - 2if(x)r_i(x)(x-a) = (\psi_i(x)t_i(x) - \phi_i(x) r_i(x))(x-a).
\]


We now compute $(x-a)r_i(x)'-gr_i(x)$.
First, we note that
\begin{align*}
r_i(x)' & = \sum_{k=1}^{i-1} k (-1)^{g-k} \binom{g}{k} a^{g-k} x^{k-1} \\
& = \sum_{k=0}^{i-2} (k+1) (-1)^{g-k-1} \binom{g}{k+1}a^{g-k-1} x^k.
\end{align*}
From this it follows that
\begin{align*}
r_i(x)'(x-a) & = x \sum_{k=1}^{i-1} k (-1)^{g-k} \binom{g}{k} a^{g-k} x^{k-1} - a \sum_{k=0}^{i-2} (k+1) (-1)^{g-k-1} \binom{g}{k+1}a^{g-k-1} x^k \\
& = \sum_{k=1}^{i-1} k (-1)^{g-k} \binom{g}{k} a^{g-k} x^k  + \sum_{k=0}^{i-2} (k+1) (-1)^{g-k} \binom{g}{k+1}a^{g-k} x^k \\
& = gr_i(x) + (-1)^{g-i+2}i \binom{g}{i}a^{g-i+1}x^i,
\end{align*}
since
\begin{align*}
k\binom{g}{k} + (k+1)\binom{g}{k+1} & = k \left( \frac{g!}{k!(g-k)!} \right) + (k+1) \left( \frac{g!}{(k+1)!(g-k-1)!} \right) \\
& = \frac{g!}{(k-1)!(g-k)!} + \frac{g!}{k!(g-k-1)!} \\
& = \frac{g\cdot g!}{k!(g-k)!} \\
& = g \binom{g}{k}.
\end{align*}
Hence $r_i(x)'-gr_i(x)= (-1)^{g-i+2}i\binom{g}{i} a^{g-i+1}x^i$. 

Combining the above we conclude that
\[
\omega_{0 i } - df_{0a i} =  \frac{(\psi_i(x)t_i(x) - \phi_i(x)r_i(x))(x-a) + 2if(x)(-1)^{g-i+1}\binom{g}{i} a^{g-i+1}x^i}{2yx^{i+1}(x-a)^{g+1}}dx= \omega_{a i}.
\]

Note that the last relation ($df_{a \infty i} = \omega_{a i} - \omega_{\infty i}$) holds, since 
\[
df_{a \infty i} = df_{0 \infty i} - df_{0 a i} = \omega_{0 i} - \omega_{\infty i } - \omega_{0 i} + \omega_{a i} = \omega_{a i} - \omega_{\infty i}.
\]
\end{proof}

We note that for any $\tau \in \aut(X)$ the following diagram commutes,
\[
\begin{array}{ccc}
X & \xrightarrow[\tau] & X \\
\downarrow\pi & & \downarrow\pi \\
\mathbb P^1_k & \xrightarrow[\tau] & \mathbb P_k^1
\end{array}
\]
since the hyperelliptic involution $\sigma$ is in the centre of $\aut(X)$.\todo{reference liu}
Hence, if $a = \tau(0)$, the following diagram also commutes:
\[
\begin{array}{ccc}
\derhamhone \cong \cechderhamhone(\cU)  & \xleftarrow{\rho} & \cechderhamhone(\cU'')  \\
\tau^*\downarrow & ~ & \rho'\downarrow  \\
\derhamhone \cong \cechderhamhone(\cU)  & \xleftarrow{\tau^*} & \cechderhamhone(\cU')
\end{array}
\]
Here $\rho$ and $\rho'$ are the projections on the first, third and fifth coordinates and on the second, third and sixth coordinates respectively.


Clearly if $\tau^* \colon x \mapsto x+a$ then $a = \tau(0)$, and we assume in what follows that $\tau$ is such an automorphism.
\begin{lem}
The action of $\tau^*$ on $y$ is given by $\tau^*(y) = y$ or $\tau^*(y) = -y$.
\end{lem}
\begin{proof}
Since $y^2 \in k(x)$ then there must exist $g_1(x), g_2(x) \in k(x)$ such that 
\begin{equation*}
\tau^*(y) = g_1(x)y + g_2(x) \in k(x,y).
\end{equation*}
Hence
\begin{equation}\label{easylemma}
f(x+a) = \tau^*(y^2) = (\tau^*(y))^2 = g_1(x)^2f(x)+2g_1(x)g_2(x)y + g_2(x)^2.
\end{equation}
Firstly, note that if neither $g_1(x)$ nor $g_2(x)$ are zero then
\[
y = \frac{f(x+a) - g_1(x)^2f(x) - g_2(x)^2}{2g_1(x)g_2(x)},
\]
which contradicts the fact $K(X)$ is a degree two extension of $k(x)$.
Hence one of $g_1(x)$ or $g_2(x)$ must be zero.

If $g_1(x) = 0$ then $\tau^*$ would not be an automorphism, since $y$ would not be in the image.
Hence $\tau^*(y) = g_1(x)y$.
Also, by comparing the degrees in \eqref{easylemma} we see that $\deg(g_1(x)) = 0$, and then by comparing coefficients in the same equation we see that $g_1(x)^2 = 1$.
Hence $\tau^*(y) = \pm y$.
\end{proof}
\begin{rem}
If $\tau^*(y) = -y$ we can, without loss of generality, replace $\tau$ by $\tau \circ \sigma$, where $\sigma$ denotes the hyperelliptic involution.
Hence we will assume throughout the rest of the paper that $\tau^*(y) = y$.
\end{rem}

\begin{ex}
If $2g+1$ is an odd prime then exists an automorphism $\tau$ of the form $\tau\colon (x,y) \mapsto (x+a,y)$ if and only if $p=2g+1$ and $f(x) = x^p - a^{p-1}x + a_0$, for some $a_0 \in k$.
In this case we may assume that $a_0 = 0$, since if it doesn't we can apply the automorphism $x \mapsto x+b, y\mapsto y$ to $K(X)$, where $b$ is a root of $f(x)$.
Moreover, given an equation of the form $y^2 = x^p - a^{p-1}x$, we can apply automorphism of $K(X)$ to replace the coefficient $a^{p-1}$ by $1$, namely $x \mapsto ax, y \mapsto a^{\frac{p}{2}}y$.
Hence we see that all such curves are isomorphic to those defined by $y^2 = x^p - x$.
In particular, it follows from \cite{canonicalrepresentation} that the short exact sequence in Proposition \ref{ses} doesn't split when the above conditions hold.
\end{ex}


We now describe the action of $\tau^*$ on $\lambda_i$.
\begin{lem}
For each $i \in \{ 0, \ldots, g-1\}$ then 
\[
\tau^*(\lambda_i) = \sum_{k = 0}^i \binom{i}{k}a^{i-k}x^k.
\]
\end{lem}
\begin{proof}
Since $\tau^*$ acts trivially on $y$, it follows that
\[
\tau^*\left( \frac{x^i}{y} \right) = \sum_{k=0}^i \binom{i}{k}a^{i-k}\frac{x^k}{y}.
\]
The statement follows from this.
\end{proof}


\begin{thm}
Suppose that $\deg(f(x)) = p^n $ for some $n \in \NN$.
Then the short exact sequence of $k[G]$-modules
\[
0 \ra \hzero \ra \derhamhone \ra \hone \ra 0
\]
does not split.
\end{thm}
\begin{proof}
We suppose that the sequence does split, and that $s \colon \hone \ra \derhamhone$ is the splitting map.

We now examine the action of $\tau^*$ on $\gamma_g$ and $\bar \gamma_g$.
We first look at the sixth entry in $\nu_g$ from Proposition \ref{basis22}.
This entry is
\[
\frac{t_g(x)y}{x^g(x-a)^g} = \frac{y}{(x-a)^g},
\]
and clearly $\tau^*(y/(x-a)^g) = y/x^g$.
Hence $\tau^*(\bar\gamma_g) = \bar \gamma_g$, and $\tau^*(\gamma_g) = \gamma_g + \sum_{i =0}^{g-1}c_i\lambda_i$, for some $c_i \in k$.
We now compute $c_{g-1}$.

We begin this by computing the lead term of the first entry of $\tau^*(\gamma_g) \in \cechderhamhone(\cU)$.
This is equal to the lead term of $\omega_{\tau g}$ in $\nu_g$ by Proposition \eqref{basis22}, since for any polynomial $h(x)$ it follows that $h(x)$ and $h(x+a)$ have teh same lead coefficient.
As
\[
\deg(\psi_g(x)t_g(x)x) \leq g+1 + g+ 1 = 2g+1 < 3g+1 = \deg(\phi_g(x)r_g(x)x) = \deg(f(x)x^g)
\]
we need only compute the coefficient of $x^{3g+1}$ in $2gf(x)(-1)ax^g - \phi_g(x)r_g(x)x$ and show that it is non-zero to find the coefficient of the lead term.
Rearranging $2g+1 = p^n $ gives us the identity
\[
g = \frac{p^n - 1}{2}.
\]
From this we see that the lead coefficient of $2gf(x)(-1)ax^g$ is 
\[
2\left( \frac{p^n-1}{2} \right) (-1)a = a,
    \]
    since $\cha(k) = p$.
On the other hand, the lead term of $-\phi_g(x)r_g(x)x$ is
\[
-(p^n-2g)(-1)\binom{g}{g-1}a = 2\left(\frac{p^n -1 }{2}\right) (-1)\left( \frac{p^n - 1}{2} \right)a = -\frac{a}{2}.
\]
Finally, it follows that the lead coefficient of the numerator in the second term of is
\[
a - \frac{a}{2} = \frac{a}{2}.
\]


Since the denominator of $\omega_{a g}$ is of degree $2g+2$, we see that overall the degree of the first entry of $\gamma_g$ (as a polynomial in $x$) is $g-1$.
Now the degree of $\frac{\psi_g(x)}{2yx^{g+1}}$ is less than this, as is the degree of the first entry in $\lambda_i$ for $i< g-1$.
However, the degree of the first entry of $\lambda_{g-1}$ is precisely $g-1$.
Hence, by comparing coefficients, we see that $c_{g-1} = \frac{a}{2}$.

Now suppose that 
\[
 s( \bar \gamma_g) = \gamma_g + \sum_{i=0}^{g-1}d_i \lambda_i
\]
for some $d_i \in k$.
Then, on the one hand,
\[
s(\tau^*(\bar\gamma_g)) = s(\bar\gamma_g) =  \gamma_g + \sum_{i=0}^{g-1}d_i\lambda_i,
\]
whilst on the other hand
\begin{align}
\tau^*(s(\bar\gamma_g)) & = \tau^*(\gamma_g + \sum_{i=1}^{g-1} d_i\lambda_i ) \\
& = \gamma_g + \frac{a}{2}\lambda_{g-1} + \sum_{i=0}^{g-2} c_i \lambda_k + \sum_{i=0}^{g-1} d_i \sum_{k=0}^{i} \binom{i}{k}a^{i-k}\lambda_k.
\end{align}
Hence we see that for $s(\tau^*(\gamma_g))$ to equal $\tau^*(s(\gamma_g))$ we require $\frac{a}{2} + d_{g-1} = d_{g-1}$; \ie that $\frac{a}{2} = 0$. 
But then $a=0$, which is a contradiction.

\end{proof}











\begin{comment}
We now suppose that $X$ is of genus 2, and that
\[
f(x) = a_6x^6 + a_5x^5 + a_4x^4 + a_3x^3 + a_2x^2 + a_1x + a_0.
\]
The existence of $\sigma^*$ places restrictions on the characteristic of $k$ and the coefficients of $f(x)$, as stated in the following lemma.

\begin{lem}
The characteristic of $k$ is 3 or 5. 
Moreover, if $\cha(k)=5$ then the coefficients $a_6, a_4, a_3$ and $a_2$ in \eqref{definingequation} are zero.
\end{lem}
\begin{proof}
Since $\sigma^*(y) = y$, it follows that $f(x+a) = f(x)$.
Expanding the left hand side we see that the coefficient of $x^5$ is $6aa_6 + a_5$.
Since the coefficient of $x^5$ in $f(x)$ is $a_5$.
We conclude that $6aa_6 = 0$, and hence either $6=0$ or $a_6=0$.
If $6=0$ then it follows that $\cha (k)=3$.

On the other hand, if $a_6=0$ then we compare the coefficients of $x^4$ in $f(x+a)$ and $f(x)$, which are $5aa_5+a_4$ and $a_4$ respectively.
We conclude that $5aa_5=0$. Since the degree of $f(x)$ is either $2g+2=6$ or $2g+1=5$ it follows that at least one of $a_6$ and $a_5$ must be non-zero.
Hence if $a_6=0$ it follows that $\cha(k) = 5$.

The final statement follows again by comparing coefficients.
\end{proof}


Recall that $\sigma^*$ fixes $y$ and maps $x$ to $x+a$. 
Note that this means that $\sigma^*(dx) = d(\sigma^*(x)) = d(x+a) = dx$.
The action of $\sigma^*$ on $\nu_i$ is given by
\begin{equation*}
\sigma^*(\nu_i) = \left( \sigma^*( \omega_{\sigma i}), \sigma^*(\omega_{\infty i}), \sigma^*( f_{\sigma \infty i})\right).
\end{equation*}

Note that in this specific case the basis of $\cechderhamhone(\cU)$ is formed by
\begin{align*}
\gamma_1  = & \left( \frac{1}{y}dx, \frac{1}{y}dx, 0\right) \\
\gamma_2 = & \left(\frac{x}{y}dx, \frac{x}{y}dx, 0\right) \\
\gamma_3 = & \left( \frac{a_3x^3-a_1x-2a_0}{2yx^2}dx, \frac{-(4a_6x^4+3a_5x^3+2a_4x^2)}{2y}dx, \frac{y}{x} \right)\\
\gamma_4 = & \left(\frac{-(a_3x^3+2a_2x^2+3a_1x+4a_0)}{2yx^3},\frac{-(2a_6x^3+a_5x^2)}{2y}dx , \frac{y}{x^2} \right).
\end{align*}
Also, the pre-image $\rho^{-1}(\tau_i)$ computed in the previous proposition can be more explicitly written as 
\begin{multline*}
\nu_1 = \left(\frac{a_3x^3-a_1x-2a_0}{2yx^2}dx, \frac{\lambda_1}{2y(x-a)^3}dx, \frac{-(4a_6x^4+3a_5x^3+2a_4x^2)}{2y}dx,  \right. \\ \left. \frac{a^2y}{x(x-a)^2}, \frac{y}{x},  \frac{y(x-2a)}{(x-a)^2} \right)
\end{multline*}
and
\begin{multline*}
\nu_2 = \left( \frac{-(a_3x^3+2a_2x^2+3a_1x+4a_0)}{2yx^3},  \frac{\lambda_2}{2y(x-a)^3}dx, \frac{-(2a_6x^3+a_5x^2)}{2y}dx, \right. \\ \left.  \frac{y(a^2-2ax)}{x^2(x-a)^2},\frac{y}{x^2}, \frac{y}{(x-a)^2} \right),
\end{multline*}
where
\begin{multline*}
\lambda_1  := (4a^3a_6+a^2a_5+a_3)x^4 + (3a^3a_5+2a^2a_4-3aa_3)x^3 \\
 + (2a^3a_4 + 6a^2a_3 - a_1)x^2 + (4a^2a_2 +3aa_1-2a_0)x+(2a^2a_1+6aa_0)
\end{multline*}
and
\begin{multline*}
\lambda_2  := (-6a^2a_6-2aa_5)x^4 + (2a^3a_6-3a^2a_5-4aa_4-a_3)x^3 \\ + (a^3a_5-3aa_3-2a_2)x^2 + (-2a_2a-3a_1)x +(-aa_1-4a_0).
\end{multline*}

We use this to prove the following proposition about the action of $\sigma^*$ on $\cechderhamhone(\cU)$.

\begin{prop}
The action of $\sigma^*$ on the basis elements of $\cechderhamhone(\cU)$ is
\begin{align*}
\sigma^*(\gamma_1) & = \gamma_1 \\
\sigma^*(\gamma_2) & = \gamma_2 + a\gamma_1 \\
\sigma^*(\gamma_3) & = \gamma_3 - a\gamma_4 + \frac{4a^3a_6 + a^2a_5}{2}\gamma_2 + a^3a_5\gamma_1 \\
\sigma^*(\gamma_4) & =  \gamma_4 - \frac{2a^3a_6+a^2a_5}{2} \gamma_2 - aa_5\gamma_1.
\end{align*}
\end{prop}
\begin{proof}
The action on $\gamma_1$ and $\gamma_2$ is clear.

For $\gamma_3$ and $\gamma_4$ we will consider this action entry by entry, starting with $\sigma^*(\omega_{\sigma 1})$, which is
\begin{align*}
\sigma^*( \omega_{\sigma}) & = \sigma^* \left( \frac{\lambda_1}{2y(x-a)^3} dx \right) \\
& = \frac{\sigma^*(\lambda_1)}{2yx^3}dx.
\end{align*}
Now we compute $\sigma^*(\lambda_1)$, which is
\begin{align*}
\sigma^*(\lambda_1) & = (4a^3a_6+a^2a_5+a_3)x^4 + (16a^4a_6+7a^3a_5+2a^2a_4+aa_3)x^3 \\
& + (24a^5a_6+15a^4a^5+8a^3a_4+3a^2a_3-a_1)x^2 \\
& + (16a^6a_6+13a^5a_5+10a^4a_4+7a^3a_3+4a^2a_2+aa_1-2a_0)x \\
& + a(4a^6a_6+4a^5a_5+4a^4a_4+4a^3a_3+4a^2a_2+4aa_1+4a_0).
\end{align*}
We use the fact $f(x+a) = f(x)$ to simplify the above expression, by equating the coefficients on both side.
We list the identities we get from doing this here:
\begin{enumerate}
\item \label{1} $6aa_6=0$;
\item \label{2} $15a^2a_6+5aa_5 = 0$;
\item \label{3} $20a^3a_6+10a^2a_5+4aa_4 = 0$;
\item \label{4} $15a^4a_6 + 10a^3a_5 + 6a^2a_4+3aa_3= 0$;
\item \label{5} $6a^5a_6 + 5a^4a_5+4a^3a_4+3a^2a_3+2aa_2= 0$;
\item \label{6} $a^6a_6+a^5a_5+a^4a_4+a^3a_3+a^2a_2+aa_1 = 0$.
\end{enumerate}
Hence the constant coefficient of $\sigma^* (\lambda_1)$ simplifies to
\begin{equation*}
 a(4a^6a_6+4a^5a_5+4a^4a_4+4a^3a_3+4a^2a_2+4aa_1+4a_0) =  4aa_0
\end{equation*}
by \ref{6}; the coefficient of $x$ simplifies to
\[
16a^6a_6+13a^5a_5+10a^4a_4+7a^3a_3+4a^2a_2+aa_1-2a_0 = 3aa_1 -2a_0
\]
by \ref{5} and \ref{6}; the coefficient of $x^2$ simplifies to 
\[
24a^5a_6+15a^4a^5+8a^3a_4+3a^2a_3-a_1 = 2a_2a^2-a_1
\]
by \ref{4} and \ref{5}; the coefficient of $x^3$ simplifies to 

\[
16a^4a_6+7a^3a_5+2a^2a_4+aa_3 = 2a^3a_5 +aa_3
\]
by \ref{1} and \ref{3}.
Hence
\begin{multline*}
\sigma^*(\lambda_1) = \\ \frac{(4a^3a_6+a^2a_5+a_3)x^4+(2a^3a_5+aa_3)x^3+(2a^2a_2-a_1)x^2+(3aa_1-2a_0)x+4aa_0}{2yx^3}dx.
\end{multline*}
We also have
\begin{align}\label{i=1sigmaaction}
\sigma^*(f_{\sigma \infty 1}) & = \sigma^*\left(\frac{y(x-2a)}{(x-a)^2} \right) \nonumber\\
& = \frac{y(x-a)}{x^2}  \
 = \frac{y}{x} - \frac{ay}{x^2}.
\end{align}
From \eqref{i=1sigmaaction} we see that
\begin{equation}\label{firstidentity}
\sigma^*(\gamma_3) = \gamma_3 - a\gamma_4 + c_1\gamma_1 + c_2 \gamma_2
\end{equation}
for some $c_1,c_2 \in k$.
Finally we have
\begin{align*}
\sigma^* (\omega_{\infty 1}) & = \sigma^*\left( \frac{-(4a_6x^4+3a_5x^3+2a_4x^2)}{2y} dx \right) \\
& = - \left( \frac{4a_6x^4+(16aa_6+3a_5)x^3+(24a^2a_6+9aa_5+2a_4)x^2}{2y}\right. \\ 
& \left. +\frac{(16a^3a_6+9a^2a_5+4aa_4)x+4a^4a_6+3a^3a_5+2a^2a_4}{2y}\right) dx.
\end{align*}
Using the above we can compute the $c_1$ and $c_2$ mentioned above.
We compute the coefficient of $\omega_{\infty}$ (using the notation of \eqref{sixtupleconditions}) in $\sigma^*(\rho'(\nu_1)) - \gamma_3 + a\gamma_4$, which is
\begin{multline*}
 -\left( \frac{18aa_6x^3 +(24a^2a_6+10aa_5)x^2+(16a^3a_6+9a^2a_5+4aa_4)x}{2y} \right. \\
+ \left. \frac{4a^4a_6+3a^3a_5+2a^2a_4}{2y}\right) dx  = \frac{(4a^3a_6+a^2a_5)x +2a^3a_5}{2y}dx.
\end{multline*}
From this we see that $c_1= a^3a_5$ and that $c_2 = \frac{4a^3a_6+a^2a_5}{2}$.


Now when $i=2$ we have
\begin{align*}
\sigma^*(\lambda_2) & = -(6a^2a_6+2aa_5)x^4 - (22a^3a_6+11a^2a_5+4aa_4+a_3)x^3 \\
& - (30a^4a_6+20a^3a_5+12a^2a_4+6aa_3+2a_2)x^2 \\
& - (18a^5a_6+15a^4a_5+12a^3a_4+9a^2a_3+6aa_2+3a_1)x \\
& -4(a^6a_6+a^5a_5+a^4a_4+a^3a_3+a^2a_2+aa_1+a_0).
\end{align*}
Again, using the relations \ref{1} through \ref{6} we can simplify these coefficients:
by \ref{6} the constant coefficient of $\sigma^*(\lambda_2)$ is
\[
-4(a^6a_6+a^5a_5+a^4a_4+a^3a_3+a^2a_2+aa_1+a_0) = -4a_0;
\]
by \ref{5} the coefficient of $x$ is
\[
- (18a^5a_6+15a^4a_5+12a^3a_4+9a^2a_3+6aa_2+3a_1) = -3a_1;
\]
by \ref{4} the coefficient of $x^2$ is 
\[
- (30a^4a_6+20a^3a_5+12a^2a_4+6aa_3+2a_2) = -2a_2;
\]
by \ref{3} the coefficient of $x^3$ is
\[
- (22a^3a_6+11a^2a_5+4aa_4+a_3) = -2a^3a_5-a^2a_5-a_3;
\]
and by \ref{1} the coefficient of $x^4$ is
\[
-(6a^2a_6+2aa_5) = -2aa_5.
\]

Next we compute the action of $\sigma^*$ on $f_{\sigma \infty 2}$, as follows:
\begin{align*}
\sigma^*( f_{\sigma \infty 2}) & = \sigma^* \left( \frac{y}{(x-a)^2} \right) \\
& = \frac{y}{x^2}.
\end{align*}
Hence we see that
\[
\sigma^*( \gamma_4) = \gamma_4 + d_1\gamma_1 + d_2 \gamma_2
\]
for some $d_1, d_2 \in k$.
Lastly, we compute the action on $\omega_{\infty 2}$, which is
\begin{align*}
\sigma^* ( \omega_{\infty 2}) & = \sigma^* \left( \frac{-(2a_6x^3+a_5x^2)}{2y}dx \right) \\
& = \frac{-2a_6x^3-(6aa_6+a_5)x^2-(6a^2a_6+2aa_5)x-2a^3a_6-a^2a_5}{2y}dx \\
& = \frac{-2a_6x^3-a_5x^2 -2aa_5x -2a^3a_6 - a^2a_5}{2y}dx,
\end{align*}
using \ref{6} for the last equality.
We now use this to compute the $d_1$ and $d_2$ above, by computing the coefficient of $\omega_\infty$ of $\sigma^*(\rho'(\nu_1)) - \gamma_4$, as follows
\begin{multline*}
 \frac{-2a_6x^3-a_5x^2 -2aa_5x -2a^3a_6 - a^2a_5 + 2a_6x^3+a_5x^2}{2y}dx \\
 = \frac{-(2aa_5x +(2a^3a_6+a^2a_5))}{2y}dx.
\end{multline*}
Hence $d_2 = -aa_5$ and $d_1 = -\frac{2a^3a_6+a^2a_5}{2}$.
\end{proof}

Having computed the action we can now prove the following proposition regarding the splitting of the short exact sequence described in \cite{derhamactions}.

\begin{prop}
The short exact sequence of $k[G]$-modules
\begin{equation*}
0 \ra \hzero \ra \derhamhone \xrightarrow{p} \hone \ra 0
\end{equation*}
does not split if $\deg(f) = 5$.
\end{prop}
\begin{rem}
Note that $\deg(f)=5$ is equivalent to $\cha(k)=5$.
\end{rem}
\begin{proof}
Suppose there is a $k[G]$-module homomorphism $s\colon \hone \ra \derhamhone$ such that $p \circ s  = {\rm id} \colon \hone \ra \hone$.
Hence $s(\bar \gamma_3) = \gamma_3 + b_{31} \gamma_1 + b_{32} \gamma_2$ and $s(\bar \gamma_4) = \gamma_4 + b_{41} \gamma_1 + b_{42} \gamma_2$, for some $ b_{ij} \in k$.
Since $s$ is a $k[G]$-module homomorphism it follows that
\[
s(\sigma^* ( \bar \gamma_4)) = s ( \bar \gamma_4) = \gamma_4 + b_{41}\gamma_1 + b_{42} \gamma_2
\]
and also that
\begin{align*}
\sigma^*(s(\bar \gamma_4))  & = \sigma^* (\gamma_4 + b_{14}\gamma_1 b_{24}\gamma_2 ) \\
& = \gamma_4 -aa_5\gamma_2 - \left( \frac{2a^3a_6+a^2a_5}{2} \right) \gamma_1 + b_{14}\gamma_1 + b_{24}\gamma_2 + ab_{24}\gamma_1 \\
& = \gamma_4 + (b_{42}-aa_5)\gamma_2 + \left( \frac{2b_{14}+2ab_{24}-2a^3a_6-a^2a_5}{2} \right) \gamma_1
\end{align*}
are equal.
In particular we must have $b_{24}-aa_5 = b_{24}$, which implies that $aa_5 =0$.
If $a=0$ then $\sigma^*$ acts trivially. 
Hence $a \neq 0$ and it follows that $a_5=0$.
But if $\deg(f) = 5$ this is a contraction.
\end{proof}
\begin{rem}
Note that if $\deg(f) = 6$ then there is no contradiction with either term, and the short exact sequence can split. 
However, because we are not considering the entire automorphism group we cannot say that it does split.
\end{rem}
\todo[inline]{say what restrictions on $s$ we get from the above?}
\todo[inline]{state which exact sequences split from kontogeorgis paper}

The following is for the case where $g = 4$.
In this case our curve must have 
\[
y^2 = x^9 + a_6x^6+ a^2a_6x^4+a_3x^3a^4a_6x^2 +2(a^8 + a^2a_3)x +a_0
\]
as its defining equation, where we also have the relations
\begin{itemize}
\item $2a^6a_6 + a^4a_4 = 0$;
\item $a^a_4 + 2a^2a_2 = 0$;
\item $a^9 + a^3a_3 + aa_1 = 0$.
\end{itemize}
Also, the characteristic of $k$ is necessarily three.
As before, $sigma \colon x \to x+a$.
We denote the basis elements that come from $\hzero$ by $\lambda_i$, as $i$ ranges from $0$ to $g-1$, and the basis elements that map tpo $\hone$ by $\gamma_i$ (same range for $i$, but plus one).
Then we have the following lemma.
\begin{lem}
For $i = 0, \ldots , g-1$ then
\[
\sigma^*(\lambda_i) = \sum_{j=0}^1 \binom{i}{j} a^{i-j} \lambda_j.
\]
\end{lem}
\begin{proof}
Clear
\end{proof}

Next we explicilty write out the elements $\gamma_1, \ldots, \gamma_4$.
\begin{align*}
\gamma_1 & = \left( \frac{a^2a_6x^4 + 2a_3x^3 + 2(a^8 + a^2a_3)x + 2a_0}{x^2y}, \frac{x^7 +a_6x^3}{y}, \frac{y}{x} \right)\\
\gamma_2 & = \left(\frac{a^4a_6x^2 +2a_3x^3 +2a_0}{2x^3y} , \frac{2(x^6 +a_6x^3)}{y}, \frac{y}{ x^2 }\right)\\
\gamma_3 & = \left(\frac{a^2a_6x^4 + 2a^4a_6x^2 + 2(a^8 + a^2a_3)x }{2x^4y } , 0, \frac{y }{x^3 } \right)\\
\gamma_4 & = \left(\frac{a^2a_6x^4 +2a_3x^3 + 2(a^8 +a^2a_3)x + 2a_0 }{x^5y } , \frac{x^4+a_6x }{ y}, \frac{y }{ x^4} \right)
\end{align*}



We now compute the action of $\sigma^*$ on the above elements, computed via the sixtuple elements given earlier:
\begin{multline}
\left( \frac{a^4x^8 + 2a^2a_6x^7 + (2a^6 + a^3a_6 + a_3)x^6 + (2a^7 + aa_3)x^5 + (a^8  + a^a_3)x^4 + 2(a^6a_6+2a^3a_3+2a_0)x^3 + 2(a^{10}+a^4a_3+2aa_0)x^2 + 2(a^{11}+a^5a_3)x + 2a^3a_0}{2yx^5} \right. \\ \left. \frac{x^7 + a^6 + (2a^3+a_6)x^4 + (2a^4+aa_6)x^3 + (a^6 + a^3a_6)x + a^4a_6 + a^7}{y}, \frac{y(x^3-ax^2+a^2x-a^3)}{x^4} \right).
\end{multline}
From this we see that
\[
\sigma^*(\gamma_1) = \gamma_1 - a\gamma_2 + a^2\gamma_3 - a^3\gamma_4 + a^4\lambda_4 + (2a^6 + a^3a_6)\lambda_2 + (2a^7 + 2a^4a_6)\lambda_1.
\]
\todo[inline]{there are sign errors in the above, when comparing the second entry. Happens in lower order terms}

\begin{multline}
\sigma^*(\gamma_2) = \left( \frac{2a^3x^8 + (a^6 + 2a^3a_6 + a_3)x^5 + a^4a_6x^4 + 2a^5a_6x^3 + 2(a^9 + a^3+a_0)x^2}{2yx^5} \right., \\
\left. \frac{2(x^6 + (2a^3+a_6)x^3 + (a^6 + a^3a_6))}{y} , \frac{y(x^2 + ax}{x^4} \right).
\end{multline}
 

From this we deduce that
\[
\sigma^*(\gamma_2) = \gamma_2 + a \gamma_3 + 2a^3 \lambda_3 + (a^6 + a^3a_3 + 2a_3)\lambda_0.
\]





Next we compute $\sigma^*(\gamma_3)$, which is
\begin{equation*}
\left( \frac{a^2a_6x^5 + 2a^4a_6x^3 + (2a^8 + 2a^2a_3)x^2}{2yx^5}, 0, \frac{y}{x^3} \right)
\end{equation*}


Hence we conclude that $\sigma^*(\gamma_3) = \gamma_3)$.

Lastly, we compute $\sigma^*(\gamma_4)$, which is
\begin{multline*}
\left( \frac{2ax^8 + 2a^3x^6 + (2a^4 +aa_6)x^5 + 2a^2a_6x^4 (a^3a_6 + a_3)x^3 + 2a^4a_6x^2 + (a^8 + a^2a_3)x + a^6a_6 +a_0 }{2yx^5},\right. \\ \left. \frac{x^4 +ax^3 +(a^3+a_6)x + a^4+aa_6}{y}, \frac{y}{x^4} \right)
\end{multline*}

Then the action of $\sigma^*(\gamma_4)$ is 
\[
\sigma^*(\gamma_4) = \gamma_4 + 2a\lambda_3 + 2a^3\lambda_1 + 2(a^4 + aa_0)\lambda_0.
\]

%--------
%Could be useful to describe why the action of sigma changes covers
Suppose that $v = (\omega_\sigma , \omega_\infty, f_{\sigma \infty})$ is an element of $\cechderhamhone(\cU')$.
Then
\begin{equation*}
\sigma^*(v) := (\sigma^*(\omega_\sigma), \sigma^*(\omega_\infty), \sigma^*(f_{\sigma \infty})) \in \Omega_X(U_0) \times \Omega_X(U_\infty) \times \cO_X(U_\infty \cap U_0).
\end{equation*}
Since $\sigma^*(\omega_\sigma)-\sigma^*(\omega_\infty) = \sigma^*(\omega_\sigma - \omega_\infty) = \sigma^*(df_{\sigma \infty})$ we see that $\sigma^*(v)$ is an element of $\cechderhamhone(\cU)$.


\end{comment}
















\bibliography{biblio}
\bibliographystyle{amsalpha}
\end{document}
