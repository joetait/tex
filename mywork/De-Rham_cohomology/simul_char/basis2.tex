% !TEX TS-program = pdflatex
% !TEX encoding = UTF-8 Unicode

% This is a simple template for a LaTeX document using the "article" class.
% See "book", "report", "letter" for other types of document.

\documentclass[draft, 11pt]{article} % use larger type; default would be 10pt

\usepackage[utf8]{inputenc} % set input encoding (not needed with XeLaTeX)

%%% Examples of Article customizations
% These packages are optional, depending whether you want the features they provide.
% See the LaTeX Companion or other references for full information.

%%% PAGE DIMENSIONS
\usepackage{geometry} % to change the page dimensions
\geometry{a4paper} % or letterpaper (US) or a5paper or....
% \geometry{landscape} % set up the page for landscape
% read geometry.pdf for detailed page layout information

\usepackage{graphicx} % support the \includegraphics command and options
\usepackage[obeyDraft]{todonotes}

%\usepackage[parfill]{parskip} % Activate to begin paragraphs with an empty line rather than an indent

%%% PACKAGES
\usepackage[all]{xy}
\usepackage{mathtools}
\usepackage{booktabs} % for much better looking tables
\usepackage{array} % for better arrays (eg matrices) in maths
\usepackage{paralist} % very flexible & customisable lists (eg. enumerate/itemize, etc.)
\usepackage{verbatim} % adds environment for commenting out blocks of text & for better verbatim
\usepackage{subfig} % make it possible to include more than one captioned figure/table in a single float
%\usepackage{hyperref}
% These packages are all incorporated in the memoir class to one degree or another...

%\usepackage[activate={true,nocompatibility},final,tracking=true,kerning=true,spacing=true,factor=1100,stretch=10,shrink=10]{microtype}
%\microtypecontext{spacing=nonfrench}
% activate={true,nocompatibility} - activate protrusion and expansion
% final - enable microtype; use "draft" to disable
% tracking=true, kerning=true, spacing=true - activate these techniques
% factor=1100 - add 10% to the protrusion amount (default is 1000)
% stretch=10, shrink=10 - reduce stretchability/shrinkability (default is 20/20)

%%% HEADERS & FOOTERS
\usepackage{fancyhdr} % This should be set AFTER setting up the page geometry
\pagestyle{fancy} % options: empty , plain , fancy
\renewcommand{\headrulewidth}{0pt} % customise the layout...
\lhead{}\chead{}\rhead{}
\lfoot{}\cfoot{\thepage}\rfoot{}

%%% SECTION TITLE APPEARANCE
\usepackage{sectsty}
\allsectionsfont{\sffamily\mdseries\upshape} % (See the fntguide.pdf for font help)
\usepackage{amsmath}
\usepackage{amsthm}
\usepackage{amsfonts}
\usepackage{mathrsfs}
\usepackage{amsopn}
\usepackage{amssymb}
\usepackage{etex}
%\usepackage{natbib}
% (This matches ConTeXt defaults)

%%% ToC (table of contents) APPEARANCE
\usepackage[nottoc,notlof,notlot]{tocbibind} % Put the bibliography in the ToC
\usepackage[titles,subfigure]{tocloft} % Alter the style of the Table of Contents
%\renewcommand{\cftsecfont}{\rmfamily\mdseries\upshape}
%\renewcommand{\cftsecpagefont}{\rmfamily\mdseries\upshape} % No bold!
%\renewcommand{\familydefault}{\sfdefault}
%\usepackage{cabin}
%\usepackage{libertine}
%\usepackage[T1]{fontenc}

%Theorems and stuff
\theoremstyle{plain}
\newtheorem{defn}{Definition}[section]
\newtheorem{thm}[defn]{Theorem}
\newtheorem{cor}[defn]{Corollary}
\newtheorem{lem}[defn]{Lemma}
\newtheorem{prop}[defn]{Proposition}
\newtheorem{ex}[defn]{Example}
\newtheorem*{unnumthm}{Theorem}
\newtheorem{defnlem}[defn]{Definition/Lemma}
\newtheorem{defnthm}[defn]{Theorem/Definition}
\theoremstyle{remark}
\newtheorem*{rem}{Remark}
\newtheorem*{note}{Note}


\newcommand{\cO}{{\cal O}}
\newcommand{\ra}{\rightarrow}
\newcommand{\NN}{{\mathbb N}}
\newcommand{\PP}{{\mathbb P}}
\newcommand{\ZZ}{{\mathbb Z}}
\newcommand{\cL}{{\mathcal L}}
\newcommand{\cA}{{\mathcal A}}
\newcommand{\cD}{{\mathcal D}}
\newcommand{\cU}{{\mathcal U}}
\newcommand{\cech}{\v{C}ech }
\newcommand{\hzero}{{H^0(X,\Omega_X)}}
\newcommand{\hone}{H^1(X,\mathcal{O}_X)}
\newcommand{\cechhone}{\check{H}^1(\mathcal U,\mathcal O_X)}
\newcommand{\derhamhone}{H_{\text {dR}}^1(X/k)}
\newcommand{\cechhzero}{{\check{H}^0(X,\Omega_X)}}
\newcommand{\ubar}{\underset{\bar{}}}


\DeclareMathOperator{\aut}{Aut}
\DeclareMathOperator{\res}{Res}
\DeclareMathOperator{\ord}{ord}
\DeclareMathOperator{\di}{div}
\DeclareMathOperator{\cha}{char}
\DeclareMathOperator{\gal}{Gal}
\DeclareMathOperator{\Tr}{Tr}
\DeclareMathOperator{\Ima}{Im}

%%% END Article customizations

%%% The "real" document content comes below...

\title{Group actions on de Rham cohomology of hyperelliptic curves}
\author{}
%\date{} % Activate to display a given date or no date (if empty),
         % otherwise the current date is printed

\begin{document}
\maketitle

\listoftodos

\todo[inline]{change references to preprint when in thesis}
Let $X$ be a smooth, projective, connected hyperelliptic curve of genus $g \geq 2$ over an algebraically closed field $k$ of characteristic $p \geq 0$.
We  fix a map $\pi \colon X \rightarrow \mathbb P_k^1$ of degree two, which is unique up to an automorphism of $\mathbb P_k^1$.
We let $G$ denote a finite group acting on $X$.

We begin by recalling some facts about Serre duality, which will be used along with \cech cohomology to compute a $k$-basis of $\hone$, in a similar vein to \cite{canonicalrepresentation}.


We begin by recalling some of the details of Serre duality.
We let $K(X)$ denote the function field of $X$, and then let $\Omega_{K(X)}$ be the module of differentials of $K(X)$ over $k$.
The following lemma will give us a particular description of $H^1(X,\Omega_X)$, which we will use later.
\begin{lem}\label{exactsequencelemma}
The following sequence is exact:
\begin{equation}\label{dualityses}
0 \rightarrow \hzero \ra \Omega_{K(X)} \ra \bigoplus_{P \in X}\Omega_{K(X)}/\Omega_{X,P} \xrightarrow{\delta} H^1(X,\Omega_X) \ra 0,
\end{equation}
\end{lem}
\begin{rem}
Note that the sequence formed by the last three terms can be found in \cite[Pg. 248]{hart}.
\end{rem}
\begin{proof}
We let $\underline{\Omega}_{K(X)}$ be the constant sheaf of $\Omega_{K(X)}$.
Then the short exact sequence
\begin{equation}\label{serredualityses}
0 \rightarrow \Omega_X \rightarrow \underline{\Omega}_{K(X)} \rightarrow \underline{\Omega}_{K(X)}/\Omega_X \rightarrow 0
\end{equation}
is a flasque resolution of $\Omega_X$ by \cite[Chap II, ex. 1.16]{hart}.

We view the module $\Omega_{K(X)}/\Omega_{X,P}$ as a sheaf on the singleton $\{P\}$, which has a natural embedding $i\colon \{P\} \rightarrow X$.
Hence for each $P\in X$ we have the induced sheaf $i_*\left(\Omega_{K(X)}/\Omega_{X,P}\right)$ on $X$.
If we consider the direct sum of these induced sheaves over all points $P\in X$ we have the following isomorphism
\begin{equation}\label{sheafisomorphism}
\underline{\Omega}_{K(X)}/\Omega_X\cong \bigoplus_{P\in X} i_*\left(\Omega_{K(X)}/\Omega_{X,P}\right).
\end{equation}
To explain this isomorphism we first construct a map from $\underline{\Omega}_{K(X)}/\Omega_{X,P}$ in to the product $\prod_{P \in X} i_*\left(\Omega_{K(X)}/\Omega_{X,P}\right)$, and then showing that this map has finite support.

Given $i\colon \{P\} \hookrightarrow X$ we have the following equalities
\begin{align*}
i^{-1}\left(\underline{\Omega}_{K(X)}/\Omega_X\right) & = \left(\underline{\Omega}_{K(X)}/\Omega_X\right)_P \\
& = \underline{\Omega}_{K(X),P}/\Omega_{X,P} \\
& = \Omega_{K(X)}/\Omega_{X,P}.
\end{align*}
Moreover, since $i^{-1}$ and the direct image functor $i_*$ are adjoint, we have a map $\left(\underline{\Omega}_{K(X)}/\Omega_X\right) \ra i_* i^{-1} \left( \underline{\Omega}_{K(X)}/\Omega_X \right) = i_* \left( \underline{\Omega}_{K(X)}/\Omega_{X.P} \right)$.\todo{check whether this is an isomorphism just because adjoint} This map is an isomorphism at the level of stalks, and hence is an isomorphism when we sum over all $P \in X$.\todo{finish this}
\end{proof}

For any $P\in X$ the residue map $\res_P \colon \Omega_{K(X)} \ra k$ is defined by the following properties:
\begin{itemize}
\item $\res_P(\omega) = 0$ for all $\omega \in \Omega_{P}$;
\item $\res_P(f^ndf) = 0$ for all $f \in K(X)*$ and all $n \neq 1$;
\item $\res_P(f^{-1}df) = \ord_P(f)$, where $\ord_P$ is the order of the function at $P$.
\end{itemize}
See \cite[Chap III, Thm. 7.14.1]{hart} for details.
Now $\res_P$ is well defined on the quotient $\Omega_{K(X)}/\Omega_P$, since $\Omega_P \subseteq \ker (\res_P)$.
The residue theorem \cite[Chap. III, Thm. 7.14.2]{hart} states that for any $\omega \in \Omega_{K(X)}$ we have $\sum_{P\in X} \res_P(\omega) = 0$.
From this it follows that the map $t \colon \bigoplus_{P \in X} \Omega_K(X)/\Omega_P \ra k$ given by $(\omega_P)_{P \in X} \mapsto \sum_{P\in X} \res_P(\omega_P)$ vanishes on the image of $\Omega_{K(X)}$.
Hence this map is well defined on the quotient, which is $H^1(X,\Omega_X)$, by Lemma \ref{exactsequencelemma}.
Given some $\bar \omega$ in the quotient $H^1(X,\Omega_X)$, which is mapped to by some $\omega \in \bigoplus_{P \in X} \Omega_K(X)/\Omega_P$, we can define the trace map $t$ by
\[
t \colon H^1\left(X, \Omega_X\right) \ra k,\ \bar \omega \mapsto \sum_{P \in X} \res_P(\omega).
\]
The following lemma will make it easier to compute $t( \bar \omega)$.

\begin{lem}
For any $\bar \omega \in H^1(X,\Omega_X)$ we have the following equality
\[
t(\bar \omega) = \sum_{P \in \pi^{-1}(\infty)}\res_P(\omega).
\]
\end{lem}
\begin{proof}
The proof follows from a diagram chase on \eqref{dualitydiagram2}.
Given a cocycle in $\bar \omega \in \hone$ we take a representative $\omega \in \Omega_X(U_0 \cap U_\infty)$.
This then injects in to $\Omega_{K(X)}$, and since $d_2$ is surjective we can choose an element of $\Omega_{K(X)} \times \Omega_{K(X)}$ mapping to $\omega$.
In particular, we could choose $(\omega,0)$.
This then maps to $\psi = ((\bar{\omega}|_P)_{P\in U_0}, 0) \in \left( \bigoplus_{P \in U_0} \Omega_{K(X)}/\Omega_{X,P}\right) \times \left( \bigoplus_{P \in U_\infty} \Omega_{K(X)}/\Omega_{X,P} \right)$.
Moreover, by commutativity of the diagram $\psi \in \ker(d_3)$.
Since $\omega$ is regular on $U_0 \cap U_\infty$, and $0$ is regular on $U_\infty$, then $\res_P(\psi) = 0$ for all $P \in U_\infty$.
Hence $t(\bar \omega) = \sum_{P \in X}\res_P(\bar \omega) = \sum_{P \in \pi^{-1}(\infty)} \res_P(\bar \omega) = \sum_{P \in \pi^{-1}(\infty)} \res_P(\omega)$.
\end{proof}
\todo[inline]{proof doesn't make sense, since cech cohomology is not covered yet}






\bibliography{biblio}
\bibliographystyle{amsalpha}
\end{document}
