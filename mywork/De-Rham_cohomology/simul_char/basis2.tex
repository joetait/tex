% !TEX TS-program = pdflatex
% !TEX encoding = UTF-8 Unicode

% This is a simple template for a LaTeX document using the "article" class.
% See "book", "report", "letter" for other types of document.

\documentclass[draft, 11pt]{article} % use larger type; default would be 10pt

\usepackage[utf8]{inputenc} % set input encoding (not needed with XeLaTeX)

%%% Examples of Article customizations
% These packages are optional, depending whether you want the features they provide.
% See the LaTeX Companion or other references for full information.

%%% PAGE DIMENSIONS
\usepackage{geometry} % to change the page dimensions
\geometry{a4paper} % or letterpaper (US) or a5paper or....
% \geometry{landscape} % set up the page for landscape
% read geometry.pdf for detailed page layout information

\usepackage{graphicx} % support the \includegraphics command and options
\usepackage[obeyDraft]{todonotes}

%\usepackage[parfill]{parskip} % Activate to begin paragraphs with an empty line rather than an indent

%%% PACKAGES
\usepackage[all]{xy}
\usepackage{mathtools}
\usepackage{booktabs} % for much better looking tables
\usepackage{array} % for better arrays (eg matrices) in maths
\usepackage{paralist} % very flexible & customisable lists (eg. enumerate/itemize, etc.)
\usepackage{verbatim} % adds environment for commenting out blocks of text & for better verbatim
\usepackage{subfig} % make it possible to include more than one captioned figure/table in a single float
%\usepackage{hyperref}
% These packages are all incorporated in the memoir class to one degree or another...

%\usepackage[activate={true,nocompatibility},final,tracking=true,kerning=true,spacing=true,factor=1100,stretch=10,shrink=10]{microtype}
%\microtypecontext{spacing=nonfrench}
% activate={true,nocompatibility} - activate protrusion and expansion
% final - enable microtype; use "draft" to disable
% tracking=true, kerning=true, spacing=true - activate these techniques
% factor=1100 - add 10% to the protrusion amount (default is 1000)
% stretch=10, shrink=10 - reduce stretchability/shrinkability (default is 20/20)

%%% HEADERS & FOOTERS
\usepackage{fancyhdr} % This should be set AFTER setting up the page geometry
\pagestyle{fancy} % options: empty , plain , fancy
\renewcommand{\headrulewidth}{0pt} % customise the layout...
\lhead{}\chead{}\rhead{}
\lfoot{}\cfoot{\thepage}\rfoot{}

%%% SECTION TITLE APPEARANCE
\usepackage{sectsty}
\allsectionsfont{\sffamily\mdseries\upshape} % (See the fntguide.pdf for font help)
\usepackage{amsmath}
\usepackage{amsthm}
\usepackage{amsfonts}
\usepackage{mathrsfs}
\usepackage{amsopn}
\usepackage{amssymb}
\usepackage{etex}
%\usepackage{natbib}
% (This matches ConTeXt defaults)

%%% ToC (table of contents) APPEARANCE
\usepackage[nottoc,notlof,notlot]{tocbibind} % Put the bibliography in the ToC
\usepackage[titles,subfigure]{tocloft} % Alter the style of the Table of Contents
%\renewcommand{\cftsecfont}{\rmfamily\mdseries\upshape}
%\renewcommand{\cftsecpagefont}{\rmfamily\mdseries\upshape} % No bold!
%\renewcommand{\familydefault}{\sfdefault}
%\usepackage{cabin}
%\usepackage{libertine}
%\usepackage[T1]{fontenc}

%Theorems and stuff
\theoremstyle{plain}
\newtheorem{defn}{Definition}[section]
\newtheorem{thm}[defn]{Theorem}
\newtheorem{cor}[defn]{Corollary}
\newtheorem{lem}[defn]{Lemma}
\newtheorem{prop}[defn]{Proposition}
\newtheorem{ex}[defn]{Example}
\newtheorem*{unnumthm}{Theorem}
\newtheorem{defnlem}[defn]{Definition/Lemma}
\newtheorem{defnthm}[defn]{Theorem/Definition}
\theoremstyle{remark}
\newtheorem*{rem}{Remark}
\newtheorem*{note}{Note}


\newcommand{\cO}{{\cal O}}
\newcommand{\ra}{\rightarrow}
\newcommand{\NN}{{\mathbb N}}
\newcommand{\PP}{{\mathbb P}}
\newcommand{\ZZ}{{\mathbb Z}}
\newcommand{\cL}{{\mathcal L}}
\newcommand{\cA}{{\mathcal A}}
\newcommand{\cD}{{\mathcal D}}
\newcommand{\cU}{{\mathcal U}}
\newcommand{\cech}{\v{C}ech }
\newcommand{\hzero}{{H^0(X,\Omega_X)}}
\newcommand{\hone}{H^1(X,\mathcal{O}_X)}
\newcommand{\cechhone}{\check{H}^1(\mathcal U,\mathcal O_X)}
\newcommand{\derhamhone}{H_{\text {dR}}^1(X/k)}
\newcommand{\cechhzero}{{\check{H}^0(X,\Omega_X)}}
\newcommand{\ubar}{\underset{\bar{}}}


\DeclareMathOperator{\aut}{Aut}
\DeclareMathOperator{\res}{Res}
\DeclareMathOperator{\ord}{ord}
\DeclareMathOperator{\di}{div}
\DeclareMathOperator{\cha}{char}
\DeclareMathOperator{\gal}{Gal}
\DeclareMathOperator{\Tr}{Tr}
\DeclareMathOperator{\Ima}{Im}

%%% END Article customizations

%%% The "real" document content comes below...

\title{Group actions on de Rham cohomology of hyperelliptic curves}
\author{}
%\date{} % Activate to display a given date or no date (if empty),
         % otherwise the current date is printed

\begin{document}
\maketitle

\listoftodos

\todo[inline]{change references to preprint when in thesis}
Let $X$ be a smooth, projective, connected hyperelliptic curve of genus $g \geq 2$ over an algebraically closed field $k$ of characteristic $p \geq 0$.
We  fix a map $\pi \colon X \rightarrow \mathbb P_k^1$ of degree two, which is unique up to an automorphism of $\mathbb P_k^1$.
We let $G$ denote a finite group acting on $X$.

\section{Serre duality}


We begin by recalling some of the details of Serre duality.
This will be used along with \cech cohomology to compute a $k$-basis of $\hone$, in a similar vein to \cite{canonicalrepresentation}.
We let $K(X)$ denote the function field of $X$, and then let $\Omega_{K(X)}$ be the module of differentials of $K(X)$ over $k$.
The following lemma gives us a useful description of $H^1(X,\Omega_X)$.
\begin{lem}\label{exactsequencelemma}
The following sequence is exact:
\begin{equation}\label{dualityses}
0 \rightarrow \hzero \ra \Omega_{K(X)} \ra \bigoplus_{P \in X}\Omega_{K(X)}/\Omega_{X,P} \xrightarrow{\delta} H^1(X,\Omega_X) \ra 0,
\end{equation}
\end{lem}
\begin{rem}
Note that the sequence formed by the last three terms can be found in \cite[Pg. 248]{hart}.
\end{rem}
\begin{proof}
We let $\underline{\Omega}_{K(X)}$ be the constant sheaf of $\Omega_{K(X)}$.
Then the short exact sequence
\begin{equation}\label{serredualityses}
0 \rightarrow \Omega_X \rightarrow \underline{\Omega}_{K(X)} \rightarrow \underline{\Omega}_{K(X)}/\Omega_X \rightarrow 0
\end{equation}
is a flasque resolution of $\Omega_X$ by \cite[Chap II, ex. 1.16]{hart}.

We view the module $\Omega_{K(X)}/\Omega_{X,P}$ as a sheaf on the singleton $\{P\}$, which has a natural embedding $i\colon \{P\} \rightarrow X$.
Hence for each $P\in X$ we have the induced sheaf $i_*\left(\Omega_{K(X)}/\Omega_{X,P}\right)$ on $X$.
If we consider the direct sum of these induced sheaves over all points $P\in X$ we have the following isomorphism
\begin{equation}\label{sheafisomorphism}
\underline{\Omega}_{K(X)}/\Omega_X\cong \bigoplus_{P\in X} i_*\left(\Omega_{K(X)}/\Omega_{X,P}\right).
\end{equation}
Given this isomorphism, we can take the cohomology of \eqref{serredualityses} and we get the exact sequence in the statement of the lemma.


To explain the isomorphism in \eqref{sheafisomorphism} we first construct a map from $\underline{\Omega}_{K(X)}/\Omega_{X,P}$ in to the product $\prod_{P \in X} i_*\left(\Omega_{K(X)}/\Omega_{X,P}\right)$, and then showing that this map has finite support.

Given $i\colon \{P\} \hookrightarrow X$ we have the following equalities
\begin{align*}
i^{-1}\left(\underline{\Omega}_{K(X)}/\Omega_X\right) & = \left(\underline{\Omega}_{K(X)}/\Omega_X\right)_P \\
& = \underline{\Omega}_{K(X),P}/\Omega_{X,P} \\
& = \Omega_{K(X)}/\Omega_{X,P}.
\end{align*}
It follows that for each $P \in X$ we can map $f \in \underline{\Omega}_{K(X)}/\Omega_X$ to $i_*i^{-1}(f) \in \left( \Omega_{K(X)}/\Omega_{X,P} \right)$.
Recall that for any $f \in \Omega_{K(X)}$then $f$ lies in $\Omega_{X,P}$ for all but a finite number of points $P \in X$.
Hence the image of $f$ in $\prod_{P \in X} i_*\left(\Omega_{K(X)}/\Omega_{X,P}\right)$ is zero in almost all factors.
From this the isomorphism in \eqref{sheafisomorphism} follows.

Replacing $\underline{\Omega}_{K(X)}/\Omega_X$ by $\bigoplus_{P\in X} i_*\left(\Omega_{K(X)}/\Omega_{X,P}\right)$ in \eqref{serredualityses} and taking cohomology then yields the exact sequence in the statement of the lemma.\todo{repeated line earlier}
\end{proof}

For any $P\in X$ the residue map $\res_P \colon \Omega_{K(X)} \ra k$ is defined by the following properties:
\begin{itemize}
\item $\res_P(\omega) = 0$ for all $\omega \in \Omega_{P}$;
\item $\res_P(f^ndf) = 0$ for all $f \in K(X)^*$ and all $n \neq 1$;
\item $\res_P(f^{-1}df) = \ord_P(f)$, where $\ord_P(f)$ is the order of $f$ at $P$.
\end{itemize}
See \cite[Chap III, Thm. 7.14.1]{hart} for details.



Now since $\Omega_P \subseteq \ker (\res_P)$, it follows that $\res_P$ is well defined on the quotient $\Omega_{K(X)}/\Omega_P$.
Moreover, the residue theorem \cite[Chap. III, Thm. 7.14.2]{hart} states that for any $\omega \in \Omega_{K(X)}$ we have $\sum_{P\in X} \res_P(\omega) = 0$, and so the map $t \colon \bigoplus_{P \in X} \Omega_K(X)/\Omega_P \ra k$ given by $(\omega_P)_{P \in X} \mapsto \sum_{P\in X} \res_P(\omega_P)$ vanishes on the image of $\Omega_{K(X)}$.
It follows that given some $\bar \omega$ in $H^1(X,\Omega_X)$, which is mapped to by some $(\omega_P)_{P \in X} \in \bigoplus_{P \in X} \Omega_K(X)/\Omega_P$, we can define the trace map $t$:
\[
t \colon H^1\left(X, \Omega_X\right) \ra k,\ \bar \omega \mapsto \sum_{P \in X} \res_P(\omega_P).
\]


We now use the trace map to define a pairing between the $k$-vector spaces $\hone$ and $\hzero$.
Roughly speaking, we can replace $\Omega_X$ by $\cO_X$ in Lemma \ref{exactsequencelemma} and get a similar result.
Hence $H^1(X,\cO_X)$ can also be viewed as a quotient of a direct sum over all $P \in X$, and using this we define a canonical map 
\begin{equation}\label{productmap}
\hzero \times \hone \ra H^1\left(X, \Omega_X\right), \ (\omega, \overline{(f_P)}_{P \in X}) \mapsto \overline {((f  \omega)_P)}_{P \in X}.
\end{equation}\todo{sort out notation}
The product here is given as follows: since $\omega$ is a global differential and $f$ is the quotient of some $(f_P)_{P \in X}~\in~\bigoplus_{P \in X} i_*\left( \cO_X/\Omega_{X,P} \right)$, at any point $P$ we can take $f_P \omega_P$, the germ of $f\omega$.
We then define $(f\omega)_P :=f_P\omega_P$.

We now combine the product map in \eqref{productmap} with the trace map $t$ to get a map 
\[
 \hzero \times \hone \ra k,\quad (\omega, f) \to \langle \omega, f \rangle := t \left( f \omega |_{U_0 \cap U_\infty}\right).
\]

\begin{thm}\label{serredualitytheorem}
Via the pairing $\langle , \rangle$, the $k$-vector spaces $\hone$ and $\hzero$ are dual to each other.
\end{thm}
\begin{proof}
This is in fact a specialisation of \cite[Thm. 2, Chap. II]{algebraicgroupsandclassfields}.
\end{proof}

If we fix any $\omega \in \hzero$ we produce a map $\theta(\omega)\colon \hone \ra k$, given by $\theta(\omega)(f) = \langle \omega , f\rangle$.
Similarly, if we fix any $f \in \hone$ then we get a map $\psi(f) \colon \hzero \ra k$.
The maps $\psi$ and $\theta$ are isomorphisms and are dual to each other: in particular, this means that given a basis $e_1, \ldots, e_n$ of $\hzero$, we can find a basis $f_1, \ldots , f_n$ of $\hone$ such that $\theta(e_i)(f_i) = 1$ for all $1 \leq i \leq n$ and $\theta(e_i)(f_j) = 0$ if $i \neq j$ (and similarly for $\psi$).


\section{\cech cohomology}
In this section we will describe $\hone$ and $\hzero$ in terms of \cech cohomology.
This will give us a concrete description, which we can then use in the next section to compute an explicit basis of $\derhamhone$.

By Leray's theorem \cite[Thm 5.2.12]{liu} and Serre's affineness criterion \cite[Thm 5.2.23]{liu} we know that the first \cech cohomology group of $\cO_X$ and $\hone$ will be isomorphic if the cover we use to compute the \cech cohomology is affine.
We let $U_a = X \backslash \pi^{-1}(a)$ for any $a \in \mathbb P_k^1$ and we let ${\cal U}$ be the affine cover of $X$ formed by $U_0$ and $U_\infty$.
Given any sheaf $\cal F$ on $X$ we have the \cech differential $\check{d}\colon {\cal F}(U_0) \times {\cal F} (U_\infty) \rightarrow {\cal F}(U_0 \cap U_\infty)$, defined by $(f_0,f_\infty) \mapsto f_0|_{U_0 \cap U_\infty} - f_\infty|_{U_0 \cap U_\infty}$.
In general we will suppress the notation denoting the restriction map.
Via this differential we have the following cochain complex
\begin{equation*}
0 \rightarrow \cO_X(U_0)\times \cO_X(U_\infty) \xrightarrow{\check{d}} \cO_X(U_0 \cap U_\infty) \rightarrow 0.
\end{equation*}
The first cohomology group of this chain is $\cechhone = \frac{\cO_X(U_0 \cap U_\infty)}{\Ima(\check{d})}$ and hence
\begin{equation}\label{cechhone}
\hone \cong \frac{\cO_X(U_0 \cap U_\infty)}{\Ima(\check{d})}  
 = \frac{\cO_X(U_0 \cap U_\infty)}{\{f_0 - f_\infty | f_i \in \cO_X(U_i) \}}.
\end{equation}

If we replace $\cO_X$ by $\Omega_X$ in the previous paragraph then everything still holds, and we can conclude that
\[
H^1(X,\Omega_X) \cong \frac{\Omega_X(U_0 \cap U_\infty)}{\Ima(\check{d})} = \frac{\Omega_X(U_0 \cap U_\infty)}{\{\omega_0 - \omega_ \infty | \omega_i \in \Omega_X(U_i)\}}.
\]

We now describe the trace map using the \cech cohomology representative of $\hone$ above.
\begin{lem}\label{tracemaplemma}
For any $\bar \omega \in \cechhone$ we have the following equality:
\[
t(\bar \omega) = \sum_{P \in \pi^{-1}(\infty)}\res_P(\omega).
\]
\end{lem}
\begin{proof}
We take the \cech complex of \eqref{serredualityses} over the cover $\cU$, yielding the following bicomplex
\begin{equation}\label{dualitydiagram2}
\xymatrix{\Omega_X(U_0)\times\Omega_X(U_\infty) \ar@{^{(}->}[r] \ar[d]^{d_1} & \underline{\Omega}_{K(X)} \times \underline{\Omega}_{K(X)} \ar[d]^{d_2} \ar@{->>}[r] & \bigoplus \limits_{P \in U_0} \Omega_{K(X)}/\Omega_{X,P} \times \bigoplus \limits_{P \in U_\infty} \Omega_{K(X)}/\Omega_{X,P} \ar[d]^{d_3} \\
\Omega_X(U_0 \cap U_\infty) \ar@{^{(}->}[r]  & \underline{\Omega}_{K(X)} \ar@{->>}[r] & \bigoplus \limits_{P\in U_0 \cap U_\infty} \Omega_{K(X)}/\Omega_{X,P} }
\end{equation}
We can now apply the snake lemma to this commutative diagram.
In the next paragraph we conclude the proof by showing that the exact sequence that arises from the applying snake lemma is precisely the sequence \eqref{dualityses} in the statement of the lemma.\todo{check underlining}

The fact that $H^0(X,\Omega_X) \cong \ker(d_1)$ and $H^1(X,\Omega_X) \cong {\rm coker}(d_1)$ follows from the above discussion of \cech cohomology.
The map $d_2\colon \Omega_{K(X)} \times \Omega_{K(X)} \ra \Omega_{K(X)}$ is the \cech differential given by $(\omega_1,\omega_2) \mapsto \omega_1 - \omega_2$.
Hence the map $\omega \mapsto (\omega, \omega)$ gives an isomorphism from $\Omega_{K(X)}$ to $\ker(d_2)$.
Finally, $d_3$ is defined by $(\omega_0, \omega_\infty) \mapsto \omega_0|_{U_0 \cap U_\infty} - \omega_\infty|_{U_0 \cap U_\infty}$.
Hence the kernel of $d_3$ is formed of pairs $(\omega_0, \omega_\infty) \in \bigoplus_{P \in U_0} \Omega_{K(X)}/\Omega_{X,P} \times \bigoplus_{P \in  U_\infty} \Omega_{K(X)}/\Omega_{X,P}$ such that $\omega_0$ and $\omega_\infty$ agree on $U_0 \cap U_\infty$.
It follows that the map $\bigoplus_{P \in X} \Omega_{K(X)}/\Omega_{X,P}\ra\ker(d_3)  $ given by 
\begin{equation*}
(\omega_P)_{P \in X} \to \left( (\omega_P)_{ P \in U_0}, (\omega_P)_{P \in U_\infty}) \right)
\end{equation*}
is an isomorphism.

The proof follows from a diagram chase on \eqref{dualitydiagram2}.
Given a cocycle in $\bar \omega \in \hone$ we take a representative $\omega \in \Omega_X(U_0 \cap U_\infty)$.
This then injects in to $\underline{\Omega}_{K(X)}$, and since $d_2$ is surjective we can choose an element of $\underline{\Omega}_{K(X)} \times \underline{\Omega}_{K(X)}$ mapping to $\omega$.
In particular, we could choose $(\omega,0)$.
This then maps to 
\[
\psi = ((\bar{\omega}|_P)_{P\in U_0}, 0) \in \left( \bigoplus_{P \in U_0} \Omega_{K(X)}/\Omega_{X,P}\right) \times \left( \bigoplus_{P \in U_\infty} \Omega_{K(X)}/\Omega_{X,P} \right).
\]
Moreover, by commutativity of the diagram $\psi \in \ker(d_3)$.
Since $\omega$ is regular on $U_0 \cap U_\infty$, and $0$ is regular on $U_\infty$, then $\res_P(\psi) = 0$ for all $P \in U_\infty$.
Hence $t(\bar \omega) = \sum_{P \in X}\res_P(\bar \omega) = \sum_{P \in \pi^{-1}(\infty)} \res_P(\bar \omega) = \sum_{P \in \pi^{-1}(\infty)} \res_P(\omega)$.
\end{proof}


































\bibliography{biblio}
\bibliographystyle{amsalpha}
\end{document}
