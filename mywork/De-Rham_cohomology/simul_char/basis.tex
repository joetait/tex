% !TEX TS-program = pdflatex
% !TEX encoding = UTF-8 Unicode

% This is a simple template for a LaTeX document using the "article" class.
% See "book", "report", "letter" for other types of document.

\documentclass[draft, 11pt]{article} % use larger type; default would be 10pt

\usepackage[utf8]{inputenc} % set input encoding (not needed with XeLaTeX)

%%% Examples of Article customizations
% These packages are optional, depending whether you want the features they provide.
% See the LaTeX Companion or other references for full information.

%%% PAGE DIMENSIONS
\usepackage{geometry} % to change the page dimensions
\geometry{a4paper} % or letterpaper (US) or a5paper or....
% \geometry{landscape} % set up the page for landscape
% read geometry.pdf for detailed page layout information

\usepackage{graphicx} % support the \includegraphics command and options
\usepackage[obeyDraft]{todonotes}

%\usepackage[parfill]{parskip} % Activate to begin paragraphs with an empty line rather than an indent

%%% PACKAGES
\usepackage{mathtools}
\usepackage{booktabs} % for much better looking tables
\usepackage{array} % for better arrays (eg matrices) in maths
\usepackage{paralist} % very flexible & customisable lists (eg. enumerate/itemize, etc.)
\usepackage{verbatim} % adds environment for commenting out blocks of text & for better verbatim
\usepackage{subfig} % make it possible to include more than one captioned figure/table in a single float
% These packages are all incorporated in the memoir class to one degree or another...

\usepackage[activate={true,nocompatibility},final,tracking=true,kerning=true,spacing=true,factor=1100,stretch=10,shrink=10]{microtype}
\microtypecontext{spacing=nonfrench}
% activate={true,nocompatibility} - activate protrusion and expansion
% final - enable microtype; use "draft" to disable
% tracking=true, kerning=true, spacing=true - activate these techniques
% factor=1100 - add 10% to the protrusion amount (default is 1000)
% stretch=10, shrink=10 - reduce stretchability/shrinkability (default is 20/20)

%%% HEADERS & FOOTERS
\usepackage{fancyhdr} % This should be set AFTER setting up the page geometry
\pagestyle{fancy} % options: empty , plain , fancy
\renewcommand{\headrulewidth}{0pt} % customise the layout...
\lhead{}\chead{}\rhead{}
\lfoot{}\cfoot{\thepage}\rfoot{}

%%% SECTION TITLE APPEARANCE
\usepackage{sectsty}
\allsectionsfont{\sffamily\mdseries\upshape} % (See the fntguide.pdf for font help)
\usepackage{amsmath}
\usepackage{amsthm}
\usepackage{amsfonts}
\usepackage{mathrsfs}
\usepackage{amsopn}
\usepackage{amssymb}
\usepackage{natbib}
% (This matches ConTeXt defaults)

%%% ToC (table of contents) APPEARANCE
\usepackage[nottoc,notlof,notlot]{tocbibind} % Put the bibliography in the ToC
\usepackage[titles,subfigure]{tocloft} % Alter the style of the Table of Contents
\renewcommand{\cftsecfont}{\rmfamily\mdseries\upshape}
\renewcommand{\cftsecpagefont}{\rmfamily\mdseries\upshape} % No bold!
%\renewcommand{\familydefault}{\sfdefault}
%\usepackage{cabin}
\usepackage{libertine}
%\usepackage[T1]{fontenc}

%Theorems and stuff
\theoremstyle{plain}
\newtheorem{defn}{Definition}[section]
\newtheorem{thm}[defn]{Theorem}
\newtheorem{cor}[defn]{Corollary}
\newtheorem{lem}[defn]{Lemma}
\newtheorem{prop}[defn]{Proposition}
\newtheorem{ex}[defn]{Example}
\newtheorem*{unnumthm}{Theorem}
\newtheorem{defnlem}[defn]{Definition/Lemma}
\newtheorem{defnthm}[defn]{Theorem/Definition}
\theoremstyle{remark}
\newtheorem*{rem}{Remark}
\newtheorem*{note}{Note}


\newcommand{\cO}{{\cal O}}
\newcommand{\ra}{\rightarrow}
\newcommand{\NN}{{\mathbb N}}
\newcommand{\PP}{{\mathbb P}}
\newcommand{\ZZ}{{\mathbb Z}}
\newcommand{\cL}{{\mathcal L}}
\newcommand{\cA}{{\mathcal A}}
\newcommand{\cD}{{\mathcal D}}
\newcommand{\cU}{{\mathcal U}}
\newcommand{\cech}{\v{C}ech }
\newcommand{\hzero}{{H^0(X,\Omega_X)}}
\newcommand{\hone}{H^1(X,\mathcal{O}_X)}
\newcommand{\cechhone}{\check{H}^1(\mathcal U,\mathcal O_X)}
\newcommand{\derhamhone}{H_{\text {dR}}^1(X/k)}


\DeclareMathOperator{\aut}{Aut}
\DeclareMathOperator{\res}{Res}
\DeclareMathOperator{\ord}{ord}
\DeclareMathOperator{\di}{div}
\DeclareMathOperator{\cha}{char}
\DeclareMathOperator{\gal}{Gal}
\DeclareMathOperator{\Tr}{Tr}
\DeclareMathOperator{\Ima}{Im}

%%% END Article customizations

%%% The "real" document content comes below...

\title{Group actions on de-Rham cohomology of hyperelliptic curves}
\author{}
%\date{} % Activate to display a given date or no date (if empty),
         % otherwise the current date is printed

\begin{document}
\maketitle

\section{Background}

Let $X$ be a smooth, projective, connected hyperelliptic curve of genus $g \geq 2$ over an algebraically closed field $k$ of characteristic $p \geq 0$.
We let $\pi \colon X \rightarrow \mathbb P_k^1$ be the corresponding degree two map.
We also suppose that we have a finite group $G$ acting fatihfully on $X$.

We will be using \cech cohomology to compute both $\hone$ and the de-Rham hypercohomology.
We therefore recall the \cech cohomology relevant to both of these, starting with $\hone$.

By Leray's theorem \cite[Thm 5.2.12]{liu} and Serre's affineness criterion \cite[Thm 5.2.23]{liu} we know that the first \cech cohomology group and $\hone$ will be isomorphic if the cover we use to compute the \cech cohomology is affine.
The cover we shall be using is $U_1 = X\backslash \pi^{-1}(0)$ and $U_2 = X\backslash  \pi^{-1}(\infty) $, both of which are affine sets.
We let ${\cal U} = \{ U_1, U_2\}$.
Given any sheaf $\cal F$ on $X$ we have a canonical map ${\cal F}(U_0) \times {\cal F} (U_1) \rightarrow {\cal F}(U_0 \cap U_1)$, given by taking the difference, $(f_0,f_1) \mapsto f_0|_{U_0 \cap U_1} - f_1|_{U_0 \cap U_1}$.
In general we will not explicitly show that we are considering the restriction when applying this map, unless it adds clarity.
It is this map which gives us the differential $d$ in the \cech cochain
\begin{equation*}
0 \rightarrow \cO_X(U_1)\times \cO_X(U_2) \xrightarrow{d} \cO_X(U_1 \cap U_2) \rightarrow 0.
\end{equation*}
The first cohomology group of this chain is $\cechhone = \frac{\cO_X(U_1 \cap U_2)}{\Ima(d)}$ and hence
\begin{equation}
\hone \cong \frac{\cO_X(U_1 \cap U_2)}{\Ima(d)}  
 = \frac{\cO_X(U_1 \cap U_2)}{\{f_1 - f_2 | f_i \in \cO_X(U_i) \}}.
\end{equation}
We will use this characterisation of $\hone$ later.

We now recall how we compute the algebraic de-Rham cohomology of $X$ via \cech cohomology.
The de-Rham cohomology of an algebraic curve is the hypercohomology of the de Rham complex on $X$. 
Since a curve doesn't have differentials of degree greater than one, this is just the hypercohomology of the complex
\begin{equation}\label{res}
0 \rightarrow \cO_X \rightarrow \Omega_X \rightarrow 0.
\end{equation}

We then use the open cover $\cal U$ and the \cech differentials as defined earlier to give us the \cech complex of \eqref{res}, which is
\begin{equation}\label{bicomplex} \begin{array}{ccccccc}
~ & ~ & 0 & ~ & 0 & ~ & ~ \\
~ & ~ & \downarrow & ~ & \downarrow & ~ & ~ \\
0 & \rightarrow & \cO_X(U_0)\times \cO_X(U_1) & \rightarrow & \Omega_X(U_1)\times \Omega_X(U_2) & \rightarrow & 0 \\
~ & ~ & \downarrow & ~ & \downarrow & ~ & ~ \\
0 & \rightarrow & \cO_X(U_1 \cap U_2) & \rightarrow & \Omega_X(U_1 \cap U_2) & \rightarrow & 0 \\
~ & ~ & \downarrow & ~ & \downarrow & ~ & ~ \\
~ & ~ & 0 & ~ & 0 & ~ & ~ \\
\end{array}
\end{equation}
where the horizontal arrows are the standard differential map $d$.
By \cite[Cor. 12.4.7]{EGA0III}, a generalisation of Leray's theorem, we know that the $\derhamhone$ is isomorphic to the first cohomology of the total complex of \eqref{bicomplex}.
Note that this relies on $\cechhone$ and ${\check H}^p(\cU, \Omega_X)$ being zero for all $p\geq 1$ ---
since $U_1$ and $U_2$ are affine, this follows from Serre's affineness criterion \cite[Thm 5.2.23]{liu}.



After computing the first cohomology group of the total complex \eqref{bicomplex} we see that $H^1_{{\text dR}}(X/k)$ is isomorphic to the space
\begin{equation}\label{derhamconditions}
\left\{(\omega_1, \omega_2, f_{12}) | \omega_i\in \Omega_{X/k}(U_i), f_{12}\in \cO_X(U_1 \cap U_2), df_{12} = \omega_1|_{U_1\cap U_2} - \omega_2|_{U_1\cap U_2} \right\}
\end{equation}
quotiented by the subspace
\[
\left\{  (df_1, df_2, f_1|_{U_1\cap U_2} -f_2|_{U_1\cap U_2} )|f_i \in \cO_X(U_i)\right\}.
\]

We wish to compute a $k$-basis of $\derhamhone$ in order that we can see how $G$ acts on $\derhamhone$.
To do this we will compute $k$-bases for the two outer terms of the short exact sequence
\begin{equation}\label{ses}
0 \ra H^0(X,\Omega_X) \ra \derhamhone \ra H^1(X,\cO_X) \ra 0.
\end{equation}
A basis of $H^0(X,\Omega_X)$ is already known in the literature (see, for example, \cite[Prop. 7.4.26]{liu}) so we need only compute a basis of $H^1(X,\cO_X)$.

\begin{comment}
In order to do this we will need to use Serre duality, which we briefly remind the reader of presently.
Serre duality states that $H^0(X,\Omega_X)$ and $H^1(X,\cO_X)$ are dual to each other, and that there is a canonical map ${\text Res}:H^0(X,\omega_X) \times H^1(X,\cO_X) \rightarrow k$.
Details of the definition of ${\text Res}$ can be found in appendix B of \todo{in progress}
\todo[inline]{add details - in particular, relate to the basis we have for de rham}
\end{comment}
\section{Basis of $\derhamhone$}

The bases of the $k$-vector spaces in \eqref{ses} will be given in terms of defining equations, so we describe these now.

If $p \neq 2$ then the extension $K(X)$ of $k(x)$ will be $k(x,y)$ where $y$ satisfies
\begin{equation}\label{definingequationpnot2}
y^2 = f(x)
\end{equation}
for some polynomial $f(x) \in k[x]$, of degree $2g+1$ or $2g+2$ \cite[Prop 7.4.24]{liu}.

For any $a\in \mathbb P_k^1$ we let $D_a$ denote the divisor $\pi^*([a])$.
Hence we have
\begin{equation}\label{divisorofpoints}
D_a= 
\begin{cases}
 [P_a] & \text{if $a$ is a branch point;} \\
 [P_a] + [P_a'] & \text{otherwise.}
\end{cases}
\end{equation}
We then let $R$ be the ramification divisor of $\pi$, and recall that 
\begin{equation}\label{pnot2divisors}
\di(y)  = R - (g+1)D_\infty \text{ and that } \di( dx) = R - 2D_\infty.
\end{equation}


On the other hand, if $p=2$, then the extension $K(X)$ will be $k(x,y)$, this time with $y$ defined by
\begin{equation}\label{definep=2}
y^2 - H(x)y = F(x),
\end{equation}
for some $H(x),F(x) \in k[x]$, such that $H(x)$ and $H'(x)^2F(x) + F'(x)^2$ share no roots.
Here we require that $\deg(H(x)) \leq g+1$, with equality if and only $\infty$ is not a branch point, and $\deg(F(x)) \leq 2g+2$, and in particular $\deg(F(x)) = 2g+1$ if $\infty$ is a branch point  \cite[Prop 7.4.24]{liu}.



Again, we let $R$ denote the ramification divisor, which we write explicilty below.
In what follows it will be clear from context which characteristic we are in whenever referring to $R$, and hence which ramification divisor is being considered at any time.
We now recall the following divisors, which we will make use of throughout this article:
\begin{align}
\di (dx) & = R - 2D_\infty \label{divdxp=2}\\
\di (H(x)) & = R - (g+1)D_\infty \label{divhp=2}\\
\di (x) & = D_0 - D_\infty \label{divxp=2}\\
\end{align}
We also use the divisor of $y$, which depends on whether or not $\infty$ is a branch point, as follows:
\begin{equation}\label{divyp=2}
\di(y) = \begin{cases}
 \di_0(y) -(2g+1)[P_\infty] & \text{if $\infty$ is a branch point} \\
 \di_0(y) +(g+1-\deg(F(x)))[P_\infty] - (g+1)[P_\infty'] & \text{otherwise.}
\end{cases}
\end{equation}
Here $\di_0(y)$ denotes the divisor of zeroes of the function $y$.


To state the ramification points, and for later use, we suppose that 
\begin{equation}\label{hcoefficients}
H(x) = \prod_{i=1}^d (x-a_i)^{n_i} = x^d + b_{d-1}x^{d-1} + \ldots + b_1x
\end{equation}
for some $a_i, b_i \in  k$, $d \leq g+1$ and $n_i \in \mathbb N$.
Then the $a_i \in \mathbb P_k^1$ are the branch points of $\pi$ and we let $P_i \in X$ be the corresponding ramification points of $\pi$.
Given this, we can now explicitly describe the ramification divisor as
\[
R = \sum_{i=1}^d 2n_i[P_i] + (g+1-d)D_\infty.
\]
Details can be found in \cite[\S 6]{faithfulaction}.

We now compute the basis of $H^1(X,\cO_X)$, and then state the basis of $\derhamhone$ in the following theorem.\\
\todo[inline]{not edited, since may use different proof}
\begin{lem}\label{basish1}
 
 The elements $\frac{y}{x}, \ldots , \frac{y}{x^g}$ form a basis of $H^1(X,\cO_X)$.
 
\end{lem}
\begin{proof}
 
We prove this directly by \cech cohomology.
By Leray's theorem we know that $H^1(X,\cO_X) \cong \check{H}^1(\cU,\cO_X)$.
The \cech complex is as follows:
\begin{equation*}
0 \rightarrow \cO_X(U_1) \times \cO_X(U_2) \xrightarrow{d_0} \cO_X(U_1 \cap U_2) \xrightarrow{d_1} 0.
\end{equation*}

So we wish to show that when $1 \leq i \leq g$ the elements $x^{-i}y$ form a basis of 
\begin{equation*}
\frac{\ker (d_1)}{\Ima (d_0)} = \frac{ \cO_X(U_1 \cap U_2)}{\Ima (d_1)}.
\end{equation*}


We first compute the divisor of $x^{-i}y$ in order to check that it is indeed regular on $U_1 \cap U_2$.

When $p \neq 2$ then
\begin{align*}
    \di(x^{-i}y) & = \di(y) - \di(x^i) \\
    & = R - (g+1)D_\infty - iD_0 + iD_\infty \\
    & = (R - D_0 - D_\infty) -(2i-1)D_0 - (2g +1 -2i)D_\infty.
\end{align*}

On the other hand, if $p =2$ then 
\begin{align*}
    \di(x^{-i}y) & = \di(y) - \di(x^i) \\
    & = \di_0(y) - (2g+1)D_0 -iD_0 + iD_\infty \\
    & = \di_0(y) -iD_0 - ( 2g + 1 -2i)D_\infty.
\end{align*}
So we see that, regardless of characteristic, the only poles are at $P_0$ and $P_\infty$, and hence $x^{-i}y \in \cO_X(U_1 \cap U_2)$.

It remains to show that the elements are not in the image of $d_0$ and hence are non-zero in the first cohomology group.

We  fix $i \in \{1, \ldots, g\}$ and suppose that there exists $f_1 \in \cO_X(U_1)$ and $f_2 \in \cO_X(U_2)$ such that their difference is $x^{-i}y$.


If $p \neq 2$ then $f_1$ necessarily has a pole of order $1-2i$ at $P_0$.
We write $f_1$ as $y \frac{p(x)}{q(x)}$, where $p(x)$ and $q(x)$ have no common factor.
Note that we require a $y$ term since we have an odd order at $P_0$, and since $y^2$ is a function of $x$, we can assume that the power of $y$ is just 1.
We also require $q(x)$ has a factor of $x^i$, and we let $q'(x) : = \frac{q(x)}{x^i}$.
In order for this function to be in $\cO_X(U_1)$ we require that $\frac{p(x)}{q'(x)}$ of order at least $2i-2g-1$ at $P_\infty$, else we will have a pole at $P_\infty$.
This is only possible if $\frac{p(x)}{q(x)}$ has a pole or poles at points other than $P_0$ and $P_\infty$, since $\deg(\di(f_1)) = 0$.
The only zeroes of our function so far are at the ramification points, so clearly the poles of $\frac{p(x)}{q'(x)}$ must be at the ramification points too.
Since $p$ and $q'$ are functions of $x$ any pole at a ramification point must be of even order; in particular they must be at least order 2.
ince we have the zeroes at the ramification points of $y$ and $x^i$ are exactly order 1 (except at the points $P_0$ and $P_\infty$, which we are not considering), this contradicts $f_1$ being regular on $U_1$.

Hence such an $f_1$ cannot exist, and the image of $x^{-i}y$ in $\check{H}^1(\cU,\cO_X)$ is non-zero (for all $i$).


If $p=2$ then we use a similar proof, but 
\end{proof}


In order to state the basis of $\derhamhone$, as well as to shorten the proof of the following proposition, we define the following polynomial and differentials. 
We suppose that $1 \leq i \leq g$.
Then, when $p\neq 2$ we define
\[
s_i(x) := f'(x) - \frac{2if(x)}{x} \in k[x]
\]
and if $p = 2$ we define
\begin{equation}\label{capitals}
S_i(x,y) := xF'(x) + xyH'(x) + iyH(x)\in k[x]\oplus yk[x] \subseteq k(x,y).
\end{equation}

We now decompose these polynomials in to two parts, which will be used in the sequel.
Firstly, we write $s_i(x)$ as $s_i(x) = \phi_i(x) + \psi_i(x)$, where $\psi_i(x), \phi_i(x) \in k[x]$ are the unique polynomials such that the degree of $\phi_i (x))$ is at most $g$ and $x^{g+1}$ divides $\psi_i(x)$.



Secondly, we write $S_i(x,y)$ as $S_i(x,y) = \Phi_i(x,y) + \Psi_i(x,y)$, where $\Phi_i(x,y), \Psi_i(x,y) \in k[x,y]$ are the unique polynomials such that the degree of $\Psi_i(x,y)$ has degree less than $i$ as a polynomial in $x$, and $\Phi_i(x,y)$ is divisible by $x^i$.

Viewing $\derhamhone$ as a quotient of \eqref{derhamconditions}, we now give a $k$-vector space basis of the space.
\begin{thm}\label{basis}

If $p \neq 2$ then the residue classes of the elements 
\begin{equation}\label{one}
 \left( \left( \frac{-\psi_i(x)}{2yx^i}\right) dx, \left(\frac{\phi_i(x)}{2yx^i}\right) dx, x^{-i}y\right), i=1, \ldots ,g,
\end{equation}
along with the residue classes of 
\begin{equation}\label{two}
 \left( \frac{x^{i}}{y} dx , \frac{x^{i}}{y} dx, 0 \right), i = 0,\ldots ,g-1,
\end{equation}
form a basis of $\derhamhone$.

On the other hand, if $p=2$ then the residue classes of the elements 
\begin{equation}\label{three}
\left( \left(\frac{-\Psi_i(x,y)}{x^{i+1}H(x)}\right) dx, \left( \frac{\Phi_i(x,y)}{x^{i+1}H(x)} \right) dx, x^{-i}y \right), i =1, \ldots , g, 
\end{equation}
together with the residue classes of 
\begin{equation}\label{four}
\left( \frac{x^{i}}{H(x)} dx, \frac{x^{i}}{H(x)} dx, 0 \right), i=0, \ldots, g-1,
\end{equation}
form a basis of $\derhamhone$.
\end{thm}

Before proving this proposition we use it to prove the following corollary.

\begin{cor}
The action of $G$ on $\derhamhone$ is faithful.
\end{cor}

\begin{proof}
Recall that $H^0(X,\Omega_X)$ injects in to $\derhamhone$.
Then if $p \neq 2$, or if $G$ does not contain a hyperelliptic involution, it follows from the main theorem of \cite{faithfulaction} that $G$ acts faithfully on $H^0(X,\Omega_X)$, and hence $G$ acts faithfully on $\derhamhone$.

We now suppose that $p=2$ and $G$ does contain a hyperelliptic involution $\sigma$.
By the same theorem from \cite{faithfulaction} as used in the last paragraph, we know that $G$ does not act faithfully on $\hzero$.
Since $\hzero$ is dual to $\hone$ we know that $G$ also does not act faithfully on $\hone$.
In particular, $\sigma$ will act trivially on $\hone$.

We can see exactly why this is from the view of \cech cohomology, and this will also help to determine the action on $\derhamhone$.
If we fix a non-zero natural number $i$ then $\frac{y}{x^i}$, considered as an element of $\cO_X(U_1\cap U_2)$, is mapped to $\frac{y}{x^i} + \frac{H(x)}{x^i}$. 
Now we can split $\frac{H(x)}{x^i}$ as follows, 
\begin{equation*}
\frac{H(x)}{x^i} =  \frac{b_{i-1}x^{i-1} + \ldots + b_1x}{x^i} - \left( - \frac{x^d + b_{d-1}x^{d-1} + \ldots + b_ix^i}{x^i} \right),
\end{equation*}
and since this is clearly the difference of an element of $\cO_x(U_1)$ and an element of $\cO_x(U_2)$ we see that $\frac{H(x)}{x^i}$ is zero in $\hone$.
We let 
\[
H_1(x) = \frac{b_{i-1}x^{i-1} + \ldots + b_1x}{x^i}
\]
and 
\[
H_2(x) = - \frac{x^d + b_{d-1}x^{d-1} + \ldots + b_ix^i}{x^i}.
\]
\todo{this can be made shorter/rearranged - kept for now as may be useful}

Despite having a trivial action on both of the outside terms of \eqref{ses}, when we apply this action to $\derhamhone$ it is not trivial. 
We can describe exactly how sigma acts on the elements of \eqref{three} using $H_1(x)$ and $H_2(x)$:
\begin{multline}
\sigma \left( \left( \left(\frac{-\Psi_i(x,y)}{x^{i+1}H(x)}\right) dx, \left( \frac{\Phi_i(x,y)}{x^{i+1}H(x)} \right) dx, x^{-i}y \right)\right) = \\
 \left( \left(\frac{-\Psi_i(x,y)}{x^{i+1}H(x)} \right) dx + d\left(H'_1(x)\right),  \left( \frac{\Phi_i(x,y)}{x^{i+1}H(x)} \right) dx+ d\left(H'_2(x)\right) , x^{-i}y \right).
\end{multline}
Hence the action is trivial if and only if $d\left(H_1'(x)\right) = d\left(H_2'(x)\right) =0$.
This cannot happen though, since we can always find an $i\leq d$ such that $i \not\equiv d\ \text{mod}\ 2$, and then $d\left(H_2(x)\right) = H_2'(x) \neq 0$.
\todo[inline]{this is a lie if d is very small - i.e. one and maybe two. solve this}
\end{proof}


We now prove Theorem \ref{basis}.

\begin{proof}
We make use of the fact that the short exact sequence \eqref{ses} splits as a sequence of vector spaces over $k$, and that we know bases of the outer two terms.

It is clear that \eqref{two} and \eqref{four} are elements of \eqref{derhamconditions}, and it follows from \cite[Thm 6.1]{faithfulaction} that they are the image of the basis of $H^0(X,\Omega_X)$ in $\derhamhone$.

Moreover, it is obvious that if the elements \eqref{one} and \eqref{three} are in \eqref{derhamconditions} then they will map to the basis of $\hone$ given in Lemma \ref{basish1}.
So it only remains to show that the first two terms in \eqref{one} and \eqref{three} do indeed belong to \eqref{derhamconditions}.


We start with the case $p\neq 2$, and observe that
\begin{align*}
\left(  \frac{\psi_i(x)}{2yx^i}  - \frac{-\phi_i(x)}{2yx^i} \right) dx & =  \frac{s_i(x)}{2yx^i} dx \\
& =  \frac{1}{2yx^i} \left( f' - \frac{2if}{x} \right) dx \\
& =  \frac{x^i}{2y} \left( \frac{f'}{x^{2i}}dx -\frac{2if}{x^{2i+1}} dx \right) \\
& =  \frac{x^i}{2y} \left( fd\left(\frac{1}{x^{2i}}\right) + \frac{1}{x^{2i}}df \right) \\
& =  \frac{x^i}{2y}d(fx^{-2i}) \\
& =  \frac{x^i}{2y} d\left(yx^{-i}\right)^2 \\
& =  d(yx^{-i}),
\end{align*}
with the penultimate line following from the defining equation \eqref{definingequationpnot2}.
This shows that \eqref{one} satisfies one of the conditions to be in \eqref{derhamconditions}
Now it remains to show that $\frac{\phi_i}{2yx^i}dx$ and $\frac{-\psi_i}{2yx^i}dx$ are regular on $U_2$ and $U_1$ respectively.



In order to do this we define $\alpha^i_j \in k$ for $0\leq j \leq 2g+1$ by the equation
\[
s_i(x) = \alpha^i_{2g+1}x^{2g+1} + \ldots + \alpha^i_0,
\]
so that
\[
\phi_i(x) = \alpha^i_{2g+1}x^{2g+1} + \ldots + \alpha^i_{g+1}x^{g+1} \ {\rm and } \ \psi_i(x) = \alpha^i_gx^g + \ldots + \alpha^i_0.
\]
Note that it is possible for any of $\alpha_i^j$ to be zero. Indeed, it is possible for either $\phi_i(x)$ or $\psi_i(x)$ to be zero.
Whenever they are non-zero we denote their degrees as polynomials in $x$ by $d_\phi$ and $d_\psi$ respectively. From the definition of $\phi_i(x)$ and $\psi_i(x)$ we know that $0 < d_\psi \leq g+1$ and $g < d_\phi \leq 2g$ respectively, for all $1 \leq i \leq g$
\todo{remove d psi? - will keep for now in case useful later}


We now show that $\frac{\phi_i}{2yx^i}dx$ and $\frac{\psi_i}{2yx^i}dx$ are regular on $U_2$ and $U_1$ respectively.
We may assume that $\phi_i$ and $\psi_i$ are non-zero, since zero is regular everywhere.
The divisor of $\frac{-\phi_i}{2yx^i}dx$ is

\begin{align*}
\di\left( \frac{-\phi_i}{2yx^i}dx \right) & =  \di(\phi_i) -\di(y) - \di(x^i) + \di (dx) \\
& =  \di(\phi_i) - ( R - (g+1)D_\infty) - (iD_0 - iD_\infty) + (R - 2D_\infty) \\
& =  \left( \di_0\left( \frac{\phi_i}{x^{g+1}}\right) + (g+1)D_0 - d_\phi D_\infty\right) - iD_0 + (g+i-1)D_\infty \\
& \geq  \di_0\left( \frac{\phi_i}{x^{g+1}}\right) + (g+1)D_0 - 2gD_\infty - iD_0 + (g+i-1)D_\infty \\
& =  \di_0\left( \frac{\phi_i}{x^{g+1}} \right) + (i-g-1)D_\infty + (g-i+1)D_0,
\end{align*}
where the first line makes use of \eqref{pnot2divisors}.
Since $i \leq g$ clearly the differential is regular on $U_2 = X\backslash \pi^{-1}(\infty)$.

Similarly the divisor of $\frac{\psi_i}{2yx^i}dx$ is

\begin{align*}
\di \left( \frac{\psi_i}{2yx^i}dx\right) & =  \di(\psi_i) - \di(y) - \di(x^i) + \di (dx) \\
& =  \di (\psi_i ) -(R - (g+1)D_\infty) - (iD_0 - iD_\infty) + (R -2D_\infty) \\
& =  \di(\psi_i) + (g+i-1)D_\infty -iD_0 \\
& \geq \left( \di_0(\psi_i) - gD_\infty \right) + (g+i-1)D_\infty -iD_0 \\
& =  \di_0(\psi_i) + (i-1)D_\infty - iD_0.
\end{align*}
Again, the first line uses \eqref{pnot2divisors}, and since $i\geq 1$ we conclude that $\frac{\psi_i(x)}{2yx^i}dx$ is regular on $U_1 = X \backslash \pi^{-1}(0)$, concluding the $p\neq 2$ case.


We now consider the case when $p=2$.
We start by noting that when we take the differential of \eqref{definep=2} we derive
\[
dF = d\left(y^2 + yH \right) = d(yH) = Hdy + ydH
\]
and from this follows 
\[
dy = \left(\frac{F'-yH'}{H}dx\right).
\]
Now we can see that
\begin{align*}
\left( \left( \frac{ \Psi_i}{x^{i+1}H} \right) - \left( \frac{\Phi_i}{x^{i+1}H} \right) \right) dx & =  \frac{S_i}{x^{i+1}H}dx \\
& =  \left( \frac{F'}{x^iH} + \frac{yH'}{x^iH} + \frac{iy}{x^{i+1}} \right) dx \\
& =  \frac{1}{x^i}\left( \frac{F' + yH'}{H} \right) dx + \frac{iy}{x^{i+1}}dx \\
& =  x^{-i}dy + yd \left( x^{-i}\right) \\
& =  dyx^{-i},
\end{align*}
and so it only remains to show that $\frac{\Phi_i}{x^{i+1}H}dx$ and $\frac{\Psi_i}{x^{i+1}H}dx$ are regular on $U_2$ and $U_1$ respectively.


We define $A^i_{j+1} \in k$ for $0 \leq j \leq 2g+1$, and $B_k^i \in k$ for $1\leq k \leq g$ by the equation
\[
S_i(x,y) = A_{2g+2}^ix^{2g+2} + \ldots + A^i_1 x + y(B_g+1^i x^g+1 + \ldots + B_1^i x + B_0^i).
\]
Note that many of these coefficients may be zero.
In particular we note that $B_i^i = 0$ for all $i$.
If $i$ is even then $B_{g+1}^ix^{g+1} + \ldots + B_1^ix + B_0^i = H'x$ and so the coefficient of $x^i$ in $H'x$ is $ib_i = 0$ (recall that we defined $b_i$ to be the coefficients of $H$ in \eqref{hcoefficients}).
Similarly, if $i$ is odd then $B_{g+1}^ix^{g+1} + \ldots + B_1^ix + B_0^i = H'x + H$, and then the coefficient of $x^i$ is $ib_i+ ib_i = 2b_i = 0$.


We can now define the following polynomials:
\begin{equation}\label{Split}
\begin{split}
\Phi_i^x(x) & =  A^i_{2g+2}x^{2g+2} + \ldots + A^i_{i+1}x^{i+1} \\
\Psi_i^x(x) & =  A^i_ix^i + \ldots + A^i_1x \\
\Phi_i^y(x) & =  B_g^ix^g + \ldots B_{i+1}^ix^{i+1} \\
\Psi_i^y(x) & =  B_{i-1}^ix^{i-1} + \ldots + B_1^ix + B_0^i.
\end{split}
\end{equation}
We denote the degrees of these polynomials by $d_{\Phi}^x, d_{\Psi}^x, d_{\Phi}^y$ and $d_{\Psi}^y$ respectively.

Clearly $\Phi_i(x) = \Phi_i^x(x) + y\Phi_i^y(x)$ and $\Psi_i (x)= \Psi_i^x(x) + y\Psi_i^y(x)$, and we will use these splittings of to show that $\frac{ \Phi_i(x) dx}{x^{i+1}H}$ and $\frac{\Psi_i(x) dx}{x^{i+1}H}$ are regular on $U_2$ and $U_1$ respectively.

We start by computing the divisor of $\frac{dx}{x^{i+1}H}$, since it is a common component to all the differentials we need to look at.
This yields
\begin{align}
\di \left( \frac{dx}{x^{i+1}H} \right) & = \di(dx) - \di (x^{i+1}) - \di (H) \nonumber \\
& = R-2D_\infty) - ((i+1)D_0 - (i+1)D_\infty) - (R - (g+1)D_\infty) \nonumber \\
& = (g+i)D_\infty - (i+1)D_0,
\end{align}
using \eqref{divdxp=2}, \eqref{divhp=2} and \eqref{divxp=2}.
We now use this, along with \eqref{divyp=2} and the polynomials we defined in the preceding paragraphs to complete the proof.

We begin by computing the divisors associated to $\Phi_i$.
Firstly,
\begin{align*}
\di \left( \frac{\Phi_i^x dx}{x^{i+1} H} \right)  = &  \di(\Phi_i^x) -(i+1)D_0 + (g+i)D_\infty\\
 = & \left( \di_0(\Phi_i^x) -d_\Phi^xD_\infty\right) -(i+1)D_0 + (g+i)D_\infty\\
 \geq & \di_0(\Phi_i^x) - (2g+1)D_\infty - (i+1)D_0 + (g+i)D_\infty \\
 = &  \di_0(\Phi_i^x) - (i+1)D_0 + (i-1-g)D_\infty \\
 =  & \di_0 \left( \frac{\Phi_i^x}{x^{i+1}} \right) + (i-g-1)D_\infty,
\end{align*}
From this we see that the differential is clearly regular on $U_2 = X \backslash \pi^{-1}(\infty)$.

We now compute the divisor of the other half of $\frac{\Phi_i}{x^{i+1}H}dx$, namely
\begin{align*}
\di\left(\frac{y\Phi_i^y dx}{x^{i+1}H} \right)  = & \di(y) + \di(\Phi_i^y) -(i+1)D_0 + (g+i)D_\infty\\
 = & \di(y) + \di_0(\Phi_i^y) - d_\Phi^yD_\infty -(i+1)D_0 + (g+i)D_\infty \\
 = & \di(y) + \di_0\left(\frac{\Phi_i^y}{x^{i+1}} \right) + (g-d_\Phi^y -1)D_\infty.
\end{align*}
Since $y$ and $\frac{\Phi_i^y}{x^{i+1}}$ both only have poles at points in $\pi^{-1}(\infty)$, this completes the proof that $\Phi_i$ is regular on $U_2 = X \backslash \pi^{-1}(\infty)$.

Now we complete the same computations on $\Psi_i$, starting with $\Psi_i^x$:
\begin{align*}
\di\left( \frac{\Psi_i^x dx}{x^{i+1}H} \right)  = &  \di(\Psi_i^x)  - (i+1)D_0 + (g+i)D_\infty \\
 \geq &  \di_0(\Psi_i^x ) - iD_\infty - (i+1)D_0 + (g+i)D_\infty \\
 = &  \di_0(\Psi_i^x) - iD_0 + (g+1)D_\infty,
\end{align*}
and it is clear that the divisor is positive on $U_1 = X \backslash \pi^{-1}(0)$.

For the other half of the differential we need to take cases.
If we assume that $\infty$ is branch point then we see that
\begin{align*}
\di\left(\frac{y\Psi_i^y dx}{x^{i+1}H} \right)  =  & \di_0(y) - (2g+1)D_\infty + \di(\Psi_i^y) - (i+1)D_0 + (g+i)D_\infty \\
 =  & \di_0(y) + \di(\Psi_i^y) -(i+1)D_0 + (2i -1)D_\infty \\
 = &  \di_0(y) + \di_0(\Psi_i^y) - d_\Psi^yD_\infty - (i+1)D_0 + (2i-1)D_\infty \\
 \geq &  \di_0(y) + \di_0(\Psi_i^y) -(i-1)D_\infty -(i+1)D_0 + (2i-1)D_\infty \\
 =   &\di_0(y) + \di_0(\Psi_i^y) -(i+1)D_0 + D_\infty.
\end{align*}
which is clearly positive on $U_1$.
If, on the other hand, $\infty$ is not a branch point then we have
\begin{align*}
\di\left(\frac{y\Psi_i^y dx}{x^{i+1}H} \right)  =  & \di(y) + \di(\Psi_i^y) - (i+1)D_0 + (g+i)D_\infty \\
= & \di(y) + \di_0(\Psi_i^y) - (i+1)D_0 + (g+i - d_\Psi^y)D_\infty \\
\geq & \di(y) + \di_0(\Psi_i^y) - (i+1)D_0 + (g+1)D_\infty. \\
\end{align*}
Since we know from \eqref{divyp=2} that $y$ cannot have a pole of order greater $g+1$ at $P_\infty$ or $P_\infty'$, and only has poles at these points, it follows that the differential is regular on $U_1 = X \backslash \pi^{-1}(0)$, and we have completed the proof.


\end{proof}

\section{Computing the residue}

Introduction needs to be finalised once the proof in the previous section is completed.

\begin{lem}
If $p \neq 2$ then ${\rm res}_{P_\infty}(\frac{1}{x}dx) = -2$.
If $p=2$ then ${\rm res}_{P_\infty}\left(\frac{ydx}{xH(x)}\right) = 1$.
\end{lem}

\begin{proof}

We first consider the case $p\neq 2$.
We note that $t:= \frac{y}{x^{g+1}}$ is a uniformising parameter at $P_\infty$ as can be seen by computing the order of $t$ at $P_\infty$ as follows
\begin{eqnarray}
\ord_{P_\infty}(t) & = & \frac{1}{2}\ord_{P_\infty}(t^2) \\
  & = & \frac{1}{2}\ord_{P_\infty}\left( \frac{f}{x^{2g+2}} \right) \\
& = & \frac{1}{2}\ord_{P_\infty}(f(x)) - \frac{1}{2}\ord_{P_\infty}(x^{2g+2})\\
& = & -(2g+1) + (2g+2) \\
& = & 1.
\end{eqnarray}

We now write $\frac{1}{x}dx$ in terms of $dt$.
By the quotient rule we know that
\begin{eqnarray*}
dt^2 & = & d \left( \frac{f(x)}{x^{2g+2}} \right) \\
& = & \frac{x^{2g+2}f' + x^{2g+1}f}{x^{4g+4}} dx \\
& = & \frac{1}{x^{2g+2}} \left( f' + \frac{f}{x} \right) dx
\end{eqnarray*}
from which we conclude that
\[
\frac{1}{x}dx = \frac{-2tx^{2g+1}}{\left(\frac{(2g+2)f}{x} - f'\right)} dt.
\]


We now let $p(x) = \left(\frac{(2g+2)f(x)}{x} - f'(x)\right)$, and noting that the coefficient of $x^{2g}$ in $p(x)$ is $(2g+2)a_{2g+1} - (2g+1)a_{2g+1} = a_{2g+1}$, we see that $h(x)$ is a degree $2g$ polynomial in $x$.
We wish to compute the coefficient of $t^{-1}$ in the expansion of $\frac{1}{x}dx$ at $P_\infty$ and for this we require the following expansions
\begin{align}
\frac{p(x)}{x^{2g+1}} = \frac{a_{2g+1}x^{2g}}{x^{2g+1}} + \ldots = \frac{a_{2g+1}}{x} + \ldots \qquad \text{and} \qquad t^2 = \frac{f}{x^{2g+2} } = \frac{a_{2g+1}}{x} + \ldots
\end{align}

Since $\ord_{P_\infty}\left(\frac{p(x)}{x^{2g+1}}\right) = 2$ we know that $\frac{p(x)}{x^{2g+1}} = \sum_{j\geq 2} c_j t^j$ for some $c_j\in k$, and from the above computations we can see that $c_2 = 1$.
We also know that $\frac{x^{2g+1}}{p(x)} = \sum_{k\geq -2} d_kt^k$, for some $d_k\in k$, and clearly $d_{-2} = 1$.
Now
\[
\frac{1}{x}dx = \left( -2t \cdot \sum_{i\geq -2} d_it^i\right) dt 
\]
so we see that the residue is $-2$.
This complets the proof of the lemma when $p\neq 2$.

We now turn to the case when $p=2$.
We now wish to compute the residue of $\frac{ydx}{xH(x)}$ at $P_\infty$.
We start by noting that $t = \frac{y}{x^{g+1}}$ is a uniformising parameter at $P_\infty$ which we check by computing the divisor:
\begin{align*}
\di(t) & = \di_0(y) - (2g+1)D_\infty -(g+1)D_0 + (g+1)D_\infty \\
& = \di_0(y)-(g+1)D_0 + D_\infty,
\end{align*}
using \eqref{divxp=2} and \eqref{divyp=2}.
So clearly $t$ is a uniformising parameter at $P_\infty$.

We now wish to write $\frac{y}{xH(x)}dx$ as $r(x,y)dt$ for some $r \in k(x,y)$.
We first write $dy$ in terms of $dx$.
Since
\begin{eqnarray*}
0 & = & dy^2 \\
& = & d(F+yH) \\
& = & F'dx + Hdy + yH'dx
\end{eqnarray*}
we conclude that
\[
dy = \left( \frac{F'+yH'}{H} \right) dx.
\]

We also rewrite $dt$ as follows:
\begin{eqnarray*}
dt & = & d\left( \frac{y}{x^{g+1}} \right) \\
& = & yd\frac{1}{x^{g+1}} + \frac{1}{x^{g+1}}dy \\
& = & \frac{1}{x^{g+1}} \left( \frac{(g+1)y}{x} + \frac{F'+yH'}{H} \right) dx \\
& = & \frac{1}{x^{g+1}} \left( \frac{xF'}{y} + xH' + (g+1)H \right) \frac{y}{Hx} dx.
\end{eqnarray*}

In total we then have
\[
\frac{y}{xH(x)}dx = \frac{x^{g+1}y}{S_{g+1}(x,y)}dt
\]
where $S_{g+1}(x,y)$ is as defined in \eqref{capitals}.

We know that $\frac{x^{g+1}y}{S_{g+1}(x,y)} = \sum_{i\geq -1} c_i t^i$, with $c_i \in k$, and we wish to compute $c_{-1}$.
We shall do this by computing the coefficient of $t$ in the expansion $\frac{S_{g+1}(x,y)}{x^{g+1}y} = \sum_{i\geq 1}d_it^i$.
We can split up $\frac{S_{g+1}(x,y)}{x^{g+1}y}$ in to three terms, namely $\frac{xF'}{x^{g+1}y}$, $\frac{yxH'}{x^{g+1}y}$ and $\frac{yH}{x^{g+1}y}$.
Each of these terms can of course be written as a power series in $t$, but only the first term mentioned has order 1 at $P_\infty$, and hence this term will uniquely determine $d_1$.
Suppose that $F = \alpha_{2g+1}x^{2g+1} + \alpha_{2g}x^{2g} + \ldots + \alpha_1x^1 + \alpha_0$.
Then $xF'= \alpha_{2g+1}x^{2g+1} + \alpha_{2g-1}x^{2g-1} + \ldots + \alpha_1x^1$; i.e. the terms with an even power are removed.

So the only term in $\frac{xF'}{x^{g+1}y}$ of order 1 at $P_\infty$ is $\frac{\alpha_{2g+1}x^{2g+1}}{x^{g+1}y} = \frac{\alpha_{2g+1}x^{g}}{y}$.
Since
\[
\frac{\alpha_{2g+1}x^g}{y} = \frac{\alpha_{2g+1}}{x}t^{-1}
\]
and $\frac{1}{x} = \sum_{i\geq 2}e_it^i$ for some $e_i \in k$, if we compute $e_2$ then we will have effectively computed $d_1$.
Now $t^2 = \frac{F }{x^{2g+2}}+ \frac{Hy}{x^{2g+2}}$, and clearly $\frac{Hy}{x^{2g+2}}$ has no terms of the form $\frac{c}{x}$ for some $c \in k$.
On the other hand
\[
\frac{F}{x^{2g+2}} = \frac{\alpha_{2g+1}}{x} + \ldots
\]
Hence we conclude that $e_2 = \frac{1}{\alpha_{2g+1}}$.
It follows that $d_1 = \alpha_{2g+1} \cdot \frac{1}{\alpha_{2g+1}} = 1$.


We finally use this to compute $c_{-1}$.
Since 
\begin{equation*}
1 = \frac{S_{g+1}(x,y)}{x^{g+1}y}\cdot \frac{x^{g+1}y}{S_{g+1}(x,y)} = \left( \sum_{i\geq 1}d_it^i \right) \cdot \left( \sum_{i\geq -1}c_it^i\right)
\end{equation*}
we conclude that $c_{-1} = \frac{1}{d_{1}} = 1$.




\end{proof}



\begin{comment}
We define $\alpha^i_j$ and $\Alpha^i_{j+1}$ for $0 \leq j \leq 2g$, and $B_k^i$ for $1\leq k \leq g$, such that
\[
$s_i(x) = \alpha^i_{2g}x^{2g} + \ldots + \alpha^i_0 \ {\rm and } \ S_i(x) = A_{2g+1}^ix^{2g+1} + \ldots + A^i_1 x + y(B_g^i x^i + \ldots + B_1^i x).
\]
\end{comment}


\begin{note}
For $\infty$ is not a branch point the same basis should work. 
Consider $y^2 = f(x)$ where the degree of $f(x)$ is $2g+2$ (i.e. $\infty$ is not ramified), with projection map $\pi : X \rightarrow \mathbb P_k^1$.
Then we can take the cover to be $U_1 := X \backslash \{\pi^{-1}(0)\}$ and $U_2 := X \backslash \{\pi^{-1}(\infty)\}$.
Then we have $\frac{x^idx}{y} \cdot \frac{y}{x^{j}} = x^{i-j}dx$.
The divisor of this is then $(i-j)D_0 -(i-j)D_\infty + R - 2D_\infty$.
Then if $i-j = -1$ this has poles at at all points supported at by $D_0$ and $D_\infty$.


If $p=2$ and $\infty$ is not a branch point then we need to consider the defining equation $y^2 - h(x)y = f(x)$ where $\deg(h)=g+1$, and $\deg(f)$ can be anything from 0 to $2g+2$.
From work (which I think was not in 18 month report), if we suppose that $P_\infty$ and $P_\infty'$ are the two points in the pre-image of $\infty \in \mathbb P_k^1$, then the divisor of $y$ is
\[
\di(y) = \di_0(y) +(g+1 - \deg(f))D_\infty - (g+1)[P_\infty'],
\]
up to parity of $P_\infty$ and $P_\infty'$.
So then, after some computation, we get
\begin{multline*}
\di\left(\frac{yx^{i-j}}{H(x)}dx \right) = \di_0(y) + (i-j)D_0 + (g-1-(i-j))D_\infty \\ + (g+1-\deg(f))D_\infty -(g+1)[P_\infty'].
\end{multline*}
So if $i-j = -1$ we clearly have a pole of order 1 at $P_\infty'$.
\end{note}

Now a short note about whether the main theorem still holds if we do not specify the ramification points.
\begin{note}
Clearly when $p\neq 2$ the same basis holds. In the proof we use the fact that $\phi_i$ has a factor of at least $x^{g+1}$, giving a sufficient zero at $P_0$.
This factor is still in the polynomial if we increase its potential degree.
Of course, we may have to replace $P_0$ with $D_0$, but the essential point still holds.
Similarly, after possibly replacing $P_\infty$ by $D_\infty$ we have almost the same argument for $\psi_i$.
So the basis is the same when $p\neq2$.

When $p=2$ the basis is also the same.
We first note that whilst the degree of the highest and lowest ordered terms may change in \eqref{Split}, it is still the case that te coefficient $B_i^i =0$, using exactly the same arguments.
This allows us to use essentially the same arguments, again with the exception of changing $P_\infty$ and $P_0$ to $D_\infty$ and $D_0$.
\end{note}

\begin{note}[Why action on H one is trivial]
We suppose that $p=2$ and $G$ does contain a hyperelliptic involution $\sigma$.
By the main theorem from \cite{faithfulaction}, we know that $G$ does not act faithfully on $\hzero$.
Since $\hzero$ is dual to $\hone$ we know that $G$ also does not act faithfully on $\hone$.
In particular, $\sigma$ will act trivially on $\hone$.

We can see exactly why this is from the view of \cech cohomology.
If we fix a non-zero natural number $i$ then $\frac{y}{x^i}$, considered as an element of $\cO_X(U_1\cap U_2)$, is mapped to $\frac{y}{x^i} + \frac{H(x)}{x^i}$. 
Now we can split $\frac{H(x)}{x^i}$ as follows, 
\begin{equation*}
\frac{H(x)}{x^i} =  \frac{b_{i-1}x^{i-1} + \ldots + b_1x}{x^i} - \left( - \frac{x^d + b_{d-1}x^{d-1} + \ldots + b_ix^i}{x^i} \right),
\end{equation*}
and since this is clearly the difference of an element of $\cO_x(U_1)$ and an element of $\cO_x(U_2)$ we see that $\frac{H(x)}{x^i}$ is zero on $\hone$.
We let 
\[
H_1(x) = \frac{b_{i-1}x^{i-1} + \ldots + b_1x}{x^i}
\]
and 
\[
H_2(x) = - \frac{x^d + b_{d-1}x^{d-1} + \ldots + b_ix^i}{x^i}.
\]
\end{note}


\bibliography{biblio}
\bibliographystyle{plain}


\end{document}
