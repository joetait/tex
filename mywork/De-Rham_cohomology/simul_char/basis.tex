% !TEX TS-program = pdflatex
% !TEX encoding = UTF-8 Unicode

% This is a simple template for a LaTeX document using the "article" class.
% See "book", "report", "letter" for other types of document.

\documentclass[draft, 11pt]{article} % use larger type; default would be 10pt

\usepackage[utf8]{inputenc} % set input encoding (not needed with XeLaTeX)

%%% Examples of Article customizations
% These packages are optional, depending whether you want the features they provide.
% See the LaTeX Companion or other references for full information.

%%% PAGE DIMENSIONS
\usepackage{geometry} % to change the page dimensions
\geometry{a4paper} % or letterpaper (US) or a5paper or....
% \geometry{landscape} % set up the page for landscape
% read geometry.pdf for detailed page layout information

\usepackage{graphicx} % support the \includegraphics command and options
\usepackage[obeyDraft]{todonotes}

%\usepackage[parfill]{parskip} % Activate to begin paragraphs with an empty line rather than an indent

%%% PACKAGES
\usepackage[all]{xy}
\usepackage{mathtools}
\usepackage{booktabs} % for much better looking tables
\usepackage{array} % for better arrays (eg matrices) in maths
\usepackage{paralist} % very flexible & customisable lists (eg. enumerate/itemize, etc.)
\usepackage{verbatim} % adds environment for commenting out blocks of text & for better verbatim
\usepackage{subfig} % make it possible to include more than one captioned figure/table in a single float
%\usepackage{hyperref}
% These packages are all incorporated in the memoir class to one degree or another...

%\usepackage[activate={true,nocompatibility},final,tracking=true,kerning=true,spacing=true,factor=1100,stretch=10,shrink=10]{microtype}
%\microtypecontext{spacing=nonfrench}
% activate={true,nocompatibility} - activate protrusion and expansion
% final - enable microtype; use "draft" to disable
% tracking=true, kerning=true, spacing=true - activate these techniques
% factor=1100 - add 10% to the protrusion amount (default is 1000)
% stretch=10, shrink=10 - reduce stretchability/shrinkability (default is 20/20)

%%% HEADERS & FOOTERS
\usepackage{fancyhdr} % This should be set AFTER setting up the page geometry
\pagestyle{fancy} % options: empty , plain , fancy
\renewcommand{\headrulewidth}{0pt} % customise the layout...
\lhead{}\chead{}\rhead{}
\lfoot{}\cfoot{\thepage}\rfoot{}

%%% SECTION TITLE APPEARANCE
\usepackage{sectsty}
\allsectionsfont{\sffamily\mdseries\upshape} % (See the fntguide.pdf for font help)
\usepackage{amsmath}
\usepackage{amsthm}
\usepackage{amsfonts}
\usepackage{mathrsfs}
\usepackage{amsopn}
\usepackage{amssymb}
%\usepackage{natbib}
% (This matches ConTeXt defaults)

%%% ToC (table of contents) APPEARANCE
\usepackage[nottoc,notlof,notlot]{tocbibind} % Put the bibliography in the ToC
\usepackage[titles,subfigure]{tocloft} % Alter the style of the Table of Contents
%\renewcommand{\cftsecfont}{\rmfamily\mdseries\upshape}
%\renewcommand{\cftsecpagefont}{\rmfamily\mdseries\upshape} % No bold!
%\renewcommand{\familydefault}{\sfdefault}
%\usepackage{cabin}
%\usepackage{libertine}
%\usepackage[T1]{fontenc}

%Theorems and stuff
\theoremstyle{plain}
\newtheorem{defn}{Definition}[section]
\newtheorem{thm}[defn]{Theorem}
\newtheorem{cor}[defn]{Corollary}
\newtheorem{lem}[defn]{Lemma}
\newtheorem{prop}[defn]{Proposition}
\newtheorem{ex}[defn]{Example}
\newtheorem*{unnumthm}{Theorem}
\newtheorem{defnlem}[defn]{Definition/Lemma}
\newtheorem{defnthm}[defn]{Theorem/Definition}
\theoremstyle{remark}
\newtheorem*{rem}{Remark}
\newtheorem*{note}{Note}


\newcommand{\cO}{{\cal O}}
\newcommand{\ra}{\rightarrow}
\newcommand{\NN}{{\mathbb N}}
\newcommand{\PP}{{\mathbb P}}
\newcommand{\ZZ}{{\mathbb Z}}
\newcommand{\cL}{{\mathcal L}}
\newcommand{\cA}{{\mathcal A}}
\newcommand{\cD}{{\mathcal D}}
\newcommand{\cU}{{\mathcal U}}
\newcommand{\cech}{\v{C}ech }
\newcommand{\hzero}{{H^0(X,\Omega_X)}}
\newcommand{\hone}{H^1(X,\mathcal{O}_X)}
\newcommand{\cechhone}{\check{H}^1(\mathcal U,\mathcal O_X)}
\newcommand{\derhamhone}{H_{\text {dR}}^1(X/k)}
\newcommand{\cechhzero}{{\check{H}^0(X,\Omega_X)}}
\newcommand{\ubar}{\underset{\bar{}}}


\DeclareMathOperator{\aut}{Aut}
\DeclareMathOperator{\res}{Res}
\DeclareMathOperator{\ord}{ord}
\DeclareMathOperator{\di}{div}
\DeclareMathOperator{\cha}{char}
\DeclareMathOperator{\gal}{Gal}
\DeclareMathOperator{\Tr}{Tr}
\DeclareMathOperator{\Ima}{Im}

%%% END Article customizations

%%% The "real" document content comes below...

\title{Group actions on de Rham cohomology of hyperelliptic curves}
\author{}
%\date{} % Activate to display a given date or no date (if empty),
         % otherwise the current date is printed

\begin{document}
\maketitle

\listoftodos

\section{Background}

\todo[inline]{change references to preprint when in thesis}
Let $X$ be a smooth, projective, connected hyperelliptic curve of genus $g \geq 2$ over an algebraically closed field $k$ of characteristic $p \geq 0$.
We  fix a map $\pi \colon X \rightarrow \mathbb P_k^1$ of degree two, which is unique up to an automorphism of $\mathbb P_k^1$.
We let $G$ denote a finite group acting faithfully on $X$.

In a similar vein to \cite{canonicalrepresentation}, we will be using \cech cohomology to compute both $\hone$ and the de Rham hypercohomology, denoted $\derhamhone$.
We therefore recall the \cech cohomology relevant to both of these, starting with $\hone$.

By Leray's theorem \cite[Thm 5.2.12]{liu} and Serre's affineness criterion \cite[Thm 5.2.23]{liu} we know that the first \cech cohomology group and $\hone$ will be isomorphic if the cover we use to compute the \cech cohomology is affine.
We let $U_a = X \backslash \pi^{-1}(a)$ for any $a \in \mathbb P_k^1$ and we let ${\cal U}$ be the affine cover of $X$ formed by $U_0$ and $U_\infty$.
Given any sheaf $\cal F$ on $X$ we have the \cech differential $\check{d}\colon {\cal F}(U_0) \times {\cal F} (U_\infty) \rightarrow {\cal F}(U_0 \cap U_\infty)$, defined by $(f_0,f_\infty) \mapsto f_0|_{U_0 \cap U_\infty} - f_\infty|_{U_0 \cap U_\infty}$.
In general we will suppress the notation denoting the restriction map.
Via this differential we have the following cochain complex
\begin{equation*}
0 \rightarrow \cO_X(U_0)\times \cO_X(U_\infty) \xrightarrow{\check{d}} \cO_X(U_0 \cap U_\infty) \rightarrow 0.
\end{equation*}
The first cohomology group of this chain is $\cechhone = \frac{\cO_X(U_0 \cap U_\infty)}{\Ima(\check{d})}$ and hence
\begin{equation}\label{cechhone}
\hone \cong \frac{\cO_X(U_0 \cap U_\infty)}{\Ima(\check{d})}  
 = \frac{\cO_X(U_0 \cap U_\infty)}{\{f_0 - f_\infty | f_i \in \cO_X(U_i) \}}.
\end{equation}

We now recall how to compute the algebraic de Rham cohomology of $X$ via \cech cohomology.
Since $X$ is a curve any differentials of degree greater than one on $X$ are zero.
Hence the de Rham complex $X$ is the hypercohomology of the complex
\begin{equation}\label{res}
0 \rightarrow \cO_X \xrightarrow{d} \Omega_X \rightarrow 0.
\end{equation}
Here $d$ denotes the differential map $f \mapsto df$, as defined in \cite[Chap II, \S 8, Pg. 172]{hart}.

We use the cover $\cal U$ and the \cech differentials as defined earlier to give us the \cech bicomplex of \eqref{res}, which is
\begin{equation}\label{bicomplex}
\xymatrix{ & 0 \ar[d] & 0 \ar[d] & \\
0 \ar[r] & \cO_X(U_0) \times \cO_X(U_\infty) \ar[d] \ar[r] & \Omega_X(U_0) \times \Omega_X(U_\infty) \ar[d] \ar[r] & 0 \\
0 \ar[r] & \cO_X(U_0\cap U_\infty) \ar[d] \ar[r] & \Omega_X(U_0 \cap U_\infty) \ar[r] \ar[d] & 0 \\
& 0 & 0 &}
\end{equation}
By a generalisation of Leray's theorem \cite[Cor 12.4.7]{EGA0III} we know that the $\derhamhone$ is isomorphic to the first cohomology of the total complex of \eqref{bicomplex}.
Note that this requires ${\check H}^p(U_\sigma, \cO_X)$ and ${\check H}^p(U_\sigma, \Omega_X)$ to be zero for any $\sigma$ in the nerve of $\cU$ and any $p \geq 1$ ---
since $U_0$ and $U_\infty$ are affine, this again follows from Serre's affineness criterion \cite[Thm 5.2.23]{liu}.



After computing the first cohomology group of the total complex of \eqref{bicomplex} we see that $\derhamhone$ is isomorphic to the space
\begin{equation}\label{derhamconditions}
\left\{(\omega_0, \omega_\infty, f_{0,\infty}) | \omega_i\in \Omega_{X/k}(U_i), f_{0,\infty}\in \cO_X(U_0 \cap U_\infty), df_{0,\infty} = \omega_0|_{U_0\cap U_\infty} - \omega_\infty|_{U_0\cap U_\infty} \right\}
\end{equation}
quotiented by the subspace
\begin{equation}\label{quotient}
\left\{  (df_0, df_\infty, f_0|_{U_0\cap U_\infty} -f_\infty|_{U_0\cap U_\infty} )|f_i \in \cO_X(U_i)\right\}.
\end{equation}

We wish to compute a $k$-basis of $\derhamhone$ in order that we can see how $G$ acts on $\derhamhone$ and to study the $k[G]$-module structure.
The following lemma shows that $\derhamhone$ fits in to a short exact sequence which we will use to compute the $k$-basis.
\begin{prop}\label{ses}
The following is a short exact sequence
\begin{equation}\label{equationses}
0 \ra H^0(X,\Omega_X) \xrightarrow{i} \derhamhone \xrightarrow{p} H^1(X,\cO_X) \ra 0, 
\end{equation}
where $i \colon \omega \mapsto (\omega, \omega)$ and $p \colon (\omega_0, \omega_\infty, f_{0 \infty}) \to f_{0 \infty}$.
\end{prop}
\begin{proof}
Let $T$ be the total complex of \eqref{bicomplex}.
Moreover we let $\cO$ and $\Omega$ be the complexes formed from the first and second (non-trivial) columns of \eqref{bicomplex} respectively.
Then let $\Omega[1]$ denote the complex obtained from shifting $\Omega$ by one, i.e. $\Omega[1]^{n+1} = \Omega^n$.
Then we obtain the following short exact sequence of complexes 
\[
\Omega[1] \hookrightarrow T \twoheadrightarrow \cO,
\]
giving rise to the following long exact sequence
\begin{equation*}
0 \ra H^0_{\text {dR}}(X/k) \ra H^0(X,\cO_X) \ra \\
\end{equation*}
\begin{equation}\label{longexactsequence}
 H^0(X,\Omega_X) \ra \derhamhone \ra \hone \ra \\
\end{equation}
\begin{equation*}
 H^1(X,\Omega_X) \ra H^2_{\text {dR}}(X/k) \ra 0.
\end{equation*}
The map $H^0(X,\cO_X) \ra \hzero$ is the map $f \mapsto df$.
Since the only globally holomorphic functions on $X$ are constant functions, it follows that this is the zero map, and hence $\hzero \ra \derhamhone$ is injective.

Since \eqref{longexactsequence} is exact, $p$ is surjective if and only if $\alpha \colon H^1(X,\Omega_X) \ra H^2_{\text {dR}}(X/k)$ is injective.
Now $H^1(X,\Omega_X)$ is isomorphic to $k$ via the residue map \cite[Chap. III, Thm. 7.14.1]{hart}, and if we can show that this isomorphism factors through $\alpha$ it will follow that $\alpha$ is injective.
Considering the \cech cohomology constructions of $H^1(X,\Omega_X)$ and $H^2_{\text {dR}}(X/k)$, it suffices to show that the residue map is zero on $\Ima \left( d \colon \cO_X(U_0 \cap U_\infty) \ra \Omega_X(U_0 \cap U_\infty) \right)$.
This follows from \cite[Chap. III, Thm. 7.14.1 (b)]{hart}, which says that $\res_P(df)=0$ for any $f \in \cO_X(U_0 \cap U_\infty)$.
Hence the residue isomorphism factors through $\alpha$, and hence $p$ is surjective.
\end{proof}


A basis of $H^0(X,\Omega_X)$ is already available in the literature (see, for example, \cite[Prop. 7.4.26]{liu}).
In the next section we compute a basis of $\hone$, in order that we can understand how $G$ acts on both ends of the short exact sequence described in Lemma \ref{ses}.

\section{Basis of $\derhamhone$}

The bases of the $k$-vector spaces in \eqref{equationses} will be given in terms of defining equations of the hyperelliptic curve $X$, so we describe these now.
Throughout we let $P_a$ and $P_a'$ denote the unique elements of $\pi^{-1}(a)$ for any point $a \in \mathbb P_k^1$ that is not a branch point.
If $a \in \mathbb P_k^1$ is a branch point we denote the unique point in $\pi^{-1}(a)$ by $P_a$.
We also define $D_a$ to be the divisor $\pi^*\left([a]\right)$ for any $a \in \mathbb P_k^1$, and hence
\begin{equation*}
D_a= 
\begin{cases}
 2[P_a] & \text{if $a$ is a branch point}, \\
 [P_a] + [P_a'] & \text{otherwise.}
\end{cases}
\end{equation*}
With this notation we also have
\begin{equation}\label{divxp=2}
\di (x)  = D_0 - D_\infty
\end{equation}
regardless of the characteristic of $k$.
Also, as in \cite[\S 6]{faithfulaction}, we have
\begin{equation}\label{differentialdivisor}
\di(dx) = R - 2D_\infty,
\end{equation}
where $R$ denotes the ramification divisor of $\pi$.
This is again regardless of characteristic.

If $p \neq 2$ then the extension $K(X)$ of $K(\mathbb P_k^1) = k(x)$ will be $k(x,y)$ where $y$ satisfies
\begin{equation}\label{definingequationpnot2}
y^2 = f(x)
\end{equation}
for some polynomial $f(x) \in k[x]$ which has no repeated roots and is of degree $2g+1$ or $2g+2$ \cite[Prop 7.4.24]{liu}.

Recall from \cite[\S 6]{faithfulaction} that
\begin{equation}\label{pnot2divisors}
\di(y)  = R - (g+1)D_\infty.
\end{equation}


If $p=2$, then the extension $K(X)$ of $k(x)$ will be $k(x,y)$, this time with $y$ satisfying the equation
\begin{equation}\label{definep=2}
y^2 - H(x)y = F(x)
\end{equation}
for some $H(x),F(x) \in k[x]$, such that $H(x)$ and $H'(x)^2F(x) + F'(x)^2$ share no roots.
We require that $\deg(H(x)) \leq g+1$, with equality if and only $\infty$ is not a branch point, and that $\deg(F(x)) \leq 2g+2$ with $\deg(F(x)) = 2g+1$ if $\infty$ is a branch point  \cite[Prop 7.4.24]{liu}.


We now recall from \cite[\S 6]{faithfulaction} that the divisor of $H(x)$ is
\begin{equation}\label{divisorofH}
\di (H(x))  = R - (g+1)D_\infty. 
\end{equation}
Finally, we describe the divisor of $y$ when $p=2$.
In order to do this we need to distinguish the zeroes of $F(x)$.
Suppose that $F(x)$ has $l \leq \deg(F)$ distinct zeroes, and let $\gamma_1, \ldots, \gamma_l \in k \subseteq \mathbb P_k^1$ be these zeroes.
Then if $\gamma_i$ is a branch point let $Q_i = (\gamma_i, 0)$ be the unique point in the pre-image $\pi^{-1}(\gamma_i)$.
If $\gamma_i$ is not a branch point then let $Q_i = (\gamma_i, 0)$ and $Q_i' = (\gamma_i, H(\gamma_i))$ be the unique points that form the pre-image $\pi^{-1}(\gamma_i)$.
Also, we denote the order of the zero of $F(x)$ at $\gamma_i \in k$ by $m_i$.


\begin{prop}\label{divyp=2}
Suppose that $p=2$.
Then, if $\infty$ is a branch point, the divisor of $y$ is
\begin{equation*}
\di(y) = 
 {\displaystyle \sum_{i=1}^l} m_i[Q_i] -(2g+1)[P_\infty].
\end{equation*}
On the other hand, if $\infty$ is not a branch point, then, after possibly swapping the notations for $P_\infty$ and $P_\infty'$ for the two points in $\pi^{-1}(\infty)$, we have
\begin{equation*}
 \di(y) = {\displaystyle \sum_{i=1}^l} m_i[Q_i] +(g+1-\deg(F(x)))[P_\infty] - (g+1)[P_\infty'].
\end{equation*}
\end{prop}
\begin{proof}
We first show that $\di_0(y)$, the divisor of the zeroes of $y$, is $\sum_{i=1}^l m_i [Q_i]$.

It is clear that the zeroes of $y$ can only occur in the affine part of the curve $X$ defined by \eqref{definep=2} i.e. in $U_\infty$.
Suppose $P\in U_\infty$.
If $\left. F \right|_P \neq 0$ then it follows that $y|_P \neq 0$, since $F(x) = y (y + H(x))$.
Hence $\di_0(y)$ has zero coefficients for any point in $U_\infty$ other than the $Q_i$.

Suppose that $P= Q_i = (\gamma_i, 0)$ is an unramified point in $U_\infty$.
Then $H(\gamma_i) \neq 0$ and $\left. y \right|_P = 0$, so $y + H(x)$ is a unit at $P$.
Since $y(y+H(x)) = F(x)$ we find that
\begin{equation*}
\ord_P(y) = \ord_P\left( \frac{F(x)}{y + H(x)} \right) = \ord_P(F(x)) = m_i.
\end{equation*}

We now look at when $P = Q_i = (\gamma_i, 0)$ is a ramification point.
Since $H(x)$ and $H'(x)^2F(x) + F'(x)^2$ cannot share roots it follows that $m_i = 1$.
Hence the function $\tilde F(x) := (x- \gamma_i)^{-1}F(x)$ is a unit at $P$.
We let $\tilde H(x) = (x- \gamma_i)^{-1}H(x)$.
Now $y^2 = F(x) - y H(x) = (x- \gamma_i) (\tilde F(x) - y \tilde H(x))$, and hence
\[
\ord_P(y^2 ) = \ord_P(x-\gamma_i) + \ord_P(\tilde F(x) - y \tilde H(x)).
\]
Since $\ord_P(x-\gamma_i) = 2$ and $\ord_P(\tilde F(x) - y \tilde H(x)) \geq 0$ we know that $\ord_P(y) \geq 1$.
Hence $\left. (y \tilde H(x)) \right|_P = 0$, and since $\tilde F(x)$ is a unit at $P$, we conclude that $\tilde F(x) - y \tilde H(x)$ is a unit at $P$.
Hence $\ord_P(y^2) = 2$, and so $\ord_P(y) = 1 = m_i$.
It follows that $\di_0(y) =  \sum_{i=1}^l m_i [Q_i]$.

We now consider the poles of $y$.
If $\infty$ is a branch point we know that $\deg(F(x)) = 2g+1$ and hence $\sum_{i=1}^l m_i = 2g+1$.
Since $y$ can only have a pole at $P_\infty$, we conclude that the degree of this pole is $2g+1$, and hence
\[
\di(y) = \sum_{i=1}^l m_i [Q_i] - (2g+1)[P_\infty]
\]
if $\infty$ is a branch point.

If $\infty$ is not a branch point then there are two points at which $y$ may have a pole, namely $P_\infty$ and $P_\infty'$.
We consider three cases, and recalling that $\ord_{P_\infty}(y + H(x)) = \ord_{P_\infty'}(y)$, using the automorphism given by $y \mapsto y+H(x)$.


Firstly, we suppose that $\ord_{P_\infty}(y) < -(g+1)$.
Then $\ord_{P_\infty}(y) < \ord_{P_\infty}(H(x))$ and hence $ \ord_{P_\infty}(y) = \ord_{P_\infty}(y+H(x))$.
But this contradicts $\ord_{P_\infty}(y) + \ord_{P_\infty}(y+H(x)) = \ord_{P_\infty}(F(x))$, since the left hand side is less than $-2(g+1)$, which is the minimum value of the right hand side.

We now suppose that $\ord_{P_\infty} (y) = -(g+1)$. Since $y(y+H(x)) = F(x)$ it follows that $-(g+1) + \ord_{P_\infty}(y+H(x)) = \ord_{P_\infty}(F(x))$, and hence $\ord_{P_\infty'}(y) = \ord_{P_\infty}(y+H(x)) = -\deg(F(x)) + g + 1$.

Finally, if $\ord_{P_\infty}(y) > -(g+1)$, then since $\ord_{P_\infty}(H(x)) = -(g+1)$ it follows that $\ord_{P_\infty'}(y) = \ord_{P_\infty} (y+H(x)) = -(g+1)$.
Then, from a computation similar to that in the previous paragraph we see that $\ord_{P_\infty}(y) = -\deg(F(x)) + g +1$, completing the proof.
\end{proof}



Finally, we will compute the divisor of $dy$, reminding the reader that we are considering the characteristic two case.
We start by noting that when we take the differential of \eqref{definep=2} we obtain
\[
dF = d\left(y^2 + yH \right) = d(yH) = Hdy + ydH
\]
and from this it follows that
\begin{equation}\label{divdyp=2}
dy = \frac{F'-yH'}{H}dx.
\end{equation}


To state the ramification points, and for later use, we suppose that 
\begin{equation*}
H(x) = \prod_{i=1}^d (x-a_i)^{n_i} = x^d + b_{d-1}x^{d-1} + \ldots + b_1x + b_0
\end{equation*}
for some $a_i, b_i \in  k$, $d \leq g+1$ and $n_i \in \mathbb N$.
Then the $a_i \in \mathbb A_k^1 \subset \mathbb P_k^1$ are the branch points of $\pi$ and we let $P_i \in X$ be the corresponding ramification points above $a_i$.
Given this, we can write the ramification divisor as
\[
R = \sum_{i=1}^d 2n_i[P_i] + (g+1-d)D_\infty.
\]
Details can be found in \cite[\S 6]{faithfulaction}.

We now take a brief detour to recall some of the details of Serre duality.
We let $\Omega_{K(X)}$ be the module of differentials of $K(X)$ over $k$.
\begin{lem}
The following sequence is exact
\begin{equation}\label{dualityses}
0 \rightarrow \hzero \ra \Omega_{K(X)} \ra \bigoplus_{P \in X}\Omega_{K(X)}/\Omega_{X,P} \xrightarrow{\delta} H^1(X,\Omega_X) \ra 0.
\end{equation}
\end{lem}
\begin{rem}
Note that the sequence formed by the last three terms can be found in \cite[Pg. 248]{hart}.
\end{rem}
\begin{proof}
We let $\underline{\Omega}_{K(X)}$ be the constant sheaf of $\Omega_{K(X)}$.
Then the short exact sequence
\begin{equation}\label{serredualityses}
0 \rightarrow \Omega_X \rightarrow \underline{\Omega}_{K(X)} \rightarrow \underline{\Omega}_{K(X)}/\Omega_X \rightarrow 0
\end{equation}
is a flasque resolution of $\Omega_X$ by \cite[Chap II, ex. 1.16]{hart}.

We view the module $\Omega_{K(X)}/\Omega_{X,P}$ as a sheaf on the singleton $\{P\}$, which has a natural embedding $i\colon \{P\} \rightarrow X$.
Hence for each $P\in X$ we have the induced sheaf $i_*\left(\Omega_{K(X)}/\Omega_{X,P}\right)$ on $X$.
If we consider the direct sum of these induced sheaves over all points $P\in X$ we have the following isomorphism
\begin{equation}\label{sheafisomorphism}
\underline{\Omega}_{K(X)}/\Omega_X\cong \bigoplus_{P\in X} i_*\left(\Omega_{K(X)}/\Omega_{X,P}\right).
\end{equation}
To explain this isomorphism we first construct a map from $\underline{\Omega}_{K(X)}/\Omega_{X,P}$ in to the product $\prod_{P \in X} i_*\left(\Omega_{K(X)}/\Omega_{X,P}\right)$, and then showing that this map has finite support.

Given $i\colon \{P\} \hookrightarrow X$ we have the following equalities
\begin{align*}
i^{-1}\left(\underline{\Omega}_{K(X)}/\Omega_X\right) & = \left(\underline{\Omega}_{K(X)}/\Omega_X\right)_P \\
& = \underline{\Omega}_{K(X),P}/\Omega_{X,P} \\
& = \Omega_{K(X)}/\Omega_{X,P}.
\end{align*}\todo{blame it on the adjoint}
It follows that for each $P\in X$ we can map $f \in \underline{\Omega}_{K(X)}/\Omega_X$ to $i_*i^*(f) \in i_*\left( \Omega_{K(X)}/\Omega_{X,P}\right)$, and that this map will be an isomorphism.
We recall that for any $f \in \Omega_{K(X)}$ we have $f \in \Omega_{X,P}$  for all but a finite number of points $P \in X$, and hence the image of any such $f$ in $\prod_{P\in X} i_*\left( \Omega_{K(X)}/\Omega_{X,P}\right)$ is zero in almost all factors.
In particular, the image lies in $\bigoplus_{P \in X} i_*(\Omega_{K(X)}/\Omega_{X,P} ) \subset \prod_{P\in X} i_*\left( \Omega_{K(X)}/\Omega_{X,P}\right)$.
Since the map $\underline{\Omega}_{K(X)}/\Omega_X \cong \bigoplus_{P \in X}i_*(\Omega_{K(X)}/\Omega_{X,P}$ given by $f \mapsto (i_*i^*(f))_{P \in X}$ is an isomorphism of stalks it follows that it is also an isomorphism at the level of sheaves, as claimed in \eqref{sheafisomorphism}.


Now we can take the \cech complex of \eqref{serredualityses} over the cover $\cU$, yielding
\begin{equation}\label{dualitydiagram2}
\xymatrix{\Omega_X(U_0)\times\Omega_X(U_\infty) \ar@{^{(}->}[r] \ar[d]^{d_1} & \underline{\Omega}_{K(X)} \times \underline{\Omega}_{K(X)} \ar[d]^{d_2} \ar@{->>}[r] & \bigoplus \limits_{P \in U_0} \Omega_{K(X)}/\Omega_{X,P} \times \bigoplus \limits_{P \in U_\infty} \Omega_{K(X)}/\Omega_{X,P} \ar[d]^{d_3} \\
\Omega_X(U_0 \cap U_\infty) \ar@{^{(}->}[r]  & \underline{\Omega}_{K(X)} \ar@{->>}[r] & \bigoplus \limits_{P\in U_0 \cap U_\infty} \Omega_{K(X)}/\Omega_{X,P} }
\end{equation}
We can now apply the snake lemma to this commutative diagram.
In the next paragraph we conclude the proof by showing that the exact sequence that arises from the applying snake lemma is precisely the sequence \eqref{dualityses} in the statement of the lemma.\todo{check underlining}

The fact that $H^0(X,\Omega_X) \cong \ker(d_1)$ and $H^1(X,\Omega_X) \cong {\rm coker}(d_1)$ follows from the above discussion of \cech cohomology.
The map $d_2\colon \Omega_{K(X)} \times \Omega_{K(X)} \ra \Omega_{K(X)}$ is the \cech differential given by $(\omega_1,\omega_2) \mapsto \omega_1 - \omega_2$.
Hence the map $\omega \mapsto (\omega, \omega)$ gives an isomorphism from $\Omega_{K(X)}$ to $\ker(d_2)$.
Finally, $d_3$ is defined by $(\omega_0, \omega_\infty) \mapsto \omega_0|_{U_0 \cap U_\infty} - \omega_\infty|_{U_0 \cap U_\infty}$.
Hence the kernel of $d_3$ is formed of pairs $(\omega_0, \omega_\infty) \in \bigoplus_{P \in U_0} \Omega_{K(X)}/\Omega_{X,P} \times \bigoplus_{P \in  U_\infty} \Omega_{K(X)}/\Omega_{X,P}$ such that $\omega_0$ and $\omega_\infty$ agree on $U_0 \cap U_\infty$.
It follows that the map $\bigoplus_{P \in X} \Omega_{K(X)}/\Omega_{X,P}\ra\ker(d_3)  $ given by 
\begin{equation*}
(\omega_P)_{P \in X} \to \left( (\omega_P)_{ P \in U_0}, (\omega_P)_{P \in U_\infty}) \right)
\end{equation*}
is an isomorphism.\todo{This is needed for the proof - it shows it is a short exact sequence}
\end{proof}


We now recall the definition of the trace map $t\colon H^1(X,\Omega_X) \ra k$.
Given a point $P\in X$ we have the residue map $\res_P \colon \Omega_{K(X)} \ra k$, defined by 
\begin{enumerate}
\item $\res_P(\omega) = 0$ for all $\omega \in \Omega_P$;
\item $\res_P(f^ndf)=0$ for all $f \in K(X)^*$, all $n \neq 1$;
\item $\res_P(f^{-1}df) = \ord_P(f)$, where $\ord_P(f)$ is the order of $f$ at $P$;
\end{enumerate}
as in \cite[Chap III, Thm. 7.14.1]{hart}.
This also gives a well defined map on the quotient $\Omega_{K(X)}/\Omega_P$, since elements of $\Omega_P$ have zero residue at $P$, and so $\Omega_P \subseteq \ker \left(\res_P\right)$.
Since we know that $\sum_{P\in X}\res_P(\omega) = 0$ for any $\omega \in \Omega_K(X)$ by the residue theorem \cite[Chap. III, Thm. 7.14.2]{hart},  it follows that the map $t \colon \bigoplus_{P \in X} \Omega_{K(X)}/\Omega_P \ra k$ given by $(\omega_P)_{P \in X} \mapsto \sum_{P\in X} \res_P(\omega_P)$ vanishes on the image of $\Omega_{K(X)}$.
Hence this map is well defined on the quotient, which is $H^1(X,\Omega_X)$, by Lemma \ref{dualityses}.
Given some $\bar \omega$ in the quotient $H^1(X,\Omega_X)$, which can be lifted to $\omega \in \bigoplus_{P \in X} \Omega_K(X)/\Omega_P$, we can define the trace map $t$ by
\[
t \colon H^1\left(X, \Omega_X\right) \ra k,\ \bar \omega \mapsto \sum_{P \in X} \res_P(\omega).
\]\todo{check this}

The following lemma will make it easier to compute $t(\bar \omega)$ for some $\bar \omega \in H^1(X,\Omega_X)$.
\begin{lem}\label{tracemaplemma}
For any $\bar \omega \in H^1(X,\Omega_X)$, with representative $\omega \in \bigoplus_{P \in X} \Omega_{K(X)}/\Omega_P$, we have the following equality:
\[
t(\bar \omega) = \sum_{P \in \pi^{-1}(\infty)}\res_P(\omega).
\]
\end{lem}
\begin{proof}
The proof follows from a diagram chase on \eqref{dualitydiagram2}.
Given a cocycle in $\bar \omega \in \hone$ we take a representative $\omega \in \Omega_X(U_0 \cap U_\infty)$.
This then injects in to $\underline{\Omega}_{K(X)}$, and since $d_2$ is surjective we can choose an element of $\underline{\Omega}_{K(X)} \times \underline{\Omega}_{K(X)}$ mapping to $\omega$.
In particular, we could choose $(\omega,0)$.
This then maps to 
\[
\psi = ((\bar{\omega}|_P)_{P\in U_0}, 0) \in \left( \bigoplus_{P \in U_0} \Omega_{K(X)}/\Omega_{X,P}\right) \times \left( \bigoplus_{P \in U_\infty} \Omega_{K(X)}/\Omega_{X,P} \right).
\]
Moreover, by commutativity of the diagram $\psi \in \ker(d_3)$.
Since $\omega$ is regular on $U_0 \cap U_\infty$, and $0$ is regular on $U_\infty$, then $\res_P(\psi) = 0$ for all $P \in U_\infty$.
Hence $t(\bar \omega) = \sum_{P \in X}\res_P(\bar \omega) = \sum_{P \in \pi^{-1}(\infty)} \res_P(\bar \omega) = \sum_{P \in \pi^{-1}(\infty)} \res_P(\omega)$.
\end{proof}

We now use the trace map to define a pairing between the $k$-vector spaces $\hone$ and $\hzero$.
There exists a canonical map 
\begin{equation}\label{productmap}
\hzero \times \hone \ra H^1\left(X, \Omega_X\right), \ (\omega, f) \mapsto f  \omega|_{U_0 \cap U_\infty},
\end{equation}
where the product $f \omega|_{U_0\cap U_\infty}$ is just the usual product of a function and a differential, but restricted to $U_0 \cap U_\infty$, and we consider $H^1(X,\Omega_X)$ and $\hone$ as $\check{H}^1(X,\Omega_X)$ and $\cechhone$ respectively.
We check that this map is well defined.
All elements of $\hzero$ and $\hone$ are regular on $U_0 \cap U_\infty$.
Hence the product $f \omega|_{U_0 \cap U_\infty}$ in \eqref{productmap} is regular on $U_0 \cap U_\infty$.
Moreover, if $f \in \cO_X(U_0 \cap U_\infty)$ is a representative of zero in $\hone$, i.e.~$f = f_0 - f_\infty$ for $f_i \in \cO_X(U_i)$, and $\omega  \in \hzero$ then $f \omega|_{U_0 \cap U_\infty} = f_0\omega|_{U_0 \cap U_\infty} - f_\infty \omega|_{U_0 \cap U_\infty}$, and $f_i \omega|_{U_0\cap U_\infty} \in \Omega_X(U_i)$ for $i\in \{0, \infty\}$.
Hence $f\omega$ is zero in $H^1(X,\Omega_X)$ and the map is well defined.

\begin{thm}\label{serredualitytheorem}
Via the pairing $\langle , \rangle$, the $k$-vector spaces $\hone$ and $\hzero$ are dual to each other.
\end{thm}
\begin{proof}
This is in fact a specialisation of \cite[Thm. 2, Chap. II]{algebraicgroupsandclassfields}.
\end{proof}

We now combine the product map in \eqref{productmap} with the trace map $t$ to get a map 
\[
(\omega, f) \to \langle \omega, f \rangle := t \left( f \omega |_{U_0 \cap U_\infty}\right) , \hzero \times \hone \ra k. 
\]
Thus if we fix any $\omega \in \hzero$ we produce a map $\theta(\omega)\colon \hone \ra k$, given by $\theta(\omega)(f) = \langle \omega , f\rangle$.
Similarly, if we fix any $f \in \hone$ then we get a map $\psi(f) \colon \hzero \ra k$.
Now $\psi$ and $\theta$ are isomorphisms and are dual to each other: in particular, this means that given a basis $e_1, \ldots, e_n$ of $\hzero$, we can find a basis $f_1, \ldots , f_n$ of $\hone$ such that $\theta(e_i)(f_i) = 1$ for all $1 \leq i \leq n$ and $\theta(e_i)(f_j) = 0$ if $i \neq j$ (and similarly for $\psi$).
Hence we have the following lemma.


We now use this fact to compute a basis of $H^1(X,\cO_X)$, and then give a basis of $\derhamhone$ in the following theorem.

\begin{prop}\label{basish1}
 Via the isomorphism \eqref{cechhone} the residue classes of $\frac{y}{x}, \ldots , \frac{y}{x^g} \in K(X)$, restricted to $U_0 \cap U_\infty$, form a basis of $H^1(X,\cO_X)$.
\end{prop}
\begin{proof}
We start by considering the case $p \neq 2$ and first check that the functions $\frac{y}{x}, \ldots, \frac{y}{x^g}$ are indeed regular on $U_0 \cap U_\infty$ (as required by \eqref{cechhone}) by computing their divisors.
From \eqref{divxp=2} and \eqref{pnot2divisors} we see that
\begin{align}\label{divisorofyoverx}
\di \left( \frac{y}{x^i} \right) & = \di (y) - \di ( x^i) \nonumber \\
& = R - (g+1)D_\infty - iD_0 + iD_\infty \nonumber \\
& = R - iD_0 - (g+1 - i)D_\infty.
\end{align}
Since $R$ is a positive divisor this is non-negative on $U_0 \cap U_\infty$ if $i\in \{0, \ldots, g-1\}$.


Recall that the differentials $\omega_0 = y^{-1}dx, \ldots, \omega_{g-1} = x^{g-1}y^{-1}dx$ form a basis of $\hzero$ (see, for example, \cite[Chap 7, Prop. 4.26]{liu}).
By Lemma \ref{tracelemma} we know that $\langle x^iy^{-1}dx, yx^{-j} \rangle = \sum_{P \in \pi^{-1}(\infty)}\res_P(x^{i-j}dx)$.
It follows immediately from \cite[Chap. III, Thm. 7.14.1(b)]{hart} that $\sum_{P \in \pi^{-1}(\infty)}\res_P(x^{i-j}dx) = -2$ if and is zero otherwise.
It then follows from Theorem \ref{serredualitytheorem} that the collection of elements $\{ yx^{-j}|_{U_0\cap U_\infty}\}_{ 1 \leq j \leq g}$ form a basis of $\hone$.



We now suppose that $p=2$, and again start by checking that for $i \in \{1, \ldots , g\}$ the function $yx^{-i}$ is regular on $U_0 \cap U_\infty$.
This follows once we compute the divisor of $yx^{-i}$, which is
\begin{equation*}
\di \left( \frac{y}{x^i} \right)  =  
{\displaystyle \sum_{i=1}^l} m_i[Q_i] -iD_0 -(2g+1 - 2i)[P_\infty]
\end{equation*}
if $\infty$ is a branch point and
\begin{equation*}
\di \left( \frac{y}{x^i} \right)  =  
{\displaystyle \sum_{i=1}^l} m_i[Q_i] - iD_0 +(g+1-\deg(F(x)) + i)[P_\infty] - (g+1-i)[P_\infty']
\end{equation*}
otherwise.
These equalities follow from Proposition \ref{divyp=2} and \eqref{divxp=2}.
The divisors are clearly positive on $U_0 \cap U_\infty$.

Next we recall from \cite[Chap 7, Prop. 4.26]{liu} that if $p=2$ a basis of $\hzero$ is given by $\frac{1}{H(x)}dx, \ldots, \frac{x^{g-1}}{H(x)}dx$.
We then deduce from Lemma \ref{tracemaplemma} that
\[
\left \langle \frac{x^i}{H(x)}dx, \frac{y}{x^j} \right \rangle = \res_{P_\infty} \left( \frac{yx^{i-j}}{H(x)}dx \right) + \res_{P_\infty'}\left( \frac{yx^{i-j}}{H(x)} dx \right).
\]
Then recall that in characteristic two we have an involution $\sigma \colon X \ra X$ given by $(x,y) \mapsto (x, y + H(x))$, and that $\res_P(\sigma^*(\omega)) = \res_{\sigma(P)}(\omega)$ for any $P \in X$ and $\omega\in \hzero$.
Then it follows that
\begin{align*}
\sum_{P \in \pi^{-1}(\infty)} \res_P \left( \frac{yx^{i-j}}{H(x)}dx \right) & = \res_{P_\infty} \left( \frac{yx^{i-j}}{H(x)} dx \right) + \res_{P_\infty'}\left( \frac{yx^{i-j}}{H(x)} dx\right) \\
& = \res_{P_ \infty} \left( \frac{yx^{i-j}}{H(x)}dx \right) + \res_{P_ \infty} \left( \frac{(y+H(x))x^{i-j}}{H(x)}dx \right) \\
& = \res_{P_\infty}(x^{i-j}dx),
\end{align*}
since we are assuming that $\cha(k) = 2$.
As in the previous case, it then follows from the definition of $\res_P$ that $\res_{P_infty}(x^{i-j}dx) = -1$ if $i-j = -1$ and is zero otherwise.



If $P_\infty$ is a branch point then we start by computing the divisor of $ \frac{y}{x^j} \cdot \frac{x^i}{H(x)}dx$, using \eqref{divxp=2}, \eqref{differentialdivisor}, \eqref{divisorofH} and Proposition \ref{divyp=2}:
\begin{align*}
\di\left( \frac{yx^{i-j}}{H(x)}dx \right) & = \di(y) + \di(x^{i-j}) + \di( dx) - \di(H(x)) \\
& = \sum_{i=1}^l m_i[Q_i] - (2g+ 1 )[P_\infty] + (i-j)D_0 - (i-j)D_\infty + R - 2D_\infty \\
& \qquad - R + (g+1)D_\infty\\
& = \sum_{i=1}^l m_i[Q_i] + (2j-3-2i)[P_\infty] + (i-j)D_0
\end{align*}
We see that there is a pole of order one at $P_\infty$ if $2j - 3 - 2i = -1$, or equivalently if $j = i+1$.
Hence $\langle \frac{x^i}{H(x)}dx, \frac{y}{x^j} \rangle \neq 0$ in this case.\todo{what is the residue?}

We also need to check that if $j \neq i+1$ then $\langle \frac{x^i}{H(x)}dx, \frac{y}{x^j} \rangle = 0$.
Indeed, if $j-i \geq 2$ then clearly $\frac{yx^{i-j}}{H(x)}dx$ does not have a pole at $P_\infty$.
On the other hand, if $j-i \leq 0$ then the differential $\frac{yx^{i-j}}{H(x)}dx$ is regular on $U_\infty$, and hence the residue on this set is zero.
Since $X \backslash U_\infty = \{P_\infty\}$ it follows from the residue theorem the residue of $\frac{yx^{i-j}}{H(x)}dx$ at $P_\infty$ is also zero.
\end{proof}


\begin{cor}
For each $P \in X$ fix some $f_P \in K(X) / \cO_{X,P}$.
Then there are unique $\alpha_1, \ldots , \alpha_g \in k$ such that if $f_P$ is replaced by $f_P - \left( \alpha_1 \frac{y}{x} + \ldots + \alpha_n \frac{y}{x^n} \right)$ for all $P \in \pi^{-1}(\infty)$ then there will exist an $f \in K(X)$ such the laurent tail of $f$ at each $P$ in $X$ will be $f_P$.
\end{cor}
\todo[inline]{add mittag-leffler corollary}



In order to state a basis of $\derhamhone$, as well as to shorten the proof of the following theorem, we define the following polynomials. 
We suppose that $1 \leq i \leq g$.
Then when $p\neq 2$ we define
\[
s_i(x) := xf'(x) - 2if(x) \in k[x]
\]
and when $p = 2$ we define
\begin{equation}\label{capitals}
S_i(x,y) := xF'(x) + y(xH'(x) + iH(x))\in k[x]\oplus yk[x] \subseteq k(x,y).
\end{equation}

We now decompose these polynomials into two parts, which will be used in the sequel.
Firstly, we write $s_i(x)$ as $s_i(x) = \phi_i(x) + \psi_i(x)$, where $\psi_i(x)\in k(x)$ and $\phi_i(x) \in k[x]$ are the unique polynomials such that the degree of $\psi_i (x)$ is at most $g+1$ and $x^{g+2}$ divides $\phi_i(x)$.

We define $A_{j,i} \in k$ for $1 \leq j \leq 2g+2$, and $B_{k,i} \in k$ for $0\leq k \leq g+1$ by the equation
\[
S_i(x,y) = A_{2g+2,i}x^{2g+2} + \ldots + A_{1,i} x + y(B_{g+1,i} x^{g+1} + \ldots + B_{1,i} x + B_{0,i}).
\]
Note that many of these coefficients may be zero.
In particular we remark that the $x^i$ term of $xH'(x) + iH(x)$, and hence $B_{i,i}$, is always zero, since it is precisely $x \cdot b_ix^{i-1} + b_i x^i = 2B_{i,i}x^i = 0$.\todo{move back to where it previously was}


We now define the following polynomials:
\begin{equation}\label{Split}
\begin{split}
\Phi_i^x(x) & =  {}_iA_{2g+2}x^{2g+2} + \ldots + {}_iA_{i+1}x^{i+1} \\
\Psi_i^x(x) & =  {}_iA_ix^i + \ldots + {}_iA_1x \\
\Phi_i^y(x) & =  {}_iB_gx^g + \ldots {}_iB_{i+1}x^{i+1} \\
\Psi_i^y(x) & =  {}_iB_{i-1}x^{i-1} + \ldots + {}_iB_1x + {}_iB_0.
\end{split}
\end{equation}
Finally, we define $\Phi_i(x,y) = \Phi_i^x(x) + y \Phi^y_i(x)$ and $\Psi_i(x,y) = \Psi_i^x(x) + y \Psi_i^y(x)$, so that $S_i(x,y) = \phi_i(x,y) + \psi_i(x,y)$.

Viewing $\derhamhone$ as a quotient of \eqref{derhamconditions}, we now give a $k$-vector space basis of $\derhamhone$.
\begin{thm}\label{basis}

If $p \neq 2$ then the residue classes of 
\begin{equation}\label{one}
 \left( \left( \frac{\psi_i(x)}{2yx^{i+1}}\right) dx, \left(\frac{-\phi_i(x)}{2yx^{i+1}}\right) dx, x^{-i}y\right), i=1, \ldots ,g,
\end{equation}
along with the residue classes of 
\begin{equation}\label{two}
 \left( \frac{x^{i}}{y} dx , \frac{x^{i}}{y} dx, 0 \right), i = 0,\ldots ,g-1,
\end{equation}
form a basis of $\derhamhone$.

On the other hand, if $p=2$ then the residue classes of the elements 
\begin{equation}\label{three}
\left( \left(\frac{\Psi_i(x,y)}{x^{i+1}H(x)}\right) dx, \left( \frac{\Phi_i(x,y)}{x^{i+1}H(x)} \right) dx, x^{-i}y \right), i =1, \ldots , g,
\end{equation}
together with the residue classes of 
\begin{equation}\label{four}
\left( \frac{x^{i}}{H(x)} dx, \frac{x^{i}}{H(x)} dx, 0 \right), i=0, \ldots, g-1,
\end{equation}
form a basis of $\derhamhone$.
\end{thm}

Before proving this theorem we use it to prove the following corollary.

\begin{cor}
The action of $G$ on $\derhamhone$ is faithful unless $G$ contains a hyperelliptic involution and $p=2$, in which case the action of the hyperelliptic involution is trivial.
\end{cor}

\begin{proof}
Recall from Proposition \ref{ses} that $H^0(X,\Omega_X)$ injects into $\derhamhone$.
Then if $p \neq 2$ or $G$ does not contain a hyperelliptic involution it follows from \cite[Thm. 4.2]{faithfulaction} that $G$ acts faithfully on $H^0(X,\Omega_X)$, and hence $G$ acts faithfully on $\derhamhone$.

We now suppose that $p=2$ and that $G$ contains a hyperelliptic involution, which we denote by $\sigma$.
By the same theorem from \cite{faithfulaction} as used in the last paragraph, we know that $\sigma$ acts trivially on $\hzero$.

Since $\hzero$ is dual to $\hone$ we know that $\sigma$ acts trivially on $\hone$.
We can study exactly why this is from the view of \cech cohomology, and this will also help to determine the action of $\sigma$ on $\derhamhone$.
If we fix a natural number $i\in \{1, \ldots ,g\}$ then $\sigma$ maps $\frac{y}{x^i}$ to $\frac{y}{x^i} + \frac{H(x)}{x^i}$. 
Now we can split $\frac{H(x)}{x^i}$ as follows, 
\begin{equation*}
\frac{H(x)}{x^i} =  \frac{b_{i-1}x^{i-1} + \ldots + b_1x + b_0}{x^i} - \left( - \frac{x^d + b_{d-1}x^{d-1} + \ldots + b_ix^i}{x^i} \right),
\end{equation*}
and since this is clearly the difference of an element of $\cO_X(U_0)$ and an element of $\cO_X(U_\infty)$ we see that $\frac{H(x)}{x^i}$ is zero in $\hone$.
We let 
\[
H_{1,i}(x) = b_{i-1}x^{i-1} + \ldots + b_1x + b_0
\]
and 
\[
H_{2,i}(x) = -( x^d + b_{d-1}x^{d-1} + \ldots + b_ix^i).
\]

We now consider the action of $\sigma$ on the entries in \eqref{three}.
Firstly we see that
\begin{align*}
\sigma \left( \frac{-\Psi_i(x,y)}{x^{i+1}H(x)} dx\right) & = \frac{-\sigma(\Psi_i(x,y))}{x^{i+1} H(x)} dx \\
& = \frac{-\Psi_i(x,y)}{x^{i+1}H(x)}dx + \frac{H(x)(xH_{1,i}'(x) + iH_{1,i}(x))}{x^{i+1}H(x)}dx\\
& = \frac{-\Psi_i(x,y)}{x^{i+1}H(x)}dx + \frac{xH_{1,i}'(x) + iH_{1,i}(x)}{x^{i+1}}dx \\
& = \frac{-\Psi_i(x,y)}{x^{i+1}H(x)}dx +  \frac{H_{1,i}'(x)}{x^i}dx + \frac{iH_{1,i}(x)}{x^{i+1}}dx \\
& = \frac{-\Psi_i(x,y)}{x^{i+1}H(x)}dx +  \frac{1}{x^i}d\left( H_{1,i}(x) \right) + H_{1,i}(x) d \left( \frac{1}{x^i} \right) \\
& = \frac{-\Psi_i(x,y)}{x^{i+1}H(x)}dx + d\left( \frac{H_{1,i}(x)}{x^i} \right),
\end{align*}
where the second equality follows from \eqref{capitals} and the fact that $\sigma(y) = y + H(x)$.

Similarly we can derive
\begin{equation*}
\sigma \left( \frac{\Phi_i(x,y)}{x^{i+1}H(x)} dx \right)  = \frac{\Phi_i(x,y)}{x^{i+1}H(x)} dx + d \left( \frac{H_{2,i}(x)}{x^i} \right).
\end{equation*}
Lastly, it is clear that $\sigma(x^{-i}y) = x^{-i}(y+H(x))$.


We can now describe exactly how sigma acts on the elements of \eqref{three} using $H_{1,i}(x)$ and $H_{2,i}(x)$:
\begin{multline*}
\sigma \left( \left( \left(\frac{-\Psi_i(x,y)}{x^{i+1}H(x)}\right) dx, \left( \frac{\Phi_i(x,y)}{x^{i+1}H(x)} \right) dx, x^{-i}y \right)\right) = \\
 \left( \left(\frac{-\Psi_i(x,y)}{x^{i+1}H(x)} \right) dx + d\left(\frac{H_{1,i}(x)}{x^i}\right),  \left( \frac{\Phi_i(x,y)}{x^{i+1}H(x)} \right) dx+ d\left(\frac{H_{2,i}(x)}{x^i} \right), \frac{y+H(x)}{x^i} \right).
\end{multline*}
So the action of $\sigma$ on the basis elements in \eqref{three} amounts to adding 
\[
\left( d\left(\frac{H_{1,i}}{x^i}\right), d\left(\frac{H_{2,i}}{x^i}\right), \frac{H(x)}{x^i} \right),
\]
which clearly satisfies the conditions of \eqref{quotient} and hence is zero.
So the action of the involution $\sigma$ on $\derhamhone$ is trivial and hence the action of the group $G$ is not faithful.
\end{proof}

\begin{rem}
We briefly study the action of $\sigma$ on the elements \eqref{one} (when $p\neq 2$).
When $p \neq 2$ then $\sigma$ acts by $(x,y) \mapsto (x,-y)$.
If we let
\[
\nu_i = \left( \left( \frac{\psi_i(x)}{2yx^{i+1}}\right) dx, \left(\frac{-\phi_i(x)}{2yx^{i+1}}\right) dx, x^{-i}y\right)
\]
then 
\begin{equation*}
\sigma(\nu_i) = -\nu_i.
\end{equation*}
Similarly, if 
\[
\eta_i = \left( \frac{x^i}{y}dx, \frac{x^i}{y}dx, 0 \right)
\]
then 
\[
\sigma(\eta_i) = - \eta_i.
\]
Hence $\sigma$ acts by multiplication with $-1$ on $\derhamhone$.
\end{rem}


We now prove Theorem \ref{basis}.

\begin{proof}
We make use of the fact that the short exact sequence in Lemma \ref{ses} splits as a sequence of vector spaces over $k$, and that we know bases of the outer two terms.

It is clear that the elements in \eqref{two} and \eqref{four} are elements of the space \eqref{derhamconditions}. 
In fact, it follows from \cite[Thm 6.1]{faithfulaction} that they are the image of a basis of $H^0(X,\Omega_X)$ in $\derhamhone$.

Moreover, it is obvious that if the elements in \eqref{one} and \eqref{three} are well defined elements of the space \eqref{derhamconditions} then they will map to the basis of $\hone$ given in Lemma \ref{basish1}.
So we need only show that the terms in \eqref{one} and \eqref{three} satisfy the conditions stated in \eqref{derhamconditions}.
For the rest of the proof we fix $i \in \{1, \ldots ,g\}$.


We start with the case $p\neq 2$, and observe that
\begin{align*}
\left(  \frac{\psi_i(x)}{2yx^{i+1}}  - \frac{-\phi_i(x)}{2yx^{i+1}} \right) dx & =  \frac{s_i(x)}{2yx^{i+1}} dx \\
& =  \frac{1}{2yx^i} \left( f' - \frac{2if}{x} \right) dx \\
& =  \frac{x^i}{2y} \left( \frac{f'}{x^{2i}}dx -\frac{2if}{x^{2i+1}} dx \right) \\
& =  \frac{x^i}{2y} \left( fd\left(\frac{1}{x^{2i}}\right) + \frac{1}{x^{2i}}df \right) \\
& =  \frac{x^i}{2y}d(fx^{-2i}) \\
& =  \frac{x^i}{2y} d\left(\left(yx^{-i}\right)^2\right) \\
& =  d(yx^{-i}),
\end{align*}
with the penultimate line following from the defining equation \eqref{definingequationpnot2}.
This shows that the elements in \eqref{one} satisfy $df_{0, \infty} = \omega_0 - \omega_\infty$, one of the conditions of \eqref{derhamconditions}.
Since we saw in the proof of Lemma \ref{basish1} that $\frac{y}{x^i}$ is regular on $U_0\cap U_\infty$ it only remains to show that $\frac{\phi_i}{2yx^{i+1}}dx$ and $\frac{-\psi_i}{2yx^{i+1}}dx$ are regular on $U_\infty$ and $U_0$ respectively.


In order to do this we define $\alpha_{j,i} \in k$ for $0\leq j \leq 2g+2$ to satisfy the equation
\[
s_i(x) = \alpha_{2g+2,i}x^{2g+2} + \ldots + \alpha_{0,i},
\]
so that
\[
\phi_i(x) = \alpha_{2g+2,i}x^{2g+2} + \ldots + \alpha_{g+2,i}x^{g+2} \ {\rm and } \ \psi_i(x) = \alpha_{g+1,i}x^{g+1} + \ldots + \alpha_{0,i}.
\]
Note that it is possible for any of $\alpha_{j,i}$ to be zero. In fact, it is possible for either $\phi_i(x)$ or $\psi_i(x)$ to be zero.
Whenever they are non-zero we denote their degrees as polynomials in $x$ by $d_\phi$ and $d_\psi$ respectively. From the definition of $\phi_i(x)$ and $\psi_i(x)$ we know that $0 \leq d_\psi \leq g+1$ and $g+1 < d_\phi \leq 2g+2$.
\todo{remove d psi? - will keep for now in case useful later}


We now show that $\frac{-\phi_i}{2yx^{i+1}}dx$ and $\frac{\psi_i}{2yx^{i+1}}dx$ are regular on $U_\infty$ and $U_0$ respectively.
We may assume that $\phi_i$ and $\psi_i$ are non-zero, since the zero function is regular everywhere.
The divisor of $\frac{-\phi_i}{2yx^{i+1}}dx$ is

\begin{align*}
\di\left( \frac{-\phi_i}{2yx^{i+1}}dx \right) & =  \di(\phi_i) -\di(y) - \di(x^{i+1}) + \di (dx) \\
& =  \di(\phi_i) - ( R - (g+1)D_\infty) - ((i+1)D_0 - (i+1)D_\infty) \\
& \qquad + (R - 2D_\infty) \\
& =  \left( \di_0\left( \frac{\phi_i}{x^{g+2}}\right) + (g+2)D_0 - d_\phi D_\infty\right) - (i+1)D_0 + (g+i)D_\infty \\
& \geq  \di_0\left( \frac{\phi_i}{x^{g+2}}\right) + (g+2)D_0 - (2g+2)D_\infty - (i+1)D_0 + (g+i)D_\infty \\
& =  \di_0\left( \frac{\phi_i}{x^{g+2}} \right) + (i-g-2)D_\infty + (g-i+1)D_0,
\end{align*}
where the second equality makes use of \eqref{divxp=2} and \eqref{pnot2divisors}.
Since $i \leq g$ the differential $\frac{-\phi_i}{2yx^{i+1}}dx$ is regular on $U_\infty = X\backslash \pi^{-1}(\infty)$.

Similarly the divisor of $\frac{\psi_i}{2yx^{i+1}}dx$ is

\begin{align*}
\di \left( \frac{\psi_i}{2yx^{i+1}}dx\right) & =  \di(\psi_i) - \di(y) - \di(x^{i+1}) + \di (dx) \\
& =  \di (\psi_i ) -(R - (g+1)D_\infty) - ((i+1)D_0 - (i+1)D_\infty) \\ 
& \qquad + (R -2D_\infty) \\
& =  \di(\psi_i) + (g+i)D_\infty -(i+1)D_0 \\
& =  (\di_0(\psi_i) -d_\psi D_\infty) + (g+i)D_\infty -(i+1)D_0 \\
& \geq \left( \di_0(\psi_i) - (g+1)D_\infty \right) + (g+i)D_\infty -(i+1)D_0 \\
& =  \di_0(\psi_i) + (i-1)D_\infty - (i+1)D_0.
\end{align*}
Again, the second equality uses \eqref{divxp=2} and \eqref{pnot2divisors}, and since $i\geq 1$ we conclude that $\frac{\psi_i(x)}{2yx^{i+1}}dx$ is regular on $U_0 = X \backslash \pi^{-1}(0)$, completing the $p\neq 2$ case.


We now suppose that $p=2$.
We remind the reader that this allows us to change signs between positive and negative as we wish.
We see that
\begin{align*}
\left( \left( \frac{ \Psi_i}{x^{i+1}H} \right) + \left( \frac{\Phi_i}{x^{i+1}H} \right) \right) dx & =  \frac{S_i}{x^{i+1}H}dx \\
& =  \left( \frac{F'}{x^iH} + \frac{yH'}{x^iH} + \frac{iy}{x^{i+1}} \right) dx \\
& =  \frac{1}{x^i}\left( \frac{F' + yH'}{H} \right) dx + \frac{iy}{x^{i+1}}dx \\
& =  x^{-i}dy + yd \left( x^{-i}\right) \\
& =  d\left( yx^{-i}\right),
\end{align*}
with the fourth equality following from \eqref{divdyp=2}.
We have also already seen in the proof of Lemma \ref{basish1} that $\frac{y}{x^i}$ is regular on $U_0 \cap U_\infty$.
So in order to prove that for $i\in \{1, \ldots, g\}$ the elements of \eqref{three} are satisfy the conditions of \eqref{derhamconditions} it only remains to show that the differentials $\frac{\Phi_i}{x^{i+1}H(x)}dx$ and $\frac{\Psi_i}{x^{i+1}H(x)}dx$ are regular on $U_\infty$ and $U_0$ respectively.
We denote the degrees of the polynomials defined in \eqref{Split} by $d_{\Phi}^x, d_{\Psi}^x, d_{\Phi}^y$ and $d_{\Psi}^y$.


By \eqref{Split} $\Phi_i(x,y) = \Phi_i^x(x) + y\Phi_i^y(x)$ and $\Psi_i (x,y)= \Psi_i^x(x) + y\Psi_i^y(x)$, and we will use these splittings to show that $\frac{ \Phi_i(x) }{x^{i+1}H(x)}dx$ and $\frac{\Psi_i(x) }{x^{i+1}H(x)}dx$ are regular on $U_\infty$ and $U_0$ respectively.\todo{note that some stuff was removed before this para}

We start by computing the divisor of $\frac{1}{x^{i+1}H(x)}dx$, since it is a common component to all the differentials we need to look at.
This yields
\begin{align*}
\di \left( \frac{1}{x^{i+1}H(x)}dx \right) & = \di(dx) - \di (x^{i+1}) - \di (H(x)) \nonumber \\
& = (R-2D_\infty) - ((i+1)D_0 - (i+1)D_\infty) - (R - (g+1)D_\infty) \nonumber \\
& = (g+i)D_\infty - (i+1)D_0,
\end{align*}
using \eqref{differentialdivisor}, \eqref{divisorofH} and \eqref{divxp=2}.
We now use this along with Proposition \ref{divyp=2} and the polynomials \eqref{Split} to complete the proof.

We begin by computing the divisors associated to $\Phi_i(x,y)$.
Firstly,
\begin{align*}
\di \left( \frac{\Phi_i^x(x) }{x^{i+1} H(x)}dx \right)  = &  \di(\Phi_i^x(x)) -(i+1)D_0 + (g+i)D_\infty\\
 = & \left( \di_0(\Phi_i^x(x)) -d_\Phi^xD_\infty\right) -(i+1)D_0 + (g+i)D_\infty\\
 \geq & \di_0(\Phi_i^x(x)) - (2g+2)D_\infty - (i+1)D_0 + (g+i)D_\infty \\
 = &  \di_0(\Phi_i^x(x)) - (i+1)D_0 + (i-2-g)D_\infty \\
 =  & \di_0 \left( \frac{\Phi_i^x(x)}{x^{i+1}} \right) + (i-g-2)D_\infty.
\end{align*}
From this we see that the differential $\frac{\Phi_i^x(x)}{x^{i+1}H(x)}dx$ is clearly regular on $U_\infty = X \backslash \pi^{-1}(\infty)$.

We now compute the divisor of the other half of $\frac{\Phi_i(x,y)}{x^{i+1}H(x)}dx$, namely
\begin{align*}
\di\left(\frac{y\Phi_i^y(x) dx}{x^{i+1}H(x)} \right)  = & \di(y) + \di(\Phi_i^y(x)) -(i+1)D_0 + (g+i)D_\infty\\
 = & \di(y) + \di_0(\Phi_i^y(x)) - d_\Phi^yD_\infty -(i+1)D_0 + (g+i)D_\infty \\
 \geq & \di(y) + \di_0(\Phi_i^y(x)) - (g+1)D_\infty - (i+1)D_0 + (g+i)D_\infty \\
 = & \di(y) + \di_0\left(\frac{\Phi_i^y(x)}{x^{i+1}} \right) + (i-1)D_\infty.
\end{align*}
From Proposition \ref{divyp=2} we see that $y$ only has poles at points in $\pi^{-1}(\infty)$, and hence this completes the proof that $\frac{\Phi_i(x,y) }{x^{i+1}H(x)}dx$ is regular on $U_\infty = X \backslash \pi^{-1}(\infty)$.

Now we complete the same computations on $\Psi_i(x,y)$, starting with $\Psi_i^x(x)$:
\begin{align*}
\di\left( \frac{\Psi_i^x(x) }{x^{i+1}H(x)}dx \right)  & =   \di(\Psi_i^x(x))  - (i+1)D_0 + (g+i)D_\infty \\
& = (\di_0(\Psi_i^x(x)) -d_\Psi^xD_\infty) - (i+1)D_0 + (g+i)D_\infty \\
 & \geq   \di_0(\Psi_i^x(x) ) - iD_\infty - (i+1)D_0 + (g+i)D_\infty \\
 & =   \di_0(\Psi_i^x(x)) - (i+1)D_0 + gD_\infty,
\end{align*}
and it is clear that the divisor is positive on $U_0 = X \backslash \pi^{-1}(0)$.

For the other half of the differential we need to consider separate cases.
If we assume that $\infty$ is branch point then  using Proposition \ref{divyp=2} we see that
\begin{align*}
\di\left(\frac{y\Psi_i^y(x) }{x^{i+1}H(x)}dx \right)  =  & \di_0(y) - (2g+1)[P_\infty] + \di(\Psi_i^y(x)) - (i+1)D_0 + (g+i)[P_\infty] \\
 =  & \di_0(y) + \di(\Psi_i^y(x)) -(i+1)D_0 + (2i -1)[P_\infty] \\
 = &  \di_0(y) + \di_0(\Psi_i^y(x)) - d_\Psi^y[P_\infty] - (i+1)D_0 + (2i-1)[P_\infty] \\
 \geq &  \di_0(y) + \di_0(\Psi_i^y(x)) -(i-1)[P_\infty] -(i+1)D_0 + (2i-1)[P_\infty] \\
 =   &\di_0(y) + \di_0(\Psi_i^y(x)) -(i+1)D_0 + [P_\infty],
\end{align*}
which is clearly positive on $U_0$.
On the other hand, if $\infty$ is not a branch point then we have
\begin{align*}
\di\left(\frac{y\Psi_i^y(x) }{x^{i+1}H(x)}dx \right)  =  & \di(y) + \di(\Psi_i^y(x)) - (i+1)D_0 + (g+i)D_\infty \\
= & \di(y) + \di_0(\Psi_i^y(x)) - (i+1)D_0 + (g+i - d_\Psi^y)D_\infty \\
\geq & \di(y) + \di_0(\Psi_i^y(x)) - (i+1)D_0 + (g+1)D_\infty. \\
\end{align*}
Since we know from Proposition \ref{divyp=2} that $y$ cannot have a pole of order greater $g+1$ at $P_\infty$ or $P_\infty'$, and only has poles at these points, it follows that the differential $\frac{y\Psi_i^y(x) }{x^{i+1}H(x)}dx$ is regular on $U_0 = X \backslash \pi^{-1}(0)$.
Thus we have completed the proof.


\end{proof}

\section{Computing the residue}
\todo[inline]{check that don't need to change any basis elements after multiplying $s_i$ by $x$}

Introduction needs to be finalised once the proof in the previous section is completed.
In particular, if the exact residue can be computed then we may as well do so.
Currently we are assuming in this section that $\infty$ and 0 are a branch point.

\begin{lem}
Suppose that $\infty\in \mathbb P_k^1$ is a branch point of $\pi$.
If $p \neq 2$ then ${\rm res}_{P_\infty}(\frac{1}{x}dx) = -2$.
If $p=2$ then ${\rm res}_{P_\infty}\left(\frac{y}{xH(x)}dx\right) = 1$.
In particular, this shows that the basis of $\hone$ in Lemma \ref{basish1} is dual to the basis given for $\hzero$ in \cite{faithfulaction}.
\end{lem}

\begin{proof}

We first consider the case $p\neq 2$.
We note that $t:= \frac{y}{x^{g+1}}$ is a uniformising parameter at $P_\infty$ as can be seen by computing the order of $t$ at $P_\infty$ as follows
\begin{align*}
\ord_{P_\infty}(t) & =  \frac{1}{2}\ord_{P_\infty}(t^2) \\
  & =  \frac{1}{2}\ord_{P_\infty}\left( \frac{f}{x^{2g+2}} \right) \\
& =  \frac{1}{2}\ord_{P_\infty}(f(x)) - \frac{1}{2}\ord_{P_\infty}(x^{2g+2})\\
& =  -(2g+1) + (2g+2) \\
& =  1.
\end{align*}

We now write $\frac{1}{x}dx$ in terms of $dt$.
By the quotient rule we know that
\begin{align*}
dt^2 & =  d \left( \frac{f}{x^{2g+2}} \right) \\
& =  \frac{x^{2g+2}f' - (2g+2)x^{2g+1}f}{x^{4g+4}} dx \\
& =  \frac{1}{x^{2g+2}} \left( f' - \frac{(2g+2)f}{x} \right) dx
\end{align*}
from which we conclude that
\[
\frac{1}{x}dx = \frac{2tx^{2g+1}}{\left(\frac{f' - (2g+2)f}{x} \right)} dt.
\]


We now let $p(x) = \left(\frac{(2g+2)f(x)}{x} - f'(x)\right)$, and noting that the coefficient of $x^{2g}$ in $p(x)$ is $(2g+2)a_{2g+1} - (2g+1)a_{2g+1} = a_{2g+1}$, we see that $h(x)$ is a degree $2g$ polynomial in $x$.
We wish to compute the coefficient of $t^{-1}$ in the expansion of $\frac{1}{x}dx$ at $P_\infty$ and for this we require the following expansions
\begin{align*}
\frac{p(x)}{x^{2g+1}} = \frac{a_{2g+1}x^{2g}}{x^{2g+1}} + \ldots = \frac{a_{2g+1}}{x} + \ldots \qquad \text{and} \qquad t^2 = \frac{f}{x^{2g+2} } = \frac{a_{2g+1}}{x} + \ldots
\end{align*}

Since $\ord_{P_\infty}\left(\frac{p(x)}{x^{2g+1}}\right) = 2$ we know that $\frac{p(x)}{x^{2g+1}} = \sum_{j\geq 2} c_j t^j$ for some $c_j\in k$, and from the above computations we can see that $c_2 = 1$.
We also know that $\frac{x^{2g+1}}{p(x)} = \sum_{k\geq -2} d_kt^k$, for some $d_k\in k$, and clearly $d_{-2} = 1$.
Now
\[
\frac{1}{x}dx = \left( -2t \cdot \sum_{i\geq -2} d_it^i\right) dt 
\]
so we see that $\res_{P_\infty}\left( \frac{1}{x} dx\right) = 2$.
This completes the proof of the lemma when $p\neq 2$.

We now turn to the case when $p=2$.
We now wish to compute the residue of $\frac{y}{xH(x)}dx$ at $P_\infty$.
We start by noting that $t = \frac{y}{x^{g+1}}$ is a uniformising parameter at $P_\infty$ which we check by computing the divisor:
\begin{align*}
\di(t) & = \di_0(y) - (2g+1)[P_\infty] -(g+1)D_0 + (g+1)D_\infty \\
& = \di_0(y)-(g+1)D_0 + [P_\infty],
\end{align*}
using \eqref{divxp=2} and Proposition \ref{divyp=2}.
So clearly $t$ is a uniformising parameter at $P_\infty$.

We now wish to write $\frac{y}{xH(x)}dx$ as $r(x,y)dt$ for some $r \in k(x,y)$.
We first write $dy$ in terms of $dx$.
Since
\begin{align*}
0 & =  dy^2 \\
& =  d(F+yH) \\
& =  F'dx + Hdy + yH'dx
\end{align*}
we conclude that
\[
dy = \left( \frac{F'+yH'}{H} \right) dx.
\]

We also rewrite $dt$ as follows:
\begin{align*}
dt & =  d\left( \frac{y}{x^{g+1}} \right) \\
& =  yd\frac{1}{x^{g+1}} + \frac{1}{x^{g+1}}dy \\
& =  \frac{1}{x^{g+1}} \left( \frac{(g+1)y}{x} + \frac{F'+yH'}{H} \right) dx \\
& =  \frac{1}{x^{g+1}} \left( \frac{xF'}{y} + xH' + (g+1)H \right) \frac{y}{Hx} dx.
\end{align*}

In total we then have
\[
\frac{y}{xH(x)}dx = \frac{x^{g+1}y}{S_{g+1}(x,y)}dt
\]
where $S_{g+1}(x,y)$ is as defined in \eqref{capitals}.

Since we saw in the proof of Lemma \ref{basish1} that $\frac{y}{xH(x)}dx$ has a pole of order at $P_\infty$, we have the Taylor expansion $\frac{x^{g+1}y}{S_{g+1}(x,y)} = \sum_{i\geq -1} c_i t^i$, with $c_i \in k$, and we wish to compute $c_{-1}$.
We shall do this by computing the coefficient $d_1$ of $t$ in the expansion $\frac{S_{g+1}(x,y)}{x^{g+1}y} = \sum_{i\geq 1}d_it^i$.
We can split up $\frac{S_{g+1}(x,y)}{x^{g+1}y}$ into three terms, namely $\frac{xF'}{x^{g+1}y}$, $\frac{yxH'}{x^{g+1}y}$ and $\frac{(g+1)yH}{x^{g+1}y}$.
Each of these terms can of course be written as a power series in $t$.
However, to compute $d_1$ we need only compute the coefficient of $t$ in these power series.
Since $\infty$ is a branch point then $H$ has degree at most $g$ (and hence $xH'$ also has degree at most $g$).
It then follows that the order of both $\frac{xH'}{x^{g+1}}$ and $\frac{(g+1)H}{x^{g+1}}$ at $P_\infty$ is at least 2, and hence the coefficient of $t$ in their power series expansion is zero.
The order of $\frac{xF'}{x^{g+1}y}$ at $P_\infty$ on the other hand is precisely 1, and hence the coefficient of $t$ in the power series expansion of $\frac{xF'}{x^{g+1}y}$ will determine $d_1$.


Suppose that $F = \alpha_{2g+1}x^{2g+1} + \alpha_{2g}x^{2g} + \ldots + \alpha_1x^1 + \alpha_0$.
Then $xF'= \alpha_{2g+1}x^{2g+1} + \alpha_{2g-1}x^{2g-1} + \ldots + \alpha_1x^1$; i.e. the terms with an even power are removed.

So the only term in $\frac{xF'}{x^{g+1}y}$ of order 1 at $P_\infty$ is $\frac{\alpha_{2g+1}x^{2g+1}}{x^{g+1}y} = \frac{\alpha_{2g+1}x^{g}}{y}$.
Since
\[
\frac{\alpha_{2g+1}x^g}{y} = \frac{\alpha_{2g+1}}{x}t^{-1}
\]
and $\frac{1}{x} = \sum_{i\geq 2}e_it^i$ for some $e_i \in k$, if we compute $e_2$ then we will have effectively computed $d_1$.
Now $t^2 = \frac{F }{x^{2g+2}}+ \frac{Hy}{x^{2g+2}}$, and clearly $\frac{Hy}{x^{2g+2}}$ has no monomial term of the form $\frac{c}{x}$ with $c \in k$.
On the other hand
\[
\frac{F}{x^{2g+2}} = \frac{\alpha_{2g+1}}{x} + \ldots
\]
Hence we conclude that $e_2 = \frac{1}{\alpha_{2g+1}}$.
It follows that $d_1 = \alpha_{2g+1} \cdot \frac{1}{\alpha_{2g+1}} = 1$.
We finally use this to conclude that $c_{-1} = \frac{1}{d_{1}} = 1$.




\end{proof}
\begin{comment}

\section{Stability of modular curves in characteristic greater than 2}
In this section we consider some modular hyperelliptic curves and how their automorphism groups at on the first de Rham cohomology group of the curves.
The automorphism groups of these curves can be found in \cite{automorphismshyperellipticmodular}.
We start by considering the curve $X_0(22)$, which can defined by the equation
\begin{equation*}
y^2 = (x^3+ 4x^2 + 8x + 4)(x^3 + 8x^2 + 16x + 16).
\end{equation*}
The automorphism group of this curve is, in general, $\mathbb Z_2 \times \mathbb Z_2$, with the action given by 
\begin{align}
\omega_2\colon (x,y) & \mapsto \left( \frac{4}{x} , \frac{8y}{x^3} \right), \\
\omega_{11}\colon (x,y) & \mapsto (x, -y ).
\end{align}
Note that we are using the notation of \cite{automorphismshyperellipticmodular} for consistency.
We now study how this group action affects the basis of $\derhamhone$.
In particular, we show that the basis associated to $\hone$ in $\derhamhone$ is not stable under the action of the automorphism group.
For each $i \in \{ 1, \ldots , g\}$ we let 
\begin{equation*}
\nu_i = \left( \left( \frac{-\psi_i(x)}{2yx^i} \right) dx , \left( \frac{\phi_i(x)}{2yx^i} \right) dx, x^{-i}y \right).
\end{equation*}
We then note that 
\[
\omega_2(dx) = d( \omega_2(x)) = d\left( \frac{4}{x} \right) = \frac{-4}{x^2}dx.
\]
If we recall that 
\[
s_i(x) = \alpha_{2g+1}^ix^{2g+1} + \ldots + \alpha_0^i
\]
then we can see that the $\omega_2$ acts as follows:
\begin{equation}\label{omega11action}
\begin{split}
\omega_2 (\nu_i) & = \left( -\sum_{j = g+1}^{2g+1} \left( \frac{\alpha_j^i 4^{j-i}x^3}{2x^{j-i}y} \right) \frac{-4}{x^2} dx, \sum_{j = 0}^{g} \left(\frac{\alpha_j^i 4^{j-i} x^3}{2x^{j-i}y} \right) \frac{-4}{x^2}dx , \frac{8yx^i}{4^i x^3} \right) \\
& = \left( \sum_{j=g+1}^{2g+1} \left( \frac{\alpha_j4^{i-j+1}}{2yx^{j-i-1}} \right)dx, \sum_{j = 0}^{g} \left( \frac{-\alpha_j^i 4^{j-i+1}}{2yx^{j-i-1}} \right) dx, \frac{2yx^{i-3}}{4^{i-1}} \right).
\end{split}
\end{equation}

Now it appears that these terms are similar to those of $s_i \left( \frac{4}{x} \right)$, so it would be good to see if we can somehow write $y^2$ in terms of  some function of $\omega_2(x) = \frac{4}{x}$.
We can do this, since of course $\omega_2(y)^2 = \omega_2(y^2) = \omega_2(f(x)) = \omega_2\left( f \left(\frac{4}{x} \right) \right)$.
From this we obtain the identity
\begin{equation*}
y^2 = \frac{x^6 f \left(\frac{4}{x} \right)}{8^2}.
\end{equation*}
From this we obtain
\begin{align*}
dy & = \frac{1}{2y \cdot 8^2} \left( f'\left(\frac{4}{x} \right) x^6 d\left(  \frac{4}{x} \right) + 6 f\left( \frac{4}{x} \right) x^5 dx \right) \\
& = \frac{1}{2y \cdot 8^2} \left( f'\left(\frac{4}{x} \right) \cdot (-4 x^4) + 6 f\left( \frac{4}{x} \right) x^5  \right) dx \\
\end{align*}

We can then use this to compute $d (\omega_2(x^{-i}y)) = d \left( \frac{8yx^{i-3}}{4^i} \right)$, which we now do:
\begin{align*}
d\left( \frac{8yx^{i-3}}{4^i} \right) & = \frac{8}{4^i} \left( x^{i-3}dy + ydx^{i-3} \right) \\
& = \frac{8}{4^i} \left( \frac{x^{i-3}}{2y8^2} \left( \frac{-4}{x^2} \cdot f'\left( \frac{4}{x}  \right) x^6 + 6x^5 f\left( \frac{4}{x} \right) \right) + (i-3)x^{i-4}y \right) dx \\
& = \frac{8}{4^i} \left( \frac{1}{2y8^2} \left(\frac{-4}{x^2} \cdot  f'\left( \frac{4}{x} \right) x^{i+3} + 6x^{i+2} f\left( \frac{4}{x} \right) \right) + \frac{(i-3)x^{i-4}x^6f \left( \frac{4}{x} \right)  }{y8^2} \right) dx \\
& = \frac{8}{4^i} \cdot \frac{x^{i+3}}{2y8^2} \left(\frac{-4}{x^2} \cdot  f' \left( \frac{4}{x} \right)  + \frac{6}{x} f \left( \frac{4}{x} \right)  + \frac{2(i-3)f \left( \frac{4}{x} \right)  }{x} \right) dx \\
& = \frac{x^{i+3}}{2 \cdot 4^i \cdot 8 \cdot y} \left(\frac{-4}{x^2} \cdot  f'\left( \frac{4}{x} \right)  + \frac{ 2i f \left( \frac{4}{x} \right) }{x} \right) dx \\
& = \frac{x^{i+1}}{4^{i+1}y} \left( -f' \left(\frac{4}{x} \right) + \frac{2if \left( \frac{4}{x} \right) x}{4} \right) dx.
\end{align*}
\todo[noline]{Apart from the minus sign in front of $f' \left( \frac{4}{x} \right)$ and a factor of two almost matches up - check computations}Note that the function before $dx$ here is precisely $\omega_2\left(\frac{s_i(x)}{2yx^i} \right)$.

This (almost) shows that
\[
\left( \omega_2 \left( \left( \frac{s_i(x)}{2yx^i} \right) dx \right) , 0, \omega_2( x^{-i}y)\right)
\]
is the zero element in $\derhamhone$.

We will now finally show that the basis of $\hone$ in $\derhamhone$ is not stable under the group action --- in particular, it is not stable under the action of $\omega_2$.
Now we fix $i = g-1$.
We can then clearly see from \eqref{omega11action} that if $\alpha_{g}^i$ is zero all the $x$ terms in the second co-ordinate will have degree at most $g-1$.
\begin{note}
By the action of $\omega_2$ we have $\frac{8^2y^2}{x^6} = f\left(\frac{4}{x} \right)$.
\end{note}

\begin{note}
For $\infty$ is not a branch point the same basis should work. 
Consider $y^2 = f(x)$ where the degree of $f(x)$ is $2g+2$ (i.e. $\infty$ is not ramified), with projection map $\pi : X \rightarrow \mathbb P_k^1$.
Then we can take the cover to be $U_0 := X \backslash \{\pi^{-1}(0)\}$ and $U_\infty := X \backslash \{\pi^{-1}(\infty)\}$.
Then we have $\frac{x^idx}{y} \cdot \frac{y}{x^{j}} = x^{i-j}dx$.
The divisor of this is then $(i-j)D_0 -(i-j)D_\infty + R - 2D_\infty$.
Then if $i-j = -1$ this has poles at at all points supported at by $D_0$ and $D_\infty$.


If $p=2$ and $\infty$ is not a branch point then we need to consider the defining equation $y^2 - h(x)y = f(x)$ where $\deg(h)=g+1$, and $\deg(f)$ can be anything from 0 to $2g+2$.
From work (which I think was not in 18 month report), if we suppose that $P_\infty$ and $P_\infty'$ are the two points in the pre-image of $\infty \in \mathbb P_k^1$, then the divisor of $y$ is
\[
\di(y) = \di_0(y) +(g+1 - \deg(f))[P_\infty] - (g+1)[P_\infty'],
\]
up to parity of $P_\infty$ and $P_\infty'$.
So then, after some computation, we get
\begin{multline*}
\di\left(\frac{yx^{i-j}}{H(x)}dx \right) = \di_0(y) + (i-j)D_0 + (g-1-(i-j))D_\infty \\ + (g+1-\deg(f))D_\infty -(g+1)[P_\infty'].
\end{multline*}
So if $i-j = -1$ we clearly have a pole of order 1 at $P_\infty'$.
\end{note}

Now a short note about whether the main theorem still holds if we do not specify the ramification points.
\begin{note}
Clearly when $p\neq 2$ the same basis holds. In the proof we use the fact that $\phi_i$ has a factor of at least $x^{g+1}$, giving a sufficient zero at $P_0$.
This factor is still in the polynomial if we increase its potential degree.
Of course, we may have to replace $P_0$ with $D_0$, but the essential point still holds.
Similarly, after possibly replacing $P_\infty$ by $D_\infty$ we have almost the same argument for $\psi_i$.
So the basis is the same when $p\neq2$.

When $p=2$ the basis is also the same.
We first note that whilst the degree of the highest and lowest ordered terms may change in \eqref{Split}, it is still the case that the coefficient $B_i^i =0$, using exactly the same arguments.
This allows us to use essentially the same arguments, again with the exception of changing $P_\infty$ and $P_0$ to $D_\infty$ and $D_0$.
\end{note}
\end{comment}


\bibliography{biblio}
\bibliographystyle{amsalpha}


\end{document}
