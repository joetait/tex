% !TEX TS-program = pdflatex
% !TEX encoding = UTF-8 Unicode

% This is a simple template for a LaTeX document using the "article" class.
% See "book", "report", "letter" for other types of document.

\documentclass[draft, 11pt]{article} % use larger type; default would be 10pt

\usepackage[utf8]{inputenc} % set input encoding (not needed with XeLaTeX)

%%% Examples of Article customizations
% These packages are optional, depending whether you want the features they provide.
% See the LaTeX Companion or other references for full information.

%%% PAGE DIMENSIONS
\usepackage{geometry} % to change the page dimensions
\geometry{a4paper} % or letterpaper (US) or a5paper or....
% \geometry{margins=2in} % for example, change the margins to 2 inches all round
% \geometry{landscape} % set up the page for landscape
%   read geometry.pdf for detailed page layout information

\usepackage{graphicx} % support the \includegraphics command and options

%\usepackage[parfill]{parskip} % Activate to begin paragraphs with an empty line rather than an indent

%%% PACKAGES
\usepackage{mathtools}
\usepackage{booktabs} % for much better looking tables
\usepackage{array} % for better arrays (eg matrices) in maths
\usepackage{paralist} % very flexible & customisable lists (eg. enumerate/itemize, etc.)
\usepackage{verbatim} % adds environment for commenting out blocks of text & for better verbatim
\usepackage{subfig} % make it possible to include more than one captioned figure/table in a single float
\usepackage[obeyDraft]{todonotes}
% These packages are all incorporated in the memoir class to one degree or another...

%%% HEADERS & FOOTERS
\usepackage{fancyhdr} % This should be set AFTER setting up the page geometry
\pagestyle{fancy} % options: empty , plain , fancy
\renewcommand{\headrulewidth}{0pt} % customise the layout...
\lhead{}\chead{}\rhead{}
\lfoot{}\cfoot{\thepage}\rfoot{}

%%% SECTION TITLE APPEARANCE
\usepackage{sectsty}
\allsectionsfont{\sffamily\mdseries\upshape} % (See the fntguide.pdf for font help)
\usepackage{amsmath}
\usepackage{amsthm}
\usepackage{amsfonts}
\usepackage{mathrsfs}
\usepackage{amsopn}
\usepackage{amssymb}
%\usepackage{natbib}
% (This matches ConTeXt defaults)

%%% ToC (table of contents) APPEARANCE
\usepackage[nottoc,notlof,notlot]{tocbibind} % Put the bibliography in the ToC
\usepackage[titles,subfigure]{tocloft} % Alter the style of the Table of Contents
\renewcommand{\cftsecfont}{\rmfamily\mdseries\upshape}
\renewcommand{\cftsecpagefont}{\rmfamily\mdseries\upshape} % No bold!

%Theorems and stuff
\theoremstyle{plain}
\newtheorem{defn}{Definition}[section]
\newtheorem{thm}[defn]{Theorem}
\newtheorem{cor}[defn]{Corollary}
\newtheorem{lem}[defn]{Lemma}
\newtheorem{prop}[defn]{Proposition}
\newtheorem{ex}[defn]{Example}
\newtheorem*{unnumthm}{Theorem}
\newtheorem{defnlem}[defn]{Definition/Lemma}
\newtheorem{defnthm}[defn]{Theorem/Definition}
\theoremstyle{remark}
\newtheorem*{rem}{Remark}


\newcommand{\cO}{{\cal O}}
\newcommand{\ra}{\rightarrow}
\newcommand{\NN}{{\mathbb N}}
\newcommand{\PP}{{\mathbb P}}
\newcommand{\ZZ}{{\mathbb Z}}
\newcommand{\cL}{{\mathcal L}}
\newcommand{\cA}{{\mathcal A}}
\newcommand{\cD}{{\mathcal D}}
\newcommand{\cU}{{\mathcal U}}
\newcommand{\cech}{\v{C}ech }
\newcommand{\hzero}{{H^0(X,\Omega_X)}}
\newcommand{\hone}{H^1(X,\mathcal{O}_X)}
\newcommand{\cechhone}{\check{H}^1(\mathcal U,\mathcal O_X)}
\newcommand{\derhamhone}{H_{\text {dR}}^1(X/k)}
\newcommand{\cechderhamhone}{\check{H}_{\text {dR}}^1(X/k)}


\DeclareMathOperator{\aut}{Aut}
\DeclareMathOperator{\ord}{ord}
\DeclareMathOperator{\di}{div}
\DeclareMathOperator{\cha}{char}
\DeclareMathOperator{\gal}{Gal}
\DeclareMathOperator{\Tr}{Tr}

%%% END Article customizations

%%% The "real" document content comes below...

\title{Group actions on de Rham cohomology of $X_0(22)$}
\author{J. Tait}
%\date{} % Activate to display a given date or no date (if empty),
         % otherwise the current date is printed 

\begin{document}
\maketitle
\listoftodos
\bigskip

We suppose that $k$ is a field of characteristic three and that $X$ is the curve defined by
\begin{align*}
y^2 = f(x)  & = (x^3 + x^2 + 2x + 1)(x^3 + 2x^2 + x + 1)\\
& = x^6 + 2x^4 + x^3 + 2x^2 + 1.
\end{align*}
As in \cite[Table 2]{automorphismshyperellipticmodular}, this is the modular curve $X_0(22)$, and has genus 2.
We see from \cite[Table 1]{automorphismshyperellipticmodular} that the automorphism group of this curve is $D_6$, the dihedral group of order 12.
In \cite[\S 3.3]{automorphismshyperellipticmodular} we see that this can be written in the form $C_2 \rtimes (C_2 \times C_3)$, where the cyclic groups of order two are generated by
\begin{align*}
\sigma_1 & \colon (x,y) \mapsto (x,-y) \\
\sigma_2 & \colon (x,y) \mapsto \left( \frac{1}{x}, \frac{2y}{x^3} \right)
\end{align*}
respectively (the description of the action is given in \cite[Table 1]{automorphismshyperellipticmodular}).

We see from \cite[\S 3.3]{automorphismshyperellipticmodular} that to describe how the element of order three acts we need our defining equation to be of the form
\[
y^2 = \sum_{i=0}^3 a_{2i}x^{2i}.
\]
If we consider a function field $k(x,y)$, with some relation on $y^2$, then the map sending $x$ to $x-1$ is an isomorphism.
If we apply this map to $K(X)$, the function field of $X$, we get that
\begin{equation}\label{int}
y^2 = x^6 + 2x^4 + 2x^2 + 2.
\end{equation}
If $g(x) = a_sx^s + \ldots + a_0$ where $a_0 \neq 0 \neq a_s$, then we define $g^*(x) = a_0^{-1}x^sf\left( \frac{1}{x} \right)$.
Applying this to \eqref{int} we get $x^6 + x^4 + x^2 + 2$.
Since this is function we will be using we relabel $f(x)$ so that $y^2 = f(x) = x^6 + x^4 + x^2 + 2$, and we henceforth let $X$ be the curve defined by this equation (which is isomorphic to our previous curve).
We check that this equation has no double roots, and hence is actually a hyperelliptic curve, by computing $f'(x) = x^3 + 2x$. 
We then notice that $f(x) = f'(x)^2 + 2$, and hence $f(x)$ and $f'(x)$ are coprime, and hence share no roots.
In particular, this implies that $f(x)$ has no repeated, hence $X$ is a hyperelliptic curve.

Moreover, since it is clear that both $f'(x+1) = f'(x)$ we see that
\[
f(x+1) = f'(x+1)^2 + 2 = f'(x)^2 + 2 = f(x).
\]
Hence $y^2 = f(x)$ is invariant under the action $\sigma \colon x \mapsto x+1$, and in particular this means that if we let $\sigma(y) = y$ then $\sigma$ is a well defined automorphism of $X$, of order three.


Now we wish to see how our element of order three acts on $\derhamhone$ of our curve.
However, since the basis we have already computed in \cite[Thm. 2.3]{derhamactions} is not stable under this action (in particular, the open set $U_0$ is not preserved) we must refine our cover.
Since $\sigma(U_0) = U_1$ we add the open set $U_1$ to our cover and compute a basis of the de Rham cohomology with respect to this new cover.
We let $\cU' = \{ U_\infty, U_1\}$ and $\cU'' = \cU \cup \cU' = \{U_0, U_1, U_\infty\}$.
Note that $\cechderhamhone(\cU')$ can be written in the same form as $\cechderhamhone(\cU)$ in \cite[Pg. 2]{derhamactions}, after changing the appropriate subscripts.
On the other hand, we write $\cechderhamhone(\cU'')$ as the vector space
\begin{multline}\label{sixtupleconditions}
\left\{ (\omega_0, \omega_\infty , \omega_1, f_{0\infty}, f_{01},f_{\infty 1}) | \omega_i \in \Omega_X(U_i), f_{ij} \in \cO_X(U_i \cap U_j), \right. \\ \left. f_{0\infty} - f_{01} + f_{\infty1} = 0, df_{ij} = \omega_i - \omega_j \right\}
\end{multline}
quotiented by the subspace 
\[
\left\{ df_0, df_\infty, df_1, f_0- f_\infty, f_0 - f_1, f_\infty - f_1 | f_i \in \cO_X(U_i)\right\}.
\]
The following proposition will allow us to compute how $\sigma$ acts on $\cechderhamhone(\cU)$, as in \cite[\S 3]{canonicalrepresentation}.
\begin{prop}
The residue classes of
\[
\nu_1 = \left(\frac{1}{yx^2}dx, \frac{x^4 + 2x^2}{y}dx, \frac{x^4 + 2x^3 +  2x^2}{2y(x-1)^3}dx, \frac{y}{x}, \frac{y}{x(x-1)^2}, \frac{2y(x+1)}{(x-1)^2} \right)
\]
and
\[
\nu_2 = \left( \frac{x^2 + 1}{2yx^3}, \frac{2x^3}{y}dx, \frac{x^3 + x^2 + x+ 1}{2y(x-1)^3}dx, \frac{y}{x^2}, \frac{x(y +1)}{x^2(x-1)^2}, \frac{2y}{(x-1)^2} \right)
\]
are well defined elements of $\cechderhamhone(\cU'')$.
Moreover, the canonical projections on to $\cechderhamhone(\cU)$ are the first two basis elements described in \cite[Thm. 2.3]{derhamactions}.
\end{prop}
\begin{proof}
We let $\omega_{ji}$ and $f_{jki}$ be the elements of $\nu_i$, so that
\[
\nu_i = (\omega_{0i}, \omega_{\infty i}, \omega_{1i}, f_{0\infty i}, f_{01i}, f_{\infty 1 i} ).
\]
We first show that the first, second and fourth entry of both $\nu_1$ and $\nu_2$ are those described in \cite[Thm. 2.2]{derhamactions}.
It will then follow that they satisfy the relevant conditions of \eqref{sixtupleconditions}.

First note that $f'(x) = x^3 + 2x$, and hence
\[
s_1(x) = xf'(x) + f(x) = x^6 + 2x^4 + 2 \qquad s_2(x) = xf'(x) - f(x) = 2x^6 + x^2 +1.
\]
Recall that we defined $\psi_i(x)$ and $\phi_i(x)$ to the be the unique polynomials in $k[x]$ summing to $s_i$ such that $\psi_i(x)$ is degree at most $g+1$ and $x^{g+2}$ divides $\phi_i(x)$.
Since $g=2$ it follows that
\[
\psi_1(x) = 2 \qquad {\rm and}\qquad \phi_1(x) = x^6 + 2x^4
\]
and that
\[
\psi_2(x) = x^2 +1 \qquad {\rm and} \qquad \phi_2(x) = 2x^6.
\]

Hence
\[
\omega_{01} = \frac{\psi_1(x)}{2yx^2}dx = \frac{1}{yx^2}dx
\]
and
\[
\omega_{\infty 1} = \frac{-\phi_1(x)}{2yx^2}dx = \frac{x^4+2x^2}{y}dx.
\]
On the other hand
\[
\omega_{02} = \frac{\psi_2(x)}{2yx^3}dx = \frac{x^2+1}{2yx^3}dx
\]
and
\[
\omega_{\infty 2} = \frac{-\phi_2(x)}{2yx^3}dx = \frac{2x^3}{y}dx.
\]

We then show that the rest of the conditions are also satisfied, starting with $\nu_i$.
We first check that the elements are regular on the appropriate sets by computing their divisors, which are as follows
\begin{align*}
\di \left( \frac{x^4 + 2x^3 + x^2}{2y(x-1)^3}dx \right)  = & \di(x^4 + 2x^3 + x^2) -3\di(x-1) - \di(y) + \di(dx) \\
= &\di_0(x^4 + 2x^3 + x^2) - 4D_\infty + 3D_\infty  - R +3D_\infty\\
&  -3D_1 +R - 2D_\infty \\
= & \di_0(x^4 + 2x^3 + x^2) - 3D_1;
\end{align*}
\begin{align*}
\di \left(\frac{y}{x(x-1)^2} \right) & = \di(y) - \di(x) - 2\di(x-1) \\
& = R - 3D_\infty - D_0 + D_\infty - 2D_1 + 2D_\infty \\
& = R - D_0 - 2D_1;
\end{align*}
\begin{align*}
\di \left( \frac{2y(x+1)}{(x-1)^2} \right) & = \di(y) + \di(x+1) - 2\di(x-1) \\
& = R - 3D_\infty +D_1 - D_\infty - 2D_1 + 2D_\infty \\
& = R + D_1 - 2D_\infty - 2D_1.
\end{align*}
These are clearly non-negative on $U_1$, $U_0 \cap U_1$ and $U_\infty \cap U_1$ respectively.

Once we note that 
\[
\frac{y}{x(x-1)^2} = \frac{y}{x} -\left(\frac{y}{x-1} - \frac{y}{(x-1)^2} \right)
\]
and that
\[
\frac{2y(x+1)}{(x-1)^2} = -\left( \frac{y}{x-1} - \frac{y}{(x-1)^2}\right)
\]
then it is clear that
\[
\frac{y}{x} - \frac{y}{x(x-1)^2} + \frac{2y(x+1)}{(x-1)^2} = 0.
\]

Now we check the other relations, starting with
\begin{align}
d \left( \frac{y(x+1)}{(x-1)^2} \right) & = \frac{x+1}{(x-1)^2}dy + y d \left( \frac{x+1}{(x-1)^2} \right) \\
& = \frac{f'(x+1)}{2y(x-1)^2}dx - \frac{yx}{(x-1)^3} dx \\
& = \frac{f'(x^2-1) +fx}{2y(x-1)^3}dx \\
& = \frac{x^7 + 2x^5 + 2x^3}{2y(x-1)^3}dx
\end{align}
and
\begin{align}
\omega_{\infty 1} - \omega_{11} & = \frac{x^4 + 2x^2}{y}dx - \frac{x^4 + 2x^3 + 2x^2}{2y(x-1)^3} dx \\
& = \frac{2(x^4 + 2x^2)(x-1)^3 + 2x^4 + x^3 + x^2}{2y(x-1)^3}dx \\
& = \frac{2x^7 + x^5 + x^3}{2y(x-1)^3}dx\\
& = 2d \left( \frac{y(x+1)}{(x-1)^2}\right).
\end{align}
Hence 
\[
df_{\infty 1 1} = \omega_{\infty 1} - \omega_{ 1 1}.
\]

Similarly we have
\begin{align}
df_{0 1 1} & = d \left( \frac{y}{x(x-1)^2} \right) \\
& = \frac{1}{x(x-1)^2}dy + y d \left( \frac{1}{x(x-1)^2} \right) \\
& = \frac{f'}{2yx(x-1)^2}dx + \frac{y}{x^2(x-1)^3}dx \\
& = \frac{2x^6 + x^5 + x^4 + 2x^3 + 1}{2yx^2(x-1)^3} dx\\
& = \frac{2(x^3-1)}{2yx^2(x-1)^3}dx - \frac{x^6 + 2x^5 + 2x^4}{2yx^2(x-1)^3}dx \\
& = \omega_{01} - \omega_{11}.
\end{align}
So we know that our first element is an element of $\derhamhone (\cU'')$.

We now turn to the second element, where we again start by checking that all the differentials and functions are regular on the correct open sets.
We have
\begin{align}
\di \left( \frac{x^3 + x^2 + x + 1}{2y(x-1)^3} dx \right) = & \di(x^3+x^2+x+1) -\di(y) - 3\di(x-1) + \di(dx) \\
= & \di_0(x^3 + x^2 + x + 1) - 3D_\infty - R + 3D_\infty - 3D_1 \\
& + 3D_\infty + R - 2D_\infty \\
 = &  \di_0(x^3 + x^2 + x + 1) + D_\infty - 3D_1
\end{align}
which is clearly non-negative on $U_1$.
Similarly
\begin{align}
\di \left( \frac{y(x+1)}{x^2(x-1)^2} \right) & = \di(y) + \di(x+1) - 2\di(x) - \di(x-1) \\
& = R - 3D_\infty + D_{-1} -D_\infty -2D_0 + 2D_\infty - 2D_1 + 2D_\infty \\
& = R + D_{-1} - 2D_0 - 2D_1
\end{align}
and
\begin{align}
\di \left( \frac{2y}{(x-1)^2} \right) & = \di(y) - 2\di(x-1) \\
& = R - 3D_\infty - 2D_1 + 2D_\infty \\
& = R - D_\infty - 2D_1,
\end{align}
are non-negative on $U_0 \cap U_1$ and $U_1 \cap U_\infty$ respectively.

It is also obvious that
\[
f_{0 \infty 2} - f_{0 1 2 } + f_{\infty 1 2} = \frac{y}{x^2} - \left( \frac{y}{x^2} - \frac{y}{(x-1)^2} \right) + \frac{2}{(x-1)^2} = 0.
\]

We finally check the relations between the functions and differentials, starting with
\begin{align}
df_{0 1 2} & = \frac{1}{2yx^3(x-1)^3} (x(x-1)(x+1)f' + f)dx \\
& = \frac{1}{2yx^3(x-1)^3}(2x^6 + 2x^4 + 2x^2 + 2)dx \\
& = \frac{1}{2yx^3(x-1)^3}((x^5 + x^3 + 2x^2 + 2) + 2(x^6 + x^5 + x^4 + x^3))dx \\
& = \omega_{02} - \omega_{12}.
\end{align}
We also have
\begin{align}
df_{\infty 1 2} & = \frac{1}{2y(x-1)^3}(f-(x-1)f')dx \\
& = \frac{1}{2y(x-1)^3}(x^6 + x^3 + 2x^2 + 2x+ 2)dx \\
& = \frac{1}{2y(x-1)^3}(x^3(x-1)^3 - x^3 - x^2 - x - 1) dx \\
& = \left(\frac{x^3}{2y} - \frac{x^3 + x^2 + x+1}{2y(x-1)^3}\right) dx \\
& = \omega_{\infty 2} - \omega_{12},
\end{align}
which completes the proof.

\end{proof}

We are now in a position to study the group action on $\cechderhamhone(\cU)$.
First note that $\sigma(U_\infty) = U_\infty$, so this part of the cover $\cU$ is preserved.
However, we can easily see that $\sigma(U_0) = U_1$, and hence we have the following commutative diagram
\[
\begin{array}{ccc}
\derhamhone \cong \cechderhamhone(\cU)  & \xleftarrow{\rho} & \cechderhamhone(\cU'')  \\
\sigma\uparrow & ~ & \rho'\downarrow  \\
\derhamhone \cong \cechderhamhone(\cU)  & \xrightarrow{\sigma} & \cechderhamhone(\cU')
\end{array}
\]

Since the image $\sigma (U_0)$ is equal to $U_1$ it follows that $\sigma$ will map from $\derhamhone(\cU')$ to $\derhamhone(\cU)$, and hence we can compute the group action by studying $\sigma \colon (\omega_{2i}, \omega_{3i}, f_{23i}) \mapsto (\sigma(\omega_{3i}), \sigma (\omega_{2i}), \sigma(f_{23i}))$.

In particular
\begin{equation}\label{sigmanu1}
\sigma(\nu_1) = \left( \frac{x^4+2x^2+2x+2}{yx^3}dx, \frac{x^4+x^3+2x^2+2x}{y}dx, \frac{2y(x+2)}{x^2} \right)
\end{equation}
and
\[
\sigma(\nu_2) = \left( \frac{x^3+x^2+1}{2yx^3}dx, \frac{2x^3+2}{y}dx, \frac{2y}{x^2} \right).
\]
Note that $dx$ is invariant under the action of $\sigma$, since $\sigma(dx) = d (\sigma (x)) = d(x+1) = dx$.


We wish to write the above elements in terms of the basis elements for $\cechderhamhone(\cU)$ that we have already described.
Recall that this basis is formed of
\begin{align}
\gamma_1  = & \left( \frac{1}{y}dx, \frac{1}{y}dx, 0\right) \\
\gamma_2 = & \left(\frac{x}{y}dx, \frac{x}{y}dx, 0\right) \\
\gamma_3 = & \left( \frac{1}{yx^2}dx, \frac{x^4 + 2x^2}{y}dx, \frac{y}{x} \right)\\
\gamma_4 = & \left(\frac{x^2+1}{2yx^3}dx, \frac{2x^3}{y}dx, \frac{y}{x^2} \right).
\end{align}

Note that
\[
2\gamma_3 + \gamma_4 = \left( \frac{2x^2+2x+2}{yx^3}dx, \frac{2x^4+2x^3+x^2}{y}dx, \frac{2y(x+2)}{x^2} \right).
\]
Hence, if the second entry in $\sigma(\nu_1)$ were the negative of what is shown in \eqref{sigmanu1} then it would follow that
\[
\sigma(\nu_1) - (2\gamma_3+\gamma_4) = \left( \frac{x}{y}dx, \frac{x}{y}dx, 0\right) = \gamma_2.
\]

We provide a similar computation for $\nu_2$, starting by computing the image of $\nu_2$ under the action of $\sigma$, which is
\[
\sigma(\nu_2) = \left( \frac{x^3+x^2+1}{2yx^3}dx, \frac{2x^3+2}{y}dx, \frac{2y}{x^2} \right).
\]
In this case, if the final term of $\sigma(\nu_2)$ were minus what it is then we would have
\[
\sigma(\nu_2) - \gamma_4 = \left( \frac{1}{2y}dx, \frac{1}{2y}dx, 0\right) = 2\gamma_1.
\]

\section{Modular curve $X_0(50)$}
We now consider the modular curve $X_0(50)$.
When $\cha(k) = 3$ we see from \cite[Table 1]{automorphismshyperellipticmodular} that the automorphism group is $D_6$, the dihedral group with 12 elements.
We see from \cite[Table 2]{automorphismshyperellipticmodular} that this has 
\[
y^2 = f(x) = x^6 + 2x^5 + 2x^3 + 2 + 1
\]
as a defining equation.

However, as in the last equation we need to write this in the form $\sum_i=0^3 a_{2i}x^{2i}$ for some $a_i \in k$.
\todo[inline]{Not sure what automorphism can be used to remove the $x^5$ term}
We can deduce this, up to a constant, from the \cite[Table 7]{automorphismshyperellipticmodular}; namely $f(x) = 2x^6 + x^4 +2x^2$.
We can also deduce that the automorphism of order three is $x \mapsto x+i$, where $i^2 = -1$.
Indeed, we can easily verify that $2x^6 + x^4 + 2x^2 = 2(x+i)^6 + (x+i)^4 + 2(x+i)^2$.
However, without being sure of what the constant term is we cannot compute the basis.

\section{Modular curve $X_0(28)$}
Let $k$ be an algebraically closed field of characteristic three.
We now consider the modular curve $X_0(28)$, which by \cite[Table 1]{automorphismshyperellipticmodular} has defining equation 
\begin{align}
y^2 &  = (x^2 + 1)(x^2 + x+2)(x^2+2x+2) \\
& = x^6 + x^4 + x^2 + 1.
\end{align}
We also see from \cite[Table 1]{automorphismshyperellipticmodular} that the automorphism group when $\cha(k) = 3$ is $GL_2(\mathbb F_3)$.

We start by computing explicitly what this action.
\todo[inline]{There appear to be some issues in the case of the generic automorphism group. In Table 1 the group is listed as $D_6$, yet in 3.1 it is claimed that the reduced group is $\ZZ_2$. Also, there are three generators listed in Table 2, but they are all of order 2. Check this with Bernhard}
\todo[inline]{Given that the explicit elements on the generic fibre are all of order two, and that on page 154 it is mentioned that the reduced group on the generic fibre is $D_4$, I am guessing that the automorphism group is meant to be $D_8$ (or $D_4$?)}

Although the automorphism group is said to be $GL_2(\mathbb F_3)$, we need only find one element that acts in the correct way to show that the sequence in \cite{derhamactions} does not split.
We can construct at least one such element in the same way that we did for $X_0(22)$.
We already have our defining equation in the form
\[
y^2 = \sum_{i=0}^3 a_{2i}x^{2i}.
\]
When we compute $f(x+c) - f(x)$ we get $(2c^3+c)x^3 + (c^3 + 2c)x + (c^6 + c^4 + c^2)$.
If we set $c=1$ or $c=2$ then all the coefficients are clearly zero.
So the map $\sigma \colon x \mapsto x-1$ is an isomorphism of our curve.

We now compute what the basis elements of $\derhamhone$ are, using \cite[Thm. 2.3]{derhamactions}.
We first note that if $f(x) = x^6 + x^4 + x^2 +1$ then $f'(x) = x^3 + 2x$.
Hence
\begin{align}
s_1 & = xf'(x) -2f(x) = x^6 + 2x^4 + 1 \\
s_2 & = xf'(x) - f(x) = 2x^6 + x^2 + 2.
\end{align}
Then it follows that the first two basis elements in \cite[Thm. 2.3]{derhamactions} are
\begin{align}
\left( \frac{1}{2yx^2}dx, \frac{2(x^6 + 2x^4)}{2yx^2}dx, \frac{y}{x} \right) & = \left( \frac{1}{2yx^2}dx, \frac{x^4+2x^2}{y}dx, \frac{y}{x} \right) \label{28basisone} \\
\left( \frac{x^2 + 2}{2yx^3}dx, \frac{2x^6}{2yx^3}dx, \frac{y}{x^2} \right) & = \left( \frac{x^2 + 2}{2yx^3}dx, \frac{2x^3}{y}dx, \frac{y}{x^2} \right). \label{28basistwo}
\end{align}

As in the $X_0(22)$ case, we need to refine our cover, since the open set $U_0$ is not closed under the action of $\sigma$.
In particular, we let $\cU' = \{U_\infty, U_{1} \}$ and $\cU'' = \cU \cup \cU'$.
Then we wish to find elements of $\cechderhamhone(\cU'')$ that surject on to \eqref{28basisone} and \eqref{28basistwo}.

\todo[inline]{Rough work from here - here be dragons}
We first posit that
\[
\left( \frac{1}{2yx^2}dx, \frac{x^4+2x^2}{y}dx, \frac{x^4 + 2x^3 + 2x^2 + 2x}{2y(x-1)^3}dx, \frac{y}{x}, \frac{y}{x(x-1)^2}, \frac{-y(x+1)}{(x-1)^2} \right)
\]
is such an element.
We start by checking that the alternating sum of the functions is zero:
\begin{equation*}
\frac{y}{x} - \frac{y}{x(x-1)^2} - \frac{y(x+1)}{(x-1)^2}  = \frac{y(x-1)^2 - y - yx(x+1)}{x(x-1)^2} = 0
\end{equation*}

We next check that the third differential is regular on the correct open set, as follows:
\begin{align*}
\di \left( \frac{x^4 + 2x^3 + 2x^2 + 2x}{2y(x-1)^3}dx \right) = & 3D_\infty -R +3D_\infty -3D_1 +\di_0(x^4+2x^3+2x^2+2x) \\
&  -4D_\infty +R -2D_\infty \\
= & \di_0(x^4 + 2x^3 + 2x^2 + 2x) - 3D_1.
\end{align*}
Clearly this is positive on $U_1$, as desired.

We now verify the other relations, starting with
\begin{align*}
df_{0 1 1} & = d \left( \frac{y}{x(x-1)^2} \right) \\
& = \frac{1}{x(x-1)^2}dy + y d \left( \frac{1}{x(x-1)^2} \right) \\
& = \frac{f'}{2yx(x-1)^2}dx + \frac{y}{(x-1)^3x^2}dx \\
& = \frac{1}{2yx^2(x-1)^3}\left( f'x(x-1) +2f\right)dx \\
& = \frac{1}{2yx^2(x-1)^3}(2x^6+x^5+x^4+2x^3+2)dx \\
& = \omega_{0 1} - \omega_{1 1}.
\end{align*}
Similarly
\begin{align*}
df_{\infty 1} & = d \left( \frac{2y(x+1)}{(x-1)^2} \right) \\
& = 
\end{align*}





\bibliography{biblio}
\bibliographystyle{amsalpha}


\end{document}
