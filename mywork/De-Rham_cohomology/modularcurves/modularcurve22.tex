% !TEX TS-program = pdflatex
% !TEX encoding = UTF-8 Unicode

% This is a simple template for a LaTeX document using the "article" class.
% See "book", "report", "letter" for other types of document.

\documentclass[draft, 11pt]{article} % use larger type; default would be 10pt

\usepackage[utf8]{inputenc} % set input encoding (not needed with XeLaTeX)

%%% Examples of Article customizations
% These packages are optional, depending whether you want the features they provide.
% See the LaTeX Companion or other references for full information.

%%% PAGE DIMENSIONS
\usepackage{geometry} % to change the page dimensions
\geometry{a4paper} % or letterpaper (US) or a5paper or....
% \geometry{margins=2in} % for example, change the margins to 2 inches all round
% \geometry{landscape} % set up the page for landscape
%   read geometry.pdf for detailed page layout information

\usepackage{graphicx} % support the \includegraphics command and options

%\usepackage[parfill]{parskip} % Activate to begin paragraphs with an empty line rather than an indent

%%% PACKAGES
\usepackage{mathtools}
\usepackage{booktabs} % for much better looking tables
\usepackage{array} % for better arrays (eg matrices) in maths
\usepackage{paralist} % very flexible & customisable lists (eg. enumerate/itemize, etc.)
\usepackage{verbatim} % adds environment for commenting out blocks of text & for better verbatim
\usepackage{subfig} % make it possible to include more than one captioned figure/table in a single float
\usepackage[obeyDraft]{todonotes}
% These packages are all incorporated in the memoir class to one degree or another...

%%% HEADERS & FOOTERS
\usepackage{fancyhdr} % This should be set AFTER setting up the page geometry
\pagestyle{fancy} % options: empty , plain , fancy
\renewcommand{\headrulewidth}{0pt} % customise the layout...
\lhead{}\chead{}\rhead{}
\lfoot{}\cfoot{\thepage}\rfoot{}

%%% SECTION TITLE APPEARANCE
\usepackage{sectsty}
\allsectionsfont{\sffamily\mdseries\upshape} % (See the fntguide.pdf for font help)
\usepackage{amsmath}
\usepackage{amsthm}
\usepackage{amsfonts}
\usepackage{mathrsfs}
\usepackage{amsopn}
\usepackage{amssymb}
%\usepackage{natbib}
% (This matches ConTeXt defaults)

%%% ToC (table of contents) APPEARANCE
\usepackage[nottoc,notlof,notlot]{tocbibind} % Put the bibliography in the ToC
\usepackage[titles,subfigure]{tocloft} % Alter the style of the Table of Contents
\renewcommand{\cftsecfont}{\rmfamily\mdseries\upshape}
\renewcommand{\cftsecpagefont}{\rmfamily\mdseries\upshape} % No bold!

%Theorems and stuff
\theoremstyle{plain}
\newtheorem{defn}{Definition}[section]
\newtheorem{thm}[defn]{Theorem}
\newtheorem{cor}[defn]{Corollary}
\newtheorem{lem}[defn]{Lemma}
\newtheorem{prop}[defn]{Proposition}
\newtheorem{ex}[defn]{Example}
\newtheorem*{unnumthm}{Theorem}
\newtheorem{defnlem}[defn]{Definition/Lemma}
\newtheorem{defnthm}[defn]{Theorem/Definition}
\theoremstyle{remark}
\newtheorem*{rem}{Remark}


\newcommand{\cO}{{\cal O}}
\newcommand{\ra}{\rightarrow}
\newcommand{\NN}{{\mathbb N}}
\newcommand{\PP}{{\mathbb P}}
\newcommand{\ZZ}{{\mathbb Z}}
\newcommand{\cL}{{\mathcal L}}
\newcommand{\cA}{{\mathcal A}}
\newcommand{\cD}{{\mathcal D}}
\newcommand{\cU}{{\mathcal U}}
\newcommand{\cech}{\v{C}ech }
\newcommand{\hzero}{{H^0(X,\Omega_X)}}
\newcommand{\hone}{H^1(X,\mathcal{O}_X)}
\newcommand{\cechhone}{\check{H}^1(\mathcal U,\mathcal O_X)}
\newcommand{\derhamhone}{H_{\text {dR}}^1(X/k)}
\newcommand{\cechderhamhone}{\check{H}_{\text {dR}}^1(X/k)}


\DeclareMathOperator{\aut}{Aut}
\DeclareMathOperator{\ord}{ord}
\DeclareMathOperator{\di}{div}
\DeclareMathOperator{\cha}{char}
\DeclareMathOperator{\gal}{Gal}
\DeclareMathOperator{\Tr}{Tr}

%%% END Article customizations

%%% The "real" document content comes below...

\title{title}
\author{J. Tait}
%\date{} % Activate to display a given date or no date (if empty),
         % otherwise the current date is printed 

\begin{document}
\maketitle

We suppose that $k$ is a field of characteristic three and that $X$ is the curve defined by
\begin{align*}
y^2  & = (x^3 + x^2 + 2x^2 + 1)(x^3 + 2x^2 + x + 1)\\
& = x^6 + 2x^4 + x^3 + 2x^2 + 1.
\end{align*}
As in \cite[Table 2]{automorphismshyperellipticmodular}, this is the modular curve $X_0(22)$.
We see from \cite[Table 1]{automorphismshyperellipticmodular} that the automorphism group of this curve is $D_6$, the dihedral group of order 12.
In \cite[\S 3.3]{automorphismshyperellipticmodular} we see that this can be written in the form $C_2 \rtimes (C_2 \times C_3)$, where the cyclic groups of order two are generated by
\begin{align*}
\sigma_1 & \colon (x,y) \mapsto (x,-y) \\
\sigma_2 & \colon (x,y) \mapsto \left( \frac{4}{x}, \frac{8y}{x^3} \right)
\end{align*}
respectively.

To describe the element of order three we need to write our defining equation in a different form.
From \cite[\S 3.3]{automorphismshyperellipticmodular} we see that we need our defining equation to be of the form 
\[
y^2 = \sum_{i=0}^3 a_{2i}x^{2i};
\]
i.e. to have only even terms.
If we consider a function field $k(x,y)$, with some relation on $y^2$, then the map sending $x$ to $x-1$ is an isomorphism.
If we apply this map to $K(X)$, the function field of $X$, we get that
\[
y^2 = x^6 + 2x^4 + 2x^2 + 2.
\]
We let $f(x) = x^6 + 2x^4 + 2x^2 + 2$.
Then $y^2 = f(x)$ and without loss of generality we now let $X$ be the projective curve defined by this equation.
If $g(x) = a_sx^s + \ldots + a_0$ where $a_0 \neq 0 \neq a_s$, then we define $g^*(x) = a_0^{-1}x^sf\left( \frac{1}{x} \right)$.
So in our case we have $f^*(x) = x^6  + x^4 + x^2 + 2$.


\bibliography{biblio}
\bibliographystyle{amsalpha}


\end{document}
