% !TEX TS-program = pdflatex
% !TEX encoding = UTF-8 Unicode

% This is a simple template for a LaTeX document using the "article" class.
% See "book", "report", "letter" for other types of document.

\documentclass[draft, 11pt]{article} % use larger type; default would be 10pt

\usepackage[utf8]{inputenc} % set input encoding (not needed with XeLaTeX)

%%% Examples of Article customizations
% These packages are optional, depending whether you want the features they provide.
% See the LaTeX Companion or other references for full information.

%%% PAGE DIMENSIONS
\usepackage{geometry} % to change the page dimensions
\geometry{a4paper} % or letterpaper (US) or a5paper or....
% \geometry{margins=2in} % for example, change the margins to 2 inches all round
% \geometry{landscape} % set up the page for landscape
%   read geometry.pdf for detailed page layout information

\usepackage{graphicx} % support the \includegraphics command and options

%\usepackage[parfill]{parskip} % Activate to begin paragraphs with an empty line rather than an indent

%%% PACKAGES
\usepackage{mathtools}
\usepackage{booktabs} % for much better looking tables
\usepackage{array} % for better arrays (eg matrices) in maths
\usepackage{paralist} % very flexible & customisable lists (eg. enumerate/itemize, etc.)
\usepackage{verbatim} % adds environment for commenting out blocks of text & for better verbatim
\usepackage{subfig} % make it possible to include more than one captioned figure/table in a single float
\usepackage[obeyDraft]{todonotes}
% These packages are all incorporated in the memoir class to one degree or another...

%%% HEADERS & FOOTERS
\usepackage{fancyhdr} % This should be set AFTER setting up the page geometry
\pagestyle{fancy} % options: empty , plain , fancy
\renewcommand{\headrulewidth}{0pt} % customise the layout...
\lhead{}\chead{}\rhead{}
\lfoot{}\cfoot{\thepage}\rfoot{}

%%% SECTION TITLE APPEARANCE
\usepackage{sectsty}
\allsectionsfont{\sffamily\mdseries\upshape} % (See the fntguide.pdf for font help)
\usepackage{amsmath}
\usepackage{amsthm}
\usepackage{amsfonts}
\usepackage{mathrsfs}
\usepackage{amsopn}
\usepackage{amssymb}
%\usepackage{natbib}
% (This matches ConTeXt defaults)

%%% ToC (table of contents) APPEARANCE
\usepackage[nottoc,notlof,notlot]{tocbibind} % Put the bibliography in the ToC
\usepackage[titles,subfigure]{tocloft} % Alter the style of the Table of Contents
\renewcommand{\cftsecfont}{\rmfamily\mdseries\upshape}
\renewcommand{\cftsecpagefont}{\rmfamily\mdseries\upshape} % No bold!

%Theorems and stuff
\theoremstyle{plain}
\newtheorem{defn}{Definition}[section]
\newtheorem{thm}[defn]{Theorem}
\newtheorem{cor}[defn]{Corollary}
\newtheorem{lem}[defn]{Lemma}
\newtheorem{prop}[defn]{Proposition}
\newtheorem{ex}[defn]{Example}
\newtheorem*{unnumthm}{Theorem}
\newtheorem{defnlem}[defn]{Definition/Lemma}
\newtheorem{defnthm}[defn]{Theorem/Definition}
\theoremstyle{remark}
\newtheorem*{rem}{Remark}


\newcommand{\cO}{{\cal O}}
\newcommand{\ra}{\rightarrow}
\newcommand{\NN}{{\mathbb N}}
\newcommand{\PP}{{\mathbb P}}
\newcommand{\ZZ}{{\mathbb Z}}
\newcommand{\cL}{{\mathcal L}}
\newcommand{\cA}{{\mathcal A}}
\newcommand{\cD}{{\mathcal D}}
\newcommand{\cU}{{\mathcal U}}
\newcommand{\cech}{\v{C}ech }
\newcommand{\hzero}{{H^0(X,\Omega_X)}}
\newcommand{\hone}{H^1(X,\mathcal{O}_X)}
\newcommand{\cechhone}{\check{H}^1(\mathcal U,\mathcal O_X)}
\newcommand{\derhamhone}{H_{\text {dR}}^1(X/k)}
\newcommand{\cechderhamhone}{\check{H}_{\text {dR}}^1(X/k)}


\DeclareMathOperator{\aut}{Aut}
\DeclareMathOperator{\ord}{ord}
\DeclareMathOperator{\di}{div}
\DeclareMathOperator{\cha}{char}
\DeclareMathOperator{\gal}{Gal}
\DeclareMathOperator{\Tr}{Tr}

%%% END Article customizations

%%% The "real" document content comes below...

\title{title}
\author{J. Tait}
%\date{} % Activate to display a given date or no date (if empty),
         % otherwise the current date is printed 

\begin{document}
\maketitle

\section{$X_0(22)$ in characteristic three}
We suppose that $k$ is characteristic 3 and then we consider the curve $X_0(22)$, where the affine part of the curve is defined by the equation
\begin{equation}\label{x22}
y^2 = (x^3+ 4x^2 + 8x + 4)( x^3 + 8x^2 + 16x + 16).
\end{equation}
By \cite{automorphismshyperellipticmodular} we see that in this case the automorphism group of $X_0(22)$ is $D_6$ (where $D_n$ denotes the group of order $2n$).
This arises from $C_2 \rtimes (C_2 \times C_3)$, where the first $C_2$ is the involution, and hence $C_2 \times C_3$ is the reduced automorphism group.
Furthermore, from \cite{automorphismgrouphyperelliptic}, we discover how the group $C_2 \times C_3$ acts.

{\bf From here on is just my best guess from the Konotgeorgis and Yang paper}
The non-exceptional automorphism group is $\ZZ_2 \times \ZZ_2$.
One of these elements is of course the involution.
The other is described by 
\[
w_2:(x,y) \mapsto \left( \frac{4}{x}, \frac{8y}{x^3} \right).
\]

Now from the first two paragraphs of \cite[\S 3.3]{automorphismshyperellipticmodular} I think that the reduced group is given by the group of order two generated by $w_2$ and another group of order three (denoted $E$ in the paper).
How does this group $E \cong C_3$ act?

It is stated that $E$ will fix either $\infty$ or 0.
If it fixes $\infty$ then an arbitrary element $\tau \in E$ can be described by $\tau \colon x \mapsto  x+ c( \tau)$ for some $c (\tau) \in k$.
It is then stated that in this case, $c$ is a root of $G_1(x):= f(x + c_i) - f(x)$, but what $c_i$ is is not specified.

If $E$ fixes 0 then we are told that $f^*(x) := a_0^{-1}x^s f\left( \frac{1}{x} \right)$ defines an isomorphic curve to $f(x)$, so we can consider this curve instead, and then we are in the previous case again.

From \cite[Table 7]{automorphismshyperellipticmodular} we see that $G_2 = 0$, so $0$ is not fixed, but $G_1 \neq 0$, so $\infty$ is fixed by $E$.
In particular we are given that
\begin{equation*}
G_1(x) = (2c^3+2c)x^3 + (2c^3 + c) x + c^6 + 2c^4 + 2c^2
\end{equation*}
we need to find the $c$ in this equation.

However, once we expand \eqref{x22} to 
\begin{equation}
x^6 + 2x^4 + x^3 + 2x^2 + 1
\end{equation}
we see that any such $c$ will satisfy the equation, we may as well assume that $c=1$ and see what happens.
In this case the equation comes out to be just $x^3+1$, and the solutions are the roots of unity.
Hence we can choose our generater of $C_3$ to be $\tau \colon x \mapsto x+1$.


{\bf How does this act on the basis of $\derhamhone$. Also, how does it affect the cover $\cal U$.}

In this case we can explicitly write down the the part of the basis of $\derhamhone$ that maps to $\hone$, which is formed of
\[
\left( \frac{x^2+x^1}{y} dx, \frac{2x^4}{y} dx, \frac{y}{x} \right) \text{ and } \left( \frac{2(x + 1)}{y} dx, \frac{x^3}{y} dx, \frac{y}{x^2} \right).
\]

{\bf How does the group act on the cover?}

We recall that $U_1 = X \backslash \pi^{_1}(0)$ and $U_2 = X \backslash \pi^{-1}(\infty)$.
We will focus on how $\tau$ acts on the space initially, since this is the element whose order is divisible by the characteristic, and hence most likely to make this interesting.
In this case it is clear that $\tau(U_2) = U_2$.
However, $\tau$ does not fix $U_1$. 
Note that the two points in $\pi^{-1}(0)$ are $(0,1)$ and $(0,2)$.
Clearly $\tau$ maps these to $(1,1)$ and $(1,2)$, so it must map something else to $(0,1)$ and $(0,2)$, and hence does not fix the subspace.
In a similar manner to \cite{canonicalrepresentation} we will define a new open set $U_3 := \tau(U_1)$, so $U_1 = \tau^{-1}(U_3) = \tau^2(U_3)$.
We let $\mathcal U' = \{ U_2, U_3\}$ and $\mathcal U'' = \mathcal U \cup \mathcal U'$.

Now since $\mathcal U''$ is a refinement of $\mathcal U$, and since $\derhamhone (\mathcal U) \cong \derhamhone (X)$, we know that $\derhamhone (\mathcal U) \cong \derhamhone (\cU'')$.
We then have the following commutative diagram:
\[
\begin{array}{ccc}
\cechderhamhone (\cU) & \overset{\rho}{\leftarrow} & \cechderhamhone (\cU'') \\
\tau \uparrow & & \downarrow  \rho' \\
\cechderhamhone(\cU) & \overset{\tau}{\rightarrow} & \cechderhamhone(\cU')
\end{array}
\]
where $\rho$ and $\rho'$ are the restriction maps.
Since these are isomorphisms we see that if we compute how $\tau$ acts on $\cechderhamhone (\cU')$ then we can see how it acts on $\cechderhamhone(\cU)$.

{\bf Finding a basis of $\cechderhamhone (\cU '')$}

Since the isomorphism goes via $\cechderhamhone (\cU'')$ we wish to find a basis of this space.
We should first describe the space --- it is precisely the vector space
\begin{multline}\label{u''conditions}
\{ ( \omega_1, \omega_2, \omega_3, f_{12}, f_{13}, f_{23}) | \omega_j \in \Omega_X(U_j), f_{jk} \in \cO_X(U_j \cap U_k),\\ df_{jk} = \omega_j - \omega_k, f_{23} - f_{13} + f_{12} = 0\}
\end{multline}
quotiented by the subspace generated by
\begin{equation*}
\{ (df_1, df_2, df_3, f_1-f_2, f_1-f_3, f_2-f_3) | f_j \in \cO_X(U_j) \}.
\end{equation*}

We actually just want to find a basis for the part that will project on to $\cechderhamhone (\cU')$, and in order to do this we start by determining a basis of $\cechhone (\cU')$.

When looking for a basis of $\cechhone (\cU)$ we found $g_i$ such that $\langle x^iy^{-1}dx, g_i \rangle$ had non-zero residue on $U_1$, and such that $g_i$ was regular on $U_1 \cap U_2$.
We now do the same but replacing $U_2$ by $U_3$.
So naively we guess that $g_i = \frac{-y}{(x-1)^i}$.
Then we can check that $f_{13} = f_{12} + f_{23}$ is regular at $\infty$ (i.e. regular on $U_1 \cap U_3$) by computing the residue at $\infty$.
If $i = 1$ then the residue of both $f_{12}$  at $\infty$ is 1 and $f_{23}$.
To see this we can consider the uniformising parameter $\frac{1}{x}$.
Then $\left( \frac{y}{x^2} \right)^2 = \frac{f}{x^4}$.
The only term of this with a pole at $\infty$ is $\frac{x^6}{x^4}= \frac{1}{x^2}$, which when expanded as a sum
\[
\sum_{i=-2} c_i x^i
\]
has $c_{-2} = 1$, so the residue of $\frac{y}{x^2}$ is $\pm 1$.
Similarly when we consider $\left( \frac{y}{(x-1)^2} \right)^2 = \frac{f}{x^4 - x^3 - x +1}$ we see that only the $x^6$ term has a pole at $\infty$.
So the square of the residue is determined by the $-2$ term in the expansion of $\frac{x^6}{x^4 - x^3 - x +1}$, and this in turn is determined by the first term in expansion of $\frac{x^4 - x^3 - x +1}{x^6}$. 
We then see that clearly this term is 1, and hence the residue of $\frac{y}{(x-1)^2}$ is $\pm 1$.
So after a sign change we have that the function 
\[
\frac{y}{x^2}  - \frac{y}{(x-1)^2}
\]
does note have a pole at $\infty$.
\todo[inline]{can I do this by just counting zeroes and poles. Also, need to check that there are not other poles that have appeared}.





Indeed, if we then consider $\langle x^iy^{-1}dx, g_i \rangle$ we get $\frac{1}{(x-1)}dx$.
The divisor of this is
\[
\di \left( \frac{1}{x-1}dx \right) = R - D_\infty - D_1.
\]
So we have order one poles at $P_0, P_0', P_\infty$ and $P_\infty'$.
So now we just have to check that the residue at $P_1$ is not the negative of the residue at $P_1'$.
This is obvious, since $x-1$ is a uniformising parameter at both $P_1$ and $P_1'$, and since $d(x-1) = dx$ we see that the coefficient at both points will just be one.\todo{check this}

So far, assuming that the previous functions work we have
\begin{align*}
\omega_{1i} & = \frac{-\psi(x)}{2yx^i} dx \\
\omega_{2i} & = \frac{\phi(x)}{2yx^i} dx \\
f_{12i} & = \frac{y}{x^i} \\
f_{13i} & = \frac{y}{x^i(x-1)}.
\end{align*}

From this we will have defined $f_{23i}$, and then we can check if this is possible (i.e. if it is regular on $U_2 \cap U_3$).
By \eqref{u''conditions} we see that
\begin{align*}
f_{23} & = f_{13} - f_{12} \\
& = \frac{y}{x^i(x-1)} - \frac{y}{x^i} \\
& = \frac{y}{x^i} \left( \frac{2-x}{x-1} \right).
\end{align*}
This will not work since we have a pole at both points in $ \pi^{-1}(0)$.


Since the above did not work, try something else.
We already have obvious choices for $f_{12i}$ and $f_{23i}$, namely $\frac{y}{x^i}$ and $\frac{-y}{(x-1)^i}$ respectively.
Let's see if they are suitable - i.e. if $f_{13i}=f_{12i} + f_{23i}$ is regular on $U_1 \cap U_3$.
When $i=2$ this is $\frac{y}{x^2} - \frac{y}{(x-1)^2}$.
We can compute the poles of this more easily by computing the pole of its square.
When $i=2$ this is
\begin{align*}
\left( \frac{y}{x^2} - \frac{y}{(x-1)^2} \right)^2 & = \frac{y^2}{x^4} + \frac{y^2}{(x-1)^4} - \frac{2y^2}{x^2(x-1)^2} \\
& = \frac{((x-1)^4 + x^4 - 2x^2(x-1)^2)y^2}{x^4(x-1)^4} \\
& = \frac{(x^2-x+1)f(x)}{(x-1)^4x^4},
\end{align*}
from which one can clearly see that the function is regular on $U_1 \cap U_3$.


Now we check whether the differentials work for this particular choice of functions.
We start by noting that 
\begin{equation*}
f' = 2x^3 + x
\end{equation*}
and hence that
\begin{align*}
s_2(x) & = f' - \frac{f}{x} \\
& = 2\left(x^5 + x^2 + x + \frac{1}{x} \right).
\end{align*}

We also compute the differential of $\frac{y}{(x-1)^2}$, which is
\begin{align*}
d\left( \frac{y}{(x-1)^2} \right) & = \frac{1}{(x-1)^2}dy + y d \left( \frac{1}{(x-1)^2} \right) \\
& = \frac{f'}{2y(x-1)^2}dx - \frac{2y}{(x-1)^3}dx \\
& = \frac{1}{2y(x-1)^2} \left( f' - \frac{f}{x-1} \right) dx.
\end{align*}
Also $\phi_2(x) = 2x^5 = -x^5$, so $\omega_1 = \frac{x^3}{2y}dx$ when $i=2$.


Combining all of the above we compute what our divisor should be, namely
\begin{align*}
\omega_{32} &  = d\left( \frac{y}{(x-1)^2} \right) + \frac{x^3}{2y}dx \\
& = \frac{1}{2y(x-1)^3} \left( f'(x-1) - f + x^3(x-1)^3 \right) dx
& = \frac{1}{2y(x-1)^3}(2x^3-x^2-x-1)dx
\end{align*}
\todo{check signs are all correct!!}
 
We can then compute the divisor of this, which, if we let $g(x) = 2x^3 - x^2 -x -1$, is
\begin{align*}
\di(g(x)) + \di( dx) - \di (y) - \di ((x-1)^3)  & =  \di_0(g(x)) - 3D_\infty + R - 2D_\infty - R  \\
& + 3D_\infty + 3D_\infty - 3D_1 \\
 & =  \di_0(g(x)) + D_\infty - 3D_1.
\end{align*}
So the only poles of this differential occur at the points in $\pi^{-1}(1)$, and hence the differential is regular on $U_3$.

The only condition that it remains to check is that $d(f_{132}) = \omega_1 - \omega_3$.
First we compute $d(f_{132})$, which is
\begin{align*}
d(f_{132})  & =  d \left( \frac{y}{x^2} - \frac{y}{(x-1)^2} \right) \\
& =  \left( \frac{1}{x^2} - \frac{1}{(x-1)^2} \right) dy + y \left( d \left( \frac{1}{x^2} \right) - d \left( \frac{1}{(x-1)^2} \right) \right) \\
&=  \frac{f'}{2y} \left( \frac{1}{x^2} - {1}{(x-1)^2} \right) dx + y \left( \frac{1}{x^3} - \frac{1}{(x-1)^3} \right) dx \\
&=  \frac{1}{2y} \left( f'\left( \frac{(x-1)^2 - x^2}{x^2(x-1)^2} \right) - f \left( \frac{(x-1)^3 - x^3}{(x-1)^3x^3} \right) \right) dx \\
&=  \frac{1}{2yx^3(x-1)^3} ( xf'(x-1)((x-1)^2-x^2) + f ) \\
&=  \frac{1}{2yx^3(x-1)^3} ( x^4 + x^3 + x^2 + 1)dx.
\end{align*}

Now we compute $\omega_1 - \omega_3$, which is
\begin{align*}
\omega_1 - \omega_3 & = \frac{x^3 + x^2 + 1}{yx^3}dx -  \frac{1}{2y(x-1)^3}(2x^3-x^2-x-1)dx \\
& = (\frac{1}{2yx^3(x-1)^3} (2(x-1)^3(x^3 + x^2 + 1) - x^3(2x^2 - x^2 - x - 1)) dx \\
& = \frac{1}{2yx^3(x-1)^3} (x^4 + x^3 + x^2 + 1) dx.
\end{align*}
Hence we see that the functions and differentials above give us an element of $\cechderhamhone(\cU '')$.

Need to find one more element. Based on \cite{canonicalrepresentation} we will see if $-\left(\frac{y}{x-1} - \frac{y}{(x-1)^2}$ could be a legitimate element as $f_{231}$.
Firstly, we note that
\begin{equation*}
\frac{y}{x-1} - \frac{y}{(x-1)^2} = y \left( \frac{x-2}{(x-1)^2} \right).
\end{equation*}

Next we check that $f_{131} = f_{121} + f_{231}$, as follows 
\begin{align*}
\left( \frac{y}{x} - \frac{y(x-2)}{(x-1)^2} \right)^2 & = f \left( \frac{1}{x^2} - \frac{(x-2)^2}{(x-1)^4} - \frac{2(x-2)}{(x-1)^2x} \\
& = frac{f}{x^2(x-1)^4}.
\end{align*}
Clearly this has no poles in $U_1 \cap U_3$, as desired.

Now we want to check that this would form a basis of $\cechhone (\cU ')$.


\bibliography{biblio}
\bibliographystyle{amsalpha}


\end{document}
