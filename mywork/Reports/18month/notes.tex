The divisor of $y$ relies on the solutions to $f(x)$, so we will now introduce some notation for these.
Let $Q_1,\ldots, Q_l$ denote the points in $X$ for which $f(x)=0$.
These all lie in the affine part of the curve, $X'$, and hence we can let $Q_i=(c_i,d_i)$ for some $c_i, d_i\in \mathbb A_k^1$.
We let $m_i$ denote the order of of the zero of $f(x)$ at $c_i\in \mathbb A_k^1$, and we denote the degree of $f$ by $d'$.



%theorem about divisor of y goes here

 The zeroes of $y$ clearly only occur in the affine part, $X'$.
 In this case, if $(a,0)\in X'$ then $f(a)=0$.
 So each zero of $y$ gives a corresponding zero of $f(x)$.
 Similarly, if we have some $Q_i$ then either $d_i=0$ of $d_i=h(c_i)$.
 If $Q_i$ is not a ramified point then we have 
 \[
 m_i = \ord_{Q_i}(f(x)) = \ord_{Q_i}(y(y-h(x))) = \ord_{Q_i}(y) + \ord_{Q_i}(y-h(x)) = \ord_{Q_i}(y),
 \]
 since $y-h(a)\neq 0$.
 If, on the other hand, $Q_i$ is ramified, we have 
 \[
  2m_i = \ord_Q_i(f(x)) = 
  \]
 
 
 