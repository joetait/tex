% !TEX TS-program = pdflatex
% !TEX encoding = UTF-8 Unicode

% This is a simple template for a LaTeX document using the "article" class.
% See "book", "report", "letter" for other types of document.

\documentclass[11pt]{article} % use larger type; default would be 10pt

\usepackage[utf8]{inputenc} % set input encoding (not needed with XeLaTeX)

%%% Examples of Article customizations
% These packages are optional, depending whether you want the features they provide.
% See the LaTeX Companion or other references for full information.

%%% PAGE DIMENSIONS
\usepackage{geometry} % to change the page dimensions
\geometry{a4paper} % or letterpaper (US) or a5paper or....
% \geometry{margins=2in} % for example, change the margins to 2 inches all round
% \geometry{landscape} % set up the page for landscape
%   read geometry.pdf for detailed page layout information

\usepackage{graphicx} % support the \includegraphics command and options

\usepackage[parfill]{parskip} % Activate to begin paragraphs with an empty line rather than an indent

%%% PACKAGES
\usepackage{booktabs} % for much better looking tables
\usepackage{array} % for better arrays (eg matrices) in maths
\usepackage{paralist} % very flexible & customisable lists (eg. enumerate/itemize, etc.)
\usepackage{verbatim} % adds environment for commenting out blocks of text & for better verbatim
\usepackage{subfig} % make it possible to include more than one captioned figure/table in a single float
% These packages are all incorporated in the memoir class to one degree or another...

%%% HEADERS & FOOTERS
\usepackage{fancyhdr} % This should be set AFTER setting up the page geometry
\pagestyle{fancy} % options: empty , plain , fancy
\renewcommand{\headrulewidth}{0pt} % customise the layout...
\lhead{}\chead{}\rhead{}
\lfoot{}\cfoot{\thepage}\rfoot{}

%%% SECTION TITLE APPEARANCE
\usepackage{sectsty}
\allsectionsfont{\sffamily\mdseries\upshape} % (See the fntguide.pdf for font help)
\usepackage{amsmath}
\usepackage{amsthm}
\usepackage{amsfonts}
\usepackage{mathrsfs}
\usepackage{amsopn}
\usepackage{amssymb}
\usepackage{natbib}
% (This matches ConTeXt defaults)

%%% ToC (table of contents) APPEARANCE
\usepackage[nottoc,notlof,notlot]{tocbibind} % Put the bibliography in the ToC
\usepackage[titles,subfigure]{tocloft} % Alter the style of the Table of Contents
\renewcommand{\cftsecfont}{\rmfamily\mdseries\upshape}
\renewcommand{\cftsecpagefont}{\rmfamily\mdseries\upshape} % No bold!

%Theorems and stuff
\newtheorem{defn}{Definition}
\newtheorem{thm}{Theorem}
\newtheorem{cor}{Corollary}
\newtheorem{lem}{Lemma}
\newtheorem{prop}{Proposition}
\newtheorem{ex}{Example}

\DeclareMathOperator{\ord}{ord}
\DeclareMathOperator{\di}{div}
\DeclareMathOperator{\cha}{char}
\DeclareMathOperator{\gal}{Gal}
%%% END Article customizations

%%% The "real" document content comes below...

\title{Nine month report}
\author{J Tait}
%\date{} % Activate to display a given date or no date (if empty),
         % otherwise the current date is printed 

\begin{document}
\maketitle

\section{Overview}

Since starting I have attended a number of MAGIC courses. In the first semester I attended a 10 hour category theory course, covering the basic definitions and theorems of the topic, in a general manner that could be easily applied to which ever area of maths it is required in. I also attended a 10 hour number theory course, studying rings of integers. This included Dirichlet's unit theorem and mentioned both the  Dedekind zeta function and Dirichlet's class number formula.

In the second semester I attended three courses. Commutative algebra was one, which started by covering all the essential results for rings and modules, and then moved on to more geometric relations with them, such as Nullstellensatz and algebraic sets. I also attended algebraic geometry, which used the abstract definition of a variety to develop the link between algebra and geometry, and eventually gave a (rather quick) proof of the Riemann-Roch theorem. Finally, I went to the Representation of Groups lecture course. This course began with defining what a representation is and what characters are, and finally covered the representation theory of the symmetric groups.

The main literature that I have read since starting is William Fulton's Algebraic Curves \citep{fulton}. This book takes a classical approach to algebraic geometry (i.e. without sheaves or schemes),  works through from basic definitions to divisors and a full proof of the Riemann-Roch theorem. Exercises from the book were used to check and test my understanding of the topics covered. In particular I looked at a proof of the Hurwitz formula, which I did in detail and is presented in full below.

Another piece of literature I looked at was Sheaf Theory by B. R. Tennison \citep{tennison}. I followed this text along with notes from Bernhard's MAGIC course that ran in Spring 2010. In this I learnt all the basic theorems and definitions required to understand Sheaf theory, as well as working through examples, and eventually seeing some applications, such as the Riemann-Roch theorem.

I have also been reading the Geometry of Schemes by Eisenbud and Harris \citep{geoschemes}, in a small reading group with some other people in the department that followed from the algebraic geometry MAGIC course. This will be useful throughout my PhD as another way of viewing varieties and understanding their underlying structure.

In terms of research, I started by reading and understanding \citep{faithfulaction}, which looks at when a faithful action of a group on a curve induces a faithful action on $H^0(X,\Omega_X)$. Then I looked at extending the results in this paper to the poly-differentials $\Omega_X^{\otimes m}$. This is given in more detail in section 2 of this paper.

In the future I intend to do more to describe the action of $G$ on a general Riemann-Roch space and use this to get results in equivariant deformation theory. I will firstly look at \citep{cohogsheaves} and \citep{galiosstruc}, which look at these generalisations and how to construct them. Then I will look at \citep{quaddiffequi}, which demonstrates applications to equivariant deformation theory.



\section{The Hurwitz Formula}
This section gives an account on the Hurwtiz formula for algebraic curves. It is the solution to an exercise in \citep{fulton}, and the approach differs from that used in most textbooks about algebraic curves.

The set up involves projective non-singular curves $X$ and $Y$ over an algebraically closed field $k$, with a surjective morphism $f:X\rightarrow Y$. Let $g_X$ and $g_Y$ be the genus of $X$ and $Y$ respectively. We denote the induced map between the fields of rational functions of $Y$ and $X$ by $\tilde{f}$. Let $n=[K(X):K(Y)]$ be the degree of $f$. We denote by $\mathscr{O}_Q(Y)$ the local ring of rational functions of $Y$ at $Q\in Y$.\\


\begin{defn}
	Let $t\in \mathscr{O}_{Q}(Y)$ be a uniformising parameter for some $Q\in Y$. Then the ramification 	index of $f$ at $P\in f^{-1}(Q)$ is $e(P):= \ord_{P}(\tilde{f} (t))$.
\end{defn}

Note that this is independent of the choice of uniformising parameter; if $s$ is another uniformising parameter then $s=ut$ for some unit $u$. Then 
	\[
		\textrm{ord}_{P}(\tilde{f}(s))=\ord_{P}(\tilde{f}(ut))=\textrm{ord}_{P}(\tilde{f}				(u))+\textrm{ord}_{P}(\tilde{f}(t))=0+\ord_{P}(\tilde{f}(t))
	\]
as the order of any unit is zero.

Now we want to show the Hurwitz formula, which relates the genus of $X$ to the genus of $Y$ using the ramification indices (see theorem 1). The first step towards this is the next proposition.\\


\begin{prop}
	For each $Q\in Y$ we have $\sum_{P\mapsto Q}e(P)=n$.
\end{prop}
\begin{proof}
	Choose a uniformising parameter $t\in \mathscr{O}_{Q}(Y)$ and let $m=[K(Y):k(t)]$.

	To start with we show that $\sum_{P\mapsto Q} e(P)\leq n$. Let $Z=\sum_{P\mapsto Q}e(P)[P]$, $q=\deg (Z)$ 
	and $S=f^{-1}(Q)$. For a divisor $D=\sum_{P\in X}n_P[P]$ we define $L^S(D):=\{f\in K(X)|\ord_P(f)\geq n_P\mbox{ for all }P\in S\}$.
	Now by \citep[Lem. 1, pg.193]{fulton} we can choose $v_{1},\ldots,v_{q}\in L^{S}(0)$ such that the residues
	$\bar{v}_{1},\ldots ,\bar{v}_{q}\in L^{S}(0)/L^{S}(-Z)$ form a basis over $k$. It suffices to show that 
	$v_{1},\ldots ,v_{q}$ are linearly independent over $K(Y)$. If not then there exist (after multiplying by a 
	suitable power of $t$ if necessary) $g_{i}=\lambda_{i} +h_{i}\in K(Y)$ with $\ord_{Q}(h_{i})>0$, $\lambda_{i}\in k$ 
	and at least one $\lambda_{i}\neq 0$, such that $\sum_{i}g_{i}v_{i}=0$. But then $\sum_{i}\lambda_{i}v_{i}=-\sum_{i}h_{i}v_{i}\in L^{S}(-Z)$. 
	Hence $\sum_{i}\lambda_{i}\bar{v}_{i}=0$, contradicting the choice of $\bar{v}_{i}$'s as a basis.

	Now let
		\[
			(t)_{0}^{Y}=\sum_{\textrm{ord}_{Q}(t)>0}m_{Q}[Q]
		\]
	 the divisor of zeros of $t$ in $Y$. Then
		\[
			(t)_{0}^{X}=\sum_{\textrm{ord}_{Q}(t)>0}m_{Q} \sum_{P\mapsto Q}e(P)[P]
		\]
	is obviously the divisor of zeroes of $t$ in $X$.

	Now
		\begin{eqnarray}
			\deg\Big((t)_{0}^{X}\Big) & = & \sum_{\textrm{ord}_{Q}(t)>0}m_{Q} \sum_{P\mapsto Q}e(P) \nonumber \\
			& \leq & \sum_{\textrm{ord}_{Q}(t)>0}m_{Q} \cdot n \nonumber \\
			& = & \deg\Big((t)_{0}^{Y}\Big)\cdot n \nonumber \\
			& = & mn. \citep[\mbox{prop 4., p.194}]{fulton}
		\end{eqnarray}
	On the other hand
		\begin{equation}
			\deg\Big((t)_{0}^{X}\Big)=[K(X):k(t)]=m\cdot n
		\end{equation}
	by \citep[prop.4, pg.194]{fulton} and the tower law.
	

	Combining (1) and (2) we have that $\sum_{\textrm{ord}_{Q}(t)>0}m_{Q} \sum_{P\mapsto Q}e(P)=m\cdot n$. As we know $\sum_{\textrm{ord}_{Q}(t)>0}m_{Q}=\deg\Big((t)_{0}^{Y}\Big)=m$ by \citep[prop.4, pg.194]{fulton} and that $\sum_{P\mapsto Q}e(P)\leq n$ for each $Q$ by the above, it follows that $\sum_{P\mapsto Q} e(P)=n$. 
\end{proof}

Now as we are wanting to relate $X$ and $Y$ we next look at the image of an element of $K(Y)$ under $\tilde{f}$.\\


\begin{cor}
	For any $h\in K(Y)$ and any $Q\in Y$ we have
		\begin{equation*}
			\sum_{P\mapsto Q} \ord_{P} (\tilde{f}(h))=n\cdot \ord_{Q}(h).
		\end{equation*}
	In particular we have $\deg( \di(\tilde{f}(h)))=n\cdot \deg( \di (h))$.
\end{cor}
\begin{proof}
	Assume that $Q$ is not a pole of $h$. Then $\ord_{Q}(h)=r\geq 0$ and $h=ut^{r}$ for some unit $u$ and uniformising parameter $t$ both in $\mathscr{O}_{Q}(Y)$. Now for each $P\in f^{-1}(Q)$ we have that 
		\begin{eqnarray*}
			\ord_{P}(\tilde{f}(h)) & = & \ord_{P}\Big(\tilde{f}(u)\tilde{f}(t^{r})\Big) \nonumber \\
			&  = & \ord_{P}\Big(\tilde{f}(u)\Big)+\ord_{P}\Big(\tilde{f}(t^{r})\Big) \nonumber\\
			& = &  0 + r\cdot \ord_{P}(\tilde{f}(t)). \nonumber \\
		\end{eqnarray*}
	The last line follows as $u$ is a unit in $\mathscr{O}_Q(Y)$, hence for any $P\in f^{-1}(Q)$ we have $\tilde{f}(u)(P)=u(f(P))\neq 0$ and $\ord_P(\tilde{f}(u))=0$.

	So
		\begin{align*}
			\sum_{P\mapsto Q}\ord_{P}(\tilde{f}(h)) & =  r\cdot \sum_{P\mapsto Q} \ord_{P}(\tilde{f}				(t)) \\
			& = r\cdot \sum_{P\mapsto Q} e(P) \\
			& = r\cdot n && \mbox{[Prop. 1]} \\
			& = \ord_{Q}(h)\cdot n \\
		\end{align*}
	Finally, if $Q$ is a pole of $h$ then we can take inverses to make it a zero (of $h^{-1}$) and the 	result follows from the above.
\end{proof}

The final step before we are able to prove the Hurwitz formula is the following lemma.\\

\begin{lem}
	If $t$ is a uniformising parameter in $\mathscr{O}_{Q}(Y)$ then $\ord_{P}(d(\tilde{f}(t)))=e(P)-1$ 	for each $P\in f^{-1}(Q)$ whenever $\cha(k)\nmid e(P)$.
\end{lem}
\begin{proof}
	By definition of $e(P)$ there is a unit $u$ and a uniformising parameter $s$ in $\mathscr{O}_{P}		(Y)$ such that $\tilde{f}(t)=us^{e(P)}$. So $d\tilde{f}(t)=d(us^{e(P)})$.

	As $d$ is a derivation we have that $d(us^{e(P)})=ud(s^{e(P)})+s^{e(P)}d(u)$, and by a simple 			inductive argument we can also see that $d(s^{e(P)})=e(P)s^{e(P)-1}d(s)$. Combining these gives 		$d(us^{e(P)})=ue(P)s^{e(P)-1}d(s)+s^{e(P)}d(u)$. Now we have
		\begin{equation}
			\ord_{P}\left(e(P)s^{e(P)-1}d(s)\right)= \ord_{P}(e(P)s^{e(P)-1})=e(P)-1.
		\end{equation}
	The last equality holds as $\mbox{char}(k)\nmid e(P)$. The first holds by definition \citep[pg. 207]{fulton}.

	Also
		\begin{equation}
			\ord_{P}(s^{e(P)}d(u))\geq e(P).
		\end{equation}
	Hence we see that
		\begin{align*}
			\ord_{P}(d\tilde({f}(t))) & = \ord_{P}(ue(P)s^{e(P)-1}ds+s^{e(P)}d(u)) \\
			& =  \mbox{min}\{ \ord_{P}(ue(P)s^{e(P)-1}ds),s^{e(P)}d(u) \} && \mbox{\citep[ex.2-29, pg. 48]{fulton}}\\
			& = e(P)-1. \\
		\end{align*}
\end{proof}

Note that we required that $\mbox{char}(k)\nmid e(P)$. If $k$ is of finite characteristic and $\mbox{char}(k)\nmid e(P)$ 
for any $P\in X$ such that $e(P)\neq 0$ then we say that $f$ is \em tamely ramified\em. Otherwise we say $f$ is 
\em wildly ramified\em.


We are now in a position to prove the main theorem of this section.\\

\begin{thm}[Hurwitz Formula]
	If $f$ is tamely ramified then we have
		\begin{equation*}
			2g_{X}-2=n(2g_{Y}-2)+\sum_{P\in X}(e(P)-1).
		\end{equation*}
\end{thm}

\begin{proof}
	It suffices to show that the degree of a canonical divisor on $X$ is precisely 
		\begin{equation*}
			n\cdot (2g_{Y}-2)+\sum_{P\in X}(e(P)-1)
		\end{equation*}
	as, by \citep[cor, pg.209]{fulton}, for any canonical divisor $W$ on $X$ it is true that $\deg(W)=2g_X-2$.

	Suppose that $\omega_{Y}$ is a non-zero meromorphic differential on $Y$. Then $\tilde{f}(\omega_{Y})$ is 
	a non-zero differential on $X$.

	
	Since
		\begin{equation*}
			\deg(\di(\tilde{f}(\omega_{Y}))) = \sum_{Q\in Y} \sum_{P\mapsto Q} \ord_{P}(\tilde{f}(\omega_{Y}))
		\end{equation*}
	and
		\begin{eqnarray*}
			\sum_{Q\in Y} \Big(n\cdot \ord_{Q}(\omega_{Y}) +\sum_{P\mapsto Q}(e(P)-1)\Big)
			& = & n\cdot \deg(\di(\omega_{Y})) +\sum_{P\in X} (e(P)-1) \\
			& = & n\cdot (2g_{Y}-2) + \sum_{P\in X}(e(P)-1)
		\end{eqnarray*}
	
	it suffices to show that for each $Q\in Y$ the following holds:
		\begin{equation*}
			\sum_{P\mapsto Q}\ord_{P}\Big(\tilde{f}(\omega_{Y})\Big)=n\cdot \ord_{Q}(\omega_{Y})+\sum_{P\mapsto Q}(e(P)-1)
		\end{equation*}
	Fix $Q\in Y$. Then for some uniformising parameter $t\in \mathscr{O}_{Q}(Y)$ and some $h\in K(Y)$ 
	we have that $\omega_{Y}=hdt$ by \citep[prop 6, pg. 205]{fulton}.

	So now we have
		\begin{eqnarray}
			\sum_{P\mapsto Q}\ord_{P}(\tilde{f}(\omega_{Y})) & = & \sum_{P\mapsto Q}\ord_{P}(\tilde{f}(hdt)) \nonumber \\
			& = & \sum_{P\mapsto Q} \ord_{P}(\tilde{f}(h)) + \sum_{P\mapsto Q} \ord_{P}(d(\tilde{f}(t))) \nonumber \\
			& = & n\cdot \ord_{Q}(h) + \sum_{P\mapsto Q} (e(P)-1) \\
		      & = & n\cdot \ord_{Q}(\omega_{Y}) + \sum_{P\mapsto Q} (e(P)-1) \nonumber
		\end{eqnarray}
	with line (5) following from the previous corollary and lemma, and the proof is complete.
\end{proof}

The $\sum_{P\in X}(e(P)-1)$ term is precisely the degree of the \em ramification divisor \em (see definition below) in the case that the map is tamely ramified. 
To define the ramification divisor in general we need to define the ramification groups.\\

\begin{defn}
	Let $G:=\gal(X/Y)$ and let $t$ be a uniformising parameter at $P\in X$. Then for $i\geq -1$ we define the 
	$i^{th}$ ramifciation group at $P$, denoted $G_i$, to be the sub-group of $s\in G$ such that $\ord_P(s(t)-t)\geq i+1$. This is
	independent of choice the choice of $t$, as shown in \citep[$\S$1, Ch. IV]{localfields}.
\end{defn}

Note that $G_{-1}=G$, if $i$ is sufficently large then $G_i$ is trivial and $G_i\supseteq G_{i+1}$. Also, $G_1$ is a $p$-group and 
$\ord(G_0/G_1)$ is coprime to $p$. In particular, $\pi$ is tamely ramified at $P$ if and only if $G_1$ is the trivial group. More details
can be found in \citep[Ch. IV]{localfields}.\\

\begin{defn}[Ramification Divisor]
	For each $P\in X$ we define $n_P:=\sum_{i=0}^{\infty}(\ord(G_i)-1)$. Then we define the ramification divisor to be $R:=\sum_{P\in X}n_P[P]$.
\end{defn}
~
\begin{thm}
	Suppose that $P_1,\ldots ,P_r$ are the ramification points of a map $\pi:X\rightarrow Y$. Suppose further that for each 
	$i=1,\ldots ,r$ $\pi_i$ is a uniformising parameter in $\mathscr{O}_{P_i}(X)$. Then we define 
	$i_G(s):=\ord_{P_i}(s(\pi_i)-\pi_i)$ for each $s\in G$. If $e\in G$ is the identity then Hilbert's formula says
	\[
	    \sum_{s\neq e}i_G(s)=\sum_{i=0}^{\infty}\left(\ord(G_i)-1\right).
	\]
	This can be of use in calculations involving the Hurwitz formula:
	\[
	    2(g_X-1)=2n(g_Y-1)+\deg(R).
	\]
	This formula is true in all cases, both ramified and unramified.
\end{thm}
\begin{proof}
	For the sake of brevity we do not prove these statements here. See \citep[Prop. 4, $\S$1, Ch. IV]{localfields} for a proof of 
	Hilbert's formula. For a proof of the Hurwitz formula, see \citep[Cor. 2.4, Ch. IV]{hart}.
\end{proof}

\begin{cor}
	If $Y=\mathbb{P}^{1}$ and $n>1$ then there exist ramification points (i.e. points  $P\in X$ such that $e(P)>1$).
\end{cor}
\begin{proof}
	As $Y=\mathbb{P}^{1}$ we have that $g_{Y}=0$. It follows that 
		\begin{equation*}
			2g_{X}-2=-2n+\sum_{P\in X}(e(P)-1)
		\end{equation*}
	 and hence
		\begin{equation*}
			2g_{X}=-2(n-1)+\sum_{P\in X} (e(P)-1).
		\end{equation*}
	So, as we know that $g_{X}\geq 0$ and that $n>1$, it follows that there exists a $P\in X$ such that $e(P)-1\geq 1$,
	and $P$ is a ramification point.
\end{proof}

\section{Faithful Action on the Space of Global Poly-Differentials of an Algebraic Curve}


Let $X$ be a connected smooth projective algebraic curve over an algebraically closed field $k$ with a faithful action 
by a group $G$ of order $n$. Then $G$ induces an action on the vector space $H^0(X,\Omega_X)$ of global holomorphic 
differentials on $X$. Now $G$ also induces the map $\pi:X\rightarrow Y$, where $Y:= X/G$ is the quotient curve of $X$ by $G$. 
We will denote the genus of $X$ and $Y$ by $g_X$ and $g_Y$ respectively, and we let $p:=\mbox{char}(k)$. 
Much has been done in studying the action of $G$ on $H^0(X,\Omega_X)$ when this map is tamely ramified. 
However much less has been done when it is wildly ramified, so the question asked in \citep{faithfulaction} is when the 
action of $G$ on $H^0(X,\Omega_X)$ is faithful if $\pi$ is wildly ramified. 
The answer that \citep{faithfulaction} came to was the following theorem:\\

\begin{thm}\label{main}
	We assume that $g_X\geq 2$. Then $G$ acts faithfully on $H^0(X,\Omega_X)$ if and only if $p\neq 2$ or if $G$ 
	does not contain any hypperelliptic involution.
\end{thm}

Remember that a hyperelliptic involution is an automorphism $\sigma$ of $X$ such that the quotient $X/\langle\sigma\rangle$ 
is isomorphic to $\mathbb{P}_k^1$.

The first lemma in \citep{faithfulaction} is used later, but before it can be stated some notation needs to be introduced. 
If $D$ is a $G$-invariant divisor on $X$ then let $\pi_*^G(\mathscr{O}_X(D))$ be the subsheaf of the direct image 
$\pi_*(\mathscr{O}_X(D))$ which is fixed by the $G$-action on $\pi_*(\mathscr{O}_X(D))$. Also, 
by $\Big{\lfloor}\frac{\pi_*(D)}{n}\Big{\rfloor}$ we denote the divisor on $Y$ obtained by replacing the 
coefficient $m_Q$ of $Q$ in $\pi_*(D)$ by the integral part $\Big{\lfloor}\frac{m_Q}{n}\Big{\rfloor}$ for each $Q\in Y$. 
Then the following lemma is a vital step for computing the dimension $H^0(X,\Omega_X)^G$, which is heavily used 
in proving \ref{main}.\\

\begin{lem}
	Let $D$ be a $G$-invariant divisor on $X$. Then the sheaves $\pi_*^G(\mathscr{O}_X(D))$ and 
	$\mathscr{O}_Y\Big(\Big{\lfloor}\frac{\pi_*(D)}{n}\Big{\rfloor}\Big)$ are equal as subsheaves of 
	the constant sheaf $K(Y)$ on $Y$ . In particular the sheaf $\pi_*^G(\mathscr{O}_X(D))$ is an 
	invertible $\mathscr{O}_Y$-module.
\end{lem}

After this, the dimension of the subspace of $H^0(X,\Omega_X)$ that is fixed by the action of $G$ is computable, 
and turns out to depend  solely on the genus of $Y$ and on the degree of the divisor 
$\Big{\lfloor}\frac{\pi_*(R)}{n}\Big{\rfloor}$, where $R$ is the ramification divisor of $\pi$. More specifically, 
the following proposition is true.\\


\begin{prop}
\begin{equation*}
			\dim_k(H^0(X,\Omega_{X})^G)  =
				\begin{cases}
					g_Y & \mbox{if }\deg\Big{\lfloor}\frac{\pi_*(R)}{n}\Big{\rfloor}=0\\
					g_Y-1+\deg\Big{\lfloor}\frac{\pi_*(R)}{n}\Big{\rfloor} & \mbox{if } \deg\Big{\lfloor}\frac{\pi_*(R)}{n}\Big{\rfloor}>0.
				\end{cases}
		\end{equation*}
\end{prop}

The proof of the main theorem then follows and has a couple of parts. 
The first involves showing that if $G$ does not act faithfully then $\pi$ is not tamely ramified, using the Hurwitz formula 
and a dimension argument to get a contradiction. The second shows that $g_Y=0$, again using the Hurwitz formula and 
another dimension argument. These two facts allow the argument to be reduced to the following proposition, which is the 
final part of the paper:\\

\begin{prop}
	Let $p>0$ and let $G$ be a cyclic group of order $p$. We furthermore assume that $g_Y=0$. Then $G$ 
	acts trivially on $H^0(X,\Omega_X)$ if and only if one of the following three conditions hold: 
		\begin{itemize}
			\item $p=2$
			\item $g_X=0$
			\item $p=3$ and $g_X=1$.
		\end{itemize}
\end{prop}

\subsection{Extending to poly-differentials: first approach}
Next I trid to find a similar result for $H^0(X,\Omega_X^{\otimes m})$ for $m\geq 2$, 
starting with the following proposition, which is an analogue of proposition $2$.\\


\begin{prop}\label{prop4}
	Let $m\geq 2$. Then
		\begin{equation*}
			\dim_k(H^0(X,\Omega_{X}^{\otimes m})^G)  =
				\begin{cases}
					0 & \mbox{if } g_Y=0\ \text{and } \\ & \deg\Big(\Big{\lfloor}\frac{m\pi_*(R)}{n}\Big{\rfloor}\Big)<2m,\\
					1 & \mbox{if } g_Y=1\ \text{and } \\ & \deg\Big(\Big{\lfloor}\frac{m\pi_*(R)}{n}\Big{\rfloor}\Big)=0,\\
					(2m-1)(g_Y-1) +\deg\Big(\Big{\lfloor}{\frac{m\pi_* (R)}{n}}\Big{\rfloor}\Big) & \mbox{otherwise}.
				\end{cases}
		\end{equation*}
\end{prop}

\begin{proof}
	Let $K_Y$ be a canonical divisor on $Y$. Then we define $K_X:=\pi^*(K_Y)+R$, and hence $mK_X=m\pi^*(K_Y)+mR$; 
	note that $K_X$ is a canonical divisor by \citep[Chap. IV,\ prop 2.3]{hart} and is also $G$-invariant by definition. 
	Moreover, we have an equivariant isomorphism between 	the $G$-sheaves $\mathscr{O}_X(K_X)$ and $\Omega_X$. 
	We have that
		\[ 
		\Big{\lfloor}{\frac{\pi_*(mK_X)}{n}} \Big{\rfloor}
		= \Big{\lfloor}{\frac{\pi_*(m\pi^*(K_Y))+\pi_*(mR)}{n}} \Big{\rfloor}
		= mK_Y+\Big{\lfloor} {\frac{m\pi_*(R)}{n}}\Big{\rfloor}.
		\]
	
	Hence, by lemma 2 above, we see that 
		\[
		\pi_*^G(\Omega_X^{\otimes m})\cong \pi_*^G(\mathscr{O}_X(mK_X))\cong \mathscr{O}_Y\Big(mK_Y +\Big{\lfloor}{\frac{m\pi_*(R)}{n}}\Big{\rfloor}\Big).
		\]

	and so
		\begin{eqnarray*}
			\mbox{dim}_k(H^0(X,\Omega_{X}^{\otimes m})^G)&=&
			\mbox{dim}_k(H^0(Y,\pi_*^G(\Omega_{X}^{\otimes m})))\\
			&=&\mbox{dim}_k\Big(H^0\Big(Y,\mathscr{O}_Y\Big(mK_Y+\Big{\lfloor}{\frac{m\pi_*(R)}{n}}\Big{\rfloor}\Big)\Big)\Big).
		\end{eqnarray*}

	First we consider the case when $g_Y=0$. Suppose that $\deg\Big(\Big{\lfloor}{\frac{m\pi_* (R)}{n}}\Big{\rfloor}\Big) < 2m$. Then as $\deg(K_Y)=-2$ we have that $\deg(mK_Y)=-2m$ and hence that 		$\mbox{deg}\Big(mK_Y +\Big{\lfloor}{\frac{m\pi_*(R)}{n}}\Big{\rfloor}\Big)<0$. So by \citep[prop. 		3, {\S}8]{fulton} then $\mbox{dim}_k(H^0(X,\Omega_{X}^{\otimes m})^G)=0.$


	Otherwise we have 
		\[
		\deg\Big(mK_Y+\Big{\lfloor}\frac{m\pi_*(R)}{n}\Big{\rfloor}\Big)>-2=2g_Y-2=\deg(K_Y).
		\]

	Similarly, in the case when $g_Y\geq2$, after noting that $\deg(K_Y)>0$ we can see that
		\[
		\deg\Big(mK_Y+\Big{\lfloor}\frac{m\pi_*(R)}{n}\Big{\rfloor}\Big)\geq\deg(mK_Y)>\deg(K_Y).
		\]
	Also, if $g_Y=1$ and $\deg\Big(\Big{\lfloor}\frac{m\pi_*(R)}{n}\Big{\rfloor}\Big)\geq1$ then
		\[
		\deg\Big(mK_Y+\Big{\lfloor}\frac{m\pi_*(R)}{n}\Big{\rfloor}\Big)>0=\deg(K_Y).
		\]
	In all three of these cases it then follows from the Riemann-Roch theorem \citep[Cor. 2, {\S}8]{fulton} that 
		\begin{eqnarray*}
			\mbox{dim}_k\Big(H^0\Big(Y,\mathscr{O}_Y\Big(mK_Y+\Big{\lfloor}{\frac{m\pi_*(R)}{n}}\Big{\rfloor} \Big)\Big)\Big)= & 1-g_Y+\mbox{deg}\Big(mK_Y+\Big{\lfloor}							{\frac{m\pi_*(R)}{n}}\Big{\rfloor}\Big) \\
			= & (2m-1)(g_Y-1)+\mbox{deg}\Big(\Big{\lfloor}{\frac{m\pi_*(R)}{n}}\Big{\rfloor}\Big).
		\end{eqnarray*}

	Lastly, if $g_Y=1$ and $\deg\Big(\Big{\lfloor}\frac{m\pi_*(R)}{n}\Big{\rfloor}\Big)=0$ then $\Big(\Big{\lfloor}\frac{m\pi_*(R)}{n}\Big{\rfloor}\Big)=0$ (as it is a non-negative divisor of degree 		zero) and $mK_Y$ is equivalent to the zero divisor (by \citep[Chap. IV,\ Example 1.3.6]{hart}). 		So
		\begin{eqnarray*}
			\mbox{dim}_k\Big(H^0\Big(Y,\mathscr{O}_Y\Big(mK_Y+\Big{\lfloor}{\frac{m\pi_*(R)}{n}}\Big{\rfloor} \Big)\Big)\Big) &= & \mbox{dim}_k(H^0(Y,\mathscr{O}_Y(0))) \\ 
			& = & 1.
		\end{eqnarray*}
\end{proof}

Next I proved the following proposition, corresponding to proposition 3.\\


\begin{prop}
	Let $m\geq 2$. We assume that $p>0$, that $G$ is a cyclic group of order $p$ and also that $g_Y=0$. Then $G$ acts 
	trivially on $H^0(X,\Omega_X^{\otimes m})$ if and only if
		\begin{itemize}
			\item
				$g_X=0$ or
			\item
				$g_X=1$ or
			\item
				$p=g_X=m=2$.
		\end{itemize}
\end{prop}
\begin{proof}

	Let $P_1,\ldots ,P_r\in X$ be the ramification points of $\pi:X\rightarrow Y$, and for $i=1,\ldots ,r$
	define $N_i\in \mathbb{N}$ by $\ord_{P_i}(\sigma(\pi_i)-\pi_i)=N_i+1$, where $\pi_i$ is a 
	local parameter at $P_i$ and $\sigma$ is a generator of $G$. Note that $p\nmid N_i$ for any $i$ by 
	\citep[Lem. 1, pg. 87]{naka}. Also, note that $R=\sum_{i=1}^r(N_i+1)(p-1)[P_i]$ by 
	\citep[Prop. 4, {\S}1, Ch. IV]{localfields} and let $N=\sum_{i=1}^rN_i$ and $k=N+r=\sum_{i=1}^r(N_i+1)$.

	If $g_X=0$ then $\mbox{deg}(mK_X)=-2m<0$; hence $\mbox{dim}_k(H^0(X,\Omega_X^{\otimes m}))=0$ by 
      \citep[prop. 3, {\S}8]{fulton} and the action must be trivial.

	We now look at the case when $g_X=1$. By \citep[Chap. IV,\ Example 1.3.6]{hart} we know that $K_X$ 
	is equivalent to the zero divisor. Hence 
	$H^0(X,\Omega_X^{\otimes m})\cong H^0(X,\mathscr{O}_X(mK_X)) \cong H^0(X,\mathscr{O}_X(0))=k$, 
	the space of constant functions, which is $G$ invariant. Hence the action of $G$ on $H^0(X,\Omega_X^{\otimes m})$ 
	is trivial.

	From now on we consider the last case, i.e. we assume that $g_X\geq 2$, so then we have that 
	$\mbox{deg}(mK_X)=m(2g_X-2)>2g_X-2=\mbox{deg}(K_X)$. So by the Riemann-Roch theorem \citep[Cor. 2, {\S}8]{fulton},
		\begin{equation}
		\mbox{dim}_k(H^0(X,\Omega_X^{\otimes m}))=\mbox{deg}(mK_X)+1-g_X=2m(g_X-1)+1-g_X=(2m-1)(g_X-1).
		\end{equation}

	Again, due to the Hurwitz formula we have $2g_X-2=-2p+k(p-1)$, and hence we can write
		\[
		\mbox{dim}_k(H^0(X,\Omega_X^{\otimes m}))=(2m-1)(g_X-1)=(2m-1)\Big(\frac{k(p-1)-2p}{2}\Big).
		\]

	Now if $\mbox{deg}\Big(\Big{\lfloor}{\frac{m\pi_*(R)}{p}}\Big{\rfloor}\Big)\geq 2m$ then by proposition 2 we have
		\begin{eqnarray*}	
			\mbox{dim}_k(H^0(X,\Omega_X^{\otimes m})^G) & = & 1-2m+\mbox{deg}\Big(\Big{\lfloor}{\frac{m\pi_*(R)}{p}}\Big{\rfloor}\Big) \\
			&= & 1-2m+\sum_{i=1}^r \Big{\lfloor}{\frac{m(N_i+1)(p-1)}{p}}\Big{\rfloor} \\
			&= & 1-2m+mk+\sum_{i=1}^r\Big{\lfloor}{\frac{-m(N_i+1)}{p}}\Big{\rfloor}. 
		\end{eqnarray*}
	
	If we have $p=g_X=m=2$ then on the one hand we see that $\mbox{dim}_k(H^0(X,\Omega_X^{\otimes m}))=3$. 
	On the other hand, we first note that the equation $2g_X-2=-2p+k(p-1)$ implies that $k=6$. 
	So $\mbox{deg}\Big(\Big{\lfloor}{\frac{m\pi_*(R)}{p}}\Big{\rfloor}\Big)=mk+\sum_{i=1}^r\Big{\lfloor}{\frac{-m(N_i+1)}{p}}\Big{\rfloor}=2k-k>2m$. 		Then we can compute $\mbox{dim}_k(H^0(X,\Omega_X^{\otimes m})^G)=9+\sum\lfloor-N_i-1\rfloor=3$. So 	the dimensions are equal and the action of $G$ on $H^0(X,\Omega_X^{\otimes m})$ is trivial. This 		completes the if direction of the proof.

	Now we assume that the action is trivial. This first implies that 
	$\deg\Big(\Big{\lfloor}\frac{m\pi_*(R)}{n}\Big{\rfloor}\Big)\geq 2m$ because otherwise we would 
	have $\dim_k(H^0(X,\Omega_X^{\otimes m})^G)=0$ by the previous proposition, but we know that 
	$\dim_k(H^0(X,\Omega_X^{\otimes m}))=(2m-1)(g_X-1)$ (6) is strictly positive. So we have:
		\begin{eqnarray}
			(2m-1)\frac{k(p-1)-2p}{2} & = & \dim_k(H^0(X,\Omega_X^{\otimes m})) \nonumber\\
			& = & \dim_k(H^0(X,\Omega_X^{\otimes m})^G) \nonumber\\
			& = & 1-2m+mk+\sum_{i=1}^r\Big{\lfloor}\frac{-m(N_i+1)}{p}\Big{\rfloor}\nonumber \\
			& \leq & 1-2m+mk+\sum_{i=1}^r\frac{-m(N_i+1)}{p}\nonumber \\
			& = & 1-2m+mk-\frac{mk}{p}.
		\end{eqnarray}

	If we then multiply by $2p$ we obtain
		\begin{equation}
			(2mk-k-4m+2)p^2+(-4mk+k-2+4m)p+2mk\leq 0.
		\end{equation}

	The left hand side of this then factorises as
		\begin{multline*}
			(k-2)(2m-1)p^2-((k-2)(2m-1)+2mk)p+2mk = \\
			(p-1)((k-2)(2m-1)p-2mk)
		\end{multline*}

	As the Hurwitz formula tells us that $-2p+k(p-1)=2g_X-2 \geq 2$ then we see that 
		\begin{equation}
			k\geq \frac{2+2p}{p-1}=2+\frac{4}{p-1}>2.
		\end{equation}

	So from (7) and (8) we see that
		\begin{eqnarray}
			p & \leq & \frac{2mk}{(k-2)(2m-1)}\nonumber\\
			& = & \frac{k}{k-2}\cdot\frac{2m}{2m-1}\nonumber\\
			& = & \Big( 1+\frac{2}{k-2} \Big) \Big(1+\frac{1}{2m-1} \Big)\\
			& \leq & 4, \nonumber	
		\end{eqnarray}

	i.e. $p=2$ or $p=3$. 

	Suppose that $p=3$. Then from (8) we have $k\geq 4$. However, from (10) and (9) we also have that 
		\begin{eqnarray*}
			3 & \leq &\Big( 1+\frac{2}{k-2} \Big) \Big(1+\frac{1}{2m-1} \Big)\\
			& \leq & \Big( 1+\frac{2}{k-2} \Big) \frac{4}{3}\\
			& \leq & \frac{8}{3},
		\end{eqnarray*}

	a contradiction.

	Lastly, we come to the case when $p=2$. From (9) we see that $2\leq \Big(1+\frac{2}{k-2}\Big)\frac{4}{3}$ 
	and hence $k\leq 6$. However, from (8) we also see that $k\geq 6$, so $k=6$. Then from (6) $2m-1=1-2m+6m-3m$
	and $m=2$. Finally, the Hurwitz formula gives us that $2g_X-2=-4+6=2$ and hence $g_X=2$. 
	This completes the only if direction of the proof.
\end{proof}

\subsection{Extending to poly-differentials: Nakajima's paper}


In this subsection we present another approach to proving Proposition 5. This approach is only slightly shorter, it uses a 
comparatively deep theorem by Nakajima \citep{naka} but it will apply to more general Riemann-Roch spaces.

Assume that $G$ is cyclic of order $p>0$. Let $P_1,\ldots ,P_r\in X$ be the ramification points of $\pi$. 
For $i=1,\ldots ,r$, let $N_i$ be defined by $N_i+1=\ord_{P_i}(\sigma\cdot\pi_i-\pi_i)$, where $\sigma$ is a generator 
of $G$, and $\pi_i$ is a function such that $\ord_{P_i}(\pi_i)=1$. We know from \citep[lem 1, pg.87]{naka} that $p\nmid N_i$. 
Also, for a rational number $a$ we will denote the fractional part by $<a>$.

Also, if we let $V$ be the $k[G]$ module with $k$-basis $\{e_1,\ldots ,e_p\}$ and $G$-action defined by 
$\sigma\cdot e_l=e_l+e_{l-1}$, with $1\leq l\leq p$ and $e_0=0$, then the subspaces $V_j$ spanned by $\{e_1,\ldots ,e_j\}$ 
are all the indecomposable $k[G]$-modules up to isomorphism \citep{naka}.
\\

\begin{prop}
	If $D$ is a $G$-invariant divisor on $X$ such that $\deg(D)>2g_X-2$ then the action of $G$ on 
	$H^0(X,\mathscr{O}_X(D))$ is trivial if and only if
		\[ 
		(p-1)\deg(D)=p\Big(g_X-g_Y-\sum_{i=1}^r\Big<\frac{n_i}{p}\Big>\Big),
		\]
	where the $n_i$ are the coefficients of the ramification points in $D$. 
\end{prop}
\begin{proof}
	First note that the action is trivial if and only if
		\[
		\dim_kH^0(X,\mathscr{O}_X(D))=\dim_kH^0(X,\mathscr{O}_X(D))^G.
		\]

	Now from Nakajimas paper \citep{naka}, we know that $H^0(X,\mathscr{O}_X(D))\cong \oplus _{j=1}^p m_j\cdot V_j$, with 
	\begin{equation*}
		m_j=
			\begin{cases}
				\sum_{i=1}^r\Big(\frac{N_i}{p}+\Big{<}\frac{n_i-jN_i}{p}\Big{>}-\Big{<}\frac{n_i-(j-1)N_i}{p}\Big{>}\Big) & \mbox{if }1\leq j\leq p-1 \\
				\frac{1}{p}\deg(D)-g_Y+1-\sum_{i=1}^r\Big(\frac{(p-1)N_i}{p}+\Big<\frac{n_i-(p-1)N_i}{p}\Big>\Big) & \mbox{if }j=p.
			\end{cases}
	\end{equation*}
	It is also clear that the only $G$-invariant part of each submodule $V_j$ is $e_1$. Hence $\dim_kH^0((X,\mathscr{O}_X(D))^G) = \sum_{j=1}^p m_j$.

	Note that this sum cancels in a very natural manner; we have that
	\begin{eqnarray*}
		\lefteqn{\dim_k(H^0((X,\mathscr{O}_X(D))^G) = \sum_{j=1}^p m_j} \\
		& = & \sum_{j=1}^{p-1}  \sum_{i=1}^r\Big(\frac{N_i}{p}+\Big<\frac{n_i-jN_i}{p}\Big>-\Big<\frac{n_i-(j-1)N_i}{p}\Big>\Big) \\
		& + & \frac{1}{p}\deg(D)-g_Y+1+\sum_{i=1}^r\Big(\frac{N_i}{p}-N_i-\Big<\frac{n_i-(p-1)N_i}{p}\Big>\Big) \\
		& = & \frac{1}{p}\deg(D)-g_Y+1-\sum_{i=1}^r \Big<\frac{n_i}{p}\Big>.
	\end{eqnarray*}

	Now as $\deg(D)>2g_X-2$ then $\dim_kH^0(X,\mathscr{O}_X(D)) =\deg(D)+1-g_X$ by the Riemann Roch theorem. 
	So the action of $G$ on $H^0(X,\mathscr{O}_X(D))$ is trivial if and only if
	\begin{equation*}
		\deg(D)+1-g_X  = \frac{1}{p}\deg(D)-g_Y+1-\sum_{i=1}^r\Big<\frac{n_i}{p}\Big>. \label{hi}
	\end{equation*}

This then clearly rearranges to $(p-1)\deg(D)=p\Big(g_X-g_Y-\sum_{i=1}^r\Big<\frac{n_i}{p}\Big>\Big)$, as we want.
\end{proof}

\begin{cor}
      Suppose that $\deg(D)\geq 2g_X$ and $g_Y=0$. Then the action of $G$ on $H^0(X,\mathscr{O}_X(D))$ is trivial if and 
      only if $p | n_i$ for all $i$, $\deg(D)=2g_X$ and, unless $g_X=0$, $p=2$.
\end{cor}
\begin{proof}
The following inequalities always hold under the stated assumptions:
	\[
		(p-1)\deg(D)\geq (p-1)2g_X \geq pg_X \geq pg_X-p\sum_{i=1}^r\Big<\frac{n_i}{p}\Big>.
	\]
Now the first inequality is an equality if and only if $\deg(D)=2g_X$. The second is an equality if and 
only if either $g_X=0$ or $p=2$. Lastly, the third inequality is an equality if and only if 
$\sum_{i=1}^r\Big<\frac{n_i}{p}\Big>=0$, which is the case if and only if each $n_i$ is divisible by $p$. 
Now proposition 4 implies corollary 3.
\end{proof}

\begin{cor}
Suppose that $\deg(D)= 2g_X-1$ and $g_Y=0$. Then the action of $G$ on $H^0(X,\mathscr{O}_X(D))$ is 
trivial if and only if one of the following conditions hold:
	\begin{itemize}
		\item
			$g_X=0$
		\item
			$p=2$ and $\sum_{i=1}^r\Big<\frac{n_i}{p}\Big>=\frac{1}{2}$
		\item
			$g_X=1$ and $\sum_{i=1}^r\Big<\frac{n_i}{p}\Big>=\frac{1}{p}$
		\item
			$g_X=2$, $p=3$ and $p\mid n_i$ for all $i$.
	\end{itemize}
\end{cor}
It should be noted that in the last two cases the Hurwitz formula implies that $r\leq 2$. If $r=1$ then the conditions 
``$\sum_{i=1}^r\Big<\frac{n_i}{p}\Big>=\frac{1}{p}$" and ``$p\mid n_i$ for all $i$" are already implied by ``$\deg(D)=2g_X-1$".
\begin{proof}
Firstly, if $g_X=0$ then $\deg(D)=-1<0$, so $\dim_kH^0(X,\mathscr{O}_X(D))=0$ and the action is trivial.

Now note that as $\deg(D)=2g_X-1$ we conclude from proposition 4 that the action is trivial if and only if 
	\begin{equation*}
		(p-1)(2g_X-1)=p\Big(g_X-\sum_{i=1}^r\Big<\frac{n_i}{p}\Big>\Big).
	\end{equation*}
If $p=2$ then this is equivalent to $2g_X-1=2g_X-2\sum_{i=1}^r\Big<\frac{n_i}{p}\Big>$, and hence 
to $\sum_{i=1}^r\Big<\frac{n_i}{p}\Big>=\frac{1}{2}$.

If $g_X=1$ then this is equivalent to $p-1=p-p\sum_{i=1}^r\Big<\frac{n_i}{p}\Big>$, and hence 
to $\sum_{i=1}^r\Big<\frac{n_i}{p}\Big>=\frac{1}{p}$.

Lastly, if $p\geq 3$ and $g_X\geq 2$ then we have that $g_X\geq \frac{p-1}{p-2}$ which is equivalent 
to the first inequality in the chain
\begin{equation*}
	(p-1)(2g_X-1)\geq pg_X\geq pg_X-p\sum_{i=1}^r\Big<\frac{n_i}{p}\Big>.
\end{equation*}
Hence the action is trivial if and only if both inequalities are equalities, which is the case if and 
only if $p=3,\ g_X=2$ and $p\mid n_i$ for all $i$.
\end{proof}
We will now assume that $K_X$ denotes a $G$-invariant divisor on $X$ such that $\mathscr{O}_X(K_X)\cong \Omega_X$ 
as $G$-sheaves. Remembering that $\Omega_X^{\otimes m}\cong \mathscr{O}_X(mK_X)$, we will now use these results to prove 
the following statement about the action of $G$ on the group $H^0(X,\Omega_X^{\otimes m})$, where $m\geq 2$. \\

\begin{cor}
Let $m\geq 2$. We assume that $g_X\geq 2$ and that $g_Y=0$. Then $G$ acts trivially on $H^0(X,\Omega_X^{\otimes m})$ if 
and only if $p=g_X=m=2$.
\end{cor}
\begin{proof}

As $g_X\geq 2$ then we have that $\deg(mK_X)\geq 2g_X$. So by Corollary 3, we have a trivial action if and only if $p=2,\ \deg(mK_X)=2g_X$ and $p|n_i$ for all $i$. Now $\deg(mK_X)=2g_X$ means that $m(2g_X-2)=2g_X$, so $m(g_X-1)=g_X$, and hence that $m=g_X=2$. This proves the only-if-direction of the proposition. To prove the if-direction it suffices to show that the co-efficents of the ramification points $K_X$ (and hence of $mK_X$) are always divisible by $p$ if $p=2$. To start with note that we can write $K_X$ as $\pi^*(K_Y)+R$, where $R=\sum_{i=1}^r(N_i+1)(p-1)[P]$ is the ramification divisor. First note that as $p\nmid N_i$ and $p=2$ then $p\mid N_i+1$. Also, $\pi^*$ multiplies the co-efficents of ramification points by $p$, hence the coefficents of all ramification points of $\pi^*(K_Y)$ are divisible by $p$. So in the sum $\pi^*(K_Y)+R$ all co-efficients of ramification points are divisible by $p$, and we are done.
\end{proof}

\subsection{Main theorem}

If $G$ is a $p$-group, we prove in this section that the action of $G$ on $H^0(X,\Omega_X^{\otimes m})$ is faithful 
except in rare cases, see theorem 4 below. In the proof of theorem 3 (see \citep{faithfulaction}) we see that the action of 
$G$ on $H^0(X,\Omega_X)$ is always faithful in the tamely ramified case, provided that $g_X\geq2$. However, if $m\geq 2$, 
the action of $G$ on $H^0(X,\Omega_X^{\otimes m})$ is not necessarily faithful in the tamely ramified case. An example of this is given below.\\

\begin{ex}
Suppose that $\cha(k)\geq 3$, $Y=\mathbb{P}^1_k$ and that the map $X\rightarrow \mathbb{P}^1_k$ is of degree $2$, 
ramified at six points, each with ramification index of 2. Note that $G$ is of order two in this case and that as 
$\cha(k)>2$ the map to the quotient space must be tamely ramified. Now by the Hurwitz formula $2(g_X-1)=2$ and 
hence $g_X=2$, so by equation (6) $\dim_kH^0(X,\Omega_X^{\otimes 2})=3$. Also, by prop. 4, we have 
$\dim_k(H^0(X,\Omega_X^{\otimes 2})^G)=-3+\deg\lfloor\pi_*(R)\rfloor=3$, so the action is trivial despite being 
tamely ramified.
\end{ex}
~

\begin{thm}
	Let $g_X\geq 2$ and $m\geq 2$. Suppose further that $G$ is a $p$-group. Then $G$ does not act 
faithfully on $H^0(X,\Omega_X^{\otimes m})$ if and only if $m=p=g_X=2$ and $g_Y=0$.
\end{thm}
\begin{proof}
	We first prove the if direction. Note that $G$ cannot be the trivial group, as otherwise we would have $Y=X$ and 
	hence $g_Y=g_X$. So as $p=2$ and $G$ is a non-trivial $p$-group, then $G$ has a subgroup of order $2$. Looking at 
	proposition 5 this subgroup must act trivially on $H^0(X,\Omega_X^{\otimes m})$, hence $G$ cannot act faithfully.
	
	We now prove the only if direction. As $G$ is a p-group and does not act faithfully on $H^0(X,\Omega_X^{\otimes m})$, 
	there must be a subgroup of order $p$, say $H$, of $G$ such the action of $H$ on $H^0(X,\Omega_X^{\otimes m})$ is 
	trivial. Now let $Z=X/H$, the quotient space of $X$ by $H$ with genus $g_Z$. We have a canonical map $Z\rightarrow Y$, 
	so if $g_Z=0$ then we have $-2=2n(g_Y-1)+\deg(R)$, and so as $\deg(R)$ is non-negative we must have $g_Y=0$. 
	So we may as well assume that the action of $G$ on $H^0(X,\Omega_X^{\otimes m})$ is itself trivial, and that 
	the order of $G$ is $p$. Now $p=g_X=m=2$ follows from the proposition 5.
	
	Now we show that $g_Y=0$. We have that $\dim_kH^0(X,\Omega_X^{\otimes m})=\dim_k(H^0(X,\Omega_X^{\otimes m}))^G$. 
	But $H^0(X,\Omega_X^{\otimes m})$ has dimension $(2m-1)(g_X-1)$ by (6), and as $m,g_X\geq 2$ then this the 
	dimension is greater than $1$. But $H^0(X,\Omega_X^{\otimes m})>1$ only in the last case of proposition 
	\ref{prop4}, hence we have
		\begin{equation*}
			(2m-1)(g_X-1)=(2m-1)(g_Y-1)+\deg\Big(\Big{\lfloor}\frac{m\pi_*(R)}{p}\Big{\rfloor}\Big).
		\end{equation*}
	The Hurwitz formula also says
		\begin{equation*}
			2(g_X-1)=2p(g_Y-1)+\deg(R).
		\end{equation*}
	Combining the last two equations we see that
		\begin{eqnarray*}
			2(2m-1)(g_Y-1)+2\deg\Big(\Big{\lfloor}\frac{m\pi_*(R)}{p}\Big{\rfloor}\Big) & = & 2(2m-1)(g_X-1)\\
			& = & 2p(2m-1)(g_Y-1)+(2m-1)\deg(R)
		\end{eqnarray*}
	which can be rearranged as
		\begin{equation*}
			(2m-1)(2p-2)(g_Y-1)=2\deg\Big(\Big{\lfloor}\frac{m\pi_*(R)}{p}\Big{\rfloor}\Big)-(2m-1)\deg(R).
		\end{equation*}
	So if we can show that $2\deg\Big(\Big{\lfloor}\frac{m\pi_*(R)}{p}\Big{\rfloor}\Big)-(2m-1)\deg(R)<0$ 
	then we will have $g_Y-1<0$ and hence $g_Y=0$.

	To show this, let $P_1,\ldots ,P_r\in X$ be the ramification poinbts of $\pi:X\rightarrow Y$ and let $n_i$, 
	$i=1,\ldots ,r$, be defined by $R=\sum_{i=1}^rn_i[P_i]$. Note that the ramifiation 
	index at each $P_i$ is equal to $p$ and that the fibre $\pi^{-1}(\pi(P_i))$ contains only $P_i$.
		\begin{eqnarray*}
			\lefteqn{2\deg\Big(\Big{\lfloor}\frac{m\pi_*(R)}{p}\Big{\rfloor}\Big)-(2m-1)\deg(R)}\\
			&=\sum_{i=1}^r\Big(2\Big{\lfloor}m\cdot \frac{n_i}{p}\Big{\rfloor}-(2m-1)\cdot n_i\Big) \\
			&\leq  \sum_{i=1}^r\Big(2m\cdot\frac{n_i}{p}-(2m-1)p\cdot\frac{n_i}{p}\Big) \\
			& =  (2m-p(2m-1))\sum_{i=1}^r\frac{n_i}{p}.
		\end{eqnarray*}
	Now as $p,m\geq 2$ then we have $2m-p(2m-1)\leq 2m-2(2m-1)=2(1-m)<0$ and we are done as 
	$\sum_{i=1}^r\frac{n_i}{p}$ is positive.
	
	
\end{proof}



\bibliography{/home/jtait/files/Documents/Maths/Bibliography/biblio.bib}
\bibliographystyle{alpha}


\end{document}