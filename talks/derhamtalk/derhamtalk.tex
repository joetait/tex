%%%%%%%%%%%%%%%%%%%%%%%%%%%%%%%%%%%%%%%%%
% Beamer Presentation
% LaTeX Template
% Version 1.0 (10/11/12)
%
% This template has been downloaded from:
% http://www.LaTeXTemplates.com
%
% License:
% CC BY-NC-SA 3.0 (http://creativecommons.org/licenses/by-nc-sa/3.0/)
%
%%%%%%%%%%%%%%%%%%%%%%%%%%%%%%%%%%%%%%%%%

%----------------------------------------------------------------------------------------
%	PACKAGES AND THEMES
%----------------------------------------------------------------------------------------

\documentclass{beamer}
\usepackage{etex}

\mode<presentation> {

% The Beamer class comes with a number of default slide themes
% which change the colors and layouts of slides. Below this is a list
% of all the themes, uncomment each in turn to see what they look like.

%\usetheme{default}
%\usetheme{AnnArbor}
%\usetheme{Antibes}
%\usetheme{Bergen}
%\usetheme{Berkeley}
%\usetheme{Berlin}
%\usetheme{Boadilla}
%\usetheme{CambridgeUS}
%\usetheme{Copenhagen}
%\usetheme{Darmstadt}
%\usetheme{Dresden}
%\usetheme{Frankfurt}
%\usetheme{Goettingen}
%\usetheme{Hannover}
%\usetheme{Ilmenau}
%\usetheme{JuanLesPins}
%\usetheme{Luebeck}
\usetheme{Madrid}
%\usetheme{Malmoe}
%\usetheme{Marburg}
%\usetheme{Montpellier}
%\usetheme{PaloAlto}
%\usetheme{Pittsburgh}
%\usetheme{Rochester}
%\usetheme{Singapore}
%\usetheme{Szeged}
%\usetheme{Warsaw}

% As well as themes, the Beamer class has a number of color themes
% for any slide theme. Uncomment each of these in turn to see how it
% changes the colors of your current slide theme.

%\usecolortheme{albatross}
%\usecolortheme{beaver}
%\usecolortheme{beetle}
%\usecolortheme{crane}
%\usecolortheme{dolphin}
%\usecolortheme{dove}
%\usecolortheme{fly}
%\usecolortheme{lily}
%\usecolortheme{orchid}
%\usecolortheme{rose}
%\usecolortheme{seagull}
%\usecolortheme{seahorse}
%\usecolortheme{whale}
%\usecolortheme{wolverine}

\setbeamertemplate{footline} % To remove the footer line in all slides uncomment this line
%\setbeamertemplate{footline}[page number] % To replace the footer line in all slides with a simple slide count uncomment this line

\setbeamertemplate{navigation symbols}{} % To remove the navigation symbols from the bottom of all slides uncomment this line
}

\usepackage[all]{xy}  % Tool for drawing commutative diagrams
%\usepackage{tikz}     % A tool for drawing
\usepackage{graphicx} % Allows including images
\usepackage{booktabs} % Allows the use of \toprule, \midrule and \bottomrule in tables

%----------------------------------------------------------------------------------------
%	TITLE PAGE
%----------------------------------------------------------------------------------------

\title[de Rham cohomology of hyperelliptic curves]{Group actions on the algebraic de Rham cohomology of hyperelliptic curves} % The short title appears at the bottom of every slide, the full title is only on the title page

\author{Joe Tait} % Your name
\institute[Southampton University] % Your institution as it will appear on the bottom of every slide, may be shorthand to save space
{
The University of Southampton   \\ % Your institution for the title page
\medskip
\textit{joe.tait@soton.ac.uk} % Your email address
}
\date{\today} % Date, can be changed to a custom date

\begin{document}


%%%%%%%%%%%%%%%%%%%%%%%%%%%%%%%%%%%%%%%%%%%%%%%%%%%%%%%%%%%%%%%%%%%%%%%%%%%%%%%%%%%%

%%%%%%%%%%%%  Binary Complex Stuff %%%%%%%%%%%%%%%%%%%%%%%%%%%%%%

%%%%%%%%%%%%%%%%%%%%%%%%%%%%%%%%%%%%%%%%%%%%%%%%%%%%%%%%%%%%%%%%%%%%%%%%%%%%%%%%%%%%



\begin{frame}
\only<1>{\begin{displaymath}
    \xymatrix{
    0 \ar[r] & \Omega^0_X \ar@[white][d] \ar[r] & \Omega_X^1 \ar[r] \ar@[white][d] & \ldots \ar[r] & \Omega_X^n \ar@[white][d] \ar[r] & 0 \\ 
    {\color{white}0 \ar@[white][r]} & {\color{white}C^0(\Omega^0_X) \ar@[white][d] \ar@[white][r] } & {\color{white}C^0(\Omega_X^1) \ar@[white][r] \ar@[white][d] } & {\color{white}\ldots \ar@[white][r] } & {\color{white}C^0(\Omega_X^n) \ar@[white][d] \ar@[white][r] } & {\color{white}0} \\
    {\color{white}0 \ar@[white][r]} & {\color{white}C^1(\Omega^0_X) \ar@[white][d] \ar@[white][r] } & {\color{white}C^1(\Omega_X^1) \ar@[white][r] \ar@[white][d] } & {\color{white}\ldots \ar@[white][r] } & {\color{white}C^1(\Omega_X^n) \ar@[white][d] \ar@[white][r] } & {\color{white}0} \\
    &   {\color{white} \vdots}                       & {\color{white}\vdots                        } & {\color{white}\ldots        } & {\color{white}\vdots                        } &  \\
    }
\end{displaymath}}

\only<2>{\begin{displaymath}
    \xymatrix{
    0 \ar[r] & \Omega^0_X \ar[d] \ar[r] & \Omega_X^1 \ar[r] \ar@[white][d] & \ldots \ar[r] & \Omega_X^n \ar@[white][d] \ar[r] & 0 \\ 
    {\color{white}0 \ar@[white][r]} & C^0(\Omega^0_X) \ar[d] \ar@[white][r]  & {\color{white}C^0(\Omega_X^1) \ar@[white][r] \ar@[white][d] } & {\color{white}\ldots \ar@[white][r] } & {\color{white}C^0(\Omega_X^n) \ar@[white][d] \ar@[white][r] } & {\color{white}0} \\
    {\color{white}0 \ar@[white][r]} & C^1(\Omega^0_X) \ar[d] \ar@[white][r]  & {\color{white}C^1(\Omega_X^1) \ar@[white][r] \ar@[white][d] } & {\color{white}\ldots \ar@[white][r] } & {\color{white}C^1(\Omega_X^n) \ar@[white][d] \ar@[white][r] } & {\color{white}0} \\
    &    \vdots                       & {\color{white}\vdots                        } & {\color{white}\ldots        } & {\color{white}\vdots                        } &  \\
    }
\end{displaymath}}

\only<3>{\begin{displaymath}
    \xymatrix{
    0 \ar[r] & \Omega^0_X \ar[d] \ar[r] & \Omega_X^1 \ar[r] \ar[d] & \ldots \ar[r] & \Omega_X^n \ar[d] \ar[r] & 0 \\ 
    0 \ar[r] & C^0(\Omega^0_X) \ar[d] \ar[r] & C^0(\Omega_X^1) \ar[r] \ar[d] & \ldots \ar[r] & C^0(\Omega_X^n) \ar[d] \ar[r] & 0 \\
    0 \ar[r] & C^1(\Omega^0_X) \ar[d] \ar[r] & C^1(\Omega_X^1) \ar[r] \ar[d] & \ldots \ar[r] & C^1(\Omega_X^n) \ar[d] \ar[r] & 0 \\
    &   \vdots                       & \vdots                        & \ldots        & \vdots                        & \\
    }
\end{displaymath}}

\only<4>{\begin{displaymath}
    \xymatrix{
    0 \ar[r] & \Omega^0_X \ar[d] \ar[r] & \Omega_X^1 \ar[r] \ar[d] & \ldots \ar[r] & \Omega_X^n \ar[d] \ar[r] & 0 \\ 
    0 \ar[r] & {\color{red}C^0(\Omega^0_X)} \ar[d] \ar[r] & C^0(\Omega_X^1) \ar[r] \ar[d] & \ldots \ar[r] & C^0(\Omega_X^n) \ar[d] \ar[r] & 0 \\
    0 \ar[r] & C^1(\Omega^0_X) \ar[d] \ar[r] & C^1(\Omega_X^1) \ar[r] \ar[d] & \ldots \ar[r] & C^1(\Omega_X^n) \ar[d] \ar[r] & 0 \\
    &   \vdots                       & \vdots                        & \ldots        & \vdots                        & \\
    }
\end{displaymath}

\begin{displaymath}
\xymatrix{
0 \ar[r]  & C^0(\Omega_X^0) \ar@[white][r] & {\color{white} C^0(\Omega_X^0) \oplus \Omega_X^1 } \ar@[white][r] & {\color{white} C^1(\Omega_X^0) \oplus C^0(\Omega_X^1) \oplus \Omega_X^2} \ar@[white][r] &
}
\end{displaymath}}


\only<5>{\begin{displaymath}
    \xymatrix{
    0 \ar[r] & \Omega^0_X \ar[d] \ar[r] & \Omega_X^1 \ar[r] \ar[d] & \ldots \ar[r] & \Omega_X^n \ar[d] \ar[r] & 0 \\ 
    0 \ar[r] & C^0(\Omega^0_X) \ar[d] \ar[r] & {\color{red}C^0(\Omega_X^1)} \ar[r] \ar[d] & \ldots \ar[r] & C^0(\Omega_X^n) \ar[d] \ar[r] & 0 \\
    0 \ar[r] & {\color{red}C^1(\Omega^0_X)} \ar[d] \ar[r] & C^1(\Omega_X^1) \ar[r] \ar[d] & \ldots \ar[r] & C^1(\Omega_X^n) \ar[d] \ar[r] & 0 \\
    &   \vdots                       & \vdots                        & \ldots        & \vdots                        & \\
    }
\end{displaymath}

\begin{displaymath}
\xymatrix{
0 \ar[r]  & C^0(\Omega_X^0) \ar[r] & C^1(\Omega_X^0) \oplus C^0(\Omega_X^1)  \ar@[white][r] & {\color{white} C^1(\Omega_X^0) \oplus C^0(\Omega_X^1) \oplus \Omega_X^2} \ar@[white][r] &
}
\end{displaymath}}

%\only<6>{\begin{displaymath}
%    \xymatrix{
%    0 \ar[r] & \Omega^0_X \ar[d] \ar[r] & \Omega_X^1 \ar[r] \ar[d] & \ldots \ar[r] & \Omega_X^n \ar[d] \ar[r] & 0 \\ 
%    0 \ar[r] & C^0(\Omega^0_X) \ar[d] \ar[r] & C^0(\Omega_X^1) \ar[r] \ar[d] & \ldots \ar[r] & C^0(\Omega_X^n) \ar[d] \ar[r] & 0 \\
%    0 \ar[r] & C^1(\Omega^0_X) \ar[d] \ar[r] & {\color{red}C^1(\Omega_X^1)} \ar[r] \ar[d] & \ldots \ar[r] & C^1(\Omega_X^n) \ar[d] \ar[r] & 0 \\
%    &   \vdots                       & \vdots                        & \ldots        & \vdots                        & \\
%    }
%\end{displaymath}
%
%\begin{displaymath}
%\xymatrix{
%0 \ar[r]  & C^0(\Omega_X^0) \ar[r] & C^1(\Omega_X^0) \oplus C^0(\Omega_X^1)  \ar[r] & C^2(\Omega_X^0) \oplus C^1(\Omega_X^1) \oplus C^0(\Omega_X^2) & 
%}
%\end{displaymath}}
\end{frame}

%%%%%%%%%%%%%%%%%%%%%%%%%%%%%%%%%%%%%%%%%%%%%%%%%%%%%%%%%%%%%%%%%%%%%%%%%%%%%%%%%%%%

%%%%%%%%%%%%           Cech Complex                        %%%%%%%%%%%%%%%%%%%%%%%%%

%%%%%%%%%%%%%%%%%%%%%%%%%%%%%%%%%%%%%%%%%%%%%%%%%%%%%%%%%%%%%%%%%%%%%%%%%%%%%%%%%%%%

%\begin{frame}
%\frametitle{$\check{\text C}$ech complex}
%The $\check {\text C}$ech complex of $X$ with respect to this cover is
%\begin{displaymath}
%    \xymatrix{
%        \Omega^0_X(U_0)\times \Omega_X^0(U_\infty) \ar[r]^{d \times d} \ar[d]^{\check d} & \Omega^1_X(U_0) \times \Omega_X^1(U_\infty) \ar[d]^{\check d} \\
%        \Omega_X^0(U_0\cap U_\infty) \ar[r]^{d}                      & \Omega_X^1(U_0 \cap U_\infty)
%    }
%\end{displaymath}
%\pause
%
%which gives rise to the following total complex
%\begin{equation*}
%\Omega^0_X(U_0) \times \Omega_X^0(U_\infty) \xrightarrow[]{d_0} \Omega_X^1(U_0)\times \Omega_X^1(U_\infty) \times \Omega_X^0(U_0 \cap U_\infty) \xrightarrow[]{d_1} \Omega^1_X(U_0 \cap U_\infty).
%\end{equation*}
%\pause
%Then $H^1_{\text dR}(X) \cong \check{H}^1_{\text dR}(U) = \operatorname{ker}(d_1)/\operatorname{im}(d_0)$, which has elements of the form $(f_0, f_\infty , \omega)$.
%\end{frame}
%
%%%%%%%%%%%%%%%%%%%%%%%%%%%%%%%%%%%%%%%%%%%%%%%%%%%%%%%%%%%%%%%%%%%%%%%%%%%%%%%%%%%%

%%%%%%%%%%%%        Basis characteristic not two      %%%%%%%%%%%%%%%%%%%%%%%%%%%%%%

%%%%%%%%%%%%%%%%%%%%%%%%%%%%%%%%%%%%%%%%%%%%%%%%%%%%%%%%%%%%%%%%%%%%%%%%%%%%%%%%%%%%

\begin{frame}
For each $i \in \{1, \ldots, g\}, k \in \{0, \ldots , 2g+2\}$ we define $\alpha_{i,k}\in k$ such that
\[
xf'(x) - 2if(x) = \alpha_{i,2g+2}x^{2g+2} + \alpha_{i, 2g+1}x^{2g+2} + \ldots + \alpha_{i,0}
\]
and then define
\[
\phi_i(x)  := \alpha_{i,2g+2}x^{2g+2} + \ldots + \alpha_{i,g+2}x^{g+2} 
\]
and
\[
\psi_i(x) := \alpha_{i,g+1}x^{g+1} + \ldots + \alpha_{i,0}.
\]
\pause
\begin{theorem}
Suppose that $p \neq 2$. Then the residue classes of
\[
\mu_i := \left( \left( \frac{\psi_i(x)}{2yx^{i+1}} \right) dx, \left(\frac{-\phi_i(x)}{2yx^{i+1}} \right)dx, \frac{y}{x^i} \right), i= 1, \ldots, g
\]
and
\[
\nu_i := \left( \frac{x^i}{y}dx, \frac{x^i}{y}dx, 0 \right), i = 0, \ldots, g-1
\]
form a basis of $\check{H}^1_X(U)$.
\end{theorem}
\end{frame}

%%%%%%%%%%%%%%%%%%%%%%%%%%%%%%%%%%%%%%%%%%%%%%%%%%%%%%%%%%%%%%%%%%%%%%%%%%%%%%%%%%%%

%%%%%%%%%%%%        Basis characteristic two          %%%%%%%%%%%%%%%%%%%%%%%%%%%%%%

%%%%%%%%%%%%%%%%%%%%%%%%%%%%%%%%%%%%%%%%%%%%%%%%%%%%%%%%%%%%%%%%%%%%%%%%%%%%%%%%%%%%

\begin{frame}
For $1 \leq i \leq g$, $1 \leq j \leq 2g+2$ and $0 \leq k \leq g+1$ we define $A_{j,i}, B_{k,i} \in k$ such that $xF'(x) + y(H'(x) + iH(x))$ equals
\[
A_{2g+2,i}x^{2g+2} + \dots + A_{1,i}x + y (B_{g+1,i}x^{g+1} + \dots + B_{1,i}x + B_{0,i}).
\]
\pause
We then define
\begin{align*}
\Phi_i^x(x) & = A_{2g+2,i}x^{2g+2} + \dots + A_{i+1,i}x^{i+1};\\
\Psi_i^x(x) & = A_{i,i}x^i + \dots + A_{1,i}x;\\
\Phi_i^y(x) & = B_{g+1,i}x^g + \dots + B_{i+1,i}x^{i+1};\\
\Psi_i^y(x) & = B_{i-1,i}x^{i-1} + \dots + B_{0,i},
\end{align*}
and finally we define 
\[
\Phi_i(x,y) = \Phi_i^x(x) + y\Phi_i^y(x)
\]
and 
\[
\Psi_i(x,y) = \Psi_i^x(x) + y\Psi_i^y(x).
\]

%\begin{theorem}
%The residue classes of
%\[
%\left( \left( \frac{\Psi_i(x,y)}{x^{i+1}H(x)} \right) \only<1>{dx}, \left(\frac{-\Phi_i(x,y)}{x^{i+1}H(x)} \right)dx, \frac{y}{x^i} \right), i= 1, \ldots, g
%\]
%and
%\[
%\nu_i := \left( \frac{x^i}{H(x)}dx, \frac{x^i}{H(x)}dx, 0 \right), i = 0, \ldots, g-1
%\]
%form a basis of $\check{H}^1_X(U)$.
%\end{theorem}
\end{frame}

%%%%%%%%%%%%%%%%%%%%%%%%%%%%%%%%%%%%%%%%%%%%%%%%%%%%%%%%%%%%%%%%%%%%%%%%%%%%%%%%%%%%

%%%%%%%%%%%%        Basis of triple cover             %%%%%%%%%%%%%%%%%%%%%%%%%%%%%%

%%%%%%%%%%%%%%%%%%%%%%%%%%%%%%%%%%%%%%%%%%%%%%%%%%%%%%%%%%%%%%%%%%%%%%%%%%%%%%%%%%%%


\begin{frame}
We define 
\[
r_i(x) = \sum_{k=0}^{i-1} (-1)^{g-k} \binom{g}{k} a^{g-k}x^k \quad {\text and} \quad t_i(x) = \sum_{k=i}^g (-1)^{g-k}\binom{g}{k} a^{g-k}x^k.
\]
\begin{theorem}
The pre-image $\rho^{-1}(\tau_i)$ is the residue class of is
\begin{equation*}
\left( \frac{\psi_i(x)}{2yx^{i+1}}dx, \frac{h_i}{2yx^{i+1}(x-a)^{g+1}}dx, \frac{-\phi_i(x)}{2yx^{i+1}}dx, \frac{r_i(x)y}{x^i(x-a)^g}, \frac{y}{x^i}, \frac{t_i(x)y}{x^i(x-a)^g} \right)
\end{equation*}
where
\[
h_i = 
(\psi_i(x)t_i(x) + \phi_i(x)r_i(x))(x-a) - 2if(x)(-1)^{g-i+1}\binom{g}{i}a^{g-i+1}x^i.
\]
\end{theorem}

\end{frame}

\end{document} 
